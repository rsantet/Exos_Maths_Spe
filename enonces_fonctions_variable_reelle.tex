\documentclass[12pt]{article}
\usepackage{style/style}

\begin{document}

\begin{titlepage}
	\centering
	\vspace*{\fill}
	\Huge \textit{\textbf{Exercices MP/MP$^*$\\ Fonction d'une variable réelle}}
	\vspace*{\fill}
\end{titlepage}

\begin{exercise}[Polnômes de Legendre]
	On pose, pour tout $n\in\N$, $L_{n}=P_{n}^{(n)}$ où 
	$$P_{n}=\frac{(X^{2}-1)^{n}}{2^{n}n!}$$
	\begin{enumerate}
		\item On munit $\mathcal{C}^{0}([-1,1],\R)$ du produit scalaire
		$$(f|g)=\int_{-1}^{1}f(t)g(t)dt$$
		Montrer que $(L_{n})_{n\in\N}$ est orthogonale pour ce produit scalaire.

		\item Montrer que 
		$$L_{n}(x)=\frac{1}{2^{n}}\sum_{k=0}^{n}\binom{n}{k}^{2}(x+1)^{n-k}(x-1)^{k}$$
		
		\item Montrer que $L_{n}$ admet $n$ zéros simples sur $]-1,1[$.
		\item Montrer que pour $n\geqslant2$,
		$$L_{n}=\frac{2n-1}{n}XL_{n-1}-\frac{n-1}{2n-1}L_{n-2}$$
	\end{enumerate}
\end{exercise}

\begin{exercise}
	Soit $f\in\mathcal{C}^{n}([a,b],\R)$, $(x_{0},\dots,x_{n})\in[a,b]^{n+1}$ avec $a<b$ et
	$$V(x_{0},\dots,x_{n})
	=
	\begin{vmatrix}
		1 & \dots & \dots & 1\\
		x_{0} & \dots & \dots &x_{n}\\
		\vdots & & &\vdots\\
		x_{0}^{n-1} & \dots &\dots  &x_{n}^{n-1}\\
		f(x_{0}) & \dots  &\dots & f(x_{n})
	\end{vmatrix}
	=\prod_{i>j}(x_{i}-x_{j})\Delta f(x_{0},\dots,x_{n})$$
	S'il existe $i\neq j$ tel que $x_{i}=x_{j}$, alors $\Delta f(x_{0},\dots,x_{n})$ prend n'importe quelle valeur, sinon $\prod_{i>j}(x_{i}-x_{j})\neq0$. Montrer qu'il existe $\xi\in]a,b[$ tel que 
	$$\Delta f(x_{0},\dots,x_{n})=\frac{f^{(n)}(\xi)}{n!}$$
\end{exercise}

\begin{exercise}
	Soit 
	$$E=\Biggl\{f\in\mathcal{C}^{2}([0,1],\R)\Biggm|\Vert f''\Vert_{\infty}\leqslant 1\Biggr\}$$
	Soit pour $f\in E$,
	$$A(f)=f(0)-2f\Bigl(\frac{1}{2}\Bigr)+f(1)$$
	Déterminer $\sup\limits_{f\in E}A(f)$.
\end{exercise}

\begin{exercise}
	Trouver toutes les fonctions $\mathcal{C}^{1}$ de $\R$ dans $\C$ telles que pour tout $(x,y)\in\R^{2}$, $f(x)-f(y)=(x-y)f'\Bigl(\frac{x+y}{2}\Bigr)$.
\end{exercise}

\begin{exercise}
	Soit $f:\R_{+}^{*}\to\R$ convexe.
	\begin{enumerate}
		\item Montrer que $\lim\limits_{x\to+\infty}\frac{f(x)}{x}=l\in\overline{\R}$ existe.
		\item Montrer que si $l\geqslant0$, $f$ est décroissante.
		\item Montrer que si $l\in\R$, $\lim\limits_{x\to+\infty}f(x)-lx$ existe dans $\overline{\R}$.
	\end{enumerate}
\end{exercise}

\begin{exercise}
	Soit $p\in\N^{*}$.
	\begin{enumerate}
		\item Calculer 
		$$l_{p}=\lim\limits_{n\to+\infty}\sum_{k=0}^{np}\frac{1}{n+k}$$
		\item Soit $f\colon\R_{+}\to\R$, $f$ de classe $\mathcal{C}^{1}$ avec $f(0)=0$. Montrer que 
		$$v_{n}=\sum_{k=0}^{np}f\Bigl(\frac{1}{k+n}\Bigr)\xrightarrow[n\to+\infty]{}\ln(p+1)f'(0).$$
		\item Si on suppose seulement $f$ continue et $f(0)=0$, montrer que l'on peut avoir $(v_{n})_{n\in\N}$ divergente.
		\item Si $f$ est de classe $\mathcal{C}^{2}$ avec $f(0)=f'(0)=0$ et $f''(0)\neq0$, trouver un équivalent de $v_{n}$.
	\end{enumerate}
\end{exercise}

\begin{exercise}
	Soit $f\in\mathcal{C}^{1}(\R_{+},\R)$, $\lim\limits_{x\to+\infty}f(x)=m$ existe et $f'$ est uniformément continue. Montrer que $\lim\limits_{x\to+\infty}f'(x)=0$. Et si $f\in\mathcal{C}^{1}(\R_{+},\C)$ ? Et si $f$ est seulement $\mathcal{C}^{1}$ sans $f'$ uniformément continue ?
\end{exercise}

\begin{exercise}
	Trouver toutes les fonctions $f$ et $g$ continues de $\R\to\R$ telles que pour tout $(x,y)\in\R^{2}$,
	$$f(x+y)-f(x-y)=2yg(x)$$
\end{exercise}

\begin{exercise}
	Soit $f\colon]0,+\infty[\to\R$ convexe de classe $\mathcal{C}^{1}$. Soit 
	$$S_{n}=\frac{1}{2}f(1)+f(2)+\dots+f(n-1)+\frac{1}{2}f(n)-\int_{1}^{n}f(t)dt$$
	Montrer que pour tout $n\geqslant2$, 
	$$0\leqslant S_{n}\leqslant\frac{1}{8}(f'(n)-f'(1))$$
\end{exercise}

\begin{exercise}
	\phantom{}
	\begin{enumerate}
		\item Soit $f\colon\R\to E$ où $E$ est un $\R$-espace vectoriel normé de dimension finie avec $f$ de classe $\mathcal{C}^{2}$ et telle que $f$ et $f''$ soient bornées sur $\R$. On poe $M_{0}=\sup\limits_{t\in\R}\Vert f(t)\Vert$ et $M_{2}=\sup\limits_{t\in\R}\Vert f''(t)\Vert$. Montrer que $f'$ est bornée sur $\R$ et que 
		$$M_{1}=\sup\limits_{t\in\R}\Vert f'(t)\Vert\leqslant\sqrt{2M_{0}M_{2}}$$
		Pour $x\in\R$ et $h>0$, on formera 
		$$
		\left\{
			\begin{array}[]{l}
				A=f(x+h)-f(x)-hf'(x)\\
				B=f(x-h)-f(x)+hf'(x)
			\end{array}
		\right.
		$$
		et on exprimera $f'(x)$ en fonction de $A$ et $B$.

		\item Si $f$ est de classe $\mathcal{C}^{n}$ et telle que $f$ et $f^{(n)}$ soient bornées sur $\R$, montrer que pour tout $k\in\{1,\dots,n-1\}$, $f^{(k)}$ l'est aussi. On pourra former 
		$$
		\left\{
			\begin{array}[]{l}
				A_{1}=f(x+1)-f(x)-f'(x)-\dots-\frac{f^{(n-1)}(x)}{(n-1)!}\\
				A_{k}=f(x+k)-f(x)-kf'(x)-\dots-k^{n-1}\frac{f^{(n-1)}(x)}{(n-1)!}
			\end{array}
		\right.
		$$
	\end{enumerate}
\end{exercise}

\begin{exercise}[Longueur d'un arc]
	Soit $\gamma\colon[a,b]\to E$ un arc de classe $\mathcal{C}^{1}$. Pour $\sigma=(a_{0},\dots,a_{n})\in\Sigma([a,b])$ (ensemble des subdivisions de $[a,b]$), on définit 
	$$l_{\sigma,\gamma}=\sum_{i=0}^{n-1}\Vert \gamma(a_{i+1})-\gamma(a_{i})\Vert$$
	On dit que $\gamma$ est de longueur finie si et seulement il existe $l(\gamma)=\sup\limits_{\sigma\in\Sigma([a,b])}l_{\sigma,\gamma}$ appelée longueur de $\gamma$.

	\begin{enumerate}
		\item Montrer que pour tout $\sigma\in\Sigma([a,b])$, 
		$$l_{\sigma,\gamma}\leqslant\int_{a}^{b}\Vert\gamma'(t)\Vert dt$$
		
		\item Soit $\sigma=(a_{1},\dots,a_{n})\in\Sigma([a,b])$, montrer que 
		$$\Bigl\vert l_{\sigma,\gamma}-\sum_{i=0}^{n-1}\Vert\gamma'(a_{i})\Vert (a_{i+1}-a_{i})\Bigr\vert\leqslant\sum_{i=0}^{n-1}\int_{a_{i}}^{a_{i+1}}\Vert\gamma'(t)-\gamma'(a_{i})\Vert dt$$
		
		\item Soit $\varepsilon>0$, justifier qu'il existe $\alpha_{0}>$ tel que si $\delta(\sigma)\leqslant\alpha_{0}$ (où $\delta$ est le pas de la subdivision, c'est-à-dire la longueur maximale entre deux $a_{i}$), alors 
		$$\Bigl\vert\int_{a}^{b}\Vert \gamma'(t)\Vert dt-\sum_{i=0}^{n-1}\Vert\gamma'(a_{i})\Vert(a_{i+1}-a_{i})\Bigr\vert\leqslant\frac{\varepsilon}{3}$$
		Puis montrer qu'il existe $\alpha_{1}>0$ tel que si $\delta(\sigma)\leqslant\alpha_{1}$, alors 
		$$\Bigl\vert l_{\sigma,\gamma}-\sum_{i=0}^{n-1}\Vert\gamma'(a_{i})\Vert(a_{i+1}-a_{i})\Bigr\vert\leqslant\frac{\varepsilon}{2}$$
		En déduire que 
		$$l(\gamma)=\int_{a}^{b}\Vert\gamma'(t)\Vert dt$$
		
		\item Étudier \function{\gamma}{[0,2\pi]}{\R^{2}}{t}{\begin{pmatrix}
			R\cos(t)\\ R\sin(t)
		\end{pmatrix}}
	\end{enumerate}
\end{exercise}

\begin{exercise}[Théorème de relèvement]
	Soit $\gamma\colon I\to\C^{*}$ un arc $\mathcal{C}^{k}$ avec $k\geqslant0$. On appelle relèvement continu de $\gamma$ toute application continue $\theta\colon I\to\R$ telle que pour tout $t\in I$, $\gamma(t)=\vert\gamma(t)\vert e^{\mathrm{i}\theta(t)}$.
	\begin{enumerate}
		\item Montrer que si $\theta_{1}$ et $\theta_{2}$ sont deux relèvements continue de $\gamma$, alors il existe $k_{0}\in\Z$ tel que pour tout $t\in I$, $\theta_{2}(t)-\theta_{1}(t)=2k_{0}\pi$.
		\item On suppose $k\geqslant1$. On pose $f(t)=\frac{\gamma(t)}{\vert\gamma(t)\vert}$. Montrer que $f$ est $\mathcal{C}^{k}$ et que s'il existe $\theta$ relèvement $C^{1}$ de $\gamma$, alors pour tout $t\in I$, 
		$$\theta'(t)=-\mathrm{i}\frac{f'(t)}{f(t)}$$
		\item Pour $k\geqslant1$, en déduire qu'il existe un relèvement $\mathcal{C}^{k}$ de $\gamma$.
	\end{enumerate}
\end{exercise}

\begin{exercise}
	Existe-t-il un $\mathcal{C}^{1}$ difféomorphisme de $\R\to\R$ telle que pour tout $x\in\R$, $f(2x)=3f(x)$.
\end{exercise}

\end{document}