\documentclass[12pt]{article}
\usepackage{style/style}

\begin{document}

\begin{titlepage}
	\centering
	\vspace*{\fill}
	\Huge \textit{\textbf{Exercices MP/MP$^*$\\ Calcul matriciel}}
	\vspace*{\fill}
\end{titlepage}

\begin{exercise}
	Soit $M=\bigl(\omega^{(k-1)(l-1)}\bigr)_{1\leqslant k,l\leqslant n}$ où
	$\omega=e^{\frac{2\mathrm{i}\pi}{n}}$. Montrer que $M\in GL_{n}(\C)$ et
	calculer $M^{-1}$. Que vaut $\det(M)$ ?
\end{exercise}

\begin{exercise}
	On dit que $A\in\M_{n,p}(\R)$ est positive et on note $A\geqslant0$ si et
	seulement si tous ses coefficients le sont.
	\begin{enumerate}
		\item
		Soit $A\in\M_{n}(\R)$. Montrer que $A\geqslant0$ si et seulement si pour
		tout $X\in\M_{n,1}(\R)$, si $X\geqslant0$ alors $AX\geqslant0$.
		\item
		Quelles sont les matrices $A\in GL_{n}(\R)$ telles que $A\geqslant0$ et
		$A^{-1}\geqslant0$?
	\end{enumerate}
\end{exercise}

\begin{exercise}
	Soit $A=\Bigl(\binom{j-1}{i-1}\Bigr)_{1\leqslant i,j\leqslant n}$. Calculer
	$A^{-1}$ et $A^{k}$ pour $k\in\Z$.
\end{exercise}

\begin{exercise}
	Soit $\K$ un corps de caractéristique non nulle ($\Q,\R,\C,\dots$). Soit
	$A\in\M_{n}(\K)$ avec $\Tr(A)=0$.
	\begin{enumerate}
		\item
		Montrer que $A$ est semblable à une matrice dont tous les coefficients
		diagonaux sont nuls. On pourra procéder par récurrence, en distinguant
		selon qu'il existe $\lambda\in\K,~A=\lambda I_{n}$ ou non. Dans le
		deuxième cas, pour $u\in\L(\K^{n})$ canoniquement associée à $A$, on
		montrera qu'il existe $e_{1}\in\K^{n}$ telle que $(e_{1},u(e_{1}))$ est
		libre.
		\item
		Montrer qu'il existe $(X,Y)\in\M_{n}(\K)^{2}$ tel que $A=[X,Y]=XY-YX$. On
		pourra considérer \function{\varphi}{\M_{n}(\K)}{\M_{n}(\K)}{M}{DM-MD}
		avec $D=\diag(1,2,\dots,n)$ et déterminer $\ker(\varphi)$.
	\end{enumerate}
\end{exercise}

\begin{exercise}
	Soit $(X,Y)\in\M_{n,1}(\K)^{2}$.
	\begin{enumerate}
		\item
		Pour quelles valeurs de $\lambda$, $I_{n}+\lambda XY^\mathsf{T}$ est
		inversible?
		\item
		Soit $A\in GL_{n}(\R)$. A quelle condition nécessaire et suffisante
		$A+\lambda XY^\mathsf{T}\in GL_{n}(\R)$? Montrer alors que 
		$$(A+\lambda XY^\mathsf{T})^{-1}=A^{-1}-\frac{\lambda}{1+\lambda
		Y^{\mathsf{T}}A^{-1}X}A^{-1}XY^{\mathsf{T}}A^{-1}$$
	\end{enumerate}
\end{exercise}

\begin{exercise}
	Soit $n\geqslant1$ et pour tout $j\in\{0,\dots,n\}$, $S_{j}=X^{j}(1-X)^{n-j}$.
	Montrer que c'est une base de $\R_{n}[X]$ et exprimer $(1,X,\dots,X^{n})$ en
	fonction de $(S_{0},\dots,S_{n})$.
\end{exercise}

\begin{exercise}
	Soit $\K$ un corps et $H$ un hyperplan de $\M_{n}(\K)$. Montrer que $H\cap
	GL_{n}(\K)\neq 0$ pour $n\geqslant2$.
\end{exercise}

\begin{exercise}
	Soit $N:\M_{n}(\C)\to\R_{+}$ telle que:
	\begin{enumerate}
		\item
		[(i)] $\forall \lambda\geqslant0,\forall A\in\M_{n}(\C),~N(\lambda
		A)=\lambda N(A)$,
		\item
		[(ii)] $\forall (A,B)\in\M_{n}(\C)^{2},~N(A+B)\leqslant N(A)+N(B)$,
		\item
		[(iii)] $\forall (A,B)\in\M_{n}(\C)^{2},~N(AB)=N(BA)$.
	\end{enumerate}
	\begin{enumerate}
		\item
		Calculer $N(0)$.
		\item
		Évaluer $N(E_{i,j})$ pour $i\neq j$ (matrice élémentaire de la base
		canonique de $\M_{n}(\C)$).
		\item
		Montrer que si $A\in\M_{n}(\C)$ est telle que $\Tr(A)=0$, alors $A$ est
		semblable à une matrice dont tous les coefficients diagonaux sont nuls. On
		pourra procéder par récurrence, en distinguant selon qu'il existe
		$\lambda\in\K,~A=\lambda I_{n}$ ou non. Dans le deuxième cas, pour
		$u\in\L(\K^{n})$ canoniquement associée à $A$, on montrera qu'il existe
		$e_{1}\in\K^{n}$ telle que $(e_{1},u(e_{1}))$ est libre.
		\item
		En déduire $N(A)$ si $\Tr(A)=0$.
		\item
		Montrer qu'il existe $a\in\R^{+}$ telle que pour tout $A\in\M_{n}(\C)$,
		$N(A)=a\vert\Tr(A)\vert$.
	\end{enumerate}
\end{exercise}

\begin{exercise}
	Soit $E$ un espace vectoriel de dimension $n$, $f\in GL(E)$ et $g$ un
	endomorphisme de $E$ de rang 1. Montrer que $f+g\in GL(E)$ si et seulement
	$\Tr(g\circ f^{-1})\neq 1$.
\end{exercise}

\begin{exercise}
	On considère un carré dans $\Z^{2}$. Pour $n\in\N$, quel est le nombre de
	chemins de longueur $n$ qui relient un sommet à un autre ? Généraliser à un
	cube dans $\Z^{3}$.
\end{exercise}

\begin{exercise}[Matrice à diagonale strictement dominante]
	Soit $A=(a_{i,j})_{1\leqslant i,j\leqslant n}\in\M_{n}(\C)$ telle que pour
	tout $i\in\{1,\dots,n\}$$\vert a_{i,i}\vert>\sum_{j\neq i}\vert a_{i,j}\vert$.
	On dit que $A$ est à diagonale strictement dominante. Montrer que $A\in
	GL_{n}(\C)$. Est-ce encore vrai si on a seulement l'inégalité large ?
\end{exercise}

\begin{exercise}
	Calculer, pour $n\geqslant1$, $\det\Bigl((i\wedge j)\Bigr)_{1\leqslant
	i,j\leqslant n}$. On pourra utiliser, pour tout $n\in\N^{*},~n=\sum_{k\mid
	n}\varphi(k)$.
\end{exercise}

\begin{exercise}
	Soit $A=(a_{i,j})_{1\leqslant i,j\leqslant n}\in\M_{n}(\C)$. On pose, pour
	$k\in\{1,\dots, n\}$,$A_{k}=(a_{i,j})_{1\leqslant i,j\leqslant k}$. On suppose
	que pour tout $k\in\{1,\dots,n\},~A_{k}\in GL_{k}(\C)$. Montrer qu'il existe
	une unique décomposition
	$(L,U)\in\mathcal{T}_{n}^{-}(\C)\times\mathcal{T}_{n}^{+}(\C)$ (matrices
	triangulaires inférieures et supérieures) où $L$ a des $1$ sur la diagonale et
	$A=LU$.
\end{exercise}

\begin{exercise}
	Soit $n\in\N$ et $(a_{1},\dots,a_{2n+1})\in\R^{2n+1}$ tel que pour tout
	$i\in\{1,\dots,2n+1\}$, il existe des parties disjointes $A_{i}$ et $B_{i}$ de
	$\{1,\dots,2n+1\}\setminus\{i\}$ avec $\vert A_{i}\vert=\vert B_{i}\vert=n$ et
	$\sum_{k\in A_{i}}a_{k}=\sum_{k\in B_{i}}a_{k}$.

	Monter que $a_{1}=\dots=a_{2n+1}$.
\end{exercise}

\begin{exercise}
	Soit $M\in GL_{n}(\C)$, montrer qu'il existe une unique permutation
	$\sigma\in\Sigma_{n}$ et il existe $(T,T')\in(\mathcal{T}_{n}^{+})^{2}$ telles
	que $M=TP_{\sigma}T'$ où $P_{\sigma}=(\delta_{i,\sigma(j)})_{1\leqslant
	i,j\leqslant n}$ et $\delta$ est le symbole de Kronecker. Cette décomposition
	est-elle unique ?
\end{exercise}

\begin{exercise}
	Soit $\K$ un sous-corps de $\C$ et $J$ un idéal non nul de $\M_{n}(\K)$.
	\begin{enumerate}
		\item
		Montrer que si $J\cap GL_{n}(\K)\neq\emptyset$, alors $J=\M_{n}(\K)$.
		\item
		Montrer que $J$ contient une matrice de rang 1.
		\item
		Montrer que $J=\M_{n}(\K)$.
	\end{enumerate}
\end{exercise}

\begin{exercise}
	Soit $(A,B)\in\M_{n}(\C)$ et $\lambda\neq 0$ avec $\lambda AB+A+B=0$, montrer
	que $A$ et $B$ commutent.
\end{exercise}

\begin{exercise}
	Soit $(a_{2},\dots,a_{n})\in\R^{n-1}$. Inverser, si possible,
	$$
	A=
	\begin{pmatrix}
		1 		& -a_{2}	& \dots		& -a_{n}\\
		a_{2} 	& \ddots 	& 0			& 0\\
		\vdots 	& 0			& \ddots 	& 0\\
		a_{n}	& 0			& 0			& 1
	\end{pmatrix}
	$$
\end{exercise}

\begin{exercise}
	Soit $A=(a_{i,j})_{1\leqslant i,j\leqslant n}\in\M_{n}(\R)$ telle que pour
	tout $i\in\{1,\dots,n\}$, $a_{i,i}=0$ et pour tout $i\neq j$,
	$a_{i,j}+a_{j,i}=1$. Soit $u\in\L(\R^{n})$ canoniquement associé à $A$. Soit
	$H=\{(x_{1},\dots,x_{n})\in\R^{n}\mid\sum_{i=1}^{n}x_{i}=0\}$.
	\begin{enumerate}
		\item
		Déterminer $\ker(u)\cap H$. En déduire que $\rg(A)\in\{n-1,n\}$.
		\item
		Est-il possible que toutes les matrices $A$ vérifiant ces conditions
		soient de rang $n-1$ ?
		\item
		Même question avec $n$.
	\end{enumerate}
\end{exercise}

\begin{exercise}
	Soit $(M,N)\in\M_{n}(\C)^{2}$ tel que $\rg(M)=\rg(N)=1$. Montrer que $M$ et
	$N$ sont semblables si et seulement si $\Tr(M)=\Tr(N)$.
\end{exercise}

\begin{exercise}
	Soit $F$ un sous-espace vectoriel de $\M_{n}(\K)$, on note $r=\max\{\rg(M)\mid
	M\in F \}$.
	\begin{enumerate}
		\item
		Montrer qu'il existe $(P_{0},Q_{0})\in GL_{n}(\R)$ telle que 
		$$
		P_{0}^{-1}
		\underbrace{
			\left(
				\begin{array}{@{}c|c@{}}
					I_{r}
					& 0_{r,n-r} \\
					\hline
					0_{n-r,r} &
					O_{n-r,n-r}
				\end{array}\right)
		}_{\displaystyle J_{r}}
		Q_{0}\in F
		$$
		On note $F_{0}=\{P_{0}MQ_{0}^{-1}\mid M\in F\}$.
		\item
		Montrer que $F_{0}$ est un sous-espace vectoriel de $\M_{n}(\R)$ isomorphe
		à $F$, et que \\$r=\max\{\rg(M_{0})\mid M_{0}\in F_{0}\}$.
		\item
		On définit
		
		$$G_{0}= \left(
				\begin{array}{@{}c|c@{}}
					0_{r}
					& B^{\mathsf{T}} \\
					\hline
					B & C \end{array}\right)
		$$
		où $B\in\M_{n-r,r}(\R)$ et $C\in\M_{n-r}(\R)$. 
		Quelle est la dimension de l'espace vectoriel engendré par $G_{0}$ ?
		\item Soit $M_{0}\in G_{0}\cap F_{0}$ avec 
		$$
			M_{0}= \left(
				\begin{array}{@{}c|c@{}}
					0_{r}
					& B^{\mathsf{T}} \\
					\hline
					B & C
				\end{array}
			\right)
			\in
			F_{0}
		$$
		Montrer que pour tout $\lambda\in\R$,
		$$
			\left(
				\begin{array}{@{}c|c@{}}
				\lambda
				I_{r} & B^{\mathsf{T}} \\
					\hline
					B & C
				\end{array}
			\right)
			\in
			F_{0}
		$$
		En déduire que pour tout $(i,j)\in\{1,\dots, n-r\}^{2}$, pour tout $\lambda\neq0$,
		$$
		\det\left(
				\begin{array}{@{}c|c@{}}
				\lambda
				I_{r} &
				\begin{matrix}
				b_{j,1}\\
						\vdots\\
						b_{j,r}
						\end{matrix}
						\\
					\hline
					\begin{matrix}
						b_{i,1} &
						\dots
						& b_{i,r}
						\end{matrix}
						& c_{i,j}
				\end{array}
			\right)=0
		$$
		\item Montrer que $C=0$, puis que $B=0$.
		\item Conclure.
		\item Si $\dim(F)\geqslant n^{2}-n+1$, montrer que $F\cap GL_{n}(\R)\neq\emptyset$.
		\item Et sur $\M_{n}(\C)$ ?
	\end{enumerate}
\end{exercise}

\begin{exercise}
	Soit $f:\M_{n}(\C)\to\C$ non constante telle que pour tout $(A,B)\in\M_{n}(\C)^{2}$, $f(AB)=f(A)\times f(B)$. 
	Montrer que $f(M)\neq0$ si et seulement si $M\in GL_{n}(\C)$.
\end{exercise}

\begin{exercise}
	Soit $(A_{1},\dots,A_{k})\in\M_n(\K)^{k}$ tels que
	\begin{enumerate}[label=(\roman*)]
		\item $\forall i\in\left\llbracket1,k\right\rrbracket$, $A_{i}^{2}=A_{i}$,
		\item $\sum_{i=1}^{k}A_i=I_n$.
	\end{enumerate}
	Montrer que pour tout $i\neq j$, $A_iA_j=0$.
\end{exercise}

\end{document}