\section{Intégration}

\begin{exercise}
    Soit $f$ continue strictement positive de $[a,b]\subset\R$ dans $\R_{+}^{*}$ et \function{S}{[a,b]}{\R}{x}{\int_{a}^{x}f}
    Montrer que pour tout $n\geqslant1$, pour tout $k\in\left\llbracket1,n\right\rrbracket$, il existe un unique $x_{k}\in[a,b]$ tel que $S(x_{k})=k\frac{S(b)}{n}$. Évaluer ensuite $\lim\limits_{n\to+\infty}\sum_{k=1}^{n}f(x_{k})$.
\end{exercise}

\begin{exercise}
    Soit $f$ continue non identiquement nulle et $g(x)=\left(\int_{0}^{1}\left\lvert f(t)\right\rvert^{x}\mathrm{d}t\right)^{\frac{1}{x}}$.
    \begin{enumerate}
        \item Montrer que $\lim\limits_{x\to+\infty}g(x)=\left\lVert f\right\rVert_{\infty}$.
        \item On suppose $\left\lvert f\right\rvert>0$, calculer $\lim\limits_{x\to0}g(x)$.
    \end{enumerate}
\end{exercise}

\begin{exercise}
    Soit $f\colon[0,a]\to\R$ strictement croissante continue avec $f(0)=0$. Soit $g\colon[0,f(a)]\to[0,a]=f^{-1}$ (continue strictement croissante). Soit $(x,y)\in[0,a]\times[0,f(a)]$. Montrer que 
    \begin{equation}
        xy\leqslant\int_{0}^{x}f+\int_{0}^{y}g.    
    \end{equation} 
    Expliciter le cas d'égalité.
\end{exercise}

\begin{exercise}
    Existence et calcul de \begin{equation}
        I=\int_{\frac{1}{2}}^{1}\frac{\ln(x)}{(1+x)\sqrt{1-x^{2}}}\mathrm{d}x.
    \end{equation}
\end{exercise}

\begin{exercise}
    Pour $n\in\N$, on pose $I_{n}=\int_{0}^{\frac{\pi}{4}}\tan^{n}(x)\mathrm{d}x$.
    \begin{enumerate}
        \item Exprimer $I_{n}$ en fonction de $n$.
        \item Que vaut $\lim\limits_{n\to+\infty}I_{n}$ (sous réserve d'existence) ?
        \item En déduire $\frac{\pi}{4}$ et $\ln(2)$ comme somme de séries.
    \end{enumerate}
\end{exercise}

\begin{exercise}
    Soit $E=\left\lbrace f\in\mathcal{C}^{0}\left([a,b],\R_{+}^{*}\right)\right\rbrace$. On définit \function{\phi}{E}{\R}{f}{\int_{a}^{b}f\times\int_{a}^{b}\frac{1}{f}}
    \begin{enumerate}
        \item Montrer que l'on peut définir $m=\min_{f\in E}\phi(f)$ et évaluer $m$. Déterminer les $f\in E$ tels que $\phi(f)=m$.
        \item Montrer que $f$ n'est pas majorée sur $E$.
        \item Déterminer $\phi(E)$.
    \end{enumerate}
\end{exercise}

\begin{exercise}
    Existence et calcul de $I=\int_{0}^{+\infty}\frac{\sqrt{x}\ln(x)}{(1+x)^{2}}\mathrm{d}x$.
\end{exercise}

\begin{exercise}
    Existence et calcul de $I=\int_{0}^{1}\frac{\ln(t)}{\sqrt{t(1-t)^{3}}}\mathrm{d}t$.
\end{exercise}

\begin{exercise}
    Existence et calcul de $I=\int_{0}^{\frac{\pi}{4}}\frac{\cos^{3}(t)}{\sqrt{\cos(2t)}}\mathrm{d}t$.
\end{exercise}

\begin{exercise}
    Soit $f\colon[a,b]\to\R$ ou $\C$ continue par morceaux et $g\colon\R\to\R$ ou $\C$ continue par morceaux $T$-périodique ($T>0$). Évaluer $\lim\limits_{\lambda\to+\infty}\int_{a}^{b}f(t)g(\lambda t)\mathrm{d}t$. Cas particulier: pour $f\colon[0,2\pi]\to\R$ continue, évaluer $\lim\limits_{n\to+\infty}\int_{0}^{2\pi}\frac{f(t)}{3+2\cos(nt)}\mathrm{d}t$.
\end{exercise}

\begin{exercise}
    Soit $f\colon\R_{+}\to\R$ uniformément continue et intégrable.
    \begin{enumerate}
        \item Montrer que $\lim\limits_{x\to+\infty}f(x)=0$.
        \item Montrer que $f^{2}\in\mathcal{L}^{1}(\R_{+})$.
    \end{enumerate}
\end{exercise}

\begin{exercise}
    Soit \function{f_n}{\R}{\R}{x}{\frac{n}{\sqrt{\pi}}\e^{-n^{2}x^{2}}}
    \begin{enumerate}
        \item Étudier la convergence simple, la convergence uniforme et en moyenne.
        \item Soit $g$ continue et bornée sur $\R$, évaluer $\lim\limits_{x\to+\infty}\int_{-\infty}^{+\infty}f(t)f_n(t)\d t$.
    \end{enumerate}
\end{exercise}

\begin{exercise}
    Existence et calcul de $I=\int_{1}^{+\infty}\frac{1}{x}-\arcsin\left(\frac{1}{x}\right)\d x$.
\end{exercise}

\begin{exercise}
    Existence et calcul de $I=\int_{0}^{\frac{\pi}{2}}\ln(\sin(t))\d t$. On pourra poser $J=\int_{0}^{\frac{\pi}{2}}\ln(\cos(t))\d t$.
\end{exercise}

\begin{exercise}
    Pour $\alpha>1$, on note \function{f_\alpha}{\R_{+}}{\R}{x}{\frac{1}{1+x^{\alpha}\left\lvert\sin(x)\right\rvert}}
    Montrer que $f_{\alpha}$ est intégrable sur $\R_{+}$. 
\end{exercise}

\begin{exercise}
    \phantom{}
    \begin{enumerate}
        \item Soit $f\colon[a,b]\to\R$ continue telle que pour tout $n\in\N$, $\int_{a}^{b}f(t)\d t=0$. Montrer que $f=0$.
        \item Calculer, pour $n\in\N$, $I_{n}=\int_{0}^{+\infty}t^{n}\e^{(\i-1)t}\d t$.
        \item En déduire $f\colon\R_{+}\to\R$ non nulle telle que pour tout $n\in\N$, $\int_{0}^{+\infty}t^{n}f(t)\d t=0$ (avec pour tout $n\in\N$, $t\to t^{n}f(t)$ intégrables sur $\R_{+}$). 
    \end{enumerate}
\end{exercise}

\begin{exercise}[Transformée de Laplace]
    Soit $f\colon\R_{+}\to\R$ ou $\C$ continue par morceaux. On suppose qu'il existe $a\in\R$ telle que $\int_{0}^{+\infty}\e^{-at}f(t)\d t=\mathcal{L}f(a)$ converge. On définit $g(t)=\int_{0}^{t}\e^{-a u}f(u)\d u$.
    \begin{enumerate}
        \item Montrer que pour tout $b>a$, $\mathcal{L}f(b)$ converge et que 
        \begin{equation}
            \mathcal{L}f(b)=(b-a)\int_{0}^{+\infty}\e^{-(b-a)t}g(t)\d t.
        \end{equation}
        \item Soit $h$ vérifiant les mêmes conditions que $f$, montrer que si pour tout $b\geqslant a$, $\mathcal{L}f(b)=\mathcal{L}h(b)$, alors $f=h$ (injectivité de la transformée de Laplace).
    \end{enumerate}
\end{exercise}

\begin{exercise}
    On dit que $f$ est continue à support compact si $f\colon\R\to\R$ ou $\C$ est continue et il existe $A\geqslant0$ tel que pour tout $x\in\R$, si $\left\lvert x\right\rvert\geqslant A$, $f(x)=0$.
    \begin{enumerate}
        \item Montrer que l'on peut définir, pour tout $x\in\R$, $\widehat{f}(x)=\int_{-\infty}^{+\infty}\e^{\i tx}f(t)\d t$, et que $\widehat{f}$ est $\mathcal{C}^{\infty}$ est $\mathcal{C}^{\infty}$ sur $\R$.
        \item On suppose que $\widehat{f}$ est continue à support compact, montrer que $f=0$.
    \end{enumerate}
\end{exercise}

\begin{exercise}
    Soit $f\colon I\to\R$ continue, montrer que $f$ est convexe si et seulement si pour tout $(a,b)\in I^{2}$ avec $a<b$, $(b-a)f\left(\frac{a+b}{2}\right)\leqslant\int_{a}^{b}f(t)\d t$.
\end{exercise}

\begin{exercise}
    Soit $x>0$.
    \begin{enumerate}
        \item Montrer que 
        \begin{equation}
            \lim\limits_{n\to+\infty}\underbrace{\int_{0}^{n}\left(1-\frac{t}{n}\right)^{n}t^{x-1}\d t}_{I_n(x)}=\Gamma(x)=\int_{0}^{+\infty}t^{x-1}\e^{-t}\d t.
        \end{equation}
        \item En déduire que 
        \begin{equation}
            \Gamma(x)=\lim\limits_{n\to+\infty}\frac{n!n^{x}}{x(x+1)\dots(x+n)}.
        \end{equation}
        \item Montrer que 
        \begin{equation}
            \frac{1}{\Gamma(x)}=x\e^{\gamma x}\prod_{k=1}^{+\infty}\left(1+\frac{x}{k}\right)\e^{-\frac{x}{k}},
        \end{equation}
        où $\gamma$ est la constante d'Euler.
        \item Donner un développement en série de $\frac{\Gamma'}{\Gamma}$.
    \end{enumerate}
\end{exercise}

\begin{exercise}[Théorème de D'Alembert-Gauss par l'indice]
    \phantom{}
    \begin{enumerate}
        \item Soit $f\colon\R\to\C^{*}$ $2\pi$-périodique et $\mathcal{C}^{1}$. On définit l'indice de $f$ par 
        \begin{equation}
            d(f)=\frac{1}{2\i\pi}\int_{0}^{2\pi}\frac{f'}{f}.
        \end{equation}
        Montrer que $\e^{2\i\pi d(f)}=1$ si et seulement si $d(f)\in\Z$.
        
        \item Soit $P\in\C[X]$ de degré $n\geqslant1$, de coefficient $a_n\neq0$, on suppose que $P$ ne s'annule pas sur $\C$. Soit $t\geqslant0$, on pose \function{f_r}{\R}{\C^{*}}{t}{P(r\e^{\i t})}
        \item Évaluer $d(f_{0})$ et $\lim\limits_{r\to+\infty}d(f_{r})$, montrer que $r\mapsto d(f_{r})$ est continue. Conclure.
    \end{enumerate}
\end{exercise}