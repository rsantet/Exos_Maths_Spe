\documentclass[12pt]{article}
\usepackage{style/style}

\begin{document}

\begin{titlepage}
	\centering
	\vspace*{\fill}
	\Huge \textit{\textbf{Exercices MP/MP$^*$\\ Intégration}}
	\vspace*{\fill}
\end{titlepage}

\begin{exercise}
    Soit $f$ continue strictement positive de $[a,b]\subset\R$ dans $\R_{+}^{*}$ et \function{S}{[a,b]}{\R}{x}{\int_{a}^{x}f}
    Montrer que pour tout $n\geqslant1$, pour tout $k\in\left\llbracket1,n\right\rrbracket$, il existe un unique $x_{k}\in[a,b]$ tel que $S(x_{k})=k\frac{S(b)}{n}$. Évaluer ensuite $\lim\limits_{n\to+\infty}\frac{1}{n}\sum_{k=1}^{n}f(x_{k})$.
\end{exercise}

\begin{exercise}
    Soit $f$ continue non identiquement nulle et $g(x)=\left(\int_{0}^{1}\left\lvert f(t)\right\rvert^{x}\mathrm{d}t\right)^{\frac{1}{x}}$.
    \begin{enumerate}
        \item Montrer que $\lim\limits_{x\to+\infty}g(x)=\left\lVert f\right\rVert_{\infty}$.
        \item On suppose $\left\lvert f\right\rvert>0$, calculer $\lim\limits_{x\to0}g(x)$.
    \end{enumerate}
\end{exercise}

\begin{exercise}
    Soit $f\colon[0,a]\to\R$ strictement croissante continue avec $f(0)=0$. Soit $g\colon[0,f(a)]\to[0,a]=f^{-1}$ (continue strictement croissante). Soit $(x,y)\in[0,a]\times[0,f(a)]$. Montrer que 
    \begin{equation*}
        xy\leqslant\int_{0}^{x}f+\int_{0}^{y}g.    
    \end{equation*} 
    Expliciter le cas d'égalité.
\end{exercise}

\begin{exercise}
    Existence et calcul de \begin{equation*}
        I=\int_{\frac{1}{2}}^{1}\frac{\ln(x)}{(1+x)\sqrt{1-x^{2}}}\mathrm{d}x.
    \end{equation*}
\end{exercise}

\begin{exercise}
    Pour $n\in\N$, on pose $I_{n}=\int_{0}^{\frac{\pi}{4}}\tan^{n}(x)\mathrm{d}x$.
    \begin{enumerate}
        \item Exprimer $I_{n}$ en fonction de $n$.
        \item Que vaut $\lim\limits_{n\to+\infty}I_{n}$ (sous réserve d'existence) ?
        \item En déduire $\frac{\pi}{4}$ et $\ln(2)$ comme somme de séries.
    \end{enumerate}
\end{exercise}

\begin{exercise}
    Soit $E=\left\lbrace f\in\mathcal{C}^{0}\left([a,b],\R_{+}^{*}\right)\right\rbrace$. On définit \function{\phi}{E}{\R}{f}{\int_{a}^{b}f\times\int_{a}^{b}\frac{1}{f}}
    \begin{enumerate}
        \item Montrer que l'on peut définir $m=\min_{f\in E}\phi(f)$ et évaluer $m$. Déterminer les $f\in E$ tels que $\phi(f)=m$.
        \item Montrer que $f$ n'est pas majorée sur $E$.
        \item Déterminer $\phi(E)$.
    \end{enumerate}
\end{exercise}

\begin{exercise}
    Existence et calcul de $I=\int_{0}^{+\infty}\frac{\sqrt{x}\ln(x)}{(1+x)^{2}}\mathrm{d}x$.
\end{exercise}

\begin{exercise}
    Existence et calcul de $I=\int_{0}^{1}\frac{\ln(t)}{\sqrt{t(1-t)^{3}}}\mathrm{d}t$.
\end{exercise}

\begin{exercise}
    Existence et calcul de $I=\int_{0}^{\frac{\pi}{4}}\frac{\cos^{3}(t)}{\sqrt{\cos(2t)}}\mathrm{d}t$.
\end{exercise}

\begin{exercise}
    Soit $f\colon[a,b]\to\R$ ou $\C$ continue par morceaux et $g\colon\R\to\R$ ou $\C$ continue par morceaux $T$-périodique ($T>0$). Évaluer $\lim\limits_{\lambda\to+\infty}\int_{a}^{b}f(t)g(\lambda t)\mathrm{d}t$. Cas particulier: pour $f\colon[0,2\pi]\to\R$ continue, évaluer $\lim\limits_{n\to+\infty}\int_{0}^{2\pi}\frac{f(t)}{3+2\cos(nt)}\mathrm{d}t$.
\end{exercise}

\begin{exercise}
    Soit $f\colon\R_{+}\to\R$ uniformément continue et intégrable.
    \begin{enumerate}
        \item Montrer que $\lim\limits_{x\to+\infty}f(x)=0$.
        \item Montrer que $f^{2}\in\mathcal{L}^{1}(\R_{+})$.
    \end{enumerate}
\end{exercise}

\begin{exercise}
    Soit \function{f_n}{\R}{\R}{x}{\frac{n}{\sqrt{\pi}}\e^{-n^{2}x^{2}}}
    \begin{enumerate}
        \item Étudier la convergence simple, la convergence uniforme et en moyenne.
        \item Soit $g$ continue et bornée sur $\R$, évaluer $\lim\limits_{x\to+\infty}\int_{-\infty}^{+\infty}g(t)f_n(t)\d t$.
    \end{enumerate}
\end{exercise}

\begin{exercise}
    Existence et calcul de $I=\int_{1}^{+\infty}\frac{1}{x}-\arcsin\left(\frac{1}{x}\right)\d x$.
\end{exercise}

\begin{exercise}
    Existence et calcul de $I=\int_{0}^{\frac{\pi}{2}}\ln(\sin(t))\d t$. On pourra poser 
    \begin{equation*}
        J=\int_{0}^{\frac{\pi}{2}}\ln(\cos(t))\d t.
    \end{equation*}
\end{exercise}

\begin{exercise}
    Pour $\alpha>1$, on note \function{f_\alpha}{\R_{+}}{\R}{x}{\frac{1}{1+x^{\alpha}\left\lvert\sin(x)\right\rvert}}
    Montrer que $f_{\alpha}$ est intégrable sur $\R_{+}$. 
\end{exercise}

\begin{exercise}
    \phantom{}
    \begin{enumerate}
        \item Soit $f\colon[a,b]\to\R$ continue telle que pour tout $n\in\N$, $\int_{a}^{b}f(t)\d t=0$. Montrer que $f=0$.
        \item Calculer, pour $n\in\N$, $I_{n}=\int_{0}^{+\infty}t^{n}\e^{(\i-1)t}\d t$.
        \item En déduire $f\colon\R_{+}\to\R$ non nulle telle que pour tout $n\in\N$, $\int_{0}^{+\infty}t^{n}f(t)\d t=0$ (avec pour tout $n\in\N$, $t\to t^{n}f(t)$ intégrables sur $\R_{+}$). 
    \end{enumerate}
\end{exercise}

\begin{exercise}[Transformée de Laplace]
    Soit $f\colon\R_{+}\to\R$ ou $\C$ continue par morceaux. On suppose qu'il existe $a\in\R$ telle que $\int_{0}^{+\infty}\e^{-at}f(t)\d t=\mathcal{L}f(a)$ converge. On définit $g(t)=\int_{0}^{t}\e^{-a u}f(u)\d u$.
    \begin{enumerate}
        \item Montrer que pour tout $b>a$, $\mathcal{L}f(b)$ converge et que 
        \begin{equation*}
            \mathcal{L}f(b)=(b-a)\int_{0}^{+\infty}\e^{-(b-a)t}g(t)\d t.
        \end{equation*}
        \item Soit $h$ vérifiant les mêmes conditions que $f$, montrer que si pour tout $b\geqslant a$, $\mathcal{L}f(b)=\mathcal{L}h(b)$, alors $f=h$ (injectivité de la transformée de Laplace).
    \end{enumerate}
\end{exercise}

\begin{exercise}
    On dit que $f$ est continue à support compact si $f\colon\R\to\R$ ou $\C$ est continue et il existe $A\geqslant0$ tel que pour tout $x\in\R$, si $\left\lvert x\right\rvert\geqslant A$, $f(x)=0$.
    \begin{enumerate}
        \item Montrer que l'on peut définir, pour tout $x\in\R$, $\widehat{f}(x)=\int_{-\infty}^{+\infty}\e^{\i tx}f(t)\d t$, et que $\widehat{f}$ est $\mathcal{C}^{\infty}$ est $\mathcal{C}^{\infty}$ sur $\R$.
        \item On suppose que $\widehat{f}$ est continue à support compact, montrer que $f=0$.
    \end{enumerate}
\end{exercise}

\begin{exercise}
    Soit $f\colon I\to\R$ continue, montrer que $f$ est convexe si et seulement si pour tout $(a,b)\in I^{2}$ avec $a<b$, $(b-a)f\left(\frac{a+b}{2}\right)\leqslant\int_{a}^{b}f(t)\d t$.
\end{exercise}

\begin{exercise}
    Soit $x>0$.
    \begin{enumerate}
        \item Montrer que 
        \begin{equation*}
            \lim\limits_{n\to+\infty}\underbrace{\int_{0}^{n}\left(1-\frac{t}{n}\right)^{n}t^{x-1}\d t}_{I_n(x)}=\Gamma(x)=\int_{0}^{+\infty}t^{x-1}\e^{-t}\d t.
        \end{equation*}
        \item En déduire que 
        \begin{equation*}
            \Gamma(x)=\lim\limits_{n\to+\infty}\frac{n!n^{x}}{x(x+1)\dots(x+n)}.
        \end{equation*}
        \item Montrer que 
        \begin{equation*}
            \frac{1}{\Gamma(x)}=x\e^{\gamma x}\prod_{k=1}^{+\infty}\left(1+\frac{x}{k}\right)\e^{-\frac{x}{k}},
        \end{equation*}
        où $\gamma$ est la constante d'Euler.
        \item Donner un développement en série de $\frac{\Gamma'}{\Gamma}$.
    \end{enumerate}
\end{exercise}

\begin{exercise}[Théorème de D'Alembert-Gauss par l'indice]
    \phantom{}
    \begin{enumerate}
        \item Soit $f\colon\R\to\C^{*}$ $2\pi$-périodique et $\mathcal{C}^{1}$. On définit l'indice de $f$ par 
        \begin{equation*}
            d(f)=\frac{1}{2\i\pi}\int_{0}^{2\pi}\frac{f'}{f}.
        \end{equation*}
        Montrer que $\e^{2\i\pi d(f)}=1$ si et seulement si $d(f)\in\Z$.
        
        \item Soit $P\in\C[X]$ de degré $n\geqslant1$, de coefficient $a_n\neq0$, on suppose que $P$ ne s'annule pas sur $\C$. Soit $t\geqslant0$, on pose \function{f_r}{\R}{\C^{*}}{t}{P(r\e^{\i t})}
        \item Évaluer $d(f_{0})$ et $\lim\limits_{r\to+\infty}d(f_{r})$, montrer que $r\mapsto d(f_{r})$ est continue. Conclure.
    \end{enumerate}
\end{exercise}

\begin{exercise}
    Soit $f\colon[0,1]\to\R$ de classe $\mathcal{C}^{2}$. On pose, pour tout $n\in\N^{*}$,
    \begin{equation*}
        v_n=\sum_{k=1}^{n-1}f\left(\frac{k}{n}\right)+\frac{1}{2n}\left(f(0)+f(1)\right).
    \end{equation*}
    Donner un développement à l'ordre 2 de $v_n$ quand $n\to+\infty$. On pourra montrer l'égalité de Taylor-Lagrange : si $f$ est de classe $\mathcal{C}^{n}$ sur $[a,b]$, $a<b$, alors il existe $\xi\in]a,b[$ tel que 
    \begin{equation*}
        f(b)-f(a)-\sum_{k=1}^{n-1}\frac{f^{(k)}(a)}{k!}=\frac{(b-a)^{n}}{n!}f^{(n)}(\xi),
    \end{equation*}
    et et l'appliquer à $F(x)=\int_{0}^{x}f$.
\end{exercise}

\begin{exercise}
    Pour $n\in\N$, on pose 
    \begin{equation*}
        I_n=\int_{0}^{\frac{\pi}{4}}\tan^{n}(t)\d t.
    \end{equation*}
    Exprimer $I_n$ sous forme d'une somme, puis donner un équivalent de $I_n$ quand $n\to+\infty$. Qu'en déduit-on sur $\frac{\pi}{4}$ ?
\end{exercise}

\begin{exercise}
    Soit \function{f}{]0,1]}{\R}{x}{\int_{x^{2}}^{x}\frac{\e^{t}}{\arcsin(t)}\d t}
    Analyser la continuité, la dérivabilité et le comportement au voisinage de 0.
\end{exercise}

\begin{exercise}
    Existence et calcul de 
    \begin{equation*}
        I_n=\int_{-\frac{1}{2}}^{+\infty}\frac{\d x}{(x^{2}+x+1)^{n}},
    \end{equation*}
    pour $n\geqslant1$.
\end{exercise}

\begin{exercise}
    Soit $f\colon[0,1]\to\R$ de classe $\mathcal{C}^{1}$. On pose, pour $n\in\N^{*}$,
    \begin{equation*}
        u_n=\frac{1}{n}\sum_{i=0}^{n-1}f\left(\frac{i}{n}\right)f'\left(\frac{i}{n}\right).
    \end{equation*}
    Déterminer $\lim\limits_{n\to+\infty}u_n$.
\end{exercise}

\begin{exercise}
    Soit $a,b>0$. Montrer que 
    \begin{equation*}
        \lim\limits_{x\to1}\int_{x^{a}}^{x^{b}}\frac{\d t}{\ln(t)}=\ln\left(\frac{b}{a}\right).
    \end{equation*}
\end{exercise}

\begin{exercise}
    Soit $\varphi$ convexe de $\R\to\R$ (donc continue). Soit $f\colon[a,b]\to\R$ continue avec $a<b$.
    \begin{enumerate}
        \item Montrer que
        \begin{equation*}
            \varphi\left(\frac{1}{b-a}\int_{a}^{b}f(t)\d t\right)\leqslant\frac{1}{b-a}\int_{a}^{b}\varphi(f(t))\d t.
        \end{equation*}

        \item On suppose de plus $\varphi$ strictement convexe, montrer que l'on a égalité dans ce qui précède si et seulement si $f$ est constante.
    \end{enumerate}
\end{exercise}

\begin{exercise}
    Trouver toutes les fonctions $f\colon\R\to\R$ continues telles que pour tout $(x,y)\in\R^{2}$, on a 
    \begin{equation*}
        \int_{x-y}^{x+y}f(t)\d t=f(x)f(y).
    \end{equation*}
\end{exercise}

\begin{exercise}
    Soit $f$ continue de $[a,b]$ dans $\R$ avec $a<b$. On pose 
    \begin{equation*}
        I_n=\left(\int_{a}^{b}\left\lvert f(t)\right\rvert^{n}\right)^{\frac{1}{n}}=\left\lVert f\right\rVert_{n}.
    \end{equation*}
    Déterminer $\lim\limits_{n\to+\infty}I_n$.
\end{exercise}

\begin{exercise}
    Soit $\theta\in\R$ et $\rho\in\R_{+}^{*}\setminus\left\lbrace1\right\rbrace$.
    \begin{enumerate}
        \item Montrer que $F(\rho,\theta)=\int_{-\pi}^{\pi}\ln\left\lvert\e^{\i t}-\rho\e^{\i\theta}\right\rvert\d t$ existe.
        \item Montrer que $F(\rho,\theta)$ ne dépend pas de $\theta$.
        \item Calculer $F(\rho,\theta)$ en utilisant une somme de Riemann sur $[0,2\pi]$.
    \end{enumerate}
\end{exercise}

\begin{exercise}
    On définit $C_{0}$ l'ensemble des fonctions continues à support compact de $\R$ dans $\R$, i.e. si $f\in C_{0}$, alors il existe $A\geqslant0$ tel que pour tout $\left\lvert t\right\rvert\geqslant A$, alors $f(t)=0$. On note $C_{1}$ l'ensemble des fonctions de $C_{0}$ de classe $\mathcal{C}^{1}$.
    \begin{enumerate}
        \item Soit $f\colon\R\to\R$ continue telle que pour tout $\varphi\in C_{0}$, $\int_{\R}f\varphi=0$. Montrer que $f=0$.
        \item Soit $f\colon\R\to\R$ continue telle que pour tout $\varphi\in C_{1}$, $\int_{\R}f\varphi'=0$. Montrer que $f$ est constante.
        \item Soit $f\in C_{0}$ telle qu'il existe $g\colon\R\to\R$ de classe $\mathcal{C}^{1}$ telle que pour tout $\varphi\in C_{1}$, $\int_{\R}f\varphi'=\int_{\R}g\varphi$. Montrer que $f$ est de classe $\mathcal{C}^{1}$ et $f'=-g$.
    \end{enumerate}
\end{exercise}

\begin{exercise}
    Existence et calcul de 
    \begin{equation*}
        I=\int_{0}^{+\infty}\frac{\e^{-t}-\e^{-2t}}{t}\d t,
    \end{equation*}
    et de 
    \begin{equation*}
        J=\int_{0}^{+\infty}\frac{\cos(t)-\cos(2t)}{t}\d t.
    \end{equation*}
\end{exercise}

\begin{exercise}
    On note \function{f}{]0,1]}{\R}{t}{\frac{1}{t}-\left\lfloor\frac{1}{t}\right\rfloor}
    Calculer 
    \begin{equation*}
        I=\int_{0}^{1}f(t)\d t.
    \end{equation*}
\end{exercise}

\begin{exercise}
    Existence de \function{f}{\R_{+}^{*}}{\R}{x}{\int_{x}^{+\infty}\frac{\d t}{\e^{t}-1}}
    Montrer que 
    \begin{equation*}
        I=\int_{0}^{+\infty}f(x)\d x,
    \end{equation*}
    est définie et donner sa valeur.
\end{exercise}

\begin{exercise}
    \phantom{}
    \begin{enumerate}
        \item Soit $n\geqslant1$, calculer 
        \begin{equation*}
            I_n=\int_{n}^{+\infty}\left(1+\frac{t}{n}\right)^{n}\e^{-t}\d t.
        \end{equation*}
        \item DOnner un équivalent en $+\infty$ de 
        \begin{equation*}
            J_n=\int_{-n}^{n}\left(1+\frac{t}{n}\right)^{n}\e^{-t}\d t.
        \end{equation*}
        On rappelle que $\int_\R\e^{-u^{2}}\d u=\sqrt{\pi}$.

        \item En déduire la formule de Stirling.
    \end{enumerate}
\end{exercise}

\begin{exercise}
    \phantom{}
    \begin{enumerate}
        \item Montrer que pour tout $x\in\R$, 
        \begin{equation*}
            I(x)=\int_{0}^{+\infty}\frac{1-\cos(tx)}{t^{2}}\e^{-t}\d t=\int_{0}^{+\infty}f_x(t)\d t,
        \end{equation*}
        est définie.
        \item Montrer que $I$ est de classe $\mathcal{C}^{2}$ sur $\R$.
        \item Calculer $I(x)$.
    \end{enumerate}
\end{exercise}

\begin{exercise}
    Montrer que pour tout $x\geqslant0$, on peut définir $f(x)=\int_{x}^{+\infty}\frac{\sin(t)}{t}\d t$. Prouver l'existence et calculer $I=\int_{0}^{+\infty}f(x)\d x$.
\end{exercise}

\begin{exercise}
    Soit $f\colon\R_{+}\to\R$ continue telle que $f(x)=\underset{x\to+\infty}{O}\left(\frac{1}{x^{2}}\right)$. On pose, pour $h>0$, 
    \begin{equation*}
        \phi(h)=\sum_{n=0}^{+\infty}hf(nh).
    \end{equation*}
    Calculer $\lim\limits_{h\to0^{+}}\phi(h)$.
\end{exercise}

\begin{exercise}
    Déterminer le domaine de définition et calculer 
    \begin{equation*}
        f(x)=\int_{0}^{+\infty}\frac{\sinh(xt)}{t}\e^{-t}\d t.
    \end{equation*}
\end{exercise}

\begin{exercise}
    Déterminer le domaine de définition et calculer 
    \begin{equation*}
        F(x)=\int_{0}^{+\infty}f(x,t)\d t=\int_{0}^{+\infty}\frac{\e^{t(\i x-1)}}{\sqrt{t}}\d t.
    \end{equation*}
\end{exercise}

\begin{exercise}[Transformée de Fourier]
    Soit $f$ continue, bornée de $\R$ dans $\C$ intégrable sur $\R$. On peut définir \function{\widehat{f}}{\R}{\C}{\nu}{\int_{-\infty}^{+\infty}f(t)\e^{-\i\nu t}\d t}
    On suppose que $\widehat{f}$ est intégrable sur $\R$ et on veut montrer que pour tout $x\in\R$,
    \begin{equation*}
        f(x)=\frac{1}{2\pi}\int_{-\infty}^{+\infty}\widehat{f}(\nu)\e^{\i\nu x}\d\nu.
    \end{equation*}

    \begin{enumerate}
        \item On définit, pour tout $\lambda\geqslant0$ et $x\in\R$,
        \begin{equation*}
            g_x(\lambda)=\frac{1}{2\pi}\int_{-\infty}^{+\infty}\widehat{f}(\nu)\e^{\i\nu x}\e^{-\lambda\left\lvert \nu\right\rvert}\d\nu.
        \end{equation*}
        Montrer que pour $\lambda>0$, on a 
        \begin{equation*}
            g_x(\lambda)=\frac{1}{2\pi}\int_{-\infty}^{+\infty}\frac{2\lambda}{\lambda^{2}+t^{2}}f(x+t)\d t.
        \end{equation*}
        \item Montrer que $\lim\limits_{\lambda\to0}g_x(\lambda)=f(x)$.
        \item Conclure.
    \end{enumerate}
\end{exercise}

\begin{exercise}[Intégrale de Dirichlet]
    \phantom{}
    \begin{enumerate}
        \item On forme, pour $n\in\N$, \function{D_n}{\R}{\C}{t}{\sum_{k=-n}^{n}\e^{\i kt}}
        Montrer que $D_n$ est paire, $\mathcal{C}^{\infty}$, $2\pi$-périodique, et calcule $\frac{1}{2\pi}\int_{0}^{2\pi}D_n(t)\d t$. Montrer que pour tout $t\in\R\setminus2\pi\Z$, on a 
        \begin{equation*}
            D_n(t)=\frac{\sin\left((2n+1)\frac{t}{2}\right)}{\sin\left(\frac{t}{2}\right)}.
        \end{equation*}

        \item On pose $u_n=\int_{0}^{(2n+1)\frac{\pi}{2}}\frac{\sin(t)}{t}\d t$. Montrer que 
        \begin{equation*}
            u_n=\int_{0}^{\pi}\frac{\sin\left((2n+1)\frac{u}{2}\right)}{u}\d u.
        \end{equation*}

        \item Montrer que l'on peut prolonger $u\mapsto \frac{1}{\sin\left(\frac{u}{2}\right)}-\frac{1}{\left(\frac{u}{2}\right)}$ en une fonction $\mathcal{C}^{1}$, notée $\varphi$, sur $[0,1]$.
        \item Calculer $\lim\limits_{n\to+\infty}\int_{0}^{\pi}\varphi(u)\sin\left((2n+1)\frac{u}{2}\right)\d u$. Conclure.
    \end{enumerate}
\end{exercise}

\begin{exercise}[Transformée de Laplace]
    Soit $f$ continue de $\R_{+}\to\R$ ou $\C$. On suppose qu'il existe $a\in\R$ tel que $\int_{0}^{t}f(t)\e^{-a t}\d t$ converge.
    \begin{enumerate}
        \item Montrer que pour tout $x>0$, on peut définir $Lf(a+x)=\int_{0}^{+\infty}f(t)\e^{-(a+x)t}\d t$ et que $Lf(a+x)=x\int_{0}^{+\infty}g(t)\e^{-xt}\d t$, où $g(t)=\int_{0}^{t}f(v)\e^{-av}\d v$.
        \item On suppose que pour tout $x\geqslant 0$, $Lf(a+x)=0$, montrer que $f=0$. On pourra montrer que $Lf(a+x)=x\int_{0}^{1}h(u)u^{x-1}\d u$, où $h\colon[0,1]\to\R$ ou $\C$ est continue.
    \end{enumerate}
\end{exercise}

\begin{exercise}
    Soit $(a_n)_{n\in\N}\in(\R_{+}^{*})^{\N}$ croissante telle que $\lim\limits_{n\to+\infty}a_n=+\infty$. Montrer que 
    \begin{equation*}
        \int_{0}^{+\infty}\sum_{n=0}^{+\infty}(-1)^{n}\e^{-a_n x}\d x=\int_{0}^{+\infty}\sum_{n=0}^{+\infty}g_n(x)\d x=\int_{0}^{+\infty}g(x)\d x=\sum_{n=0}^{+\infty}\frac{(-1)^{n}}{a_n}.
    \end{equation*}
\end{exercise}

\begin{exercise}
    Soit $(a_n)_{n\in\N}\in\left(\R_{+}^{*}\right)^{\N}$ une suite croissance telle que $\lim\limits_{n\to+\infty}a_n=+\infty$. Montrer que l'on a 
    \begin{equation*}
        \int_{0}^{+\infty}\underbrace{\sum_{n=0}^{+\infty}(-1)^{n}\e^{-a_n x}}_{S(x)}\d x=\sum_{n=0}^{+\infty}\frac{(-1)^{n}}{a_n}.
    \end{equation*}
\end{exercise}

\begin{exercise}
    Calculer $\sum_{n=1}^{+\infty}\frac{\sin(nx)}{3^{n}}$. En déduire $\int_{0}^{\pi}\frac{3\sin(x)}{5-3\cos(x)}\d x$.
\end{exercise}

\begin{exercise}
    Donner le domaine de définition et calculer
    \begin{equation*}
        I(x)=\int_{0}^{2\pi}\ln\left(x^{2}-2x\cos(t)+1\right)\d t.
    \end{equation*}
\end{exercise}

\begin{exercise}
    Montrer que \function{f}{\R}{\R}{x}{\int_{0}^{+\infty}\e^{-t^{2}}\sin(2xt)\d t}
    et \function{g}{\R}{\R}{x}{\int_{0}^{x}\e^{t^{2}}\d t\e^{-x^{2}}} sont définies et égales. Donner la limite de $f$ en $+\infty$.
\end{exercise}

\begin{exercise}
    Convergence et valeur de 
    \begin{equation*}
        I=\int_{1}^{+\infty}\left(\frac{1}{t}-\arcsin\left(\frac{1}{t}\right)\right)\d t.
    \end{equation*}
\end{exercise}

\end{document}