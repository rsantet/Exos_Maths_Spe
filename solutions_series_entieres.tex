\documentclass[12pt]{article}
\usepackage{style/style_sol}

\begin{document}

\begin{titlepage}
	\centering
	\vspace*{\fill}
	\Huge \textit{\textbf{Solutions MP/MP$^*$\\ Séries Entières}}
	\vspace*{\fill}
\end{titlepage}

\begin{proof}
    \phantom{}
    \begin{enumerate}
        \item On pose, pour tout $n\geqslant1$, $u_n=\left(\cosh\left(\frac{1}{n}\right)\right)^{n^{\alpha}}>0$. On va chercher un équivalent. On a $u_n=\e^{n^{\alpha}\ln\left(\cosh\left(\frac{1}{n}\right)\right)}$. Comme $\cosh(x)\underset{0}{=}1+\frac{x^{2}}{2}+O(x^{4})$, on a 
        \begin{align}
            \ln\left(\cosh\left(\frac{1}{n}\right)\right)
            &\underset{+\infty}{=}\ln\left(1+\frac{1}{2n^{2}}+O\left(\frac{1}{n^{4}}\right)\right),\\
            &\underset{+\infty}{=}\frac{1}{2n^{2}}+O\left(\frac{1}{n^{4}}\right).
        \end{align}

        Ainsi, $u_n\underset{+\infty}{=}\e^{\frac{n^{\alpha-2}}{2}+O\left(n^{\alpha-4}\right)}$. Donc :
        \begin{itemize}
            \item si $\alpha<2$, $\lim\limits_{n\to+^infty}u_n=1$ et $R=1$,
            \item si $\alpha=2$, $\lim\limits_{n\to+^infty}u_n=\e^{\frac{1}{2}}$ et $R=1$,
            \item si $\alpha>2$, on a 
            \begin{equation}
                \frac{u_{n+1}}{u_n}\underset{+\infty}{=}\e^{\left(\frac{(n+1)^{\alpha-2}}{2}\right)-\frac{n^{\alpha-2}}{2}+O\left(n^{\alpha-4}\right)}.
            \end{equation}
            Or
            \begin{align}
                \left(n+1\right)^{\alpha-2}-n^{\alpha-2}
                &= n^{\alpha-2}\left(\left(1+\frac{1}{n}\right)^{\alpha-2}-1\right),\\
                &\underset{+\infty}{=}n^{-2}\left(\frac{\alpha-2}{n}+O\left(\frac{1}{n^{3}}\right)\right),\\
                &\underset{+\infty}{=}\left(\alpha-2\right)n^{\alpha-3}+O\left(n^{\alpha-4}\right).
            \end{align}

            Donc $\frac{u_{n+1}}{u_{n}}\underset{+\infty}{=}\e^{\frac{(\alpha-2)n^{\alpha-3}}{2}+O\left(n^{\alpha-4}\right)}$.
            Ainsi,
            \begin{itemize}
                \item si $\alpha<3$, $\lim\limits_{n\to+\infty}\frac{u_{n+1}}{u_{n}}=1$ et $R=1$,
                \item si $\alpha=3$, $\lim\limits_{n\to+\infty}\frac{u_{n+1}}{u_{n}}=\e^{\frac{1}{2}}$ et $R=\e^{-\frac{1}{2}}$,
                \item si $\alpha>3$, comme $\frac{(\alpha-2)n^{\alpha-3}}{2}+O\left(n^{\alpha-4}\right)\underset{+\infty}{\sim}\frac{(\alpha-2)}{2}n^{\alpha-3}$, il existe $N_{0}\in\N$ tel que pour tout $n\geqslant N_{0}$,
                \begin{equation}
                    \frac{(\alpha-2)n^{\alpha-3}}{2}+O\left(n^{\alpha-4}\right)\geqslant \frac{\alpha-2}{4}n^{\alpha-3}\xrightarrow[+\infty]{}+\infty.
                \end{equation}
                Ainsi, $\lim\limits_{n\to+\infty}\frac{u_{n+1}}{u_{n}}=+\infty$ et $R=0$.
            \end{itemize}
        \end{itemize}

        \item On note $u_n=\e^{n^{2}\ln\left(1+\frac{(-1)^{n}}{n^{2}}\right)}>0$. Comme $\ln(1+x)\underset{0}{=}x+O(x^{2})$, on a $u_n\underset{+\infty}{=}\e^{(-1)^{n}n+O\left(\frac{1}{n}\right)}\underset{+\infty}{\sim}\e^{(-1)^{n}n}=v_n$. On ne peut pas appliquer la règle de d'Alembert à $v_n$, ça ne va pas converger. Mais on peut encadrer $v_n$ : $0<v_n\leqslant\e^{n}$ et donc $R\geqslant\frac{1}{\e}$. On a $\frac{u_n}{\e^{n}}\underset{+\infty}{=}\e^{n\left((-1)^{n}-1\right)+O\left(\frac{1}{n}\right)}$ et $\frac{u_{2n}}{\e^{2n}}\xrightarrow[n\to+\infty]{}1$ donc $\sum \frac{u_{n}}{\e^{n}}$ diverge. Ainsi, $R=\frac{1}{\e}$.
    \end{enumerate}
\end{proof}

\begin{proof}
    \phantom{}
    \begin{enumerate}
        \item On remarque
        \begin{equation}
            \underbrace{\begin{pmatrix}
                u_n\\
                u_{n+1}\\
                \dots\\
                u_{n+p-1}
            \end{pmatrix}}_{X_n}
            =
            \underbrace{
                \begin{pmatrix}
                    m_1 & \dots & m_p\\
                    m_{1}\e^{\i\theta_{1}} & \dots & m_p\e^{\i\theta_{p}}\\
                    \vdots & & \vdots\\
                    m_1\e^{\i(p-1)\theta_p} & & m_p\e^{\i(p-1)\theta_p}
                \end{pmatrix}
            }_{A}
            \underbrace{
                \begin{pmatrix}
                    \e^{\i n\theta_1}\\
                    \vdots\\
                    \e^{\i n\theta_p}
                \end{pmatrix}
            }_{Y_n}.
        \end{equation}

        $A$ est inversible car $\det(A)=\left(\prod_{i=1}^{p}m_{i}\right)\times\text{VdM}(\e^{\i\theta_1},\dots,\e^{\i\theta_p})\neq0$. Donc si $u_n\xrightarrow[n\to+\infty]{}0$, on a $X_n\xrightarrow[n\to+\infty]{}0$ et $Y_n=A^{-1}X_{n}\xrightarrow[n\to+\infty]{}0$ ce qui n'est pas car $\left\lVert Y_n\right\rVert_{\infty}=1$.

        \item On pose $\rho(A)=\max\limits_{\lambda\in\Sp(A)}\left\lvert\lambda\right\rvert$. Si $\chi_{A}=(X-\lambda_{1})\dots(X-\lambda_{p})$ avec $\left\lvert \lambda_{1}\right\rvert=\dots=\left\lvert\lambda_{j}\right\rvert=\rho(A)$ et $\left\lvert\lambda_{i}\right\rvert<\rho(A)$ pour tout $i\in\left\lbrace j+1,\dots,p\right\rbrace$. On écrit $a_n=\sum_{i=1}^{p}\lambda_{i}^{n}=\sum_{i=1}^{j}\lambda_{i}^{n}+\sum_{i=j+1}^{p}\lambda_{i}^{n}$. D'après la règle de d'Alembert, on a $R\geqslant\frac{1}{\rho(A)}$ (et $R=+\infty$ si $\rho(A)=0$ et $A$ est nilpotente). De plus, on a 
        \begin{equation}
            \frac{a_n}{\rho(A)^{n}}=\sum_{k=1}^{j}m_{k}\e^{\i n\theta_k}+\sum_{i=j+1}^{p}\left(\frac{\lambda_{i}}{\rho(A)}\right)^{n},
        \end{equation}
        et le premier terme ne tend pas vers 0 d'après ce qui précède tandis que le deuxième tend vers 0. Donc $\sum \frac{a_{n}}{\rho(A)}$ diverge grossièrement, donc $R=\frac{1}{\rho(A)}$.

        Soit $z\in\C$ avec $\left\lvert z\right\rvert<\frac{1}{\rho(A)}$, on a 
        \begin{align}
            \sum_{n=0}^{+\infty}a_{n}z^{n}
            &=\sum_{i=1}^{p}\left(\sum_{n=0}^{+\infty}\lambda_{i}^{n}z^{n}\right),\\
            &=\sum_{i=1}^{p}\frac{1}{1-\lambda_{i}z},\\
            &=\Tr\left(I_{p}-z A\right)^{-1},
        \end{align}
        car pour tout $i\in\left\llbracket1,p\right\rrbracket,\left\lvert \lambda_{i}z\right\rvert<1$ et on peut trigonaliser dans la même base $A$ et $I_{p}-zA$.
    \end{enumerate}
\end{proof}

\begin{proof}
    D'après la règle de d'Alembert, on a $R=1$. De plus, $\left\lvert a_n\right\rvert=\underset{+\infty}{O}\left(\frac{1}{n^{3}}\right)$ donc il y a convergence uniforme sur $\overline{D(0,1)}$. Ainsi, la somme $S$ est continue sur $\overline{D(0,1)}$. Soit $t\in]-1,1[$, comme $\frac{1}{X(X+1)(2X+1)}=\frac{a}{X}+\frac{b}{X+1}+\frac{c}{2X+1}$ avec $a=b=1$ et $c=4$, on a 
    \begin{equation}
        \frac{S(t)}{6}=\sum_{n=1}^{+\infty}\left(\frac{t^{n}}{n}+\frac{t^{n}}{n+1}-4\frac{t^{n}}{2n+1}\right)=-\ln(1-t)+\left(\frac{-\ln(1-t)}{t}-1\right)-4\underbrace{\sum_{n=1}^{+\infty}\frac{t^{n}}{2n+1}}_{g(t)}.
    \end{equation}

    Si $t>0$, on a $\sqrt{t}g(t)=\sum_{n=1}^{+\infty}\frac{(\sqrt{t})^{2n+1}}{2n+1}$. On pose $h(x)=\sum_{n=1}^{+\infty}\frac{x^{2n+1}}{2n+1}$, $\mathcal{C}^{\infty}$ sur $[0,1[$ et $h(0)=0$. On a $h'(x)=\sum_{n=1}^{+\infty}x^{2n}=\frac{x^{2}}{1-x^{2}}=-1-\frac{1}{2}\frac{1}{x+1}+\frac{1}{2}\frac{1}{x-1}$ donc $h(x)=-x-\frac{1}{2}\ln(x+1)+\frac{1}{2}\ln(x-1)$ d'où $g(t)=-\sqrt{t}-\frac{1}{2}\ln(\sqrt{t}+1)+\frac{1}{2}\ln(\sqrt{t}-1)$.

    Si $t<0$, $\sqrt{-t}g(t)=\sum_{n=1}^{+\infty}\frac{(-1)^{n}\sqrt{-t}^{2n+1}}{2n+1}=\arctan(\sqrt{-t})-\sqrt{-t}$. Donc $g(t)=\frac{\arctan(\sqrt{-t})}{\sqrt{-t}}-1$. L'expression de $S$ reste valable en -1 et 1 par continuité de $S$.
\end{proof}

\begin{proof}
    Soit $t\in]-1,1[$, on a 
    \begin{align}
        I(t)
        &=\int_{0}^{1}\e^{u\ln(1+t)}\d u,\\
        &=\left[\frac{1}{\ln(1+t)}\e^{u\ln(1+t)}\right]_{u=0}^{u=1},\\
        &=\frac{1+t}{\ln(1+t)}-\frac{1}{\ln(1+t)},\\
        &=\frac{t}{\ln(1+t)}=f(t).
    \end{align}

    Soit $u\in[0,1]$, on a $(1+t)^{u}=\sum_{n=0}^{+\infty}\frac{u(u-1)\dots(u-n+1)}{n!}t^{n}=\sum_{n=0}^{+\infty}f_n(u)$. $f_n$ est continue sur $[0,1]$. On a 
    \begin{align}
        \left\lvert f_n(u)\right\rvert
        &=\frac{u(1-u)\dots(n-1-u)}{n!}\left\lvert t\right\rvert^{n},\\
        &\leqslant\frac{(n-1)!}{n!}\left\lvert t\right\rvert^{n},\\
        &=\frac{1}{n}\left\lvert t\right\rvert^{n},\\
        &\leqslant\left\lvert t\right\rvert^{n},
    \end{align}
    car pour tout $u\in[0,1]$, $0\leqslant k-u\leqslant k$. Comme $\left\lvert t\right\rvert<1$, $\left\lvert t\right\rvert^{n}$ est le terme général d'une série convergente. Donc $\sum f_{n}$ converge normalement sur $[0,1]$ et on peut intervertir :
    \begin{equation}
        f(t)=\sum_{n=0}^{+\infty}\underbrace{\left(\int_{0}^{1}\frac{u(u-1)\dots(u-n+1)}{n!}\d u\right)}_{a_n}t^{n},
    \end{equation}
    encore vrai pour $t=0$ car $a_{0}=1$. Donc $f$ est développable en série entière sur $]-1,1[$ et $f$ est $\mathcal{C}^{\infty}$ sur $]-1,1[$. Par ailleurs, $f$ est $\mathcal{C}^{\infty}$ sur $\left[\frac{1}{2},+\infty\right[$, donc $f$ est $\mathcal{C}^{\infty}$ sur $]-1,+\infty[$.
\end{proof}

\begin{remark}
    On a 
    \begin{equation}
        a_n=\frac{(-1)^{n}}{n!}\int_{0}^{1}u(1-u)\dots(n-1-u)\d u.
    \end{equation}
    De plus,
    \begin{align}
        \left\lvert a_{n+1}\right\rvert
        &=\frac{1}{(n+1)!}\int_{0}^{1}u(1-u)\dots(n-1-u)\underbrace{(n-u)}_{\leqslant n}\d u,\\
        &\leqslant \frac{1}{n!}\int_{0}^{1}u(1-u)\dots(n-1-u)\d u=\left\lvert a_n\right\rvert.
    \end{align}

    Enfin, $\left\lvert a_n\right\rvert\leqslant\frac{(n-1)!}{n!}=\frac{1}{n}$. D'après le critère spécial des séries alternées, $\sum a_{n}$ converge. Puis $\sum a_{n}t^{n}$ converge uniformément sur $[0,1]$ (majorer le reste par le critère spécial des séries alternées), donc il y a convergence et continuité en 1. On vérifie que
    \begin{equation*}
        \left\lvert a_n\right\rvert=\frac{1}{n}\int_{0}^{1}u\e^{\sum_{k=1}^{n-1}\ln\left(1-\frac{u}{k}\right)}\d u=\frac{1}{n}\int_{0}^{1}\e^{-\ln(n)u+g_n(u)}
    \end{equation*}, où $g_n(u)$ est majorée par $M$ indépendant de $n$ et de $u$. Ainsi, par convergence dominée, $\left\lvert a_n\right\rvert\underset{+\infty}{\sim}\frac{1}{n}\int_{0}^{1}\frac{u}{n^{u}}\d u$, terme général d'une série divergente.
\end{remark}

\begin{proof}
    On a $a_n=\e^{\ln(n)\ln\left(\sum_{k=1}^{n}\frac{1}{k}\right)}\underset{+\infty}{=}\e^{\ln(n)\ln\left(\ln(n)+\gamma+o(1)\right)}$.
    On a 
    \begin{align}
        \ln\left(\ln(n)+\gamma+o(1)\right)
        &=\ln(\ln(n))+\ln\left(1+\frac{\gamma}{\ln(n)}+O\left(\frac{1}{\ln(n)}\right)\right),\\
        &=\ln(\ln(n))+\frac{\gamma}{\ln(n)}+o\left(\frac{1}{\ln(n)}\right).
    \end{align}

    Donc $a_n=\e^{\ln(n)\ln(\ln(n))+\gamma+o(1)}\underset{+\infty}{\sim}\e^{\gamma}\underbrace{\e^{\ln(n)\ln(\ln(n))}}_{b_n}$. On a 
    \begin{equation}
        \frac{b_{n+1}}{b_n}=\e^{\ln(n+1)\ln(\ln(n+1))-\ln(n)\ln(\ln(n))},
    \end{equation}
    mais 
    \begin{equation}
        \ln(n+1)=\ln(n)+\ln\left(1+\frac{1}{n}\right)\underset{+\infty}{=}\ln(n)+O\left(\frac{1}{n}\right),    
    \end{equation}
    et
    \begin{equation}
        \ln(n+1)\ln(\ln(n+1))=\ln(n)\ln(\ln(n+1))+\underbrace{O\left(\frac{\ln(\ln(n+1))}{n}\right)}_{=o(1)\xrightarrow[+\infty]{}0},
    \end{equation}
    puis
    \begin{align}
        \ln(\ln(n+1))
        &\underset{+\infty}{=}\ln\left(\ln(n)+O\left(\frac{1}{n}\right)\right),\\
        &\underset{+\infty}{=}\ln(\ln(n))+\ln\left(1+O\left(\frac{1}{n\ln(n)}\right)\right),\\
        &\underset{+\infty}{=}\ln(\ln(n))+O\left(\frac{1}{n\ln(n)}\right).
    \end{align}

    Donc $\ln(n+1)\ln(\ln(n+1))-\ln(n)\ln(\ln(n))=\underset{+\infty}{o}(1)$, et $\frac{b_{n+1}}{b_{n}}\lim\limits_{n\to+\infty}1$, d'où $R=1$.

    De plus, $\lim\limits_{n\to+\infty}a_n=+\infty$ donc il y a divergence sur le cercle de convergence.
\end{proof}

\begin{remark}
    On peut aussi écrire $a_n\leqslant n^{\ln(n)}=\e^{(\ln(n))^{2}}=c_n$, et 
    \begin{equation}
        \frac{c_{n+1}}{c_{n}}=\e^{\left(\ln(n+1)\right)^{2}-\left(\ln(n)\right)^{2}}\underset{+\infty}{=}\e^{\left(\ln(n)+O\left(\frac{1}{n}\right)\right)^{2}-(\ln(n))^{2}}\xrightarrow[n\to+\infty]{}1.
    \end{equation}
    Donc $\sum c_{n}z^{n}$ a pour rayon de convergence 1, donc $R\geqslant1$, et $\sum a_{n}$ diverge donc $R=1$.
\end{remark}

\begin{proof}
    Le nombre de diviseurs est compris entre 1 et $n$. Comme $\sum z^{n}$ et $\sum nz^{n}$ ont un rayon de convergence égal à 1, on a $R=1$ par encadrement.
\end{proof}

\begin{proof}
    On pose $u_n=\frac{a_{n+1}}{a_{n}}$. Alors $\frac{a_{n-1}a_{n+1}}{a_{n}^{2}}=\frac{u_{n}}{u_{n-1}}$. 
    \begin{itemize}
        \item Si $l<1$, alors d'après la règle de d'Alembert, $\sum u_{n}$ converge donc $\lim\limits_{n\to+\infty}u_{n}=0$ donc $R=+\infty$.
        \item Si $l>1$, il existe $N_{0}\in\N$ tel que pour tout $n\geqslant N_{0}$, $\frac{u_{n}}{u_{n-1}}\geqslant\frac{l+1}{2}$ et pour tout $n\geqslant N_{0}$, $u_{n}\geqslant u_{N_{0}}\times\left(\frac{l+1}{2}\right)^{n-N_{0}}\xrightarrow[n\to+\infty]{}+\infty$ donc $R=0$.
        \item Si $l=1$ : si $a_n=n!$, on a $u_n=n+1$ donc $R=0$, si $a_n=\frac{1}{n!}$, on a $u_n=\frac{1}{n+1}$ donc $R=+\infty$, si $a_n=\lambda^{n}$ avec $\lambda>0$, on a $u_n=\lambda$ et $R=\frac{1}{\lambda}$. Donc on ne peut rien dire.
    \end{itemize}
\end{proof}

\begin{proof}
    D'après la règle de d'Alembert, avec $a_{n}=\frac{(-1)^{n}}{n}$, on a $\left\lvert \frac{a_{n+1}}{a_{n}}\right\rvert=\frac{n}{n+1}\xrightarrow[n\to+\infty]{}1$ donc le rayon de convergence de $\phi$ est $R=1$ donc $\phi$ est bien définie.

    Fixons $z\in\C^{*}$ avec $\left\lvert z\right\rvert<1$, formons \function{f}{[0,1]}{\C}{t}{\e^{\phi(tz)}}
    $z$ étant fixé, le rayon de convergence de la série entière $\sum_{n\geqslant1}(-1)^{n-1}\frac{t^{n}z^{n}}{n}=\phi(tz)$ vaut $\frac{1}{\left\lvert z\right\rvert}>1$, donc l'application $t\mapsto\phi(tz)$ est $\mathcal{C}^{\infty}$ sur $[0,1]\subset\left]-\frac{1}{\left\lvert z\right\rvert},\frac{1}{\left\lvert z\right\rvert}\right[$. $f$ est donc $\mathcal{C}^{\infty}$ sur $[0,1]$ et pour tout $t\in[0,1]$,
    \begin{equation}
        f'(t)=\sum_{n=1}^{+\infty}(-1)^{n-1}z^{n}t^{n-1}\times f(t)=\frac{z}{1+tz}f(t),
    \end{equation}
    car $\left\lvert zt\right\rvert<1$ et $f(0)=1$. On pose $g(t)=1+tz$. Alors $g'(t)=z=\frac{z}{1+tz}g(t)$ et $g(0)=1$. Ainsi, par unicité (d'après le théorème de Cauchy-Lipschitz), pour tout $t\in[0,1]$, $f(t)=g(t)$. En particulier, $f(1)=\e^{\phi(z)}=1+z$.
\end{proof}

\begin{remark}
    On vient de définir, pour $\left\lvert z\right\rvert<1$, $\phi(z)$ qui est un logarithme complexe continue de $1+z$. Si $1+z=\rho\e^{\i\theta}$ avec $\theta\in]-\pi,\pi[$, $\phi(z)=\ln(\rho)+\i\theta$.
\end{remark}

\begin{proof}
    On a $a_n=\frac{1}{\cos\left(\frac{2n\pi}{3}\right)}$ et $1\leqslant\left\lvert a_n\right\rvert\leqslant2$ donc $R=1$. Si $\left\lvert z\right\rvert<1$, on a 
    \begin{equation}
        \sum_{n=0}^{+\infty}z^{3n}-2\left(\sum_{n=0}^{+\infty}-z^{3n+1}+z^{3n+2}\right)=\frac{1}{1-z^{3}}+\frac{2z}{1-z^{3}}-\frac{2z^{2}}{1-z^{3}}=\frac{1+2z-2z^{2}}{1-z^{3}}.
    \end{equation}
\end{proof}

\begin{proof}
    \phantom{}
    \begin{enumerate}
        \item On a $b_n\geqslant0$ donc $g$ est croissante sur $[0,1[$. $g$ admet donc une limite $l\in\overline{\R}_{+}$ en $1^{-}$. Pour tout $x<1$, $g(x)\leqslant l$. Pour tout $N\in\N$, pour tout $x\in[0,1[$, comme $b_nx^{n}\geqslant0$, on a $\sum_{n=0}^{N}b_{n}x^{n}\leqslant g(x)\leqslant l$. $N$ étant fixé, quand $x\to1$, on a $\sum_{n=0}^{N}b_n\leqslant l$ et quand $N\to+\infty$, on a $l=+\infty$.
        \item Soit $\varepsilon>0$, il existe $n_0\in\R$, pour tout $n\geqslant n_0$, $\left\lvert a_n-b_n\right\rvert<\frac{\varepsilon}{2}\times b_n$. Pour tout $x\in[0,1[$, on a $\left\lvert f(x)-g(x)\right\rvert\leqslant\sum_{n=0}^{n_{0}-1}\left\lvert a_n-b_n\right\rvert x^{n}$ + $\sum_{n=n_0}^{+\infty}\left\lvert a_n-b_n\right\rvert x^{n}$. Le terme de gauche est en polynôme en $x$ qui a une limite finie en $1^{-}$, le terme de droite majoré par $\frac{\varepsilon}{2}\sum_{n=n_0}^{+\infty}b_nx^{n}\leqslant\frac{\varepsilon}{2}g(x)$, car les $b_n$ sont positifs. Ainsi, ce terme de droite est un $\underset{x\to1^{-}}{O}(g(x))$ donc majoré par $\frac{\varepsilon}{2}g(x)$ pour $x$ suffisamment proche de 1, d'où $\left\lvert f(x)-g(x)\right\rvert\leqslant\varepsilon g(x)$ et $f(x)\underset{1^{-}}{\sim}g(x)$.
        \item On a $n^{p}\underset{+\infty}{\sim}n(n-1)\dots(n-p+1)$, donc 
        \begin{equation}
            h_p(x)\underset{1}{\sim}\sum_{n=0}^{+\infty}n(n-1)\dots(n-p+1)x^{n}=\sum_{n=p}^{+\infty}n(n-1)\dots(n-p+1)x^{n},
        \end{equation}
        et $f(x)=\frac{1}{1-x}=\sum_{n=0}^{+\infty}x^{n}$, $f'(x)=\frac{1}{(1-x)^{2}}=\sum_{n=1}^{+\infty}nx^{n-1}$. De proche en proche, on a $f^{(p)}(x)=\frac{p!}{(1-x)^{p+1}}=\sum_{n=p}^{+\infty}n\dots(n-p+1)x^{n-p}$, d'où 
        \begin{equation}
            \boxed{
                h_p(x)\underset{1}{\sim}\frac{p!}{(1-x)^{p+1}}.
            }
        \end{equation}
    \end{enumerate}
\end{proof}

\begin{proof}
    Soit $\varepsilon>0$. Il existe $N_{0}\in\N$ tel que pour tout $n\geqslant N_{0}$, $\left\lvert a_n\right\rvert\leqslant\frac{\varepsilon}{n}$. Alors si $S_n=\sum_{h=0}^{n}a_h$, on a 
    \begin{equation}
        \left\lvert S_n-S\right\rvert\leqslant \left\lvert S_n-f\left(1-\frac{1}{n}\right)\right\rvert+\left\lvert f\left(1-\frac{1}{n}\right)-S\right\rvert.
    \end{equation}
    Puisque $f\left(1-\frac{1}{n}\right)\xrightarrow[n\to+\infty]{}S$, il existe $N_{1}\in\N$ tel que pour tout $n\geqslant N_{1}$, $\left\lvert f\left(1-\frac{1}{n}\right)-S\right\rvert\leqslant\frac{\varepsilon}{4}$. Pour $n\geqslant N_{0}$, on a alors 
    \begin{equation}
        \left\lvert S_n-f\left(1-\frac{1}{n}\right)\right\rvert\leqslant A_n+B_n+C_n,
    \end{equation}
    avec $A_n=\sum_{h=0}^{N_{0}}\left\lvert a_h\right\rvert\left(1-\left(1-\frac{1}{n}\right)^{h}\right)\xrightarrow[n\to+\infty]{}0$ et il existe $N_{1}$ pour tout $n\geqslant N_{1}$, $A_n\leqslant\frac{\varepsilon}{4}$. On a 
    \begin{align}
        B_n
        &= \sum_{h=N_{0}+1}^{n}\left\lvert a_{h}\right\rvert\left(1-\left(1-\frac{1}{n}\right)^{h}\right),\\
        &\leqslant\frac{\varepsilon}{4}\sum_{h=N_{0}+1}^{n}\left(\frac{1}{h}\times h\left(1-\left(1-\frac{1}{n}\right)\right)\right),\\
        &\leqslant\frac{\varepsilon}{4}\sum_{h=N_{0}}^{n}\frac{1}{n},\\
        &\leqslant\frac{\varepsilon}{4}\times\frac{n-N_{0}}{n},\\
        &\leqslant\frac{\varepsilon}{4}.
    \end{align}
    Cela est dû au fait que $x\mapsto1-x^{h}$ est concave sur $[0,1]$ donc 
    \begin{equation*}
        \left(1-\left(1-\frac{1}{n}\right)^{h}\right)\leqslant h\left(1-\left(1-\frac{1}{n}\right)\right),
    \end{equation*}(ou par accroissement fini). Enfin, on a 
    \begin{align}
        C_n
        &=\sum_{h\geqslant n}a_h\left(1-\frac{1}{n}\right)^{h},\\
        &\leqslant\frac{\varepsilon}{4}\sum_{h\geqslant n}\frac{\left(1-\frac{1}{n}\right)^{h}}{h},\\
        &\leqslant\frac{\varepsilon}{4n}\sum_{h\geqslant n}\left(1-\frac{1}{n}\right)^{h},\\
        &\leqslant\frac{\varepsilon}{4n}\sum_{h=0}^{+\infty}\left(1-\frac{1}{n}\right)^{h},\\
        &=\frac{\varepsilon}{4}.
    \end{align}

    Ainsi, on a $\left\lvert S_n-S\right\rvert\leqslant\varepsilon$ et donc $S_n\xrightarrow[n\to+\infty]{}S$.
\end{proof}

\begin{remark}
    C'est une réciproque du lemme d'Abel radial i.e.~si $\sum a_n$ converge alors \begin{equation}
        \lim\limits_{x\to1^{-}}\sum_{n=0}^{+\infty}a_nx^{n}=\sum_{n=0}^{+\infty}a_n.
    \end{equation}
\end{remark}

\begin{remark}
    Ce n'est pas valable par exemple pour $a_n=(-1)^{n}$, car $f(x)=\frac{1}{1+x}\xrightarrow[x\to1^{-}]{}\frac{1}{2}$ mais $\sum(-1)^{n}$ diverge.
\end{remark}

\begin{proof}
    On note $f(z)=\sum_{n=0}^{+\infty}a_{n}z^{n}$ avec $a_0=f(0)=\rho\e^{\i\theta}\neq0$. Alors
    \begin{equation}
        f(z)=f(0)\left(1+\underbrace{\sum_{n=1}^{+\infty}\frac{a_n}{a_0}z^{n}}_{=g(z)}\right),
    \end{equation}
    avec $g(z)\xrightarrow[z\to0]{}0$ car $g$ est somme d'une série entière donc continue. Il existe $r>0$, si $\left\lvert z\right\rvert<r$, $\left\lvert g(z)\right\rvert<1$. Alors on a vu, d'après l'Exercice 8, que l'on a 
    \begin{equation}
        f(z)=\exp\left(\ln\rho+\i\theta+\sum_{p=1}^{+\infty}(-1)^{p-1}\frac{g(z)^{p}}{p}\right).
    \end{equation}
    Pour $p\in\N$ fixé, on peut développer chaque terme $g(z)^{p}=\sum_{n=0}^{+\infty}a_{n,p}z^{n}$ (produit de Cauchy). On vérifie alors (théorème de Fubini) que l'on peut intervertir les sommations.
\end{proof}

\begin{remark}
    Autre méthode : si $T$ existe avec $T(z)=\sum_{n=0}^{+\infty}b_nz^{n}$. Pour $t\in]-r,r[$, on a $f(t)=\e^{T(t)}$. En dérivant, on a $f'(t)=T'(t)f(t)=\left(\sum_{n}(n+1)b_{n+1}t^{n}\right)f(t)$. Par unicité de développement, et par produit de Cauchy, pour tout $n\in\N$, on a 
    \begin{align}
        (n+1)a_{n+1}
        &=\sum_{h=0}^{n}(h+1)b_{h+1}a_{n-h},\\
        &=(n+1)b_{n+1}\underbrace{a_{0}}_{\neq0}+\sum_{h=1}^{n}hb_{h}a_{n-h+1}.
    \end{align}
    On a $b_{0}=T(0)$, on choisit $b_{0}$ tel que $\e^{b_{0}}=a_{0}\neq0$ et on définit univoquement $(b_n)_{n\in\N}$ par récurrence. On vérifie alors, en majorant, que $\sum b_{n}z^{n}$ a un rayon de convergence $r>0$ (montrer qu'il existe $M\geqslant0,A\geqslant0$ tels que pour tout $n\in\N$, $\left\lvert b_n\right\rvert AM^{n}$). Alors $f'(t)=T'(t)f(t)$ et en posant $g(t)=\e^{T(t)}$, on a $g=f$ par unicité via le théorème de Cauchy-Lipschitz.
\end{remark}

\begin{proof}
    \phantom{}
    \begin{enumerate}
        \item Pour tout $n\geqslant1$, on a $\left\lvert\frac{1}{\sin(n\pi a)}\right\rvert\geqslant1$, donc $R_a\leqslant1$.
        \item On rappelle que si $a$ est irrationnel algébrique de degré $d\geqslant2$, il existe $C>0$ tel que pour tout $\frac{p}{q}\in\Q$, on a $\left\lvert a-\frac{p}{q}\right\rvert\geqslant\frac{C}{q^{d}}$. Soit $n\in\N^{*}$. On fixe $p\in\N$ tel que $n\pi a-p\pi\in\left]-\frac{\pi}{2},\frac{\pi}{2}\right[$. On a alors
        \begin{align}
            \left\lvert \sin(n\pi a)\right\rvert
            &=\left\lvert\sin(n\pi a-p\pi)\right\rvert,\\
            &\geqslant\frac{2}{\pi}\left\lvert n\pi a-p\pi\right\rvert,\\
            &\geqslant2\left\lvert na-p\right\rvert,\\
            &\geqslant2n\frac{C}{n^{d}}=\frac{2C}{n^{d-1}},
        \end{align}
        car par concavité, on a pour tout $t\in\left[-\frac{\pi}{2},\frac{\pi}{2}\right],\left\lvert\sin(t)\right\rvert\geqslant\frac{2}{\pi}\left\lvert t\right\rvert$. On a donc $\left\lvert a_n\right\rvert\leqslant\frac{n^{d-1}}{2C}$, et comme le rayon de convergence de $\sum \frac{n^{d-1}}{2C}z^{d-1}$ vaut 1, on a $R_a=1$.
        \item On a $\left\lvert\sin(n!\pi e)\right\vert=\left\lvert\sin\left(n!\pi\sum_{k=0}^{+\infty}\frac{1}{k!}\right)\right\rvert\underset{+\infty}{=}\left\lvert\sin\left(\frac{\pi}{n+1}+O\left(\frac{1}{n^{2}}\right)\right)\right\rvert\underset{+\infty}{\sim}\frac{\pi}{n}$. Pour $x\in]0,1]$, $\sum nx^{n!}$ converge. L'idée est donc de former $a$ tel que pour tout $x\in]0,1]$, on puisse extraire 
        \begin{equation}
            \left(\frac{x^{\sigma(n)}}{\sin\left(\sigma(n)\pi a\right)}\right)_{n\in\N},
        \end{equation} 
        qui ne tend pas vers 0.
        
        \begin{lemma}
            \label{lem:serie_entiere_1}
            Soit $(a_n)_{n\in\N}\in\left(\N^{*}\right)^{\N}$ strictement croissante, et 
            \begin{equation}
                a=\sum_{n=0}^{+\infty}\frac{1}{a_0\dots a_n}.
            \end{equation}
            On a 
            \begin{equation}
                a-\sum_{k=0}^{N}\frac{1}{a_0\dots a_k}\underset{N\to+\infty}{\sim}\frac{1}{a_0\dots a_{N+1}}.
            \end{equation}
        \end{lemma}
        \begin{proof}[Preuve du Lemme~\ref{lem:serie_entiere_1}]
            On a pour tout $n\in\N^{*}$, $\frac{1}{a_{0}\dots a_{n}}\leqslant\frac{1}{a_{0}a_{1}^{n}}$ et $a_{1}\geqslant2$ donc 
            \begin{equation*}
                \sum_{n\geqslant0}\frac{1}{a_{0}\dots a_{n}}    
            \end{equation*}
            converge. On a
            \begin{equation}
                \left\lvert a-\sum_{n=0}^{N}\frac{1}{a_{0}\dots a_{n}}\right\rvert=\sum_{k=N+1}^{+\infty}\frac{1}{a_0\dots a_k},
            \end{equation}
            donc 
            $\frac{1}{a_0\dots a_{N+1}}\leqslant \sum_{k=N+1}^{+\infty}\frac{1}{a_0\dots a_k}\leqslant\frac{1}{a_0\dots a_N}\sum_{k=1}^{+\infty}\frac{1}{a_{N+1}^{k}}=\frac{1}{a_0\dots a_N}\times\frac{1}{a_{N+1}}\times\frac{1}{1-\frac{1}{a_{N+1}}}$. Donc 
            \begin{equation}
                a-\sum_{k=0}^{N}\frac{1}{a_0\dots a_k}\underset{N\to+\infty}{\sim}\frac{1}{a_0\dots a_{N+1}}.
            \end{equation}
        \end{proof}

        On a donc $(a_{0}\dots a_{N})a-\underbrace{(a_0\dots a_{N})\sum_{k=0}^{N}\frac{1}{a_{0}\dots a_{k}}}_{\in\N}\underset{N\to+\infty}{\sim}\frac{1}{a_{N+1}}$. Ainsi,
        \begin{equation}
            \left\lvert\sin\left(\underbrace{(a_0\dots a_N)}_{=\sigma(N)}\pi a\right)\right\rvert=\left\lvert\sin\left((a_0\dots a_N)\pi a-(a_0\dots a_N)\sum_{k=0}^{N}\frac{\pi}{a_{0}a_{k}}\right)\right\rvert\underset{N\to+\infty}{\sim}\frac{\pi}{a_{N+1}}.
        \end{equation}
        Pour $x\in]0,1]$, on a $\frac{x^{\sigma(N)}}{\left\lvert\sin(\sigma(N)\pi a)\right\rvert}\underset{N\to+\infty}{\sim}\frac{1}{\pi}\exp\left(\sigma(N)\ln(x)+\ln(a_{N+1})\right)$. Il suffit de choisir $a_{N+1}$ tel que $\ln(a_{N+1})\geqslant N(a_{0}\dots a_{N})$, par exemple $a_{N+1}=\left\lfloor \e^{N(a_{0}\dots a_{N})}\right\rfloor+1$. Donc pour tout $x\in]0,1]$, $\lim\limits_{N\to+\infty}\frac{x^{\sigma(N)}}{\left\lvert\sin(\sigma(N)\pi a)\right\rvert}=+\infty$. Ainsi, $R_a=0$.
    \end{enumerate}
\end{proof}

\begin{proof}
    Pour $\left\lvert z\right\rvert<1$, par produit de Cauchy, ces séries sont définies et absolument convergentes, par sommabilité,
    \begin{equation}
        \left(\sum_{p_1=0}^{+\infty}z^{a_1 p_1}\right)\times\dots\times\left(\sum_{p_N=0}^{+\infty}z^{a_Np_N}\right)-\frac{1}{(1-z^{a_1})\dots(1-z^{a_N})}=\sum_{(p_1,\dots,p_N)\in\N^{N}}z^{a_1p_1+\dots+a_Np_N}.
    \end{equation}
    Par associativité, on regroupe selon les valeurs de l'exposant et on note l'expression précédente $\sum_{n=0}^{+\infty}c_nz^{n}$. On factorise la fraction rationnelle [les pôles sont des racines de l'unité] :
    \begin{equation}
        \frac{1}{\prod_{\xi\in\U}(z-\xi)^{m\left(\xi\right)}},
    \end{equation}
    avec $m(1)=N$, $m\left(\xi\right)<N$ si $\xi\neq1$ car $a_1\wedge\dots\wedge a_N=1$ : si $\xi^{a_1}=\dots=\xi^{a_N}=1$, l'ordre de $\xi$ divise $a_1,\dots,a_N$ donc divise $a_1\wedge\dots\wedge a_N=1$. Cette expression vaut alors 
    $\sum_{k=1}^{N}\frac{\alpha_{1,k}}{(-z+1)^{k}}+\sum_{\xi\in\U\setminus\left\lbrace1\right\rbrace}\left(\sum_{k=1}^{N-1}\frac{\alpha_{\xi,k}}{(-z+\xi)^{k}}\right)$ (somme finie). Pour $\left\lvert z\right\rvert<1$, on a 
    \begin{equation}
        \frac{1}{\left(-z+\xi\right)^{k}}=\left(-\frac{1}{\xi}\right)^{k}\sum_{n=0}^{+\infty}\frac{(n-k+1)\dots(n+1)}{(k-1)!}\left(\frac{z}{\xi}\right)^{n}.
    \end{equation}
    Ainsi, le coefficient en $z^{n}$ et équivalent à $\frac{n^{k-1}}{(k-1)!}\left(-\frac{1}{\xi}\right)^{k}$ en $+\infty$. Donc $c_n$ est un polynôme en $n$, équivalent en $+\infty$ à $\alpha_{1,N}\times \frac{n^{N-1}}{(n-1)!}$.

    Si $F=\frac{1}{(1-X^{a_1})\dots (1-X^{a_N})}$, en évaluant $(1-X)^{N}F$ et en prenant la limite en $X\to 1$, on a $\frac{X^{a_k}-1}{X-1}=1+X+\dots+X^{a_k-1}\xrightarrow[X\to1]{}a_k$. Finalement, $\alpha_{1,N}=\frac{1}{\prod_{k=1}^{n}a_k}$ et $c_n\geqslant1$ pour $n$ suffisamment grand. Ainsi,
    \begin{equation}
        \boxed{
            c_n\underset{+\infty}{\sim}\frac{n^{N-1}}{\left(\prod_{k=1}^{N}a_k\right)(N-1)!}.
        }
    \end{equation}
\end{proof}

\begin{proof}
    $f$ est $\mathcal{C}^{\infty}$ sur $\R$ par somme et composée. Pour $x\neq1$, on a 
    \begin{equation}
        f(x)=\sqrt{\frac{1-x^{3}}{1-x}}=\sqrt{1-x^{3}}\times\sqrt{\frac{1}{1-x}},
    \end{equation}
    produit de deux fonctions développable en série entière sur $]-1,1[$. Il existe donc $(a_n)_{n\in\N}\in\R^{\N}$ telle que pour tout $x\in]-1,1[$, $f(x)=\sum_{n=0}^{+\infty}a_nx^{n}$. On a $f^{2}(x)=1+x+x^{2}$ et $(f^{2})'(x)=2f'(x)f(x)=1+2x$ d'où pour tout $x\in]-1,1[$,
    \begin{equation}
        2\left(\sum_{n=0}^{+\infty}(n+1)a_{n+1}x^{n}\right)\left(\sum_{n=0}^{+\infty}a_nx^{n}\right)=1+2x,
    \end{equation}
    encore vrai pour $z\in D(0,1)$ par unicité du développement en série entière.

    Si $R>1$, me rayon de convergence de $\sum(n+1)a_{n+1}z^{n}$ est $R$. On aurait alors pour tout $z\in D(0,R)$
    \begin{equation}
        2\left(\sum_{n=0}^{+\infty}(n+1)a_{n+1}z^{n}\right)\left(\sum_{n=0}^{+\infty}a_nz^{n}\right)=1+2z,
    \end{equation}
    i.e.~si $S(z)=\sum_{n=0}^{+\infty}a_nz^{n}$, alors $2S'(z)S(z)=1+2z$. En $\j$, on a $2S'(\j)S(\j)=1+2\j$. Comme pour tout $x\in]-1,1[$, $S^{2}(x)=1+x+x^{2}$, par unicité, on a pour tout $z\in D(0,R)$, $S^{2}(z)=1+z+z^{2}$. Donc $S^{2}(\j)=1+\j+\j^{2}=0$ d'où $S(\j)=0$ : impossible car sinon $0=1+2\j$. Ainsi, $R=1$.
\end{proof}

\begin{proof}
    \phantom{}
    \begin{enumerate}
        \item $\sum_{k\in\N}\frac{f^{(k)(0)}}{k!}x^{k}$ est une série à termes positifs, d'après la formule de Taylor reste intégral, on a 
        \begin{equation}
            f(x)=\underbrace{\sum_{k=0}^{n}\frac{f^{(k)}(0)}{k!}x^{k}}_{S_n(x)}+\underbrace{\int_{0}^{x}\frac{(x-t)^{n}}{n!}f^{(n+1)}(t)\d t}_{R_n(x)\geqslant0}.
        \end{equation}
        On a $0\leqslant S_n(x)\leqslant f(x)$, donc la série converge et la suite $(R_n(x))_{n\in\N}$ converge aussi.

        \item On pose $t=xu$ et on a 
        \begin{equation}
            R_n(x)=x^{n+1}\int_{0}^{1}\frac{(1-u)^{n}}{n!}f^{(n+1)}(u)\d u.
        \end{equation}
        Pour tout $t\in[0,A[$, $f^{(n+2)}(t)\geqslant0$, $f^{(n+1)}$ est croissante. On a donc 
        \begin{equation}
            0\leqslant R_n(x)\leqslant\frac{x^{n+1}}{y^{n+1}}y^{n+1}\int_{0}^{1}\frac{(1-u)^{n}}{n!}f^{(n+1)}(xu)\d u,    
        \end{equation}
        d'où $0\leqslant R_n(x)\leqslant\left(\frac{x}{y}\right)^{n+1}R_n(y)$.

        \item $(R_n(y))_{n\in\N}$ est bornée d'après a), donc $R_n(x)\xrightarrow[x\to0]{}0$ d'où $f(x)=\sum_{k=0}^{+\infty}\frac{f^{(k)}(0)}{k!}x^{k}$.
        
        \item On a $\tan\geqslant0$ sur $\left[0,\frac{\pi}{2}\right[$ et $\tan'=1+\tan^{2}\geqslant0$. Soit $n\in\N$, on suppose que pour tout $k\in\left\llbracket0,n\right\rrbracket$, $\tan^{(k)}\geqslant0$ sur $\left[0,\frac{\pi}{2}\right[$. On dérive $n$ fois, d'après la formule de Leibniz, on a 
        \begin{equation}
            \tan^{(n+1)}=\sum_{k=0}^{n}\binom{n}{k}\tan^{(k)}\tan^{(n-k)}\geqslant0.
        \end{equation}
        Par imparité, on a pour tout $t\in\left[0,\frac{\pi}{2}\right[$
        \begin{equation}
            \tan(x)=\sum_{k=0}^{+\infty}\frac{\tan^{(k)}(0)}{k!}x^{k}=\sum_{q=0}^{+\infty}\frac{\tan^{(2p+1)}(0)}{(2p+1)!}x^{2p+1}.
        \end{equation}
        Par imparité, c'est aussi vrai sur $\left]-\frac{\pi}{2},\frac{\pi}{2}\right[$.
    \end{enumerate}
\end{proof}

\begin{remark}
    Si $\tan(x)=\sum_{k=0}^{+\infty}a_k x^{k}$, $\tan'=1+\tan^{2}$ fournit, pour tout $n\geqslant1$, $(n+1)a_{n+1}=\sum_{k=0}^{n}a_{k}a_{n-k}$.
\end{remark}

\begin{proof}
    \phantom{}
    \begin{enumerate}
        \item D'après le critère spécial des séries alternées, $a_n$ est du signe de $(-1)^{n}$, et $a_n\xrightarrow[n\to+\infty]{}0$. De plus, il existe $M>0$ tel que pour tout $n\geqslant1$, $\left\lvert a_n\right\rvert\leqslant M$. Donc par comparaison, $R\geqslant1$.
        
        D'autre part, on a $\left\lvert a_n\right\rvert+\left\lvert a_{n+1}\right\rvert=\frac{1}{n}$ et $\left\lvert a_n\right\rvert=\sum_{k=0}^{+\infty}\frac{1}{(n+2k)(n+1+2k)}$. On a $\left\lvert a_{n+1}\right\rvert\leqslant\left\lvert a_n\right\rvert$. On a alors $2\left\lvert a_{n+1}\right\rvert\leqslant\frac{1}{n}\leqslant2\left\lvert a_n\right\rvert$, et $\frac{1}{2n}\leqslant\left\lvert a_n\right\rvert\leqslant\frac{1}{2(n-1)}$. D'où $\left\lvert a_n\right\rvert\underset{+\infty}{\sim}\frac{1}{2n}$ et $R=1$. $a_n(-1)^{n}=\left\lvert a_n\right\rvert$ est le terme général d'une série divergente.

        Pour tout $\theta\in\R$ tel que $\e^{\i\theta}\neq1$, on a $a_n\left(-\e^{\i\theta}\right)^{n}=\left\lvert a_n\right\rvert\e^{\i n\theta}$. $n\mapsto\left\lvert a_n\right\rvert$ est décroissante tandis que $n\mapsto\e^{\i n\theta}$ est bornée. D'après la règle d'Abel, $\sum a_{n}\left(-\e^{\i\theta}\right)^{n}$ converge. On a convergence sur le cercle sauf en -1.

        \item On a toujours $\left\lvert a_n\right\rvert\leqslant3^{\frac{n-1}{3}}$. Si $b_n=3^{\frac{n-1}{3}}$, $\sum b_nz^{n}$ a un rayon de convergence égal à $\frac{1}{\sqrt[3]{3}}$. Donc $R\geqslant\frac{1}{\sqrt[3]{3}}$. De plus, $a_{3p+1}\left(\frac{1}{\sqrt[3]{3}}\right)^{3p+1}=3^{p}\times\frac{1}{3^{p}}\times\frac{1}{\sqrt[3]{3}}=3^{-\frac{1}{3}}$. Donc $\sum a_n\left(\frac{1}{\sqrt[3]{3}}\right)^{n}\not\to0$ quand $\nrightarrow+\infty$. Donc $R=3^{-\frac{1}{3}}$.
        
        Sur le cercle, si $z=3^{-\frac{1}{3}}\e^{\i\theta}$ avec $\theta\in\R$, on a $\left\lvert a_{3p+1}z^{3p+1}\right\rvert=3^{-\frac{1}{3}}$ donc $a_{n}z^{n}\nrightarrow0$ : il y a divergence.

        Pour le calcul effectif, si $z\in\C$ tel que $\left\lvert z\right\rvert<3^{-\frac{1}{3}}$, on a 
        \begin{equation}
            \sum_{n=0}^{+\infty}a_{n}z^{n}=\sum_{p=0}^{+\infty}\frac{(-1)^{p}}{2^{p}}z^{3p}+\sum_{p=0}^{+\infty}3^{p}z^{3p+1}=\frac{1}{1+\frac{z^{3}}{2}}+\frac{z}{1-3z^{3}}.
        \end{equation}

        \item Soit $n\geqslant0$, on a $\frac{1}{3}\int_{0}^{1}t^{n}\d t\leqslant a_n\leqslant\int_{0}^{1}t^{n}\d t$. Ainsi, d'après la règle de d'Alembert, on a $R=1$.
        \item 
        Comme $a_n\geqslant\frac{1}{3}\frac{1}{n+1}$, en $x=R$, $\sum a_nx^{n}=\sum a_n$ est divergente. 
        
        $\sum a_n(-R)^{n}$ est alternée, et comme $t^{n+1}\leqslant t^{n}$ pour tout $t\in[0,1]$, $n\mapsto\left\lvert a_n\right\rvert$ décroît vers 0. Donc $\sum a_n(-R)^{n}$ est convergente d'après le critère spécial des séries alternées. 

        Pour le calcul, soit $x\in]-1,1[$. Soit \function{f_n}{[0,1]}{\R}{t}{\frac{(tx)^{n}}{1+t+t^{2}}}
        $f_n$ est continue sur $[0,1]$ et $\left\lvert f_n(t)\right\rvert\leqslant\left\lvert x\right\rvert^{n}$ terme général d'une série à termes positifs convergente. Donc $\sum f_n$ converge normalement donc uniformément sur $[0,1]$, on peut intervertir :
        \begin{equation}
            \sum_{n=0}^{+\infty}\int_{0}^{1}\frac{t^{n}}{1+t+t^{2}}\d t x^{n}=\int_{0}^{1}\frac{1}{1+t+t^{2}}\times \frac{1}{1-tx}\d t.
        \end{equation}
        On pose $F(X)=\frac{1}{1+X+X^{2}}\times\frac{1}{1-Xx}=\frac{\alpha X+\beta}{1+X+X^{2}}+\frac{\gamma}{1-Xx}$. Si $x\neq0$, on a $\gamma=\frac{x^{2}}{1+x+x^{2}}$ et $\lim\limits_{X\to+\infty}XF(X)=0=\alpha-\frac{\gamma}{x}$ et $\alpha=\frac{x}{1+x+x^{2}}$. Enfin, $F(0)=1=\beta+\gamma$ donc $\beta=\frac{1}{1+x+x^{2}}$. Finalement, on a 
        \begin{equation}
            S(x)=\frac{1}{1+x+x^{2}}\times\left[\int_{0}^{1}\frac{xt+1+x}{1+t+t^{2}}\d t+x^{2}\int_{0}^{1}\frac{\d t}{1-tx}\right].
        \end{equation}
        Le calcul est laissé aux soins du lecteur.

        Pour la valeur en -1, on note que pour tout $x\in[0,1[$, $\sum_{n=0}^{+\infty}(-1)^{n}a_nx^{n}=S(-x)$. D'après le critère spécial des séries alternées, le $n$-ième reste est majoré par $a_n\to0$ donc on a convergence uniforme et $\lim\limits_{x\to1}S(-x)=\sum_{n=0}^{+\infty}(-1)^{n}a_n$ (continuité en -1).
    \end{enumerate}
\end{proof}

\begin{proof}
    \phantom{}
    \begin{enumerate}
        \item On partitionne les relations d'équivalence sur $\left\llbracket1,n+1\right\rrbracket$ selon le cardinal de la classe de $n+1$, $k$. On a alors $\omega_{n+1}=\sum_{k=0}^{n}\binom{n}{k}\omega_{n-k}=\sum_{k=0}^{n}\binom{n}{k}\omega_k$ (choisir les $k$ éléments en relation avec $n+1$). On a $\omega_{0}=1=0^{0}$. Soit $n\in\N$, supposons que pour tout $k\leqslant n$, on ait $\omega_{k}\leqslant k^{k}$. Alors 
        \begin{equation}
            \omega_{n+1}\leqslant\sum_{k=0}^{n}\binom{n}{k}k^{k}\leqslant\sum_{k=0}^{n}\binom{n}{k}n^{k}=(1+n)^{n}\leqslant(n+1)^{n+1}.
        \end{equation}
        Donc pour tout $n\in\N$, on a $\omega_{n}\leqslant n^{n}$.

        On a $\frac{\omega_{n}}{n!}\leqslant\frac{n^{n}}{n!}\underset{+\infty}{\sim}\frac{\e^{n}}{\sqrt{2\pi n}}$ donc $R\geqslant\frac{1}{\e}>0$.

        \item Pour tout $n\leqslant n_0$, $\frac{\omega_n r^{n}}{n!}\leqslant A$. Soit $n\geqslant n_0$, supposons que pour tout $k\leqslant n$, $\omega_k x^{k}\leqslant Ak!$. Alors on a 
        \begin{align}
            \omega_{n+1}r^{n+1}
            &=\sum_{k=0}^{n}\binom{n}{k}\omega_{k}r^{k}r^{n+1-k},\\
            &\leqslant n!Ar\sum_{k=0}^{n}\frac{r^{n-k}}{(n-k)!},\\
            &\leqslant n! A r\e^{r}\leqslant (n+1)! A.
        \end{align}
        Donc pour tout $n\in\N$, $\frac{\omega_n}{n!}\leqslant \frac{A}{r^{n}}$. On a donc $R\geqslant r$ pour tout $r\geqslant 1$ donc $R=+\infty$.

        \item Soit $x\in\R$, on a 
        \begin{align}
            f'(x)
            &=\sum_{n=0}^{+\infty}(n+1)\frac{\omega_{n+1}}{(n+1)!}z^{n},\\
            &=\sum_{n=0}^{+\infty}\frac{\omega_{n+1}}{n!}z^{n},\\
            &=\sum_{n=0}^{+\infty}\sum_{k=0}^{n}\frac{\omega_{k}}{k!}\frac{1}{(n-k)!}x^{n},\\
            &=\e^{x}f(x).
        \end{align}

        Donc il existe $K\in\R$ tel que pour tout $x\in\R$, $f(x)=K\e^{\e^{x}}$, et $K=\frac{f(0)}{\e}=\frac{1}{\e}$. On a donc 
        \begin{equation}
            f(x)=\frac{1}{\e}\e^{\e^{x}}=\frac{1}{\e}\sum_{k=0}^{+\infty}\frac{\e^{kx}}{k!}=\frac{1}{\e}\sum_{k=0}^{+\infty}\sum_{n=0}^{+\infty}\underbrace{\frac{1}{k!}\frac{(kx)^{n}}{n!}}_{a_{k,n}}.
        \end{equation}
        On a $\sum_{k=0}^{+\infty}\sum_{n=0}^{+\infty}\left\lvert a_{k,n}\right\rvert=\e^{\e^{\left\lvert x\right\rvert}}<+\infty$. D'après le théorème de Fubini, on a donc 
        \begin{equation}
            f(x)=\frac{1}{\e}\sum_{n=0}^{+\infty}\sum_{k=0}^{+\infty}\frac{(kx)^{n}}{k!n!},
        \end{equation}
        et donc pour tout $n\in\N$, on a 
        \begin{equation}
            \boxed{
                \omega_{n}=\frac{1}{\e}\sum_{k=0}^{+\infty}\frac{k^{n}}{k!}.
            }
        \end{equation}
    \end{enumerate}
\end{proof}

\begin{proof}
    \phantom{}
    \begin{enumerate}
        \item Pour tout $n\geqslant1$, on a $p_n\leqslant\sum_{j=1}^{n}p_{n-j}$ (car $p_{n-j}$ est le nombre maximal de possibilité si le premier terme vaut $j$, $t_{1}=j$). On a $p_0=2^{0}$ et par récurrence forte, soit $n\in\N^{*}$, supposons que pour tout $k\in\left\llbracket0,n-1\right\rrbracket$, $p_{k}\leqslant2^{k}$, alors 
        \begin{equation}
            p_n\leqslant\sum_{j=0}^{n-1}p_j\leqslant\sum_{j=0}^{n-1}2^{j}=2^{n}-1.
        \end{equation}
        Donc pour tout $n\in\N$, $p_n\leqslant 2^{n}$. D'après la règle de d'Alembert, $R\geqslant\frac{1}{2}$. De plus, pour tout $n\in\N$, $p_n\geqslant1$ donc $R\leqslant 1$.

        \item Soit $x\in[0,R[$, on a $x<1$. Alors $0\leqslant-\ln\left(1-x^{k}\right)\underset{k\to+\infty}{\sim}x^{k}$, terme générale d'une série à termes positifs convergente car $x<1$. Donc $\prod_{k\geqslant1}\frac{1}{1-x^{k}}$ converge.
        
        Soit $N\geqslant1$, on a 
        \begin{align}
            \prod_{k=1}^{N}\frac{1}{1-x^{k}}
            &=\prod_{k=1}^{N}\left(\sum_{n_k=0}^{+\infty}(x^{k})^{n_k}\right),\\
            &=\sum_{(n_{1},\dots,n_{N})\in\N^{N}}x^{n_{1}+2n_{2}+\dots+Nn_{N}},\\
            &=\sum_{n=0}^{+\infty}\alpha_{n,N}x^{n},
        \end{align}
        où $\alpha_{n,N}=\left\lvert\left\lbrace(n_1,\dots,n_N)\in\N^{N}\middle| n_1+2n_2+\dots+Nn_{N}=n\right\rbrace\right\rvert$. Par sommabilité, on a $\alpha_{n,N}=\left\lvert\left\lbrace\text{partitions }(t_k)_{k\geqslant1}\text{ de }n\middle| t_1\leqslant N\right\rbrace\right\rvert\leqslant p_n$, et si $n\leqslant N, \alpha_{n,N}=p_n$. 
        
        On a $f(x)=\sum_{n=0}^{N}p_nx^{n}+\sum_{n=N+1}^{+\infty}\alpha_{n,N}x^{n}$ d'où $\prod_{k=1}^{N}\frac{1}{1-x^{k}}\leqslant f(x)$. Ainsi,
        \begin{equation}
            0\leqslant f(x)-\prod_{k=1}^{N}\frac{1}{1-x^{k}}\leqslant\sum_{n=N+1}^{+\infty}\left(p_n-\alpha_{n,N}\right)x^{n}\leqslant\sum_{n=N+1}^{+\infty}p_n x^{n}\xrightarrow[N\to+\infty]{}0,
        \end{equation}
        reste d'une série convergente. Donc 
        \begin{equation}
            f(x)=\prod_{k=1}^{+\infty}\frac{1}{1-x^{k}}.
        \end{equation}

        Soit $z\in D(0,R)$, 
        \begin{equation}
            \left\lvert f(z)-\prod_{k=1}^{N}\frac{1}{1-z^{k}}\right\rvert\leqslant\sum_{n=N+1}^{+\infty}\left(p_n-\alpha_{n,N}\right)\left\lvert z\right\rvert^{n}\leqslant\sum_{n=N+1}^{+\infty}p_n\left\lvert z\right\rvert^{n}\xrightarrow[N\to+\infty]{}0.
        \end{equation}

        Cela reste vrai sur $D(0,R)$.

        \item Si $x\in[0,1[$, on peut développer et on obtient
        \begin{equation}
            \prod_{k=1}^{+\infty}\frac{1}{1-x^{k}}=\sum_{n=0}^{+\infty}a_nx^{n},
        \end{equation}
        et par unicité, $a_n=p_n$ pour tout $n\in\N$ donc $R=1$.
    \end{enumerate}
\end{proof}

\begin{proof}
    \phantom{}
    \begin{enumerate}
        \item On a $f(z_{0}+r\e^{\i t})=\sum_{n=0}^{+\infty}a_n r^{n}\e^{\i nt}=\sum_{n=0}^{+\infty}f_n(t)$. On a pour tout $n\in\N$, $\left\lvert f_n(t)\right\rvert=\left\lvert a_n\right\rvert r^{n}$, et comme $r<d(z_{0},\partial U)$ donc $\sum \left\lvert a_n\right\rvert r^{n}$ converge. On a convergence normale des $(f_n)_{n\in\N}$ sur $[0,2\pi]$.
        Ainsi,
        \begin{equation}
            \int_{0}^{2\pi}f\left(z_{0}+r\e^{\i t}\right)\d t=\sum_{n=0}^{+\infty}a_n r^{n}\int_{0}^{2\pi}\e^{\i nt}\d t=2\pi a_0=2\pi f(z_{0}).
        \end{equation}
        Donc 
        \begin{equation}
            \boxed{
                f(z_{0})=\frac{1}{2\pi}\int_{0}^{2\pi}f\left(z_0+r\e^{\i t}\right)\d t.
            }
        \end{equation}

        \item $\overline{U}$ est un compact donc $\left\lvert f\right\rvert$ atteint son maximum sur $\overline{U}$. De plus, pour tout $r\in[0,d(z_{0},\partial U)]$ et pour tout $t\in[0,2\pi]$, on a $\left\lvert f\left(z_{0}+r\e^{\i t}\right)\right\rvert\leqslant\left\lVert f\right\rVert_{\infty}$, intégrable sur $[0,2\pi]$ et $f$ continue. Donc d'après le théorème de continuité, 
        \begin{equation*}
            f(z_{0})=\frac{1}{2\pi}\int_{0}^{2\pi}f\left(z_0+R\e^{\i t}\d t\right),    
        \end{equation*}
        où $R=d(z_{0},\partial U)$.
        
        Si $\left\lvert f\right\rvert$ atteint son maximum en $z_{0}\in U$, on a 
        \begin{equation}
            \left\lvert f(z_{0})\right\rvert=\left\lVert f\right\rVert_{\infty}=\left\lvert \frac{1}{2\pi}\int_{0}^{2\pi}f\left(z_{0}+R\e^{\i t}\right)\d t\right\rvert\leqslant\frac{1}{2\pi}\int_{0}^{2\pi}\left\lvert f\left(z_{0}+R\e^{\i t}\right)\right\rvert\d t\leqslant\left\lVert f\right\rVert_{\infty}.
        \end{equation}

        On a donc égalité partout : $\frac{1}{2\pi}\int_{0}^{2\pi}\left(\left\lVert f\right\rVert_{\infty}-\left\lvert f\left(z_0+R\e^{\i t}\right)\right\rvert\right)\d t=0$. Comme l'intégrande est une fonction continue positive, donc pour tout $t\in[0,2\pi]$, $\left\lVert f\right\rVert_{\infty}=f\left(z_{0}+R\e^{\i t}\right)$. On a $d\left(C(0,R),\partial U\right)=0$ et comme $C(0,R)$ est un compact la distance est atteinte : il existe $t_{0}\in[0,2\pi]$ tel que $z_{0}+R\e^{\i t_0}\in\partial U$, donc $\left\lvert f\right\rvert$ atteint son maximum et son minimum sur $\partial U$.

        \item Si $f=0$ sur $\partial U$, alors $f=0$ sur $\overline{U}$.
    \end{enumerate}
\end{proof}

\begin{remark}
    S'il existe $z_{0}\in U$ tel que $\left\lvert f(z_{0})\right\rvert=\left\lVert f\right\rVert_{\infty}$, on a pour tout $r\leqslant R$,
    \begin{equation}
        \left\lvert f(z_{0})\right\rvert=\left\lVert f\right\rVert_{\infty}=\left\lvert \frac{1}{2\pi}\int_{0}^{2\pi}f\left(z_{0}+r\e^{\i t}\right)\d t\right\rvert\leqslant\frac{1}{2\pi}\int_{0}^{2\pi}\left\lvert f\left(z_{0}+r\e^{\i t}\right)\right\rvert\d t\leqslant\left\lVert f\right\rVert_{\infty}.
    \end{equation}

    On en déduit que pour tout $t\in[0,2\pi]$, $\left\lvert f\left(z_0+r\e^{\i t}\right)\right\rvert=\left\lvert f(z_{0})\right\rvert$, et on a aussi 
    \begin{equation*}
        \arg\left(f\left(z_{0}+r\e^{\i t}\right)\right)\equiv\arg\left(f(z_0)\right)[2\pi].    
    \end{equation*}
    Donc $f\left(z_{0}+r\e^{\i t}\right)=f(z_{0})$, et on peut vérifier que $f$ est constante.
\end{remark}

\begin{proof}
    \phantom{}
    \begin{enumerate}
        \item Passer au $\ln$ de la valeur absolue, équivalent, convergence géométrique.
        \item On cherche une équation fonctionnelle satisfaite par $f$.
        On a 
        \begin{equation*}
            f(qz)=\prod_{k=1}^{+\infty}\left(1-q^{k+1}z\right)=\prod_{k=2}^{+\infty}\left(1-q^{k}z\right),
        \end{equation*}donc $(1-qz)f(qz)=f(z)$.
        Si $f$ est développable en série entière avec $f(z)=\sum_{n=0}^{+\infty}a_nz^{n}$ pour tout $z\in D(0,R)$ avec $R>0$, on a par unicité du développement, $a_n(q^{n}-1)=a_{n-1}q^{n}$ et comme $\left\lvert q\right\rvert<1$, pour tout $n\in\N^{*}$, $q^{n}\neq1$ d'où $a_{n}=a_{n-1}\frac{q^{n}}{q^{n}-1}=\prod_{i=1}^{n}\frac{q^{i}}{q^{i}-1}$.

        Réciproquement si pour tout $n\in\N$, $a_n=\prod_{i=1}^{n}\frac{q^{i}}{q^{i}-1}$, alors pour tout $n\in\N$, $a_n\neq0$ et par la règle de d'Alembert, $R=+\infty$. Si $S\colon\C\to\C$ est définie apr $\S(z)=\sum_{n=0}^{+\infty}a_n z^{n}$, en reportant les calculs, $S$ vérifie la même équation fonctionnelle que $f$. En itérant, on a $S(z)=\prod_{i=1}^{n}(1-q^{i}z)S(q^{n}z)$. $S$ étant continue en 0 (car développable en série entière sur $\C$), on a $S(q^{n}z)\xrightarrow[n\to+\infty]{}S(0)=a_0=1$. En passant à la limite, on a donc $S(z)=f(z)$ et $f$ est développable en série entière.

        \item Si $f(z)\neq0$ et $f(qz)\neq0$, on pose $g(z)=\frac{1}{f(z)}$. On a alors $g(qz)=(1-qz)g(z)$, et on procède de même façon qu'à la question précédente. On trouve alors $R=\frac{1}{\left\lvert q\right\rvert}$.
    \end{enumerate}
\end{proof}

\begin{proof}
    \phantom{}
    \begin{enumerate}
        \item Par continuité, $a_0=f(z_{0})=0$. Supposons qu'il existe $n\geqslant1$ tel que $a_n\neq0$. Soit $n_{0}=\min\left\lbrace n\in\N\middle| a_n\neq0\right\rbrace$. Il vient, si $\left\lvert h\right\rvert<r_0$, 
        \begin{equation}
            f(z_{0}+h)=\sum_{k\geqslant n_{0}}^{+\infty}a_k h^{k}=a_{n_{0}}h^{n_{0}}\left(1+\underbrace{\sum_{k=1}^{+\infty}\frac{a_{k+n_0}}{a_{n_{0}}}h^{k}}_{g(h)}\right).
        \end{equation}

        $g$ est continue (série entière de rayon de convergence plus grand que $r_0>0$) et $g(0)=0$ donc il existe $\alpha_{0}>0$ tel que si $\left\lvert h\right\rvert\leqslant\alpha_{0}$, alors $\left\lvert g(h)\right\rvert\leqslant\frac{1}{2}$. Alors $1+g(h)\neq0$ et si $h\neq0$, $f(z_{0}+h)=a_{n_{0}}h^{n_{0}}\left(1+g(h)\right)\neq0$. Il existe $N\in\N$ tel que pour tout $k\geqslant N$, $\left\lvert \xi_{k}-z_{0}\right\rvert\leqslant\alpha_{0}$, d'où pour tout $k\geqslant N$, $f(\xi_{k})\neq0$, ce qui est absurde. Les $(a_n)_{n\in\N}$ sont donc tous non nuls.

        \item Soit $z_1\in U$. Il existe $\gamma\colon[0,1]\to U$ continue telle que $\gamma(0)=z_{0}$ et $\gamma(1)=z_{1}$. Soit $t_{0}=\sup\left\lbrace t\in[0,1]\middle|\forall x\in[0,t],f(\gamma(x))=0\right\rbrace$. Supposons $t_{0}\neq1$. Par définition de la borne supérieure, pour tout $x\in[0,t_0[$, $f(\gamma(x))=0$. On peut appliquer ce qui précède à $\gamma(t_0)$ à la place de $z_{0}$ : il existe $\alpha_{0}$ tel que pour tout $z\in D(\gamma(t_{0}),\alpha_{0})$ tel que $f(z)=0$. Par continuité de $\gamma$, il existe $\beta>0$ tel que si $\left\lvert t-t_{0}\right\rvert<\beta$, alors $\left\lvert\gamma(t)-\gamma(t_{0})\right\rvert\leqslant\alpha_{1}$/ Pour $t=t_{0}+\frac{\beta}{2}$, on a $f(\gamma(t))=0$ pour tout $x\in[0,t]$. C'est absurde. Donc $t_{0}=1$ et $f(z_{1})=0$.
    \end{enumerate}
\end{proof}

\begin{remark}
    Deux fonctions analytiques définies sur un ouvert connexe par arcs et qui coïncident sur une suite injective convergente sont égales.
\end{remark}

\begin{proof}
    \phantom{}
    \begin{enumerate}
        \item On a $f(x)=\ln\left(\left(x-\cos(\theta)\right)^{2}+\sin^{2}(\theta)\right)$. L'argument est positif et égal à 0 si et seulement si $x=\cos(\theta)$ et $\sin(\theta)=0$, ce qui est absurde car $\theta\in]0,\pi[$. $f$ est définie sur $\R$ et est $\mathcal{C}^{\infty}$.
        On a
        \begin{equation}
            f'(x)=\frac{2(x-\cos(\theta))}{1-2x\cos(\theta)+x^{2}}=\frac{2(x-\cos(\theta))}{\left(x-\e^{\i\theta}\right)\left(x+\e^{\i\theta}\right)}=\frac{a}{x-\e^{\i\theta}}+\frac{b}{x-\e^{-\i\theta}}.
        \end{equation},
        où $a=\frac{2\left(\e^{\i\theta}-\cos(\theta)\right)}{\e^{\i\theta}-\e^{-\i\theta}}=\frac{2\i\sin(\theta)}{2\i\sin(\theta)}=1$ et $b=\overline{a}=1$.

        On sait alors que $f'(x)=\frac{1}{x-\e^{\i\theta}}+\frac{1}{x-\e^{-\i\theta}}$ est développable en série entière avec un rayon de convergence égal à 1 (fonction rationnelle dont 0 n'est pas un pôle). Soit $x\in]-1,1[$, on a 
        \begin{align}
            f'(x)
            &=-\e^{-\i\theta}\times\frac{1}{1-x\e^{-\i\theta}}-\e^{\i\theta}\times\frac{1}{1-x\e^{\i\theta}},\\
            &=-\e^{-\i\theta}\sum_{k=0}^{+\infty}\left(x\e^{-\i\theta}\right)^{k}-\e^{\i\theta}\sum_{k=0}^{+\infty}\left(x\e^{\i\theta}\right)^{k},\\
            &=\sum_{k=0}^{+\infty}\left(-\left(\e^{-\i\theta}\right)^{k+1}-\left(\e^{\i\theta}\right)^{k+1}\right)x^{k},\\
            &=\sum_{k=0}^{+\infty}\left(-2\cos\left((k+1)\theta\right)\right)x^{k}.
        \end{align}
        On a $f(0)=0$. Ainsi, pour tout $x\in]-1,1[$, $f(x)=\sum_{k=1}^{+\infty}\frac{x^{k}}{k}\times\left(-2\cos(k\theta)\right)$.

        \item D'après la règle d'Abel, on a convergence pour $x=1$ donc $\sum_{n\geqslant1}\frac{\cos(n\theta)}{n}$ converge, et on a convergence uniforme sur $[0,1]$. $f$ est alors continue en 1 et on a 
        \begin{equation}
            \sum_{k=1}^{+\infty}\frac{\cos(k\theta)}{k}=-\frac{1}{2}f(1)=-\frac{1}{2}\ln\left(2-2\cos(\theta)\right).
        \end{equation}

        \item Pour $x\in]-1,1[$ pour tout $\theta\in[0,\pi]$, $x^{2}-2x\cos(\theta)+1=\left(x-\cos(\theta)\right)^{2}+\sin^{2}(\theta)$ qui est égal à 0 si et seulement si $x=\cos(\theta)$ et $\theta\in\left\lbrace0,\pi\right\rbrace$ : impossible car $x\in]-1,1[$. On a 
        \begin{equation}
            I(x)=\int_{0}^{\pi}\left(-2\sum_{k=1}^{+\infty}\frac{x^{k}}{k}\cos(k\theta)\right)\d\theta=\int_{0}^{\pi}f_k(\theta)\d\theta.
        \end{equation}

        On pose $u_k=\int_{0}^{\pi}\left\lvert f_k(\theta)\right\rvert\d\theta\leqslant\frac{x^{k}}{k}\times\pi$, terme général d'une série convergente car $\left\lvert x\right\rvert<1$, donc $\sum u_{k}$ converge. On peut donc intervertir :
        \begin{equation}
            I(x)=-2\sum_{k=1}^{+\infty}\frac{x^{k}}{k}\int_{0}^{\pi}\cos(k\theta)\d\theta=0.
        \end{equation}
        Pour $\left\lvert x\right\rvert>1$, on a 
        \begin{equation}
            I(x)=\int_{0}^{\pi}\left(\ln(x^{2})+\ln\left(1-\frac{2\cos(\theta)}{x}+\frac{1}{x^{2}}\right)\right)\d\theta=2\pi\ln\left(\left\lvert x\right\rvert\right).
        \end{equation}
    \end{enumerate}
\end{proof}

\begin{remark}
    Pour $x=1$, on a 
    \begin{align}
        \ln(2-2\cos(\theta))
        &=\ln(2)+\ln(1-\cos(\theta)),\\
        &\underset{\theta\to0}{=}\ln(2)+\ln\left(\frac{\theta^{2}}{2}+o(\theta^{2})\right),\\
        &\underset{\theta\to0}{=}2\ln(\theta)+O(1)+\ln(2),\\
        &\underset{\theta\to0}{\sim}2\ln(\theta),\\
        &\underset{\theta\to0}{O}\left(\frac{1}{\sqrt{\theta}}\right).
    \end{align}

    $I(1)$ est donc bien définie et $I(1)=\pi\ln(2)+\int_{0}^{\pi}\ln\left(2\sin^{2}\left(\frac{\theta}{2}\right)\right)$. On se ramène ainsi au calcul de $\int_{0}^{\frac{\pi}{2}}\ln(\sin(\theta))\d\theta$.
\end{remark}

\begin{proof}
    \phantom{}
    \begin{enumerate}
        \item On pose $a_k=0$ si $k\not\in\left\lbrace p_n,n\in\N\right\rbrace$ et $a_k=1$ sinon. On a toujours $\left\lvert a_k\right\rvert\leqslant 1$, donc $R\geqslant1$. De plus, $\sum_{n\geqslant}1^{p_n}=+\infty$, donc $R=1$.
        \item Soit $\varepsilon>0$, il existe $N_{0}\in\N$ tel que pour tout $n\geqslant N_0$, $\frac{n}{p_n}\leqslant\frac{\varepsilon}{2}$ et donc $p_n\geqslant\frac{2n}{\varepsilon}$. Soit $x\in[0,1[$. Pour tout $n\geqslant N_0$, on a $x^{p_n}\leqslant x^{\frac{2n}{\varepsilon}}$. Ainsi,
        \begin{equation}
            f(x)=\sum_{n=0}^{N_0-1}x^{p_n}+\sum_{n=N_0}^{+\infty}x^{p_n}\leqslant\sum_{n=0}^{N_0-1}x^{p_n}+\sum_{n=N_0}^{+\infty}x^{\frac{2n}{\varepsilon}}=\sum_{n=0}^{N_0-1}x^{p_n}+\frac{1}{1-x^{\frac{2}{\varepsilon}}}.
        \end{equation}
        On a $\lim\limits_{x\to1^{-}}(1-x)\sum_{n=0}^{N_0-1}x^{p_n}=0$, donc il existe $\alpha_{1}>0$ tel que pour tout $x\in[1-\alpha_{1}[$, $(1-x)\sum_{n=0}^{N_0-1}x^{p_n}\leqslant\frac{\varepsilon}{3}$, et 
        \begin{equation}
            \frac{1-x}{1-x^{\frac{2}{\varepsilon}}}=\frac{u}{1-\left(1-\frac{2u}{\varepsilon}+\underset{u\to0}{o}(u)\right)}\xrightarrow[u\to0]{}\frac{\varepsilon}{2},
        \end{equation}
        en posant $u=1-x$. Ainsi, il existe $\alpha_{2}>0$ tel que pour tout $x\in[1-\alpha_{2},1[$, $\frac{1-x}{1+x^{\frac{2}{\varepsilon}}}\leqslant\frac{2\varepsilon}{3}$. Ainsi, en posant $\alpha=\min\left(\alpha_{1},\alpha_{2}\right)$, si $x\in[1-\alpha,1[$, alors $f(x)(1-x)\leqslant\varepsilon$.

        \item On suppose $\lim\limits_{x\to1^{-}}(1-x)f(x)=0$. On pose, pour tout $k\geqslant1,x_{k}=1-\frac{1}{k}$. Alors on a 
        \begin{align}
            (1-x_k)f(x_k)
            &=\frac{1}{k}\left(\sum_{n=0}^{N_0-1}\left(1-\frac{1}{k}\right)^{p_n}\right)+\sum_{n=k}^{+\infty}\left(1-\frac{1}{k}\right)^{p_n},\\
            &\geqslant \sum_{n=k}^{+\infty}\frac{1}{k}\left(1-\frac{1}{k}\right)^{p_n},\\
            &\geqslant\sum_{n=k}^{+\infty}\frac{1}{n}\left(1-\frac{1}{k}\right)^{p_n}.
        \end{align}
        Donc $\lim\limits_{k\to+\infty}\sum_{n=k}^{+\infty}\frac{1}{n}\left(1-\frac{1}{k}\right)^{p_n}=0$.

        %% A voir
    \end{enumerate}
\end{proof}

\begin{proof}
    On suppose que $R_{1}$ et $R_{2}$, les rayons de convergence de $U(z)$ et $V(z)$ sont strictement positifs. Soit $R=\min(R_1,R_2)>0$. Pour tout $n\in\N$, pour tout $z\in D(0,R)$, on a $u_{n+1}z^{n+1}=\left(u_nz^{n}-v_nz^{n}\right)z$ et $v_{n+1}z^{n+1}=\left(u_nz^{n}-2v_nz^{n}\right)z$. On somme sur $\N$ et on obtient
    \begin{equation}
        \begin{array}[]{rcl}
            U(z)-u_0 &=& z\left(U(z)-V(z)\right),\\
            V(z)-v_0 &=& z\left(U(z)-2V(z)\right).
        \end{array}
    \end{equation}

    Ainsi,
    \begin{equation}
        \begin{array}[]{rcl}
            U(z)(z-1)-zV(z) &=& -U_0,\\
            zU(z)-V(z)(2z+1) &=& -V_0.
        \end{array}
    \end{equation}

    Le déterminant du système est 
    \begin{equation}
        \begin{vmatrix}
            z-1 & -z\\
            z -(2z+1)
        \end{vmatrix}=-z^{2}+z+1.
    \end{equation}
    Le discriminant est $\Delta=5$. Soit $z_{1}=\frac{1-\sqrt{5}}{2}$ et $z_2=\frac{1+\sqrt{5}}{2}$. On a $\left\lvert z_1\right\rvert=\frac{\sqrt{5}-1}{2}<\left\lvert z_2\right\rvert$.

    Si $z\not\in\left\lbrace z_1,z_2\right\rbrace$, on peut utiliser les formules de Cramer. Ainsi,
    \begin{equation}
        \begin{array}[]{l}
            U(z) = \frac{
                \begin{vmatrix}
                    -u_0 & -z\\
                    -v_0 & -(2z+1)
                \end{vmatrix}
            }{-z^{2}+z+1}=\frac{(2z+1)u_0-v_0z}{-z^{2}+z+1},\\
            V(z)= \frac{
                \begin{vmatrix}
                    z-1 & -u_0\\
                    z & -v_0
                \end{vmatrix}
            }{-z^{2}+z+1}=\frac{-v_0(z-1)+u_0z}{-z^{2}+z+1}.
        \end{array}
    \end{equation}

    Réciproquement, en définissant ainsi $U$ et $V$, ce sont des fractions rationnelles de $z_1$ et $z_2$ donc développable en séries entières avec un rayon de convergence à $\left\lvert z_1\right\rvert=\frac{\sqrt{5}-1}{2}$. En remontant les calculs, les coefficients vérifient les relations de récurrence.
\end{proof}

\begin{proof}
    \phantom{}
    \begin{enumerate}
        \item Si $\theta\in\left\lbrace0,\pi\right\rbrace$, $f(z)=0$. Sinon, on a $f(z)=\frac{\sin(\theta)}{\left(z-\e^{\i\theta}\right)\left(z-\e^{-\i\theta}\right)}\in\R(z)$. On prend $\left\lvert z\right\rvert<1$. Il existe $(A,B)\in\C^{2}$ tel que $f(z)=\frac{A}{z-\e^{\i\theta}}+\frac{B}{z-\e^{-\i\theta}}$. On trouve $A=-\frac{\i}{2}$ et $B=\overline{A}=\frac{\i}{2}$. En remplaçant, on a 
        \begin{align}
            f(z)
            &= \frac{\i}{2}\left(\frac{1}{z-\e^{-\i\theta}}-\frac{1}{z-\e^{\i\theta}}\right),\\
            &= \frac{\i}{2}\left(-\frac{\e^{\i\theta}}{1-z\e^{\i\theta}}+\frac{\e^{-\i\theta}}{1-z\e^{-\i\theta}}\right),\\
            &=\frac{\i}{2}\left(\sum_{n=0}^{+\infty}\left(-\e^{\i\theta}\left(z\e^{\i\theta}\right)^{n}+\e^{-\i\theta}\left(z\e^{-\i\theta}\right)^{n}\right)\right),\\
            &=\sum_{n=0}^{+\infty}\sin\left((n+1)\theta\right)z^{n}.
        \end{align}

        \item Pour $\left\lvert z\right\rvert<1$, défini car $z\not\in\left\lbrace\e^{\i\theta},\e^{-\i\theta}\right\rbrace$, on a 
        \begin{equation}
            I(z)=\int_{0}^{+\infty}\sum_{n=0}^{+\infty}\underbrace{\sin\left((n+1)\theta\right)z^{n}}_{f_n(\theta)}\d\theta.
        \end{equation}
        Comme $\left\lvert f_n(\theta)\right\rvert\leqslant\left\lvert z\right\rvert^{n}$, terme général d'une série à termes positifs convergentes car $\left\lvert z\right\rvert<1$, $\sum f_n$ convergent normalement sur $[0,\pi]$. On peut donc intervertir, et on a 
        \begin{equation}
            I(z)=\sum_{n=0}^{+\infty}\int_{0}^{\pi}\sin\left((n+1)\theta\right)\d\theta=2\sum_{k=0}^{+\infty}\frac{z^{2k}}{2k+1}.
        \end{equation}
        Pour $x\in]-1,1[$, soit $g(x)=xI(x)=2\sum_{k=0}^{+\infty}\frac{x^{2k+1}}{2k+1}$. On a $g'(x)=\frac{2}{1-x^{2}}=\frac{1}{1-x}+\frac{1}{1+x}$. On a $g(0)=0$, donc $g(x)=\ln\left(\frac{1+x}{1-x}\right)$ et $I(x)=\frac{1}{x}\ln\left(\frac{1+x}{1-x}\right)$.
    \end{enumerate}
\end{proof}

\begin{remark}
    Si $\left\lvert x\right\rvert>1$, on a $I(x)=\frac{1}{x^{2}}I\left(\frac{1}{x}\right)=\frac{1}{x}\ln\left(\frac{x+1}{x-1}\right)$.
\end{remark}

\begin{proof}
    \phantom{}
    \begin{enumerate}
        \item On a $\mathbb{E}(Y^{k})=\sum_{p=1}^{n}p^{k}\mathbb{P}\left(Y=p\right)$ pour tout $k\in\left\llbracket1,n\right\rrbracket$. Ainsi,
        \begin{equation}
            \begin{pmatrix}
                \mathbb{E}(Y)\\
                \vdots\\
                \mathbb{E}(Y^{n})\\
            \end{pmatrix}=
            \underbrace{
                \begin{pmatrix}
                    1&2&\dots&n\\
                    \vdots&2^{2}&\dots&n^{2}\\
                    \vdots & \vdots &\vdots & \vdots\\
                    1&2^{n}&\dots&n^{n}
                \end{pmatrix}
            }_{A}
            \begin{pmatrix}
                \mathbb{P}(Y=1)\\
                \dots\\
                \mathbb{P}(Y=n)
            \end{pmatrix}.
        \end{equation}
        On a $\det(A)=n! \text{VdM}(1,\dots,n)\neq0$. $A$ est inversible puis
        \begin{equation}
            \begin{pmatrix}
                \mathbb{P}(Y=1)\\
                \dots\\
                \mathbb{P}(Y=n)
            \end{pmatrix}=
            A^{-1}
            \begin{pmatrix}
                \mathbb{E}(Y)\\
                \vdots\\
                \mathbb{E}(Y^{n})\\
            \end{pmatrix}.
        \end{equation}

        Donc $\left(\mathbb{E}(Y^{k})\right)_{k\in\left\llbracket1,n\right\rrbracket}$ caractérise la loi de $Y$.

        \item Soit $n\geqslant1$. On a $k^{n}\mathbb{P}(Y=k)=\underset{K\to+\infty}{O}\left(\frac{1}{k^{2}}\right)$ (car $a<1$). Donc $Y$ possède un moment à tout ordre. Formons la série génératrice de $Y$: $G_Y(t)=\sum_{k=0}^{+\infty}\mathbb{P}\left(Y=k\right)t^{k}$ de rayon $R$ supérieur à $\frac{1}{a}$. $G_Y$ est $\mathcal{C}^{\infty}$ sur $\left[0,\frac{1}{a}\right[$. Ainsi,
        \begin{equation}
            \begin{array}[]{rcl}
                G_Y'(1) &=&\sum_{k=0}^{+\infty}y\mathbb{P}(Y=k)=\mathbb{E}(Y),\\
                G_Y''(1) &=& \sum_{k=0}^{+\infty}k(k-1)\mathbb{P}(Y=k)=\mathbb{E}(Y^{2})-\mathbb{E}(Y),\\
                \vdots\\
                G_Y^{(n)}(1) &=& \sum_{k=0}^{+\infty}k(k-1)\dots(k-n+1)\mathbb{P}(Y=k)\\
                &&=\mathbb{E}(Y^{n})+\sum_{k=0}^{+\infty}A(k)\mathbb{P}(Y=k),
            \end{array}
        \end{equation}
        avec $A\in\Z[X]$ de degré inférieur ou égal à $n-1$. Donc $\left(\mathbb{E}(Y^{n})\right)_{n\in\N^{*}}$ déterminent $\left(G_Y(1)\right)_{n\in\N^{*}}$.

        \begin{lemma}
            \label{lem:2}
            Soit $f(z)=\sum_{n=0}^{+\infty}a_nz^{n}$ de rayon de convergence supérieur ou égal à $R$. Soit $z_0\in D(0,R)$, alors il existe $(b_n)_{n\in\N}\in\C^{\N}$ tel que pour tout $h\in D(0,R-\left\lvert z_0\right\rvert)$, $f(z_{0}+h)=\sum_{n=0}^{+\infty}b_n h^{n}$.
        \end{lemma}
        \begin{proof}[Preuve du lemme~\ref{lem:2}]
            Soit $h\in D(0,R-\left\lvert z_0\right\rvert)$, on a $\left\lvert z_{0}+h\right\rvert\leqslant\left\lvert z_0\right\rvert+\left\lvert h\right\rvert<R$. On a donc 
            \begin{equation}
                f(z_{0}+h)=\sum_{n=0}^{+\infty}a_n (z_{0}+h)^{n}=\sum_{n=0}^{+\infty}a_n\sum_{k=0}^{+\infty}\binom{n}{k}z_{0}^{n-k}h^{k}=\sum_{n=0}^{+\infty}\sum_{k=0}^{+\infty}\alpha_{n,k}h^{k},
            \end{equation}
            avec $\alpha_{n,k}=\binom{n}{k}a_nz_{0}^{n-k}$ si $k\leqslant n$ et 0 sinon. Or 
            \begin{equation}
                \sum_{n=0}^{+\infty}\sum_{k=0}^{+\infty}\left\lvert \alpha_{n,k}\right\rvert\left\lvert h\right\rvert^{k}=\sum_{n=0}^{+\infty}\left\lvert a_n\right\rvert\sum_{k=0}^{n}\binom{n}{k}\left\lvert z_{0}\right\rvert^{n-k}\left\lvert h\right\rvert^{k}=\sum_{n=0}^{+\infty}\left\lvert a_n\right\rvert\left(\left\lvert h\right\rvert+\left\lvert z_{0}\right\rvert\right)^{n}<+\infty,
            \end{equation}
            
            car $\left\lvert h\right\rvert+\left\lvert z_{0}\right\rvert<R$. D'après le théorème de Fubini, on a 
            \begin{equation*}
                f(z_{0}+h)=\sum_{k=0}^{+\infty}\underbrace{\left(\sum_{n=0}^{+\infty}\alpha_{n,k}\right)}_{b_k}h^{k}.
            \end{equation*}
        \end{proof}

        On a pour tout $h\in\R$ tel que $\left\lvert h\right\rvert<\frac{1}{a}-1$. On a $G_Y(1+h)=\sum_{n=0}^{+\infty}b_nh^{n}$ et $b_n=\frac{G_Y^{(n)}(1)}{n!}$. Or $\mathbb{P}(Y=k)=k!G_Y^{(k)}(0)$. On peut encore développer $G_Y$ au voisinage de $2-\frac{1}{a}$, et de proche en proche, au voisinage de $1-2^{k}\left(\frac{1}{a}-1\right)$, jusqu'à 0. On retrouve ainsi la loi de $Y$.
    \end{enumerate}
\end{proof}

\begin{proof}
    \phantom{}
    \begin{enumerate}
        \item Pour tout $t\in[0,2\pi]$, on a $\left\lvert r\e^{\i t}\right\rvert>\left\lvert z\right\rvert$, donc $r\e^{\i t}-z\neq0$ et $g$ est bien définie. Soit 
        \function{F}{[0,1]\times[0,2\pi]}{\C}{(\lambda,t)}{
            \frac{f\left((1-\lambda)z+\lambda r\e^{\i t}\right)-f(z)}{r\e^{\i t}-z}r\e^{\i t}
        }
        $F$ est continue sur $[0,1]\times[0,2\pi]$ et $[0,1]\times[0,2\pi]$ est compact donc $F$ est bornée. Ainsi, $g$ est continue (théorème de continuité des intégrales à paramètres). 

        On a $\frac{\partial F}{\partial \lambda}(\lambda,t)=f'\left((1-\lambda)z+\lambda r\e^{\i t}\right)r\e^{\i t}$. C'est une fonction continue sur $[0,1]\times[0,2\pi]$, donc bornée. D'après le théorème de Leibniz, $g$ est de classe $\mathcal{C}^{1}$. Ainsi, pour tout $\lambda\in]0,1]$, on a 
        \begin{equation}
            g'(\lambda)=\int_{0}^{2\pi}f'\left((1-\lambda)z+\lambda r\e^{\i t}\right)r\e^{\i t}\d t=\left[\frac{1}{\i\lambda}f\left((1-\lambda)z+\lambda r\e^{\i t}\right)\right]_{0}^{2\pi}=0,
        \end{equation}
        par continuité de $g'$, c'est aussi vrai en $\lambda=0$. Donc $g$ est constante sur $[0,1]$. De plus, $g(0)=0=g(1)=\int_{0}^{2\pi}\frac{f\left(r\e^{\i t}\right)-f(z)}{r\e^{\i t}-z}r\e^{\i t}\d t$. Donc 
        \begin{equation}
            f(z)\int_{0}^{2\pi}\frac{r\e^{\i t}}{r\e^{\i t}-z}\d t=\int_{0}^{2\pi}\frac{f\left(r\e^{\i t}\right)-f(z)}{r\e^{\i t}-z}r\e^{\i t}\d t.
        \end{equation}
        Pour tout $t\in[0,2\pi]$, on a $\frac{r\e^{\i t}}{r\e^{\i t}-z}=\frac{1}{1-\frac{z}{r}\e^{\i t}}$. Comme $\left\lvert\frac{z}{r}\right\rvert<1$, on a 
        \begin{equation}
            \frac{r\e^{\i t}}{r\e^{\i t}-z}=\sum_{n=0}^{+\infty}\left(\frac{z\e^{-\i t}}{r}\right)^{n}.
        \end{equation}
        De plus, $\left\lvert \frac{z\e^{-\i t}}{r}\right\rvert^{n}=\left\lvert\frac{z}{r}\right\rvert^{n}$, terme général d'une série à termes positifs convergente indépendant de $t$, et on a 
        \begin{equation}
            \left\lvert f\left(r\e^{\i t}\right)\frac{z\e^{-\i t}}{r}\right\rvert^{n}\leqslant\left\lVert f\right\rVert_{\infty,\mathcal{C}(0,r)}\left\lvert\frac{z}{r}\right\rvert^{n}.
        \end{equation}

        Ainsi, on a 
        \begin{equation}
            f(z)\times\sum_{n=0}^{+\infty}\int_{0}^{2\pi}\left(\frac{z\e^{-\i t}}{r}\right)^{n}\d t=\sum_{n=0}^{+\infty}\int_{0}^{2\pi}\left(f\left(r\e^{\i t}\right)\frac{z\e^{-\i t}}{r}\right)^{n}\d t=\sum_{n=0}^{+\infty}\left(\frac{z}{r}\right)^{k}\underbrace{\int_{0}^{2\pi}\e^{-\i nt}\d t}_{2\pi\delta_{n,0}}.
        \end{equation}

        Ainsi,
        \begin{equation}
            f(z)=\sum_{n=0}^{+\infty}\left(\frac{z}{r}\right)^{n}\int_{0}^{2\pi}f\left(r\e^{\i t}\right)\e^{-\i nt}\d t.
        \end{equation}

        Ceci valant pour $t\in]0,R[$ fixé, pour tout $z\in D(0,r)$, $f(z)=\sum_{n=0}^{+\infty}a_n z^{n}$. Donc pour tout $n\in\N$, $a_n=\frac{f^{(n)}(0)}{n!}$, qui ne dépend pas de r. Ainsi, $f$ est développable en série entière sur tout $D(0,R)$.

        \item On applique ce qui précède à $h\mapsto f(z_{0}+h)$.
    \end{enumerate}
\end{proof}

\begin{remark}
    $f$ est $\mathcal{C}^{1}$ au sens complexe si et seulement si $f$ est développable en série entière au voisinage de tout point $z_{0}\in U$ (avec un rayon de convergence plus grand que $d(z_{0},\partial U)$) si et seulement si $f$ est $\mathcal{C}^{\infty}$ au sens complexe.
\end{remark}

\begin{remark}[Théorème de Liouville]
    Si $f$ est $\mathcal{C}^{1}$ au sens complexe de $\C\to\C$, alors il existe $(a_n)_{n\in\N}$ telle que pour tout $z\in\C$, $f(z)=\sum_{n=0}^{+\infty}a_nz^{n}$ (rayon de convergence $+\infty$) et pour tout $r>0$,
    \begin{equation}
        a_n=\frac{1}{2\pi r^{n}}\int_{0}^{2\pi}f\left(r\e^{\i t}\right)\e^{-\i nt}\d t.
    \end{equation}
    Si de plus $f$ est bornée sur $\C$ et pour tout $n\geqslant1$, pour tout $r>0$, on a $\left\lvert a_n\right\rvert\leqslant\frac{\left\lVert f\right\rVert_{\infty}}{r^{n}}$. Quand $r\to+\infty$, on a $a_n=0$ donc $f$ est constante.

    Application : soit $P\in\C[X]$ non constant. Comme $\lim\limits_{\left\lvert z\right\rvert}\left\lvert P(z)\right\rvert=+\infty$. On sait qu'il existe $m=\min\limits_{z\in\C}\left\lvert P(z)\right\rvert$, et si $m>0$, $f=\frac{1}{P}$ est $\mathcal{C}^{1}$ au sens complexe et bornée sur $\C$ donc constante : impossible. On vient de redémontrer le théorème de d'Alembert Gauss.
\end{remark}

\begin{proof}
    Soit $x\neq0$. Si, pour tout $p\in\N$, $u_p=\left\lvert\frac{x^{3p}}{(3p)!}\right\rvert>0$, on a 
    \begin{equation}
        \frac{u_{p+1}}{u_{p}}=\frac{x^{3}}{(3p+1)(3p+2)-3p+3}\xrightarrow[p\to+\infty]{}0.
    \end{equation}
    Donc le rayon de convergence est $R=+\infty$.

    Notons $S_1(x)=\sum_{n=0}^{+\infty}\frac{x^{3n+1}}{(3n+1)!}$ et $S_2(x)=\sum_{n=0}^{+\infty}\frac{x^{3n+2}}{(3n+2)!}$. On a 
    \begin{equation}
        \begin{array}[]{rcl}
            S_0(x)+S_1(x)+S_2(x) &=& \sum_{n=0}^{+\infty}\frac{x^{n}}{n!}=\e^{x},\\
            S_0(x)+\j S_{1}(x)+\j^{2}S_{2}(x) &=& \e^{\j x},\\
            S_0(x)+\j^{2} S_{1}(x)+\j S_{2}(x) &=& \e^{\j^{2} x}.
        \end{array}
    \end{equation}

    En effet, $\j=\j^{3n+1}$ et $\j^{2}=\j^{3n+2}$ pour tout $n\in\N$. En sommant, on a $3S_{0}(x)=\e^{x}+\e^{\j x}+\e^{\j^{2}x}$. Donc 
    \begin{equation}
        \boxed{
            S_0(x)=\frac{1}{3}\left(\e^{x}+\e^{-\frac{1}{2}x}+2\cos\left(\frac{\sqrt{3}}{2}x\right)\right).
        }
    \end{equation}
\end{proof}

\begin{remark}
    Autre méthode possible : on a $S_2'(x)=S_1(x)$ et $S_1'(x)=S_0(x)$. Donc $S_{2}''(x)+S_2'(x)+S_2(x)=\e^{x}$. L'équation homogène a pour solution générale $\lambda \e^{\j x}+\mu\e^{\j^{2}x}$, avec une solution particulière $P(x)\e^{x}$, avec $P$ constante car $1$ n'est pas racine de $X^{2}+X+1$. On trouve $\frac{\e^{x}}{3}$, donc $S_2(x)=\frac{\e^{x}}{3}+\lambda\e^{\j x}+\mu\e^{\j^{2}x}$, on identifie $S_2(0)=0$ et $S_2'(0)=0$, puis $S_0=S_2''$.
\end{remark}

\begin{proof}
    \phantom{}
    \begin{enumerate}
        \item Si \function{f_1}{\R}{\R}{x}{f(x)}
        on a pour tout $n\in\N$, $a_n=\frac{f_1^{(n)}(0)}{n!}\in\R$.
        \item Soit $\theta\in[0,\pi]$, on a 
        \begin{align}
            v\left(r\e^{\i\theta}\right)
            &=\sum_{n=0}^{+\infty}a_n\Im\left(r^{m}\e^{\i\theta n}\right),\\
            &=\sum_{n=0}^{+\infty}a_n r^{n}\sin\left(n\theta\right),
        \end{align}
        car les $a_n\in\R$.

        Pour $m\geqslant1$ fixé, on a
        \begin{equation}
            v\left(r\e^{\i\theta}\right)\sin(m\theta)=\sum_{n=0}^{+\infty}a_n r^{n}\sin(n\theta)\sin(m\theta)=\sum_{n=0}^{+\infty}f_n(\theta),
        \end{equation}
        avec $f_n$ continue sur $[0,\pi]$. Pour tout $n\in\N$, pour tout 
        $\theta\in[0,\pi]$, $\left\lvert f_n(\theta)\right\rvert\leqslant\left\lvert a_n r^{n}\right\rvert$, terme général d'une série à termes positifs convergente car $R=+\infty$. Donc $\sum f_n$ converge normalement sur $[0,\pi]$, on peut intervertir
        \begin{align}
            \int_{0}^{\pi}v\left(r\e^{\i\theta}\right)\sin(m\theta)\d\theta
            &=\sum_{n=0}^{+\infty}a_n r^{n}\int_{0}^{\pi}\sin(n\theta)\sin(m\theta)\d\theta,\\
            &=a_m r^{m}\frac{\pi}{2}.
        \end{align}
        
        \item
        \begin{lemma}
            \label{lem:3}
            Pour tout $m\in\N$, on a $\left\lvert\sin(m\theta)\right\rvert\leqslant m\left\lvert\sin(\theta)\right\rvert$.
        \end{lemma}
        \begin{proof}[Preuve du lemme~\ref{lem:3}]
            Par récurrence, car 
            \begin{align}
                \left\lvert\sin\left((m+1)\theta\right)\right\rvert
                &=\left\lvert\sin(m\theta)\cos(\theta)+\sin(\theta)\cos(m\theta)\right\rvert,\\
                &\leqslant \left\lvert\sin(m\theta)\right\rvert+\left\lvert\sin(\theta)\right\rvert,\\
                &\leqslant (m+1)\left\lvert\sin(\theta)\right\rvert.
            \end{align}
        \end{proof}

        Donc 
        \begin{equation}
            \left\lvert r^{m}a_m\right\rvert\leqslant\frac{2}{\pi}\int_{0}^{\pi}\left\lvert v(r\e^{\i\theta})\right\rvert m\left\lvert \sin(\theta)\right\rvert\d\theta.
        \end{equation}
        $\sin$ est positif sur $[0,\pi]$, et pour tout $\theta\in]0,\pi[$, si $v(r\e^{\i\theta})=0$, $f(r\e^{\i\theta})\in\R$ donc $r\e^{\i\theta}\in\R$ ce qui est exclu. Ainsi, $v(r\e^{\i\theta})$ a un signe constant sur $[0,pi]$, et 
        \begin{equation}
            \left\lvert\int_{0}^{\pi}v(r\e^{\i\theta})\sin(m\theta)\d\theta\right\rvert=\int_{0}^{\pi}\left\lvert v(r\e^{\i\theta})\right\rvert\sin(\theta)\d\theta.
        \end{equation}

        Finalement, on a $\left\lvert r^{m}a_m\right\rvert\leqslant mr\left\lvert a_1\right\rvert$, d'où $\left\lvert a_m\right\rvert\leqslant\frac{m}{r^{m-1}}\left\lvert a_1\right\vert$. Pour $m\geqslant2$, lorsque $r\to+\infty$, on obtient $a_m=0$. Donc $f$ est affine.
    \end{enumerate}
\end{proof}

\begin{proof}
    \phantom{}
    \begin{enumerate}
        \item On a 
        \begin{equation}
            \int_{0}^{2\pi}r^{\left\lvert k\right\rvert}f\left(\e^{\i(x-t)}\right)\d t=\int_{0}^{2\pi}\sum_{n=0}^{+\infty}\underbrace{a_n\e^{\i n(x-t)+\i kt}}r^{\left\lvert k\right\rvert}_{f_n(t)}\d t.
        \end{equation}
        $f_n$ est continue sur $[0,2\pi]$, avec $\left\lvert f_n(t)\right\rvert\leqslant\left\lvert a_n r^{\left\lvert k\right\rvert}\right\rvert$ terme général d'une série à termes positifs convergente. donc $\sum_{n\geqslant0}f_n(t)$ converge normalement sur $[0,2\pi]$. Ainsi,
        \begin{align}
            \int_{0}^{2\pi}r^{\left\lvert k\right\rvert}\e^{\i kt}f\left(\e^{\i(x-t)}\right)\d t
            &=\sum_{n=0}^{+\infty}a_n\e^{\i nx}r^{\left\lvert k\right\rvert}\int_{0}^{2\pi}\e^{\i kt}\e^{-\i nt}\d t,\\
            &=
            \left\lbrace
                \begin{array}[]{ll}
                    2\pi r^{\left\lvert k\right\rvert}a_{k}\e^{\i kx} &\text{si }k\geqslant0,\\
                    0 &\text{sinon.}
                \end{array}
            \right.
        \end{align}

        Puis 
        \begin{equation}
            \int_{0}^{+\infty}P_r(t)f\left(\e^{\i(x-t)}\right)\d t=\int_{0}^{2\pi}\sum_{k=-\infty}^{+\infty}\underbrace{r^{\left\lvert k\right\rvert}\e^{\i kt}f\left(\e^{\i (x-t)}\right)}_{g_k(t)}\d t.
        \end{equation}
        $g_k$ est continue sur $[0,2\pi]$, et $\left\lvert g_k(t)\right\rvert\leqslant r^{\left\lvert k\right\rvert}\left\lVert f\right\rVert_{\infty,\overline{D(0,1)}}$ et $\sum_{k\in\Z}r^{\left\lvert k\right\rvert}<\infty$. On a donc convergence normale sur $[0,2\pi]$, et 
        \begin{align}
            \int_{0}^{2\pi}P_r(t)f\left(\e^{\i(x-t)}\right)\d t
            &=\sum_{k=-\infty}^{+\infty} r^{\left\lvert k\right\rvert}\int_{0}^{2\pi}\e^{\i kt}f\left(\e^{\i(x-t)}\right)\d t,\\
            &=\sum_{k=0}^{+\infty} 2\pi r^{k}a_k\e^{\i kx},\\
            &=2\pi f\left(r\e^{\i x}\right).
        \end{align}

        \item On a 
        \begin{align}
            P_r(x)
            &=\sum_{k=0}^{+\infty}r^{k}\e^{\i kx}+\sum_{k=0}^{+\infty}r^{k}\e^{-\i kx}-1,\\
            &=\frac{1}{1-r\e^{\i x}}+\frac{1}{1-r\e^{-\i x}}-1,\\
            &=\frac{1-r^{2}}{1+r^{2}-2r\cos(x)},
        \end{align}
        et $1+r^{2}-2r\cos(x)=\left(1-r\cos(x)\right)^{2}+r^{2}\sin^{2}(x)>0$, donc $P_r>0$. On applique le résultat du a) pour $f=1$ et on obtient 
        \begin{equation}
            \frac{1}{2\pi}\int_{0}^{2\pi}P_r(t)\d t=1.
        \end{equation}

        \item Si $f(\U)\subset\U$, prenons $z\in D(0,1)$, soit $z+r\e^{\i x}$, $r\in[0,1[$ et $x\in\R$,
        \begin{align}
            \left\lvert f\left(r\e^{\i x}\right)\right\rvert 
            &= \left\lvert \frac{1}{2\pi}\int_{0}^{2\pi}P_r(t)f\left(\e^{\i(x-t)}\right)\d t\right\rvert,\\
            &\leqslant\frac{1}{2\pi}\int_{0}^{2\pi}P_r(t)\left\lvert f\left(\e^{\i(x-t)}\right)\right\rvert\d t,\\
            &\leqslant\frac{1}{2\pi}\int_{0}^{2\pi}P_r(t)\d t=1.
        \end{align}
        Donc $f(z)\in\overline{D(0,1)}$ et $f\left(\overline{D(0,1)}\right)\subset\overline{D(0,1)}$.
    \end{enumerate}
\end{proof}

\begin{proof}
    \phantom{}
    \begin{enumerate}
        \item L'espérance vaut la série harmonique $H_n\underset{n\to+\infty}{\sim}\ln(n)$ par linéarité. Par indépendance, la variance vaut $\sum_{k=1}^{n}\frac{1}{k}\left(1-\frac{1}{k}\right)\underset{n\to+\infty}{\sim}\ln(n)$.
        \item On a 
        \begin{equation}
            \left(\left\lvert \frac{R_n}{\ln(n)}-1\right\rvert\right)\subset \underbrace{\left(\left\lvert \frac{R_n}{\ln(n)}-\frac{\mathbb{E}(R_n)}{\ln(n)}\right\rvert>\frac{\varepsilon}{2}\right)}_{B_n}\bigcup\underbrace{\left(\left\lvert\frac{\mathbb{E}(R_n)}{\ln(n)}-1\right\rvert>\frac{\varepsilon}{2}\right)}_{C_n}.
        \end{equation}
        $C_n$ est nul à partir d'un certain rang car $\lim\limits_{n\to+\infty}\frac{\mathbb{E}(R_n)}{\ln(n)}=1$. De plus, 
        \begin{equation}
            B_n=\left(\left\lvert R_n-\mathbb{E}(R_n)\right\rvert\frac{\varepsilon}{2}\ln(n)\right),    
        \end{equation}
        donc d'après l'inégalité de Bienaymé-Tchebychev, $B_n<\frac{4\mathbb{V}(R_n)}{\varepsilon^{2}\ln^{2}(n)}\underset{n\to+\infty}{\sim}\frac{4}{\varepsilon^{2}\ln(n)}\xrightarrow[n\to+\infty]{}0$.

        \item On a 
        \begin{align}
            G_{R_n}(t)
            &=\mathbb{E}\left(t^{R_n}\right),\\
            &=\prod_{k=1}^{n}\mathbb{E}(t^{\chi_{A_k}}),\\
            &=\prod_{k=1}^{n}\left(1-\frac{1}{k}+\frac{t}{k}\right),\\
            &=\frac{t}{n!}\prod_{k=1}^{n}\left(k-1+t\right),
        \end{align}
        car les $(\chi_{A_k})_{k\geqslant1}$ sont indépendants. $\mathbb{P}(R_n=1)$ est le coefficient en $t$ de $G_{R_n}$, et vaut donc $\frac{(n-1)!}{n!}=\frac{1}{n}$. De même, $\mathbb{P}(R_n=2)$ est le coefficient en $t^{2}$ de $G_{R_n}$ et vaut donc $\frac{1}{n!}\sum_{k=2}^{n}\frac{(n-1)!}{k-1}=\frac{1}{n}\sum_{k=1}^{n-1}\frac{1}{k}$.

        \item On a $T_n=\sum_{k=na+1}^{nb}\chi_{A_k}$, donc 
        \begin{equation}
            G_{T_n}(t)=\prod_{k=na+1}^{nb}\left(1-\frac{1}{k}+\frac{t}{k}\right).
        \end{equation}
        Ainsi, $\ln\left(G_{T_n}(t)\right)=\sum_{k=na+1}^{nb}\ln\left(1+\frac{t-1}{k}\right)$. Pour $x>1$, soit $g(x)=\ln(1+x)-x+\frac{x^{2}}{2}$. On a $g'(x)=\frac{x^{2}}{1+x}\geqslant0$ et $g(0)=0$ donc $g\geqslant0$.

        On a 
        \begin{equation}
            \sum_{k=na+1}^{nb}\frac{t-1}{k}-\frac{1}{2}\sum_{k=na+1}^{nb}\left(\frac{t-1}{k}\right)^{2}\leqslant\ln\left(G_n(t)\right)\leqslant\sum_{k=na+1}^{nb}\frac{t-1}{k}.
        \end{equation}
        Comme $0\leqslant\frac{1}{2}\sum_{k=na+1}^{nb}\left(\frac{t-1}{k}\right)^{2}\leqslant\frac{(t-1)^{2}}{2}\sum_{k=na+1}^{+\infty}\frac{1}{k^{2}}\xrightarrow[n\to+\infty]{}0$, et $\sum_{k=na+1}^{nb}\frac{t-1}{k}=(t-1)(H_{nb}-H_{na})\xrightarrow[n\to+\infty]{}(t-1)\ln\left(\frac{b}{a}\right)$.
        Donc $\lim\limits_{n\to+\infty}G_{T_n}(t)=\e^{\ln\left(\frac{b}{a}\right)(t-1)}$. Il s'agit de la fonction génératrice d'une variable aléatoire suivant une loi de Poisson de paramètre $\ln\left(\frac{a}{b}\right)$.
    \end{enumerate}
\end{proof}

\begin{proof}
    \phantom{}
    \begin{enumerate}
        \item 
        \begin{enumerate}
            \item Soit $a_n=\P(S_n=0)$. On a $0\leqslant a_n\leqslant1$ donc le rayon de convergence de $f$ est plus grand que 1 et $f$ est définie et $\mathcal{C}^{\infty}$, de même pour $g$.
            \item Soit $g_k(t)=\P(T=k)t^{k}$. Pour tout $t\in[0,1]$, on a $\left\lvert g_k(t)\right\rvert\leqslant\P(T=k)$, terme général d'une série à termes positifs convergente indépendant de $t$, car $\sum_{k=0}^{+\infty}\P(T=k)=\pi\leqslant1$, donc $\sum g_k$ converge normalement sur $[0,1]$. Donc $\lim\limits_{t\to1^{-}}g(t)=\sum_{k=1}^{+\infty}\P(T=k)=\pi$.
        \end{enumerate}

        \item On a $(S_{m+1}-S_m,\dots,S_{m+k}-S_{m})=(X_{m+1},X_{m+1}+X_{m+2},\dots,X_{n+1}+\dots+X_{m+k})$. $X_{m+1}+\dots X_{m+r}$ a pour loi la convoluée de la loi de $X$ r fois par elle-même car $X_i\sim X$ et les $(X_i)_{1\geqslant i}$ sont indépendants. Donc $X_m+\dots+X_{m+r}\sim X_1+\dots+X_r$, d'où le résultat.
        
        \item 
        \begin{enumerate}
            \item $\P(S_n=0)=\sum_{k=1}^{n}\P(T=k)\P_{T=k}(S_n=0)$ car si on revient à 0 à l'instant $n$, le 1er retour en 0 a eu lieu à un instant $k\in\left\llbracket1,n\right\rrbracket$. D'après ce qui précède, on a donc 
            \begin{equation}
                \P_{T=k}(S_n=0)=\P_{S_k=0}(S_n-S_k=0)=\P(S_{n-k}=0).
            \end{equation}

            \item Posons $P(T=0)=0$, d'où $g(t)=\sum_{k=0}^{+\infty}\P(T=k)t^{k}$ et pour tout $n\geqslant1$,
            \begin{equation}
                \sum_{k=0}^{n}\P(T=k)\P(S_{n-k}=0)=\P(S_n=0),
            \end{equation}
            puis par produit de Cauchy, 
            \begin{equation}
                f(t)=\P(S_0=0)+\sum_{n=1}^{+\infty}\P(S_n=0)t^{n}=1+f(t)g(t).
            \end{equation}
        \end{enumerate}

        \item 
        \begin{enumerate}
            \item Soit $j=\left\lvert\left\lbrace i\in\left\llbracket1,n\right\rrbracket\middle|X_i=1\right\rbrace\right\rvert$. On a $S_n=j-(n-j)=2j-n$ (on est allé $j$ fois à droite et $n-j$ fois à gauche). Si $n=2p+1$, $S_n=S_{2p+1}=2j-2p-1\neq0$. Pour que $S_{2n}=0$, il faut et il suffit que $j=n$, ce qui revient à une loi binomiale $\mathcal{B}(2n,p)$ d'où $\P(S_{2n}=0)=\binom{2n}{n}(pq)^{n}$.
            
            \item On a $f(t)=\sum_{n=0}^{+\infty}\P(S_n=0)t^{n}=\sum_{n=0}^{+\infty}\binom{2n}{n}(pq)^{n}t^{2n}$. Comme $p\in]0,1[$, $p(1-p)=pq\leqslant\frac{1}{4}$, et si $t\in[0,1[$, 
            \begin{align}
                \frac{1}{\sqrt{1-4pqt^{2}}}
                &=(1-4pqt^{2})^{-1},\\
                &=\sum_{n=0}^{+\infty}\frac{(-1)^{n}\left(-\frac{1}{2}\right)\left(-\frac{1}{2}-1\right)\dots\left(-\frac{1}{2}-n+1\right)}{n!}(4pqt^{2})^{n},\\
                &=\sum_{n=0}^{+\infty}\frac{\frac{1}{2}\left(1+\frac{1}{2}\right)\dots\left(n-1+\frac{1}{2}\right)}{n!}(4pqt^{2})^{n},\\
                &=\sum_{n=0}^{+\infty}\frac{1\times 3\dots \times (2n-1)}{2^{n}n!}(4pqt^{2})^{n},\\
                &=\sum_{n=0}^{+\infty}\frac{(2n)!}{2^{2n}(n!)^{2}}(4pqt^{2})^{n},\\
                &=\sum_{n=0}^{+\infty}\binom{2n}{n}p^{n}q^{n}t^{2n}=f(t).
            \end{align}
        \end{enumerate}

        \item 
        \begin{enumerate}
            \item Pour tout $t\in]0,1[$, on a 
            \begin{equation}
                g(t)=\frac{f(t)-1}{f(t)}=1-\sqrt{1-4pqt^{2}},
            \end{equation}
            car $f(t)\neq0$. D'après 1.(b), on a 
            \begin{equation}
                \pi=\lim\limits_{t\to1^{-}}g(t)=1-\sqrt{1-4p(1-p)}=1-\left\lvert 2p-1\right\rvert=1-\left\lvert p-q\right\rvert,
            \end{equation}
            et $\pi=1$ si et seulement si $p=q=\frac{1}{2}$.

            \item Pour tout $t\in[0,1]$, $g(t)=1-\sqrt{1-4p(1-p)t^2}$. Or $\left\lvert pqt^{2}\right\rvert<1$ donc $g$ est développable en série entière sur $[0,1[$, et on a 
            \begin{align}
                g(t)
                &=1-\left(1+\sum_{n=1}^{+\infty}\frac{\frac{1}{2}\left(\frac{1}{2}-1\right)\dots\left(\frac{1}{2}-n+1\right)}{n!}(-4pqt^{2})^{n}\right),\\
                &=\sum_{n=1}^{+\infty}\frac{1\times 1\times 3\times\dots\times(2n-3)}{2^{n}n!}(4pqt^{2})^{n},\\
                &=\sum_{n=1}^{+\infty}\frac{(2n-2)!}{2^{n-1}(n-1)!2^{n}n!}4^{n}(pq)^{n}t^{2n},\\
                &=\sum_{n=1}^{+\infty}\frac{(2n-2)!}{n(n-1)!^{2}}2(pq)^{n}t^{2n}.
            \end{align}
            Par unicité du développement, on a $\P(T=2n+1)=0$ et $\P(T=2n)=\binom{2n-2}{n-1}\frac{2(pq)^{n}}{n}$.
        \end{enumerate}

        \item 
        \begin{enumerate}
            \item Si $p=\frac{1}{2}$, on a $\pi=1$ d'où $\P(T=+\infty)=0$. Donc 
            \begin{equation}
                \mathbb{E}(T)=\sum_{n=1}^{+\infty}\binom{2n-2}{n-1}\frac{2n}{n4^{n-1}}=\sum_{n=0}^{+\infty}\binom{2n}{n}\frac{1}{4^{n}}.
            \end{equation}
            Or $\binom{2n}{n}\frac{1}{4^{n}}=\frac{(2n)!}{(n!)^{2}}\frac{1}{4^{n}}\underset{n\to+\infty}{\sim}\frac{(2n)^{2n}}{n^{2n}}\frac{\sqrt{4\pi n}}{2\pi n}\frac{1}{4^{n}}\underset{n\to+\infty}{\sim}\frac{1}{\sqrt{\pi n}}$, terme général d'une série divergente, d'où le résultat.

            \item Si $p\neq\frac{1}{2}$, on a 
            \begin{equation}
                \mathbb{E}(T\times \mathbf{1}_{T<+\infty})=\sum_{k=1}^{+\infty}k\P(T=k)=g'(1)=\frac{4pq}{\sqrt{1-4pq}},
            \end{equation}
            car $g(t)=1-\sqrt{1-4pqt^{2}}$. On a $\P(T<+\infty)=\pi=1-\left\lvert p-q\right\rvert$ et comme $\sqrt{1-4pq}=\left\lvert p-q\right\rvert$, d'où 
            \begin{equation}
                \mathbb{E}_{T<+\infty}(T)=\frac{4pq}{\left\lvert p-q\right\rvert(1-\left\lvert p-q\right\rvert)}.
            \end{equation}
        \end{enumerate}

        \item $\pi=1$ si et seulement si $p=\frac{1}{2}$ d'après 6.
        
        \item $f$ est croissante sur $[0,1[$ donc il existe $l\in\overline{\R}$ telle que $\lim\limits_{t\to1^{-}}f(t)=l=\sup\limits_{t<1}f(t)$.
        
        \begin{itemize}
            \item Si $\sum_{n=0}^{+\infty}\P(S_n=0)=+\infty$, soit $N\in\N$, on a $\sum_{n=0}^{N}\P(S_n=0)t^{n}\leqslant f(t)\leqslant l$. $N$ étant fixé, on peut faire tendre vers 1 d'où $\sum_{n=0}^{+\infty}\P(S_n=0)\leqslant l$. Lorsque $N\to+\infty$, on obtient $\lim\limits_{t\to1^{-}}f(t)=+\infty$.
            
            \item Si $(S_n)_{n\geqslant0}$ n'est pas récurrente, alors $\pi\neq1$ et $p\neq\frac{1}{2}$. Pour tout $t\in[0,1[$,
            \begin{equation}
                f(t)=\frac{1}{\sqrt{1-4pqt^{2}}}\xrightarrow[t\to1^{-}]{}\frac{1}{\sqrt{1-4pq}},
            \end{equation}
            défini car $4pq=4p(1-p)<1$.

            \item Si $p=\frac{1}{2}$, $\P(S_n=0)=\binom{2n}{n}\frac{1}{4^{n}}$ d'après 4.(a), et $\P(S_n=0)\underset{n\to+\infty}{\sim}\frac{1}{\sqrt{\pi n}}$, terme général d'une série divergente donc $\sum_{n=0}^{+\infty}\P(S_n=0)=+\infty$.
        \end{itemize}

        \item 
        \begin{enumerate}
            \item On a 
            \begin{align}
                \mathbb{E}(N)
                &=\lim\limits_{n\to+\infty}\mathbb{E}(N_n),\\
                &=\sum_{k=1}^{+\infty}\mathbb{E}(\mathbf{1}_{\left\lbrace S_n=0\right\rbrace}),\\
                &=\sum_{k=1}^{+\infty}\P(S_k=0),\\
                &=f(1)-\P(S_0=0),\\
                &=f(1)-1.
            \end{align}
            Or $f(1)=1+f(1)g(1)$ et $g(1)=\pi$, donc $f(1)=\frac{1}{1-\pi}$, d'où 
            \begin{equation}
                \mathbb{E}(N)=\frac{\pi}{1-\pi}.
            \end{equation}

            \item Si $p=\frac{1}{2}$, $\sum\P(S_n=0)$ diverge donc $\mathbb{E}(N)=+\infty$.
        \end{enumerate}

        \item Pour tout $k\geqslant1$, on a 
        \begin{equation}
            \P(S_k=x)=\sum_{j=0}^{k}\P(T^{x}=j)\P_{T^{x}=j}(S_k=x).
        \end{equation}
        Par produit de polynômes, $f_{n,x}(t)=\P(S_0=x)+\sum_{k=1}^{n}\P(S_k=x)t^{k}$ et $\P(S_0=x)=0$ car $x\neq0$. Soit 
        \begin{equation}
            f_{n,x}(t)=\sum_{k=1}^{n}\P(T^{x}=j)\P(X_{k-j}=0)t^{k},
        \end{equation}
        pour $t\in[0,1]$. Soit $f_n(t)=\sum_{k=0}^{n}\P(S_k=0)t^{k}$, on a 
        \begin{equation}
            f_n(t)g_{n,x}(t)=\left(\sum_{l=0}^{n}\P(S_l=0)t^{k}\right)\left(\sum_{j=0}^{n}\P(T^{x}=j)t^{j}\right),
        \end{equation}
        donc $f_{n,x}(t)\leqslant f_n(t)g_{n,x}(t)$. Or $N_{n,x}=\sum_{j=1}^{n}\mathbf{1}_{\left\lbrace S_j=x\right\rbrace}$ et 
        \begin{equation}
            \mathbb{E}(N_{n,x})=f_{n,x}(1)\leqslant f_n(1)g_{n,x}(1)=\mathbb{E}(N_{n})\sum_{k=1}^{n}\P(T^{x}=k)\leqslant\mathbb{E}(N_{n}),
        \end{equation}
        donc $\mathbb{E}(N_{n,x})\leqslant\mathbb{E}(N_{n})$.

        \item 
        \begin{equation}
            \left\lbrace \left(\left\lVert S_n\right\rVert\right)_{n\geqslant1}\right\rbrace=\left\lbrace\exists A\geqslant0\middle|\left\lbrace n\in\N\middle|\left\lVert S_n\right\rVert\leqslant A\right\rbrace\text{ est fini}\right\rbrace=\mathcal{A}.    
        \end{equation}
        Comme $S_n$ est à valeur dans $\Z^{d}$, il y a un nombre fini de points dans $\overline{B(0,A)}$. On a 
        \begin{equation}
            \mathcal{A}=\left\lbrace x\in\Z^{d}\middle|\left\lbrace n\in\N\middle| S_n=x\right\rbrace\text{ est fini}\right\rbrace.
        \end{equation}
        Soit $x\in\Z^{d}$ fixé, et $\mathcal{A}^{x}=\left\lbrace\left\lbrace n\in\N\middle| S_n=x\right\rbrace\text{ est infini}\right\rbrace$, on a 
        \begin{equation}
            \mathcal{A}=\biguplus_{x\in\Z^{d}}\mathcal{A},
        \end{equation}
        (union disjointe) et $\P(\mathcal{A})=\sum_{x\in\Z^{d}}\P(\mathcal{A}^{x})$ dénombrable.

        Soit $N^{x}=\left\lvert\left\lbrace j\in\N\middle| S_j=x\right\rbrace\right\rvert$ (à valeurs dans $\N\cup \lbrace+\infty\rbrace$). $(N_{n}^{x})_{n\geqslant1}$ converge vers $N^{x}$ en croissant. Or $(\mathbb{E}(N_{n}^{x}))_{n\geqslant1}$ converge vers $\mathbb{E}(N^{x})$ dans $\R_{+}\cup\lbrace+\infty\rbrace$, et comme il s'agit d'une marche transitoire, on a 
        \begin{equation}
            \mathbb{E}(N_{n}^{x})\leqslant\mathbb{E}(N_{n})\leqslant\mathbb{E}(N)<+\infty,
        \end{equation}
        donc $\mathbb{E}(N^{x})$ est fini et 
        \begin{equation}
            \mathbb{E}(N^{x})=\sum_{k=1}^{+\infty}k\P(N^{x}=k)+(+\infty)\underbrace{\P(N^{x}=+\infty)}_{\P(\mathcal{A}^{x})}.
        \end{equation}
        Nécessairement, $\P(\mathcal{A}^{x})=0$ pour tout $x\in\Z^{d}$ et $\P(\mathcal{A})=0$.
    \end{enumerate}
\end{proof}

\begin{proof}
    On vérifie d'abord que $\sum_{n\in\N}\P(X=n)=1$. Ensuite, si $u_n=\P(X=n)$, l'étude de l'équation caractéristique donne $u_n=\lambda_{1}\left(\frac{1}{2}\right)^{n}+\lambda_{2}\left(\frac{1}{3}\right)^{n}$, et on trouve $\lambda_{1}=-3\alpha+2$ et $\lambda_{2}=4\alpha-2$ grâce à la condition initiale et la condition $\sum_{n}u_n=1$. Noter que $u_n\sim\frac{\lambda_1}{2^n}$ donc il faut (pour que cela reste positif) $0<\alpha<2/3$. On écrit ensuite $G_X(t)=\sum_{n}u_nt^{n}$. Grâce à l'équation caractéristique, on a 
    \begin{equation*}
        G_X(t)=\frac{6u_0+6u_1t-5tu_0}{6-5t+t^{2}},
    \end{equation*}
    et le rayon de convergence est 2. L'espérance est $G_X'(1)$ et la variance est $G_X''(1)-G_X'(1)^{2}+G_X'(1)$.
\end{proof}
\end{document}