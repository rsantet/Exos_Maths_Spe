\documentclass[12pt]{article}
\usepackage{style/style}

\begin{document}

\begin{titlepage}
	\centering
	\vspace*{\fill}
	\Huge \textit{\textbf{Exercices MP/MP$^*$\\ Probabilités sur un univers dénombrable}}
	\vspace*{\fill}
\end{titlepage}

\begin{exercise}
	On lance une seule fois une pièce équilibrée, puis on effectue des tirages
	avec remise dans une urne contenant initialement 1 boule noire et 1 boule
	blanche: si la pièce a donné pile (respectivement face), on rajoute à chaque
	fois une boule blanche (respectivement noire).
	\begin{enumerate}
		\item
		Quelle est la probabilité de tirer une boule blanche au $k$-ième tirage?
		\item
		Sachant que l'on a tiré une boule blanche au $k$-ième tirage, quelle est
		la probabilité $p_{k}$ d'avoir obtenu pile au lancer initial de la pièce ?
		\item
		Quelle est la probabilité d'obtenir $k$ boules blanches au cours des $k$
		premiers tirages?
		\item
		On note $B_{k}$ l'évènement où la $k$-ième boule est blanche. $B_{k}$ et
		$B_{k+1}$ sont-ils indépendants?
	\end{enumerate}
\end{exercise}

\begin{exercise}
	$A$ et $B$ s'affrontent dans une partie de pile ou face:
	$\P(P)=p,~\P(F)=1-p=q$ avec $p\in]0,1[$. Au départ, ils possèdent un total de
	$N$ euros. Après chaque lancer, le perdant donne un euro au gagnant. $A$ gagne
	si pile, perd si face. Le jeu s'arrête lorsqu'un des joueurs est ruiné (ou
	s'ils avaient 0 au départ).

	On note $p_{a}$ (respectivement $q_{a}$) la probabilité que $A$
	(respectivement $B$) soit ruiné (en temps fini) si $A$ a $a$ euros au départ.
	\begin{enumerate}
		\item
		Évaluer $p_{0},p_{N},q_{0},q_{N}$.
		\item
		Montrer que pour tout $a\in\{1,\dots,N-1\}$, $p_{a}=pp_{a+1}+qp_{a-1}$. En
		déduire l'expression de $p_{a}$.
		\item
		Calculer de même $q_{a}$, puis $p_{a}+q_{a}$. Qu'en déduit-on?
	\end{enumerate}
\end{exercise}

\begin{exercise}
	Deux archers tirent alternativement sur une cible, jusqu'à ce que l'un des
	deux la touche. $A$ commence. Il touche la cible avec une probabilité
	$a\in]0,1[$. $B$ touche la cible avec une probabilité $b\in]0,1[$. On note
	$G_{A}$ (respectivement $G_{B}$) l'évènement où $A$ (respectivement $B$)
	l'emporte.
	\begin{enumerate}
		\item
		Soit $n\in\N$, quelle est la probabilité pour que $A$ (respectivement $B$)
		l'emporte au rang $2n+1$ (respectivement $2n+2$). On note $A_{n}$
		(respectivement $B_{n}$) l'évènement correspondant.
		\item
		En déduire $\P(G_{A})$ et $\P(G_{B})$. Que vaut $\P(G_{A})+\P(G_{B})$?
		\item A quelle condition a-t-on $\P(G_{A})=\P(G_{B})$?
	\end{enumerate}
\end{exercise}

\begin{exercise}
	Un joueur lance une pièce équilibrée jusqu'à l'obtention du premier pile. S'il
	lui a fallu $n$ lancers pour l'obtenir, on lui fait tirer un billet de loterie
	parmi $n$ (un seul billet gagnant).
	\begin{enumerate}
		\item
		Quelle est la probabilité pour que le joueur gagne ?
		\item
		Sachant qu'il a gagné, quelle est la probabilité qu'il ait obtenu pile au
		$n$-ième lancer? Et qu'il ait obtenu pile ?
	\end{enumerate}
\end{exercise}

\begin{exercise}
	Dans une famille donnée, la probabilité pour qu'il y ait $k$ enfants est
	$p_{k}$ ($k\in\N$) avec $p_{0}=p_{1}=\alpha\in]0,\frac{1}{2}[$ et pour tout
	$k\geqslant2$, $p_{k}=\frac{1-2\alpha}{2^{k-1}}$. La probabilité qu'il y ait
	un garçon ou une fille est la même.
	\begin{enumerate}
		\item
		Vérifier que c'est une probabilité sur $\N$.
		\item
		Quelle est la probabilité qu'une famille ait exactement deux garçons?
		\item
		Quelle est la probabilité qu'une famille au au moins deux filles sachant
		qu'elle a au moins deux garçons?
	\end{enumerate}
\end{exercise}

\begin{exercise}
	Deux joueurs jouent avec deux dés non pipés. $A$ (respectivement) gagne s'il
	obtient un total de 6 (respectivement 7). $A$ commence. On s'arrête lorsqu'un
	des deux joueurs gagne. Quelle est la probabilité de succès des deux joueurs ?
\end{exercise}

\begin{exercise}
	Une urne contient $a$ boules blanches et $b$ noires. On en tire successivement
	$n$ au hasard avec remise. Quelle est la probabilité pour que le nombre de
	boules blanches tirées soit pair ?
\end{exercise}

\begin{exercise}
	Soit $n\in\N^{*}$, calculer la probabilité $p_{n}$ pour qu'une bijection de
	$\{1,\dots,n\}$ possède au moins un point fixe. Donner la limite de $p_{n}$
	quand $n\to+\infty$.
\end{exercise}

\begin{exercise}
	Soit $N\in\N^{*}$, $p\in]0,1[$ et $q=1-p$. On joue à pile ou face avec une
	probabilité $p$ d'obtenir pile. On gagne 1 euro si on tombe sur face, on perd
	un euro sinon. Le jeu s'arrête lorsque l'on a 0 ou $N$ euros. Pour
	$n\in\{0,\dots,N\}$, on note $p_{N}(n)$ la probabilité de gagner
	(respectivement de perdre) si on dispose au départ de $n$ euros.
	\begin{enumerate}
		\item
		Calculer $p_{N}(0)$ et $p_{N}(N)$.
		\item
		Calculer $p_{N}(n)$, $q_{N}(n)$ et $\lim\limits_{N\to+\infty}p_{N}(n)$ à
		$n$ fixé.
	\end{enumerate}
\end{exercise}

\begin{exercise}
	Soit $c\in\N^{*}$. Soit une urne contenant initialement une boule blanche et
	une noire. Après tirage, la boule tirée est remise avec $c$ autres boules de
	sa couleur. Soit, pour $n\geqslant1$, $p_{n}$ la probabilité pour que la
	première boule blanche apparaisse au $n$-ième tirage. Calculer $p_{n}$ et
	$\sum_{n\geqslant1}p_{n}$ lorsque 
	\begin{enumerate}
		\item
		$c=1$
		\item
		$c$ quelconque.
	\end{enumerate}
\end{exercise}

\begin{exercise}
	Soit $p\in]0,1[$ et $q=1-p$. Une bactérie vit un seul jour. \`A l'issue de
	cette journée, elle peut se diviser en deux avec la probabilité $p$ ou bien
	disparaître tristement sans laisser de traces avec la probabilité $q$. Pour
	$n\in\N$, on désigne par $U_{n}$ l'évènement indiquant que la lignée d'une
	bactérie donnée est éteinte au $n$-ième jour. On note $u_{n}=\P(U_{n})$.
	Montrer que $\lim\limits_{n\to+\infty}u_{n}=\min(1,\frac{p}{q})$. Interpréter.
	Dans le cas $p=\frac{1}{2}$, chercher un développement asymptotique en
	$o\bigl(\frac{1}{n}\bigr)$ de $u_{n}$.
\end{exercise}

\begin{exercise}
	Une puce se déplace sur une droite. Elle part de 0 et fait des sauts
	successifs de longueur 1. A chaque saut, elle avec avec la probabilité
	$p\in]0,1[$ et recule avec la probabilité $q=1-p$. Quelle est la probabilité
	pour que la puce repasse en 0 ? Montrer que si $p\neq\frac{1}{2}$, le nombre
	de retours à l'origine est presque sûrement fini.
\end{exercise}

\begin{exercise}
	Soit un lancer infini d'une pièce donnant pile avec la probabilité $p\in]0,1[$
	et face avec la probabilité $q=1-p$. On désigne, pour $n\in\N^{*}$, $A_{n}$
	l'évènement qu'après le $n$-ième lancer, on a obtenu pour la première fois
	deux piles consécutifs (donc au $n-1$-ième et au $n$-ième). On note
	$a_{n}=\P(A_{n})$.
	
	\begin{enumerate}
		\item
		Calculer $a_{1},a_{2},a_{3}$.
		\item
		Calculer $a_{n}$ et $\sum_{n=1}^{+\infty}a_{n}$. Interpréter.
	\end{enumerate}
\end{exercise}

\begin{exercise}
	On dispose de $N+1$ urnes numérotées de $0$ à $N$. Pour tout
	$k\in\{0,\dots,N\}$, la $k$-ième urne possède $k$ boules blanches et $N-k$
	boules noires. On choisit une urne au hasard et on tire $n$ fois une boule
	avec remise.

	Quelle est la probabilité qu'au $n+1$-ième tirage, on ait obtenu une
	blanche sachant qu'au cours des $n$ premiers lancers on a obtenu des
	blanches? On note cette probabilité $\P_{N}(n)$.
	Que vaut $\lim\limits_{N\to+\infty}\P_{N}(n)$ à $n$ fixé ?
\end{exercise}

\begin{exercise}
	Quelle est la probabilité pour que deux entiers naturels soient premiers entre
	eux ?
\end{exercise}

\begin{exercise}
	$A$ écrit à $B$ avec une probabilité $p_{1}$ s'il lui a écrit la veille,
	$p_{2}$ sinon (avec $(p_{1},p_{2})\in[0,1]^{2}$). Soit, pour $n\in\N^{*}$,
	$X_{n}$ qui vaut 1 si $A$ a écrit à $B$ le jour $n$ et 0 sinon. Déterminer la
	loi et l'espérance (sous réserve d'existence) de $X_{n}$.
\end{exercise}

\begin{exercise}
	On lance deux dés non pipés. On note $D_{i}$ le résultat du $i$-ième dé,
	$X=\max(D_{1},D_{2})$ et $Y=\min(D_{1},D_{2})$.
	\begin{enumerate}
		\item
		Déterminer les lois de $X$ et $Y$ ainsi que leurs espérances et variances.
		\item
		$X$ et $Y$ sont-elles indépendantes?
		\item
		Et si $\P(D_{i}=k)=p_{k,i}$ avec $\sum_{k=1}^{6}p_{k,i}=1$?
	\end{enumerate}
\end{exercise}

\begin{exercise}
	Soit $(a,b)\in]0,1[\times\R_{+}^{*}$, $X$ et $Y$ deux variables aléatoires
	discrètes à valeurs dans $\N$ dont la loi conjointe est
	$$
	p_{i,j}=
	\left\{
		\begin{array}{cc}
			0 & \text{si }i<j\\
			\frac{b^{i}e^{-b}a^{j}(1-a)^{i-j}}{j!(i-j)!} & \text{si }i\geqslant j
		\end{array}
	\right.
	$$
	\begin{enumerate}
		\item
		Vérifier que cette définition est cohérente.
		\item
		Déterminer les lois marginales. $X$ et $Y$ sont-elles indépendantes?
		\item
		Déterminer la loi de $Z=X-Y$. $Y$ et $Z$ sont-elles indépendantes?
	\end{enumerate}
\end{exercise}

\begin{exercise}
	Soit $x\in]0,1[$. Soit une succession d'épreuves de Bernoulli, indépendantes,
	de probabilité d'échec $x$. On définit deux suites de variables aléatoires:
	pour tout $n\in\N^{*}$, $S_{n}$ est la nombre d'épreuves nécessaires pour
	obtenir le $n$-ième succès, et $T_{1}=S_{1}$ et pour $n\geqslant2$, $T_{n}$
	est le nombre d'épreuves séparant le $n$-ième succès du $n-1$-ième.
	\begin{enumerate}
		\item
		Exprimer, pour tout $n\geqslant1$, $S_{n}$ en fonction des
		$(T_{i})_{i\geqslant1}$.
		\item
		Déterminer la loi, l'espérance et la variance de $T_{n}$.
		\item
		Déterminer la loi, l'espérance et la variance de $S_{n}$.
		\item
		Montrer que pour tout $x\in]0,1[$, pour tout $n\in\N^{*}$,
		$$\sum_{k=n}^{+\infty}\binom{k-1}{n-1}x^{k}=\frac{x^{n}}{(1-x)^{n}}$$
	\end{enumerate}
\end{exercise}

\begin{exercise}
Soit $X$ une variable aléatoire à valeur dans $\N$. Montrer que $X$ possède une
espérance si et seulement si $\sum_{n\in\N}\P(X>n)$ converge, et qu'on a alors
$$\E(X)=\sum_{n=0}^{+\infty}\P(X>n)$$
\end{exercise}

\begin{exercise}
	Soit $X\sim\mathcal{P}(\lambda)$ (loi de Poisson) avec $\lambda>0$. Quelle est
	la valeur que prend $X$ avec la plus grande probabilité? Évaluer la limite de
	cette valeur quand $\lambda\to+\infty$.
\end{exercise}

\begin{exercise}
	Un veilleur de nuit doit ouvrir une porte, ils possède un trousseau de 10 clés
	indiscernables dont une seule ouvre la porte. S'il est sobre, après chaque
	échec, il met de côté la mauvaise clé et poursuit avec les autres. S'il est
	ivre, il remet la clé dans le trousseau après chaque échec. Soit $X$
	(respectivement $Y$) la nombre d'essais au bout desquels il ouvre la porte
	s'il est sobre (respectivement ivre).
	\begin{enumerate}
		\item
		Donner les lois de $X$ et $Y$, leurs espérances et variances (si
		définies).
		\item
		Le gardien est ivre un jour sur 3. Sachant qu'un jour il a essayé au moins
		9 clés, quelle est la probabilité qu'il ait été sobre ce jour-là?
	\end{enumerate}
\end{exercise}

\begin{exercise}
	Soit $(n,N)\in(\N^{*})^{2}$. Une urne contient $N$ jetons à deux faces. L'une
	porte un numéro bleu, l'autre un rouge. Pour tout $1\leqslant j\leqslant
	i\leqslant n$, un seul jeton porte $i$ bleu et $j$ rouge. On tire au hasard un
	jeton. On note $B$ (respectivement $R$) le numéro bleu (respectivement rouge)
	tiré et $G=B-R$.
	\begin{enumerate}
		\item
		Déterminer $N$ en fonction de $n$.
		\item
		Donner les lois conjointes et marginales de $B$ et $R$.
		\item
		Déterminer les espérances et variances de $B$, $R$ et $G$.
	\end{enumerate}
\end{exercise}

\begin{exercise}
	Un mobile se déplace sur les points à coordonnées entières (naturelles) d'un
	axe d'origine 0. Au départ, le mobile est en 0. S'il est au point d'abscisse
	$k$ à l'instant $n$, à l'instant $n+1$ il est au point $k+1$ avec une
	probabilité $\frac{k+1}{k+2}$ et 0 avec la probabilité $\frac{1}{k+2}$. On
	note $X_{n}$ l'abscisse du mobile à l'instant $n$ et $u_{n}=\P(X_{n}=0)$.
	\begin{enumerate}
		\item
		Montrer que pour tout $n\in\N$, pour tout $k\in\{1,\dots,n+1\}$,
		$\P(X_{n+1}=k)=\frac{k}{k+1}\P(X_{n}=k-1)$.
		\item
		En déduire que pour tout $n\in\N$, pour tout $k\in\{0,\dots,n\}$,
		$\P(x_{n}=k)=\frac{1}{k+1}u_{n-k}$.
		\item
		Montrer que pour tout $n\in\N$, $\sum_{j=0}^{n}\frac{u_{n}}{n-j+1}=1$.
		Calculer $u_{0},u_{1},u_{2},u_{3}$.
		\item
		Prouver que pour tout $n\in\N$, $\E(X_{n+1})=\E(X_{n})+u_{n+1}$. En
		déduire $\E(X_{n})$ en fonction des $(u_{n})_{n\in\N}$.
		\item
		On note $T$ l'instant auquel le mobile revient pour la première fois à
		l'origine ($T=0$ s'il ne repasse pas par l'origine). Montrer que pour tout
		$n\in\N^{*}$, $\P(T=n)=\frac{1}{n(n+1)}$. En déduire $\P(T=0)$.
		\item
		$T$ admet-elle une espérance?
	\end{enumerate}
\end{exercise}

\begin{exercise}
	Le nombre $N$ de clients arrivant dans un magasin au cours d'une journée suit
	une loi de Poisson de paramètre $\lambda$. Ces clients se répartissent de
	manière équiprobable entre les $m$ caisses du magasin ($m\in\N^{*}$). Soit
	$X_{1}$ le nombre de clients qui arrivent à la caisse 1 au cours d'une
	journée. 
	\begin{enumerate}
		\item
		Soit $n\in\N^{*}$, déterminer la loi conditionnelle de $X_{1}$ sachant
		$N=n$.
		\item
		En déduire la loi de $X_{1}$.
	\end{enumerate}
\end{exercise}

\begin{exercise}
	Soit $X,Y$ deux variables aléatoires indépendantes discrètes suivant la même
	loi de Bernoulli de paramètre $p\in]0,1[$. On pose $U=X+Y$ et $V=X-Y$.
	\begin{enumerate}
		\item
		Quelle est la conjointe de $(U,V)$?
		\item
		Déterminer la covariance de $X,Y$ ?
		\item
		$U$ et $V$ sont-elles indépendantes?
	\end{enumerate}
\end{exercise}

\begin{exercise}
	On tire indéfiniment à pile ou face, avec une probabilité $p\in]0,1[$
	d'obtenir pile, $q=1-p$ d'obtenir face. On note $P$ (respectivement $F$) le
	rang d'apparition du premier pile (respectivement du premier face) et on note
	$X$ (respectivement $Y$) la longueur de la première (respectivement deuxième)
	suite de tirages égaux (ex: (pile, pile, face, pile,...) donne $X=2$ et
	$Y=1$).
	\begin{enumerate}
		\item
		Donner les lois de $P$ et $F$ et leurs espérances et variances.
		\item
		$P$ et $F$ sont-elles indépendantes?
		\item
		Donner les lois conjointes et marginales de $X$ et $Y$. Sont-elles
		indépendantes?
		\item
		Montrer que $\E(X)\geqslant2$.
		\item
		Quelle est la probabilité que $X=Y$ ?
		\item
		Donner la loi de $X+Y$ lorsque $p=\frac{1}{2}$.
	\end{enumerate}
\end{exercise}

\begin{exercise}
	Soit $(X_{n})_{n\geqslant1}$ une suite de variables aléatoires discrètes de même loi, centrées à valeurs dans $[-1,1]$.
	\begin{enumerate}
		\item
		Montrer que pour tout $\lambda\geqslant0,\forall x\in[-1,1],~\e^{\lambda
		x}\leqslant \e^{\frac{\lambda^{2}}{2}}+x\sinh(\lambda)$.
		\item
		En déduire que si $X$ est une variable aléatoire discrète centrée à
		valeurs dans $[-1,1]$, alors pour tout $\lambda\geqslant0$, $\E(\e^{\lambda
		X})\leqslant \e^{\frac{\lambda^{2}}{2}}$ et $\E(\e^{-\lambda X})\leqslant
		\e^{\frac{\lambda^{2}}{2}}$.
		\item
		Montrer que pour tout $a\in\R$, pour tout $\lambda>0$, $\P(X\geqslant
		a)\leqslant \e^{-\lambda a}\E(\e^{\lambda X})$.
		\item
		Montrer que pour tout $n\geqslant1$, pour tout $a\geqslant0$,
		$\P\Bigl(\bigl\vert\frac{1}{n}\sum_{i=1}^{n}X_{i}\bigr\vert\geqslant
		a\Bigr)\leqslant2\e^{-\frac{a^{2}}{2}}$.
	\end{enumerate}
\end{exercise}

\begin{exercise}
	\phantom{}
	\begin{enumerate}
		\item
		Soient $X$ et $Y$ deux variables aléatoires discrètes à valeurs dans $\N$
		d'espérances finies. Pour $k\in\N$, on définit
		$$\E_{(X=k)}(Y)=\sum_{l=0}^{+\infty}l\P_{(X=k)}(Y=l)$$ Montrer que
		$$\E(Y)=\sum_{k=0}^{+\infty}\E_{(X=k)}(Y)\times\P(X=k)$$
		\item
		Soit $\lambda\geqslant0$, on suppose que le nombre de descendants poules à
		la première génération d'une poule donnée suit une loi de Poisson de
		paramètre $\lambda$. Soit $X_{n}$ le nombre de poules à la $n$-ième
		génération avec $X_{0}=N\in\N^{*}$. Déterminer $\E(X_{n})$.
	\end{enumerate}
\end{exercise}

\begin{exercise}
	Au cours de sa vie, une poule pond $N$ \oe ufs où $N\sim\mathcal{P}(\lambda)$.
	Chaque \oe uf éclot avec une probabilité $p\in]0,1[$. On note $K$ le nombre de
	poussins. Donner la loi et l'espérance de $K$.
\end{exercise}

\begin{exercise}
	Soit $(A_{n})_{n\geqslant1}$ une suite d'évènements indépendants de $\Omega$
	(où $(\Omega,\mathcal{A},\P)$ est un espace probabilisé) telle que pour tout
	$n\geqslant1$, on a $\P(A_{n})=\frac{1}{n}$. Soit
	$S_{n}=\sum_{k=1}^{n}\chi_{A_{k}}$ (indicatrice de l'ensemble).
	\begin{enumerate}
		\item
		Évaluer l'espérance et variance de $S_{n}$, et en donner des équivalents.
		\item
		Soit $\varepsilon>0$ fixé, montrer que 
		$$\lim\limits_{n\to+\infty}\P\Biggl(\Bigl\vert\frac{S_{n}}{\ln(n)}-1\Bigr\vert\geqslant\varepsilon\Biggr)=0$$
	\end{enumerate}
\end{exercise}

\begin{exercise}
	Soit $n\geqslant1$, $(X_{1},\dots,X_{n})$ des variables aléatoires
	indépendantes discrètes telles que $X_{i}\sim\mathcal{G}(p)$ (loi géométrique)
	avec $p\in]0,1[$. On note $q=1-p$ et $U=\min_{1\leqslant i\leqslant n}X_{i}$
	et $V=\max_{1\leqslant i\leqslant n}X_{i}$.
	\begin{enumerate}
		\item
		Donner la loi de $U$ et son espérance (dont on justifiera l'existence).
		\item
		Donner la loi de $V$, montrer que
		$$\E(V)=\sum_{i=1}^{n}\binom{n}{i}(-1)^{i+1}\frac{1}{1-q^{i}}$$
	\end{enumerate}
\end{exercise}

\begin{exercise}
	Un joueur arrive au casino avec une fortune $k\in\{0,\dots,N\}$ où $N$ est la
	"banque", c'est-à-dire la limite que paiera le casino. A chaque étape, il
	gagne $1$ avec une probabilité $p\in]0,1[$ et perd $1$ avec une probabilité
	$q=1-p$. Il s'arrête lorsqu'il a 0 (ruine) ou $N$ (banque). On sait que
	l'arrêt en temps fini est presque sûr. On note $t_{k}$ le temps d'arrêt du
	joueur (le nombre de fois où il joue avant de s'arrêter).
	\begin{enumerate}
		\item
		Montrer que pour tout $n\in\N$,
		$$\P(t_{k}>N(n+1))\leqslant\P(t_{k}>Nn)\times(1-p^{N})$$ En déduire que
		$t_{k}$ possède une espérance notée $T_{k}$.
		\item
		Montrer que pour tout $k\in\{1,\dots,N-1\}$,
		$T_{k}=p(1+T_{k+1})+q(1+T_{k-1})$.
		\item
		En déduire que 
		$$T_{k}= \left\{
			\begin{array}[]{cc}
				\dfrac{1}{q-p}\Biggl[k-N\Biggl(\dfrac{1-\Bigl(\dfrac{q}{p}\Bigr)^{k}}{1-\Bigl(\dfrac{q}{p}\Bigr)^{n}}\Biggr)\Biggr]
& \text{si }p\neq\dfrac{1}{2}\\
				k(N-k) & \text{si }p=\dfrac{1}{2}
			\end{array}
		\right.$$
	\end{enumerate}
\end{exercise}

\begin{exercise}
	Soit $s>1$. On munit $\N^{*}$ de la probabilité:
	$$\forall n\geqslant1,~\P(\{n\})=\frac{1}{\zeta(s)n^{s}}$$
	\begin{enumerate}
		\item
		Vérifier que c'est une probabilité sur $\N^{*}$. On forme, pour
		$n\geqslant 1$, $A_{n}=n\N^{*}$. Calculer $\P(A_{n})$.
		\item
		Montrer que si $\mathcal{P}$ est l'ensemble des nombres premiers, alors
		$(A_{p})_{p\in\mathcal{P}}$ sont indépendants.
		\item
		En déduire que 
		$$\P(\{1\})=\prod_{p\in\mathcal{P}}\Bigl(1-\frac{1}{p^{s}}\Bigr)$$ puis
		que 
		$$\zeta(s)=\Biggl(\prod_{p\in\mathcal{P}}\Bigl(1-\frac{1}{p^{s}}\Bigr)\Biggr)^{-1}$$
	\end{enumerate}
\end{exercise}

\begin{exercise}
	On dispose de deux pièces de monnaie. La pièce $A$ (respectivement $B$) amène
	pile avec la probabilité $a\in]0,1[$ (respectivement $b\in]0,1[$). On choisit
	au départ de façon équiprobable une des deux pièces et on effectue le premier
	lancer avec celle là. Si on obtient pile, on garde la même, sinon on change.
	Soit, pour $n\geqslant1$, $E_{n}$ l'évènement désignant le fait que l'on
	utilise pour la première fois la pièce $A$ au $n$-ième lancer. On indique de
	même par $U_{n}$ l'évènement désignant le fait que l'on obtient $n$ piles au
	cours des $n$ premiers lancers.

	Calculer $\P(E_{n}),\P(U_{n}),\P(\cup_{n\in\N}E_{n}),\P(\cap_{n\in\N}U_{n})$.
\end{exercise}

\begin{exercise}
	Soit $N$ pièces de monnaie, dont chacune amène pile avec la probabilité
	$p\in]0,1[$. A chaque lancer, on laisse de côté les pièces tombées sur pile,
	et on relance les autres jusqu'à que toutes arrivent sur pile.
	\begin{enumerate}
		\item
		Pour $i\in\{1,\dots,N\}$ et $n\in\N^{*}$. Calculer la probabilité de
		l'évènement $A_{i,n}$ désignant le fait que la pièce numéro $i$ est lancée
		au plus $n$ fois.
		\item
		Soit $B_{n}$ l'évènement indiquant que l'on effectue au plus $n$ relances.
		Calculer $\P(B_{n})$ et $\lim\limits_{n\to+\infty}\P(B_{n})$.
		\item
		Pour $n\geqslant1$, quelle est la probabilité d'effectuer exactement $n$
		relances?
	\end{enumerate}
\end{exercise}

\begin{exercise}
	Soit $p$ premier et $\K=\Z/p\Z$ (en tant que corps). Soit $n\geqslant1$. On
	définit $\K_{<n}[X]=\{Q\in\K[X]\mid \deg(Q)<n\}$ et
	$\K_{=n}[X]=\{P\in\K[X]\mid \deg(P)=n\}$. Ces espaces sont munis de la
	probabilité uniforme. On forme $\Omega=\K_{<n}[X]\times\K_{=n}[X]$ muni de la
	probabilité uniforme.
	\begin{enumerate}
		\item
		Quelle est la loi de $\deg(Q)$?
		\item
		Quelle est la probabilité pour que $Q\mid P$ ? On pourra poser $A=\left\lbrace(Q,P)\in\Omega\middle| Q\mid P\right\rbrace$ et $B=\left\lbrace(Q,A)\in\K_{<n}[X]\times\K_{=n}[X]\middle|\deg(Q)+\deg(A)=n\right\rbrace$ et montrer que \function{f}{A}{B}{(Q,P)}{(Q,\frac{P}{Q})} est définie et bijective.
		\item
		Soit $R\in\K_{<n-1}[X]$, on pose pour $(Q,P)\in\Omega$ avec $Q\neq0$,
		$R_{1}$ le reste de la division euclidienne de $P$ par $Q$. Calculer
		$\P(R_{1}=R)$ et $\P_{\deg(Q)=k}(R_{1}=R)$.
	\end{enumerate}
\end{exercise}

\begin{exercise}
	On effectue deux tirages sans remise dans une urne contenant $n$ jetons
	numérotés de $1$ à $n$. On note $X_{1}$ le premier tirage et $X_{2}$ le
	deuxième.
	\begin{enumerate}
		\item
		Donner la loi conjointe de $(X_{1},X_{2})$.
		\item
		Donner la loi de $X_{2}$ puis la loi de $X_{{1}_{\mid X_{2}=j}}$ pour
		$j\in\{1,\dots,n\}$.
		\item
		$X_{1}$ et $X_{2}$ sont-elles indépendantes?
		\item
		Généraliser à $k$ tirages sans remise pour $1\leqslant k\leqslant n$.
		\item Quelle est l'espérance de $(X_{1},X_{2})$ ?
	\end{enumerate}
\end{exercise}

\end{document}