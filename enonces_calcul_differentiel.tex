\documentclass[12pt]{article}
\usepackage{style/style}

\begin{document}

\begin{titlepage}
	\centering
	\vspace*{\fill}
	\Huge \textit{\textbf{Exercices MP/MP$^*$\\ Calcul différentiel}}
	\vspace*{\fill}
\end{titlepage}

\begin{exercise}
	Étudier la continuité, la différentiabilité et la classe de 
	\function{f}{\R^{3}}{\R}{(x,y,z)}{
		\left\lbrace
			\begin{array}[]{rcl}
				\frac{x^{3}+y^{3}-xy^{2}+yz^{2}+xyz}{x^{2}+y^{2}+z^{2}} &\text{si }(x,y,z)\neq(0,0,0),\\
				0 &\text{sinon.}
			\end{array}
		\right.
	}
\end{exercise}

\begin{exercise}
	Soit $U$ ouvert de $\R^{n}$, $(\varphi_1,\dots,\varphi_{k})\in\left(\mathcal{C}^{0}(U,\R)\right)^{k}$.
	\begin{enumerate}
		\item Montrer que $\psi=\min\limits_{1\leqslant i\leqslant k}(\varphi_i)$ est continue.
		\item Soit $x_0\in U$, si les $(\varphi_i)_{1\leqslant i\leqslant k}$ sont différentiables en $x_0$, donner une condition nécessaire et suffisante pour que $\psi$ le soit. On pourra former 
		\begin{equation}
			J=\left\lbrace i\in\left\llbracket1,k\right\rrbracket,\psi(x_0)=\varphi_i(x_0)\right\rbrace.
		\end{equation}
		\item Si les $(\varphi_i)_{1\leqslant i\leqslant k}$ sont $\mathcal{C}^{1}$ sur $U$ et $\psi$ est différentiable, montrer que $\psi$ est $\mathcal{C}^{1}$ sur $U$.
	\end{enumerate}
\end{exercise}

\begin{exercise}
	On définit 
	\function{f}{\R^{n}}{\R}{
		X=\begin{pmatrix}
			x_1\\\vdots\\x_n
		\end{pmatrix}
	}{
		f(X)=\sum_{(i,j)\in\left\llbracket1,n\right\rrbracket^{2}}\frac{x_i x_j}{i+j+1}
	}
	Soit $H_0=\left\lbrace(x_1,\dots,x_n)\in\R^{n}\middle|\sum_{i=1}^{n}x_i=1\right\rbrace$. Déterminer les extrema de $f$ sur $H_0$.
\end{exercise}

\begin{exercise}
	Étudier la continuité, différentiabilité, classe de \function{f}{\R^{2}}{\R}{(x,y)}{
		\left\lbrace
			\begin{array}[]{ll}
				x^{2}\sin\left(\frac{y}{x}\right),&\text{si }x\neq0,\\
				0 &\text{sinon}.
			\end{array}
		\right.
	}
\end{exercise}

\begin{exercise}
	Soit $n\geqslant 2$, en quels points de $\R^{n}$, $\left\lVert\cdot\right\rVert_{1}$ et $\left\lVert\cdot\right\rVert_{\infty}$ sont-elles différentiables ?
\end{exercise}

\begin{exercise}
	Soit $n\geqslant3$. Trouver 
	\begin{equation}
		\sup\left\lbrace\frac{\prod_{i=1}^{n}x_i}{\prod_{i=1}^{n}\left(1-x_i\right)}\middle| (x_1,\dots,x_n)\in(\R_{+})^{n},\sum_{i=1}^{n}x_i=1\right\rbrace.
	\end{equation}
\end{exercise}

\begin{exercise}
	Trouver les fonctions $f\colon\R^{2}\to\R$ de classe $\mathcal{C}^{1}$ telle que pour tout $(x,y)\in\R^{2}$
	\begin{equation}
		x\frac{\partial f}{\partial x}(x,y)+y\frac{\partial f}{\partial y}(x,y)+x^{2}+y^{2}=0.
	\end{equation}
\end{exercise}

\begin{exercise}
	Soit \function{f}{\mathcal{M}_n(\R)}{\R^n}{M}{(\Tr(M),\Tr(M^{2}),\dots,\Tr(M^{n}))}
	\begin{enumerate}
		\item Montrer que $f$ est $\mathcal{C}^{\infty}$, calculer sa différentielle.
		\item Quel est le rang de $df_{M}$ ? On l'exprimera en fonction du degré du polynôme minimal de $M$, $\Pi_{M}$.
		\item Montrer que $C=\left\lbrace M\in\mathcal{M}_n(\R)\middle|\Pi_{M}=\chi_{M}\right\rbrace$ est un ouvert de $\mathcal{M}_n(\R)$.
	\end{enumerate}
\end{exercise}

\begin{exercise}
	Trouver toutes les applications $f\colon\R^{\star}\times\R\to\R$ de classe $\mathcal{C}^{1}$ telles que pour tout $(x,y)\in\R^{*}\times\R$, 
	\begin{equation}
		x\frac{\partial f}{\partial x}(x,y)+y\frac{\partial f}{\partial y}(x,y)=\frac{1}{x^{2}}.
	\end{equation}
\end{exercise}

\begin{exercise}
	Soit $U$ un ouvert convexe de $\R^{n}$ et $f\colon U\to\R$ convexe et $\mathcal{C}^{1}$.
	\begin{enumerate}
		\item Montrer que pour $(x,y)\in U^{2}$, $f(y)\geqslant f(x)+df_{x}(y-x)$.
		\item Montrer que tout point critique de $f$ est un minimum absolu.
		\item Montrer que l'ensemble $E$ des points critiques de $f$ est convexe.
		\item On suppose $U=\R^{n}$, montrer que $E$ est fermé.
	\end{enumerate}
\end{exercise}

\begin{exercise}
	Soit $\alpha\in\R$, on dit que $f\colon\R^{n}\to\R$ est homogène de degré $\alpha$ ou $\alpha$-homogène si et seulement si pour tout $(x,t)\in\R^{n}\times\R_{+}^{*}$, $f(tx)=t^{\alpha}f(x)$.

	Soit $f\colon\R^{n}\to\R$ $\mathcal{C}^{1}$, montrer que $f$ est $\alpha$-homogène si et seulement si pour tout $x=(x_1,\dots,x_n)\in\R^{n}$, $\sum_{i=1}^{n}x_i\frac{\partial f}{\partial x_i}(x_1,\dots,x_n)=\alpha f(x)$.
\end{exercise}

\begin{exercise}
	Étudier les extrema de \function{f}{\R^{3}}{\R}{(x,y,z)}{x^{2}+y^{2}+z^{2}-xyz}.
\end{exercise}

\begin{exercise}
	Soit $f\colon\R\to\R$ dérivable. On définit \function{g}{\R^{2}}{\R}{(x,y)}{
		\left\lbrace
			\begin{array}[]{ll}
				\frac{f(x)-f(y)}{x-y},&\text{si }x\neq y,\\
				f'(x), &\text{sinon.}
			\end{array}
		\right.
	}
	\begin{enumerate}
		\item Montrer que $f$ est $\mathcal{C}^{1}$ si et seulement si $g$ est $\mathcal{C}^{0}$. On pourra écrire $g(x,y)=\int_{0}^{1}\dots \d t$.
		\item Montrer que $f$ est $\mathcal{C}^{2}$ si et seulement si $g$ est $\mathcal{C}^{1}$. 
	\end{enumerate}
\end{exercise}

\begin{exercise}[Dérivation au sens complexe]
	Soit \function{f}{U\subset\C}{\C}{z}{f(z)}
	On lui associe \function{\widetilde{f}}{E\subset\R^{2}}{\R^{2}}{(x,y)}{
		(\underbrace{\Re(f(x+\i y))}_{\widetilde{f}_1(x,y)},\underbrace{\Im(f(x+\i y))}_{\widetilde{f}_2(x,y)})
	}

	On a $f(x+\i y)=\widetilde{f}_1(x,y)+\i\widetilde{f}_2(x,y)$.
	À quelles conditions nécessaires et suffisantes sur $\widetilde{f}$, la fonction est-elle dérivable au sens complexe sur $U$, c'est-à-dire que pour tout $z_0\in U$, il existe $f'(z_0)=\lim\limits_{\substack{h\to\\h\in\C^{*}}}\frac{f(z_0+h)-f(z_0)}{h}$, respectivement $\mathcal{C}^{1}$ au sens complexe (c'est-à-dire dérivable sur $U$ et $f'$ continue) ?
\end{exercise}

\begin{exercise}[Fonctions harmoniques]
	On définit, pour $f\colon U\subset\R^{2}\to\R$ de classe $\mathcal{C}^{2}$,
	\begin{equation}
		\Delta f(x,y)=\frac{\partial^{2}f}{\partial x^{2}}(x,y)+\frac{\partial^{2}f}{\partial y^{2}}(x,y).
	\end{equation}
	On dit que $f$ est harmonique sur $U$ si et seulement si $\Delta f=0$ sur $U$.

	Soit $U$ un ouvert borné, et $f\colon\overline{U}\to\R$ continue sur $\overline{U}$ et harmonique sur $U$. On veut montrer que $\max\limits_{\overline{U}}f$ est atteint sur $\partial U$.

	On suppose que $\max\limits_{\overline{U}}f$ est atteint sur $(x_0,y_0)\in U$.
	\begin{enumerate}
		\item On définit, pour tout $n\geqslant1$, \function{f_n}{\overline{U}}{\R}{(x,y)}{f(x,y)+\frac{1}{n}(x^{2}+y^{2})}
		Montrer que $\Delta f_n(x,y)>0$, en déduire que $\sup\limits_{\overline{U}}f_n$ est atteint sur $\partial U$.

		\item Montrer le résultat.
		\item En déduire que si $f,g\colon U\to\R$ sont continues sur $\overline{U}$ et harmoniques sur $U$ et vérifient $f=g$ sur $U$, alors $f=g$ sur $\overline{U}$.
	\end{enumerate}
\end{exercise}

\begin{exercise}[Laplacien en polaire]
	Soit \function{f}{U\subset\R^{2}}{\R}{(x,y)}{f(x,y)} de classe $\mathcal{C}^{2}$. On lui associe \function{\widetilde{f}}{U'\subset\R^{2}}{\R}{(r,\theta)}{\widetilde{f}(r,\theta)=f(\underbrace{r\cos\theta}_{x(r,\theta)},\underbrace{r\sin\theta}_{y(r,\theta)})}
	$\widetilde{f}$ est $\mathcal{C}^{2}$ par composition. Exprimer la laplacien en polaire $\Delta f$ en fonction des dérivées partielles de $\widetilde{f}$.
\end{exercise}

\begin{exercise}[Égalité de la moyenne]
	Soit $f$ harmonique sur $U$ continue sur $\overline{U}$. Soit $(x_0,y_0)\in U$. On veut montrer que pour tout $r\in[0,d((x_{0},y_{0}),\partial U)]$,
	\begin{equation}
		f(x_{0},y_{0})=\frac{1}{2\pi}\int_{0}^{2\pi}f(x_0+r\cos\theta,y_0+r\sin\theta)\d\theta=G(r).
	\end{equation}

	La fonction $f_1(x_0,y_0)=f(x_0+x,y_0+y)$ est harmonique. On lui associe $\widetilde{f}_1(r,\theta)=f(x_0+r\cos\theta,y_0+r\sin\theta)$ fonction de $\theta$ $2\pi$-périodique.
	\begin{enumerate}
		\item Pour $r>0$, calculer $\frac{1}{r}\frac{\d}{\d r}\left(r\frac{\d G}{\d r}\right)$.
		\item En déduire le résultat.
	\end{enumerate}
\end{exercise}

\begin{exercise}
	\phantom{}
	\begin{enumerate}
		\item Déterminer l'ensemble des vecteurs tangents à $O_n(\R)$ en $I_n$.
		\item De même en $\theta\in O_n(\R)$.
		\item On définit sur $\mathcal{M}_n(\R)$, $(A|B)=\Tr(A^{\mathsf{T}}B)$ et $\left\lVert A\right\rVert_{2}=(A|A)$. Évaluer, pour $M\in\mathcal{M}_n(\R)$, $d(M,O_n(\R))=\inf\left\lbrace\left\lVert M-\theta\right\rVert_{2}\middle| \theta\in O_n(\R)\right\rbrace$.
	\end{enumerate}
\end{exercise}

\end{document}