\documentclass[12pt]{article}
\usepackage{style/style}

\begin{document}

\begin{titlepage}
	\centering
	\vspace*{\fill}
	\Huge \textit{\textbf{Exercices MP/MP$^*$\\ Séries Entières}}
	\vspace*{\fill}
\end{titlepage}

\begin{exercise}
    Donner le rayon de convergence de 
    \begin{enumerate}
        \item $\sum_{n\geqslant1}\left(\cosh\left(\frac{1}{n}\right)\right)^{n^{\alpha}}z^{n}$,
        \item $\sum_{n\geqslant1}\left(1+\frac{(-1)^{n}}{n^{2}}\right)^{n^{3}}$.
    \end{enumerate}
\end{exercise}

\begin{exercise}
    \phantom{}
    \begin{enumerate}
        \item Soit $(\theta_{1},\dots,\theta_{p})\in[0,2\pi[^{p}$ des réels distincts, $(m_{1},\dots,m_{p})\in(\N^{*})^{p}$. Montrer que 
        \begin{equation}
            \left(u_n=\sum_{k=1}^{p}m_{k}\e^{\i n\theta_{k}}\right)_{n\in\N}
        \end{equation}
        ne tend pas vers 0.

        \item Soit $A\in\M_{p}(\C)$ et $a_n=\Tr(A^{n})$. Donner le rayon de convergence et la somme de $\sum a_{n}z^{n}$.
    \end{enumerate}
\end{exercise}

\begin{exercise}
    Donner le rayon de convergence et calculer la somme (en cas de convergence) de 
    \begin{equation}
        \sum_{n\geqslant1}\frac{z^{n}}{\sum_{k=1}^{n}k^{2}}=\sum_{n=\geqslant1}\frac{6z^{n}}{n(n+1)(2n+1)}.
    \end{equation}
\end{exercise}

\begin{exercise}
    On définit \function{f}{]-1,+\infty[}{\R}{t}{
        \left\lbrace
            \begin{array}[]{ll}
                \frac{t}{\ln(1+t)} & \text{si }t\neq0,\\
                1 & \text{si }t=0.
            \end{array}
        \right.
    }

    Montrer que $f$ est développable en série entière sur $]-1,1[$, en déduire que $f$ est $\mathcal{C}^{\infty}$. On pourra former $\int_{0}^{1}(1+t)^{u}\d u=I(t)$.
\end{exercise}

\begin{exercise}
    Donner le rayon de convergence de $\sum_{n\geqslant1}a_{n}z^{n}$ où 
    \begin{equation}
        a_n=\left(\sum_{k=1}^{n}\frac{1}{k}\right)^{\ln(n)}.
    \end{equation}
\end{exercise}

\begin{exercise}
    Donner le rayon de convergence de $\sum a_{n}z^{n}$ où $a_n$ est le nombre de diviseurs $n$.
\end{exercise}

\begin{exercise}
    Soit $(a_n)_{n\in\N}\in\left(\R_{+}^{*}\right)^{\N}$ telle que 
    \begin{equation}
        \lim\limits_{n\to+\infty}\frac{a_{n-1}a_{n+1}}{a_{n}^{2}}=l\in\R.
    \end{equation}
    Déterminer le rayon de convergence de $\sum a_{n}z^{n}$.
\end{exercise}

\begin{exercise}
    Soit $z\in\C$ tel que $\left\lvert z\right\rvert<1$. On pose $\phi(z)=\sum_{n=1}^{+\infty}(-1)^{n-1}\frac{z^{n}}{n}$. Déterminer $\e^{\phi(z)}$.
\end{exercise}

\begin{exercise}
    Donner le rayon de convergence et calculer la somme (sur le disque ouvert de convergence) de 
    \begin{equation}
        \sum_{n=0}^{+\infty}\frac{z^{n}}{\cos\left(\frac{2n\pi}{3}\right)}.
    \end{equation}
\end{exercise}

\begin{exercise}
    Soit $(a_n)_{n\in\N}$ et $(b_n)_{n\in\N}$ des suites réelles, on suppose
    \begin{enumerate}[label=\roman*)]
        \item pour tout $n\in\N$, $b_n\geqslant0$,
        \item $a_n\underset{+\infty}{\sim}b_n$,
        \item le rayon de convergence de $\sum b_{n}z^{n}$ vaut 1,
        \item $\sum b_{n}$ diverge.
    \end{enumerate}

    On forme sur $[0,1[$, $f(x)=\sum_{n=0}^{+\infty}a_{n}x^{n}$ et $g(x)=\sum_{n=0}^{+\infty}b_{n}x^{n}$.

    \begin{enumerate}
        \item Montrer que $\lim\limits_{x\to1^{-}}g(x)=+\infty$.
        \item Montrer que $f(x)=\underset{x\to1^{-}}{\sim}g(x)$.
        \item Donner un équivalent simple quand $x\to1^{-}$ de $h_p(x)=\sum_{n=1}^{+\infty}n^{p}x^{n}$ avec $p\in\N$.
    \end{enumerate}
\end{exercise}

\begin{exercise}
    Soit $f(x)=\sum_{n=0}^{+\infty}a_nx^{n}$ de rayon de convergence 1. On suppose que $\lim\limits_{x\to1^{-}}=S\in\R$ et que $a_n=\underset{+\infty}{o}\left(\frac{1}{n}\right)$. Montrer que $\sum a_{n}$ converge et vaut $S$. On pourra étudier $f\left(1-\frac{1}{n}\right)$.
\end{exercise}

\begin{exercise}
    Soit $f\colon\C\to\C$ développable en série entière avec un rayon de convergence $\rho>0$ telle que $f(0)\neq0$. Montrer qu'il existe une fonction $T$, développable en série entière, et $r>0$, telle que si $\left\lvert z\right\rvert<r$, $f(z)=\e^{T(z)}$.
\end{exercise}

\begin{exercise}
    Soit $a\in\R\setminus\Q$, on pose pour tout $n\geqslant1$, $a_n=\frac{1}{\sin(n\pi a)}$. Soit $R_a$ le rayon de convergence de $\sum a_{n}z^{n}$.
    \begin{enumerate}
        \item Montrer que $R_a\leqslant1$.
        \item Évaluer $R_a$ lorsque $a$ est irrationnel algébrique.
        \item Existe-t-il $a$ tel que $R_a=0$ ?
    \end{enumerate}
\end{exercise}

\begin{exercise}
    Soit $(a_{1},\dots,a_{N})\in\N^{\N}$ premiers entre eux dans leur ensemble. Pour $n\in\N$, on note $c_n=\left\lvert\left\lbrace (p_1,\dots,p_N)\in\N^{N}\middle| p_1a_1+\dots p_Na_N=n\right\rbrace\right\rvert$. Donner un équivalent simple de $c_n$ quand $n\to+\infty$.
\end{exercise}

\begin{exercise}
    Soit \function{f}{\R}{\R}{x}{\sqrt{1+x+x^{2}}}
    Montrer que $f$ est développable en série entière, et donner le rayon de convergence de la série entière obtenue. On pourra dériver $f^{2}(x)$.
\end{exercise}

\begin{exercise}
    Soit $f\colon[0,A[\to\R$ de classe $\mathcal{C}^{\infty}$ telle que pour tout $n\in\N$, pour tout $t\in[0,1[$, $f^{(n)}(t)\geqslant0$.
    \begin{enumerate}
        \item Soit $x\in[0,A[$, montrer que $\sum_{k\geqslant0}\frac{f^{(k)}(0)}{k!}x^{k}$ converge.
        \item On pose, pour $n\in\N$ et $x\in[0,A[$, $R_n(x)=\int_{0}^{x}\frac{(x-t)^{n}}{n!}f^{(n+1)}(t)\d t$. Montrer que si $x<y<A$, on a $0\leqslant R_n(x)\leqslant\left(\frac{x}{y}\right)^{n}R_n(y)$.
        \item En déduire que $f$ est développable en série entière sur $[0,A[$.
        \item Application à $\tan$.
    \end{enumerate}
\end{exercise}

\end{document}