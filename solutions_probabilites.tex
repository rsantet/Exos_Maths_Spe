\documentclass[12pt]{article}
\usepackage{style/style_sol}

\begin{document}

\begin{titlepage}
	\centering
	\vspace*{\fill}
	\Huge \textit{\textbf{Solutions MP/MP$^*$\\ Probabilités sur un univers dénombrable}}
	\vspace*{\fill}
\end{titlepage}

\begin{proof}
    \phantom{}
    \begin{enumerate}
        \item On note P:`le lancer initial donne pile', F:`le lancer initial donne face', $B_{k}$:`la k-ième boule est blanche', $N_{k}$:`la k-ième boule est noire'.
        
        On a 
        \begin{equation}
            \P\left(B_{k}\right)=\P\left(P\right)\P_{P}\left(B_{k}\right)+\P\left(F\right)\P_{F}\left(B_{k}\right)=\frac{1}{2}\frac{k}{k+1}+\frac{1}{2}\frac{1}{k+1}
        \end{equation}
        donc 
        \begin{equation}
            \boxed{\P\left(B_{k}\right)=\frac{1}{2}}
        \end{equation}

        \item On a 
        \begin{equation}
            \boxed{\P_{B_{k}}\left(P\right)=\P_{P}\left(B_{k}\right)\frac{\P\left(P\right)}{\P\left(B_{k}\right)}=\frac{k}{k+1}\xrightarrow[k\to+\infty]{}1}
        \end{equation}

        \item On a 
        \begin{align}
            \P\left(B_{1}\bigcap\dots\bigcap B_{k}\right)
            &=\frac{1}{2}\P_{P}\left(B_{1}\bigcap\dots\bigcap B_{k}\right)+\frac{1}{2}\P_{F}\left(B_{1}\bigcap\dots\bigcap B_{k}\right)\\
            &=\frac{1}{2}\left(\prod_{j=1}^{k}\frac{j}{j+1}+\prod_{j=1}^{k}\frac{1}{j+1}\right)\\
            &\boxed{=\frac{1}{2}\left(\frac{1}{k+1}+\frac{1}{(k+1)!}\right)}
        \end{align}

        \item On a 
        \begin{align}
            \P\left(B_{k}\bigcap B_{k+1}\right)
            &=\frac{1}{2}\left(\frac{k}{k+1}\times\frac{k+1}{k+2}+\frac{1}{k+1}\times\frac{1}{k+2}\right)\\
            &=\frac{1}{2}\left(\frac{k\left(k+1\right)+1}{\left(k+1\right)\left(k+2\right)}\right)
        \end{align}

        Donc on a indépendance si et seulement si 
        \begin{align}
            \P\left(B_{k}\bigcap B_{k+1}\right)=\P\left(B_{k}\right)\P\left(B_{k+1}\right)=\frac{1}{4}
            &\Leftrightarrow \frac{k\left(k+1\right)+1}{\left(k+1\right)\left(k+2\right)}=\frac{1}{2}\\
            &\Leftrightarrow 2k\left(k+1\right)+2=\left(k+2\right)\left(k+2\right)\\
            &\Leftrightarrow 2k^{2}+2k=k^{2}+3k\\
            &\Leftrightarrow \boxed{k=1}
        \end{align}
        Ainsi, seuls les deux premiers tirages sont indépendants.
    \end{enumerate}
\end{proof}

\begin{remark}
    Seuls les deux premiers tirages sont indépendants car le premier tirage est indépendant du lancer de pièce.
\end{remark}

\begin{proof}
    \phantom{}
    \begin{enumerate}
        \item 
        \begin{equation}
            \boxed{p_{0}=1,q_{0}=0,p_{N}=0,q_{N}=1}
        \end{equation}
        \item Soit $a\in\left\llbracket 1,N-1\right\rrbracket$. Puisque les lancers de pièce sont indépendants, on peut partitionner selon le résultat du premier lancer. On a donc [probabilités conditionnelles]
        \begin{equation}
            \boxed{p_{a}=p\times p_{a+1}+q\times p_{a-1}}
        \end{equation}

        L'équation caractéristique est 
        \begin{equation}
            pX^{2}-x+q=0
        \end{equation}
        On a $\Delta=1-4pq=1-4\left(1-p\right)p=4p^{2}-4p+1=\left(1-2p\right)^{2}$.

        Ainsi, si $p\neq\frac{1}{2}$, il existe $\left(\alpha,\beta\right)\in\R^{2}$ tels que pour tout $a\in\left\llbracket 0,N\right\rrbracket$, on a 
        \begin{equation}
            p_{a}=\alpha+\beta\left(\frac{q}{p}\right)^{a}
        \end{equation}
        Grâce aux valeurs en $a=0,a=N$, on en déduit que 
        \begin{equation}
            \boxed{p_{a}=\frac{1}{1-\left(\frac{q}{p}\right)^{N}}\times\left(\left(\frac{q}{p}\right)^{a}-\left(\frac{q}{p}\right)^{N}\right)}
        \end{equation}

        Si $p=\frac{1}{2}$, il existe $\left(\alpha,\beta\right)\in\R^{2}$ tels que 
        \begin{equation}
            p_{a}=\alpha a+\beta
        \end{equation}
        Grâce aux valeurs en $a=0,a=N$, on en déduit que 
        \begin{equation}
            \boxed{p_{a}=\frac{1}{N}\left(N-a\right)}
        \end{equation}

        \item Pour tout $a\in\left\llbracket 1,N-1\right\rrbracket$, on a 
        \begin{equation}
            q_{a}=pq_{a+1}+qp_{a-1}
        \end{equation}
        donc pour tout $a\in\left\llbracket 1,N-1\right\rrbracket$, on a 
        \begin{equation}
            p_{a}+q_{a}=p\left(p_{a+1}+q_{a+1}\right)+q\left(p_{a-1}+q_{a-1}\right)
        \end{equation}
        Comme $p_{0}+q_{0}=p_{N}+q_{N}=1$, on a pour tout $a\in\left\llbracket 0,N\right\rrbracket$,
        \begin{equation}
            \boxed{p_{a}+q_{a}=1}
        \end{equation}

        Ainsi, le jeu s'arrête presque sûrement en temps fini.
    \end{enumerate}
\end{proof}

\begin{proof}
    \phantom{}
    \begin{enumerate}
        \item Les tirs sont indépendants donc 
        \begin{equation}
            \boxed{
                \begin{array}[]{l}
                    \P\left(A_{n}\right)=\left(1-a\right)^{n}\times\left(1-b\right)^{n}\times a\\
                    \P\left(B_{n}\right)=\left(1-a\right)^{n}\times\left(1-b\right)^{n}\times \left(1-a\right)\times b
                \end{array}
            }    
        \end{equation}
        
        \item On a 
        \begin{equation}
            G_{A}=\bigcup_{n\in\N}A_{n}
        \end{equation}
        réunion disjointe. Donc 
        \begin{equation}
            \boxed{
                \begin{array}[]{l}
                    \P\left(G_{A}\right)=\sum_{n=0}^{+\infty}\P\left(A_{n}\right)=\frac{a}{1-\left(1-a\right)\left(1-b\right)}=\frac{a}{a+b-ab}\\
                    \P\left(G_{B}\right)=\sum_{n=0}^{+\infty}\P\left(B_{n}\right)=\frac{b\left(1-a\right)}{a+b-ab}\\
                \end{array}
            }
        \end{equation}
        
        Ainsi, 
        \begin{equation}
            \boxed{\P\left(G_{A}\right)+\P\left(G_{B}\right)=1}
        \end{equation}

        \item On a $\P\left(G_{A}\right)=\P\left(G_{B}\right)$ si et seulement si 
        \begin{equation}
            \frac{a}{1-a}=b
        \end{equation}
        Cela implique que $\frac{a}{1-a}\in]0,1[$ ce qui est possible uniquement (après étude de fonction) si
        \begin{equation}
            \boxed{a\in\left]0,\frac{1}{2}\right[\text{ et }b=\frac{a}{1-a}}
        \end{equation}
    \end{enumerate}
\end{proof}

\begin{proof}
    \phantom{}
    \begin{enumerate}
        \item Pour $n\in\N^{*}$, on pose $E_{n}$:`Le joueur gagne au bout du n-ième lancer' (évènement disjoints) et G:`Le joueur gagne'. On a $G\cup_{n\in\N^{*}}E_{n}$. Donc 
        \begin{equation}
            \boxed{\P\left(G\right)=\sum_{n\in\N^{*}}\P\left(E_{n}\right)=\sum_{n\in\N^{*}}\left(\frac{1}{2}\right)^{n}\times\frac{1}{n}=\ln(2)}
        \end{equation}

        \item On note $P_{n}$:`le joueur obtient pile au n-ième lancer', P:`il obtient pile'. On a 
        \begin{equation}
            \P_{G}\left(P_{n}\right)=\frac{\P\left(G\bigcap P_{n}\right)}{\P\left(G\right)}=\frac{\P_{P_{n}}\left(G\right)\times\P\left(P_{n}\right)}{\P\left(G\right)}
        \end{equation}
        donc 
        \begin{equation}
            \boxed{\P_{G}\left(P_{n}\right)=\frac{\frac{1}{n}\left(\frac{1}{2}    \right)^{n}}{\ln\left(2\right)}}
        \end{equation}

        Puis 
        \begin{equation}
            \boxed{\P_{G}\left(P\right)=\sum_{n\in\N^{*}}\P_{G}\left(P_{n}\right)=1}
        \end{equation}
    \end{enumerate}
\end{proof}

\begin{remark}
    On a utilisé le résultat suivant: pour tout $x\in\left]0,1\right[$, 
    \begin{equation}
        \sum_{n=1}^{+\infty}\frac{x^{n}}{n}=-\ln\left(1-x\right)
    \end{equation}
    Soit on connaît le résultat avec les séries entières, soit on le redémontre à la main: pour $N\geqslant1$, on a 
    \begin{align}
        \sum_{n=1}^{N}\frac{x^{n}}{n}
        &=\int_{0}^{1}\sum_{n=1}^{N}x^{n}t^{n-1}dt\\
        &=x\int_{0}^{1}\frac{1-\left(xt\right)^{N}}{1-xt}dt\\
        &=\underbrace{\int_{0}^{1}\frac{x}{1-xt}dt}_{=\left[\ln\left(1-xt\right)\right]_{0}^{1}}+R_{N}
    \end{align}
    avec $\left\lvert R_{N}\right\rvert\leqslant\frac{x^{N+1}}{1-x}\xrightarrow[N\to+\infty]{}0$ d'où le résultat.
\end{remark}

\begin{proof}
    \phantom{}
    \begin{enumerate}
        \item On a 
        \begin{equation}
            \sum_{k=0}^{+\infty}p_{k}=p_{0}+p_{1}+\sum_{k=2}^{+\infty}\frac{1-2\alpha}{2^{k-1}}=2\alpha+\left(1-2\alpha\right)\times\sum_{k=1}^{+\infty}\frac{1}{2^{k}}=1
        \end{equation}
        donc 
        \begin{equation}
            \boxed{\text{c'est une probabilité sur }\N.}
        \end{equation}

        \item Pour tout $k\in\N$, on note $E_{k}$:`la famille a $k$ enfants et exactement 2 garçons', $E$:`la famille a exactement 2 garçons', $A_{k}$:`la famille a $k$ enfants'.
        
        On a alors 
        \begin{align}
            \P\left(E\right)
            &=\sum_{k=2}^{+\infty}\P_{A_{k}}\left(E_{k}\right)\times\P\left(A_{k}\right)\\
            &=\sum_{k=2}^{+\infty}\binom{k}{2}\left(\frac{1}{2}\right)^{k}\times p_{k}\\
            &=\sum_{k=2}^{+\infty}\frac{k\left(k-1\right)}{2^{k+1}}\times\frac{1-2\alpha}{2^{k-1}}\\
            &=\left(1-2\alpha\right)\sum_{k=2}^{+\infty}\frac{k\left(k-1\right)}{2^{2k}}\\
            &=\left(1-2\alpha\right)\sum_{k=0}^{+\infty}\frac{\left(k+1\right)\left(k+2\right)}{2^{2k+4}}\\
            &=\frac{1}{16}\left(1-2\alpha\right)\sum_{k=0}^{+\infty}\frac{\left(k+1\right)\left(k+2\right)}{4^{k}}=\frac{1}{16}\left(1-2\alpha\right)\times\frac{1}{\left(\frac{3}{4}\right)^{3}}\\
            &\boxed{=\frac{4\left(1-2\alpha\right)}{27}}
        \end{align}

        \item On note $F$:`la famille a au moins 2 filles', $F_{k}$:`la famille a exactement $k$ filles et au moins 4 enfants', $G$:`la famille a au moins 2 garçons', $G_{k}$:`la famille a exactement $k$ garçons et au moins 4 enfants'.
        
        On a 
        \begin{equation}
            \P_{G}\left(G\right)=\frac{\P\left(F\cap G\right)}{\P\left(G\right)}
        \end{equation}
        et $\overline{F\cap G}=\overline{F}\cup\overline{G}=F_{0}\cup F_{1}\cup G_{0}\cup G_{1}$.
        Donc, comme $\P\left(F_{0}\right)=\P\left(G_{0}\right)$ et $\P\left(F_{1}\right)=\P\left(G_{1}\right)$, on a $\P\left(F\cap G\right)=1-2\left(\P\left(G_{0}\right)+\P\left(G_{1}\right)\right)$.

        On a alors 
        \begin{align}
            \P\left(G_{0}\right)
            &=\sum_{k=4}^{+\infty}\binom{k}{0}\left(\frac{1}{2}\right)^{k}p_{k}\\
            &=\sum_{k=4}^{+\infty}\frac{1-2\alpha}{2^{2k-1}}\\
            &=2\left(1-2\alpha\right)\frac{1}{4^{4}}\times\frac{1}{1-\frac{1}{4}}\\
            &=2\left(1-2\alpha\right)\times\frac{1}{4^{3}}\times\frac{1}{3}
        \end{align}
        et 
        \begin{align}
            \P\left(G_{1}\right)
            &=\sum_{k=4}^{+\infty}\binom{k}{1}\left(\frac{1}{2}\right)^{k}p_{k}\\
            &=\sum_{k=4}^{+\infty}k\times\frac{1}{2^{k}}\times\frac{1-2\alpha}{2^{k-1}}\\
            &=\left(1-2\alpha\right)\sum_{k=4}^{+\infty}\frac{k}{2^{2k-1}}\\
            &=\left(1-2\alpha\right)\times\frac{2}{4}\sum_{k=4}^{+\infty}\frac{k}{4^{k-1}}\\
            &=\frac{1-2\alpha}{2}\sum_{k=3}^{+\infty}\frac{k+1}{4^{k}}\\
            &=\frac{1-2\alpha}{2}\times\left(\frac{1}{\left(1-\frac{1}{4}\right)^{2}}-1-\frac{2}{4}-\frac{3}{4^{2}}\right)
        \end{align}

        et on calcule enfin 
        \begin{equation}
            \boxed{\P\left(G\right)=1-\P\left(G_{0}\right)-\P\left(G_{1}\right)}
        \end{equation}
    \end{enumerate}
\end{proof}

\begin{proof}
    Pour tout $k\geqslant1$, on note $A_{k}$:`A gagne à son lancé $k$' et $B_{k}$ de manière équivalente pour le joueur $B$. On note $G_{A}$:`A gagne' et de même pour B. On a ainsi
    \begin{equation}
        G_{A}=\bigcup_{k\geqslant 1}A_{k}
    \end{equation}
    (réunion disjointe) et pareil pour $G_{B}$. On a 
    \begin{equation}
        \P\left(A_{k}\right)=\left(1-\frac{5}{36}\right)^{k-1}\times\left(1-\frac{1}{6}\right)^{k-1}\times\frac{5}{36}
    \end{equation}
    d'où 
    \begin{equation}
        \P\left(G_{A}\right)=\frac{5}{36}\times\frac{1}{1-\left(1-\frac{5}{36}\right)\left(1-\frac{1}{6}\right)}
    \end{equation}
    et pareil 
    \begin{equation}
        \boxed{\P\left(G_{B}\right)=\frac{1}{6}\times\left(1-\frac{5}{36}\right)\times\frac{1}{1-\left(1-\frac{5}{36}\right)\left(1-\frac{1}{6}\right)}>\P\left(G_{A}\right)}
    \end{equation}
    et 
    \begin{equation}
        \P\left(G_{A}\right)+\P\left(G_{B}\right)=1
    \end{equation}
    donc $G_{A}\cup G_{B}$ est presque sur.
\end{proof}

\begin{proof}
    Soit $k\in\left\llbracket0,\left\lfloor\frac{n}{2}\right\rfloor\right\rrbracket$. La probabilité que l'on tire $2k$ boules blanches est (loi binomiale):
    \begin{equation}
        \binom{n}{2k}\times\left(\frac{a}{a+b}\right)^{2k}\times\left(\frac{b}{a+b}\right)^{n-2k}
    \end{equation}
    donc la probabilité que le nombre de boules blanches tirées soit pair est 
    \begin{equation}
        \P_{P}=\sum_{0\leqslant 2k\leqslant n}\binom{n}{2k}\times\left(\frac{a}{a+b}\right)^{2k}\times\left(\frac{b}{a+b}\right)^{n-2k}
    \end{equation}
    
    De même, la probabilité que le nombre de boules blanches tirées soit impair est 
    \begin{equation}
        \P_{I}=\sum_{0\leqslant 2k+1\leqslant n}\binom{n}{2k+1}\times\left(\frac{a}{a+b}\right)^{2k+1}\times\left(\frac{b}{a+b}\right)^{n-2k-1}
    \end{equation}

    On a alors 
    \begin{equation}
        \P_{P}+\P_{I}=1
    \end{equation}
    et 
    \begin{equation}
        \P_{P}-\P_{I}=\sum_{k'=0}^{n}\binom{n}{k'}\left(-1\right)^{k'}\left(\frac{a}{a+b}\right)^{k'}\left(\frac{b}{a+b}\right)^{n-k'}=\left(\frac{b-a}{a+b}\right)^{n}
    \end{equation}

    On a donc 
    \begin{equation}
        \boxed{\P_{P}=\frac{1}{2}\left(1+\left(\frac{b-a}{a+b}\right)^{n}\right)}
    \end{equation}
\end{proof}

\begin{remark}
    Si on note $\P_{3}$ la probabilité que le nombre de boules blanches tirées soit multiple de 3:
    \begin{equation}
        \P_{3}=\sum_{0\leqslant 3k\leqslant n}\binom{n}{3k}\left(\frac{a}{a+b}\right)^{3k}\left(\frac{b}{a+b}\right)^{n-3k}
    \end{equation}
    On note $\P_{2}$ la probabilité pour que le nombre de boules blanches tirées soit congru à 2 module 3, et on définit $\P_{1}$ de même.
    Alors on a 
    \begin{equation}
        \left\{
            \begin{array}[]{rcl}
                \P_{1}+\P_{2}+\P_{3} &= &1\\
                \mathrm{j}\P_{1}+\mathrm{j}^{2}\P_{2}+\P_{3} &= &\left(\frac{b+\mathrm{j}a}{a+b}\right)^{n}\\
                \mathrm{j}^{2}\P_{1}+\mathrm{j}\P_{1}+\P_{3} &= &\left(\frac{b+\mathrm{j}^{2}a}{a+b}\right)^{n}
            \end{array}
        \right.
    \end{equation}
    et donc 
    \begin{equation}
        \P_{3}=\frac{1}{3}\left(1+\left(\frac{b+\mathrm{j}a}{a+b}\right)^{n}+\left(\frac{b+\mathrm{j}^{2}a}{a+b}\right)^{n}\right)
    \end{equation}
\end{remark}

\begin{proof}
    Soit pour $i\in\left\llbracket 1,n\right\rrbracket$, 
    \begin{equation}
        A_{i}=\left\{\sigma\in\Sigma_{n}\middle|\sigma(i)=i\right\}
    \end{equation}
    \begin{equation}
        A=\left\{\sigma\in\Sigma_{n}\middle|\sigma\text{ a un point fixe}\right\}
    \end{equation}
    On a 
    \begin{equation}
        A=\bigcup_{i=1}^{n}A_{i}
    \end{equation}

    On a 
    \begin{equation}
        \left\lvert A\right\rvert=\sum_{k=1}^{n}(-1)^{k-1}\sum_{\substack{J\subset\left\llbracket 1,n\right\rrbracket\\\left\lvert J\right\rvert=k}}\left\lvert\bigcap_{i\in J}A_{i}\right\rvert
    \end{equation}
    Il y a $\binom{n}{k}$ tels $J$, et on a 
    \begin{equation}
        \left\lvert\bigcap_{i\in J}A_{i}\right\rvert=\left\lvert\left\{\sigma\in\Sigma_{n}\middle|\forall i\in J,\sigma(i)=i\right\}\right\rvert=(n-k)!
    \end{equation}
    Ainsi, 
    \begin{equation}
        \left\lvert A\right\rvert=\sum_{k=1}^{n}(-1)^{k-1}\binom{n}{k}(n-k)!
    \end{equation}
    donc 
    \begin{equation}
        \boxed{p_{n}=\sum_{k=1}^{n}\frac{(-1)^{k-1}}{k!}\xrightarrow[n\to+\infty]{}\sum_{k=1}^{+\infty}\frac{(-1)^{k-1}}{k!}=-\left(\frac{1}{e}-1\right)=1-\frac{1}{e}}
    \end{equation}
\end{proof}

\begin{proof}
    \phantom{}
    \begin{enumerate}
        \item 
        \begin{equation}
            \boxed{p_{N}(0)=0,p_{N}(1)=1}    
        \end{equation}

        \item Pour tout $n\in\left\llbracket 1,N-1\right\rrbracket$, on a 
        \begin{equation}
            p_{N}(n)=p\times p_{N}(n+1)+(1-n)\times p_{N}(n-1)
        \end{equation}
        et l'équation caractéristique est $X^{2}-\frac{1}{p}X+\frac{1-p}{p}$ et le discriminant vaut 
        \begin{equation*}
            \Delta=\left(\frac{1}{p}-2\right)^{2}\geqslant0.
        \end{equation*}Donc les solutions sont $r_{1}=1$ et $r_{2}=\frac{q}{p}$. Ainsi, pour tout $n\in\left\llbracket 1,N-1\right\rrbracket$,
        \begin{equation}
            p_{N}(n)=\lambda+\mu\left(\frac{q}{p}\right)^{n}
        \end{equation}
        avec $\left(\lambda,\mu\right)\in\R^{2}$.
        
        Avec les conditions initiales, on trouve 
        \begin{equation}
            \left\{
                \begin{array}[]{rcl}
                    \mu &= &\frac{1}{\left(\frac{q}{p}\right)^{N}-1}\\
                    \lambda &= &\frac{1}{1-\left(\frac{q}{p}\right)^{N}}
                \end{array}
            \right.
        \end{equation}
        donc 
        \begin{equation}
            \boxed{
            p_{N}(n)=\frac{1-\left(\frac{q}{p}\right)^{n}}{1-\left(\frac{q}{p}\right)^{N}}\xrightarrow[N\to+\infty]{}
            \left\{
                \begin{array}[]{ll}
                    0 &\text{si }q>p\text{ i.e.~}p<\frac{1}{2}\\
                    1-\left(\frac{q}{p}\right)^{n} &\text{si }q<p\text{ i.e.~}p>\frac{1}{2}
                \end{array}
            \right.}
        \end{equation}
        On vérifie d'ailleurs que l'arrêt en temps fini est presque sûr: $p_{N}(n)+q_{N}(n)=1$ (utiliser la relation de récurrence et les conditions initiales).
    \end{enumerate}
\end{proof}

\begin{proof}
    \phantom{}
    \begin{enumerate}
        \item On note $A_{n}$:`la première boule blanche apparaît au $n$-ième tirage' et $B_{n}$:`on tire une boule noire au $n$-ième tirage'. On a 
        \begin{equation}
            A_{n}=\bigcap_{i=1}^{n-1}B_{i}\bigcap\overline{B_{n}}
        \end{equation}
        ce qui implique donc 
        \begin{align}
            \P\left(A_{n}\right)
            &=p_{n}\\
            &=\P\left(B_{1}\right)\P_{B_{1}}\left(B_{2}\right)\dots\P_{B_{1}\cap \dots \cap B_{n-1}}\left(\overline{B_{n}}\right)\\
            &=\frac{1}{2}\times\frac{2}{3}\times\dots\times\frac{n-1}{n}\times\frac{1}{n+1}\\
            &=\boxed{\frac{1}{n(n+1)}}
        \end{align}
        et par sommation téléscopique, on a 
        \begin{equation}
            \boxed{\sum_{n=1}^{+\infty}p_{n}=1}
        \end{equation}
        Donc on tire une boule blanche presque sûrement.

        \item On utilise le même principe: pour $n\geqslant1$,
        \begin{equation}
            \boxed{\P\left(A_{n}\right)=p_{n}=\frac{1}{2}\times\frac{c+1}{c+2}\times\frac{2c+1}{2c+2}}\times\dots\times\frac{\left(n-2\right)c+1}{\left(n-2\right)c+2}\times\frac{1}{\left(n-1\right)c+2}
        \end{equation}
        Comme les $\left(A_{n}\right)_{n\in\N^{*}}$ sont incompatibles, on a 
        \begin{equation}
            \sum_{n\geqslant1}\P\left(A_{n}\right)=\P\left(\bigcup_{n\geqslant1} A_{n}\right)\leqslant1
        \end{equation}
        donc 
        \begin{equation}
            \boxed{\text{la série converge.}}
        \end{equation}

        On peut montrer à nouveau que le tirage d'une boule blanche reste presque sûr. En effet, on a 
        \begin{equation}
            \frac{p_{n+1}}{p_{n}}=\frac{nc+2-c-1}{nc+2}=1-\frac{c+1}{nc+2}=a-\frac{c+1}{nc}+\underset{n\to+\infty}{O}\left(\frac{1}{n^{2}}\right)
        \end{equation}
        D'après la règle de Raabe-Duhamel, il existe $K>0$ tel que 
        \begin{equation}
            p_{n}\underset{n\to+\infty}{\sim}\frac{K}{n^{\frac{c+1}{c}}}
        \end{equation}
        avec $\frac{c+1}{c}>1$. Notamment, $\lim\limits_{n\to+\infty}np_{n}=0$. Comme 
        \begin{equation}
            \left(nc+2\right)p_{n+1}=\left(\left(n-1\right)c+1\right)p_{n}
        \end{equation}
        on a 
        \begin{align}
            \sum_{n=1}^{+\infty}ncp_{n+1}-\left(n-1\right)cp_{n}
            &=\sum_{n=1}^{+\infty}p_{n}-2p_{n+1}\\
            &=\sum_{n=1}^{+\infty}p_{n}-2\left(\sum_{n=1}^{+\infty}p_{n}-u_{1}\right)
        \end{align}
        La première somme est téléscopique et vaut 0, et $u_{1}=\frac{1}{2}$ donc on trouve bien 
        \begin{equation}
            \sum_{n=1}^{+\infty}p_{n}=1
        \end{equation}
    \end{enumerate}
\end{proof}

\begin{remark}
    On peut contourner la règle de Raabe-Duhamel. On écrit 
    \begin{align}
        \ln\left(p_{n+1}\right)
        &=-\ln\left(2\right)+\sum_{k=1}^{n-1}\ln\left(\frac{kc+1}{kc+2}\right)-\ln\left(nc+2\right)\\
        &=\sum_{k=1}^{n-1}\ln\left(1-\frac{1}{kc+2}\right)-\ln\left(n\right)+\ln\left(c\right)-\ln(2)+\underset{n\to+\infty}{o}\left(1\right)\\
        &=-\sum_{k=1}^{n-1}\left(\frac{1}{kc}+\underset{k\to+\infty}{O}\left(\frac{1}{k^{2}}\right)\right)-\ln\left(n\right)-A+\underset{n\to+\infty}{o}\left(1\right)\\
        &=-\frac{1}{c}\left(\ln\left(n\right)+\gamma+\underset{n\to+\infty}{o}\left(1\right)\right)-\ln\left(n\right)-A+\underset{n\to+\infty}{o}\left(1\right)\\
        &=-\ln\left(n\right)\left(1+\frac{1}{c}\right)+A'+\underset{n\to+\infty}{o}\left(1\right)
    \end{align}
    Ainsi, 
    \begin{equation}
        p_{n+1}\underset{n\to+\infty}{\sim}\frac{K}{n^{1+\frac{1}{c}}}
    \end{equation}
    donc la série converge.
\end{remark}

\begin{proof}
    On a 
    \begin{equation}
        u_{n+1}=q\times 1+p\times u_{n}^{2}
    \end{equation}
    car soit la bactérie meure au premier jour, soit les deux descendants n'ont plus de lignée au $n$-ième jour (on a $u_{n}^{2}$ car les lignées des deux descendants sont indépendantes).

    Soit \function{f}{[0,1]}{\R}{x}{q+px^{2}}
    Si $x\in[0,1]$, on a $f(x)\in[0,1]$ car $f(1)=q+p=1$. Soit $g(x)=f(x)-x$. On a 
    \begin{equation}
        g(x)=p\left(x-1\right)\left(x-\frac{p}{q}\right)
    \end{equation}
    \begin{itemize}
        \item Si $1\leqslant\frac{p}{q}$: on a pour tout $x\in[0,1[$, $g(x)>0$ et $g(1)=0$. Donc si 
        \begin{equation}
            \boxed{\lim\limits_{n\to+\infty}u_{n}=1}
        \end{equation}
        car c'est une suite croissante, majorée, convergente vers le point fixe 1.

        \item Si $1>\frac{q}{p}$: si $x\in\left[0,\frac{q}{p}\right[$, on a $g(x)>0$, si $x\in\left]\frac{q}{p},1\right[$, $g(x)<0$ et $g\left(\frac{q}{p}\right)=0$.
        
        Par récurrence, comme $u_{0}=0$, pour tout $n\in\N$, $u_{n}\in\left[0,\frac{q}{p}\right[$ donc (suite croissante majorée qui converge vers le point fixe $\frac{q}{p}$) donc 
        \begin{equation}
            \boxed{\lim\limits_{n\to+\infty}u_{n}=\frac{q}{p}}
        \end{equation}
    \end{itemize}

    On a bien 
    \begin{equation}
        \boxed{\lim\limits_{n\to+\infty}u_{n}=\min\left(1,\frac{q}{p}\right)}
    \end{equation}

    Ainsi, la lignée s'éteint presque sûrement si et seulement si $\frac{q}{p}\geqslant1$ i.e.~$p\leqslant\frac{1}{2}$. Sinon, la probabilité d'extinction est $\frac{q}{p}$.

    Si $p=\frac{1}{2}$, on pose $\varepsilon_{n}=1-u_{n}\xrightarrow[n\to+\infty]{}0$. On a 
    \begin{equation}
        u_{n+1}=\frac{1}{2}\left(1+u_{n}^{2}\right)
    \end{equation}
    d'où 
    \begin{equation}
        \varepsilon_{n+1}=1-u_{n+1}=\varepsilon_{n}\left(1-\frac{\varepsilon}{2}\right)
    \end{equation}

    Soit $\alpha\in\R$, on a 
    \begin{equation}
        \varepsilon_{n+1}^{\alpha}=\varepsilon_{n}^{\alpha}\left(1-\frac{\varepsilon_{n}}{2}\right)^{\alpha}=\varepsilon_{n}^{\alpha}-\frac{\alpha\varepsilon_{n}^{\alpha+1}}{2}+\underset{n\to+\infty}{o}\left(\varepsilon_{n}^{\alpha+1}\right)
    \end{equation}

    On choisit $\alpha=-1$, on a
    \begin{equation}
        \frac{1}{\varepsilon_{n+1}}-\frac{1}{\varepsilon_{n}}\xrightarrow[n\to+\infty]{}\frac{1}{2}
    \end{equation}

    D'après le lemme de Césaro, on a $\frac{1}{\varepsilon_{n}}\underset{n\to+\infty}{\sim}\frac{n}{2}$ d'où 
    \begin{equation}
        \boxed{\varepsilon_{n}\underset{n\to+\infty}{\sim}\frac{2}{n}}
    \end{equation}
\end{proof}

\begin{proof}
    On note $E_{n}$:`la puce est en $0$ à l'instant $2n$' et $B_{n}$:`la puce repasse pour la première fois en $0$ à l'instant $2n$'.

    Soit $E$:`la puce repasse par l'origine'. On a 
    \begin{equation}
        E=\bigcup_{n\in\N^{*}}E_{n}=\bigcup_{n\in\N^{*}}B_{n}
    \end{equation}
    où les $B_{n}$ sont disjoints donc $\P(E)=\sum_{n\in\N^{*}}\P(B_{n})$.

    On a 
    \begin{equation}
        \P(E_{n})=\binom{2n}{n}p^{n}q^{n} 
    \end{equation}
    On écrit alors 
    \begin{equation}
        E_{n}=\bigcup_{1\leqslant k\leqslant n}\left(E_{n}\cap B_{k}\right)
    \end{equation}
    où la réunion est disjointe (on partitionne selon le premier passage en 0). D'où 
    \begin{equation}
        u_{n}=\P(E_{n})=\sum_{k=1}^{n}\P(B_{k})\P_{B_{k}}(E_{n})
    \end{equation}
    On pose $b_{k}=\P(B_{k})$ et on a $\P_{B_{k}}(E_{n})=\P(E_{n-k})=u_{n-k}$: c'est comme si on repartait de 0 à l'étape $k$. On a donc $u_{0}=\P(E_{0})=1$ et pour tout $n\geqslant1$,
    \begin{equation}
        u_{n}=\sum_{k=1}^{n}b_{k}u_{n-k}=\sum_{k=0}^{n}b_{k}u_{n-k}
    \end{equation}
    en posant $b_{0}=0$.

    Or, on a 
    \begin{equation}
        u_{n}=\frac{(2n)!}{(n!)^{2}}(pq)^{n}\underset{n\to+\infty}{\sim}\frac{\sqrt{4\pi n}\left(\frac{2n}{e}\right)^{2n}}{2\pi n\left(\frac{n}{e}\right)^{2n}}(pq)^{n}
    \end{equation}
    d'où 
    \begin{equation}
        u_{n}\underset{n\to+\infty}{\sim}\frac{(4pq)^{n}}{\sqrt{\pi n}}
    \end{equation}
    et on a $4pq<1$ si et seulement si $p\neq\frac{1}{2}$ donc $\sum_{n\in\N}u_{n}$ converge si et seulement si $p\neq\frac{1}{2}$.

    Dans le cas $p\neq\frac{1}{2}$, on pose $S=\sum_{n=0}^{+\infty}u_{n}$. On a 
    \begin{align}
        \sum_{n=1}^{+\infty}u_{n}
        &=S-u_{0}\\
        &=S-1\\
        &=\sum_{n=1}^{+\infty}\sum_{k=0}^{n}b_{k}u_{n-k}\\
        &=\sum_{n=0}^{+\infty}\sum_{k=0}^{n}b_{k}u_{n-k}\\
        &=\left(\sum_{n=0}^{+\infty}b_{n}\right)\left(\sum_{l=0}^{+\infty}u_{l}\right)
        &=S\sum_{n=0}^{+\infty}b_{n}
    \end{align}
    donc 
    \begin{equation}
        \boxed{\sum_{n=0}^{+\infty}b_{n}=\P(E)=\frac{S-1}{S}<1}
    \end{equation}

    Comme dans ce cas, on a $\sum_{n\geqslant1}\P(E_{n})<\infty$, le lemme de Borel-Cantelli indique que le nombre de retours à l'origine est presque sûrement fini.
\end{proof}

\begin{remark}

    Avec les séries entières, on peut vérifier que 
    \begin{equation}
        S=\frac{1}{\sqrt{1-4pq}}
    \end{equation}
    d'où
    \begin{equation}
        \P(E)=1-\sqrt{1-4pq}
    \end{equation}

    Dans le cas $p=\frac{1}{2}$, $\sum_{n\in\N^{*}}u_{n}$ diverge. Comme on a pour $p\neq\frac{1}{2}$, on a 
    \begin{equation}
        \sum_{n=0}^{+\infty}b_{n}(p)=1-\sqrt{4p(1-p)}
    \end{equation}
    et $b_{n}(p)\leqslant b_{n}\left(\frac{1}{2}\right)$, on peut passer à la limite donc 
    \begin{equation}
        \sum_{n=0}^{+\infty}b_{n}\left(\frac{1}{2}\right)=1
    \end{equation}
    et la retour en 0 est presque sûr si $p=\frac{1}{2}$.
\end{remark}

\begin{remark}
    Pour montrer que 
    \begin{equation}
        l(x)=\sum_{n=0}^{+\infty}\binom{2n}{n}x^{n}=\frac{1}{1-4x}
    \end{equation}
    lorsque $0\leqslant x<\frac{1}{4}$. On effectue un produit de Cauchy
    \begin{equation}
        l(x)^{2}=\sum_{n=0}^{+\infty}\left(\sum_{k=0}^{n}\binom{2k}{k}\binom{2n-2k}{n-k}\right)x^{n}=\sum_{n=0}^{+\infty}4^{n}x^{n}=\frac{1}{1-4r}
    \end{equation}
    en dénombrant les parties d'un ensemble à $2n$ éléments séparées en $n$ éléments dans $A$ et $n$ éléments dans $B$.
\end{remark}

\begin{proof}
    On note $P_{n}$:`on obtient pile au $n$-ième lancer' et $F_{n}$:`on obtient face au $n$-ième lancer'.
    \begin{enumerate}
        \item On a 
        \begin{equation}
            \boxed{a_{1}=0,a_{2}=p^{2},a_{3}=qp^{2}}
        \end{equation}
        \item Pour $n\geqslant4$, on a 
        \begin{equation}
            A_{n}=\bigcap_{k=1}^{n-3}\overline{A_{k}}\bigcap F_{n-2}\bigcap P_{n-1}\bigcap P_{n}
        \end{equation}
        Comme les évènements concernant des lancers différents sont supposés indépendants, on a 
        \begin{equation}
            a_{n}=\P\left(\bigcap_{k=1}^{n-3}\overline{A_{k}}\right)qp^{2}
        \end{equation}
        On écrit 
        \begin{equation}
            \P\left(\bigcap_{k=1}^{n-3}\overline{A_{k}}\right)=1-\P\left(\bigcup_{k=1}^{n-3}A_{k}\right)=1-\sum_{k=1}^{n-3}a_{k}
        \end{equation}
        car les $A_{k}$ sont incompatibles. Ainsi, 
        \begin{equation}
            a_{n}=p^{2}q\left(1-\sum_{k=1}^{n-3}a_{k}\right)
        \end{equation}
        et $\sum_{k\geqslant1}a_{k}$ converge puisque 
        \begin{equation}
            \sum_{k=1}^{N}a_{k}\leqslant\P\left(\bigcup_{k\geqslant1}\right)\leqslant1
        \end{equation}

        Pour calculer $a_{n}$, on remarque que 
        \begin{equation}
            B_{n}=\bigcap_{k=1}^{n}\overline{A_{k}}
        \end{equation}
        est exactement l'évènement `on n'a pas deux piles consécutifs dans les lancers $\lbrace1,\dots,n\rbrace$'.

        Si $P_{n}$, on a nécessairement $F_{n-1}$ et $B_{n-2}$, si $F_{n}$ on a nécessairement $B_{n-1}$. Ainsi,
        \begin{equation}
            \P(B_{n})=qp\P(B_{n-2})+q\P(B_{n-1})
        \end{equation}

        On a l'équation caractéristique $X^{2}-qX-pq$, le discriminant est $\Delta=q^{2}+4pq>0$. On en déduit les racines $\lambda_{1}=\frac{q+\sqrt{\Delta}}{2}$ et $\lambda_{2}=\frac{q-\sqrt{\Delta}}{2}$, et on utilise les conditions aux limites $\P(B_{0})=\P(B_{1})=1$ et $\P(B_{n})=A\lambda_{1}^{n}+B\lambda_{2}^{n}$.
    \end{enumerate}
\end{proof}

\begin{remark}
    La probabilité d'obtenir une séquence fixée de longueur $N$ est égale à 1. En effet, on pose pour tout $n\in\N$, $A_{n}$:`la séquence apparaît entre les lancers $nN+1$ et $(n+1)N$'. Les $A_{n}$ sont clairement indépendants et on a $\P(A_{n})=\P(A_{1})=\alpha>0$ et $\sum_{n\geqslant1}\P(A_{n})$ diverge. Notamment,
    \begin{equation}
        \P\left(\overline{\bigcup_{n\in\N}A_{n}}\right)=\P\left(\bigcap_{n\in\N}\overline{A_{n}}\right)=\lim\limits_{k\to+\infty}\P\left(\bigcap_{n=0}^{k}\overline{A_{n}}\right)=\lim\limits_{k\to+\infty}\prod_{n=0}^{k}(1-\alpha)=0
    \end{equation}
    On a donc presque sûrement la séquence. D'après le lemme de Borel-Cantelli, on a presque sûrement une infinité de fois la séquence.
\end{remark}

\begin{proof}
    On note $N_{n}$:`on tire une boule noire au $n$-ième tirage', et $B_{n}$:`on tire une boule blanche au $n$-ième tirage'.
    
    On a 
        \begin{equation}
            \P_{N}(n)=\P_{B_{1}\cap\dots\cap B_{n}}(B_{n+1})=\frac{\P(B_{1}\cap\dots\cap B_{n+1})}{\P(B_{1}\cap\dots\cap B_{n})}
        \end{equation}
        Or 
        \begin{equation}
            \P(B_{1}\cap\dots\cap B_{n})=\sum_{k=0}^{N}\frac{1}{N+1}\left(\frac{k}{n}\right)^{n}\xrightarrow[N\to+\infty]{}\int_{0}^{1}x^{n}dx=\frac{1}{n+1}
        \end{equation}
        car pour $k$ fixé, $\frac{1}{N+1}$ est la probabilité d'avoir l'urne $k$ et $\left(\frac{k}{n}\right)^{n}$ est la probabilité d'avoir une blanche sachant qu'on a pris l'urne $k$, et la limite vient d'une somme de Riemann.
        
        Donc 
        \begin{equation}
            \boxed{\P_{N}(n)\xrightarrow[N\to+\infty]{}\frac{\frac{1}{n+2}}{\frac{1}{n+1}}=\frac{n+1}{n+2}}
        \end{equation}
\end{proof}

\begin{remark}
    Pour $n=0$, on a 
    \begin{equation}
        \P_{N}(0)=\frac{\frac{1}{N}\sum_{k=0}^{N}k}{N+1}=\frac{1}{2}
    \end{equation}
\end{remark}

\begin{proof}
    Si cette probabilité est définie, on note $p$ la probabilité recherchée. Soit $d\in\N^{*}$, pour tout $(n_{1},n_{2})\in\left(\N^{*}\right)^{2}$, on a $n_{1}\wedge n_{2}=d$ si et seulement si $n_{1}=dn_{1}'$ et $n_{2}=dn_{2}'$ avec $n_{1}'\wedge n_{2}'=1$. Ainsi, la probabilité pour que $n_{1}\wedge n_{2}=d$ est $\frac{p}{d^{2}}$ et 
    \begin{equation}
        \sum_{d\geqslant1}\frac{p}{d^{2}}=1
    \end{equation}
    d'où 
    \begin{equation}
        \boxed{p=\frac{6}{\pi^{2}}}
    \end{equation}
\end{proof}

\begin{remark}
    Pour justifier un peu plus précisément, on note que dans l'ensemble $\llbracket 1,dN\rrbracket$, la proportion de multiplies de $d$ est de $\frac{1}{d}$, donc sur $\llbracket 1,dN\rrbracket^{2}$, la proportion de couples de multiples de $d$ est  $\frac{1}{d^{2}}$.
\end{remark}

\begin{proof}

    On note $q_{n}=\P(X_{n}=1)$ (qui détermine la loi de $X_{n}$ car c'est une variable de Bernouilli). On a $q_{1}=p_{2}$ et pour $n\geqslant2$,
    \begin{equation}
        q_{n}=p_{1}q_{n-1}+p_{2}(1-q_{n-1})=(p_{1}-p_{2})q_{n-1}+p_{2}
    \end{equation}
    La relation est vraie pour $n=1$ en posant $q_{0}=0$.

    \begin{itemize}
        \item Si $p_{1}=1$ et $p_{2}=0$, on a $q_{n}=q_{n-1}+p_{2}$ d'où 
        \begin{equation}
            \boxed{q_{n}=0}
        \end{equation}

        \item Si $(p_{1},p_{2})\neq(1,0)$, on a $p_{1}-p_{2}\neq1$ donc 
        \begin{align}
            q_{n}
            &=(p_{1}-p_{2})^{n}\times \frac{-p_{2}}{1-(p_{1}-p_{2})}+\frac{p_{2}}{1-(p_{1}-p_{2})}\\
            &=\boxed{\frac{p_{2}}{1-(p_{1}-p_{2})}\left(1-\left(p_{1}-p_{2}\right)^{n}\right)=\E(X_{n})}
        \end{align}

        \item Si $p_{1}-p_{2}=-1$, i.e.~$p_{1}=0$ et $p_{2}=1$, 
        \begin{equation}
            \boxed{q_{n}\text{ n'a pas de limite.}}
        \end{equation}

        \item Si $p_{1}-p_{2}\neq-1$,
        \begin{equation}
            \boxed{q_{n}\xrightarrow[n\to+\infty]{}\frac{p_{2}}{1-(p_{2}-p_{1})}}
        \end{equation}
    \end{itemize}
\end{proof}

\begin{proof}
    \phantom{}
    \begin{enumerate}
        \item Pour tout $k\in\llbracket1,6\rrbracket$, on veut
        \begin{equation}
            \P(X\leqslant k)=\P\left(\left(D_{1}\leqslant k\right)\cap\left(D_{2}\leqslant k\right)\right)=\P(D_{1}\leqslant k)\P(D_{2}\leqslant k)=\frac{k^{2}}{36}
        \end{equation}

        Or on a (avec $P(X\leqslant 0)=0$)
        \begin{equation}
            P(X=k)=P(X\leqslant k)-\P(X\leqslant k-1)=\frac{2k-1}{36}
        \end{equation}
        De même, on a $P(Y\geqslant k)=\frac{(7-k)^{2}}{36}$ donc
        \begin{equation}
            \P(Y=k)=\frac{13-2k}{36}
        \end{equation}
        A chaque fois, on vérifie que $\sum_{k=1}^{6}\P(X=k)=\sum_{k=1}^{6}\P(Y=k)=1$.

        Pour les calculs de variance et d'espérance, on calcule $\sum_{k=1}^{6}k\P(X=k)$ et 
        \begin{equation*}
            \sum_{k=1}^{6}k^{2}\P(X=k)-\left(\sum_{k=1}^{6}k\P(X=k)\right),
        \end{equation*}de même pour $Y$.

        \item Soit $(i,j)\in\llbracket1,6\rrbracket^{2}$, si $i<j$ on a $\P((X=i)\cap (Y=j))=0$ mais $P(X=i)\P(Y=j)\neq0$, on n'a donc pas indépendance.
        
        \item Si $P(D_{i}=k)=p_{k,i}$, on a 
        \begin{equation}
            \P(X\leqslant k)=\left(\sum_{l=1}^{k}p_{l,1}\right)\left(\sum_{l=1}^{k}p_{l,2}\right)=\sum_{1\leqslant l,r\leqslant k}p_{l,1}\times p_{r,2}
        \end{equation}
        et on calcule ensuite $P(X=k)=\P(X\leqslant k)-\P(X\leqslant k-1)$ et cela vaut ce que cela vaut.
    \end{enumerate}
\end{proof}

\begin{proof}
    \phantom{}
    \begin{enumerate}
        \item On a 
        \begin{align}
            \sum_{(i,j)\in\N^{2}}
            &=\sum_{i=0}^{+\infty}\sum_{j=0}^{i}\frac{a^{j}(1-a)^{i-j}}{j!(i-j)!}(b^{i}e^{-b})\\
            &=\sum_{i=0}^{+\infty}\frac{b^{i}e^{-b}}{i!}\sum_{j=0}^{i}\binom{i}{j}a^{j}(1-a)^{i-j}\\
            &=\sum_{i=0}^{+\infty}\frac{b^{i}e^{-b}}{i!}\\
            &=e^{b}e^{-b}\\
            &=1
        \end{align}
        donc la définition est cohérente.

        \item On a 
        \begin{equation}
            \boxed{p_{i,\cdot}=\sum_{j=0}^{i}p_{i,j}=\frac{b^{i}e^{-b}}{i!}}
        \end{equation}
        et 
        \begin{align}
            p_{\cdot,j}
            &=\sum_{i=j}^{+\infty}\frac{e^{-b}a^{j}}{j!}\left(\frac{b^{i}(1-a)^{i-j}}{(i-j)!}\right)\\
            &=\frac{e^{-b}a^{j}b^{j}}{j!}\sum_{j=0}^{+\infty}\frac{b^{i}(1-a)^{i}}{i!}\\
            &=\boxed{\frac{e^{-ab}(ab)^{j}}{j!}}
        \end{align}
        On a $p_{i,j}\neq p_{i,\cdot}\neq p_{\cdot,j}$ donc les variables ne sont pas indépendantes.

        \item $Z$ est à valeurs dans $\N$ (car $p_{i,j}=0$ si $i<j$). On a 
        \begin{align}
            \P(X-Y=k)
            &=\sum_{j=0}^{+\infty}\P(X=j+k,Y=j)\\
            &=\sum_{j=0}^{+\infty}\frac{b^{j+k}e^{-b}a^{j}(1-a)^{k}}{j!k!}\\
            &=\frac{e^{-b}b^{k}(1-a)^{k}}{k!}\sum_{j=0}^{+\infty}\frac{(ba)^{j}}{j!}\\
            &=\boxed{\frac{e^{b(a-1)}(b(1-a))^{k}}{k!}}
        \end{align}

        De plus,
        \begin{equation}
            \P(Z=k,Y=j)=\P((X,Y)=(k+j,j))=p_{k+j,j}=\frac{b^{j+k}e^{-b}a^{j}(1-a)^{k}}{j!k!}
        \end{equation}
        et
        \begin{align}
            \P(Z=k)\P(Y=j))
            &=\frac{e^{b(a-1)b^{k}(1-a)^{k}}}{k!}\frac{e^{-ab}(ab^{j})}{j!}\\
            &=\frac{e^{-b}b^{k+j}a^{j}(1-a)^{k}}{k!j!}
        \end{align}
        donc $Z$ et $Y$ sont indépendantes.
    \end{enumerate}
\end{proof}

\begin{remark}
    On a $X\sim \mathcal{P}(b)$ et $Y\sim\mathcal{P}(ab)$ donc $X$ et $Y$ ont des espérances.
\end{remark}

\begin{proof}
    \phantom{}
    \begin{enumerate}
        \item On a $S_{n}-S_{n-1}=T_{n}$ pour tout $n\geqslant2$, donc 
        \begin{equation}
            \boxed{S_{n}=\sum_{i=1}^{n}T_{i}}
        \end{equation}

        \item Pour tout $k\in\N^{*}$, comme $T_{n}\sim\mathcal{G}(1-x)$, on a 
        \begin{equation}
            \boxed{\P(T_{n}=k)=x^{k-1}(1-x)}
        \end{equation}
        et 
        \begin{equation}
            \boxed{\E(T_{n})=\sum_{k=1}^{+\infty}kx^{k-1}(1-x)=\frac{1}{1-x}}
        \end{equation}
        et 
        \begin{equation}
            \boxed{\mathbb{V}(T_{n})=\frac{x}{(1-x)^{2}}}
        \end{equation}

        \item On a
        \begin{equation}
            \boxed{\E(S_{n})=\sum_{i=1}^{n}\E(T_{i})=\frac{n}{1-x}}
        \end{equation}
        Comme les $(T_{i})_{1\leqslant i\leqslant n}$ sont indépendants, on a 
        \begin{equation}
            \boxed{\mathbb{V}(S_{n})=\sum_{i=1}^{n}\mathbb{V}(T_{i})=\frac{nx}{(1-x)^{2}}}
        \end{equation}

        Pour $k<n$, on a $\P(S_{n}=k)=0$ et sinon, on a 
        \begin{equation}
            \P(S_{n}=k)=\binom{k-1}{n-1}(1-x)^{n}x^{k-n}
        \end{equation}
        (choisir les $n-1$ succès parmi $k-1$ épreuves).

        \item On a $\sum_{k=n}^{+\infty}\P(S_{n}=k)=1$ donc 
        \begin{equation}
            \boxed{\sum_{k=n}^{+\infty}\binom{k-1}{n-1}x^{k}=\frac{x^{n}}{(1-x)^{n}}}
        \end{equation}
    \end{enumerate}
\end{proof}

\begin{proof}
    On a $\P(X>n)=\sum_{k=n+1}^{+\infty}$. On pose 
    \begin{equation}
        u_{k,n}=
        \left\lbrace
            \begin{array}[]{l}
                \P(X=k)\text{ si k>n}\\
                0 \text{ sinon}
            \end{array}
        \right.
    \end{equation}

    Alors $\sum_{n\in\N}\P(X>n)$ converge si et seulement si $(u_{k,n})_{(k,n)\in\N^{2}}$ est sommable ($u_{k,n}\geqslant0$) si et seulement si $\sum_{k=1}^{+\infty}\sum_{n=0}^{+\infty}u_{k,n}$ converge (théorème de Fubini). Or 
    \begin{equation}
        \sum_{k=1}^{+\infty}\sum_{n=0}^{+\infty}u_{k,n}=\sum_{k=1}^{+\infty}\sum_{n=0}^{k-1}\P(X=k)=\sum_{k=1}^{+\infty}k\P(X=k)
    \end{equation}
\end{proof}

\begin{proof}
    On cherche $\sup\limits_{k\in\N}\left(\frac{\lambda^{k}}{k!}e^{-\lambda}\right)=\sup\limits_{k\in\N}u_{k}$ avec $u_{k}>$. On a 
    \begin{equation}
        \frac{u_{k+1}}{u_{k}}=\frac{\lambda}{k+1}\geqslant 1 
    \end{equation}
    si et seulement si $k\leqslant\lambda-1$. On a donc $u_{k}\leqslant u_{k+1}$ si et seulement si $k\leqslant \left\lfloor\lambda\right\rfloor-1$ et le maximum est donc atteint pour $k=\left\lfloor\lambda\right\rfloor$.

    Si $\lambda=n\in\N^{*}$, le maximum vaut 
    \begin{equation}
        \frac{n^{n}e^{-n}}{n!}\underset{n\to+\infty}{\sim}\frac{1}{\sqrt{2\pi n}}
    \end{equation}
\end{proof}

\begin{proof}
    \phantom{}
    \begin{enumerate}
        \item On a 
        \begin{equation}
            \P(X=k)=\frac{9}{10}\times\frac{8}{9}\times\dots\times\frac{10-(k-1)}{10-(k-2)}\times\frac{1}{10-(k-1)}=\frac{1}{10}
        \end{equation}
        donc $X\sim\mathcal{U}\left(\llbracket1,10\rrbracket\right)$ et $Y\sim\mathcal{G}\left(\frac{1}{10}\right)$ donc $\E(X)=\frac{11}{2}$, $\mathbb{V}(X)=\frac{10^{2}-1}{12}=\frac{33}{4}$, $\E(Y)=10$, $\mathbb{V}(Y)=\frac{9}{10}\times100=90$.

        \item Soit $S$ l'événement `le gardien est sobre' et $Z$ compte le nombre d'essais au bout desquels il a réussi. Alors 
        \begin{equation}
            \P_{Z\geqslant 9}(5)=\frac{\P(S)\P_{S}(Z\geqslant9)}{\P(Z\geqslant9)}=\frac{\P(S)\P(X\geqslant9)}{\frac{1}{3}\P(Y\geqslant9)+\frac{2}{3}\P(X\geqslant9)}
        \end{equation}
        On a $\P(X\geqslant9)=\frac{1}{5}$ et 
        \begin{equation}
            \P(Y\geqslant9)=\sum_{n=9}^{+\infty}\left(\frac{9}{10}\right)^{n-1}\times\frac{1}{10}=\left(\frac{9}{10}\right)^{8}
        \end{equation}
        d'où 
        \begin{equation}
            \boxed{\P_{Z\geqslant9}(5)=\frac{2}{5\times\left(\frac{9}{10}\right)^{8}+2}<\frac{1}{2}}
        \end{equation}
    \end{enumerate}
\end{proof}

\begin{proof}
    \phantom{}
    \begin{enumerate}
        \item On a 
        \begin{equation}
            \boxed{N=\frac{n(n+1)}{2}}
        \end{equation}

        \item Soit $(i,j)\in\llbracket1,n\rrbracket^{2}$, si $i<j$ on a $p_{i,j}=0$ (où $p_{i,j}$ est la loi conjointe). Si $j\leqslant i$, on a
        \begin{equation}
            p_{i,j}=\frac{1}{N}=\frac{2}{n(n+1)}
        \end{equation}

        On a ensuite 
        \begin{equation}
            \boxed{p_{i,\cdot}=\sum_{j=1}^{i}\frac{2}{n(n+1)}}=\frac{2i}{n(n+1)}
        \end{equation}
        et 
        \begin{equation}
            \boxed{p_{j,\cdot}}=\sum_{i=j}^{n}\frac{2}{n(n+1)}=\frac{2(n-j+1)}{n(n+1)}
        \end{equation}

        \item On calcule 
        \begin{equation}
            \boxed{
                \begin{array}[]{l}
                    \E(B)=\sum_{i=1}^{n}p_{i,\cdot}=\frac{2}{n(n+1)}\sum_{i=1}^{n}i^{2}=\frac{2n+1}{3}\\
                    \E(R)=\sum_{j=1}^{n}jp_{\cdot,j}=\frac{2}{n(n+1)}\sum_{j=1}^{n}j(n+1-j)=\frac{n(n+1)^{2}}{2}-\frac{n(n+1)(2n+1)}{6}
                \end{array}
            }
        \end{equation}

        On laisse le reste en calcul facile en utilisant $\mathbb{V}(G)=\mathbb{V}(R)+\mathbb{V}(B)-2\mathrm{cov}(B,R)$ et 
        \begin{equation}
            \E(BR)=\sum_{(i,j)\in\llbracket1,n\rrbracket^{2}}ijp_{i,j}
        \end{equation}
    \end{enumerate}
\end{proof}

\begin{proof}
    \phantom{}
    \begin{enumerate}
        \item On écrit 
        \begin{equation}
            \boxed{\P(X_{n+1}=k)=\P_{X_{n}=k-1}(X_{n+1}=k)\P(X_{n}=k-1)=\frac{k}{k+1}\P(X_{n}=k)}
        \end{equation}

        \item On a $\P(X_{n}=k)=0$ si $k>n$, sinon on écrit 
        \begin{align}
            \P(X_{n}=k)
            &=\frac{k}{k+1}\P(X_{n-1}=k-1)\\
            &=\frac{k}{k+1}\times\frac{k-1}{k}\times\frac{k-2}{k-1}\times\dots\times\frac{2}{3}\times\frac{1}{2}\times u_{n-k}\\
            &=\boxed{\frac{1}{k+1}u_{n-k}}
        \end{align}

        \item On a $\sum_{j=0}^{n}\P(X_{n}=j)=1$ donc 
        \begin{equation}
            \boxed{\sum_{j=0}^{n}\frac{u_{n}}{n-j+1}=1}
        \end{equation}
        et on a $u_{0}=1,u_{1}=\frac{1}{2},u_{2}=\frac{5}{12},u_{3}=\frac{3}{8}$ (en utilisant la formule précédente).

        \item On écrit 
        \begin{equation}
            (k+1)\P(X_{n+1}=k)=k\P(X_{n}=k-1)
        \end{equation}
        donc pour tout $k\in\llbracket1,n+1\rrbracket$, 
        \begin{equation}
            k\P(X_{n+1}=k)=(k-1)\P(X_{n}=k-1)+\P(X_{n}=k-1)-\P(X_{n+1}=k)
        \end{equation}
        En sommant sur $k\in\llbracket1,n+1\rrbracket$, on trouve donc 
        \begin{equation}
            \boxed{\E(X_{n+1})=\E(X_{n})+1-(1-u_{n+1})=\E(X_{n})+u_{n+1}}
        \end{equation}
        Par récurrence, on a directement 
        \begin{equation}
            \boxed{\E(X_{n})=u_{n}+\dots+u_{1}+\underbrace{\E(X_{0})}_{{=~0}}}
        \end{equation}

        \item On écrit 
        \begin{align}
            \P(T=n)
            &=\P\left(\left(X_{0}=0\right)\cap\left(X_{1}=1\right)\cap\dots\cap\left(X_{n-1}=n-1\right)\cap\left(X_{n}=0\right)\right)\\
            &=\P(X_{0}=0)\times\P_{X_{0}=0}(X_{1}=1)\times\dots\times\P_{\left(X_{0}=0\right)\cap\dots\cap\left(X_{n-1}=n-1\right)}(X_{n=0})\\
            &=\P(X_{0}=0)\times\P_{X_{0}=0}(X_{1}=1)\times\dots\times\P_{\left(X_{n-1}=n-1\right)}(X_{n=0})\\
            &=1\times\frac{1}{2}\times\frac{2}{3}\times\dots\times\frac{n-1}{n}\times\frac{1}{n+1}\\
            &=\boxed{\frac{1}{n(n+1)}}
        \end{align}

        Comme $\frac{1}{n(n+1)}=\frac{1}{n}-\frac{1}{n+1}$, on a 
        \begin{equation}
            \sum_{n=1}^{+\infty}\P(T=n)=1
        \end{equation}
        donc 
        \begin{equation}
            \boxed{\P(T=0)=0}
        \end{equation}
        Donc le retour en temps fini à l'origine est presque sûr.

        \item Non au vu de la formule donnée par $\P(T=n)$.
    \end{enumerate}
\end{proof}

\begin{proof}
    \phantom{}
    \begin{enumerate}
        \item Soit $k\in\llbracket0,n\rrbracket$, on a 
        \begin{equation}
            \boxed{\P_{N=n}(X_{1}=k)=\binom{n}{k}\left(\frac{1}{m}\right)^{k}\left(1-\frac{1}{m}\right)^{n-k}}
        \end{equation}
        (loi binomiale, car les $m$ caisses sont équiprobables).

        \item On a $\P_{N=n}(X_{1}=k)=0$ si $k>n$ donc si $k\in\N$,
        \begin{align}
            \P_{X_{1}=k}
            &=\sum_{n=0}^{+\infty}\P_{N=n}(X_{1}=k)\P(N=n)\\
            &=\sum_{n=k}\binom{n}{k}\left(\frac{1}{m}\right)^{k}\left(1-\frac{1}{m}\right)^{n-k}\frac{\lambda^{n}}{n!}e^{-\lambda}\\
            &=\frac{1}{k!}e^{-\lambda}\left(\frac{1}{m}\right)^{k}\lambda^{k}\sum_{n=k}^{+\infty}\lambda^{n-k}\left(1-\frac{1}{m}\right)^{n-k}\frac{1}{(n-k)!}
        \end{align}
        On reconnaît la série exponentielle, après un changement d'indice, appliquée en $\lambda\left(1-\frac{1}{m}\right)$. Ainsi,
        \begin{equation}
            \boxed{\P(X_{1}=k)=e^{-\frac{\lambda}{n}}\frac{\left(\frac{\lambda}{n}\right)^{k}}{k!}}
        \end{equation}
        et donc $X_{1}\sim\mathcal{P}\left(\frac{\lambda}{n}\right)$.
    \end{enumerate}
\end{proof}

\begin{proof}
    \phantom{}
    \begin{enumerate}
        \item Si $(i,j)\in\lbrace0,1,2\rbrace\times\lbrace-1,0,1\rbrace$, on a 
        \begin{equation}
            \boxed{\P\left((U,V)=(i,j)\right)=\P\left(X=\frac{i+j}{2},Y=\frac{-j}{2}\right)}
        \end{equation}
        Cette probabilité vaut 0 si $i$ et $j$ n'ont pas la même parité. Sinon, on a 
        \begin{equation}
            \boxed{
                \begin{array}[]{l}
                    \P\left((U,V)=(0,0)\right)=\P(X=0,Y=0)=q^{2}\\
                    \P\left((U,V)=(1,-11)\right)=\P(X=0,Y=1)=qp\\
                    \P\left((U,V)=(1,1)\right)=qp\\
                    \P\left((U,V)=(2,0)\right)=\P(X=1,Y=1)=p^{2}\\
                \end{array}
            }
        \end{equation}

        \item On a 
        \begin{align}
            \mathrm{cov}(U,V)
            &=\E\left(\left(U-\E(U)\right)\left(V-\E(V)\right)\right)\\
            &=\E(UV)-\E(U)\E(V)\\
            &=\E(X^{2}-Y^{2})-\left[\left(\E(X)+\E(Y)\right)\left(\E(X)-\E(Y)\right)\right]\\
            &=\E(X^{2}-Y^{2})-\left[\E(X)^{2}-\E(Y)^{2}\right]\\
            &=\mathbb{V}(X)-\mathbb{V}(Y)\\
            &=0
        \end{align}

        \item Les variables $U$ et $V$ ne sont pas indépendantes, il suffit de voir que 
        \begin{equation}
            \boxed{\P\left((U,V)=(1,0)\right)=0\neq\P(U=1)\P(V=0)=2pq\times\left(q^{2}+p^{2}\right)}
        \end{equation}
    \end{enumerate}
\end{proof}

\begin{proof}
    \phantom{}
    \begin{enumerate}
        \item $P\sim\mathcal{G}(p)$ et $F\sim\mathcal{G}(q)$ donc 
        \begin{equation}
            \boxed{
                \begin{array}[]{l}
                    \E(P)=\frac{1}{p}\\
                    \mathbb{V}(P)=\frac{q}{p^{2}}\\
                    \E(F)=\frac{1}{q}\\
                    \mathbb{V}(F)=\frac{p}{q^{2}}
                \end{array}
            }
        \end{equation}

        \item On a $\P\left((P=1)\cap(F=1)\right)=0\neq\P(P=1)\P(F=1)$ donc $P$ et $F$ ne sont pas indépendantes.
        
        \item Soit $(i,j)\in\left(\N^{*}\right)^{2}$, on note $p_{i,j}=\P(X=i,Y=j)$. On partitionne selon si $P=1$ ou $F=1$ et donc 
        \begin{equation}
            \boxed{p_{i,j}=p^{i+1}q^{j}+q^{i+1}p^{j}}
        \end{equation}
        On note 
        \begin{equation}
            \begin{array}[]{l}
                p_{i,\cdot}=\sum_{j=1}^{+\infty}p_{i,j}=p^{i+1}\frac{q}{1-q}+q^{i+1}\frac{p}{1-p}=p^{i}q+q^{i}p\\
                p_{\cdot,j}=\sum_{i=1}^{+\infty}p_{i,j}=\frac{p^{2}}{1-p}q^{j}+\frac{q^{2}}{1-q}p^{j}=q^{j-1}p^{2}+p^{j-1}q^{2}
            \end{array}
        \end{equation}

        De plus, on a $p_{1,1}=p^{2}q+q^{2}p=pq$ et $p_{1,\cdot}\times p_{\cdot,1}=(pq+qp)(p^{2}+q^{2})=2pq(p^{2}+q^{2})$ dp,c si $X$ et $Y$ sont indépendantes, on a $1=2(p^{2}+q^{2})$ d'où $p=\frac{1}{2}$. Réciproquement, si $p=\frac{1}{2}$, on a $p_{i,\cdot}=\frac{1}{2^{i}}$, $p_{\cdot, j}=\frac{1}{2^{j}}$ et $p_{i,j}=\frac{1}{2^{i+j}}=p_{i,\cdot}p_{\cdot,j}$. Ainsi, $X$ et $Y$ sont indépendantes si et seulement si $p=\frac{1}{2}$.

        \item On a 
        \begin{equation}
            \E(X)=\sum_{i=1}^{+\infty}ip_{i,\cdot}=q\sum_{i=1}^{+\infty}ip^{i}+p\sum_{i=1}^{+\infty}iq^{i}
        \end{equation}
        On utilise alors le fait que $\sum_{n=1}^{+\infty}nz^{n-1}=\frac{1}{(1-z)^{2}}$ si $\left\lvert z\right\rvert<1$ et donc 
        \begin{equation}
            \boxed{\E(X)=\frac{p^{2}+q^{2}}{pq}\geqslant2}
        \end{equation}
        car $(p-q)^{2}\geqslant0$.

        \item On a 
        \begin{equation}
            \boxed{\P(X=Y)=\sum_{i=1}^{+\infty}p_{i,i}=\sum_{i=1}^{+\infty}p^{i}q^{i}(p+q)=pq\times\frac{1}{1-pq}}
        \end{equation}

        \item Si $p=\frac{1}{2}$, $X$ et $Y$ sont indépendantes, donc par convolution,
        \begin{equation}
            \boxed{\P(X+Y=k)=\sum_{i=1}^{k-1}\P(X=i)\P(Y=k-i)=\sum_{i=1}^{k-1}p_{i,\cdot}p_{\cdot,k-i}=\sum_{i=1}^{k-1}\frac{1}{2^{k}}=\frac{k-1}{2^{k}}}
        \end{equation}
    \end{enumerate}
\end{proof}

\begin{proof}
    \phantom{}
    \begin{enumerate}
        \item On a 
        \begin{equation}
            \boxed{\e^{\lambda x}=\sum_{k=0}^{+\infty}\frac{(\lambda x)^{2k}}{(2k)!}+\sum_{k=0}^{+\infty}\frac{(\lambda x)^{2k+1}}{(2k+1)!}\leqslant\sum_{k=0}^{+\infty}\frac{\lambda^{2k}}{2^{k}k!}+x\sinh(\lambda)=\e^{\frac{\lambda^{2}}{2}}+x\sinh(\lambda)}
        \end{equation}
        car $\left\lvert x^{2}\right\rvert\leqslant1$ et pour tout $k\in\N^{*}$, $(2k)!\geqslant2^{k}k!$ (par récurrence).

        \item $\e^{\lambda X}$ admet une espérance car $\left\lvert\e^{\lambda X}\right\rvert\leqslant \e^{\lambda}$. Comme $X$ est centrée, on a d'après l'inégalité précédente, en prenant l'espérance,
        \begin{equation}
            \boxed{\E\left(\e^{\lambda X}\right)\leqslant\E\left(e^{\frac{\lambda^{2}}{2}}\right)+\sinh(\lambda)\E(X)=\e^{\frac{\lambda^{2}}{2}}}
        \end{equation}
        En appliquant l'inégalité à $-X$, on a l'autre inégalité.

        \item Grâce à l'inégalité de Markov, on en déduit que 
        \begin{equation}
            \boxed{\P(X\geqslant a)=\P\left(\e^{\lambda X}\geqslant \e^{\lambda a}\right)}\leqslant \frac{\E\left(\e^{\lambda X}\right)}{e^{\lambda a}}=\e^{-\lambda a}\E\left(\e^{\lambda X}\right)
        \end{equation}

        \item On pose $X=\frac{1}{n}\sum_{k=1}^{n}X_{i}$. $X$ est centrée dans $[-1,1]$ ainsi que $-X$. On a donc, pour tout $\lambda\geqslant0$,
        \begin{equation}
            \P\left(\left\lvert X\right\rvert\geqslant a\right)=\P(X\geqslant a)+\P(-X\geqslant a)\leqslant 2\e^{-\lambda a}\e^{\frac{\lambda^{2}}{2}}
        \end{equation}
        On optimise ensuite cette inégalité en $\lambda\geqslant0$ (le minimum est en $\lambda=a$) et on a bien 
        \begin{equation}
            \boxed{\P\left(\left\lvert\frac{1}{n}\sum_{i=1}^{n}X_{i}\right\rvert\geqslant a\right)\leqslant 2\e^{-\frac{a^{2}}{2}}}
        \end{equation}
    \end{enumerate}
\end{proof}

\begin{proof}
    \phantom{}
    \begin{enumerate}
        \item Comme $\E(Y)<+\infty$, on a d'après le théorème de Fubini et le fait que $\left(X=l\right)_{l\in\N}$ est un système complet d'événements:
        \begin{equation}
            \E(Y)=\sum_{l=0}^{+\infty}l\P(Y=l)\sum_{l=0}^{+\infty}\sum_{k=0}^{+\infty}\P_{(X=k)}(Y=l)\P(X=k)=\sum_{k=0}^{+\infty}\E_{(X=k)}(Y)\P(X=k).
        \end{equation}

        \item Pour $\lambda=X_{n+1}$ et $X=X_{n}$, et en utilisant le fait que les poules sont indépendantes, on a 
        \begin{equation}
            \E(X_{n+1})=\sum_{k=0}^{+\infty}\E_{(X_{n}=k)}\P(X_{n}=k)=\sum_{k=0}^{+\infty}k\lambda\P(X_{n}=k)=\lambda\E(X_{n})
        \end{equation}
        Par récurrence, on a 
        \begin{equation}
            \boxed{\E(X_{n})=\lambda^{n}\E(X_{0})=\lambda^{n}N}
        \end{equation}

        On note que si $\lambda>1$, $\E(X_{n})\xrightarrow[n\to+\infty]{}+\infty$ donc la descendance est assurée. Si $\lambda<1$, on a $\E(X_{n})\xrightarrow[n\to+\infty]{}0$.
    \end{enumerate}
\end{proof}

\begin{proof}
    Soit $k\in\N$, on a 
    \begin{align}
        \P(K=k)
        &=\sum_{n=0}^{+\infty}\P_{(N=n)}(K=k)\P(N=n)\\
        &=\sum_{n=k}^{+\infty}\binom{n}{k}p^{k}(1-p)^{n-k}\e^{-\lambda}\frac{\lambda^{n}}{n!}\\
        &=\e^{-\lambda p}\frac{\left(\lambda p\right)^{k}}{k!}
    \end{align}

    Donc $K\sim\mathcal{P}(\lambda p)$ et 
    \begin{equation}
        \boxed{\E(K)=\lambda p}
    \end{equation}
\end{proof}

\begin{proof}
    \phantom{}
    \begin{enumerate}
        \item On a $\chi_{A_{k}}\sim\mathcal{B}\left(\frac{1}{k}\right)$. Ainsi,
        \begin{equation}
            \boxed{\E(S_{n})=\sum_{k=1}^{n}\frac{1}{k}\underset{n\to+\infty}{\sim}\ln(n)}
        \end{equation}

        Comme les $(A_{n})_{n\geqslant1}$ sont indépendants, on a aussi 
        \begin{equation}
            \boxed{\mathbb{V}(S_{n})=\sum_{k=1}^{n}\mathbb{V}(\chi_{A_{k}})=\sum_{k=1}^{n}\frac{1}{k}\left(1-\frac{1}{k}\right)\underset{n\to+\infty}{\sim}\ln(n)}
        \end{equation}

        \item Soit $X_{n}=\frac{S_{n}}{\ln(n)}$. On a $\E(X_{n})\xrightarrow[n\to+\infty]{}1$ et $\mathbb{V}(X_{n})=\frac{1}{\ln^{2}(n)}\mathbb{V}(S_{n})\xrightarrow[n\to+\infty]{}0$.
        
        D'après l'inégalité de Bienaymé-Tchebychev,
        \begin{equation}
            \P\left(\left\lvert X_{n}-\E(X_{n})\right\rvert\geqslant\frac{\varepsilon}{2}\right)\leqslant4\frac{\mathbb{V}(X_{n})}{\varepsilon^{2}}\xrightarrow[n\to+\infty]{}0
        \end{equation}

        Or, si $\left\lvert X_{n}-\E(X_{n})\right\rvert<\frac{\varepsilon}{2}$ et $\left\lvert\E(X_{n})-1\right\rvert<\frac{\varepsilon}{2}$, alors $\left\lvert X_{n}-1\right\rvert<\varepsilon$.
        Par contraposée, si $\left\lvert X_{n}-1\right\rvert\geqslant\varepsilon$, alors ou bien $\left\lvert X_{n}-\E(X_{n})\right\rvert\geqslant\frac{\varepsilon}{2}$ ou $\left\lvert\E(X_{n})-1\right\rvert\geqslant\frac{\varepsilon}{2}$. Ainsi,
        \begin{equation}
            \P\left(\left\lvert X_{n}-1\right\rvert\geqslant\varepsilon\right)\leqslant4\frac{\mathbb{V}(X_{n})}{\varepsilon^{2}}+\P\left(\left\lvert\E(X_{n})-1\right\rvert\geqslant\frac{\varepsilon}{2}\right)
        \end{equation}
        A partir d'un certain rang $N_{0}\in\N$, on a $\left\lvert\E(X_{n})-1\right\rvert<\frac{\varepsilon}{2}$, donc pour tout $n\geqslant N_{0}$, on a $\P\left(\left\lvert\E(X_{n})-1\right\rvert\geqslant\frac{\varepsilon}{2}\right)=0$. Ainsi,
        \begin{equation}
            \boxed{\lim\limits_{n\to+\infty}\P\left(\left\lvert\frac{S_{n}}{\ln(n)}-1\right\rvert\geqslant\varepsilon\right)=0}
        \end{equation}
    \end{enumerate}
\end{proof}

\begin{proof}
    \phantom{}
    \begin{enumerate}
        \item Soit $k\in\N^{*}$. On a 
        \begin{equation}
            \P(U\geqslant k)=\P\left(\bigcap_{i=1}^{n}\left(X_{i}\geqslant k\right)\right)=\prod_{i=1}^{k}\P\left(X_{i}\geqslant k\right)=\prod_{i=1}^{n}\sum_{j=k}^{+\infty}q^{j-1}p=(q^{k-1})^{n}
        \end{equation}

        Ainsi, 
        \begin{equation}
            \boxed{\P(U=k)=\P(U\geqslant k)-\P(U\geqslant k+1)=q^{(k-1)n}(q^{n}-1)}
        \end{equation}

        $U$ possède une espérance car $0\leqslant U\leqslant X\sim\mathcal{G}(p)$ et on a 
        \begin{align}
            \E(U)
            &=\sum_{k=1}^{+\infty}k\P(U=k)\\
            &=\sum_{k=1}^{+\infty}k\left(\left(q^{n}\right)^{k-1}-\left(q^{n}\right)^{k}\right)\\
            &=\frac{1}{\left(1-q^{n}\right)^{2}}-\frac{q^{n}}{(1-q^{n})^{2}}\\
            &=\boxed{\frac{1}{1-q^{n}}}
        \end{align}
        où l'on a utilisé le fait que $\sum_{n=0}^{+\infty}nx^{n-1}=\frac{1}{(1-x)^{2}}$ si $\left\lvert x\right\rvert<1$.

        \item Soit $k\in\N^{*}$. On a $\P(V\leqslant k)=\left(1-q^{k}\right)^{n}$ donc 
        \begin{equation}
            \boxed{\P(V=k)=\left(1-q^{k}\right)^{n}-\left(1-q^{k-1}\right)^{n}\underset{n\to+\infty}{\sim}nq^{k-1}\left(1-q\right)\xrightarrow[k\to+\infty]{}0}
        \end{equation}

        Comme $k\P(V=k)=\underset{k\to+\infty}{O}\left(\frac{1}{k^{2}}\right)$, $V$ admet une espérance est 
        \begin{align}
            \E(V)
            &=\sum_{k=1}^{+\infty}k\P(V=k)\\
            &=\sum_{k=1}^{+\infty}k\left[\left(1-q^{k}\right)^{n}-\left(1-q^{k-1}\right)^{n}\right]\\
            &=\sum_{k=1}^{+\infty}k\sum_{i=1}^{n}\binom{n}{i}\left[(-1)^{i}q^{ki}-(-1)^{i}q^{(k-1)i}\right]\\
            &=\sum_{k=1}^{+\infty}k\sum_{i=1}^{n}(-1)^{i+1}\binom{n}{i}q^{(k-1)i}(1-q^{i})\\
            &=\sum_{k=1}^{+\infty}(-1)^{i+1}\binom{n}{i}(1-q^{i})\sum_{k=1}^{+\infty}k\left(q^{i}\right)^{k-1}\\
            &=\boxed{\sum_{i=1}^{n}\binom{n}{i}(-1)^{i+1}\frac{1}{1-q^{i}}}
        \end{align}
    \end{enumerate}
\end{proof}

\begin{proof}
    \phantom{}
    \begin{enumerate}
        \item $1-p^{N}$ correspond à la probabilité que le joueur perde au moins une partie sur $N$ consécutives donc c'est aussi la probabilité pour qu'il perde une partie entre la $nN+1$-ième et la $(n+1)N$-ième (inclus), car les parties sont indépendantes. On note $A_{nN+1,(n+1)N}$:`le joueur perd une partie entre la $nN+1$-ième et la $(n+1)N$-ième (au sens large)'. $\lbrace t_{k}>nN\rbrace$ et $A_{nN+1,(n+1)N}$ sont des événements indépendants car les différentes parties sont indépendantes. Ainsi,
        \begin{equation}
            \boxed{\P\left(t_{k}>nN\right)(1-p^{N})=\P\left(\left(t_{k}>nN\right)\cap A_{nN+1,(n+1)N}\right)\geqslant\P(t_{k}>n(N+1))}
        \end{equation}
        car s'il avait gagné toutes les parties entre $nN+1$ et $(n+1)N$ on aurait $t_{k}\leqslant n(N+1)$.

        On sait que si $X$ est une variable aléatoire discrète à valeurs dans $\N^{*}$, $X$ possède une espérance finie si et seulement si $\sum_{k\in\N}\P(X>k)$ converge et on a (théorème de Fubini) $\E(X)=\sum_{k\in\N}\P(X>k)$.

        On a donc 
        \begin{equation}
            \P(t_{k}>Nn)\leqslant (1-p^{n})\P(t_{k}>0)=1-p^{n}
        \end{equation}
        par récurrence sur $n$ d'après 1. Pour $l\in\N$, soit $n=\left\lfloor\frac{l}{N}\right\rfloor$, on a $nN\leqslant l<(n+1)N$ donc 
        \begin{equation}
            \P(t_{k}>l)\leqslant\P(t_{k}>nN)\leqslant(1-p^{N})^{\left\lfloor\frac{l}{N}\right\rfloor}\leqslant\left(1-p^{N}\right)^{\frac{l}{N}}
        \end{equation}
        et le membre de droite est le terme général d'une série converge car $\left(1-p^{N}\right)^{\frac{1}{N}}<1$. Donc $t_{k}$ admet une espérance.

        \item On note $b_{i}$:`le joueur gagne au $i$-ième coup'. Alors 
        \begin{align}
            T_{k}
            &=\sum_{l=0}^{+\infty}l\P(t_{k}=l)\\
            &=\sum_{l=0}^{+\infty}l\left(\P_{b_{i}}(t_{k}=l)\P(b_{i})+\P_{\overline{b_{i}}}(t_{k}=l)\P(\overline{b_{i}})\right)\\
            &=p\sum_{l=1}^{+\infty}l\P(t_{k+1}=l-1)+q\sum_{l=1}^{+\infty}l\P(t_{k-1}=l-1)\\
            &=p\left(\sum_{l=1}^{+\infty}(l-1)\P(t_{k+1}=l-1)+\sum_{l=1}^{+\infty}1\times\P(t_{k+1}=l-1)\right)\notag\\
            &~~~+q\left(\sum_{l=1}^{+\infty}(l-1)\P(t_{k-1}=l-1)+\sum_{l=1}^{+\infty}1\times\P(t_{k-1}=l-1)\right)\\
            &=p\left(T_{k-1}+1\right)+q\left(T_{k+1}+1\right)
        \end{align}

        \item On a $qT_{k+1}-T_{k}+pT_{k-1}=-1$. Comme $q\alpha(k+1)-\alpha k+p\alpha(k-1)=1$ si et seulement si $\alpha=\frac{1}{1-2p}$, on pose $U_{k}=T_{k}-\frac{k}{1-2p}$ si $p=\frac{1}{2}$. Alors 
        \begin{equation}
            qU_{k+1}-U_{k}+pU_{k-1}=-1-q\frac{k+1}{1-2p}+\frac{k}{1-2p}-p\frac{k-1}{1-2p}=0
        \end{equation}
        car $p+q=1$.

        L'équation caractéristique est $qr^{2}-r+p=0$, les racines sont $1$ et $\frac{p}{q}$ (qui est différent de 1 car $p\neq \frac{1}{2}$). Donc 
        \begin{equation}
            q\left(1-\frac{p}{q}\right)(1-1)=0
        \end{equation}
        On a $T_{0}=T_{N}=0$ donc 
        \begin{equation}
            \boxed{T_{k}=\frac{1}{q-p}\left(k-N\left(\frac{1-\left(\frac{q}{p}\right)^{k}}{1-\left(\frac{q}{p}\right)^{N}}\right)\right)}
        \end{equation}
        si $p\neq\frac{1}{2}$. Si $p=\frac{1}{2}$, on a 
        \begin{equation}
            T_{k+1}-2T_{k}+T_{k-1}=-2    
        \end{equation}
        Ainsi, si $V_{k}=T_{k-1}-T_{k}$, on a 
        \begin{equation}
            V_{k+1}=-2+V_{k}
        \end{equation}
        On en déduit grâce aux conditions aux limites $T_{0}=T_{N}=0$ que 
        \begin{equation}
            \boxed{T_{k}=k(N-k)}
        \end{equation}
    \end{enumerate}
\end{proof}

\begin{proof}
    \phantom{}
    \begin{enumerate}
        \item On a 
        \begin{equation}
            \boxed{\sum_{n\geqslant1}\P(\lbrace n\rbrace)=\sum_{n\geqslant1}\frac{1}{\zeta(s)n^{s}}=1}
        \end{equation}

        \item On a 
        \begin{equation}
            \boxed{\P(A_{n})=\sum_{k\in A_{n}}\P(\lbrace k\rbrace)=\sum_{r=1}^{+\infty}\frac{1}{\zeta(s)r^{s}n^{s}}=\frac{1}{n^{s}}}
        \end{equation}

        \item Soient $p_{1},\dots,p_{k}$ des nombres premiers distincts. On a $A_{p_{1}}\cap\dots\cap A_{p_{r}}=A_{p_{1}\times\dots\times p_{k}}$ donc 
        \begin{equation}
            \boxed{\P\left(A_{p_{1}}\bigcap\dots\bigcap A_{p_{k}}\right)}=\frac{1}{(p_{1}\dots p_{k})^{s}}=\prod_{i=1}^{k}\frac{1}{p_{i}^{s}}=\prod_{i=1}^{k}\P(A_{p_{i}})
        \end{equation}
        donc les $(A_{p})_{p\in\mathcal{P}}$ sont indépendants.

        On remarque que l'on a $\lbrace1\rbrace=\cap_{p\in\mathcal{P}}\overline{A_{p}}$. On pose $p_{k}$ le $k$-ième nombre premier et $B_{k}=\cap_{i=1}^{k}\overline{A_{p_{i}}}$ qui est une suite décroissante d'événements. Comme les $(A_{p_{i}})_{i}$ sont indépendants, c'est aussi le cas des $(\overline{A_{p_{i}}})_{i}$, et on a donc 
        \begin{equation}
            \P(\lbrace1\rbrace)=\lim\limits_{k\to+\infty}\P(B_{k})=\lim\limits_{k\to+\infty}\prod_{i=1}^{k}\left(1-\P(A_{p_{i}})\right)=\prod_{i=}^{+\infty}\left(1-\frac{1}{p_{i}^{s}}\right)
        \end{equation}
        Or $\P(\lbrace1\rbrace)=\frac{1}{\zeta(s)}$ donc 
        \begin{equation}
            \boxed{\zeta(s)=\left(\prod_{i=1}^{+\infty}\left(1-\frac{1}{p_{i}^{s}}\right)\right)^{-1}}
        \end{equation}
    \end{enumerate}
\end{proof}

\begin{proof}
    On a $\P(E_{1})=\frac{1}{2},\P(E_{2})=\frac{1}{2}(1-b)$ et pour tout $n\geqslant2$,
    \begin{equation}
        \boxed{\P(E_{n})=\frac{1}{2}b^{n-1}(1-b)}
    \end{equation}

    Les événements $E_{n}$ sont incompatibles donc 
    \begin{equation}
        \boxed{\P\left(\bigcup_{n=1}^{+\infty}E_{n}\right)=\sum_{n=1}^{+\infty}\P(E_{n})=1}
    \end{equation}
    Il est donc presque sûr qu'on finisse par utiliser $A$.

    On a 
    \begin{equation}
        \boxed{\P(U_{n})=\frac{1}{2}a^{n}+\frac{1}{2}b^{n}}
    \end{equation}
    $(U_{n})_{n\in\N}$ forme une suite décroissante d'événements, donc 
    \begin{equation}
        \boxed{\P\left(\bigcap_{n=1}^{+\infty}U_{n}\right)=\lim\limits_{n\to+\infty}\P(U_{n})=0}
    \end{equation}
\end{proof}

\begin{proof}
    \phantom{}
    \begin{enumerate}
        \item On a 
        \begin{equation}
            \boxed{\P(A_{n})=\sum_{k=1}^{n}p(1-p)^{k-1}=1-(1-p)^{n}}
        \end{equation}

        \item $B_{n}=\bigcap_{i=1}^{N}A_{i,n}$ et les $A_{i,n}$ sont indépendants donc 
        \begin{equation}
            \boxed{\P(B_{n})=\prod_{i=1}^{N}\P(A_{i,n})=\left(1-\left(1-p\right)^{n}\right)^{N}}
        \end{equation}

        \item On note $C_{n}=B_{n}\setminus B_{n-1}$ et $B_{n-1}\subset B_{n}$ donc 
        \begin{equation}
            \boxed{\P(C_{n})=\P(B_{n})-\P(B_{n-1})=\left(1-\left(1-p\right)^{n}\right)^{N}-\left(1-\left(1-p\right)^{n-1}\right)^{N}}
        \end{equation}
    \end{enumerate}
\end{proof}

\begin{proof}
    \phantom{}
    \begin{enumerate}
        \item On a 
        \begin{equation}
            \K_{<n}[X]=\left\lbrace a_{0}+\dots+a_{n-1}X^{n-1}\middle|(a_{0},\dots,a_{n-1})\in\K^{n}\right\rbrace    
        \end{equation}
        donc $\left\lvert\K_{<n}[X]\right\rvert=p^{n}$. De même, on a 
        \begin{equation}
            \K_{=n}[X]=\left\lbrace a_{0}+\dots+a_{n-1}X^{n-1}+a_{n}X^{n}\middle|(a_{0},\dots,a_{n})\in\K^{n},a_{n}\neq\overline{0}\right\rbrace
        \end{equation}
        donc $\left\lvert\K_{=n}[X]\right\rvert=p^{n}(p-1)$ d'où $\left\lvert\Omega\right\rvert=p^{2n}(p-1)$.

        On a $\P\left(\deg(Q)=-\infty\right)=\frac{1}{p^{n}}$ et si $k\in\llbracket0,n-1\rrbracket$,
        \begin{equation}
            \boxed{\P\left(\deg(Q)=k\right)=\frac{p^{k}(p-1)}{p^{n}}}
        \end{equation}

        \item On a $(Q,P)\in A$ si et seulement si $Q\mid P$ si et seulement si il existe $A\in\K_{\leqslant n}[X]$ tel que $P=AQ$ et $\deg(A)+\deg(Q)=n$. Ainsi, $\left(Q,\frac{P}{Q}\right)\in B$ et $f$ est bien définie. On a directement
        \function{f^{-1}}{B}{A}{(Q,A)}{(Q,AQ)}
        donc $f$ est bijective et $\left\lvert A\right\rvert =\left\lvert B\right\rvert$.

        On a 
        \begin{equation}
            B=\bigcup_{k=0}^{n-1}\left\lbrace(Q,A)\in\K_{=k}[X]\times\K_{=n-k}[X]\right\rbrace
        \end{equation}
        donc 
        \begin{equation}
            \left\lvert B\right\rvert=\sum_{k=0}^{n-1}p^{k}(p-1)\times p^{n-k}(p-1)=np^{n}(p-1)^{2}=\left\lvert A\right\rvert
        \end{equation}

        Ainsi, 
        \begin{equation}
            \boxed{\P(Q\mid P)=\frac{np^{n}(p-1)^{2}}{p^{2n}(p-1)}=\frac{n(p-1)}{p^{n}}}
        \end{equation}

        \item On a $R_{1}=R$ si et seulement si il existe $A\in\K[X]$ tel que $P=AQ+R$ avec $\deg(R)<\deg(Q)$ si et seulement si $Q\mid P-R$ et $\deg(R)<\deg(Q)$. Comme $\deg(Q)<\deg(P)$, $\deg(R)<\deg(P)$ implique $\deg(P-R)=\deg(P)$.
        
        Or \function{\varphi}{\K_{=n}[X]}{\K_{=n}[X]}{P}{P-R}
        est bijective donc les lois de $P-R$ et de $P$ sont les mêmes. En notant $r=\deg(R)$, on a donc 
        \begin{align}
            \P(R_{1}=R)
            &=\P\left((Q\mid P-R)\cap(\deg(Q)>\deg(R))\right)\\
            &=\sum_{q=r+1}^{n-1}\frac{p^{n-q}(p-1)}{(p-1)^{2}}\times\frac{p^{d}(p-1)}{p^{2n}}\\
            &=\boxed{\frac{1}{p^{n}}\times(n-r-1)}
        \end{align}

        et 
        \begin{align}
            \P_{\deg(Q)=q}(R_{1}=R)
            &=\frac{\P\left((R_{1}=R)\cap(\deg(Q)=q)\right)}{\P\left(\deg(Q)=s\right)}\\
            &=\frac{\frac{1}{p^{n}}}{\frac{p^{q}(p-1)}{p^{n}}}\\
            &=\boxed{\frac{1}{p^{q}(p-1)}}
        \end{align}
    \end{enumerate}
\end{proof}

\begin{proof}
    \phantom{}
    \begin{enumerate}
        \item $(X_{1},X_{2})$ prend ses valeurs dans $\llbracket1,n\rrbracket^{2}\setminus\left\lbrace(i,i)\middle| i\in\left\llbracket1,n\right\rrbracket\right\rbrace$. Soit $(i,j)\in\llbracket1,n\rrbracket^{2}$ avec $i\neq j$, alors 
        \begin{equation}
            \boxed{\P(X_{1}=i,X_{2}=j)=\frac{1}{n(n-1)}}
        \end{equation}
        La loi conjointe est uniforme.

        \item $X_{2}$ prend ses valeurs dans $\llbracket1,n\rrbracket$. On a 
        \begin{equation}
            \P(X_{2}=j)=\sum_{\substack{i=1\\i\neq j}^{n}}\P(X_{1}=i,X_{2}=j)=\frac{n-1}{n(n-1)}=\frac{1}{n}
        \end{equation}
        donc $X_{2}\sim\mathcal{U}\left(\llbracket1,n\rrbracket\right)$.

        $X_{1_{\mid X_{2}=j}}$ prend ses valeurs dans $\llbracket1,n\rrbracket\setminus\lbrace j\rbrace$ et 
        \begin{equation}
            \P(X_{1_{\mid X_{2}=j}}=i)=\frac{\P(X_{1}=i,X_{2}=j)}{\P(X_{2}=j)}=\frac{1}{n-1}
        \end{equation}
        donc $X_{1_{\mid X_{2}=j}}\sim\mathcal{U}\left(\left\llbracket1,n\right\rrbracket\setminus\left\lbrace j\right\rbrace\right)$.

        \item D'après ce qui précède, les lois de $X_{1}$ et $X_{2}$ sont différentes et $X_{1}$ et $X_{2}$ ne sont pas indépendantes.
        
        \item On écrit 
        \begin{equation}
            (X_{1},\dots,X_{k})=\left\lbrace(x_{1},\dots,x_{k})\in\llbracket1,n\rrbracket^{k}\middle| \forall i\neq j,x_{i}\neq x_{j}\right\rbrace
        \end{equation}
        ensemble que l'on note $A_{n,k}$. On a $\left\lvert A_{n,k}\right\rvert=n(n-1)\dots(n-k+1)$. Pour $(x_{1},\dots,x_{k})\in A_{n,k}$, on a donc 
        \begin{equation}
            \boxed{\P(X_{1}=x_{1},\dots,X_{k}=x_{k})=\frac{1}{n(n-1)}\dots(n-k+1)}
        \end{equation}
        et pour tout $i\in\llbracket1,n\rrbracket$, $X_{i}\sim\mathcal{U}(\llbracket1,n\rrbracket)$ et les $X_{i}$ ne sont pas indépendants.

        \item On a 
        \begin{align}
            \E(X_{1},X_{2})
            &=\sum_{\substack{(i,j)\in\llbracket1,n\rrbracket^{2}\\ i\neq j}}(i,j)\times\frac{1}{n(n-1)}\\
            &=\frac{1}{n(n-1)}\left(\sum_{i=1}^{n}\left(\sum_{j\neq i}i\right),\sum_{j=1}^{n}\left(\sum_{i\neq j}j\right)\right)\\
            &=\frac{1}{n(n-1)}\left(\frac{n(n-1)(n+1)}{2},\frac{n(n-1)(n+1)}{2}\right)\\
            &=\boxed{\frac{n+1}{2},\frac{n+1}{2}}
        \end{align}
    \end{enumerate}
\end{proof}

\begin{proof}
    \phantom{}
    \begin{enumerate}
        \item $R_n$="les $n$ premiers tirages sont rouges". Avec la formule des probabilités conditionnelles,
        \begin{equation*}
            \P(R_n)=\P(R_1)\P_{R_1}(R_2)\dots\P_{R_{n-1}}(R_n)=\frac{1}{2}\times\frac{3}{4}\times\dots\times\frac{2n-1}{2n}.
        \end{equation*}
        car $R_n\subset R_{n-1}$.
        \item $R=\cap_{n\in\N^{*}}R_n$, suite décroissante d'évènements. Donc 
        \begin{equation*}
            \P(R)=\lim\limits_{n\to+\infty}\P(R_n)=\prod_{n=1}^{+\infty}\left(\frac{2n-1}{2n}\right),
        \end{equation*}
        et $\ln(\P(R_n))=\sum_{k=1}^{n}\ln\left(1-\frac{1}{2k}\right)\xrightarrow[n\to+\infty]{}-\infty$ car le terme est équivalent à $\frac{-1}{2k}<0$ donc $\P(R)=0$.

        \item Par symétrie, la probabilité de tirer une boule rouge ou une blanche est la même, leur somme égale à 1, donc on a une probabilité de $\frac{1}{2}$.
        \item On refait les mêmes calculs qu'à la premier étape, si ce n'est que $\ln(\P(R_n))=\sum_{k=1}^{n}\ln\left(1-\frac{2}{4+k(k+1)}\right)$ converge donc $\P(R)>0$.
    \end{enumerate}
\end{proof}

\begin{remark}
    Soit une suite infinie (dénombrable) de tirages de pile ou face indépendantes suivant chacun une loi Bernoulli de paramètre $p\in(0,1)$. La probabilité qu'une séquence donnée de $N$ résultats apparaisse une infinité de fois vaut 1 d'après le lemme de Borel-Cantelli.
\end{remark}

\end{document}