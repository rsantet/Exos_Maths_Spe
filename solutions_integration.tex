\documentclass[12pt]{article}
\usepackage{style/style_sol}

\begin{document}

\begin{titlepage}
	\centering
	\vspace*{\fill}
	\Huge \textit{\textbf{Solutions MP/MP$^*$\\ Intégration}}
	\vspace*{\fill}
\end{titlepage}

\begin{proof}
    $S$ est de classe $\mathcal{C}^{1}$ sur $[a,b]$ avec $S'=f>0$. Donc $S$ définit un $\mathcal{C}^{1}$-difféomorphisme de $[a,b]$ dans $[S(a)=0,S(b)]$. COmme pour tout $n\geqslant1$, pour tout $k\in\left\llbracket1,n\right\rrbracket$, $k\frac{S(b)}{n}\in[0,S(b)]$, il existe un unique $x_{k}\in[a,b]$ tel que $S(x_{k})=k\frac{S(b)}{n}$ qui est simplement donné par 
    \begin{equation}
        \boxed{
            x_{k}=S^{-1}\left(k\frac{S(b)}{n}\right).
        }
    \end{equation}

    On a 
    \begin{equation}
        \frac{1}{n}\sum_{k=1}^{n}f(x_{k})=\frac{1}{S(b)}\left(\frac{S(b)}{n}\sum_{k=1}^{n}f\left(S^{-1}\left(\frac{k}{n}S(b)\right)\right)\right)\xrightarrow[n\to+\infty]{}\frac{1}{S(b)}\int_{0}^{S(b)}f\left(S^{-1}(t)\right)\mathrm{d}t=I.
    \end{equation}
    On effectue le changement de variable $u=S^{-1}(t)$ pour obtenir
    \begin{equation}
        \boxed{
            I=\frac{1}{S(b)}\int_{0}^{b}f(u)^{2}\mathrm{d}u.
        }
    \end{equation}
\end{proof}

\begin{remark}
    On peut se demander si cela reste vrai si $f\geqslant0$ (mais $f\neq 0$). On définit \function{\varphi}{[0,S(b)]}{[a,b]}{y}{\min\left(\left\lbrace x\in[a,b]\middle|y=S(x)\right\rbrace\right)}
    On a $x_{k}=\varphi\left(k\frac{S(b)}{n}\right)$, $f\circ \varphi$ continue par morceaux sur $[0,S(b)]$ et 
    \begin{equation}
        \frac{1}{n}\sum_{k=1}^{n}f\left(\varphi\left(k\frac{S(b)}{n}\right)\right)\xrightarrow[n\to+\infty]{}\frac{1}{S(b)}\int_{0}^{S(b)}\left(f\circ \varphi\right)(t)\mathrm{d}t.
    \end{equation}

    Cela marche aussi si $\left\lbrace t\in[a,b], f(t)=0\right\rbrace$ est discret car $S$ reste strictement croissante. Cela marche aussi si $\int f>0$ (poser $f_p=f+\frac{1}{p}>0$ et passer à la limite).
\end{remark}

\begin{proof}
    \phantom{}
    \begin{enumerate}
        \item Pour tout $x>0$, on a $g(x)\leqslant\left\lVert f\right\rVert_{\infty}$. Soit $t_{0}\in[0,1]$ tel que $\left\lvert f(t_{0})\right\rvert=\left\lVert f\right\rVert_{\infty}$. Soit $\varepsilon>0$. Par continuité de $\left\lvert f\right\rvert$, il existe $[a,b]\subset[0,1]$ avec $a<b$ tel que pour tout $t\in[a,b]$, $0<\left\lvert f(t_{0})\right\rvert-\frac{\varepsilon}{2}<\left\lvert f(t)\right\rvert$.

        D'où 
        \begin{equation}
            \left(\int_{0}^{1}\left\lvert f(t)\right\rvert^{x}\mathrm{d}t\right)^{\frac{1}{x}}\geqslant\left(\int_{a}^{b}\left\lvert f(t)\right\rvert^{x}\mathrm{d}t\right)^{\frac{1}{x}}\geqslant(b-a)^{\frac{1}{x}}\left(\left\lvert f(t_{0})\right\rvert-\frac{\varepsilon}{2}\right)\xrightarrow[x\to+\infty]{}\left\lvert f(t_{0})\right\rvert-\frac{\varepsilon}{2}.
        \end{equation}
        Alors il existe $X_{1}>0$ tel que pour tout $x\geqslant X_{1}$, $\left\lvert f(t_{}0)\right\rvert-\frac{\varepsilon}{2}-\frac{\varepsilon}{2}=\left\lVert f\right\rVert_{\infty}-\varepsilon\leqslant g(x)$. D'où le résultat.

        \item On pose $h_{x}(t)=\exp(x\ln(\left\lvert f(t)\right\rvert))$ pour tout $t\in[0,1]$. Alors pour tout $t\in[0,1]$, on a $\lim\limits_{x\to0}h_{x}(t)=1$ et pour tout $x>0$, pour tout $t\in[0,1]$ $h_{x}(t)\leqslant\max\left(1,\left\lVert f\right\rVert_{\infty}\right)$ qui est intégrable sur $[0,1]$. D'après le théorème de convergence dominé, on a 
        \begin{equation}
            \lim\limits_{x\to0}\int_{0}^{1}\left\lvert f(t)\right\rvert^{x}\mathrm{d}t=1.
        \end{equation}

        Malheureusement, ce n'est pas suffisant.

        Posons donc $k_{x}(t)=\frac{\left\lvert f(t)\right\rvert^{x}-1}{x}$. Pour $t$ fixé, on a $\lim\limits_{x\to0}k_{x}(t)=\ln\left(\left\lvert f(t)\right\rvert\right)$. De plus, pour tout $0<x\leqslant1$, pour tout $t\in[0,1]$, on a 
        \begin{equation}
            \left\lvert\left\lvert f(t)\right\rvert^{x}-1\right\rvert=\left\lvert\e^{x\ln(\left\lvert f(t)\right\rvert)}-\e^{0}\right\rvert 
            \left\lbrace
                \begin{array}[]{ll}
                    \leqslant x\ln(\left\lvert f(t)\right\rvert),&\text{si }\left\lvert f(t)\right\rvert\leqslant1,\\\hline
                    \leqslant x\ln(\left\lvert f(t)\right\rvert)\e^{x\ln(\left\lvert f(t)\right\rvert)}\\
                    \leqslant x\ln(\left\lvert f(t)\right\rvert)\e^{\ln(\left\lvert f(t)\right\rvert)},&\text{si }\left\lvert f(t)\right\rvert>1.
                \end{array}
            \right.
        \end{equation}

        Ainsi $k_{x}(t)\leqslant\max\left(\ln\left(\left\lVert f\right\rVert_{\infty}\right), \left\lVert f\right\rVert_{\infty}\ln\left(\left\lVert f\right\rVert_{\infty}\right)\right)$ qui est intégrable sur $[0,1]$. D'après le théorème de convergence dominé,
        \begin{equation}
            \lim\limits_{x\to0}\int_{0}^{1}\frac{f(t)^{x}-1}{x}\mathrm{d}t=\int_{0}^{1}\ln\left(\left\lvert f(t)\right\rvert\right)\mathrm{d}t.
        \end{equation}

        Ainsi,
        \begin{align}
            g(x)
            &=\exp\left(\frac{1}{x}\ln\left(1+x\int_{0}^{1}k_{x}(t)\mathrm{d}t\right)\right),\\
            &=\exp\left(\frac{1}{x}\left(x\int_{0}^{1}\ln\left(\left\lvert f(t)\right\rvert\right)+\underset{x\to0}{o}(x)\right)\right),\\
            &=\exp\left(\int_{0}^{1}\ln\left(\left\lvert f(t)\right\rvert\right)\mathrm{d}t+\underset{x\to0}{o}(1)\right)\xrightarrow[x\to0]{}\exp\left(\int_{0}^{1}\ln\left(\left\lvert f(t)\right\rvert\right)\mathrm{d}t\right).
        \end{align}
    \end{enumerate}
\end{proof}

\begin{proof}
    On fixe $y\in[0,f(a)]$. On pose \function{\varphi}{[0,a]}{\R}{x}{\int_{0}^{x}f+\int_{0}^{y}g-xy}
    $\varphi$ est $\mathcal{C}^{1}$ et $\varphi'(x)=f(x)-y$ donc $\varphi$ décroît de $0$ à $g(y)$ puis croît jusqu'en $x=a$. Son minimum vaut alors $\varphi(g(y))=\int_{0}^{x}+\int_{0}^{f(x)}g-xf(x)$ avec $x=g(y)$.

    Si $f$ est $\mathcal{C}^{1}$, alors $g$ l'est aussi car $f$ définit un $\mathcal{C}^{1}$-difféomorphisme de $[0,a]$ dans $[0,f(a)]$. On effectue le changement de variable $u=f(t)$ et on obtient $\varphi(g(y))=\int_{0}^{x}(tf'(t)+f(t))\mathrm{d}t-xf(x)=[tf(t)]_{0}^{x}-xf(x)=0$. De même si $f$ est $\mathcal{C}^{1}$ par morceaux (utiliser la relation de Chasles).

    Plus généralement, on a le lemme
    \begin{lemma}
        Soit pour $n\geqslant1$, $f_n:[0,a]\to\R$ affine par morceaux continue telle que pour tout $k\in\left\llbracket0,n\right\rrbracket$, $f_{n}\left(\frac{k}{n}a\right)=f\left(\frac{k}{n}a\right)$. Alors $(f_{n})_{n\geqslant1}$ converge uniformément vers $f$ sur $[0,a]$ et $(f_{n}^{-1})_{n\geqslant1}$ converge uniformément vers $f$ sur $[0,f(a)]$.
    \end{lemma}
    \begin{proof}[Preuve du lemme]
        Soit $\varepsilon>0$. Par continuité uniforme de $f$, il existe $N_{0}\in\N$ tel que pour tout $n\geqslant N_{0}$, pour tout $k\in\left\llbracket0,n-1\right\rrbracket$, pour tout $x\in\left[\frac{ka}{n},\frac{k+1}{n}a\right]$, on a $\left\lvert f(x)-f\left(\frac{k}{n}a\right)\right\rvert\leqslant\frac{\varepsilon}{2}$. Alors 
        \begin{equation}
            \left\lvert f(x)-f_n(x)\right\rvert\leqslant\left\lvert f(x)-f\left(\frac{ka}{n}\right)\right\rvert+\left\lvert f_n\left(\frac{k}{n}a\right)-f_n(x)\right\rvert\leqslant\varepsilon.
        \end{equation}
        On fait de même pour $(f_n^{-1})_{n\geqslant1}$.
    \end{proof}
    $f_n$ et $f_n^{-1}$ sont $\mathcal{C}^{1}$ par morceaux continues et $g_n=f_{n}^{-1}$. On a $\int_{0}^{x}f_{n}+\int_{0}^{f_n(x)}f_n=xf_n(x)$. Quand $n\to+\infty$, par convergence uniforme, on a $\int_{0}^{f_n(x)}g_n=\int_{0}^{f(x)}g_n+\int_{f_n(x)}^{f(x)}g_n$ et le dernier terme est uniformément borné par $\left\lVert f^{-1}\right\rVert_{\infty}\left\lvert f(x)-f_n(x)\right\rvert\xrightarrow[n\to+\infty]{}0$. Ainsi, le cas d'égalité est quand 
    \begin{equation}
        \boxed{
            \int_{0}^{x}f+\int_{0}^{f(x)}g=xf(x).
        }
    \end{equation}
\end{proof}

\begin{proof}
    On pose $f(x)=\frac{\ln(x)}{(1+x)\sqrt{1-x^{2}}}$. $f$ est continue sur $\left[\frac{1}{2},1\right[$ et $f(x)\underset{x\to1^{-}}{\sim}\frac{x-1}{2\sqrt{2(1-x)}}\xrightarrow[x\to^{1^{-}}]{}0$. On effectue le changement de variable $x=\cos(t)$ d'où $\mathrm{d}t=-\frac{\mathrm{d}x}{\sqrt{1-x^{2}}}$. On a alors 
    \begin{equation}
        I=-\int_{\frac{\pi}{3}}^{0}\frac{\ln(\cos(t))}{1+\cos(t)}\mathrm{d}t=\int_{0}^{\frac{\pi}{3}}\frac{\ln(\cos(t))}{2\cos^{2}\left(\frac{t}{2}\right)}\mathrm{d}t.
    \end{equation}
    Or $\tan'\left(\frac{t}{2}\right)=\frac{1}{2\cos^{2}\left(\frac{t}{2}\right)}$ donc par intégrations par parties,
    \begin{equation}
        I=\left[\ln(\cos(t))\tan\left(\frac{t}{2}\right)\right]_{0}^{\frac{\pi}{3}}-\int_{0}^{\frac{\pi}{3}}\frac{-\sin(t)}{\cos(t)}\tan\left(\frac{t}{2}\right)\mathrm{d}t.
    \end{equation}
    Le premier terme vaut $\frac{\ln(\frac{1}{2})}{\sqrt{3}}$.
    Pour le deuxième terme, on utilise la formule d'addition $\tan\left(\frac{t}{2}+\frac{t}{2}\right)=\frac{2\tan\left(\frac{t}{2}\right)}{1-\tan^{2}\left(\frac{t}{2}\right)}$. Ainsi,
    \begin{equation}
        \int_{0}^{\frac{\pi}{3}}\tan(t)\tan\left(\frac{t}{2}\right)\mathrm{d}t=\int_{0}^{\frac{\pi}{3}}\frac{2\tan^{2}\left(\frac{t}{2}\right)}{1-\tan^{2}\left(\frac{t}{2}\right)}\mathrm{d}t=\int_{0}^{\frac{1}{\sqrt{3}}}\frac{4u^{2}}{1-u^{2}}\frac{\mathrm{d}u}{1+u},
    \end{equation}
    en ayant effectué le changement de variables $u=\tan\left(\frac{t}{2}\right)$, d'où $\mathrm{d}t=\frac{2\mathrm{d}u}{1+u^{2}}$. Il ne reste plus qu'à faire une décomposition en éléments simples.
\end{proof}

\begin{proof}
    \phantom{}
    \begin{enumerate}
        \item $I_{n}$ est bien définie. On a 
        \begin{align}
            I_{n}+I_{n+2}
            &=\int_{0}^{\frac{\pi}{4}}\tan^{n}(x)(1+\tan^{2}(x))\mathrm{d}x,\\
            &=[\tan^{n+1}(x)]_{0}^{\frac{\pi}{4}}-n\int_{0}^{\frac{\pi}{4}}\tan^{n}(x)(1+\tan^{2}(x))\mathrm{d}x,\\
            =1-n(I_{n}+I_{n+2}).
        \end{align}
        Donc $I_{n}+I_{n+2}=\frac{1}{n+1}$. On a $I_{0}=\frac{\pi}{4}$. On en déduit que 
        \begin{equation}
            I_{2p}=\frac{1}{2p-1}-I_{2p-2}=\dots=(-1)^{p}\left(\frac{\pi}{4}-1+\frac{1}{3}-\frac{1}{5}+\dots+\frac{(-1)^{p}}{2p-1}\right).
        \end{equation}

        On a $I_{1}=\int_{0}^{\frac{\pi}{4}}\tan(x)\mathrm{d}x=[-\ln(\cos(x))]_{0}^{\frac{\pi}{4}}=\frac{1}{2}\ln(2)$. Ainsi,
        \begin{equation}
            I_{2p+1}=(-1)^{p}\left(\frac{\ln(2)}{2}-\frac{1}{2}+\frac{1}{4}-\frac{1}{6}+\dots+\frac{(-1)^{p}}{2p}\right).
        \end{equation}

        \item On pose $f_n(x)=\tan^{n}(x)$. Si $x\in\left[0,\frac{\pi}{4}\right[$, on a $\lim\limits_{n\to+\infty}f_n(x)=0$. Si $x=\frac{\pi}{4}$, on a $\lim\limits_{n\to+\infty}f_n(x)=1$. Donc $(f_n)_{n\in\N}$ converge simplement vers $f\colon\left[0,\frac{\pi}{4}\right]\to\R$ qui vaut $0$ partout sauf en $\frac{\pi}{4}$ où elle vaut 1.
        Soit $n\in\N$ et $x\in\left[0,\frac{\pi}{4}\right]$. On a $\left\lvert f_n(x)\right\rvert\leqslant1$ intégrable sur $\left[0,\frac{\pi}{4}\right]$. D'après le théorème de convergence dominée,
        \begin{equation}
            \boxed{
                \lim\limits_{n\to+\infty}I_{n}=0.
            }
        \end{equation}

        \item D'après ce qui précède, on a 
        \begin{equation}
            \everymath={\displaystyle}
            \boxed{
                \begin{array}[]{rcl}
                    \frac{\pi}{4}&=&\sum_{n=0}^{+\infty}\frac{(-1)^{k}}{2k+1},\\
                    \ln(2)&=&\sum_{k=0}^{+\infty}\frac{(-1)^{k}}{k+1}.
                \end{array}
            }
        \end{equation}
    \end{enumerate}
\end{proof}

\begin{remark}
    On peut donner un équivalent de $I_{n}$. Comme pour tout $x\in\left[0,\frac{\pi}{4}\right]$, on a $0\leqslant\tan(x)\leqslant1$, on a $I_{n+2}\leqslant I_{n}$. Ainsi, 
    \begin{equation}
        2I_{n+2}\leqslant I_{n}+I_{n+2}=\frac{1}{n+1}\leqslant 2I_{n},
    \end{equation}
    et donc 
    \begin{equation}
        \frac{1}{2(n+1)}\leqslant I_n\leqslant\frac{1}{2(n-1)},
    \end{equation}
    d'où
    \begin{equation}
        \boxed{
            I_{n}\underset{n\to+\infty}{\sim}\frac{1}{2n}.
        }
    \end{equation}
\end{remark}

\begin{proof}
    \phantom{}
    \begin{enumerate}
        \item D'après l'inégalité de Cauchy-Schwarz appliquée à $\sqrt{f}$ et $\frac{1}{\sqrt{f}}$, on a 
        \begin{equation}
            \int_{a}^{b}f\times\int_{a}^{b}\frac{1}{f}\geqslant\left(\int_{a}^{b}1\right)^{2}=(b-a)^{2}.
        \end{equation}
        $f\colon x\mapsto 1$ pour tout $x\in[a,b]$ donne l'égalité, et d'après le cas d'égalité de l'inégalité de Cauchy-Schwarz, on a égalité si et seulement si $\sqrt{f}$ et $\frac{1}{\sqrt{f}}$ sont proportionnelles, donc si et seulement si $f$ est constante.

        \item Soit $\alpha\in\R_{+}^{*}\setminus\left\lbrace1\right\rbrace$ et $c<a$. Soit \function{f_{\alpha,c}}{[a,b]}{\R_+^*}{t}{(t-c)^\alpha}
        On a
        \begin{align}
            \phi(f_{\alpha,c})
            &=\frac{1}{\alpha^{2}-1}\left[(b-c)^{\alpha+1}-(a-c)^{\alpha+1}\right]\left[(a-c)^{-\alpha+1}-(b-c)^{-\alpha+1}\right],\\
            &\underset{\alpha\to+\infty}{\sim}\frac{1}{\alpha^{2}}\left[(b-c)^{\alpha+1}\times\frac{1}{(a-c)^{\alpha-1}}\right],\\
            &\underset{\alpha\to+\infty}{\sim}\frac{(b-a)(a-c)}{\alpha^{2}}\left(\frac{b-c}{a-c}\right)^{\alpha}\xrightarrow[\alpha\to+\infty]{}+\infty,
        \end{align}
        car $b-c>a-c$.

        \item Soit $f,g\in E^{2}$ et $\lambda\in[0,1]$. $\lambda f+(1-\alpha)g$ est continue et strictement positive. $E$ est convexe dans $\left(\mathcal{C}^{0}\left([a,b],\R_{+}^{*}\right),\left\lVert\cdot\right\rVert_{\infty}\right)$ donc connexe par arcs.
        
        Soit $f\in E$ et $(f_n)_{n\in\N}$ suite de fonctions convergent uniformément vers $f$. Par convergence uniforme, on a $\int_{a}^{b}f_n\xrightarrow[n\to+\infty]{}\int_{a}^{b}f$. De plus, pour tout $x\in[a,b]$, on a 
        \begin{equation}
            \left\lvert\frac{1}{f_n(x)}-\frac{1}{f(x)}\right\rvert=\frac{\left\lvert f_n(x)-f(x)\right\rvert}{f_n(x)\times f(x)}\leqslant\frac{\left\lVert f_n-f\right\rVert_{\infty}}{\min_{y\in[a,b]f_n(y)\times f(y)}}.
        \end{equation}
        Il existe $n_0\in\N$ tel que pour tout $n\geqslant n_0$, $\left\lVert f_n-f\right\rVert_{\infty}\leqslant\frac{\min f}{2}$ et pour tout $x\in[a,b]$, pour tout $n\geqslant n_0$, $f_n(x)\geqslant\frac{\min f}{2}$. Alors 
        \begin{equation}
            \left\lVert \frac{1}{f_n}-\frac{1}{f}\right\rVert_{\infty}\leqslant\frac{2\left\lVert f_n-f\right\rVert_{\infty}}{(\min f)^{2}}\xrightarrow[n\to+\infty]{}0.
        \end{equation}
        Ainsi, $\int_{a}^{b}\frac{1}{f_n}\xrightarrow[n\to+\infty]{}\int_{a}^{b}\frac{1}{f}$ et $\phi(f_n)\xrightarrow[n\to+\infty]{}\phi(f)$. $\phi$ est donc continue. D'après le théorème des valeurs intermédiaires, on a donc 
        \begin{equation}
            \boxed{
                \phi(E)=[(b-a)^{2},+\infty[.
            }
        \end{equation}
    \end{enumerate}
\end{proof}

\begin{proof}
    Soit \function{f}{]0,+\infty}{\R}{x}{\frac{\sqrt{x}\ln(x)}{(1+x)^{2}}}
    $f$ est continue. On a $f(x)\xrightarrow[x\to0]{}0$ donc $\int_{0}^{1}f$ converge. On a $f(x)\underset{x\to+\infty}{\sim}\ln(x)\frac{1}{x^{\frac{3}{2}}}=\underset{x\to+\infty}{O}\left(\frac{1}{x^{\frac{5}{4}}}\right)$ donc $\int_{1}^{+\infty}f$ converge.

    On pose $x=u^{2}$ et on obtient 
    \begin{align}
        I
        &=\int_{0}^{+\infty}\frac{2u^{2}\ln(u^{2})}{(1+u^{2})^{2}}\mathrm{d}u,\\
        &=4\int_{0}^{+\infty}\frac{u^{2}\ln(u)}{(1+u^{2})^{2}}\mathrm{d}u,\\
        &=2\left(\left[-\frac{1}{(1+u^{2})}\times u\ln(u)\right]_{0}^{+\infty}+\int_{0}^{+\infty}\frac{1}{1+u^{2}}\left(\ln(u)+0\right)\mathrm{d}u\right),\label{it:ipp}\\
        &=2\left(\int_{0}^{+\infty}\frac{\ln(u)}{1+u^{2}}\mathrm{d}u+\int_{0}^{+\infty}\frac{1}{1+u^{2}}\mathrm{d}u\right).
    \end{align}

    Noter que l'intégration par parties faite en~\ref{it:ipp} est correcte car tout converge en 0 et $+\infty$ (passer à la limite $\alpha,\beta\to0,+\infty$ pour être plus rigoureux).

    La première intégrale est nulle. En effet, on pose $x=\frac{1}{u}$ d'où $\mathrm{d}x=-\frac{\mathrm{d}u}{u^{2}}$ et donc 
    \begin{equation}
        \int_{0}^{+\infty}\frac{\ln(u)}{1+u^{2}}\mathrm{d}u=-\int_{+\infty}^{0}\frac{\ln\left(\frac{1}{x}\right)}{1+\frac{1}{x^{2}}}\frac{\mathrm{d}x}{x^{2}}=-\int_{0}^{+\infty}\frac{\ln(x)}{1+x^{2}}\mathrm{d}x.
    \end{equation}
    La deuxième intégrale vaut $\frac{\pi}{2}$. Finalement, on a 
    \begin{equation}
        \boxed{
            I=\pi.
        }
    \end{equation}
\end{proof}

\begin{proof}
    On note $f$ la fonction intégrande. $f$ est continue négative. On a $\left\lvert f(t)\right\rvert\underset{t\to0}{\sim}\left\lvert\frac{\ln(t)}{\sqrt{t}}\right\rvert=\underset{t\to0}{O}\left(\frac{1}{t^{\frac{3}{4}}}\right)$ donc $\int_{0}^{\frac{1}{2}}f$ converge. On a $\left\lvert f(t)\right\rvert\underset{t\to1}{\sim}\frac{1}{\sqrt{1-t}}$ donc $\int_{\frac{1}{2}}^{1}f$ converge.

    On a 
    \begin{equation}
        I=\int_{0}^{1}\frac{\ln(t)}{1-t}\frac{\mathrm{d}t}{\sqrt{t(1-t)}}.
    \end{equation}
    Comme $t(1-t)=-(t^{2}-t)=-\left(\left(t-\frac{1}{2}\right)^{2}-\frac{1}{4}\right)=\frac{1}{4}\left(1-(2t-1)^{2}\right)$, on pose $2t-1=\cos\theta$. On a alors $t=\frac{\cos\theta+1}{2}$ et $\mathrm{d}\theta=\frac{-2\mathrm{d}t}{\sqrt{1-(2t-1)^{2}}}=\frac{-\mathrm{d}t}{\sqrt{t(1-t)}}$. Ainsi, 
    \begin{equation}
        I=\int_{0}^{\pi}\frac{\ln\left(\frac{\cos\theta+1}{2}\right)}{\frac{1-\cos\theta}{2}}\mathrm{d}\theta.
    \end{equation}
    On a $\frac{1+\cos\theta}{2}=\cos^{2}\left(\frac{\theta}{2}\right)$ et $\frac{1-\cos\theta}{2}=\sin^{2}\left(\frac{\theta}{2}\right)$. En posant $u=\frac{\theta}{2}$, on a donc 
    \begin{equation}
        I=4\int_{0}^{\frac{\pi}{2}}\frac{\ln(\cos u)}{\sin^{2}u}\mathrm{d}u.
    \end{equation}
    En fixant $0<\varepsilon<\alpha<1$ et en posant $I_{\varepsilon,\alpha}=\int_{\varepsilon}^{\alpha}f$, on a en faisant une intégration par parties:
    \begin{equation}
        I_{\varepsilon,\alpha}=4\left(\left[-\cot u\times\ln(\cos u)\right]_{\varepsilon}^{\alpha}-\int_{\varepsilon}^{\alpha}1\mathrm{d}u\right).
    \end{equation}
    Le deuxième terme tend vers $\frac{\pi}{2}$,. Pour le premier, si $\alpha=\frac{\pi}{2}-h$, on a 
    \begin{equation}
        -\cot\alpha\ln\cos\alpha=-\tan h\ln\sin h=-\tan h\left[\ln h+\underset{h\to0}{o}(1)\right]\underset{h\to0}{\sim}-h\ln(h)\xrightarrow[h\to0]{}0.
    \end{equation}
    De même, on a 
    \begin{equation}
        -\cot\varepsilon\ln\cos\varepsilon\underset{\varepsilon\to0}{\sim}-\frac{1}{\varepsilon}\times \frac{-\varepsilon^{2}}{2}\underset{\varepsilon\to0}{\sim}\frac{\varepsilon}{2}\xrightarrow[\varepsilon\to0]{}0.
    \end{equation}

    Ainsi, 
    \begin{equation}
        \boxed{
            I=-2\pi.
        }
    \end{equation}
\end{proof}

\begin{proof}
    On note $f$ la fonction intégrande. Si $h=\frac{\pi}{4}-t$, on a $\cos(2t)=\cos\left(\frac{\pi}{2}-2h\right)=\sin(2h)\underset{h\to0}{\sim}2h$. Ainsi, 
    \begin{equation}
        f(t)\underset{t\to\frac{\pi}{4}}{\sim}\frac{\frac{1}{2\sqrt{2}}}{\sqrt{2\left(\frac{\pi}{4}-t\right)}},
    \end{equation}
    donc l'intégrale existe (critère de Riemann).

    En posant $u=\sin(t)$, puis $v=\sqrt{2}u$, puis $\theta=\arcsin(v)$, on a 
    \begin{align}
        I
        &= \int_{0}^{\frac{\pi}{4}}\frac{\left(1-\sin^{2}(t)\right)\cos(t)}{\sqrt{1-2\sin^{2}(t)}}\mathrm{d}t,\\
        &= \int_{0}^{\frac{\sqrt{2}}{2}}\frac{1-u^{2}}{\sqrt{1-2u^{2}}}\mathrm{d}u,\\
        &= \int_{0}^{1}\frac{1-\frac{u^{2}}{2}}{\sqrt{2}}\frac{\mathrm{d}u}{\sqrt{1-u^{2}}},\\
        &= \int_{0}^{\frac{\pi}{2}}\frac{1-\frac{\sin^{2}\theta}{2}}{\sqrt{2}}\mathrm{d}\theta,\\
        &= \frac{1}{\sqrt{2}}\left(\frac{\pi}{2}-\frac{1}{4}\int_{0}^{\frac{\pi}{2}}(1-\cos(2\theta))\mathrm{d}\theta\right),\\
        &=\frac{3\pi-1}{8\sqrt{2}}.
    \end{align}
\end{proof}

\begin{proof}
    Si $f=c\in\C$ est constante, on a 
    \begin{equation}
        \gamma=\int_{a}^{b}f(t)g(\lambda t)\mathrm{d}t=c\int_{a}^{b}g(\lambda t)\mathrm{d}t.
    \end{equation}
    On pose $u=\lambda t$ et on pose $k(\lambda)=\left\lfloor\frac{\lambda b-\lambda a}{T}\right\rfloor\underset{\lambda\to+\infty}{\sim}\frac{\lambda(b-a)}{T}$. Alors 
    \begin{equation}
        \gamma=\frac{c}{\lambda}k(\lambda)\int_{0}^{T}g+\frac{c}{\lambda}\int_{\lambda a+k(\lambda)T}^{\lambda b}g.
    \end{equation}
    Le deuxième terme est majoré par $\frac{\left\lvert c\right\rvert}{\lambda}T\left\lVert g\right\rVert_{\infty}\xrightarrow[\lambda\to+\infty]{}0$. Finalement,
    \begin{equation}
        \lim\limits_{\lambda\to+\infty}\gamma=\frac{c(b-a)}{T}\int_{0}^{T}g=\frac{1}{T}\int_{0}^{T}g\int_{a}^{b}f.
    \end{equation}

    C'est la même chose pour les fonctions en escalier (par combinaison linéaire).

    Pour une fonction quelconque $f$ continue par morceaux, soit $\varepsilon>0$. Il existe $f_{\varepsilon}$ une fonction en escalier telle que $\left\lVert f-f_{\varepsilon}\right\rVert_{\infty}\leqslant\varepsilon$. On forme 
    \begin{equation}
        \Gamma=\left\lvert\int_{a}^{b}(f(t)g(\lambda t))\mathrm{d}t-\frac{1}{T}\int_{0}^{T}g\int_{a}^{b}f\right\rvert.
    \end{equation}

    On a 
    \begin{align}
        \Gamma
        &=\left\lvert \int_{a}^{b}f_{\varepsilon}(t)g(\lambda t)\mathrm{d}t+\int_{a}^{b}(f(t)-f_{\varepsilon}(t))g(\lambda t)\mathrm{d}t-\frac{1}{T}\int_{0}^{T}g\int_{a}^{b}f_{\varepsilon}-\frac{1}{T}\int_{0}^{T}g\int_{a}^{b}(f-f_{\varepsilon})\right\rvert,\\
        &\leqslant \left\lvert \int_{a}^{b}f_{\varepsilon}(t)g(\lambda t)\mathrm{d}t-\frac{1}{T}\int_{0}^{T}g\int_{a}^{b}f_{\varepsilon}\right\rvert+\left\lvert \int_{a}^{b}(f(t)-f_{\varepsilon}(t))g(\lambda t)\mathrm{d}t\right\rvert+\left\lvert\frac{1}{T}\int_{0}^{T}g\int_{a}^{b}(f-f_{\varepsilon})\right\rvert.
    \end{align}

    Il existe $\lambda_{0}\in\R$ tel que pour tout $\lambda\geqslant\lambda_{0}$,
    \begin{equation}
        \left\lvert\int_{a}^{b}f_{\varepsilon}(t)g(\lambda t)\mathrm{d}t-\frac{1}{T}\int_{0}^{T}\int_{a}^{b}f_{\varepsilon}\right\rvert\leqslant\frac{\varepsilon}{3}.
    \end{equation}

    Ainsi, $\Gamma\leqslant\frac{\varepsilon}{3}\times 3=\varepsilon$. Donc 
    \begin{equation}
        \boxed{
            \lim\limits_{\lambda\to+\infty}\int_{a}^{b}f(t)g(\lambda t)\mathrm{d}t=\frac{1}{T}\int_{0}^{T}g\int_{a}^{b}f.
        }
    \end{equation}

    Pour le cas particulier, on a $g(t)=\frac{1}{3+2\cos(t)}$. $g$ est $2\pi$-périodique, paire et strictement positive. On pose $x=\tan\left(\frac{t}{2}\right)$, on a $\cos(t)=\frac{1-x^{2}}{1+x^{2}}$ et $\sin(t)=\frac{2x}{1+x^{2}}$. Par parité, on a $\int_{0}^{2\pi}g=2\int_{0}^{\pi}g$, et 
    \begin{align}
        \int_{0}^{\pi}g(t)\mathrm{d}t
        &=\int_{0}^{+\infty}\frac{2\mathrm{d}x}{(1+x^{2})\left(3+2\left(\frac{1-x^{2}}{1+x^{2}}\right)\right)},\\
        &= 2\int_{0}^{+\infty}\frac{\mathrm{d}x}{x^{2}+5},\\
        &= \frac{2}{\sqrt{5}}\int_{0}^{+\infty}\frac{\frac{\mathrm{d}x}{\sqrt{5}}}{\left(\frac{x}{\sqrt{5}}\right)^{2}+1},\\
        &= \frac{2}{\sqrt{5}}\times\frac{\pi}{2},\\
        &=\frac{\pi}{\sqrt{5}}.
    \end{align}

    Donc 
    \begin{equation}
        \boxed{
            \lim\limits_{n\to+\infty}\int_{0}^{2\pi}\frac{f(t)}{3+2\cos(nt)}=\frac{1}{\sqrt{5}}\int_{0}^{2\pi}f.
        }
    \end{equation}
\end{proof}

\begin{remark}
    Pour calculer $I=\int_{0}^{2\pi}\frac{\mathrm{d}t}{3+2\cos(t)}$, on peut écrire
    \begin{equation}
        \frac{1}{3+2\cos(t)}=\frac{1}{3+\e^{\i t}+\e^{-\i t}}=\frac{\e^{\i t}}{\e^{2\i t}+3\e^{\i t}+1}.
    \end{equation}
    On décompose $F(X)=\frac{X}{X^{2}+3X+1}=\frac{\alpha}{X-\lambda}+\frac{\beta}{X-\mu}$ avec $\lambda=\frac{-3+\sqrt{5}}{2}\in]-1,0[$, $\mu=\frac{-3-\sqrt{5}}{2}\in]-\infty,-1[$, $\alpha=\frac{\lambda}{\lambda-\mu}$, $\beta=\frac{\mu}{\mu-\lambda}$ avec $\lambda-\mu=\sqrt{5}$ et $\lambda=\frac{1}{\mu}$. Ainsi, 
    \begin{align}
        \frac{1}{3+2\cos(t)}
        &=\frac{\lambda}{\sqrt{5}}\frac{1}{\e^{\i t}-\lambda}-\frac{\mu}{\sqrt{5}}\frac{1}{\e^{\i t}-\mu},\\
        &=\frac{\lambda}{\sqrt{5}}\frac{\e^{-\i t}}{1-\lambda\e^{-\i t}}+\frac{1}{\sqrt{5}}\frac{1}{1-\frac{\e^{\i t}}{\mu}},\\
        &=\frac{1}{\sqrt{5}}\sum_{n\in\Z}\lambda^{n}\e^{\i nt},
    \end{align}
    car $\left\lvert \lambda\e^{-\i t}\right\rvert<1$ et $\left\lvert \frac{\e^{\i t}}{\mu}\right\rvert<1$. Comme on a $\left\lvert \lambda^{n}\e^{\i nt}\right\rvert\leqslant\left\lvert\lambda\right\rvert^{n}$, on a convergence normale sur $[0,2\pi]$ car $\left\lvert\lambda\right\rvert<1$. Ainsi,
    \begin{equation}
        \int_{0}^{2\pi}\frac{\mathrm{d}t}{3+2\cos(nt)}=\frac{1}{\sqrt{5}}\sum_{n\in\Z}\lambda^{n}\int_{0}^{2\pi}\e^{\i nt}\mathrm{d}t=\frac{2\pi}{\sqrt{5}}.
    \end{equation}
\end{remark}

\begin{proof}
    \phantom{}
    \begin{enumerate}
        \item Si $f$ ne tend pas vers 0 en $+\infty$, il existe $\varepsilon_{0}>0$ tel que pour tout $A>0$, il existe $x_{A}\geqslant A$ tel que $\left\lvert f(x_{A})\right\rvert>\varepsilon_{0}$. On sait qu'il existe $\alpha_{0}>0$ tel que pour tout $(x_{1},x_{2})\in\left(\R_{+}\right)^{2}$, si $\left\lvert x_{1}-x_{2}\right\rvert\leqslant\alpha_{0}$ alors $\left\lvert f(x_{1})-f(x_{2})\right\rvert\leqslant\frac{\varepsilon_{0}}{2}$. Alors pour tout $A\geqslant0$, pour tout $x\in[x_{A}-\alpha_{0},x_{A}+\alpha_{0}]$, on a $\left\lvert f(x)-f(x_{A})\right\rvert\leqslant\frac{\varepsilon_{0}}{2}$. Donc $f(x)$ est du signe de $f(x_{A})$ et $\left\lvert f(x)\right\rvert>\frac{\varepsilon_{0}}{2}$. Alors on a 
        \begin{equation}
            \left\lvert\int_{x_{A}-\alpha_{0}}^{x_{A}+\alpha_{0}}f(x)\mathrm{d}x\right\rvert=\int_{x_{A}-\alpha_{0}}^{x_{A}+\alpha_{0}}\left\lvert f(x)\right\rvert\d x>\varepsilon_{0}\alpha_{0}>0.
        \end{equation}
        Or 
        \begin{equation}
            \left\lvert \int_{x_{A}-\alpha_{0}}^{x_{A}+\alpha_{0}}f(x)\d x\right\rvert = \left\lvert \int_{x_{A}-\alpha_{0}}^{+\infty}f(x)\d x-\int_{x_{A}+\alpha_{0}}^{+\infty}f(x)\d x\right\rvert\xrightarrow[A\to+\infty]{}0.
        \end{equation}
        C'est absurde, donc
        \begin{equation}
            \boxed{
                \lim\limits_{x\to+\infty}f(x)=0.
            }
        \end{equation}

        \item Il existe $x_{0}>0$ tel que pour tout $x>x_{0}$, on ait $\left\lvert f(x)\right\rvert<1$. Donc pour tout $x>x_{0}$, on a $\left\lvert f^{2}(x_{0})\right\rvert\leqslant\left\lvert f(x)\right\rvert$ d'où $f^{2}=\underset{+\infty}{O}(f)$ et $f^{2}$ est intégrable.
    \end{enumerate}
\end{proof}

\begin{remark}
    Si $f$ est à valeurs dans $\C$, alors il faut raisonner sur $\Im(f)$ et $\Re(f)$ et le résultat reste vrai.
\end{remark}

\begin{proof}
    \phantom{}
    \begin{enumerate}
        \item Si $x=0$, $f_n(0)=\frac{n}{\sqrt{\pi}}\xrightarrow[n\to+\infty]{}+\infty$. Si $x\neq0$, alors $f_n$ converge simplement vers 0. On n'a pas convergence uniforme sur $\R_{+}^{*}$ car on pourrait intervertir les limites en 0.
        Soit $a>0$, soit $x\in[a,+\infty[$. $f$ étant décroissante sur $\R_{+}^{*}$, on a $\left\lvert f_n(x)\right\rvert\leqslant\frac{n}{\sqrt{\pi}}\e^{-n^{2}a^{2}}\xrightarrow[n\to+\infty]{}0$. On a donc convergence uniforme sur $[a,+\infty[$.

        Notons que $f_n$ est intégrable sur $\R$ et que son intégrable vaut 1. Enfin, pour tout $a>0$, on a $\int_{0}^{+\infty}f_n(x)\d x=\int_{n_0}^{+\infty}\frac{1}{\sqrt{\pi}}\e^{-u^{2}}\d u\xrightarrow[n\to+\infty]{}0$ (reste d'intégrale convergente).

        \item Notons $g_n(u)=\frac{g\left(\frac{u}{n}\right)}{\sqrt{\pi}}\e^{-u^{2}}$ de telle sorte que $\int_{-\infty}^{+\infty}g(t)\frac{n}{\sqrt{\pi}}\e^{-n^{2}t^{2}}\d t=\int_{-\infty}^{+\infty}g_n(u)\d u$.
        
        Soit $u$ fixé dans $\R$, on a $\lim\limits_{n\to+\infty}g_n(u)=\frac{g(0)}{\sqrt{\pi}}\e^{-u^{2}}$ par continuité de $g$, et pour tout $n\geqslant1$, pour tout $u\in\R$, on a $\left\lvert g_n(u)\right\rvert\leqslant\frac{\left\lVert g\right\rVert_{\infty}}{\sqrt{5}}\e^{-u^{2}}$ intégrable sur $\R$. D'après le théorème de converge dominée, on peut intervertir limite et intégrale, donc 
        \begin{equation}
            \boxed{
                \lim\limits_{n\to+\infty}\int_{\R}g(t)f_n(t)\d t=g(0)
            }
        \end{equation}
    \end{enumerate}
\end{proof}

\begin{remark}
    Généralement, pour tout $x\in\R$, $(f_n\star g)(x) = \int_{\R}g(x-t)f_n(t)\d t\xrightarrow[n\to+\infty]{}g(x)$ par théorème de convergence dominée.
\end{remark}

\begin{remark}
    Si $g$ est bornée et uniformément continue sur $\R$, soit $\varepsilon>0$ et $\alpha>0$ tel que si $\left\lvert t\right\rvert\leqslant\alpha$ alors pour tout $x\in\R$, $\left\lvert g(x-t)-g(x)\right\rvert\leqslant\frac{\varepsilon}{2}$. Alors 
    \begin{equation}
        \left\lvert (f_n\star g)(x)-g(x)\right\rvert\leqslant\int_{-\alpha}^{\alpha}\left\lvert g(x-t)-g(x)\right\rvert f_n(t)\d t+\int_{\R\setminus[-\alpha, \alpha]}2\left\lVert g\right\rVert_{\infty}f_n(t)\d t.
    \end{equation}
    Le deuxième terme tend vers 0 quand $n\to+\infty$, donc $(f_n\star g)_{n\in\N}$ converge uniformément vers $g$ sur $\R$.
\end{remark}

\begin{remark}
    Soit $f\colon\R\to\R_{+}$ continue par morceaux telle que $\int_{\R}f=1$. Soit pour $n\geqslant1$, \function{f_n}{\R}{\R^{+}}{t}{nf(nt)}
    Par changement de variable, on a $\int_{\R}f_n=1$ et $\lim\limits_{n\to+\infty}\int_{\alpha}^{+\infty}f=0$ pour $\alpha>0$. $(f_n)_{n\in\N}$ est une approximation de l'unité.
\end{remark}

\begin{proof}
    Si $x\geqslant2$, on a $\frac{1}{x}\in]0,1]$ donc on peut définir \function{f}{[1,+\infty[}{\R}{x}{\frac{1}{x}-\arcsin\left(\frac{1}{x}\right)}
    $f$ est continue et $\arcsin(t)\underset{t\to0}{=}t+\frac{t^{3}}{6}+o(t^{3})$ implique $f(x)\underset{x\to+\infty}{\sim}\frac{-1}{6x^{3}}$, donc d'après le critère de Riemann, $\int_{1}^{+\infty}f$ converge.

    Soit $A\geqslant1$, on pose $I_{A}=\int_{1}^{A}\frac{1}{x}-\arcsin\left(\frac{1}{x}\right)\d x=\ln(A)-\int_{1}^{A}\arcsin\left(\frac{1}{x}\right)\d x$. On a 
    \begin{align}
        \int_{1}^{A}\arcsin\left(\frac{1}{x}\right)\d x
        &=[x\arcsin\left(\frac{1}{x}\right)]_{1}^{A}+\int_{1}^{A}\frac{1}{\sqrt{x^{2}-1}}\d x,\\
        &=\arcsin\left(\frac{1}{A}\right)+\ln(A+\sqrt{A^{2}-1})-\frac{\pi}{2},\\
        &\underset{A\to+\infty}{=}1+\ln(A)+\ln(2)-\frac{\pi}{2}+o(1),
    \end{align}
    donc 
    \begin{equation}
        \boxed{
            I=\lim\limits_{A\to+\infty}=-1+\frac{\pi}{2}-\ln(2).
        }
    \end{equation}
\end{proof}

\begin{proof}
    On a $\ln(\sin(t))\underset{t\to0}{\sim}\ln(t)\underset{t\to0}{=}O\left(\frac{1}{\sqrt{t}}\right)$ donc $I$ existe, et en posant $u=\frac{\pi}{2}-t$, on a $I=J$. On a 
    \begin{align}
        I+J
        &=\int_{0}^{\frac{\pi}{2}}\ln\left(\frac{\sin(2t)}{2}\right)\d t,\\
        &=\int_{0}^{\frac{\pi}{2}}\ln(\sin(2t))\d t-\int_{0}^{\frac{\pi}{2}}\ln(2)\d t,\\
        &=\frac{1}{2}\int_{0}^{\pi}\ln(\sin(u))\d u-\frac{\pi}{2}\ln(2),\\
        &=I+\int_{\frac{\pi}{2}}^{\pi}\ln(\sin(u))\d u-\frac{\pi}{2}\ln(2).
    \end{align}
    On a 
    \begin{equation}
        I+\int_{\frac{\pi}{2}}^{\pi}\ln(\sin(u))\d u=\frac{1}{2}I+\frac{1}{2}\int_{0}^{\frac{\pi}{2}}\ln(\sin(v))\d v=I.
    \end{equation}
    Finalement, on a $I+J=I-\frac{\pi}{2}\ln(2)$ donc 
    \begin{equation}
        \boxed{
            I=J=-\frac{\pi}{2}\ln(2).
        }
    \end{equation}
\end{proof}

\begin{proof}
    $f_\alpha$ est positive, continue et $f_\alpha\leqslant1$. $f_\alpha$ est intégrable si et seulement si $\sum u_k$ converge avec 
        \begin{align}
            u_k
            &= \int_{k\pi-\frac{\pi}{2}}^{k\pi+\frac{\pi}{2}}\frac{\d x}{1+x^{\alpha}\left\lvert\sin(x)\right\rvert},\\
            &=\int_{-\frac{\pi}{2}}^{\frac{\pi}{2}}\frac{\d t}{1+(t+k\pi)^{\alpha}\left\lvert\sin(t)\right\rvert},\\
            &\leqslant\int_{-\frac{\pi}{2}}^{\frac{\pi}{2}}\frac{\d t}{1+\left(k\pi-\frac{\pi}{2}\right)^{\alpha}\frac{2}{\pi}\left\lvert t\right\rvert},\\
            &=2\int_{0}^{\frac{\pi}{2}}\frac{\d t}{1+\left(k\pi-\frac{\pi}{2}\right)^{\alpha}\frac{2}{\pi}t},\\
            &=\frac{\pi}{\left(k\pi-\frac{\pi}{2}\right)^{\alpha}}\ln\left(1+\left(k\pi-\frac{\pi}{2}\right)^{\alpha}\right),\\
            &\underset{k\to+\infty}{\sim}\frac{\alpha\ln(k)}{(k\pi)^{\alpha}}\underset{k\to+\infty}{=}O\left(\frac{1}{k^{\frac{1+\alpha}{2}}}\right).
        \end{align}
        Donc $\sum u_k$ converge et $\int_{0}^{+\infty}f_\alpha(t)\d t$ converge.
\end{proof}

\begin{proof}
    \phantom{}
    \begin{enumerate}
        \item Pour tout $P\in\R[X]$, $\int_{a}^{b}P(t)f(t)\d t=0$. D'après le théorème de Weierstrass, il existe $(P_n)\in\left(\R[X]\right)^{\N}$ telle que $\left\lVert P_n-f\right\rVert_{\infty}\xrightarrow[n\to+\infty]{}0$ sur $[a,b]$. $(P_nf)_{n\in\N}$ converge uniformément vers $f^{2}$ sur $[a,b]$, donc pour tout $n\in\N$, $\int_{a}^{b}P_nf=0$ donne par convergence uniforme $\int_{a}^{b}f^{2}=0$. Comme $f^{2}$ est continue positive, on a $f^{2}=0$ donc $f=0$.
        
        \item Pour tout $n\geqslant1$, on a par intégration par parties,
        \begin{equation}
            I_n=\frac{n}{1-\i}I_{n-1}.
        \end{equation}
        On a $I_{0}=\frac{1}{1-\i}$. Par récurrence, on a 
        \begin{equation}
            I_n=\frac{n!}{(1-\i)^{n+1}}=\frac{n!}{\left(\sqrt{2}\right)^{n+1}}\e^{\i(n+1)\frac{\pi}{4}}.
        \end{equation}
        
        \item Pour $n=4k-1$ pour $k\in\N^{*}$, on a 
        \begin{equation}
            \Im(I_{4k-1})=0=\int_{0}^{+\infty}t^{4k-1}\sin(t)\e^{-t}\d t.
        \end{equation}
        On pose $u=t^{4}$, $t=u^{\frac{1}{4}}$ et $\d t=\frac{1}{4}u^{-\frac{3}{4}}\d u$. Ainsi, en posant $f(u)=\sin\left(u^{\frac{1}{4}}\right)\e^{-u^{\frac{1}{4}}}$,
        \begin{equation}
            \boxed{
                0=\int_{0}^{+\infty}f(u)u^{k-1}\d u.
            }
        \end{equation}
    \end{enumerate}
\end{proof}

\begin{proof}
    \phantom{}
    \begin{enumerate}
        \item $g$ est continue, $\mathcal{C}^{1}$ par morceaux et en tout point de continuité de $f$, on a $g'(t)=\e^{-at}f(t)$. On a $g(0)=0$ et $\lim\limits_{t\to+\infty}g(t)=\mathcal{L}f(a)$. 
        
        Soit $X\geqslant0$, on a grâce à une intégration par parties,
        \begin{align}
            \int_{0}^{X}\e^{-bt}f(t)\d t
            &=\int_{0}^{X}\e^{-(b-a)t}\e^{-at}f(t)\d t,\\
            &=\left[g(t)\e^{-(b-a)t}\right]_{0}^{X}+(b-a)\int_{0}^{X}\e^{-(b-a)t}g(t)\d t.
        \end{align}
        Le terme entre crochet s'annule car $g(0)=0$, et $b>a$ donc $g(X)\xrightarrow[X\to+\infty]{}\mathcal{L}f(a)$. $g$ est continue, admet une limite finie en $+\infty$, donc est bornée sur $\R_{+}$. Ainsi,
        \begin{equation}
            \left\lvert\e^{-(b-a)t}g(t)\right\rvert\leqslant\left\lVert g\right\rVert_{\infty}\e^{-(b-a)t},
        \end{equation}
        qui est intégrable sur $\R_{+}$. Finalement, 
        \begin{equation}
            \int_{0}^{+\infty}\e^{-(b-a)}g(t)\d t,    
        \end{equation}
        converge absolument et $\int_{0}^{+\infty}\e^{-bt}f(t)\d t$ converge et 
        \begin{equation}
            \boxed{
                \mathcal{L}f(b)=(b-a)\int_{0}^{+\infty}\e^{-(b-a)t}g(t)\d t.
            }
        \end{equation}

        \item En raisonnant sur $f-h$, on se ramène à $\mathcal{L}f=0$. Pour tout $b>a$, $\int_{0}^{+\infty}\e^{-(b-a)t}g(t)\d t=0$, donc pour tout $x>0$, $\int_{0}^{+\infty}\e^{-xt}g(t)\d t=0$. Si $g=0$, alors en dérivant, on a $f=0$. On pose $u=\e^{-t}$ qui est un $\mathcal{C}^{1}$ difféomorphisme de $[0,+\infty[\to]0,1]$. On a donc, pour tout $x>0$, $\int_{0}^{1}u^{x-1}g\left(-\ln(u)\right)\d u=0$. Donc pour tout $n\in\N$, $\int_{0}^{1}u^{n}g\left(-\ln(u)\right)\d u=0=\int_{0}^{1}u^{n}k(u)\d u$ avec $k\colon[0,1]\to\R$ définie par $k(0)=\mathcal{L}f(a)$ et $k(x)=g\left(-\ln(x)\right)$ si $x\in]0,1]$. $k$ est continue sur $[0,1]$. D'après le théorème de Weierstrass, $k=0$ donc $g=0$ puis $f=0$.
    \end{enumerate}
\end{proof}

\begin{proof}
    \phantom{}
    \begin{enumerate}
        \item Pour tout $(x,t)\in\R^{2}$, $\left\lvert\e^{\i xt}f(t)\right\rvert=\left\lvert f(t)\right\rvert$ est intégrable sur $\R$ car $f(t)\underset{\left\lvert t\right\rvert\to+\infty}{\sim}0$. $\widehat{f}$ est définie et $\widehat{f}(x)=\int_{-A}^{A}\e^{\i tx}f(t)\d t$.
        
        Posons \function{g}{\R^{2}}{\C}{(x,t)}{\e^{\i tx}f(t)}
        Pour tout $t\in\R$, $x\mapsto g(x,t)$ est $\mathcal{C}^{\infty}$ et pour tout $k\in\N$, $\frac{\partial^{k}g}{\partial x^{k}}=(\i t)^{k}g(x,t)$. On a 
        \begin{equation}
            \left\lvert\frac{\partial^{k}g(x,t)}{\partial x^{k}}\right\rvert=\left\lvert t\right\rvert^{k}\left\lvert f(t)\right\rvert,
        \end{equation}
        majoration indépendante de $x$ et intégrable sur $[-A,A]$. Donc $\widehat{f}$ est $\mathcal{C}^{\infty}$ et pour tout $k\in\N$, pour tout $x\in\R$, $\widehat{f}^{(k)}(x)=\int_{-A}^{A}(\i t)^{k}\e^{\i tx}f(t)\d t$.

        \item Soit $B>0$ tel que si $x>B$, $\widehat{f}(x)=0$. Soit $x_{0}=B+1$, alors $\widehat{f}=0$ sur $]x_{0}-1,+\infty[$. Pour tout $k\in\N$, on a $\widehat{f}^{(k)}(x_{0})=0=\int_{-A}^{A}t^{k}\e^{\i tx_{0}}f(t)\d t$. D'après le théorème de Weierstrass, on a $f(t)=0$ pour tout $t\in[-A,A]$.
    \end{enumerate}
\end{proof}

\begin{proof}
    Si $f$ est affine avec $f(x)=\alpha x+\beta$. On a $\int_{a}^{b}f(t)\d t=\alpha\frac{(b^{2}-a^{2})}{2}+\beta(b-a)$ et $(b-a)f\left(\frac{a+b}{2}\right)=(b-a)\left(\alpha\frac{(a+b)}{2}+\beta\right)$ d'où $(b-a)f\left(\frac{a+b}{2}\right)=\int_{a}^{b}f(t)\d t$.

    Notons que l'inégalité de l'énoncé équivaut à pour tout $x\in\mathring{I}$, pour tout $h>0$, on a $a=x-h$ et $b=x+h\in I^{2}$, $2hf(x)\leqslant\int_{x-h}^{x+h}f(t)\d t$.

    Si $f$ est convexe, soit $a<b\in I^{2}$. Soit $\varphi$ affine sur $[a,b]$ telle que $\varphi\left(\frac{a+b}{2}\right)=f\left(\frac{a+b}{2}\right)$. On a 
    \begin{equation}
        \varphi'=\lambda=\frac{1}{2}\left(f'_{g}\left(\frac{a+b}{2}\right)+f'_{d}\left(\frac{a+b}{2}\right)\right)\geqslant f'_{g}\left(\frac{a+b}{2}\right),
    \end{equation}
    par convexité et en notant $\varphi(t)=\lambda t+\mu$. En notant $\varphi_{1}$ la demi-tangente à $f$ en $\frac{a+b}{2}$, on a pour tout $t\in\left[a,\frac{a+b}{2}\right]$,
    \begin{equation}
        \varphi(t)\leqslant\varphi_1(t)\leqslant f(t).
    \end{equation}
    $\varphi_{1}$ est affine sur $\left[a,\frac{a+b}{2}\right]$, $\varphi_{1}\left(\frac{a+b}{2}\right)=f\left(\frac{a+b}{2}\right)$ et $\varphi_{1}'\left(\frac{a+b}{2}\right)=f'_{g}\left(\frac{a+b}{2}\right)$. De la même façon, pour tout $t\in\left[\frac{a+b}{2},b\right]$, on a 
    \begin{equation}
        \varphi(t)\leqslant\varphi_2(t)\leqslant f(t),
    \end{equation}
    avec $\varphi_{2}$ affine sur $\left[\frac{a+b}{2},b\right]$, $\varphi_{2}\left(\frac{a+b}{2}\right)=f\left(\frac{a+b}{2}\right)$ et $\varphi'_{2}\left(\frac{a+b}{2}\right)=f'_{d}\left(\frac{a+b}{2}\right)$. 
    
    On a donc $\int_{a}^{b}\varphi(t)\d t=(b-a)f\left(\frac{a+b}{2}\right)\leqslant\int_{a}^{b}f$.

    Réciproquement, si pour tout $a<b$, $(b-a)f\left(\frac{a+b}{2}\right)\leqslant\int_{a}^{b}f(t)\d t$, soient $x<y\in I^{2}$ fixés. On pose $g=f-\varphi$ avec $\varphi(z)=\frac{f(y)-f(x)}{y-x}(z-x)+f(x)$. $g$ vérifie l'inégalité de l'énoncé car pour $\varphi$ on a égalité (car affine). On veut montrer que $g\leqslant0$ sur $[x,y]$. On a $g(x)=g(y)=0$. Soit $g(x_{0})=\max\limits_{t\in[x,y]}g(t)$. Si $g(x_{0})>0$, on a $x_{0}\in]x,y[$ car $g(x)=g(y)=0$. Soit $h>0$ tel que $x_{0}-h$ et $x_{0}+h\in[x,y]$. On applique l'inégalité de l'énoncé à $g$:
    \begin{equation}
        2hg(x_{0})\leqslant\int_{x_{0}-h}^{x_{0}+h}g(t)\d t=2g(x_{0}),
    \end{equation}
    donc 
    \begin{equation}
        \int_{x_{0}-h}^{x_{0}+h}(g(x_{0})-g(t))\d t=0,
    \end{equation}
    et l'intégrande est positive et continue. Donc pour tout $t\in[x_{0}-h,x_{0}+h]$, on a $g(t)=g(x_{0})$. On pose $h=\min(y-x_{0},x-x_{0})$. On obtient $g(x)=0=g(x_{0})>0$ (ou $g(y)=g(x_{0})$) ce qui est absurde. Donc $g\leqslant0$ sur $[x,y]$ et $f$ est convexe.
\end{proof}

\begin{remark}
    Notons que si pour tout $(h,x)\in\R_{+}\times\mathring{I}$ tels que $(x-h,x+h)\in I^{2}$, $2hf(x)=\int_{x-h}^{x+h}f(t)\d t$, alors pour $x\in\mathring{I}$ et $h$ fixé, $x\mapsto\int_{x-h}^{x+h}f$ est $\mathcal{C}^{1}$ donc $f$ l'est. Par récurrence, $f$ est $\mathcal{C}^{\infty}$, et en dérivant deux fois par rapport à $h$ (pour $x\in \mathring{I}$ fixé), on a 
    $0=f'(x+h)-f'(x-h)$ donc $f$ est affine.
\end{remark}

\begin{proof}
    \phantom{}
    \begin{enumerate}
        \item Soit $f_n\colon]0,+\infty[\to\R$ définie par $f_n(t)=0$ si $t>n$ et $f_n(t)=\left(1-\frac{t}{n}\right)^{n}t^{x-1}$ si $t\leqslant n$. $f_n$ est continue par morceaux, positive, intégrable sur $\R_{+}$ car équivalente à 0 en $+\infty$ et à $t^{x-1}$ en 0.
        
        Soit $t\in]0,+\infty[$ fixé, il existe $N_{0}\in\N$ tel que pour tout $n\geqslant N_{0}$, $n>t$ et pour tout $n\geqslant N_{0}$, 
        \begin{align}
            f_n(t)
            &=\left(1-\frac{t}{n}\right)^{n}t^{x-1},\\
            &=\e^{n\ln\left(1-\frac{t}{n}\right)}t^{x-1},\\
            &\underset{n\to+\infty}{=}\e^{n\left(-\frac{t}{n}+o\left(\frac{1}{n}\right)\right)}t^{x-1},\\
            &\xrightarrow[n\to+\infty]{}\e^{-t}t^{x-1}.
        \end{align}

        On a donc convergence simple vers $f(t)=\e^{-t}t^{x-1}$, fonction continue sur $\R_{+}^{*}$ intégrable sur $\R_{+}^{*}$. Soit $n\geqslant1$ et $t\in[0,n[$, on a 
        \begin{equation}
            0\leqslant f_n(t)\leqslant f(t),
        \end{equation}
        car $\ln(1+x)\leqslant x$ pour tout $x>-1$.
        D'après le théorème de convergence dominée, on a bien
        \begin{equation}
            \boxed{
                \lim\limits_{n\to+\infty}I_n=\lim\limits_{n\to+\infty}\int_{0}^{+\infty}f_n(t)\d t=\int_{0}^{+\infty}f(t)\d t.
            }
        \end{equation}

        \item On pose $u=\frac{t}{n}$ et on a 
        \begin{align}
            I_n(x)
            &= \int_{0}^{1}(1-u)^{n}(nu)^{x-1}n\d u,\\
            &=n^{x}\int_{0}^{1}(1-u)^{n}u^{x-1}\d u,\\
            &=n^{x}\left(\left[(1-u)^{n}\frac{u^{x}}{x}\right]_{0}^{1}+\int_{0}^{1}n(1-u)^{n-1}\frac{u^{x}}{x}\d u\right).
        \end{align}
        Le terme entre crochets est nul car $u\geqslant1$ et $x>0$.

        Si on pose $B_n(x)=\int_{0}^{1}(1-u)^{n}u^{x-1}\d u$, on a 
        \begin{equation}
            B_n(x)=\frac{n}{x}B_{n-1}(x+1)=\dots=\frac{n!}{x(x+1)\dots(x+n-1)}B_{0}(x+n)=\frac{n!}{x(x+1)\dots(x+n)}.
        \end{equation}
        On a donc 
        \begin{equation}
            \boxed{
                \Gamma(x)=\lim\limits_{n\to+\infty}\frac{n!n^{x}}{x(x+1)\dots(x+n).}
            }
        \end{equation}

        \item Par définition, on a $\gamma=\lim\limits_{n\to+\infty}\left(1+\frac{1}{2}+\dots+\frac{1}{n}-\ln(n)\right)$. On a 
        \begin{align}
            \frac{x(x+1)\dots(x+n)}{1\times2\times\dots\times n}\e^{-x\ln(n)}
            &\underset{n\to+\infty}{=} x\left[\prod_{k=1}^{n}\left(1+\frac{x}{k}\right)\right]\e^{-x\left(1+\frac{1}{2}+\dots+\frac{1}{n}+o(1)\right)}\e^{\gamma x},\\
            &\underset{n\to+\infty}{=} x\e^{\gamma x}\prod_{k=1}^{n}\left(1+\frac{x}{k}\right)\e^{-\frac{x}{k}}\times\underbrace{\e^{o(1)}}_{\xrightarrow[n\to+\infty]{}1}.
        \end{align}
        Donc on a 
        \begin{equation}
            \frac{1}{\Gamma(x)}=x\e^{\gamma x}\prod_{k=1}^{+\infty}\left(1+\frac{x}{k}\right)\e^{-\frac{x}{k}}.
        \end{equation}

        \item On remarque que $\frac{\Gamma'(x)}{\Gamma(x)}=(\ln\Gamma)'(x)$. On a 
        \begin{equation}
            \ln\Gamma(x)=-\ln(x)-\gamma x+\sum_{k=1}^{+\infty}\underbrace{\left(\frac{x}{k}-\ln\left(1+\frac{x}{k}\right)\right)}_{f_k(x)\geqslant0}.
        \end{equation}
        On a convergence simple sur $\R_{+}^{*}$. On a pour tout $k>1$, $f_{k}$ est $\mathcal{C}^{1}$, et pour tout $k\geqslant1$, pour tout $x\in]0,+\infty[$, $f'_k(x)=\frac{1}{k}-\frac{1}{k+x}\geqslant0$ car $x>0$.

        Soit $A>0$, pour tout $x\in]0,A]$, on a 
        \begin{equation}
            0<f_k(x)\leqslant\frac{1}{k}-\frac{1}{k+A}=\frac{A}{k(k+A)}\underset{k\to+\infty}{O}\left(\frac{1}{k^{2}}\right).
        \end{equation}
        Donc pour tout $A>0$, $\sum f_k(x)$ converge normalement sur $]0,A]$. $\ln\Gamma$ est donc $\mathcal{C}^{1}$ (en tant que série de fonction) et on a 
        \begin{equation}
            \boxed{
            \left(\frac{\Gamma'}{\Gamma}\right)(x)=-\frac{1}{x}-\gamma+\sum_{k=1}^{+\infty}\left(\frac{1}{k}-\frac{1}{k+x}\right).}
        \end{equation}
    \end{enumerate}
\end{proof}

\begin{remark}
    En particulier, comme $\Gamma(1)=1$, on a 
        \begin{equation}
            \Gamma'(1)=\left(\frac{\Gamma'}{\Gamma}\right)(1)=-1-\gamma+\sum_{k=1}^{+\infty}\left(\frac{1}{k}-\frac{1}{k+1}\right)=-\gamma,
        \end{equation}
    car la série est téléscopique.
\end{remark}

\begin{remark}
    On a 
    \begin{equation}
        \left(\ln\Gamma\right)''(x)=\frac{\Gamma''\Gamma-\Gamma'^{2}}{\Gamma^{2}}.
    \end{equation}
    Par ailleurs,
    \begin{equation}
        0\leqslant\Gamma'^{2}(x)=\left(\int_{0}^{+\infty}\ln(t)t^{x-1}\e^{-t}\d t\right)^{2}\leqslant\int_{0}^{+\infty}\ln^{2}(t)t^{x-1}\e^{-t}\d t\times\int_{0}^{+\infty}t^{x-1}\e^{-t}\d t=\Gamma''(x)\Gamma(x),
    \end{equation}
    d'après l'inégalité de Cauchy-Schwarz. Ona égalité stricte car $\ln(t)$ n'est pas constante. Ainsi, $\ln\Gamma$ est strictement convexe.
\end{remark}

\begin{remark}
    On peut vérifier que $\ln\Gamma$ est l'unique fonction de $\R_{+}^{*}\to\R$ telle que 
    \begin{enumerate}
        \item $\ln\Gamma$ est convexe,
        \item $\forall x>0$, $\left(\ln\Gamma\right)(x+1)=\left(\ln\Gamma\right)(x)+\ln(x)$,
        \item $\left(\ln\Gamma\right)(1)=0$.
    \end{enumerate}
\end{remark}

\begin{proof}
    \phantom{}
    \begin{enumerate}
        \item On a 
        \begin{equation}
            \e^{2\i\pi d(f)}=\exp\left(\int_{0}^{2\pi}\frac{f'(t)}{f(t)}\d t\right).
        \end{equation}
        Posons $g(x)=\exp\left(\int_{0}^{x}\frac{f'(t)}{f(t)}\d t\right)$. $g$ est $\mathcal{C}^{1}$ sur $\R$. Pour tout $x\in\R$, on a $f'(x)=\frac{f'(x)}{f(x)}g(x)$. On a $\left(\frac{g}{f}\right)'=\frac{g'f-f'g}{g^{2}}=0$ donc $\frac{g}{f}=\alpha\in\C$. En particulier, $g(0)=g(2\pi)=1$ donc $d(f)\in\Z$.

        \item $f_0$ est constante égale à $P(0)$ donc $d(f_0)=0$ car c'est une fonction constante.
        Soit $r\geqslant0$, on a 
        \begin{equation}
            d(f_r)=\frac{1}{2\i\pi}\int_{0}^{2\pi}\frac{f_r'(t)}{f_r(t)}\d t=\frac{1}{2\i\pi}\int_{0}^{2\pi}\frac{\i r\e^{\i t}P'(r\e^{\i t})}{P(r\e^{\i t})}\d t.
        \end{equation}
        On note $g(r,t)$ la fonction intégrande définit sur $\R_{+}\times[0,2\pi]\to\C$. $r\mapsto g(r,t)$ est continue sur $\R_{+}$, et soit $z\in\C$, on a 
        \begin{equation}
            P(z)=a_{0}+a_{1}z+\dots+a_nz^{n}.
        \end{equation}
        Alors 
        \begin{equation}
            h(z)=\frac{zP'(z)}{P(z)}=\frac{a_1 z+\dots+na_nz^{n}}{a_0+a_1z+\dots+a_nz^{n}}
        \end{equation}
        est continue sur $\C$ et pour $z\neq0$, on a 
        \begin{equation}
            \frac{z P'(z)}{P(z)}=\frac{\frac{a_1}{z^{n-1}}+\dots+na_n}{\frac{a_0}{z^{n}}+a_n}\xrightarrow[\left\lvert z\right\rvert\to+\infty]{}n.
        \end{equation}
        Donc $h$ est bornée sur $\C$, soit $M=\left\lVert h\right\rVert_{\infty}$. On a $\left\lvert g(r,t)\right\rvert\leqslant M\in L^{1}\left([0,2\pi]\right)$. Donc $r\mapsto d(f_r)$ est continue et pour $t$ fixé, on a $\lim\limits_{r\to+\infty}g(r,t)=n$. Par convergence dominée, on a $\lim\limits_{r\to+\infty}d(f_r)=n$. $r\mapsto d(f_r)$ est continue à valeurs dans $\Z$ donc constante et $d(f_{0})=0$, $\lim\limits_{r\to+\infty}d(f_r)=n\neq0$: c'est absurde. Donc $P$ s'annule.
    \end{enumerate}
\end{proof}

\begin{remark}
    Le théorème de relèvement permet d'écrire $f(t)=\rho(t)\e^{\i\theta(t)}$ avec $\rho(t)=\left\lvert f(t)\right\rvert$ et $(\rho,\theta)\colon\R\to\left(\R_{+}^{*},\R\right)$ est $\mathcal{C}^{1}$. On a alors 
    \begin{equation}
        \int_{0}^{2\pi}\frac{f'}{f}=\int_{0}^{2\pi}\frac{\rho'}{\rho}+\i\left(\theta(2\pi)-\theta(0)\right).
    \end{equation}
    Le premier terme vaut $\left[\ln(\rho)\right]_{0}^{2\pi}=0$ car $\rho=\left\lvert f\right\rvert$ est $2\pi$-périodique, et le deuxième terme vaut $2\i\pi\times$ le nombre de tours que décrit $f$ autour de l'origine.
\end{remark}

\begin{proof}
    En appliquant l'inégalité de Taylor avec reste intégral à $f$ de classe $\mathcal{C}^{n}$, on a 
    \begin{equation}
        R_n=f(b)-f(a)-\sum_{k=1}^{n-1}\frac{f^{(k)}(a)}{k!}=\int_{a}^{b}\frac{(b-t)^{n-1}}{(n-1)!}f^{(n)}(t)\d t.
    \end{equation}
    Soit $m_n=\min\limits_{[a,b]}f^{(n)}$ et $M_n=\max\limits_{[a,b]}f^{(n)}$. Alors 
    \begin{equation}
        m_n\frac{\left\lvert b-a\right\rvert^{n}}{n!}\leqslant\left\lvert\int_{a}^{b}\frac{(b-t)^{n-1}}{(n-1)!}f^{(n)}(t)\d t\right\rvert\leqslant M_n\frac{\left\lvert b-a\right\rvert^{n}}{n!}.
    \end{equation}
    D'après le théorème des valeurs intermédiaires, il existe $\xi\in]a,b[$ tel que $R_n=\frac{(b-a)^{n}}{n!}f^{(n)}(\xi)$.

    On a 
    \begin{equation}
        v_n-\int_{0}^{1}=\sum_{k=0}^{n-1}\frac{1}{2n}\left[f\left(\frac{k}{n}\right)+f\left(\frac{k+1}{n}\right)\right]-\int_{\frac{k}{n}}^{\frac{k+1}{n}}f.
    \end{equation}

    On prend d'abord $a=\frac{k}{n}$ et $b=\frac{k+1}{n}$, il existe $\xi_k\in\left]\frac{k}{n},\frac{k+1}{n}\right[$ tel que 
    \begin{equation}
        F\left(\frac{k+1}{n}\right)-F\left(\frac{k}{n}\right)=\frac{1}{n}f\left(\frac{k}{n}\right)+\frac{1}{2n^{2}}f'\left(\frac{k}{n}\right)+\frac{1}{6n^{3}}f^{(2)}(\xi_k).
    \end{equation}
    Puis avec $a=\frac{k+1}{n}$ et $b=\frac{k}{n}$, il existe $\eta_k\in\left]\frac{k}{n},\frac{k+1}{n}\right[$ tel que 
    \begin{equation}
        F\left(\frac{k}{n}\right)-F\left(\frac{k+1}{n}\right)=-\frac{1}{n}f\left(\frac{k+1}{n}\right)+\frac{1}{2n^{2}}f'\left(\frac{k+1}{n}\right)-\frac{1}{6n^{3}}f^{(2)}(\eta_k).
    \end{equation}

    En faisant la différence des deux égalités et en divisant par deux, on a 
    \begin{align}
        F\left(\frac{k+1}{n}\right)-F\left(\frac{k}{n}\right)
        &=\int_{\frac{k}{n}}^{\frac{k+1}{n}}f(t)\d t,\\
        &=\frac{1}{2n}\left(f\left(\frac{k+1}{n}\right)+f\left(\frac{k}{n}\right)\right)+\frac{1}{4n^{2}}\left(f'\left(\frac{k}{n}\right)-f'\left(\frac{k+1}{n}\right)\right),\\
        &\qquad+\frac{1}{12n^{3}}\left(f^{(2)(\xi_k)}+f^{(2)}(\eta_k)\right).
    \end{align}
    En sommant, on obtient (par le théorème de Riemann car on a une subdivision pointée)
    \begin{align}
        v_n-\int_{0}^{1}f
        &=\sum_{k=0}^{n-1}\frac{1}{4n^{2}}\left(f'\left(\frac{k+1}{n}\right)-f'\left(\frac{k}{n}\right)\right)-\frac{1}{12n^{3}}\left(f^{(2)}(\xi_k)+f^{(2)}(\eta_k)\right),\\
        &\underset{n\to+\infty}{=}\frac{1}{4n^{2}}\left[f'(1)-f'(0)\right]-\frac{1}{6n^{2}}\left[\int_{0}^{1}f'^{(2)}(t)\d t+o(1)\right],\\
        &\underset{n\to+\infty}{=}\frac{1}{4n^{2}}\left[f'(1)-f'(0)\right]-\frac{1}{6n^{2}}\left(f'(1)-f'(0)+o(1)\right).
    \end{align}

    Donc on a
    \begin{equation}
        \boxed{
            v_n=\int_{0}^{1}f+\frac{1}{12n^{2}}\left(f'(1)-f'(0)\right)+\underset{n\to+\infty}{o}\left(\frac{1}{n^{2}}\right).
        }
    \end{equation}
\end{proof}

\begin{proof}
    On a 
    \begin{align}
        I_n
        &=\int_{0}^{\frac{\pi}{4}}\left(\tan^{2}(x)+1-1\right)\tan^{n-2}(x)\d x,\\
        &=\int_{0}^{\frac{\pi}{4}}\left(\tan'(x)-1\right)\tan^{n-2}(x)\d x,\\
        &=\frac{1}{n-1}\left[\tan^{n-1}(x)\right]_{0}^{\frac{\pi}{4}}-I_{n-2}.
    \end{align}
    Donc
    \begin{equation}
        I_n+I_{n-2}=\frac{1}{n-1}.
    \end{equation}
    On a $I_0=\frac{\pi}{4}$ et $I_1=\left[-\ln\left\lvert\cos\right\rvert\right]_{0}^{\frac{\pi}{4}}=\frac{1}{2}\ln(2)$.

    On a donc 
    \begin{equation}
        I_{2p}=(-1)^{p}\left(\frac{\pi}{4}-1+\frac{1}{3}-\dots+\frac{(-1)^{p}}{2p+1}\right),
    \end{equation}
    et
    \begin{equation}
        I_{2p+1}=\frac{(-1)^{p}}{2}\left(\ln(2)-1+\frac{1}{2}-\frac{1}{3}+\dots+\frac{(-1)^{p}}{p}\right).
    \end{equation}

    D'après le théorème de convergence dominée, on a $I_n\xrightarrow[n\to+\infty]{}0$ donc 
    \begin{equation}
        \boxed{
            \frac{\pi}{4}=\sum_{k=0}^{+\infty}\frac{(-1)^{k}}{2k+1}.
        }
    \end{equation}

    Comme on a $2I_{n}\leqslant I_{n}+I_{n-2}=\frac{1}{n-1}\leqslant 2I_{n-2}$ d'où 
    \begin{equation}
        \frac{1}{2(n+1)}\leqslant I_n\leqslant\frac{1}{2(n-1)},
    \end{equation}
    ainsi 
    \begin{equation}
        \boxed{
            I_n\underset{n\to+\infty}{\sim}\frac{1}{2n}.
        }
    \end{equation}
\end{proof}

\begin{proof}
    On note $g$ la fonction intégrande. $g$ est $\mathcal{C}^{\infty}$ sur $]0,1]$. On a 
    \begin{equation}
        f(x)=\int_{\frac{1}{2}}^{x}g(t)\d t-\int_{\frac{1}{2}}^{x^{2}}g(t)\d t,
    \end{equation}
    donc $f$ est $\mathcal{C}^{\infty}$ et $f'(x)=g(x)-2xg'(x^{2})$.

    En 0, $\e^{t}$ se comporte comme 1 et $\frac{1}{\arcsin(t)}$ comme en $\frac{1}{t}$. Donc, au voisinage de 0,
    \begin{equation}
        h(t)=\frac{\e^{t}}{\arcsin(t)}-\frac{1}{t}=\frac{t\e^{t}-\arcsin(t)}{t\arcsin(t)}\underset{n\to+\infty}{=}\frac{t^{2}+o(t^{2})}{t^{2}+o(t^{2})}\xrightarrow[t\to0^{+}]{}1.
    \end{equation}
    Donc $h$ est bornée sur $]0,1]$, soit $M=\sup\limits_{t\in]0,1]}\left\lvert h(t)\right\rvert$. On a 
    \begin{equation}
        \left\lvert\int_{x^{2}}^{x}h(t)\d t\right\rvert\leqslant\int_{x}^{x^{2}}\left\lvert h(t)\right\rvert \d t\leqslant M(x-x^{2})\xrightarrow[x\to0]{}0.
    \end{equation}
    Comme $\int_{x^{2}}^{x}\frac{\d t}{t}=-\ln(x)$, on a $f(x)\underset{x\to0}{=}-\ln(x)+o(1)$.

    Pour aller plus loin dans le développement limité, on pousse plus loin le développement limité de $h(t)$ dans $0^{+}$.
\end{proof}

\begin{proof}
    On note $f$ la fonction intégrande. $x\mapsto x^{2}+x+1$ ne s'annule pas sur $\R$. $f$ est continue sur $\R$ et positive, $f(x)\underset{x\to+\infty}{\sim}\frac{1}{x^{2n}}$. Donc d'après le critère de Riemann, l'intégrale converge absolument.

    Pour le calcul, on on 
    \begin{equation}
        x^{2}+x+1=\left(x+\frac{1}{2}\right)^{2}+\frac{3}{4}=\frac{3}{4}\left(\left(\frac{2}{\sqrt{3}}(x+1)\right)^{2}+1\right).
    \end{equation}
    On pose $u=\frac{2}{\sqrt{3}}(x+1)$ et on a 
    \begin{equation}
        I_n=\frac{\sqrt{3}}{2}\left(\frac{4}{3}\right)^{n}\int_{0}^{+\infty}\frac{\d u}{(1+u^{2})^{n}}.
    \end{equation}

    On note $J_n$ l'intégrale:
    \begin{equation}
        J_n=\int_{0}^{+\infty}\frac{1}{(u^{2}+1)^{n-1}}\frac{\d u}{1+u^{2}}.
    \end{equation}
    On pose $\theta=\arctan(u)$, $\mathcal{C}^{1}$-difféomorphisme de $[0,+\infty[\to[0,\frac{\pi}{2}[$. On a $\d\theta=\frac{\d u}{1+u^{2}}$ et $\frac{1}{(1+u^{2})^{n-1}}=\cos^{2n-2}(\theta)$.

    On retrouve les intégrales de Wallis, d'où on en tire 
    \begin{equation}
        J_n=\frac{(2n-1)!}{2^{2(n-1)}(n-1)!}\frac{\pi}{2}.
    \end{equation}
    Donc 
    \begin{equation}
        \boxed{
            I_n=\frac{\sqrt{3}}{2}\left(\frac{4}{3}\right)^{n}\frac{(2n-1)!}{2^{2(n-1)}(n-1)!}\frac{\pi}{2}.
        }
    \end{equation}
\end{proof}

\begin{proof}
    Soit $M_{0}=\sup\limits_{[0,1]}\left\lvert f\right\rvert$ et $M_{1}=\sup\limits_{[0,1]}\left\lvert f'\right\rvert$. D'après l'inégalité des accroissements finis, on a
    \begin{equation}
        \left\lvert f\left(\frac{i+1}{n}\right)-f\left(\frac{i}{n}\right)\right\rvert\leqslant\frac{M_1}{n}.
    \end{equation}
    Donc, par la formule de la somme de Riemann, on a 
    \begin{equation}
        \left\lvert u_n-\underbrace{\frac{1}{n}\sum_{i=0}^{n-1}f\left(\frac{i+1}{n}\right)f'\left(\frac{i+1}{n}\right)}_{\xrightarrow[n\to+\infty]{}\int_{0}^{1}ff'}\right\rvert\leqslant\underbrace{\frac{M_1^{2}}{n}}_{\xrightarrow[n\to+\infty]{}0},
    \end{equation}
    donc 
    \begin{equation}
        \boxed{
            u_n\xrightarrow[n\to+\infty]{}\frac{1}{2}\left(f^{2}(1)-f^{2}(0)\right).
        }
    \end{equation}
\end{proof}

\begin{proof}
    Ici, l'intégrale diverge, mais comme on fait tendre l'intervalle d'intégration à un singleton, cela aura une limite finie.

    Formons \function{f}{\R_{+}^{*}\setminus\left\lbrace1\right\rbrace}{\R}{t}{\frac{1}{\ln(t)}}
    Si $x<1$, $x^{a}$ et $x^{b}$ sont $<1$, et si $x>1$, alors $x^{a}$ et $x^{b}$ sont $>1$. $\int_{x^{a}}^{x^{b}}f(t)\d t$ est donc bien définie.

    On a 
    \begin{equation}
        \frac{1}{\ln(t)}-\frac{1}{t-1}=\frac{1}{\ln(1+(t-1))}-\frac{1}{t-1}=\frac{(t-1)-\ln(1+(t-1))}{(t-1)\ln(1+(t-1))}\underset{t\to1}{\sim}\frac{\frac{(t-1)^{2}}{2}}{(t-1)^{2}}=\frac{1}{2}.
    \end{equation}
    Soit $h\colon\R_{+}^{*}\to\R$ définie par $h(t)=\frac{1}{\ln(t)}-\frac{1}{t-1}$ si $t\neq1$ et $h(1)=\frac{1}{2}$. $h$ est continue donc bornée au voisinage de 1. Il existe $\alpha_{0}>0$ et $M_{0}\geqslant0$ tels que pour tout $t\in[1-\alpha_{0},1+\alpha_{0}]$, on ait 
    \begin{equation}
        \left\lvert \int_{x^{a}}^{x^{b}}\frac{\d t}{\ln(t)}-\int_{x^{a}}^{x^{b}}\frac{\d t}{t-1}\right\rvert\leqslant M_{0}\left\lvert x^{b}-x^{a}\right\rvert\xrightarrow[x\to1]{}0.
    \end{equation}

    Or, si $x=1+x'$, on a
    \begin{align}
        \int_{x^{a}}^{x^{b}}\frac{\d t}{t-1}
        &=\left[\ln\left\lvert t-1\right\rvert\right]_{x^{a}}^{x^{b}},\\
        &=\ln\left\lvert x^{b}-1\right\rvert-\ln\left\lvert x^{a}-1\right\rvert,\\
        &=\ln\left\lvert (1+x')^{b}-1\right\rvert-\ln\left\lvert (1+x')^{a}-1\right\rvert,\\
        &\underset{x'\to0}{=}\ln\left\lvert bx"+o(x')\right\rvert-\ln\left\lvert ax'+o(x')\right\rvert,\\
        &\underset{x'\to0}{=}\ln(b)+\ln(x')+o(1)-\left[\ln(a)+\ln(x')+o(1)\right],\\
        &\underset{x'\to0}{=}\ln\left(\frac{b}{a}\right)+o(1)\xrightarrow[x'\to0]{}\ln\left(\frac{a}{b}\right).
    \end{align}

    D'où le résultat.
\end{proof}

\begin{proof}
    \phantom{}
    \begin{enumerate}
        \item Soit $n\geqslant1$, on a 
        \begin{equation}
            S_n=\frac{b-a}{n}\sum_{k=0}^{n-1}f\left(a+k\frac{b-a}{n}\right)\xrightarrow[n\to+\infty]{}\int_{a}^{b}f.
        \end{equation}
        Par convexité, on a 
        \begin{equation}
            \varphi\left(\frac{1}{b-a}S_n\right)\leqslant\sum_{k=0}^{n-1}\frac{1}{n}\varphi\left(f\left(a+k\frac{b-a}{n}\right)\right),
        \end{equation}
        donc en passant à la limite $n\to+\infty$, par continuité, on a 
        \begin{equation}
            \boxed{
            \varphi\left(\frac{1}{b-a}\int_{a}^{b}f(t)\d t\right)\leqslant\frac{1}{b-a}\int_{a}^{b}\varphi\circ f(t)\d t.}
        \end{equation}

        \item Soit $c\in]a,b[$. En cas d'égalité dans ce qui précède, on a 
        \begin{align}
            \varphi\left(\frac{1}{b-a}\int_{a}^{b}f(t)\d t\right)
            &=\varphi\left(\frac{c-a}{b-a}\frac{1}{c-a}\int_{a}^{c}f(t)\d t+\frac{b-c}{b-a}\frac{1}{b-c}\int_{c}^{b}f(t)\d t\right),\\
            &\leqslant\frac{c-a}{b-a}\varphi\left(\frac{1}{c-a}\int_{a}^{c}f(t)\d t\right)+\frac{b-c}{b-a}\varphi\left(\frac{1}{b-c}\int_{c}^{b}f(t)\d t\right),\\
            &\leqslant\frac{1}{b-a}\left(\int_{a}^{c}\varphi\circ f(t)\d t+\int_{c}^{b}\varphi\circ f(t)\d t\right)=\frac{1}{b-a}\int_{a}^{b}\varphi\circ f(t)\d t,
        \end{align}
        par convexité et par ce qui précède. Par hypothèse, on a égalité partout, Par stricte convexité, on a 
        \begin{equation}
            \frac{1}{c-a}\int_{a}^{c}f(t)\d t=\frac{1}{b-c}\int_{c}^{b}f(t)\d t,
        \end{equation}
        d'où $(b-c)\int_{a}^{c}g(t)\d t=(c-a)\int_{c}^{b}f(t)\d t$. En dérivant par rapport à $c$, on obtient
        \begin{equation}
            (b-a)f(c)-\int_{a}^{c}f(t)\d t=-(c-a)f(c)+\int_{c}^{b}f(t)\d t,
        \end{equation}
        soit $(b-a)f(c)=\int_{c}^{b}f(t)\d t$ et $f(c)=\frac{1}{b-a}\int_{a}^{b}f(t)\d t$ pour tout $c\in]a,b[$. Donc $f$ est constante sur $[a,b]$.
    \end{enumerate}
\end{proof}

\begin{remark}
    Pour la première question, on aurait aussi pu passer par des fonctions en escaliers qui converge uniformément vers $f$.
\end{remark}

\begin{proof}
    Si $f=0$, ça marche. Sinon, il existe $y_0\in\R$ tel que $f(y_0)\neq0$ et donc, pour tout $x\in\R$, on a 
    \begin{equation}
        f(x)=\frac{1}{f(y_0)}\int_{x-y_0}^{x+y_0}f(t)\d t.
    \end{equation}
    Par récurrence, $f$ est $\mathcal{C}^{1}$ (d'après l'expression) et si $f$ est $\mathcal{C}^{k}$, alors elle est $\mathcal{C}^{k+1}$, donc $f$ est $\mathcal{C}^{\infty}$. Par ailleurs, $f(y_{0})f(-x)=-f(y_{0})f(y)$ donc $f$ est impaire. On dérive par rapport à $x$: $f(x+y)-f(x-y)=f'(x)f(y)$, et en dérivant à nouveau par rapport à $x$, on a $f'(x+y)-f'(x-y)=f''(x)f(y)$.

    Même chose par rapport à $y$: $f(x+y)-f(x-y)=f(x)f'(y)$ puis $f'-x+y)-f'(x-y)=f(x)f''(y)$. On pose alors $\alpha=\frac{f''(y_0)}{f(y_0)}$, on a $f''(x)-\alpha f(x)=0$.

    Si $\alpha=0$, comme $f$ est impaire, on a $f(x)=ax$ avec $a\in\R$ et en reportant, on a
    \begin{equation}
        \int_{x-y}^{x+y}f(t)\d t=a\left[\frac{u^{2}}{2}\right]_{x-y}^{x+y}=2axy.
    \end{equation}
    Or $f(x)f(y)=a^{2}xy$ donc ou bien $a=0$, ce qui est exclu, ou bien $a=2$.

    Si $\alpha>0$, on a $f(x)=a_{1}\sinh(\sqrt{\alpha}x)$. En reportant, on a 
    \begin{equation}
        \int_{x-y}^{x+y}f(t)\d t=\frac{2a_{1}}{\sqrt{\alpha}}\sinh(\sqrt{\alpha}x)\sinh(\sqrt{\alpha}y),
    \end{equation}
    et $f(x)=f(y)=a_{1}^{2}\sinh(\sqrt{\alpha}x)\sinh(\sqrt{\alpha}y)$ d'où $a_{1}=\frac{2}{\sqrt{\alpha}}$.

    Si $\alpha<0$, on trouve $f(x)=a_{2}\sin(\sqrt{-\alpha}x)$ avec $a_{2}=\frac{2}{\sqrt{-\alpha}}$.
\end{proof}

\begin{proof}
    Soit $f$ une fonction constante égale à $c$ sur $[a,b]$. On a alors $I_n=(b-a)^{\frac{1}{n}}\left\lvert c\right\rvert\xrightarrow[n\to+\infty]{}\left\lvert c\right\rvert$.

    Plus généralement, on a 
    \begin{equation}
        I_n\leqslant\left(\int_{a}^{b}\left\lVert f\right\rVert_{\infty}^{n}\right)^{\frac{1}{n}}=(b-a)^{\frac{1}{n}}\left\lVert f\right\rVert_{\infty}\xrightarrow[n\to+\infty]{}\left\lVert f\right\rVert_{\infty}.
    \end{equation}

    Soit $\varepsilon>0$, il existe $N_{1}\in\N$ tel que pour tout $n\geqslant N_1$, on a $I_{n}\leqslant\left\lVert f\right\rVert_{\infty}+\varepsilon$. $\left\lvert f\right\rvert$ est continue sur le compact $[a,b]$, donc il existe $t_{0}\in[a,b]$ tel que $\left\lvert f(t_{0})\right\rvert=\left\lVert f\right\rVert_{\infty}$. Par continuité de $\left\lvert f\right\rvert$, il existe $(c,d)\in[a,b]^{2}$ avec $c<d$ tel que pour tout $t\in[c,d]$, on ait $\left\lvert f(t)\right\rvert\geqslant\left\lVert f\right\rVert_{\infty}-\frac{\varepsilon}{2}$.

    On a alors
    \begin{equation}
        \int_{a}^{b}\left\lvert f\right\rvert^{n}\geqslant\int_{c}^{b}\left\lvert f\right\rvert^{n}\geqslant(d-c)\left(\left\lVert f\right\rVert_{\infty}-\frac{\varepsilon}{2}\right)^{n},
    \end{equation}
    donc 
    
    \begin{equation}
        I_n\geqslant\left(\left\lVert f\right\rVert_{\infty}-\frac{\varepsilon}{2}\right)(d-c)^{\frac{1}{n}}\xrightarrow[n\to+\infty]{}\left\lVert f\right\rVert_{\infty}-\frac{\varepsilon}{2}.
    \end{equation}

    Il existe donc $N_{2}\in\N$ tel que pour tout $n\geqslant N_{2}$, on a $I_{n}\geqslant\left\lVert f\right\rVert_{\infty}-\varepsilon$. Donc pour tout $n\geqslant\max(N_{1},N_{2})$, on a 
    \begin{equation}
        \left\lVert f\right\rVert_{\infty}-\varepsilon\leqslant I_{n}\leqslant \left\lVert f\right\rVert_{\infty}+\varepsilon,
    \end{equation}
    donc 
    \begin{equation}
        \boxed{
            \lim\limits_{n\to+\infty}I_{n}=\left\lVert f\right\rVert_{\infty}.
        }
    \end{equation}
\end{proof}

\begin{remark}
    Soit $u_n=I_{n}^{n}$ avec $f$ continue non nulle. On a $u_n>0$ et $u_n^{\frac{1}{n}}\xrightarrow[n\to+\infty]{}\left\lVert f\right\rVert$. Avec l'inégalité de Cauchy-Schwarz appliquée à $\left\lvert f\right\rvert^{\frac{n}{2}}$ et $\left\lvert f\right\rvert^{\frac{n}{2}+1}$, on a 
    \begin{equation}
        0<u_{n+1}=\int_{a}^{b}\left\lvert f\right\rvert^{n+1}\leqslant\sqrt{u_n}\sqrt{u_n+2}
    \end{equation}
    d'où 
    \begin{equation}
        \frac{u_{n+1}}{u_n}\leqslant\frac{u_{n+2}}{u_{n+1}}.
    \end{equation}

    $\left(\frac{u_{n+1}}{u_{n}}\right)_{n\geqslant0}$ est croissante et strictement positive, donc converge vers $l\in\overline{\R_{+}^{*}}$. On a 
    \begin{equation}
        \ln(u_{n+1})-\ln(u_n)=\ln\left(\frac{u_{n+1}}{u_{n}}\right)\xrightarrow[n\to+\infty]{}\ln(l).
    \end{equation}

    D'après le théorème de Césaro, on a donc 
    \begin{equation}
        \frac{\ln(u_{n})}{n}\xrightarrow[n\to+\infty]{}\ln(l),
    \end{equation}
    d'où $\ln\left(u_{n}^{\frac{1}{n}}\right)\xrightarrow[n\to+\infty]{}\ln\left\lVert f\right\rVert_{\infty}=\ln(l)$ par unicité de la limite.

    Donc 
    \begin{equation}
        \lim\limits_{n\to+\infty}\frac{u_{n+1}}{u_{n}}=\left\lVert f\right\rVert_{\infty}.
    \end{equation}
\end{remark}

\begin{proof}
    \phantom{}
    \begin{enumerate}
        \item Pour tout $t\in[-\pi,\pi]$, comme $\rho\neq1$, on a $\e^{\i t}\neq\rho\e^{\i\theta}$, donc $t\mapsto\left\lvert\e^{\i t}-\rho\e^{\i \theta}\right\rvert>0$ et $t\mapsto\ln\left\lvert\e^{\i t}-\rho\e^{\i\theta}\right\rvert$ est continue, $2\pi$-périodique sur $[-\pi,\pi]$ donc $F(\rho,\theta)$ existe.
        \item On a 
        \begin{equation}
            F(\rho,\theta)=\int_{-\pi}^{\pi}\ln\left\lvert\e^{\i(t-\theta)}-\rho\right\rvert\d t=\int_{-\pi-\theta}^{\pi-\theta}\ln\left\lvert\e^{\i u}-\rho\right\rvert\d u,
        \end{equation}
        et comme l'intégrande est une fonction $2\pi$-périodique,
        \begin{equation}
            F(\rho,\theta)=\int_{0}^{2\pi}\ln\left\lvert\e^{\i u}-\rho\right\rvert\d u,
        \end{equation}
        est indépendant de $\theta$.

        \item Soit $n\geqslant1$, on a 
        \begin{equation}
            S_n=\frac{2\pi}{n}\sum_{k=0}^{n-1}\ln\left\lvert\e^{\frac{2\i k\pi}{n}}-\rho\right\rvert=\frac{2\pi}{n}\ln\left(\left\lvert\prod_{k=0}^{n-1}\left(\rho-\e^{\frac{2\i k\pi}{n}}\right)\right\rvert\right)=\frac{2\pi}{n}\ln\left(\left\lvert\rho^{n}-1\right\rvert\right).
        \end{equation}

        Si $\rho>1$, on a 
        \begin{align}
            S_n
            &=\frac{2\pi}{n}\ln\left(\rho^{n}-1\right),\\
            &=\frac{2\pi}{n}\left[\ln(\rho^{n})+\ln\left(1-\left(\frac{1}{\rho}\right)^{n}\right)\right]\xrightarrow[n\to+\infty]{}2\pi\ln(\rho).
        \end{align}

        Donc $F(\rho,\theta)=2\pi\ln(\rho)$.

        Si $\rho<1$, $S_n=\frac{2\pi}{n}\ln(1-\rho^{n})\xrightarrow[n\to+\infty]{}0$ donc $F(\rho,\theta)=0$.
    \end{enumerate}
\end{proof}

\begin{remark}
    On a 
    \begin{align}
        F(\rho,0)
        &=\int_{0}^{2\pi}\ln\left\lvert\cos(u)-\rho+\i\sin(u)\right\rvert\d u,\\
        &=\frac{1}{2}\int_{0}^{2\pi}\ln\left(\rho^{2}-2\rho\cos(u)+1\right)\d u,\\
        &=2\pi\ln(\rho)+F\left(\frac{1}{\rho},0\right).
    \end{align}
\end{remark}

\begin{remark}
    On peut se demander si l'on a convergence de $F(1,0)=\frac{1}{2}\int_{-\pi}^{\pi}\ln\left(2(1-\cos(u))\right)\d u$. On vérifie que 
    \begin{equation}
        \left\lvert\ln(2(1-\cos(u)))\right\vert\underset{u\to0}{\sim}2\left\lvert\ln u\right\rvert=\underset{u\to0}{o}\left(\frac{1}{\sqrt{u}}\right).
    \end{equation}
    Donc $F(1,0)$ converge. Pour le calcul, on a 
    \begin{align}
        2F(1,0)
        &=2\pi\ln(2)+2\int_{0}^{\pi}\ln(1-\cos(u))\d u,\\
        &=2\pi\ln(2)+4\int_{0}^{\pi}\ln\left(\sin\left(\frac{u}{2}\right)\right)\d u,\\
        &=2\pi\ln(2)+8\int_{0}^{\frac{\pi}{2}}\ln\left(\sin\left(v\right)\right)\d v.
    \end{align}

    D'après un exercice précédent, l'intégrale vaut $-\frac{\pi}{2}\ln(2)$ et finalement $F(1,0)=0$.
\end{remark}

\begin{proof}
    Toutes les intégrales existent car les fonctions sont à support compact.
    \begin{enumerate}
        \item Montrons la contraposée. Soit $\delta\in\R$ tel que $f(\delta)\neq0$. On suppose que $f(\delta)>0$. Par continuité, il existe $\eta\>0$ tel que $f\geqslant0$ sur $[\delta-\eta,\delta+\eta]$. $f\times\varphi$ est continue sur $[\delta-\eta,\delta+\eta]$, positive et $(f\varphi)(\delta)>0$ donc $\int_{\R}f\varphi>0$ (en choisissant $\varphi\geqslant0$ définie sur le support $[\delta-\eta,\delta+\eta]$).
        \item Montrons un petit lemme: si $\psi\in C_{0}$, il existe $\varphi\in C_{1}$ tel que $\psi=\varphi'$ si et seulement si $\int_{\R}\psi=0$. Pour le sens direct, on a $\int_{\R}\psi=\int_{\R}\varphi'=\lim\limits_{t\to+\infty}\varphi(t)-\lim\limits_{t\to-\infty}\varphi(t)=0$. Pour le sens indirect, on définit $\varphi(x)=\int_{-\infty}^{x}\psi$ (possible car $\psi\in C_{0}$). $\varphi$ est $\mathcal{C}^{1}$ et $\varphi'=\psi$. $\varphi$ est $\mathcal{C}^{1}$ et $\varphi'=\psi$. Soit $A\geqslant0$ tel que pour tout $\left\lvert t\right\rvert\geqslant A$, on a $\psi(t)=0$. Alors pour tout $t\leqslant-A$, on a $\varphi(t)=0$ et pour tout $t\geqslant A$, $\varphi(t)=\int_{-\infty}^{t}\psi=\int_{-A}^{A}\psi=0$. Donc $varphi$ est à support compact.
        
        Pour montrer le résultat, montrons la contraposée. Supposons $f$ non constante, il existe $(x_{1},x_{2})\in\R^{2}$ distincts tel que $f(x_{1})\neq f(x_{2})$. Quitte à remplacer $f$ par $f-\frac{f(x_1)+f(x_2)}{2}$, on peut supposer que $f(x_{2})=-f(x_{1})$. Il existe $\eta>0$ tel que pour tout $t\in[x_{1}-\eta,x_{1}+\eta]$, $f(t)\geqslant0$ et pour tout $t\in[x_{2}-\eta,x_{2}+\eta]$, $f(t)\leqslant0$.

        On a $\int_{\R}\psi=0$. On pose $\varphi(t)=\int_{-\infty}^{t}\psi(x)\d x$. Alors $\varphi\in C_{1}$ et $\int_{\R}f\varphi'>0$.

        \item Soit $G$ une primitive de $g$. On a alors, pour tout $\varphi\in C_{1}$,
        \begin{equation}
            \int_{\R}g\varphi=\left[G\varphi\right]_{-\infty}^{+\infty}-\int_{\R}G\varphi'=\int_{\R}f\varphi'.
        \end{equation}
        D'après ce qui précède, on a, pour tout $\varphi\in C_{1}$, $\int_{\R}(f+G)\varphi'=0$ et donc $f=-G$ à une constante près.
    \end{enumerate}
\end{proof}

\begin{proof}
    On note \function{f}{\R_{+}^{*}}{\R}{t}{\frac{\e^{-t}-\e^{-2t}}{t}}
    $f$ est continue, tend vers $1$ en 0 et $f(t)=\underset{t\to+\infty}{o}\left(\frac{1}{t^{2}}\right)$ donc $I$ existe.

    Soit $\varepsilon>0$. On a 
    \begin{align}
        \int_{\varepsilon}^{+\infty}\frac{\e^{-t}-\e^{-2t}}{t}\d t
        &=\int_{\varepsilon}^{+\infty}\frac{\e^{-t}}{t}\d t-\int_{\varepsilon}^{+\infty}\frac{\e^{-2t}}{t}\d t,\\
        &=\int_{\varepsilon}^{+\infty}\frac{\e^{-t}}{t}\d t-\int_{2\varepsilon}^{+\infty}\frac{\e^{-t}}{t}\d t,\\
        &=\int_{\varepsilon}^{2\varepsilon}\frac{\e^{-t}}{t}\d t.
    \end{align}

    La fonction $\frac{\e^{-t}-1}{t}$ tend vers -1 quand $t\to0$, donc elle est bornée au voisinage de 0 et est continue. Donc 
    \begin{equation}
        \int_{\varepsilon}^{2\varepsilon}\frac{\e^{-t}-1}{t}=\int_{\varepsilon}^{2\varepsilon}\frac{\e^{-t}}{t}\d t-\ln(2)\xrightarrow[\varepsilon\to0]{}0,
    \end{equation}
    donc 
    \begin{equation}
        \boxed{
            I=\ln(2).
        }
    \end{equation}

    On note maintenant $f$ la fonction intégrande de $J$. $f$ est continue sur $\R_{+}^{*}$, on a 
    \begin{equation}
        f(t)\xrightarrow[t\to0]{}0,    
    \end{equation}
    et 
    \begin{equation}
        \int_{1}^{X}\frac{\cos(t)}{t}\d t=\underbrace{\left[\frac{\sin(t)}{t}\right]_{1}^{X}}_{\xrightarrow[X\to+\infty]{}-\sin(1)}+\int_{1}^{X}\frac{\sin(t)}{t^{2}}\d t,
    \end{equation}
    et l'intégrale de droite converge absolument. Donc $\int_{1}^{+\infty}\frac{\cos(t)}{t}\d t$ converge, et c'est donc aussi le cas pour $\int_{1}^{+\infty}\frac{\cos(2t)}{t}\d t$. Donc $J$ existe.

    Soit $\varepsilon>0$ et $X\geqslant \varepsilon$, on a 
    \begin{align}
        \int_{\varepsilon}^{X}\frac{\cos(t)-\cos(2t)}{t}\d t
        &=\int_{\varepsilon}^{X}\frac{\cos(t)}{t}\d t-\int_{\varepsilon}^{X}\frac{\cos(2t)}{t}\d t,\\
        &=\int_{\varepsilon}^{2\varepsilon}\frac{\cos(t)}{t}\d t-\int_{X}^{2X}\frac{\cos(t)}{t}\d t.
    \end{align}
    La deuxième intégrale tend vers 0 quand $X\to+\infty$ (car l'intégrale est semi-convergente), et de même, on a $\int_{\varepsilon}^{2\varepsilon}\frac{\cos(t)-1}{t}\d t\xrightarrow[\varepsilon\to0]{}0$ donc 
    \begin{equation}
        \boxed{
            J=\ln(2).
        }
    \end{equation}
\end{proof}

\begin{remark}
    Généralement, pour $a<b$ et $f\colon\R_{+}\to\R$ continue, dérivable en 0 et telle que $\int_{1}^{+\infty}\frac{f(t)}{t}\d t$ converge. Alors on a $f(u)=f(0)+uf'(0)+\underset{u\to0}{o}(u)$ et 
    \begin{equation}
        \int_{0}^{+\infty}\frac{f(at)-f(bt)}{t}\d t,
    \end{equation}
    existe. En notant $g$ la fonction intégrande, $g$ tend vers $(b-a)f'(0)$ en 0. Et en séparant pour $\varepsilon>0$, on a 
    \begin{equation}
        \int_{\varepsilon}^{+\infty}\frac{f(at)-f(bt)}{t}\d t=\int_{a\varepsilon}^{b\varepsilon}\frac{f(u)}{u}\d u\xrightarrow[\varepsilon\to0]{}f(0)\ln\left(\frac{b}{a}\right).
    \end{equation}
\end{remark}

\begin{proof}
    $f$ est continue par morceaux, positive. On a $0\leqslant f\leqslant2$ donc $f$ est intégrable sur $]0,1]$. On écrit 
    \begin{equation}
        I=\sum_{k=1}^{+\infty}\int_{\frac{1}{k+1}}^{\frac{1}{k}}\frac{\d t}{t}-\frac{k}{k(k+1)},
    \end{equation}
    et en prenant les sommes partielles et en passant à la limite, on trouve 
    \begin{equation}
        \boxed{
            I=1-\gamma.
        }
    \end{equation}
\end{proof}

\begin{proof}
    On note \function{g}{\R_{+}^{*}}{\R}{t}{\frac{1}{\e^{t}-1}}
    $g$ est continue positive sur $\R_{+}^{*}$ et $g(t)=\underset{t\to+\infty}{O}\left(\frac{1}{t^{2}}\right)$/ Donc $f$ est bien définie sur $\R_{+}^{*}$.

    Pour tout $x\in\R_{+}^{*}$, on a 
    \begin{align}
        f(x)
        &=\int_{x}^{+\infty}\frac{\e^{-t}}{1-\e^{-t}}\d t,\\
        &=\left[\ln\left(1-\e^{-t}\right)\right]_{x}^{+\infty},\\
        &=-\ln\left(1-\e^{-x}\right).
    \end{align}
    $f$ est continue et positive sur $\R_{+}^{*}$. On a $f(x)\underset{x\to0}{=}-\ln(x+o(x))\underset{x\to0}{\sim}-\ln(x)=\underset{x\to0}{O}\left(\frac{1}{\sqrt{x}}\right)$ et $f(x)\underset{x\to+\infty}{\sim}\e^{-x}=\underset{x\to+\infty}{O}\left(\frac{1}{x^{2}}\right)$. D'où l'existence de $I$. On pose $u=\e^{-x}$ soit $x=-\ln(u)$ et $\d x=-\frac{\d u}{u}$. C'est un $\mathcal{C}^{1}$-difféomorphisme de $\R_{+}^{*}$ dans $]0,1[$. On a alors 
    \begin{equation}
        I=-\int_{0}^{1}\frac{-\ln(1-u)}{u}\d u=\int_{0}^{1}\frac{-\ln(1-u)}{u}\d u.
    \end{equation}
    On pose $v=1-u$ pour avoir 
    \begin{equation}
        I=\int_{0}^{1}\frac{-\ln(v)}{1-v}\d v.
    \end{equation}
    Pour $v\in]0,1[$, on développe
    \begin{equation}
        \frac{-\ln(v)}{1-v}=\sum_{k=0}^{+\infty}-v^{k}\ln(v)=\sum_{k=0}^{+\infty}f_k(v).
    \end{equation}
    $f_k$ est positive intégrable sur $]0,1[$.

    On forme $u_k=\int_{0}^{1}\left\lvert f_k(t)\right\rvert\d t$ pour $k\in\N$. Alors $u_k=\int_{0}^{1}-t^{k}\ln(t)\d t=\frac{1}{(k+1)^{2}}$. Donc $\sum u_{k}$ converge et on peut intervertir. Finalement, on a 
    \begin{equation}
        \boxed{
            I=\frac{\pi^{2}}{6}.
        }
    \end{equation}
\end{proof}

\begin{remark}
    On peut aussi appliquer le théorème de convergence dominée à $\left(\sum_{k=0}^{n}f_k\right)$ car les $f_k$ sont positifs.
\end{remark}

\begin{remark}
    On a 
    \begin{equation}
        f(x)=\underbrace{\int_{x}^{1}\frac{\d t}{\e^{t}-1}}_{\underset{x\to0}{\sim}\int_{x}^{1}\frac{1}{t}\d t}+\int_{1}^{+\infty}\frac{\d t}{\e^{t}-1}\underset{x\to0}{\sim}-\ln(x)=\underset{x\to0}{O}\left(\frac{1}{\sqrt{x}}\right),
    \end{equation}
    car la fonction est positive, et 
    \begin{equation}
        f(x)\underset{x\to+\infty}{\sim}\int_{x}^{+\infty}\e^{-t}\d t=\e^{-x}=\underset{x\to+\infty}{O}\left(\frac{1}{x^{2}}\right).
    \end{equation}
\end{remark}

\begin{proof}
    \phantom{}
    \begin{enumerate}
        \item On pose $f_n(t)=\left(1+\frac{t}{n}\right)^{n}\e^{-t}=\underset{t\to+\infty}{O}\left(\frac{1}{t^{2}}\right)$, continue positive sur $[-n,+\infty[$. $I_n$ est donc bien définie. On pose $u=t+n$, et alors 
        \begin{align}
            I_n
            &=\int_{0}^{+\infty}\frac{u^{n}}{n^{n}}\e^{-u}\e^{n}\d u,\\
            &=\frac{\e^{n}}{n^{n}}\int_{0}^{+\infty}u^{n}\e^{-u}\d u,\\
            &=\frac{\e^{n}}{n^{n}}\Gamma(n+1),\\
            &=\frac{\e^{n}}{n^{n}}n!.
        \end{align}

        \item On pose $t=\sqrt{n}u$ et on a 
        \begin{equation}
            J_n=\sqrt{n}\int_{-\sqrt{n}}^{\sqrt{n}}\e^{n\ln\left(1+\frac{u}{\sqrt{n}}\right)-\sqrt{n}u}\d u.
        \end{equation}
        On définit, pour tout $n\in\N^{*}$, $f_n\colon\R\to\R$ par $f(u)=\e^{n\ln\left(1+\frac{u}{\sqrt{n}}\right)-\sqrt{n}u}$ sur $[-\sqrt{n},\sqrt{n}]$ et $0$ ailleurs. $f_n$ est continue par morceaux (intégrable) sur $\R$. Soit $u\in\R$, il existe $N_0\in\N$ tel que pour tout $n\geqslant N_{0}$, $\sqrt{n}\geqslant\left\lvert u\right\rvert$. Alors pour tout $n\geqslant N_0$, on a 
        \begin{align}
            f_n(u)
            &\underset{n\to+\infty}{=}\e^{n\left(\frac{u}{\sqrt{n}}-\frac{u^{2}}{2n}+O\left(\frac{1}{n^{\frac{3}{2}}}\right)\right)-\sqrt{n}u},\\
            &\underset{n\to+\infty}{=}\e^{-\frac{u^{2}}{2}+O\left(\frac{1}{\sqrt{n}}\right)}\xrightarrow[n\to+\infty]{}\e^{-\frac{u^{2}}{2}}.
        \end{align}

        Ainsi, on a convergence simple de $f_n\xrightarrow[n\to+\infty]{}f$ sur $\R$ avec $f(u)=\e^{-\frac{u^{2}}{2}}$.

        C'est un cas particulier où trouver la dominante (pour appliquer le théorème de convergence dominée) est difficile. On propose $\varphi(u)=\e^{-\frac{u^{2}}{4}}$ (plus grand que $f$ et toujours intégrable). Soit $n\geqslant1$ et $u\in]-\sqrt{n},\sqrt{n}]$, soit 
        \begin{equation}
            g_n(u)=n\ln\left(1+\frac{u}{\sqrt{n}}\right)-\sqrt{n}u-\left(-\frac{u^{2}}{4}\right),
        \end{equation}
        dérivable sur $]-\sqrt{n},\sqrt{n}]$. On a 
        \begin{equation}
            g'_n(u)=\frac{-\frac{u}{2}\left(1-\frac{u}{\sqrt{n}}\right)}{1+\frac{u}{\sqrt{n}}},
        \end{equation}
        donc $g_n'\leqslant0$ si $u\geqslant0$ et $g_n>0$ si $u<0$. Comme $g_n(0)=0$, on a $g_n(u)\leqslant0$ pour tout $u\in]-\sqrt{n},\sqrt{n}]$ et donc pour tout $u\in\R$, $0\leqslant f_n(u)\leqslant\e^{-\frac{u^{2}}{4}}$.

        On peut donc appliquer le théorème de convergence dominée et on a
        \begin{equation}
            \boxed{
                \frac{J_n}{\sqrt{n}}=\int_\R\e^{-\frac{u^{2}}{2}}\d u=\sqrt{2\pi}.
            }
        \end{equation}

        \item On pose pour $n\in\N^{*}$,
        \begin{align}
            K_n
            &=\int_{-n}^{+\infty}\left(1+\frac{t}{n}\right)^{n}\e^{-t}\d t,\\
            &=\frac{\e^{n}}{n^{n}}\int_{2n}^{+\infty}u^{n}\e^{-u}\d u.
        \end{align}
        On écrit $u^{n}\e^{-u}=u^{n}\e^{-\frac{u}{2}}\e^{-\frac{u}{2}}=h(u)\e^{-\frac{u}{2}}$. Alors $h$ est dérivable sur $]2n,+\infty[$ et $h'(u)=u^{n-1}\left(n-\frac{u}{2}\right)\e^{-\frac{u}{2}}$ donc 
        \begin{equation}
            0\leqslant h(u)\leqslant (2n)^{n}\e^{-n}.
        \end{equation}
        Donc 
        \begin{equation}
            K_n\leqslant 2^{n}\int_{2n}^{+\infty}\e^{-\frac{u}{2}}\d u =2^{n}\times 2\e^{-n}.
        \end{equation}
        Or $\e>2$ donc $\lim\limits_{n\to+\infty}K_n=0$. Donc $K_n=\underset{n\to+\infty}{o}(I_n)$ et $I_n=J_n+K_n$ donc $I_n\underset{n\to+\infty}{\sim}J_n$ et 
        \begin{equation}
            \boxed{
                n!\underset{n\to+\infty}{\sim}\frac{n^{n}}{\e^{n}}\sqrt{2\pi n}.
            }
        \end{equation}
    \end{enumerate}
\end{proof}

\begin{proof}
    \phantom{}
    \begin{enumerate}
        \item Soit $x\in\R$, on a $f_x(t)=\underset{t\to+\infty}{O}\left(\frac{1}{t^{2}}\right)$ et $f_x(t)\xrightarrow[t\to0]{}\frac{x^{2}}{2}$ donc $I(x)$ est bien définie.
        \item Pour $t\in\R_{+}^{*}$, $x\mapsto f_x(t)$ est de classe $\mathcal{C}^{2}$. D'après l'inégalité de Taylor-Lagrange, pour tout $a\neq0$, on a 
        \begin{equation}
            \left\lvert\cos(a)-1\right\rvert=\left\lvert\cos(a)-\cos(0)-a\cos'(0)\right\rvert\leqslant\frac{a^{2}}{2}\sup\limits_{[0,a]}\left\lvert\cos''\right\rvert\leqslant\frac{a^{2}}{2}.
        \end{equation}
        Donc $\left\lvert 1-\cos(tx)\right\rvert\leqslant\frac{t^{2}x^{2}}{2}$. Ainsi, pour tout $t>0$, on a $\left\lvert f_x(t)\right\rvert\leqslant\frac{x^{2}}{2}\e^{-t}$. Fixons $a\geqslant0$, on a pour tout $x\in[-a,a]$, pour tout $t>0$, $\left\lvert f_x(t)\right\rvert\leqslant\frac{a^{2}}{2}\e^{-t}$, fonction indépendante de $x$ et intégrable sur $\R_{+}$.

        D'après le théorème de continuité, $I$ est continue sur $[-a,a]$, pour tout $a\geqslant0$, donc sur $\R$. On a $\frac{\partial f_x}{\partial x}(t)=\frac{\sin(t)x}{t}\e^{-t}$, $\frac{\partial^{2}f_x}{\partial x^{2}}(t)=\cos(tx)\e^{-t}$. Pour tout $u\in\R$, $\left\lvert\sin(u)\right\rvert\leqslant\left\lvert u\right\rvert$ donc pour tout $t>0$, pour tout $x\in[-a,a]$,
        \begin{equation}
            \left\lvert\frac{\partial f_x}{\partial x}(t)\right\rvert\leqslant\left\lvert x\right\rvert\e^{-t}\leqslant a\e^{-t},
        \end{equation}
        \begin{equation}
            \left\lvert\frac{\partial^{2} f_x}{\partial x^{2}}(t)\right\rvert\leqslant\e^{-t}.
        \end{equation}
        Donc d'après le théorème de dérivation, $I$ est de classe $\mathcal{C}^{2}$ et on a 
        \begin{align}
            I''(x)
            &=\int_{0}^{+\infty}\cos(tx)\e^{-t}\d t,\\
            &=\Re\left(\int_{0}^{+\infty}\e^{-(1-\i x)t}\d t\right),\\
            &=\Re\left(\frac{1}{1-\i x}\right),\\
            &=\frac{1}{1+x^{2}}.
        \end{align}
        Donc $I'(x)=\lambda+\arctan(x)$ avec $\lambda\in\R$. Comme $I'(0)=0$, $I'(x)=\arctan(x)$. On a $I(0)=0$ donc 
        \begin{align}
            I(x)
            &=\int_{0}^{x}\arctan(u)\d u,\\
            &=\left[u\arctan(u)\right]_{0}^{x}-\int_{0}^{x}\frac{u}{1+u^{2}}\d u,\\
            &=x\arctan(x)-\frac{1}{2}\ln\left(1+x^{2}\right).
        \end{align}
    \end{enumerate}
\end{proof}

\begin{remark}
    En posant $v=xt$ pour $x>0$, on a 
    \begin{equation}
        I(x)=x\int_{0}^{+\infty}\frac{1-\cos(v)}{v^{2}}\e^{-\frac{v}{x}}\d v,
    \end{equation}
    et 
    \begin{equation}
        \lim\limits_{x\to+\infty}\frac{I(x)}{x}=\int_{0}^{+\infty}\frac{1-\cos(v)}{v^{2}}\d v=\frac{\pi}{2},
    \end{equation}
    et par intégration par parties, cette intégrale vaut $\int_{0}^{+\infty}\frac{\sin(v)}{v}\d v$. C'est une autre preuve de l'intégrale de Dirichlet.
\end{remark}

\begin{proof}
    Soit $x\geqslant0$, $t\mapsto\frac{\sin(t)}{t}$ définie continue sur $\R$. On a 
    \begin{equation}
        \int_{x}^{X}\frac{\sin(t)}{t}\d t=\underbrace{\left[\frac{1-\cos(t)}{t}\right]_{x}^{X}}_{\xrightarrow[X\to+\infty]{}\frac{1-\cos(x)}{x}}+\int_{x}^{X}\frac{1-\cos(t)}{t^{2}}\d t.
    \end{equation}
    L'intégrale est absolument convergente, donc $f(x)$ est définie et on a 
    \begin{equation}
        f(x)=\frac{\cos(x)-1}{x}+\int_{x}^{+\infty}\frac{1-\cos(t)}{t^{2}}\d t.
    \end{equation}
    $f$ est de classe $\mathcal{C}^{1}$ et $f'(x)=-\frac{\sin(x)}{x}$. Soit $X>0$ fixé, on a 
    \begin{align}
        \int_{0}^{X}f(x)\d x
        &=\left[xf(x)\right]_{0}^{X}-\int_{0}^{X}f'(x)x\d x,\\
        &=Xf(X)+\int_{0}^{X}\sin(x)\d x,\\
        &=Xf(X)+(1-\cos(X)),\\
        &=X\int_{X}^{+\infty}\frac{1-\cos(t)}{t^{2}}\d t,\\
        &=1-X\int_{X}^{+\infty}\frac{\cos(t)}{t^{2}}\d t.
    \end{align}
    Par intégrations par parties, on a 
    \begin{align}
        \int_{X}^{+\infty}\frac{\cos(t)}{t^{2}}\d t
        &=\left[\frac{1}{t^{2}}\sin(t)\right]_{X}^{+\infty}+\int_{X}^{+\infty}\frac{2}{t^{2}}\sin(t)\d t,\\
        &=-\frac{\sin(X)}{X^{2}}+\int_{X}^{+\infty}\frac{2\sin(t)}{t^{3}}\d t,\\
        &=\underset{X\to+\infty}{O}\left(\frac{1}{X^{2}}\right),
    \end{align}
    car la deuxième intégrale est majorée par $\int_{X}^{+\infty}\frac{2}{t^{3}}\d t=\frac{1}{X^{2}}$. Finalement, on a 
    \begin{equation}
        \int_{0}^{X}f(x)\d x=1+\underset{X\to+\infty}{O}\left(\frac{1}{X}\right),
    \end{equation}
    donc 
    \begin{equation}
        \boxed{
            \int_{0}^{+\infty}f(x)\d x=1.
        }
    \end{equation}
\end{proof}

\begin{proof}
    Définissons $f_h$ pour $h>0$ par $f_h(t)=f(nh)$ si $t\in[nh,(n+1)h[$ ($n\left\lfloor\frac{t}{h}\right\rfloor$). Pour $h$ fixé, $f(nh)=\underset{n\to+\infty}{O}\left(\frac{1}{n^{2}}\right)$ donc $\phi(h)$ est bien définie. $f_h$ est continue par morceaux et $f_h(t)=\underset{t\to+\infty}{O}\left(\frac{1}{t^{2}}\right)$ donc $f_h$ est intégrable sur $\R_{+}$ et 
    \begin{equation}
        \phi(h)=\int_{0}^{+\infty}f_h(t)\d t.
    \end{equation}

    Soit $t$ fixé, on a $\lim\limits_{h\to0^{+}}h \left\lfloor\frac{t}{h}\right\rfloor=t$. Donc par continuité de $f$, $f_h(t)\xrightarrow[h\to0^{+}]{}f(t)$. 

    On sait qu'il existe $M\in\R_{+}$ tel que pour tout $x>0$, $\left\lvert f(x)\right\rvert\leqslant\frac{M}{x^{2}}$. Donc 
    \begin{equation}
        \left\lvert f_h(t)\right\rvert\leqslant\frac{M}{\left(h\left\lfloor\frac{t}{h}\right\rfloor\right)^{2}}\leqslant\frac{M}{\left(t-h\right)^{2}}.
    \end{equation}

    On s'impose $h<1$. Dans ce cas, pour tout $t>2$, $\left\lvert f_h(t)\right\rvert\leqslant\frac{M}{\left(t-1\right)^{2}}$ est intégrable sur $[2,+\infty[$ et indépendant de $h$. Pour tout $t\in[0,2]$, $\left\lvert f_h(t)\right\rvert\leqslant\left\lVert f\right\rVert_{\infty}$ intégrable sur $[0,2]$ et indépendant de $h$.

    Soit \function{\varphi}{\R_+}{\R}{t}{
        \left\lbrace
        \begin{array}[]{ll}
            \left\lVert f\right\rVert_{\infty} &\text{si }t\in[0,2],\\
            \frac{M}{(t-1)^{2}} &\text{si }t>2.
        \end{array}
        \right.
    }
    $\varphi$ est continue par morceaux intégrable sur $\R_{+}$ et indépendante de $h$. Donc, par convergence dominée,
    \begin{equation}
        \boxed{
            \lim\limits_{h\to0}\phi(h)=\int_{0}^{+\infty}f(t)\d t.
        }
    \end{equation}
\end{proof}

\begin{proof}
    $f$ est impaire donc on se limite à $x>0$. On pose $g(x,t)$ l'intégrande. $t\mapsto g(x,t)$ est continue sur $\R_{+}^{*}$, $g(x,t)\xrightarrow[t\to0]{}x$ et $g(x,t)\underset{t\to+\infty}{\sim}\frac{\e^{(x-1)t}}{2t}$, donc $t\mapsto g(x,t)$ est intégrable sur $\R_{+}^{*}$ si et seulement si $x<1$. Donc le domaine de définition est $]-1,1[$. Enfin, $x\mapsto g(x,t)$ est $\mathcal{C}^{\infty}$.

    On a $\frac{\partial g}{\partial x}(x,t)=\cosh(xt)\e^{-t}=\frac{\e^{(x-1)t}+\e^{-(x+1)t}}{2}$. Fixons $a\in[0,1[$, soit $x\in[0,a]$. Si $t\geqslant1$, on a $0\leqslant \sinh(xt)\leqslant\frac{\e^{xt}}{2}$ et
    \begin{equation}
        0\leqslant g(x,t)\leqslant\frac{\e^{(x-1)t}}{2t}\leqslant\frac{\e^{(a-1)t}}{2t}.
    \end{equation}

    Par ailleurs, $\lim\limits_{u\to0}\frac{\sinh(u)}{u}=1$, donc il existe $M\geqslant0$ tel que si $\left\lvert u\right\vert\leqslant a$, $\left\lvert\frac{\sinh(u)}{u}\right\rvert\leqslant M$. Si $t\in]0,1]$, $xt\in]0,a]$, $0\leqslant g(x,t)\leqslant M_a$. En définissant 
    \function{\phi_0}{\R_{+}^{*}}{\R_{+}}{t}{
        \left\lbrace
            \begin{array}[]{ll}
                M_a & \text{si }t\in]0,1],\\
                \frac{\e^{(a-1)t}}{2t} &\text{si } t>1.    
            \end{array}
    \right.
    }
    $\phi_0$ est intégrable sur $\R_{+}^{*}$ et $\left\lvert g(x,t)\right\rvert\leqslant\phi_0(t)$. Or $\sinh$ est croissante et $\left\lvert g(x,t)\right\rvert\leqslant\frac{\sinh(at)}{t}\e^{-t}$, intégrable sur $\R_{+}^{*}$. Par ailleurs, $\left\lvert\frac{\partial g}{\partial x}(x,t)\right\rvert\leqslant\cosh(at)\e^{-t}$ est intégrable sur $\R_{+}^{*}$ car $a<1$. D'après le théorème de continuité dérivabilité, $f$ est de classe $\mathcal{C}^{1}$ sur $[0,a]$ pour tout $a<1$, donc sur $[0,1[$.

    Alors 
    \begin{align}
        f'(x)
        &=\int_{0}^{+\infty}\cosh(xt)\e^{-t}\d t,\\
        &=\frac{1}{2}\int_{0}^{+\infty}\left(\e^{(x-1)t}+\e^{-(x+1)t}\right)\d t,\\
        &=\frac{1}{2}\left(\frac{1}{1-x}+\frac{1}{x+1}\right).
    \end{align}
    Comme $f(0)=0$, pour $x\in[0,1[$, $f(x)=\frac{1}{2}\ln\left(\frac{1+x}{1-x}\right)$ et si $x\in]-1,0]$, $f(x)=-f(-x)=\frac{1}{2}\ln\left(\frac{1+x}{1-x}\right)$.
\end{proof}

\begin{proof}
    On a $\left\lvert f(x,t)\right\rvert=\frac{\e^{-t}}{\sqrt{t}}=\underset{t\to+\infty}{O}\left(\frac{1}{t^{2}}\right)$ et est équivalent ) $\frac{1}{\sqrt{t}}$ en 0. Donc $F$ existe et est continue sur $\R$. De plus,
    \begin{equation}
        \left\lvert\frac{\partial f}{\partial x}(x,t)\right\rvert=\sqrt{t}\e^{-t},
    \end{equation}
    est intégrable sur $\R_{+}$. Donc $F$ est $\mathcal{C}^{1}$ et 
    \begin{align}
        F'(x)
        &=\int_{0}^{+\infty}\i\sqrt{t}\e^{t(\i x-1)}\d t,\\
        &=\left[\frac{\i\sqrt{t}\e^{t(\i x-1)}}{\i x-1}\right]_{0}^{+\infty}-\int_{0}^{+\infty}\frac{\i \e^{t(\i x-1)}}{2\sqrt{t}(\i x-1)}\d t,\\
        &=-\frac{\i}{2(\i x-1)}F(x).
    \end{align}
    On a
    \begin{equation}
        \frac{\i}{2(\i x-1)}=\frac{x}{2(x^{2}+1)}-\frac{\i}{2(x^{2}+1)},
    \end{equation}
    donc 
    \begin{equation}
        F(x)=A\exp\left(-\frac{1}{4}\ln\left(x^{2}+1\right)+\frac{\i}{2}\arctan(x)\right).
    \end{equation}
    Comme 
    \begin{equation}
        F(0)=A=\int_{0}^{+\infty}\frac{\e^{-t}}{\sqrt{t}}\d t=\sqrt{\pi},
    \end{equation}
    on a
    \begin{equation}
        \boxed{
            F(x)=\sqrt{\pi}\exp\left(-\frac{1}{4}\ln\left(x^{2}+1\right)+\frac{\i}{2}\arctan(x)\right).
        }
    \end{equation}
\end{proof}

\begin{proof}
    \phantom{}
    \begin{enumerate}
        \item On a 
        \begin{equation}
            \left\lvert\widehat{f}(\nu)\e^{\i\nu x}\e^{-\lambda\left\lvert\lambda\right\rvert}\right\rvert\leqslant\left\lvert \widehat{f}(\nu)\right\rvert.
        \end{equation}
        Donc $g_x$ est bien définie. On pose 
        \begin{equation}
            h(t,\nu)=f(t)\e^{\i\nu(x-t)}\e^{-\lambda\left\lvert\nu\right\rvert}.
        \end{equation}
        On a 
        \begin{equation}
            g_x(\lambda)=\frac{1}{2\pi}\int_{-\infty}^{+\infty}\left(\int_{-\infty}^{+\infty}h(t,\nu)\d t\right)\d\nu,
        \end{equation}
        et comme 
        \begin{equation}
            \int_{-\infty}^{+\infty}\int_{-\infty}^{+\infty}\left\lvert h(t,\nu)\right\rvert\d t\d\nu=\int_{\R}\e^{-\lambda\left\lvert\nu\right\rvert}\d\nu\times\int_{\R}\left\lvert f(t)\right\rvert\d t=\frac{2}{\lambda}\int_{\R}\left\lvert f(t)\right\rvert\d t<+\infty,
        \end{equation}
        d'après le théorème de Fubini,
        \begin{align}
            g_x(\lambda)
            &=\frac{1}{2\pi}\int_{\R}\left(\int_{\R}\e^{\i\nu(x-t)}\e^{-\lambda\left\lvert\nu\right\rvert}\d \nu\right)f(t)\d t,\\
            &=\frac{1}{2\pi}\int_{\R}\left(\int_{-\infty}^{0}\e^{\i\nu(x-t)}\e^{\lambda\nu}\d \nu+\int_{0}^{+\infty}\e^{\i\nu(x-t)}\e^{-\lambda\nu}\right)f(t)\d t,\\
            &=\frac{1}{2\pi}\int_{\R}\left(\frac{1}{\lambda-\i(t-x)}+\frac{1}{\lambda-\i(x-t)}\right)f(t)\d t,\\
            &=\frac{1}{2\pi}\int_{\R}\frac{2\lambda f(t)}{\lambda^{2}+(t-x)^{2}}\d t.
        \end{align}

        On pose $t'=t-x$ et on a bien 
        \begin{equation}
            \boxed{
                g_x(\lambda)=\frac{1}{2\pi}\int_{\R}\frac{2\lambda f(t+x)}{\lambda^{2}+t^{2}}\d t.
            }
        \end{equation}

        \item On pose $u=\frac{t}{\lambda}$ pour $\lambda>0$. Alors 
        \begin{equation}
            g_x(\lambda)=\frac{1}{2\pi}\int_{\R}\frac{f(\lambda u+x)}{1+u^{2}}\d u,
        \end{equation}
        et pour $u$ fixé, $\lim\limits_{\lambda\to0}\frac{2f(\lambda u+x)}{1+u^{2}}=\frac{2f(x)}{1+u^{2}}$ par continuité de $f$. Comme 
        \begin{equation}
            \left\lvert \frac{2f(\lambda u+x)}{1+u^{2}}\right\rvert\leqslant\frac{2\left\lVert f\right\rVert_{\infty}}{1+u^{2}},
        \end{equation}
        fonction de $u$ intégrable sur $\R$ indépendante de $\lambda$. Par le théorème de continuité, on a
        \begin{equation}
            \boxed{
                \lim\limits_{\lambda\to0}g_x(\lambda)=\frac{1}{2\pi}\int_{\R}\frac{2f(x)}{1+u^{2}}\d u=f(x).
            }
        \end{equation}

        \item Pour tout $\nu\in\R$, $\lambda\mapsto\widehat{f}(\nu)\e^{\i\nu x}\e^{-\lambda\left\lvert \nu\right\rvert}$ est continue sur $\R_{+}$, et on a 
        \begin{equation}
            \left\lvert \widehat{f}(\nu)\e^{\i\nu x}\e^{-\lambda\left\lvert\nu\right\rvert}\right\rvert\leqslant\left\lvert \widehat{f}(\nu)\right\rvert,
        \end{equation}
        fonction indépendante de $\lambda$ intégrable sur $\R$. Par le théorème de continuité, $g_x$ est continue sur $\R_{+}$, donc en 0. Ainsi,
        \begin{equation}
            \boxed{
                g_x(0)=f(x)=\lim\limits_{\lambda\to0}g_x(\lambda)=\frac{1}{2\pi}\int_{\R}\widehat{f}(\nu)\e^{\i\nu x}\d\nu.
            }
        \end{equation}
    \end{enumerate}
\end{proof}

\begin{remark}
    En exemple, soit $a>0$ et \function{f}{\R}{\R}{x}{\e^{-a\left\lvert x\right\rvert}}
    $f$ est continue, intégrable et bornée. On a 
    \begin{align}
        \widehat{f}(\nu)
        &= \int_{\R}f(t)\e^{-\i\nu t}\d t,\\
        &=\frac{2a}{a^{2}+\nu^{2}}.
    \end{align}
    $\widehat{f}$ est intégrable sur $\R$, donc pour tout $x\in\R$, on a 
    \begin{equation}
        2\pi\e^{-a\left\lvert x\right\rvert}=\int_{\R}\frac{2a}{a^{2}+\nu^{2}}\e^{\i\nu x}\d\nu.
    \end{equation}
\end{remark}

\begin{remark}
    Si $\widehat{f}=0$, elle est intégrable et d'après la formule, on a $f=0$.
\end{remark}

\begin{proof}
    \phantom{}
    \begin{enumerate}
        \item $D_n=1+\sum_{k=1}^{n}2\cos(kt)$ est paire, $2\pi$-périodique et $\mathcal{C}^{\infty}$. Sa moyenne est 1, et comme pour $t\in\R\setminus2\pi\Z$, $\e^{\i t}\neq1$, on a 
        \begin{align}
            D_n(t)
            &=\e^{-\i nt}\sum_{k=0}^{2n}\e^{\i kt},\\
            &=\e^{-\i nt}\left(\frac{1-\e^{\i(2n+1)t}}{1-\e^{\i t}}\right),\\
            &=\e^{-\i nt}\frac{\e^{\i \frac{(2n+1)t}{2}}}{\e^{\i \frac{t}{2}}}\left(\frac{\e^{-\i\frac{(2n+1)t}{2}}-\e^{\i\frac{(2n+1)t}{2}}}{\e^{-\i\frac{t}{2}}-\e^{\i\frac{t}{2}}}\right),\\
            &=\e^{-\i nt}\e^{\i nt}\left(\frac{-2\i\sin\left(\frac{(2n+1)t}{2}\right)}{-2\i\sin\left(\frac{t}{2}\right)}\right),\\
            &=\frac{\sin\left(\frac{(2n+1)t}{2}\right)}{\sin\left(\frac{t}{2}\right)}.
        \end{align}

        \item On pose $u=\frac{2t}{2n+1}$.
        \item Elle est de classe $\mathcal{C}^{1}$ sir $]0,\pi]$. Pour tout $u>0$,
        \begin{equation}
            \varphi(u)=\frac{\frac{u}{2}-\sin\left(\frac{u}{2}\right)}{\left(\frac{u}{2}\right)\sin\left(\frac{u}{2}\right)}\underset{u\to0}{\sim}\frac{\frac{1}{6}\left(\frac{u}{2}\right)^{3}}{\left(\frac{u^{2}}{4}\right)}\xrightarrow[u\to0]{}0.
        \end{equation}
        On pose $\varphi(0)=0$. $\varphi$ ainsi prolongée est continue sur $[0,2\pi]$. On a 
        \begin{align}
            \varphi'(u)
            &=-\frac{1}{2}\frac{\cos\left(\frac{u}{2}\right)}{\sin^{2}\left(\frac{u}{2}\right)}+\frac{2}{u^{2}},\\
            &=\frac{4\sin^{2}\left(\frac{u}{2}\right)-u^{2}\cos\left(\frac{u}{2}\right)}{2u^{2}\sin^{2}\left(\frac{u}{2}\right)},\\
            &\underset{u\to0}{\sim}\frac{u^{4}\left(-\frac{1}{12}+\frac{1}{8}\right)}{\left(\frac{u^{4}}{2}\right)}\xrightarrow[u\to0]{}\frac{1}{12}.
        \end{align}

        D'après le théorème de prolongement de la dérivée, $\varphi$ est $\mathcal{C}^{1}$ sur $[0,\pi]$.

        \item On a 
        \begin{align}
            \int_{0}^{\pi}\varphi(u)\sin\left((2n+1)\frac{u}{2}\right)\d u
            &=\left[-\frac{2}{n+1}\cos\left((2n+1)\frac{u}{2}\right)\varphi(u)\right]_{0}^{\pi}\nonumber\\
            &\qquad+\frac{2}{n+1}\int_{0}^{\pi}\varphi'(u)\cos\left((2n+1)\frac{u}{2}\right)\d u,\\
            &\xrightarrow[n\to+\infty]{}0,
        \end{align}
        car $\varphi$ et $\varphi'$ sont bornées. De plus,
        \begin{align}
            \int_{0}^{\pi}\varphi(u)\sin\left((2n+1)\frac{u}{2}\right)\d u
            &=\int_{0}^{\pi}D_n(u)\d u-\int_{0}^{\pi}\frac{2\sin\left((2n+1)\frac{u}{2}\right)}{u}\d u,\\
            &=\frac{1}{2}\int_{-\pi}^{\pi}D_n(u)\d u-2u_n,\\
            &=\pi-2u_n.
        \end{align}
        Donc
        \begin{equation}
            \boxed{
                \lim\limits_{n\to+\infty}u_n=\frac{\pi}{2}=\int_{0}^{+\infty}\frac{\sin(t)}{t}\d t.
            }
        \end{equation}
    \end{enumerate}
\end{proof}

\begin{remark}
    Soit $f\colon I\to\R$ ou $\C$ intégrable sur un intervalle $I$. On a $\lim\limits_{\lambda\to+\infty}\int_{I}f(t)\e^{\i\lambda t}\d t=0$.

    En effet, soit $\varepsilon>0$, il existe $[a,b]\subset I$ tel que $\int_{I\setminus[a,b]}\left\lvert f\right\rvert\leqslant\frac{\varepsilon}{2}$. Alors 
    \begin{equation}
        \left\lvert\int_{I}f(t)\e^{\i\lambda t}\d t\right\rvert\leqslant\int_{I\setminus[a,b]}\left\lvert f\right\rvert+\left\lvert\int_{[a,b]}f(t)\e^{\i\lambda t}\d t\right\rvert,
    \end{equation}
    et le deuxième terme tend vers 0 quand $\lambda\to \pm\infty$, donc inférieur à $\frac{\varepsilon}{2}$ pour $\lambda$ suffisamment grand. En particulier, si $f$ est intégrable sur $\R$, on a $\lim\limits_{\lambda\to+\infty}\widehat{f}(\lambda)=0$.
\end{remark}

\begin{proof}
    \phantom{}
    \begin{enumerate}
        \item $g$ est de classe $\mathcal{C}^{1}$. On a $g'(t)=f(t)\e^{-at}$, $g(0)=0$ et $\lim\limits_{t\to+\infty}g(t)=Lf(a)$. Donc $g$ est bornée sur $\R_{+}$. Soit $X>0$, on a
        \begin{equation}
            \int_{0}^{X}f(t)\e^{-(a+x)t}\d t=\left[\e^{-xt}g(t)\right]_{0}^{X}+x\int_{0}^{X}g(t)\e^{-xt}\d t.
        \end{equation}
        Le terme entre crochet tend vers $0$ quand $X\to+\infty$ car $x>0$ et $\left\lvert g(t)\e^{-xt}\right\rvert\leqslant\left\lVert g\right\rVert_{\infty}\left\lvert\e^{-xt}\right\rvert=\underset{t\to+\infty}{O}\left(\frac{1}{t^{2}}\right)$. Donc $t\mapsto g(t)\e^{-xt}$ est intégrable sur $\R_{+}$. Donc $Lf(a+x)=\int_{0}^{+\infty}f(t)\e^{-\i t(a+x)}\d t$ existe et vaut $x\int_{0}^{+\infty}g(t)\e^{-xt}\d t$.

        \item On pose $u=\e^{-t}$, $\mathcal{C}^{1}$-difféomorphisme de $[0,+\infty[\to]0,1]$. On a alors 
        \begin{equation}
            Lf(a+x)=x\int_{0}^{1}g\left(-\ln(u)\right)u^{x-1}\d u.
        \end{equation}
        Or $\lim\limits_{u\to0^{+}}g\left(-\ln(u)\right)=Lf(a)$. On définit \function{h}{[0,1]}{\R \text{ ou }\C}{u}{
            \left\lbrace
                \begin{array}[]{ll}
                    g\left(-\ln(u)\right) & \text{si }u\neq 0,\\
                    LF(a) & \text{si }u=0.
                \end{array}
            \right.
        }
        $h$ est continue et $Lf(a+x)=x\int_{0}^{1}h(u)u^{x-1}\d u$. Si pour tout $x>0$, $Lf(a+x)=0$, on a pour tout $>0$, $\int_{0}^{1}h(u)u^{x-1}\d u=0$. Par combinaison linéaire, pour tout $P\in\C[X]$, $\int_{0}^{1}\ln(u)P(u)=0$. D'après le théorème de Weierstrass, on prend $(P_n)_{n\in\N}$ une suite de polynômes qui converge uniformément sur $[0,1]$ vers $\overline{h}$. Donc $\int_{0}^{1}\left\lvert h\right\rvert^{2}=0$ donc $h=0$. Ainsi, $g=0$ et $g'(t)=0=f(t)\e^{-at}$ donc $f=0$.
    \end{enumerate}
\end{proof}

\begin{remark}
    Il suffit qu'il existe $a_{0}\in\R$ tel que pour tout $n\in\N^{*}$, $Lf(a_{0}+n)=0$.
\end{remark}

\begin{proof}
    Pour $x>0$, $\sum g_n(x)$ est alternée car $\left\lvert g_n(x)\right\rvert=\e^{-a_n x}$ est décroissante car $a_n\leqslant a_{n+1}$ et tend vers 0 car $\lim\limits_{n\to+\infty}a_n=+\infty$. Donc $g$ est bien définie sur $\R_{+}^{*}$.

    Pour tout $n\in\N$, $g_n$ est continue sur $\R_{+}^{*}$ et pour tout $N\in\N$, d'après le critère spécial des séries alternées,
    \begin{equation}
        \left\lvert\sum_{n=N}^{+\infty}(-1)^{n}\e^{-a_n x}\right\rvert\leqslant\e^{-a_N x}.
    \end{equation}

    Soit $\alpha>0$. Alors pout tout $x\geqslant\alpha$, pour tout $N\in\N$,
    \begin{equation}
        \left\lvert\sum_{n=N}^{+\infty}(-1)^{n}\e^{-a_n x}\right\rvert\leqslant\e^{-a_N x}\leqslant\e^{-a_N \alpha}\xrightarrow[N\to+\infty]{}0,
    \end{equation}
    car $\alpha>0$ et $\lim\limits_{N\to+\infty}a_N=+\infty$.

    Ainsi, $\sum g_n$ converge uniformément sur $[\alpha,+\infty[$, donc $g$ est continue sur $[\alpha,+\infty[$ pour tout $\alpha>0$ donc sur $\R_{+}^{*}$. De plus, toujours d'après le critère spécial des séries alternées, pour tout $x>0$,
    \begin{equation}
        \left\lvert g(x)\right\rvert\leqslant\e^{-a_0 x},
    \end{equation}
    fonction de $x$ intégrable sur $\R_{+}^{*}$ car continue et est $\underset{x\to+\infty}{O}\left(\frac{1}{x^{2}}\right)$. Donc $g$ est intégrable sur $\R_{+}^{*}$. 
    
    On forme $u_n=\int_{0}^{+\infty}\left\lvert g_n(x)\right\rvert\d x=\int_{0}^{+\infty}\e^{-a_n x}\d x=\frac{1}{a_n}$. On n'est pas sûr que $\sum\frac{1}{a_n}$ converge. Mais on a 
    \begin{equation}
        0\leqslant\sum_{n=0}^{N}(-1)^{n}\e^{-a_n x}=S_N(x)\leqslant\e^{-a_0 x},
    \end{equation}
    fonction intégrable sur $\R_{+}^{*}$. D'après le théorème de convergence dominée, on a 
    \begin{equation}
        \lim\limits_{N\to+\infty}\int_{0}^{+\infty}S_N(x)\d x=\sum_{n=0}^{+\infty}\frac{(-1)^{n}}{a_n}=\int_{0}^{+\infty}g(x)\d x.
    \end{equation}
\end{proof}

\begin{proof}
    Soit $x>0$, notons $f_n(x)=(-1)^{n}\e^{-a_n x}$ continue. $\sum_{n\in\N}f_n(x)$ est une série alternée. Pour tout $n\in\N$, on a $\left\lvert f_{n+1}(x)\right\rvert=\e^{-a_{n+1} x}\leqslant\left\lvert f_n(x)\right\rvert$ car $(a_n)_{n\in\N}$ est croissante, $\lim\limits_{n\to+\infty}\left\lvert f_n(x)\right\rvert=0$ car $\lim\limits_{n\to+\infty}a_n=+\infty$ et $x>0$. Donc $S(x)$ est définie sur $\R_{+}^{*}$.

    Pour tout $n\in\N$, $f_n$ est continue. Soit $\alpha>0$, pour tout $x\in[\alpha,+\infty[$, $\left\lvert\sum_{n=N}^{+\infty}f_n(x)\right\rvert\leqslant\e^{-a_{N}x}\leqslant\e^{-a_N \alpha}\xrightarrow[N\to+\infty]{}0$ d'après le critère spécial des séries alternées. Il y a donc convergence uniforme sur $[\alpha,+\infty[$, donc $S$ est continue sur $\R_{+}^{*}$. On a $\left\lvert S(x)\right\rvert\leqslant\e^{-a_0 x}$ intégrable sur $\R_{+}^{*}$ (encore une fois via le critère spécial des séries alternées). Donc $S$ est intégrable sur $\R_{+}^{*}$. Enfin, d'après le critère spécial des séries alternées, $\sum_{n=0}^{+\infty}\frac{(-1)^{n}}{a_n}$ existe.

    Problème : $\sum\int\left\lvert f_n\right\rvert$ ne converge pas forcément, on ne peut pas appliquer le théorème d'interversion.
    On le fait donc à la main : pour tout $N\in\N$, on a 
    \begin{equation}
        \sum_{n=0}^{N}\int_{0}^{+\infty}f_n(x)\d x\xrightarrow[N\to+\infty]{}\sum_{n=0}^{+\infty}\frac{(-1)^{n}}{a_n}.
    \end{equation}
    Soit $R_N(x)=\sum_{n=N}^{+\infty}f_n(x)\d x$. On a $\left\lvert R_N(x)\right\rvert\leqslant\e^{-a_N x}\leqslant \e^{-a_0 x}$ indépendant de $N$ et intégrable sur $\R_{+}^{*}$. D'après le théorème de convergence dominée, on peut intervertir, d'où $\lim\limits_{N\to+\infty}\int_{0}^{+\infty}\sum_{n=N}^{+\infty}f_n(x)\d x=0$.
\end{proof}

\begin{proof}
    On a convergence normale de la série, puis on passe par la partie imaginaire de l'exponentielle. On trouve que 
    \begin{equation*}
        \sum_{n=1}^{+\infty}\frac{\sin(nx)}{3^{n}}=\frac{3}{10-6\cos(x)}.
    \end{equation*}
    L'intégrale vaut
    \begin{equation*}
        2\int_{0}^{\pi}\sum_{k=1}^{+\infty}f_k(x)\d x,
    \end{equation*}
    avec $f_k(x)=\frac{\sin(kx)}{3^{k}}\sin(x)$. On a convergence normale sur $\R$, on peut intervertir. On utilise la formule du produit des sinus: $\sin(k)x\sin(nx)=\frac{1}{2}\left(\cos(k-n)x-\cos(k+n)x\right)$. Ainsi, l'intégrale vaut 0 si $k\neq n$ et $\pi$ sinon. On conclut grâce au calcul précédent.
\end{proof}

\begin{remark}
    Généralement, si $I_n=\int_{0}^{\pi}f(x)\sin(nx)\d x$ avec $f(x)=\sum_{k\in\Z}c_k\e^{\i kx}$ et $\sum_{k\in\Z}\left\lvert c_k\right\rvert<+\infty$, on peut intervertir par convergence normale le signe $\sum$ et $\int$.
\end{remark}

\begin{proof}
    $x^{2}-2x\cos(t)+1=(x-\cos(t))^{2}+\sin^{2}(t)\geqslant0$ et est nul si et seulement si $t=k\pi$ et $x=\cos t=(-1)^{k}$. Ainsi, si $x\not\in\left\lbrace-1,1\right\rbrace$, $I(x)$ est bien définie. Par ailleurs, on a $x^{2}-2x\cos(t)+1=(x-\e^{\i t})(x-\e^{-\i t})>0$. On passe ensuite par la somme de Riemann:
    \begin{equation*}
        I(x)=\lim\limits_{n\to+\infty}\frac{2\pi}{n}\sum_{k=0}^{n-1}\ln\left(\left\lvert x-\e^{\frac{2\i k\pi}{n}}\right\rvert\left\lvert x-\e^{-\frac{2\i k\pi}{n}}\right\rvert\right)=\lim\limits_{n\to+\infty}\frac{4\pi}{n}\ln\left\lvert x^{n}-1\right\rvert.
    \end{equation*}
    Ainsi, si $x\in]-1,1[$, $I(x)=0$. Si $\left\lvert x\right\rvert>1$, $\ln\left\lvert x^{n}-1\right\rvert=\ln\left\lvert x\right\rvert^{n}+\ln\left\lvert 1-\frac{1}{x^{n}}\right\rvert$ donc le résultat est $4\pi\ln\left\lvert x\right\rvert$.

    Si $x=1$, comme $\ln(1-\cos(t))\sim2\ln(t)$ en 0, $I(1)$ est bien définie. On passe ensuite par le calcul de $\int_{0}^{\frac{\pi}{2}}\ln\cos t\d t$.
\end{proof}

\end{document}