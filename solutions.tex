\documentclass[12pt]{article}
\usepackage[french]{babel}
\usepackage[utf8]{inputenc}
\usepackage[T1]{fontenc}

%% Mathématiques
\usepackage{amsmath}
\usepackage{amssymb}
\usepackage{stmaryrd}
\SetSymbolFont{stmry}{bold}{U}{stmry}{m}{n}
\usepackage{mathdots}
\usepackage{bm}

%% Couleur
\usepackage{xcolor}
\definecolor{bluearmy}{RGB}{16, 66, 146}

%% Référence
\usepackage[breaklinks]{hyperref}
\hypersetup{
	colorlinks=true,
	linkcolor=bluearmy,
	filecolor=blue,
	citecolor=black,      
	urlcolor=cyan,
}

%% Tikz
\usepackage{pgfplots}
\pgfplotsset{compat=1.16}
\usetikzlibrary{external}
\tikzexternalize
\tikzsetexternalprefix{figs/}

%% Géométrie
\textwidth=18cm \textheight=23cm \oddsidemargin=-1.00cm
\evensidemargin=-1.00cm
\parindent=1cm
\topmargin=-2cm
\setlength{\parindent}{2em}
\setlength{\parskip}{1em}
\renewcommand{\baselinestretch}{1.5}

%% Théorème
\usepackage{amsthm}
\renewcommand{\qedsymbol}{\hfill$\blacksquare$}
\newcounter{proofcounter}[section]
\addto\captionsfrench{\renewcommand\proofname{\stepcounter{proofcounter}\color{bluearmy}\emph{\textbf{Solution~\thesection.\theproofcounter}}}}
\newtheorem{remark}{Remarque}[section]
\numberwithin{equation}{section}
\newtheorem{lemma}{Lemme}[section]

%% Raccourcis mathématiques
\newcommand{\K}{\mathbb{K}} \newcommand{\R}{\mathbb{R}}
\newcommand{\C}{\mathbb{C}} \newcommand{\Q}{\mathbb{Q}}
\newcommand{\N}{\mathbb{N}} \newcommand{\Z}{\mathbb{Z}}
\newcommand{\U}{\mathbb{U}} \newcommand{\E}{\mathbb{E}}
\newcommand{\M}{\mathcal{M}} \renewcommand{\L}{\mathcal{L}}
\renewcommand{\P}{\mathbb{P}} \newcommand{\im}{\mathrm{Im}}
\renewcommand{\i}{\mathrm{i}} \newcommand{\e}{\mathrm{e}}
\renewcommand{\j}{\mathrm{j}} \renewcommand{\d}{\mathrm{d}}
\DeclareMathOperator{\sgn}{sgn} \DeclareMathOperator{\diag}{diag}
\DeclareMathOperator{\rg}{rg} \DeclareMathOperator{\Tr}{Tr}
\DeclareMathOperator{\Sp}{Sp} \DeclareMathOperator{\mat}{mat}
\DeclareMathOperator{\com}{com} \DeclareMathOperator{\conv}{conv}
\DeclareMathOperator{\ppcm}{ppcm} \DeclareMathOperator{\Vect}{Vect}
\newcommand{\vertiii}[1]{{\left\vert\kern-0.25ex\left\vert\kern-0.25ex\left\vert{}#1
\right\vert\kern-0.25ex\right\vert\kern-0.25ex\right\vert}}
\newcommand{\function}[5]{
	\begin{equation}
		\begin{array}{rccl}
			#1: & #2 & \to & #3 \\
			& #4 & \mapsto & #5
		\end{array}
	\end{equation}
}

\includeonly{solutions_integration}

\begin{document}

\begin{titlepage}
	\centering
	\vspace*{\fill}
	\Huge \textit{\textbf{Solutions Exercices MP/MP$^*$}}
	\vspace*{\fill}
\end{titlepage}

\cleardoublepage

\tableofcontents

\cleardoublepage

\section{Algèbre Générale}

\begin{proof}
	Soit $(x,y)\in G^{2}$. On a d'abord
	\begin{align}
		x\cdot y
		&=(x\cdot y)^{p+1}(x\cdot y)^{-p}\\
		&=x^{p+1}\cdot y^{p+1}\cdot y^{-p}\cdot x^{-p}\\
		&=x^{p+1}\cdot y \cdot x^{-p} \label{eq:1.3}
	\end{align}
	On cherche maintenant à montrer que $x^{p+1}$ et $y$ commutent.
	On a
	\begin{align}
		y^{p+2}\cdot x^{p+2}
		&=(y\cdot x)^{p+2}\\
		&=(y\cdot x)^{p+1}\cdot y\cdot x\\
		&=y^{p+1}\cdot x^{p+1}\cdot y\cdot x
	\end{align}
	Donc on a $y\cdot x^{p+1}=x^{p+1}\cdot y$. En reportant dans~\eqref{eq:1.3}, on a $x\cdot y=y\cdot x$ et donc 
	\begin{equation}
		\boxed{\text{G est abélien.}}
	\end{equation}
\end{proof}

\begin{remark}
	\phantom{}
	\begin{itemize}
		\item Pour $(\Sigma_{3},\cdot)$, on a $f_{0},f_{1}$ et $f_{6}$ des morphismes mais $\Sigma_{3}$ n'est pas commutatif.
		\item Si $f_{2}$ est un morphisme, alors on a $(x\cdot y)^{2}=x\cdot y\cdot x\cdot y=x^{2}\cdot y^{2}$ d'où $y\cdot x=x\cdot y$.
	\end{itemize}
\end{remark}

\begin{proof}
	$A$ est non vide car $\omega(e_{G})=1$ et $e_{G}\in A$. Soit $x\in A$ tel que $\omega(x)=2p+1$. Soit $k\in\Z$, on a 
	\begin{align}
		x^{2k}=e_{G}
		&\Leftrightarrow 2p+1\mid 2k\\
		&\Leftrightarrow 2p+1\mid k
	\end{align}
	d'après le théorème de Gauss.

	Ainsi, $\omega(x^{2})=2p+1$ et $x^{2}\in A$, donc \function{\varphi}{A}{A}{x}{x^{2}} est bien définie. Soit $x\in A$, il existe $p\in\N$ tel que $x^{2p+1}=e_{G}$ donc $x^{2p+2}=x$ d'où $(x^{p+1})^{2}=x$. Il suffit donc de vérifier que $x^{p+1}\in A$ pour montrer que l'application est surjective. Comme $A$ est fini, elle sera bijective.

	On a $gr\{x^{p+1}\}\subset gr\{x\}$ et $(x^{p+1})^{2}=x$ donc $gr\{x\}=gr\{x^{p+1}\}$ donc $\omega(x)=\omega(x^{p+1})=2p+1$ et donc $x^{p+1}\in A$.

	\begin{equation}
		\boxed{\text{Donc }A\text{ est bijective.}}
	\end{equation}
\end{proof}

\begin{proof}
	On note $m=\theta(\sigma)$. On suppose que $\sigma$ se décompose en produit de cycle de longueur $l_{1},\dots,l_{m}$ avec $l_{1}+\dots+l_{m}=n$. Comme
	\begin{equation}
		(a_{1},\dots,a_{l})=[a_{1},a_{2}]\circ[a_{2},a_{3}]\circ\dots\circ[a_{l-1},a_{l}]
	\end{equation}
	Donc $\sigma$ se décompose en $\sum_{i=1}^{m}(l_{i}-1)=n-m$ transpositions. Montrons par récurrence sur $k$, $\mathcal{H}(k)\colon$ "Un produit de $k$ transpositions possède au moins $n-k$ orbites".

	Pour $k=0$, $\sigma=id$ possède $n$ orbites.

	Pour $k=1$, soit $\tau$ une transposition, on a $\theta(\tau)=n-2+1=n-1$.

	Soit $k\in\N$, supposons $\mathcal{H}_{k}$, soit $\sigma\in\Sigma_{n}$ qui se décompose en produit de $k+1$ transpositions.
	\begin{equation}
		\sigma=\underbrace{\tau_{1}\circ\dots\tau_{k}}_{\sigma'}\circ\tau_{k+1}
	\end{equation}
	D'après $\mathcal{H}_{k}$, on a $\theta(\sigma')\geqslant n-k$. Notons $\tau_{k+1}=[a,b]$. 
	
	Si $a$ et $b$ appartiennent à la même orbite. On note $(a_{1},\dots,a_{r})$ le cycle correspondant avec $a_{r}=a$ et $a_{s}=b$ où $s\in\left\llbracket 1,n-1\right\rrbracket$. On a 
	\begin{equation}
		\left\{
			\begin{array}[]{ll}
				(a_{1},\dots,a_{r-1},a_{r})\circ[a,b](a_{i})=a_{i+1} &\text{où }i\notin\{r,s\}\\
				(a_{1},\dots,a_{r-1},a_{r})\circ[a,b](a_{r})=a_{s+1}&\\
				(a_{1},\dots,a_{r-1},a_{r})\circ[a,b](a_{s})=a_{1}&
			\end{array}
		\right.
	\end{equation}
	
	On n'a pas perdu d'orbites, donc $\theta(\sigma)\geqslant n-k-1$. 

	Si $a$ et $b$ n'appartiennent pas à la même orbite, notons $(a_{1},\dots,a_{r})$ et $(b_{1},\dots,b_{s})$ ces orbites avec $a=a_{r}$ et $b=b_{s}$. On a 
	\begin{equation}
		\left\{
			\begin{array}[]{ll}
				\underbrace{(a_{1},\dots,a_{r-1},a_{r})\circ(b_{1},\dots,b_{s})\circ[a_{r},b_{s}]}_{\sigma''}(a_{i})=a_{i+1} &\text{où }i\in\left\llbracket 1,\dots,r-1\right\rrbracket\\
				(a_{1},\dots,a_{r-1},a_{r})\circ(b_{1},\dots,b_{s})\circ[a_{r},b_{s}](b_{j})=b_{j+1} &\text{où }j\in\left\llbracket 1,\dots,s-1\right\rrbracket\\
				(a_{1},\dots,a_{r-1},a_{r})\circ(b_{1},\dots,b_{s})\circ[a_{r},b_{s}](a_{r})=b_{1}&\\
				(a_{1},\dots,a_{r-1},a_{r})\circ(b_{1},\dots,b_{s})\circ[a_{r},b_{s}](b_{s})=a_{1}&
			\end{array}
		\right.	
	\end{equation}

	Donc 
	\begin{equation}
		\sigma''=(a_{1},\dots,a_{r},b_{1},\dots,b_{s})
	\end{equation}
	On a perdu une orbite et donc $\theta(\sigma)\geqslant n-k-1$. 
	
	\begin{equation}
		\boxed{\text{D'où le résultat par récurrence sur k.}}
	\end{equation}
\end{proof}

\begin{proof}
	On note par $\overline{k}$ les éléments de $\Z/n\Z$ et par $\widetilde{l}$ les éléments de $\Z/m\Z$.

	Soit $f$ un morphisme. On pose $f(\overline{1})=\widetilde{x}$ où $x\in\left\llbracket 0,m-1\right\rrbracket$. On a donc $nf(\overline{1})=f(\overline{0})=\widetilde{0}$.

	On a donc $\widetilde{nx}=\widetilde{0}$ donc $m\mid nx$. On écrit $m=m_{1}(m\wedge n)$ et $n=n_{1}(m\wedge n)$. D'après le théorème de Gauss, on a donc $m_{1}\mid x$. Donc $x=km_{1}$ avec $k\in\left\llbracket 0,(n\wedge m)-1\right\rrbracket$.

	Réciproquement, soit $k\in\left\llbracket 0,(n\wedge m)-1\right\rrbracket$. On définit 
	\function{f_k}{\Z/n\Z}{\Z/m\Z}{\overline{l}}{\widetilde{lkm_{1}}}
	Si $\overline{l}=\overline{l'}$, alors $n\mid l-l'$ et donc $nm_{1}\mid (l-l')km_{1}$ puis $n_{1}(n\wedge m)m_{1}\mid (l-l')km_{1}$ donc $m\mid (l-l')km_{1}$ d'où $\widetilde{lkm_{1}}=\widetilde{l'km_{1}}$ donc $f$ est bien définie et c'est évidemment un morphisme. 
	
	Soit $k,k'\in\left\llbracket 0,n\wedge m-1\right\rrbracket$ avec $k\neq k'$. Si $\widetilde{km_{1}}=\widetilde{k'm_{1}}$ alors $m\mid (k-k')m_{1}$ et donc $n\wedge m\mid k-k'$ et $\vert k-k'\vert< n\wedge m$ donc $k=k'$ ce qui est absurde. Ainsi, les $f_{k}$ sont distincts. 
	
	\begin{equation}
		\boxed{\text{On a donc }n\wedge m\text{ morphismes.}}
	\end{equation}
\end{proof}

\begin{remark}
	Exemple pour l'exercice précédent: morphisme de $\Z/4\Z$ dans $\Z/6\Z$. On a $f(\overline{1})=\widetilde{x}$ d'où $\widetilde{4x}=\widetilde{0}$ donc $3\mid x$ d'où $x\in\left\{0,3\right\}$. On a donc le morphisme trivial $f_{0}\colon \overline{l}\mapsto\widetilde{0}$ et $f_{1}\colon\overline{l}\mapsto\widetilde{3l}$.
\end{remark}

\begin{proof}
	On considère $H=\{x\in G\bigm| x^{2}=e_{G}\}$. Si $x\notin H$, alors $x^{-1}\neq x$ et donc 
	
	\begin{equation}
		P=\prod_{x\in H}x	
	\end{equation}
	
	$H$ est le noyau du morphisme $x\mapsto x^{2}$ (morphisme car $G$ est abélien) donc $H$ est un sous-groupe. Soit $K$ un sous-groupe de $H$ et $a\in H\setminus K$. Montrons que $K\cup aK$ est un sous-groupe de $H$.
	
	On a $e_{G}\in K\cup aK$. Soit $x\in K\cup aK\subset H$, on a $x^{-1}=x\in K\cup aK$. Soit $(x_{1},x_{2})\in (K\cup aK)^{2}$, si $(x_{1},x_{2})\in K^{2},$, c'est ok. Si $(x_{1},x_{2})\in (aK)^{2}$, on note $x_{1}=a\cdot k_{1}$ et $x_{2}=a\cdot k_{2}$ avec $(k_{1},k_{2})\in K^{2}$. On a $x_{1}\cdot x_{2}=a^{2}\cdot k_{1}\cdot k_{2}=k_{1}\cdot k_{2}\in K$. Si $x_{1}\in K$ et $x_{2}\in aK$, alors $x_{1}\cdot x_{2}=a\cdot k_{1}\cdot k_{2}\in aK$. Donc $K\cup aK$ est un sous-groupe de $H$.

	Soit $x\in K\cap aK$, il existe $(k_{1},k_{2})\in K^{2}$ tel que $k_{1}=a\cdot k_{2}$ et $a\in K$ ce qui est impossible. Donc $K\cap aK=\emptyset$.

	On construit alors par récurrence $K_{n}$: on pose $K_{0}=\{e_{G}\}$ et à l'étape $n$, si $K_{n}=H$ on arrête, sinon il existe $a_{n+1}\in H\setminus K_{n}$ et on pose $K_{n+1}=K_{n}\cup a_{n+1}K$. Alors $\vert K_{n+1}\vert=2\vert K_{n}\vert$. Comme $H$ est fini, il existe $n_{0}\in\N$ tel que $H=K_{n_{0}}$. On a alors $\vert H\vert=2^{n_{0}}$.

	Ainsi, si $n_{0}=0$, on a $H=\{e_{G}\}$ et 
	\begin{equation}
		\boxed{P=e_{G}}
	\end{equation}
	
	Si $n_{0}=1$, on a $H=\{e_{G},a_{1}\}$ et 
	\begin{equation}
		\boxed{P=a_{1}\neq e_{G}}
	\end{equation}
	
	Si $n_{0}\geqslant 2$, comme chaque $a_{k}$ apparaît un nombre pair de fois dans le produit, on a 
	
	\begin{equation}
		\boxed{P=e_{G}}
	\end{equation}
\end{proof}

\begin{proof}
	Soit $x_{0}\in\R$. $(\overline{kx_{0}})_{0\leqslant k\leqslant n}$ ne sont pas deux à deux distincts. Donc il existe $l\neq l'\in\left\llbracket 0,n\right\rrbracket^{2}$ tel que $\overline{lx_{0}}=\overline{l'x_{0}}$ d'où $0<\vert l-l'\vert\leqslant n$. Donc il existe $j\in\left\llbracket 1, n\right\rrbracket$ avec $jx_{0}\in G$. Ainsi, $n!x_{0}\in G$ (itéré de $jx_{0}$). Ce raisonnement est vrai pour $x=\frac{x_{0}}{n!}$ donc $x_{0}\in G$. Ainsi, 
	\begin{equation}
		\boxed{G=\R}
	\end{equation}
\end{proof}

\begin{proof}
	Soit $f$ un isomorphisme de $\Z/n\Z$ dans lui-même. Soit $k\in\left\llbracket 0, n-1\right\rrbracket$, on a $f(\overline{k})=kf\overline{1})$. Par isomorphisme, $\omega(f(\overline{1}))=\omega(\overline{1})=n$. Notons alors $\overline{x}=f(\overline{1})$ avec $x\in\left\llbracket 0,n-1\right\rrbracket$.

	Si $x\wedge n=1$, il existe $(u,v)\in\Z^{2}$ tel que $ux+vn=1$, donc $u\overline{x}=\overline{1}\in gr\{\overline{x}\}$. Ainsi, $Z\/n\Z=gr\{\overline{x}\}$ (car les éléments de $\Z/n\Z$ sont des itérés de $\overline{1}$) donc $\omega(\overline{x})=n$.

	Réciproquement, si $\omega(\overline{x})=n$, $\overline{1}\in gr\{\overline{x}\}$ donc il existe $u\in\Z$ tel que $u\overline{x}=1=\overline{ux}$. Donc $n\mid ux-1$, c'est-à-dire qu'il existe $v\in\Z$ tel que $ux-1=vn$, d'où $ux+vn=1$. D'après Bézout, on a $x\wedge n=1$. Finalement, on a $\omega(\overline{x})=n$ si et seulement si $x\wedge n=1$.

	Ainsi, les isomorphismes sont nécessairement de la forme 
	\function{f_{x}}{\Z/n\Z}{\Z/n\Z}{\overline{k}}{\overline{kx}}
	où $x\in\left\llbracket 0,n-1\right\rrbracket$ et $x\wedge n=1$.

	Réciproquement, si $x\in\left\llbracket 0,n-1\right\rrbracket$ est tel que $x\wedge n=1$, $f_{x}$ est évidemment un morphisme. Si $\overline{k}\in\ker\left(f_{x}\right)$, on a $f_{x}\left(\overline{k}\right)=\overline{0}$ si et seulement si $\overline{kx}=\overline{0}$ si et seulement si $n\mid kx$ et comme $n\wedge x=1$, d'après le théorème de Gauss, on a $n\mid k$ donc $\overline{k}=\overline{0}$ donc $\ker\left(f_{x}\right)=\left\{\overline{0}\right\}$. Donc $f_{x}$ est injective, donc bijective car $\left\vert\Z/n\Z\right\vert=\left\vert\Z/n\Z\right\vert=n$.
\end{proof}

\begin{proof}
	Si $y\in \im\varphi$, $y$ possède $\vert\ker\varphi\vert$ antécédents. En effet, il existe $x_{0}\in G$ tel que $y=\varphi(x_{0})$. Pour tout $x\in G$, on a $\varphi(x)=y$ si et seulement si $\varphi(x)=\varphi(x_{0})$ si et seulement si $\varphi(x_{0}^{-1}\cdot x)=e_{G}$ si et seulement si $x_{0}^{-1}\cdot x\in\ker\varphi$ si et seulement si $x\in x_{0}\ker\varphi$. Comme \function{g}{\ker\varphi}{x_{0}\ker\varphi}{x}{x\cdot x_{0}}
	est bijective, on a $\vert\ker\varphi\vert=\vert x_{0}\varphi\vert$. Ainsi, on a $\vert G\vert=\vert\im\varphi\vert\times\vert\ker\varphi\vert$.

	Dans tous les cas, on a $\ker\varphi\subset\ker\varphi^{2}$ et $\im\varphi^{2}\subset\im\varphi$. On a ensuite 
	\begin{align}
		\im\varphi^{2}=\im\varphi
		&\Longleftrightarrow \vert\im\varphi^{2}\vert=\vert\im\varphi\vert\\
		&\Longleftrightarrow \vert\ker\varphi^{2}\vert\vert\im\varphi^{2}\vert=\vert\ker\varphi^{2}\vert\vert\im\varphi\vert=\vert G\vert=\vert\ker\varphi\vert\vert\im\varphi\vert\\
		&\Longleftrightarrow \vert\ker\varphi^{2}\vert=\vert\ker\varphi\vert\\
		&\Longleftrightarrow \boxed{\ker\varphi^{2}=\ker\varphi}
	\end{align}
\end{proof}

\begin{proof}
	On considère \function{f}{G}{G}{x}{x^{m}}
	l'exercice revient à montrer que $f$ est bijective. D'après le théorème de Bézout, il existe $(a,b)\in\Z^{2}$ tel que $am+bn=1$. Soit $y\in G$, on a 
	\begin{equation}
		y^{1}=y=y^{am+bn}=y^{am}\cdot \underbrace{y^{bn}}_{=e_{G}}=y^{am}=(y^{a})^{m}
	\end{equation}
	Donc $f$ est surjective et comme $G$ est fini, 
	\begin{equation}
		\boxed{\text{f est bijective.}}
	\end{equation}
\end{proof}

\begin{proof}
	\phantom{}
	\begin{enumerate}
		\item On a $e_{G}\in S_{g}$, si $(x,y)\in S_{g}^{2}$ alors $x\cdot y\cdot g=x\cdot g\cdot y=g\cdot x\cdot y$ donc $x\cdot y\in S_{g}$ et si $x\in S_{g}$ alors $x\cdot g=g\cdot x$ implique $g\cdot x^{-1}=x^{-1}\cdot g$ en multipliant par l'inverse de $x$ à gauche et à droite donc 
		\begin{equation}
			\boxed{x^{-1}\in S_{g}}
		\end{equation}
		
		\item Soit $(h,h')\in G^{2}$. On a $h\cdot g\cdot h^{-1}=h'\cdot g\cdot h'^{-1}$ si et seulement si $g\cdot h^{-1}\cdot h'=h^{-1}\cdot h\cdot g$ si et seulement si $h^{-1}\cdot h\in S_{g}$ si et seulement si $h'\in hS_{g}$. Or $\vert hS_{g}\vert=\vert S_{g}\vert$ car \function{I_{h}}{S_{g}}{hS_{g}}{x}{h\cdot x} est bijective de réciproque $I_{h^{-1}}$. Soit la relation d'équivalence $\mathcal{R}_{0}$ sur $G$ définie par $h\mathcal{R}_{0}h'$ si et seulement si $h\cdot g\cdot h^{-1}=h'\cdot g\cdot h'^{-1}$. Chaque classe à $\vert S_{g}\vert$ éléments et il y y a $\vert C(g)\vert$ classes dans $G$ d'où 
		\begin{equation}
			\boxed{\left\lvert G\right\rvert=\left\lvert S_{g}\right\rvert\left\lvert C(g)\right\rvert}
		\end{equation}
		
		\item On a $Z(G)=\cap_{g\in G}S_{g}$ donc $Z(G)$ est un sous-groupe et pour tout $g\in G$, 
		\begin{equation}
			\boxed{Z(G)\subset S_{g}}
		\end{equation}
		
		\item Pour $x\in G$, on note $\overline{x}=\{h\cdot x\cdot h^{-1}\bigm| h\in G\}=C(x)$. 
		
		On a $\vert\overline{x}\vert=1$ si et seulement si pour tout $h\in G$, $h\cdot x\cdot h^{-1}=x$ si et seulement si $x\in Z(G)$.
		
		Soit $\mathcal{A}$ une partie de $G$ telle que $(\overline{x})_{x\in\mathcal{A}}$ forme une partition de $G\setminus Z(G)$. On a 
		\begin{equation}
			\vert G\vert=p^{\alpha}=\vert Z(G)\vert+\sum_{x\in\mathcal{A}}\vert C(x)\vert
		\end{equation}
		Si $x\in\mathcal{A}$, $x\notin Z(G)$ donc $\vert S_{x}\vert <\vert G\vert$ (car $x\in Z(G)$ si et seulement si $S_{x}=G$) et donc 
		\begin{equation}
			\vert C(x)\vert=\frac{\vert G\vert}{\vert S_{x}\vert}
		\end{equation}
		d'après 2. Donc $\vert C(x)\vert=p^{\beta}$ avec $\beta\in\left\llbracket 1,\alpha\right\rrbracket$ car $\vert C(x)\vert\neq 1$. Donc 
		\begin{equation}
			p\Bigm|\sum_{x\in\mathcal{A}}\vert C(x)\vert
		\end{equation}
		d'où 
		\begin{equation}
			p\bigm|\left\lvert Z(G)\right\rvert
		\end{equation}
		donc 
		\begin{equation}
			\boxed{\left\lvert Z(G)\right\rvert\neq1}
		\end{equation}

		\item On a 
		\begin{equation}
			p^{2}=\vert Z(G)\vert+\sum_{x\in\mathcal{A}}\vert C(x)\vert
		\end{equation}
		D'après la question 4, on a $\vert Z(G)\vert\neq1$ et $\vert Z(G)\vert\bigm|\vert G\vert$.

		Si $Z(G)\neq G$, alors $\vert Z(G)\vert=p$. Pour $x\in\mathcal{A}$, $Z(G)\subset S_{x}\neq G$ donc $\vert S_{x}\vert= p$ (car $\vert S_{x}\vert\bigm|\vert G\vert$) et donc $Z(G)=S_{x}$. Or $x\in S_{x}$ et $x\notin Z(G)$ ce qui n'est pas possible, donc $\vert Z(G)\vert=p^{2}$ et $Z(G)=G$. 
		
		\begin{equation}
			\boxed{\text{Donc G est abélien.}}
		\end{equation}

		S'il existe un élément d'ordre $p^{2}$. $G$ est cyclique et est isomorphe à $\Z/p^{2}\Z$. Sinon, pour tout $x\in G\setminus\{e_{G}\}$, on a $\omega(x)=p$. Soit $x_{1}\in G\setminus\{e_{G}\}$ et $x_{2}\in G\setminus gr\{x_{1}\}$.
		Soit \function{f}{\left(\Z/p\Z\right)^{2}}{G}{(\overline{k},\overline{l})}{x_{1}^{k}\cdot x_{2}^{l}}
		$f$ est bien définie car si $\overline{k}=\overline{k'}$ et $\overline{l}=\overline{l'}$, on a $p\mid k-k'$ et $p\mid l-l'$ donc $x_{1}^{k}\cdot x_{2}^{l}=x_{1}^{k'}\cdot x_{2}^{l'}$. Comme $G$ est abélien, $f$ est un morphisme. 
		
		Montrons que $f$ est injective. Soit $(\overline{k},\overline{l})\in\ker(f)$ avec $(k,l)\in\left\llbracket 0,p-1\right\rrbracket^{2}$, on a $x_{1}^{k}\cdot x_{2}^{l}=e_{G}$ donc $x_{2}^{l}=x_{1}^{-k}$. Si $l\in\left\llbracket 1,p-1\right\rrbracket$ or $p$ est premier donc $l\wedge p=1$ donc il existe $(u,v)\in\Z^{2}$ tel que $lu+pv=1$. Alors on a 
		\begin{equation}
			x_{2}=x_{2}^{lu+pv}=x_{2}^{lu}\cdot x_{2}^{pv}=x_{2}^{lu}=x_{1}^{-k}\in gr\{x_{1}\}
		\end{equation} ce qui n'est pas possible. Donc $\overline{l}=\overline{0}$ et de même $\overline{k}=\overline{0}$ donc $f$ est injective et ainsi 
		$\vert\Z/p^{2}\Z\vert=\vert G\vert$ donc 
		\begin{equation}
			\boxed{\text{f est un isomorphisme.}}
		\end{equation}
	\end{enumerate}
\end{proof}

\begin{remark}
	Les groupes de cardinal $p^{3}$ ne sont pas nécessairement abélien, par exemple le groupe des isométries du carré $\mathcal{D}_{4}$ de cardinal 8.
\end{remark}

\begin{proof}
	Soit $f$ un morphisme de $(\Z,+)$ dans $(\Q_{+}^{*},\times)$. Pour tout $n\in\Z$, $f(n)=f(1)^{n}$ donc il existe $r_{0}\in\Q_{+}^{*}$ tel que $f(1)=r_{0}$ donc 
	\begin{equation}
		\boxed{f\colon n\mapsto r_{0}^{n}}
	\end{equation}

	Soit $f$ un morphisme de $(\Q,+)$ dans $(\Q_{+}^{*},\times)$. Pour tout $a\in\N^{*}$, $f(1)=f(\frac{1}{a})^{a}$. Pour tout $p$ premier, on a $\nu_{p}(f(1))=a\nu_{p}(f(\frac{1}{a}))$ donc pour tout $a\in\N^{*}$, $a\mid\nu_{p}(f(1))$ donc $\nu_{p}(f(1))=0$ pour tout $p$ premier, donc $f(1)=1$. Ainsi, pour tout $n\in\Z$, $f(n)=f(1)^{n}=1$ et $f(b\times\frac{a}{b})=f(a)=1=f(\frac{a}{b})^{b}$ donc $f(\frac{a}{b})=1$. Donc 
	\begin{equation}
		\boxed{f\colon r\mapsto 1}
	\end{equation}
\end{proof}

\begin{proof}
	On a $xy=y^{2}x$, $x^{2}y=xy^{2}x=y^{4}x^{2}$, $x^{3}y=x^{2}y^{2}x=xy^{4}x^{2}=y^{8}x^{3}$, $x^{5}y=y^{32}x^{5}$ donc $y^{31}=e_{G}$ et $\omega(y)=31$. 
	
	Tout élément de $G$ peut s'écrire $y^{\lambda}x^{\mu}$ avec $\left(\lambda,\mu\right)\in\left\llbracket 0,30\right\rrbracket \times\left\{0, 4\right\}$. Soit \function{f}{\left\llbracket 0,30\right\rrbracket\times\left\llbracket 0, 4\right\rrbracket}{G}{(\lambda,\mu)}{y^{\lambda}x^{\mu}} est surjective par construction. Soit $((\lambda,\mu),(\lambda',\mu'))\in(\left\llbracket 0,30\right\rrbracket\times\left\llbracket 0, 4\right\rrbracket)^{2}$ tel que $y^{\lambda}x^{\mu}=y^{\lambda'}x^{\mu'}$ donc $y^{\lambda-\lambda'}=x^{p'-p}$ d'où $y^{5(\lambda-\lambda')}=x^{5(\mu'-\mu)}=e_{G}$. Or $\omega(y)=31$ donc $31\mid 5(\lambda-\lambda')$ et d'après le théorème de Gauss, $31\mid \lambda-\lambda'$. Or $(\lambda,\lambda')\in\left\llbracket 0,30\right\rrbracket^{2}$ donc $\lambda=\lambda'$ et de même $\mu=\mu'$ donc $f$ est injective donc bijective et 
	\begin{equation}
		\boxed{\vert G\vert=155}
	\end{equation}
	
	Soit $G'$ un autre tel groupe engendré par $x'$ et $y'$, on forme \function{g}{G}{G}{y^{p}x^{\mu}}{y'^{\lambda}x'^{\mu}}
	et on vérifie que $g$ est un isomorphisme.
\end{proof}

\begin{proof}
	\phantom{}
	\begin{enumerate}
		\item Soit $i\in\left\llbracket 1,r\right\rrbracket$, il existe nécessairement $y_{i}\in G$ tel que $\nu_{p_{i}}(\omega(y_{i}))=p_{i}^{\alpha_{i}}$ (où $\nu_{p}$ est la valuation $p$-adique), sinon on ne pourrait pas avoir ce terme dans le $\ppcm$. Donc 
		\begin{equation}
			\boxed{p_{i}^{\alpha_{i}}\mid \omega(y_{i})}
		\end{equation}
		
		\item Il existe $n\in\N$ tel que $\omega(y_{i})=p_{i}^{\alpha_{i}}n$. Posons $x_{i}=y_{i}^{n}\in G$. Alors pour $k\in\N$,
		\begin{equation}
			x_{i}^{k}=e_{G}\Longleftrightarrow y_{i}^{nk}=e_{G}\Longleftrightarrow \omega(y_{i})\mid nk\Longleftrightarrow p_{i}^{\alpha_{i}}\mid k
		\end{equation}
		Donc 
		\begin{equation}
			\boxed{\omega(x_{i})=p_{i}^{\alpha_{i}}}
		\end{equation}

		\item On pose $x=\prod_{i=1}^{r}x_{i}$. Soit $k\in\N$, alors 
		\begin{equation}
			x^{k}=e_{G}\Longleftrightarrow \prod_{i=1}^{r}x_{i}^{k}=e_{G}
		\end{equation}
		Pour $i\in\left\llbracket 1,r\right\rrbracket$, on met le tout à la puissance $M_{i}=\prod_{\substack{j=1\\j\neq i}}^{r}p_{j}^{\alpha_{j}}$. On a alors, pour tout $i\in\left\llbracket 1,r\right\rrbracket$,
		\begin{equation}
			x_{i}^{kM_{i}}=e_{G}\Longleftrightarrow p_{i}^{\alpha_{i}}\mid kM_{i}\Longleftrightarrow p_{i}^{\alpha_{i}}\mid k
		\end{equation}
		la dernière équivalence venant du théorème de Gauss. Donc pour tout $i\in\left\llbracket 1,r\right\rrbracket$, $p_{i}^{\alpha_{i}}\mid k$, ce qui équivaut donc à $N\mid k$ et donc 
		\begin{equation}
			\boxed{\omega(x)=N}
		\end{equation}
	\end{enumerate}
\end{proof}

\begin{proof}
	Sur un corps commutatif, un polynôme de degré $n$ admet au plus $n$ racines. Montrons qu'il existe $x_{1}\in\K^{*}$ tel que $\omega(x_{i})=\vert\K^{*}\vert$. Par définition de $N$, pour tout $x\in\K^{*}$, $\omega(x)\mid N$. D'où $x^{N}=1_{\K}$. Donc $x$ est racine de $X^{N}-1$. Ainsi, $\vert\K^{*}\vert\leqslant N$. Par ailleurs, $N\mid\vert\K^{*}\vert$ car pour tout $x\in\K^{*}$, $x^{\vert\K^{*}\vert}=1_{\K^{*}}$. Donc $\vert\K^{*}\vert=N$ et ainsi 
	\begin{equation}
		\boxed{\K^{*}=gr\left\{x_{1}\right\}}
	\end{equation}

	On a $\vert\Z/13\Z^{*}\vert=12$ donc pour tout $\overline{x}\in(\Z/13\Z)^{*}$, $\omega(\overline{x})\in\left\{1,2,3,4,6,12\right\}$. On a $\overline{2}^{2}=\overline{4}$, $\overline{2}^{3}=\overline{8}$, $\overline{2}^{4}=\overline{16}=\overline{3}$, $\overline{2}^{6}=\overline{12}$ donc $\omega(\overline{2})=12$ et 
	\begin{equation}
		\boxed{
		\Z/13\Z^{*}=gr\left\{\overline{2}\right\}=\left\{\overline{2}^{k}\bigm| k\in\left\llbracket 0,11\right\rrbracket\right\}}
	\end{equation}
\end{proof}

\begin{proof}
	\phantom{}
	\begin{enumerate}
		\item Soit $(x,y)\in G^{2}$, on a $(x\cdot y)^{2}=(x\cdot y)\cdot (x\cdot y)=e_{G}$ donc $x\cdot y=y^{-1}\cdot x^{-1}$ et comme $x^{2}=e_{G}$, $x^{-1}=x$ d'où $xy=yx$ et 
		\begin{equation}
			\boxed{\text{G est abélien.}}
		\end{equation}
		
		\item Soit $(x_{1},\dots,x_{n})$ une famille génératrice minimale de $G$: pour tout $x\in G$, il existe$(\varepsilon_{i})\in\left\{0,1\right\}^{n}$ tel que $x=\prod_{i=1}^{n}x_{i}^{\varepsilon_{i}}$ (car $G$ est abélien).
		Soit \function{f}{(\Z/2\Z)^{n}}{G}{(\overline{\varepsilon_{1}},\dots,\overline{\varepsilon_{n}})}{\prod_{i=1}^{n}x_{i}^{\varepsilon_{i}}}
		Si pour tout $i\in\left\llbracket 1,n\right\rrbracket$ on a $\overline{\varepsilon_{i}}=\overline{\varepsilon_{i}'}$, alors $x^{\varepsilon_{i}}=x^{\varepsilon_{i}'}$ car $x_{i}^{2}=e_{G}$ et $2\mid\varepsilon_{i}-\varepsilon_{i}'$. Donc $f$ est bien définie.

		$f$ est clairement un morphisme (car $G$ est abélien). D'après la première question, $f$ est surjective. Montrons que $f$ est injective. Soit $(\overline{\varepsilon_{1}},\dots,\overline{\varepsilon_{n}})$ tel que $\prod_{i=1}^{n}x_{i}^{\varepsilon_{i}}=e_{G}$. Soit $i\in\left\llbracket 1,n\right\rrbracket$, supposons $\varepsilon_{i}$ impair, on a alors $x_{i}=\varepsilon_{i}=x_{i}$. D'où $x_{i}=\prod_{j=1}^{n}x_{j}^{-\varepsilon_{j}}=\prod_{j=1}^{n}x_{j}^{\varepsilon_{j}}$ car $x^{2}=e_{G}$. Donc $x_{i}\in gr(x_{j},j\in\left\llbracket 1,n\right\rrbracket, j\neq i)$, ce qui contredit le caractère minimal de $(x_{1},\dots,x_{n})$. 
		
		\begin{equation}
			\boxed{\text{Ainsi, f est injective donc est un isomorphisme.}}
		\end{equation}
	\end{enumerate}
\end{proof}

\begin{remark}
	En notant $+$ la loi sur $G$, on peut définir \function{f}{\Z/2\Z\times G}{G}{(\varepsilon,x)}{x^{\varepsilon}}. Alors $(G,+,\cdot)$ est un $\Z/2\Z$-espace vectoriel, de dimension finie $n$ car $G$ est fini, et le choix d'une base réalise un isomorphisme de $((\Z/2\Z)^{n},+)$ dans $(G,+)$.
\end{remark}

\begin{remark}
	Par isomorphisme, on a 
	\begin{equation}
		\prod_{x\in G}x=f\left(\sum_{(\overline{\varepsilon_{1}},\dots,\overline{\varepsilon_{n}})\in(\Z/2\Z)^{n}}\left(\overline{\varepsilon_{1}},\dots,\overline{\varepsilon_{n}}\right)\right)
	\end{equation}

	Pour $n=1$, on a $\overline{0}+\overline{1}=\overline{1}$, pour $n=2$, on a $(\overline{0},\overline{0})+(\overline{0},\overline{1})+(\overline{1},\overline{0})+(\overline{1},\overline{1})=(\overline{0},\overline{0})$. Pour $n>2$, $\overline{1}$ apparaît $2^{n+1}$ fois sur chaque coordonnée (donc un nombre pair de fois), donc la somme fait $(\overline{0},\dots,\overline{0})$.
\end{remark}

\begin{proof}
	\phantom{}
	\begin{enumerate}
		\item Si $G$ est abélien, on a 
		\begin{equation}
			\boxed{D(G)=\left\{e_{G}\right\}}
		\end{equation}
		\item Soit $\sigma\in\mathcal{A}_{n}$, $\sigma$ se décompose en un produit d'un nombre pair de transpositions. Soient $[a,b]$ et $[c,d]$ deux transpositions.
		\begin{itemize}
			\item Si $\left\{a,b\right\}=\left\{c,d\right\}$, alors $[a,b]\circ[c,d]=id$.
			\item Si $a\in\left\{c,d\right\}$, supposons par exemple $a=c$ et $b\neq d$. On a alors $[a,b]\circ[c,d]=[a,b]\circ[a,d]=[b,a,d]$.
			\item Si $\left\{a,b\right\}\cap\left\{c,d\right\}=\emptyset$, on a 
			\begin{equation}
				[a,b]\circ[c,d]=[a,b]\circ\underbrace{[b,c]\circ[b,c]}_{=id}\circ[c,d]=[a,b,c]\circ[b,c,d]
			\end{equation}
		\end{itemize}
		\begin{equation}
			\boxed{\text{Donc les 3-cycles engendrent }\mathcal{A}_{n}.}
		\end{equation}

		\item On a 
		\begin{equation}
			\sigma\circ(a_{1},a_{2},a_{3})\circ\sigma^{-1}=(\sigma(a_{1}),\sigma(a_{2}),\sigma(a_{3}))
		\end{equation}
		On peut trouver $\sigma\colon\left\llbracket 1,n\right\rrbracket \to\left\llbracket 1,n\right\rrbracket$ telle que $a_{i}$ soit envoyé sur $b_{i}$ pour $i\in\left\{1,2,3\right\}$ et les éléments $\left\llbracket 1,n\right\rrbracket \setminus\left\{a_{1},a_{2},a_{3}\right\}$ dans $\left\llbracket 1,n\right\rrbracket\setminus\left\{b_{1},b_{2}b_{3}\right\}$.
		\begin{equation}
			\boxed{\text{Donc les 3-cycles sont conjugués dans }\Sigma_{n}.}
		\end{equation}

		Si $n\geqslant5$ et $\sigma$ impair, soit $(c_{1},c_{2})\in\left\llbracket 1,n\right\rrbracket\setminus\left\{a_{1},a_{2},a_{3}\right\}$. $\sigma'=\sigma\circ[c_{1},c_{2}]$ est pair et $\sigma'(a_{i})=b_{i}$. 
		\begin{equation}
			\boxed{\text{Donc les trois cycles sont conjugués dans }\mathcal{A}_{n}\text{ pour }n\geqslant5.}
		\end{equation}
		
		C'est cependant faux pour $n=3$ et $n=4$.

		\item Soit $(\sigma,\sigma')\in\Sigma_{n}^{2}$. En notant $\mathcal{E}$ la signature d'une permutation (morphisme de $(\Sigma_{n},\circ)$ dans $(\left\{-1,1\right\},\times)$), on a
		\begin{equation}
			\mathcal{E}(\sigma\circ\sigma^{-1}\circ\sigma'\circ\sigma'^{-1})=1
		\end{equation}
		donc $\sigma\circ\sigma^{-1}\circ\sigma'\circ\sigma'^{-1}\in\mathcal{A}_{n}$. Donc $D(\Sigma_{n})\subset\mathcal{A}_{n}$.

		Soit ensuite $(a_{1},a_{2},a_{3})$ un 3-cycle. On a $(a_{1},a_{3},a_{2})^{2}=(a_{1},a_{2},a_{3})$ puis\\$(a_{1},a_{3},a_{2})^{-1}=(a_{1},a_{2},a_{3})$. Ainsi, on a 
		\begin{equation}
			\sigma\circ(a_{1},a_{3},a_{2})\circ\sigma^{-1}\circ(a_{1},a_{2},a_{3})=(a_{1},a_{3},a_{2})^{2}=(a_{1},a_{2},a_{3})
		\end{equation}
		On pose $\sigma=[a_{2},a_{3}]$, et alors $(a_{1},a_{2},a_{3})$ est un commutateur. Ainsi, $(a_{1},a_{2},a_{3})\in D(\Sigma_{n})$ et donc $\mathcal{A}_{n}\subset D(\Sigma_{n})$ (d'après la première question).

		Finalement, on a 
		\begin{equation}
			\boxed{D(\Sigma_{n})=\mathcal{A}_{n}}
		\end{equation}
	\end{enumerate}
\end{proof}

\begin{remark}
	Pour $n\geqslant5$, on a $D(\mathcal{A}_{n})=\mathcal{A}_{n}$.
\end{remark}

\begin{proof}
	\phantom{}
	\begin{enumerate}
		\item Pour $g\in G$, $\tau_{g}$ est bijective de réciproque $\tau_{g^{-1}}$. On a notamment $\tau_{g\cdot g'}=\tau_{g}\circ\tau_{g'}$ donc $\tau$ est un morphisme. Si $g\in G$ est tel que $\tau_{g}=id$, pour tout $x\in G$, on a $gx=x$ donc $g=e_{G}$. Donc $\tau$ est un morphisme injectif et 
		\begin{equation}
			\boxed{\text{G est isomorphe à }\im\tau=\tau(G)\text{, sous-groupe de }\Sigma(G)\text{, lui-même isomorphe à }\Sigma_{n}}
		\end{equation}
		
		\item Soit \function{f}{\Sigma_{n}}{GL_{n}(\C)}{\sigma}{(\delta_{i,\sigma(j)})_{1\leqslant i,j\leqslant n}=P_{\sigma}}
		$P_{\sigma}$ est la matrice de permutation associée à $\sigma$. $f$ est un morphisme, et est injectif, donc 
		\begin{equation}
			\boxed{\text{G est isomorphe à un sous-groupe de }GL_{n}(\C).}
		\end{equation}
	\end{enumerate}
\end{proof}

\begin{proof}
	Soit $(x,y,z,t)\in\N^{4}$ tel que $x^{2}+y^{2}+z^{2}=8t+7$. Dans $\Z/8\Z$, on a $\overline{0}^{2}=\overline{0}$, $\overline{1}^{2}=\overline{1}$, $\overline{2}^{2}=\overline{4}$, $\overline{3}^{2}=\overline{1}$, $\overline{4}^{2}=\overline{0}$, $\overline{5}^{2}=\overline{1}$, $\overline{6}^{2}=\overline{4}$ et $\overline{7}^{2}=\overline{1}$. Donc la somme de 3 de ces classes ne donnent pas $\overline{7}$.

	Par récurrence, prouvons la propriété. Soit $(x,y,z,t)\in\N^{4}$ tel que $x^{2}+y^{2}+z^{2}=(8t+7)4^{n+1}$. Parmi $x,y,z$ les trois sont pairs ou deux d'entre eux sont impairs. Si $x,y$ impairs et $z$ pair, on écrit $x=2x'+1,y=2y'+1,z=2z'$, alors $x^{2}+y^{2}+z^{2}\equiv 2[4]$ mais $(8t+7)4^{n+1}\equiv 0[4]$: contradiction. Nécessairement, $x,y$ et $z$ sont pairs. En divisant par $4$, on se ramène donc à l'hypothèse de récurrence.

	\begin{equation}
		\boxed{\text{D'où le résultat par récurrence.}}
	\end{equation}
\end{proof}

\begin{proof}
	On raisonne sur $\Z/7\Z$. On a $\overline{10^{10^{n}}}=\overline{3^{10^{n}}}$. Dans le groupe $((\Z/7\Z)^{*},\times)$, $\overline{3}$ a un ordre qui divise $\vert \Z/7\Z^{*}\vert=6$. On a $\overline{3}^{2}=\overline{2}$, $\overline{3}^{3}=\overline{-1}$ et $\overline{3}^{6}=\overline{1}$. Donc $\overline{3}^{6k}=\overline{1}, \overline{3}^{6k+1}=\overline{3},\overline{3}^{6k+2}=\overline{2},\overline{3}^{6k+3}=\overline{-1}, 3^{6k+4}=\overline{4}$ et $3^{6k+5}=\overline{5}$..

	On se place maintenant dans $\Z/6\Z$: $\overline{10}=\overline{4},\overline{10}^{2}=\overline{4}$ et donc par récurrence sur $n\in\N^{*}$, $\overline{10}^{n}=\overline{4}$. Donc il existe $k\in\Z$ tel que $10^{n}=6k+4$. Ainsi, \begin{equation}
		\boxed{\overline{10^{10^{n}}}=\overline{4}}
	\end{equation}
\end{proof}

\begin{proof}
	\phantom{}
	\begin{enumerate}
		\item On a $F_{1}=5$ et $2+\prod_{k=0}^{0}F_{k}=2+3=5$. Soit $n\geqslant1$, supposons que $F_{n}=2+\prod_{k=0}^{n-1}F_{k}$. Alors 
		\begin{align}
			F_{n+1}-2=2^{2^{n+1}}-1
			&=(2^{2^{n}})^{2}-1\\
			&=(2^{2^{n}}+1)(2^{2^{n}}-1)\\
			&=F_{n}(F_{n}-2)\\
			&=F_{n}\times\prod_{k=0}^{n-1}F_{k}\\
			&=\prod_{k=0}^{n}F_{k}
		\end{align}
		\begin{equation}
			\boxed{\text{d'où le résultat par récurrence.}}
		\end{equation}
		

		\item Soit $p$ un facteur premier de $F_{n}$. S'il existe $k\in\left\llbracket 0,n-1\right\rrbracket$ tel que $p\mid F_{k}$, alors d'après la première question on a $p\mid F_{n}-\prod_{k=0}^{n-1}F_{k}=2$. Donc $p=2$. Or $F_{n}$ est impair, donc non divisible par deux, ce qui est absurde. Donc $p$ ne divise aucun $F_{k}$ pour $k\in\left\llbracket 0,n-1\right\rrbracket$. Les $F_{n}$	 étant distincts deux à deux,
		\begin{equation}
			\boxed{\text{il existe donc une infinité de nombres premiers.}}
		\end{equation}
	\end{enumerate}
\end{proof}

\begin{remark}
	Si $n\neq m$ alors $F_{n}\wedge F_{m}=1$.
\end{remark}

\begin{proof}
	\phantom{}
	\begin{enumerate}
		\item On teste uniquement les puissances qui divisent 32: 2,4,8,16,32. On a $\overline{5}^{2}=\overline{-7},\overline{5}^{4}=\overline{-15},\overline{5}^{8}=\overline{1}$. Donc 
		\begin{equation}
			\boxed{\omega(\overline{5})=8}
		\end{equation}

		\item On note \function{\psi}{\Z/2\Z\times\Z/8\Z}{U}{(\dot{k},\tilde{l})}{\overline{-1}^{k}\times\overline{5}^{l}}
		
		On a $\omega(\overline{-1})=2$ et $\gamma(\overline{5})=8$ donc $\psi$ est bien définie. $\psi$ est bien un morphisme de groupes. Soit $(\dot{k},\tilde{l})\in\ker(\psi)$, on a $\overline{-1}^{k}\times \overline{5}^{l}=\overline{1}$. Si $\dot{k}=\dot{1}$, alors $\overline{-1}^{k}=\overline{-1}=\overline{5}^{-l}=\overline{5}^{l}\in gr\{\overline{5}\}$. Donc $\overline{5}^{2l}=\overline{1}$ et ainsi $8\mid 2l$ d'où $4\mid l$. Mais alors $l\in\left\{0,4\right\}$ ce qui est impossible. Donc $\dot{k}\neq\dot{1}$. De ce fait, $\dot{k}\neq\dot{1}$. Ainsi, $\overline{5}^{l}=\overline{1}$ donc $\tilde{l}=\tilde{0}$. Ainsi, $\ker(\psi)=\left\{(\dot{0},\tilde{0})\right\}$ donc $\psi$ est injective, puis bijective car $\vert\Z/2\Z\times\Z/8\Z\vert=\vert U\vert$. Donc 
		\begin{equation}
			\boxed{U=gr\left\{\overline{-1},\overline{5}\right\}}
		\end{equation}
	\end{enumerate}
\end{proof}

\begin{remark}
	$U$ n'est pas cyclique car, par isomorphisme, ses éléments ont un ordre qui divise 8.
\end{remark}

\begin{proof}
	\phantom{}
	\begin{enumerate}
		\item Soit \function{f}{G_{n}\times G_{m}}{U_{nm}}{(\xi,\xi')}{\xi\times\xi'}
		Soit $(\xi,\xi')\in G_{n}\times G_{m}$, Soit $k\in\Z$ tel que $(\xi\times\xi')^{k}=1$. Alors $(\xi\times\xi')^{km}=1$ d'où $\xi^{km}=1$ donc $n\mid km$ et $n\mid k$ d'après le théorème de Gauss. De même pour $n$, on a $m\mid k$ et donc $nm\mid k$. La réciproque est immédiate: $\xi\times\xi'\in G_{nm}$. Donc $f(G_{n}\times G_{m})\subset G_{nm}$ et $\vert G_{n}\times G_{m}\vert=\varphi(n)\times\varphi(m)=\varphi(nm)=\vert G_{nm}\vert$ où $\varphi$ est l'indicatrice d'Euler.

		Montrons que $f$ est injective: soit $(x,y,x',y')\in G_{n}^{2}\times G_{m}^{2}$ tel que $xx'=yy'$. On a alors $x^{m}=y^{m}$ et $x'^{n}=y'^{n}$ d'où $(xy^{-1})^{m}=1$ d'où $\omega(xy^{-1})\mid m$ et $\omega(xy^{-1})\mid n$. Donc $\omega(xy^{-1})=1$ donc $x=y$ et en reportant, on a $x'=y'$. Donc $f$ est injective puis bijective (égalité des cardinaux).

		On a alors 
		\begin{align}
			\mu(n)\mu(m)
			&=\sum_{\xi\in G_{n}}\xi\times\sum_{\xi'\in G_{m}}\xi'\\
			&=\sum_{(\xi,\xi')\in G_{n}\times G_{m}}\xi\xi'\\
			&=\sum_{\xi\in G_{nm}}\xi\\
			&=\boxed{\mu(nm)}
		\end{align}

		\item On a $\mu(1)=1$. Soit $p$ premier. On a 
		\begin{equation}
			\sum_{k=0}^{p-1}e^{\frac{2\mathrm{i}k\pi}{p}}=0
		\end{equation} 
		donc 
		\begin{equation}
			\mu(p)\sum_{k=1}^{p-1}e^{\frac{2\mathrm{i}k\pi}{p}}=-1	
		\end{equation}
		
		Soit alors $\alpha\in\N$ avec $\alpha\geqslant2$, on a 
		\begin{equation}
			\boxed{
			\mu(p^{\alpha})=\sum_{\substack{k=1\\ k\wedge p=1}}^{p^{\alpha}}e^{\frac{2\mathrm{i}k\pi}{p^{\alpha}}}=\sum_{k=1}^{p^{\alpha}}e^{\frac{2\mathrm{i}k\pi}{p^{\alpha}}}-\sum_{k=1}^{p^{\alpha-1}}e^{\frac{2\mathrm{i}k\pi}{p^{\alpha-1}}}=0}
		\end{equation}

		Si $n=p_{1}^{\alpha_{1}}\dots p_{r}^{\alpha_{r}}$, s'il existe $i\in\left\llbracket 1,r\right\rrbracket$ tel que $\alpha_{i}\geqslant2$ alors $\mu(n)=0$. Sinon, on a 
		\begin{equation}
			\boxed{\mu(n)=\prod_{i=1}^{r}\mu(p_{i})=(-1)^{r}}
		\end{equation}

		\item Soit $(f,g)\in(\C^{\N^{*}})^{2}$, on a 
		\begin{align}
			(f\star g)(n)
			&=\sum_{d_{1}d_{2}=n}f(d_{1})g(d_{2})\\
			&=\sum_{d_{1}d_{2}=n}g(d_{1})f(d_{2})\\
			&=(g\star f)(n)
		\end{align}
		\begin{equation}
			\boxed{\text{Donc }\star\text{ est commutative.}}
		\end{equation}

		Soit $(f,g,h)\in(\C^{\N^{*}})^{3}$, on a 
		\begin{align}
			(f\star (g\star h))(n)
			&=\sum_{d_{1}d=n}f(d_{1})(g\star h)(d)\\
			&=\sum_{d_{1}d=n}\Biggl[f(d_{1})\times \sum_{d_{2}d_{3}=d}g(d_{2})h(d_{3})\Biggr]\\
			&=\sum_{d_{1}d_{2}d_{3}=n}f(g_{1})g(d_{2})h(d_{3})\\
			&=((f\star g)\star h)(n)
		\end{align}
		\begin{equation}
			\boxed{\text{donc }\star\text{ est associative. }}
		\end{equation}

		On vérifie maintenant que l'élément neutre est $e:\N^{*}\to\C$ qui à $1$ associe $1$ et 0 si $n\geqslant2$.
		Soit \function{\psi}{\N}{\Z}{n}{\sum_{d\mid n}\mu(d)}
		On a $\psi(1)=1$. Soit $n\geqslant2$ avec $n=\prod_{i=1}^{r}p_{i}^{\alpha_{r}}$. Les diviseurs de $n$ sont dans $D=\{\prod_{i=1}^{r}p_{i}^{\beta_{i}}\Bigm| \beta_{i}\leqslant\alpha_{i}\}$. Ainsi, $\psi(n)=\sum_{d\in D}\mu(d)$. Or $\mu(d)$ vaut 0 s'il existe $\beta_{i}\geqslant2$ et $(-1)^{k}$ si $k$ $\beta_{i}$ valent 1 et les autres 0. Il y a $\binom{r}{k}$ choix possibles pour que $k$ $\beta_{i}$ valent $1$. Ainsi,
		\begin{equation}
			\psi(n)=\sum_{k=0}^{r}1^{r-k}(-1)^{k}\binom{r}{k}=0
		\end{equation}
		Donc $\mu\star 1=e$, et $\mu^{-1}=1\colon n\mapsto 1$ pour tout $n\in\N$.

		\item On note \function{id}{\N^{*}}{\N^{*}}{n}{n}
		Alors 
		\begin{align}
			\sum_{d\mid n}d\mu(\frac{n}{d})
			&=(\mu\star id)(n)\\
			&=(id\star \mu)(n)\\
			&=(1\star (\varphi\star \mu))(n)\\
			&=\boxed{\varphi(n)}
		\end{align}
		la troisième égalité venant du fait que $id=1\star\varphi$ car $n=\sum_{d\mid n}\varphi(d)$.
	\end{enumerate}
\end{proof}

\begin{proof}
	Pour $k\in\left\llbracket 1,p-1\right\rrbracket$, on a 
	\begin{equation}
		\binom{p+k}{k}=\frac{(p+k)\times\dots\times(p+1)}{k\times\dots\times 1}=1+\alpha kp
	\end{equation}
	car $(p+k)\times\dots\times(p+1)=k!+p\times\text{qqchose}$. On a $p\mid\binom{p}{k}$ donc 
	\begin{equation}
		\sum_{k=1}^{p-1}\binom{p}{k}\binom{p+k}{k}\equiv\sum_{k=1}^{p-1}\binom{p}{k}[p^{2}]
	\end{equation}

	Pour $k=0$, on a $\binom{p}{0}\binom{p}{0}=1$ et pour $k=p$, on a $\binom{p}{p}\binom{2p}{p}=\binom{2p}{p}$. Et 
	\begin{equation}
		\sum_{k=1}^{p-1}\binom{p}{k}=\sum_{k=0}^{p}\binom{p}{k}-2=2^{p}-2
	\end{equation}
	Il reste donc à prouver que $\binom{2p}{p}\equiv 2[p^{2}]$.

	Or 
	\begin{equation}
		\binom{2p}{p}=\sum_{k=0}^{p}\binom{p}{k}\binom{p}{p-k}\equiv2[p^{2}]
	\end{equation}
	la première égalité venant de l'égalité du terme en $X^{p}$ dans $(1+X)^{2p}=(1+X)^{p}(1+X)^{p}$, et la deuxième venant du fait que seuls les termes en $k=0$ et $k=p$ ne contiennent pas de $p^{2}$, et valent chacun 1.

	Finalement, on a 
	\begin{equation}
		\boxed{
		\sum_{k=0}^{p}\binom{p}{k}\binom{p+k}{k}\equiv 2^{p}-2+1+2[p^{2}]\equiv 2^{p}+1[p^{2}]}
	\end{equation}
\end{proof}

\begin{proof}
	\phantom{}
	\begin{enumerate}
		\item Soit $G$ un sous-groupe de $(\U,\times)$. On note $\vert G\vert=d$. On a donc $G\subset\U_{d}$ car pour tout $x\in G$, $x^{d}=1$. 
		\begin{equation}
			\boxed{\text{Donc }G=\U_{d}\text{ est cyclique.}}
		\end{equation}
		
		\item On pose \function{\psi}{SO_{2}(\R)}{(\U,\times)}{R_{\theta}}{e^{\mathrm{i}\theta}}
		qui est un isomorphisme. Donc les sous-groupes de $SO_{2}(\R)$ sont les $G_{n}$ pour $n\geqslant1$ avec 
		\begin{equation}
			\boxed{
			G_{n}=\left\{R_{\frac{2k\pi}{n}}\bigm| k\in\left\llbracket 0,n-1\right\rrbracket\right\}}
		\end{equation}

		\item $\varphi$ est bilinéaire et symétrique. Pour tout $X\in\R^{2}$, on $\varphi(X,X)=\sum_{M\in G}\Vert MX\Vert^{2}\geqslant0$ et si $\varphi(X,X)=0$, on a pour tout $M\in G$, $X=0$. Notamment, $I_{2}\in G$ et donc $X=0$. 
		\begin{equation}
			\boxed{\text{Donc }\varphi\text{ est bien un produit scalaire.}}
		\end{equation}
		
		Pour tout $(M_{0},X,Y)\in G\times(\R^{2})^{2}$, on a $\varphi(M_{0}X,M_{0}Y)=\sum_{M\in G}\langle MM_{0}X,MM_{0}Y\rangle$
		et $M\mapsto MM_{0}$ est bijective de $G$ dans $G$ donc $\varphi(M_{0}X,M_{0}Y)=\varphi(X,Y)$.

		Soit $\mathcal{B}_{0}$ la base canonique de $\R^{2}$ et $\mathcal{B}_{1}$ une base orthonormée pour $\varphi$. On note $P_{0}=\mat\limits_{\mathcal{B}_{0}\to \mathcal{B}_{1}}$. 
		
		Pour tout $M\in G$, $P_{0}^{-1}MP_{0}$ est la matrice d'une isométrie pour $\varphi$ dans une base orthonormée pour $\varphi$. Donc $P_{0}^{-1}MP_{0}$ est orthogonale, et $\det(P_{0}^{-1}MP_{0})=1$ car pour tout $M\in G$, $\det(M)=1$. Ainsi, $\left\{P_{0}^{-1}MP_{0}\bigm| M\in G\right\}$ est un sous-groupe fini de $SO_{2}(\R)$, donc cyclique. Il est isomorphe à $G$ donc 
		\begin{equation}
			\boxed{\text{G est cyclique.}}
		\end{equation}
	\end{enumerate}
\end{proof}

\begin{proof}
	\phantom{}
	\begin{enumerate}
		\item On a $1=1+0\sqrt{2}\in E$. On remarque ensuite que pour tout $s=x+y\sqrt{2}\in E$, on a $ss^{-1}=1$ avec $s^{-1}=x-y\sqrt{1}\in E$. Soit $(s,s')\in E^{2}$ avec $s=x+y\sqrt{2}$ et $s'=x'+y'\sqrt{2}$. Notons déjà que $x+y\sqrt{2}>0$ car $x=\sqrt{1+2y^{2}}>\vert y\vert\sqrt{2}$.
		On a donc
		\begin{equation}
			ss'=\underbrace{xx'+2yy'}_{\in\Z}+\sqrt{2}\underbrace{(yx'+y'x)}_{\in\Z}
		\end{equation}
		On a $xx'\in\N$ et $x>\sqrt{2}\vert y\vert\geqslant0$ et $x'>\sqrt{2}\vert y'\vert\geqslant0$ donc $xx'>2\vert yy'\vert$ et ainsi $xx'+2yy'\in\N^{*}$. Enfin, on a 
		\begin{align}
			(xx'+2yy')^{2}-2(yx'+y'x)^{2}
			&=(xx')^{2}+4(yy')^{2}-2(yx')^{2}2(y'x)^{2}\\
			&=(x^{2}-2y^{2})(x'^{2}-2y'^{2})\\
			&=1
		\end{align}
		Donc $ss'\in E$. Finalement, 
		\begin{equation}
			\boxed{\text{E est un sous-groupe de }(\R_{+}^{*},\times).}
		\end{equation}

		\item $\ln$ est un isomorphisme de $E$ sur $\ln(E)$, sous-groupe de $(\R,+)$. On sait que si 
		\begin{equation}
			\underbrace{\inf(\ln(E)\cap\R_{+})}_{\alpha}>0
		\end{equation}
		alors $\ln(E)=\alpha\Z$ (sous-groupe de $(\R,+)$ dans le cas $\alpha>0$, pour rappel si $\alpha=0$ alors le sous-groupe est dense dans $\R$). On cherche la borne inférieure de $E\cap]1+\infty[$ que l'on note $\beta$. $\beta$ existe car cet ensemble est non vide, par exemple $3+2\sqrt{2}$ y appartient.
		
		Si $\beta=1$, on peut trouver une suite de termes de $E$ strictement décroissante convergeant vers 1. Alors pour tout $n\in\N$, on a 
		\begin{equation}
			1<x_{n+1}+y_{n+1}\sqrt{2}<x_{n}+y_{n}\sqrt{2}
		\end{equation}
		On sait que 
		\begin{equation}
			x_{n}-y_{n}\sqrt{2}=(x_{n}+y_{n}\sqrt{2})^{-1}<1<x_{n}+y_{n}\sqrt{2}
		\end{equation}
		donc $-y_{n}\sqrt{2}<1-x_{n}<0$ donc $y_{n}>0$. Ainsi, 
		\begin{equation}
			y_{n}=\sqrt{\frac{x_{n}^{2}-1}{2}}
		\end{equation}
		Si $x_{n+1}\geqslant x_{n}$, alors $y_{n+1}\geqslant y_{n}$ d'où $x_{n+1}+\sqrt{2}y_{n+1}>x_{n}+\sqrt{2}y_{n}$ ce qui est absurde. Donc $x_{n+1}<x_{n}$ et on obtient une suite strictement décroissante d'entiers naturels ce qui est impossible. Donc $\beta>1$ et 
		\begin{equation}
			\boxed{E=\left\{(x_{0}+y_{0}\sqrt{2})^{n}\bigm| n\in\Z\right\}\text{ est monogène.}}
		\end{equation}

		On peut identifier $\beta$:
		\begin{equation}
			x_{0}=\min\left\{x\in\N^{*}\setminus\left\{1\right\},\exists y\in\Z,x+y\sqrt{2}\in E\cap],+\infty[\right\}
		\end{equation}
		Donc $\beta=3+2\sqrt{2}$ Finalement, $x^{2}-2y^{2}=1$ avec $x\in\N,y\in\N$ si et seulement s'il existe $n\in\N$ tel que $x_{n}+y_{n}\sqrt{2}=\beta^{n}$.
	\end{enumerate}
\end{proof}

\begin{remark}
	En fait, on a 
	\begin{equation}
		\left\{
			\begin{array}[]{lcl}
				x_{n} &= &\sum\limits_{k=0}^{\lfloor \frac{n}{2}\rfloor}\binom{n}{2k}2^{2k}3^{n-2k}\\[0.5cm]
				y_{n} &= &\sum\limits_{k=0}^{\lfloor \frac{n-1}{2}\rfloor}\binom{n}{2k+1}2^{2k+1}3^{n-2k-1}
			\end{array}	
		\right.
	\end{equation}
\end{remark}

\begin{proof}
	On a $7\mid n^{n}-3$ si et seulement si $\overline{n}^{n}=\overline{3}$ dans $\Z/7\Z$. $(\Z/7\Z^{*},\times)$ est un groupe de cardinal 6. Donc l'ordre de ses éléments divisent 6, et sont donc 1,2,3 ou 6. Notamment, on vérifie que $\omega(\overline{3})=6$ et donc le groupe engendré par $\overline{3}$ est exactement $(\Z/7\Z^{*},\times)$. Ainsi, 
	\begin{equation}
		(\Z/7\Z^{*},\times)=\left\{\overline{3}^{k}\bigm| k\in\left\llbracket 0,5\right\rrbracket\right\}
	\end{equation}
	(c'est un groupe cyclique). Les générateurs sont $\left\{\overline{3}^{k},k\wedge 6=1\right\}=\left\{\overline{3},\overline{3}^{5}=\overline{-2}=\overline{5}\right\}$.
	Donc $\overline{n}=\overline{3}$ ou $\overline{n}=\overline{5}$.

	Si $\overline{n}=3$, $\overline{3}^{n}=\overline{3}$ si et seulement si $n\equiv1[6]$ donc $n\equiv 3[7]$ et $n\equiv1[6]$. D'après le théorème des restes chinois, on vérifie que ceci équivaut à $n\equiv31[42]$. La réciproque est immédiate.

	Si $\overline{n}=5$, $\overline{5}^{n}=\overline{3}$ si et seulement si $n\equiv5[6]$ et $n\equiv5[7]$. D'après le théorème des restes chinois, on vérifie que ceci équivaut à $n\equiv5[42]$.

	\begin{equation}
		\boxed{\text{Donc les solutions sont }n\in\N^{*}\text{ tels que }n\equiv31[42]\text{ ou }n\equiv5[42].}
	\end{equation}
\end{proof}

\begin{proof}
	On a 
	\begin{align}
		\sum_{k=1}^{p-1}\frac{1}{k}+\frac{1}{p-k}=\frac{2a}{(p-1)!}
		&\Longleftrightarrow \sum_{k=1}^{p-1}\frac{p}{k(p-k)}=\frac{2a}{(p-1)!}\\
		&\Longleftrightarrow \sum_{k=1}^{p-1}\frac{p(p-1)!}{k(p-k)}=2a\\
		&\Longleftrightarrow p\underbrace{\sum_{k=1}^{p-1}\frac{(p-1)!^{3}}{k(p-k)}}_{\in\N}=2a\underbrace{(p-1)!^{2}}_{p\wedge (p-1)!^{2}=1}
	\end{align}
	donc $p\mid a$ d'après le théorème de Gauss.

	On écrit alors $a=p\times b$ avec $b\in\N$. On a alors
	\begin{equation}
		\sum_{k=1}^{p-1}\frac{1}{k(p-k)}=\frac{2b}{(p-1)!}
	\end{equation}
	comme $(p-1)!,k$ et $p-k$ ($1\leqslant k\leqslant p$) sont inversibles dans $\Z/p\Z$, on a alors
	\begin{equation}
		\sum_{k=1}^{p-1}\overline{-k}^{-2}=\overline{2b}\times\underbrace{\overline{(p-1)!}^{-1}}_{=\overline{-1}}
	\end{equation}
	d'après le théorème de Wilson.

	Donc 
	\begin{equation}
		\overline{2b}=\sum_{k=1}^{p-1}\overline{k}^{-2}
	\end{equation}
	Comme \function{f}{\Z/p\Z^{*}}{\Z/p\Z^{*}}{\overline{k}}{\overline{k}^{-1}}
	est bijective, on a 
	\begin{equation}
		\overline{2}\times\overline{b}=\sum_{k=1}^{p-1}\overline{k}^{2}=\overline{\frac{p(p-1)(2p-1)}{6}}
	\end{equation}
	Or $p\geqslant5$ est premier, donc $p-1$ est pair et $p$ est congru ) 1 ou 2 modulo 3. Donc $p-1\equiv0[3]$ ou $2p-1\equiv0[3]$ donc $\frac{(p-1)(2p-1)}{6}\in\N$. Ainsi, 
	\begin{equation}
		\overline{2}\times\overline{b}=\sum_{k=1}^{p-1}\overline{k}^{2}=\overline{p}\times\overline{\frac{(p-1)(2p-1)}{6}}=0
	\end{equation}
	et donc $p\mid b$ par le théorème de Gauss. Donc 
	\begin{equation}
		\boxed{p^{2}\mid a}
	\end{equation}
\end{proof}

\begin{proof}
	Les racines réelles de $P$ ont une multiplicité paire, le coefficient dominant est positif (car la limite en $+\infty$ est positive) et les racines complexes non réelles sont 2 à 2 conjuguées:
	\begin{equation}
		(X-\alpha)(X-\overline{\alpha})=X^{2}-2\Re(\alpha)X+\vert\alpha\vert^{2}=(X-\Re(\alpha))^{2}+\vert\Im(\alpha)\vert^{2}
	\end{equation}
	avec $\Im(\alpha)\neq0$.
	\begin{equation}
		\boxed{\text{D'où le résultat en décomposant P sur }\C[X].}
	\end{equation}
\end{proof}

\begin{proof}
	\phantom{}
	\begin{enumerate}
		\item $G=\Z+\alpha\Z$ est un sous-groupe de $\R$ engendré par $\alpha$ et $1$. S'il existait $a\in\R_{+}^{*}$ tel que $G=a\Z$, alors il existait $(n,m)\in(\Z^{*})^{2}$ tel que $1=na$ et $\alpha=ma$, d'où $\alpha=\frac{m}{n}\in\Q$ ce qui est absurde. Donc $G$ est dense dans $\R$. 
		\begin{equation}
			\boxed{\text{Le fait que }\Z+\alpha\N\text{ est dense dans }\R\text{ est alors immédiate.}}
		\end{equation}

		\item Posons $\beta=\frac{\alpha}{2\pi}\notin\Q$. Alors $\Z+\beta\N$ est dense dans $\R$. Soit $c<d\in\R^{2}$. Comme $\frac{c}{2\pi}<\frac{d}{2\pi}$, il existe $x\in\Z+\beta\N\cap]\frac{c}{2\pi},\frac{d}{2\pi}[$ et alors $2\pi x\in2\pi\Z+\alpha\N\cap]c,d[$. On pose $c=\arcsin(a)$ et $d=\arcsin(b)$ avec $a<b$. On a bien $c<d$ car $\arcsin$ est strictement croissante.
		
		Alors il existe $(m,n)\in\Z\times\N$ tel que $2\pi m+\alpha m=2\pi x\in]c,d[$ donc $\sin(2\pi x)=\sin(2\pi m+\alpha n)=\sin(\alpha n)\in]a,b[$.

		\begin{equation}
			\boxed{\text{Donc }(\sin(n\alpha))_{n\in\N}\text{ est dense dans }]-1,1[.}
		\end{equation}
		En particulier, cela vaut pour $\alpha=1$ car $\pi\notin\Q$. Donc $(\sin(n))_{n\in\N}$ est dense dans $[-1,1]$.

		\item Soit $n\in\N$. $2^{n}$ commence par 7 en base 10 si et seulement s'il existe $p\in\N$ avec 
		\begin{align}
			7\times10^{p}\leqslant2^{n}<8\times10^{p}
			&\Longleftrightarrow \ln(7)+p\ln(10)\leqslant n\ln(2)<\ln(8)+p\ln(10)\\
			&\Longleftrightarrow \frac{\ln(7)}{\ln(10)}\leqslant\frac{n\ln(2)}{\ln(10)}-p<\frac{\ln(8)}{\ln(10)}
		\end{align}
		On a alors 
		\begin{equation}
			p=\left\lfloor\frac{n\ln(2)}{\ln(10)}\right\rfloor\in\N
		\end{equation}

		On étudie donc $\N\frac{\ln(2)}{\ln(10)}+\Z$. Supposons que $\frac{\ln(2)}{\ln(10)}=\frac{p}{q}\in\Q$. Alors on a $2^{q}=10^{p}$ mais comme $p\neq0$, on a $5\mid 10^{p}$ mais $5\nmid 2^{q}$, donc $\frac{\ln(10)}{\ln(2)}\notin\Q$.

		On sait que 
		\begin{equation}
			u_{n}=n\frac{\ln(2)}{\ln(10)}-\left\lfloor\frac{n\ln(2)}{\ln(10)}\right\rfloor\in\left]\frac{\ln(7)}{\ln(10)},\frac{\ln(8)}{\ln(10)}\right[
		\end{equation}
		Par densité, on peut donc construire par récurrence $(u_{n_{p}})_{p\in\N}$ telle que 
		\begin{equation}
			\frac{\ln(7)}{\ln(10)}<u_{n_{p+1}}<u_{n_{p}}<\frac{\ln(8)}{\ln(10)}
		\end{equation}
		\begin{equation}
			\boxed{\text{Donc on a bien une infinité de puissance de 2 commençant par 7 en base 10.}}
		\end{equation}
	\end{enumerate}
\end{proof}

\begin{remark}
	$(e^{\mathrm{i}n\alpha})_{n\in\N}$ est de la même façon dense dans $\U$. On peut montrer qu'elle est équirépartie, c'est à dire que pour tout $a<b\in[0,2\pi[^{2}$, on a 
	\begin{equation}
		\lim\limits_{N\to+\infty}\left\vert\left\{n\in\left\llbracket 1,N\right\rrbracket\middle| n\alpha-\frac{\left\lfloor 2\pi n\alpha\right\rfloor}{2\pi}\in]a,b[\right\}\right\vert\times\frac{1}{N}=\frac{b-a}{2\pi}
	\end{equation}
\end{remark}

\begin{remark}
	Par équirépartition dans $[0,1[$ des 
	\begin{equation}
		\left\{n\frac{\ln(2)}{\ln(10)}-\left\lfloor\frac{n\ln(2)}{\ln(10)}\right\rfloor\bigm| n\in\N\right\}
	\end{equation}
	la probabilité pour qu'une puissance de $2$ commence par $k$ en base $10$ est ($k\in\left\llbracket 1,9\right\rrbracket$)
	\begin{equation}
		\frac{\ln(k+1)-\ln(k)}{\ln(10)}=\frac{\ln(1+\frac{1}{k})}{\ln(10)}
	\end{equation}
\end{remark}

\begin{proof}
	\phantom{}
	\begin{enumerate}
		\item Pour $\alpha=a+\i b$, on définit le module au carré: $\vert\alpha\vert^{2}=a^{2}+b^{2}$. Soit $\beta=c+\i d\neq0$. Si $\alpha=\beta q+r$ avec $q,r\in\Z[\i]^{2}$ et $\vert r\vert^{2}<\vert \beta\vert^{2}$, alors $\vert\alpha-\beta q\vert^{2}<\vert\beta\vert^{2}$ et $\beta\neq0$ donc
		\begin{equation}
			\left\vert\underbrace{\frac{\alpha}{\beta}}_{\in\C}-\underbrace{q}_{\in\Z[\i]}\right\vert< 1
		\end{equation}
		On pose $\frac{\alpha}{\beta}=x+\i y$. On pose 
		\begin{equation}
			u_{x}=
			\left\{
				\begin{array}[]{ll}
					\lfloor x\rfloor & \text{si }x\in[\lfloor x\rfloor,\lfloor x\rfloor+\frac{1}{2}[\\
					\lfloor x\rfloor+1 & \text{si }x\in[\lfloor x\rfloor+\frac{1}{2},\lfloor x\rfloor+1[
				\end{array}
			\right.
		\end{equation}

		et de même pour $u_{y}$. On a alors $q=u_{x}+\i u_{y}\in\Z[\i]$ et 
		\begin{equation}
			\left\vert\frac{\alpha}{\beta}-q\right\vert^{2}=\vert x-u_{x}\vert^{2}+\vert y-u_{y}\vert^{2}\leqslant2\times \left(\frac{1}{2}\right)^{2}=\frac{1}{2}<1
		\end{equation}
		On pose donc $r=\alpha-\beta q\in\Z[i]$ et ainsi 
		\begin{equation}
			\boxed{\text{l'anneau }\Z[\i]\text{ est euclidien.}}
		\end{equation}

		\item Soit $A$ un anneau euclidien et $I$ un idéal de $A$ non réduit à $\{0\}$. Il existe $x\in I$ tel que 
		\begin{equation}
			v(x_{0})=\min\{v(x)\bigm| x\in I\{0\}\}
		\end{equation}
		On a $x_{0}A\subset I$. Soit $x\in I$. Il existe $q,r\in A$ tel que 
		\begin{equation}
			x=x_{0}q+r
		\end{equation}
		avec $v(r)<v(x_{0})$ ou $r=0$. Or $r\in I$ donc $r=0$. Ainsi $x\in x_{0}A$ et donc $I=x_{0}A$. 
		\begin{equation}
			\boxed{\text{Donc tout anneau euclidien est principal.}}
		\end{equation}
	\end{enumerate}
\end{proof}

\begin{remark}
	C'est encore vrai avec $\Z[\i\sqrt{2}]=\{a+\i b\sqrt{2}\bigm|(a,b)\in\Z^{2}\}$.
\end{remark}

\begin{proof}
	\phantom{}
	\begin{enumerate}
		\item Si $\overline{x}=\overline{y}^{2}$ est un carré, d'après le petit théorème de Fermat, on a $\overline{x}^{\frac{p-1}{2}}=\overline{y}^{p-1}=\overline{1}$. Soit \function{f}{\Z/p\Z^{*}}{\Z/p\Z^{*}}{\overline{y}}{\overline{y}^{2}}
		$f$ est un morphisme multiplicatif, $\im(f)$ est un sous-groupe de $\left(\Z/p\Z^{*},\times\right)$.

		Comme $\mathbb{F}_{p}$ est un corps, chaque carré possède exactement deux antécédents. Il y a $p-1$ antécédents, donc il y a $\frac{p-1}{2}$ carrés dans $\Z/p\Z^{*}$. Donc $\vert\im(f)\vert=\frac{p-1}{2}$ et si $\overline{x}$ est un carré, x est racine de $X^{\frac{p-1}{2}}-\overline{1}$. Le polynôme $X^{\frac{p-1}{2}}-\overline{1}$ possède au plus $\frac{p-1}{2}$ racines et tout carré est racine. Donc les racines sont exactement les carrés et 
		\begin{equation}
			\boxed{\overline{x}^{\frac{p-1}{2}}=\overline{1}\text{ si et seulement si }\overline{x}\text{ est un carré.}}
		\end{equation}

		\item On a $p\equiv1[4]$ si et seulement si $\frac{p-1}{2}$ est pair si  et seulement si $(\overline{-1})^{\frac{p-1}{2}}=\overline{1}$ si et seulement si $\overline{-1}$ est un carré dans $\mathbb{F}_{p}$.
		Supposons qu'il y ait un nombre fini de nombres premiers $p_{1},\dots,p_{r}$ tous congrus à 1 modulo 4. 
		On pose $n=(p_{1}\times\dots\times p_{r})^{2}+1$.
		Soit $p$ un facteur premier de $n$, on a $n\equiv 1[n_{i}]$ donc $p\notin\{p_{1},\dots,p_{r}\}$.
		Dans $\Z/p\Z$, on a $\overline{n}=\overline{0}$ donc $\overline{-1}=\overline{p_{1}\times\dots\times p_{r}}^{2}$ donc $p\equiv1[4]$ ce qui est une contradiction.

		\begin{equation}
			\boxed{\text{Donc il y a une infinité de nombres premiers congrus à 1 modulo 4.}}
		\end{equation}
	\end{enumerate}
\end{proof}

\begin{proof}
	\phantom{}
	\begin{enumerate}
		\item On pose $P_{1}=\sum_{i=0}^{n}r'_{i}X^{i}$, et $\nu_{p}(r'_{i})$ est positif par définition de $c(P)$. Donc
		\begin{equation}
			\boxed{P_{1}\in\Z[X]}
		\end{equation}
		
		Pour tout $p\in\mathcal{P}$, il existe $i_{0}\in\left\llbracket 1,n\right\rrbracket$ tel que 
		\begin{equation}
			\min\limits_{i\in\left\llbracket 1,n\right\rrbracket}\nu_{p}(r_{i})=\nu_{p}(r_{i_{0}})
		\end{equation}
		et $\nu_{p}(r'_{i_{0}})=0$ donc $p\nmid r'_{i_{0}}$ donc 
		\begin{equation}
			\bigwedge_{i=1}^{n}r'_{i}=1
		\end{equation}

		Si on a $P=\alpha_{1}P_{1}=\alpha_{2}P_{2}$ avec les conditions requises, soit $p\in\mathcal{P}$, si $\nu_{p}(\alpha_{2})>\nu_{p}(\alpha_{1})$, alors $p$ divise tous les coefficients de $P_{1}$ ce qui n'est pas possible, donc $\nu_{p}(\alpha_{2})=\nu_{p}(\alpha_{1})$. Ceci étant vrai pour tout $p\in\mathcal{P}$, on a aussi $\alpha_{1}=\alpha_{2}$ et donc $P_{1}=P_{2}$. 
		\begin{equation}
			\boxed{\text{Donc l'écriture est unique.}}
		\end{equation}

		\item On a $P=c(P)P_{1}$ et $Q=c(Q)Q_{1}$ donc $PQ=c(P)c(Q)P_{1}Q_{1}$ et $P_{1}Q_{1}\in\Z[X]$.
		
		Soit $p\in\mathcal{P}$ divisant tous les coefficients de $P_{1}Q_{1}$. On définit, si $R=\sum_{i\in\N}\gamma_{i}X^{i}\in\Z[X]$, $\overline{R}=\sum_{i\in\N}\overline{\gamma_{i}}X^{i}\in\Z/p\Z[X]$. $R\mapsto\overline{R}$ est un morphisme d'anneaux. Par hypothèse, on a $\overline{P_{1}Q_{1}}=\overline{0}=\overline{P_{1}}\overline{Q_{1}}$ et par intégrité de $\Z/p\Z[X]$, on a $\overline{P_{1}}=\overline{0}$ ou bien $\overline{Q_{1}}=\overline{0}$, ce qui est exclu par les hypothèses. Donc
		\begin{equation}
			\boxed{\text{c(PQ)=c(P)c(Q)}}
		\end{equation}

		\item 
		Soit alors $P$ irréductible dans $\Z[X]$ (les inversibles de $\Z[X]$ étant -1 et 1). Posons 
		\begin{align}
			P
			&=QR\in\Q[X]^{2}\\
			&=c(Q)c(R)\underbrace{Q_{1}R_{1}}_{\in\Z[X]}
		\end{align}
		Or $c(Q)c(R)=c(P)$ d'après le lemme de Gauss et nécessairement, $c(P)=1$. Donc $P=Q_{1}R_{1}$, et alors $Q_{1}=\pm 1$ et $R_{1}=\pm 1$, et $Q$ ou $R$ est constant, 
		\begin{equation}
			\boxed{\text{donc P est irréductible sur }\Q[X].}
		\end{equation}

		Pour la réciproque, on a $2X$ est irréductible sur $\Q[X]$ car de degré 1, mais pas sur $\Z[X]$ car ni 2 ni $X$ ne sont inversibles.

		\item Soit $\theta=\frac{2\pi p}{q}$ avec $p\wedge q=1$ et $\cos(\theta)\in\Q$. Sur $\C[X]$, on a $P=(X-e^{\i\theta})(X-e^{-\i\theta})=X^{2}-2\cos(\theta)X+1\in\Q[X]$.
		
		Et $e^{\i\theta}\neq e^{-\i\theta}$ car $\theta\not\equiv0[\pi]$. On a $\theta=\frac{2\pi p}{q}$ donc $e^{\i\theta}\in \U_{q}$, et $e^{\i\theta}$ et $e^{-\i\theta}$ sont des racines de $A$. Donc, dans $\C[X]$, on a $P\mid A$ et $A\in\Q[X]$, donc il existe $B\in\Q[X]$ tel que 
		\begin{equation}
			\underbrace{A}_{\in\Q[X]}=\underbrace{B}_{\in\C[X]}\times\underbrace{P}_{\in\Q[X]}
		\end{equation}
		Or $B$ s'obtient par la division euclidienne de $A$ par $P$, qui est indépendante du corps de référence, il vient $B\in\Q[X]$ et donc $A\mid P$ dans $\Q[X]$.

		On a $c(A)=1=c(B)c(P)$ et $A=c(B)c(P)B_{1}P_{1}=B_{1}P_{1}\in\Z[X]$ et le coefficient dominant de $A$ est donc 1. Donc le coefficient dominant de $B_{1}$ et de $P_{1}$ est aussi 1. En reportant, on a $P=P_{1}\in\Z[X]$.

		Donc $2\cos(\theta)\in\Z\cap[-2,2]$ donc $\cos\left\{\theta\right\}\in\left\{-\frac{1}{2},\frac{1}{2},0\right\}$ ($-1$ et $1$ ne peuvent y être car on a supposé $\theta\not\equiv0[\pi]$). Les solutions sont donc 
		\begin{equation}
			\boxed{
			\theta\in\left\{0,\frac{\pi}{3},\frac{\pi}{2},\frac{2\pi}{3},\pi,\frac{4\pi}{3},\frac{3\pi}{2},\frac{5\pi}{3}\right\}}
		\end{equation}
		(en rajoutant $\theta=0$ et $\pi$).
	\end{enumerate}
\end{proof}

\begin{remark}
	On a $\frac{\arccos(\frac{1}{3})}{\pi}\notin Q$ car $\cos(\theta)=\frac{1}{3}$ n'est pas dans l'ensemble solutions.
\end{remark}

\begin{proof}
	\phantom{}
	\begin{enumerate}
		\item Soit $P=a\prod_{i=1}^{s}(X-a_{i})^{\alpha_{i}}$ avec les $a_{i}$ distincts et $\alpha_{i}\geqslant1$. $a_{i}$ est racine de $P'$ de multiplicité $\alpha_{i}-1$. Il manque donc $s$ racines. Si $\alpha=0$, le résultat est évident, sinon on pose \function{f}{\R}{\R}{x}{P(x)e^{\frac{x}{\alpha}}}
		et on a pour tout $x\in\R$,
		\begin{equation}
			f'(x)=\frac{e^{\frac{x}{\alpha}}}{\alpha}(P(x)+\alpha P'(x))
		\end{equation}
		Comme $P$ est scindé sur $\R$, $P'$ est scindé sur $\R$ (appliquer le théorème de Rolle entre les racines distinctes de $P$), donc $f'$ s'annule $s-1$ fois entre les racines de $P$ donc 
		\begin{equation}
			\boxed{P+\alpha P'\text{ aussi.}}
		\end{equation}

		La dernière racine est réelle car sinon, le conjugué de la racine complexe supposée serait aussi racine.

		\item On pose $R=\mu\prod_{i=0}^{r}(X-\beta_{i})$. On pose \function{\Delta}{\R[X]}{\R[X]}{P}{P'}
		On a alors 
		\begin{equation}
			\sum_{i=0}^{r}a_{i}P^{(i)}=\sum_{i=0}^{r}a_{i}\Delta^{i}(P)=R(\Delta)(P)=\mu\prod_{i=0}^{r}(\Delta-\beta_{i}id)(P)
		\end{equation}
		Par récurrence sur $r$, on montre que 
		\begin{equation}
			\boxed{\prod_{i=0}^{r}(\Delta-\beta_{i}id)(P)\text{ est scindé}}
		\end{equation} 
		d'après la première question.
	\end{enumerate}
\end{proof}

\begin{remark}
	On a aussi pour tout $\lambda\in\R$, $P'+\lambda P$ est aussi scindé sur $\R$ si $P$ est scindé sur $\R$.
\end{remark}

\begin{proof}
	Soit $F=\frac{P'}{P}$ définie sur $\R\setminus\{a_{1},\dots,a_{n}\}$ où $a_{i}$ sont les racines de $P$. On note $\alpha$ le coefficient dominant de $P$, et on a 
	\begin{equation}
		P'=\alpha\sum_{i=1}^{n}\left(\prod_{\substack{j=1\\ j\neq i}}^{n}(X-a_{j})\right)
	\end{equation}
	On a donc $F=\sum_{i=1}^{n}\frac{1}{X-a_{i}}$ et on a 
	\begin{equation}
		F'=-\sum_{i=1}^{n}\frac{1}{(X-a_{i})^{2}}=\frac{P''P-P'P'}{P^{2}}
	\end{equation}

	Pour $x\notin\{a_{1},\dots,a_{n}\}$, on a 
	\begin{align}
		(n-1)(P'^{2}(x))(x)\geqslant nP(x)P''(x)
		&\Longleftrightarrow n(P''(x)P(x)-P'^{2}(x))\leqslant-P'^{2}(x)\\
		&\Longleftrightarrow \frac{P'^{2}(x)}{P^{2}(x)}\leqslant n(P''(x)P(x)-P'^{2}(x))\times\frac{1}{P^{2}(x)}\\
		&\Longleftrightarrow F^{2}(x)\leqslant n(-F'(x))\\
		&\Longleftrightarrow\left(\sum_{i=1}^{n}\frac{1}{(X-a_{i})}\right)^{2}\leqslant \boxed{n\times\sum_{i=1}^{n}\frac{1}{(X-a_{i})^{2}}}
	\end{align}
	qui est l'inégalité de Cauchy-Schwarz dans $\R^{2}$ avec $(1\dots 1)$ et $(\frac{1}{x-a_{1}}\dots\frac{1}{x-a_{n}})$.
\end{proof}

\begin{remark}
	Si $P=\alpha(X-a_{1})^{m_{1}}(X-a_{r})^{m_{r}}$, alors 
	\begin{equation}
		\frac{P'}{P}=\sum_{i=1}^{r}\frac{m_{i}}{X-a_{i}}
	\end{equation}
\end{remark}

\begin{proof}
	\phantom{}
	\begin{enumerate}
		\item $P'\in\C[X]$ et $\deg(P')=\deg(P)-1$. On a $P\wedge P'=1$ car $P$ est irréductible sur $\Q[X]$. Comme le pgcd est obtenu par l'algorithme d'Euclide qui est indépendant du corps de référence, on a $P\wedge P'=1$ sur $\C[X]$ donc 
		\begin{equation}
			\boxed{P\text{ n'a que des racines simples sur }\C.}
		\end{equation}
		\item Notons $P\in\Q[X]$ le polynôme minimal de $\alpha$ sur $\Q$ (défini car $A(\alpha)=0$ donc $\alpha$ est algébrique). Comme $A(\alpha)=0$, on a $P\mid A$ et $P$ est irréductible sur $\Q[X]$. Si $\alpha\notin\Q$, on a $\deg(P)\geqslant2$, on peut donc décomposer sur $\Q[X]$:
		\begin{equation}
			A=P^{r}\times P_{1}^{r_{1}}\times\dots P_{s}^{r_{s}}
		\end{equation}
		avec les $P_{i}$ irréductibles sur $\Q[X]$ non associés.

		$\alpha$ n'est pas racine d'un $P_{i}$ car sinon $P\mid P_{i}$ ce qui est impossible. $\alpha$ est racine simple de $P$ donc $m(\alpha)=r>\frac{\deg(A)}{2}$. Par ailleurs, $\deg(P)^{r}\geqslant2r>\deg(A)$ ce qui est impossible.

		Donc 
		\begin{equation}
			\alpha\in\Q
		\end{equation}
	\end{enumerate}
\end{proof}

\begin{proof}
	Soit $x\in A$. Il existe $(n,m)\in\N^{2}$ avec $n<m$ tel que $x^{n}=x^{m}$. Alors $x^{m-n}=e_{G}\in A$.
	\function{f}{\N^{*}}{A}{n}{x^{n}}
	n'est pas injective, car $\N^{*}$ est infini et $A$ est fini. Or $m-n\in\N^{*}$ donc
	\begin{equation}
		x^{m-n}=e_{G}\Rightarrow x=x\cdot x^{m-n-1}=e_{G}
	\end{equation}
	donc $x^{-1}=x^{m-n-1}\in A$ et ainsi 
	\begin{equation}
		\boxed{\text{A est un sous-groupe.}}
	\end{equation}
\end{proof}

\begin{proof}
	Pour $\alpha=0$, on a $1+p\equiv 1+p[p^{2}]$. Pour $\alpha=1$, on a 
	\begin{equation}
		(1+p)^{p}=\sum_{k=0}^{p}\binom{p}{k}p^{k}=1+p^{2}+\binom{p}{2}p^{2}\sum_{k=3}^{p}\binom{p}{k}p^{k}
	\end{equation}
	Or $\binom{p}{2}p^{2}=\frac{p(p-1)p^{2}}{2}\equiv0[p^{3}]$ car $p$ est premier plus grand que trois donc impair, et la somme est aussi congru à 0 modulo $p^{3}$.

	Soit $\alpha\geqslant1$, supposons que l'on ait 
	\begin{equation}
		(1+p)^{p}\equiv 1+p^{\alpha+1}[p^{\alpha+2}]
	\end{equation}
	Il existe $l\in\N$ tel que 
	\begin{equation}
		(1+p)^{p^{\alpha}}=1+p^{\alpha+1}+lp^{\alpha+2}
	\end{equation}
	Alors 
	\begin{equation}
		(1+p)^{p^{\alpha+1}}=(1+\underbrace{p^{\alpha+1}+lp^{\alpha+2}}_{x})^{p}
	\end{equation}
	Or
	\begin{equation}
		(1+x)^{p}=\sum_{k=0}^{p}\binom{p}{k}x^{k}=1+px+\sum_{k=2}^{p}\binom{p}{k}x^{k}=1+p^{\alpha+2}+lp^{\alpha+3}+\underbrace{\sum_{k=2}^{p}\binom{p}{k}x^{k}}_{\text{divisible par }x^{2}}
	\end{equation}
	Comme $p^{\alpha+1}\mid x$, $p^{2\alpha+2}\mid x^{2}$ avec $2\alpha+2\geqslant\alpha+3$ ($\alpha\geqslant1)$. D'où 
	\begin{equation}
		p^{\alpha+3}\Bigm| x^{2}\Bigm|\sum_{k=2}^{p}\binom{p}{k}x^{k}
	\end{equation}
	et donc
	\begin{equation}
		\boxed{(1+p)^{p^{\alpha+1}}\equiv1+p^{\alpha+2}[p^{\alpha+3}]}
	\end{equation}
\end{proof}

\begin{remark}
	Pour $p=2,\alpha=1$, on a $3^{2}=9\not\equiv 5[8]$.
\end{remark}

\begin{proof}
	Si $7=2x^{2}-5y^{2}$, on a $\overline{0}=2\overline{x}^{2}-5\overline{y}^{2}=\overline{2}(\overline{x}^{2}+\overline{y}^{2})$ dans $\Z/7\Z$. Comme 2 et 7 sont premiers entre eux donc $\overline{2}$ est inversible. Donc $\overline{x}^{2}+\overline{y}^{2}=\overline{0}$. La seule possibilité est $\overline{x}=\overline{0}$ et $\overline{y}=\overline{0}$. Donc $7\mid x$ et $y\mid y$. Si $x=7k$ alors $x^{2}=49k^{2}$ donc $49\mid x^{2}$ et $49\mid y^{2}$ donc $47\mid 2x^{2}-5y^{2}=7$ ce qui est faux. 

	Ainsi, pour tout $(x,y)\in\Z^{2}$,
	\begin{equation}
		\boxed{7\neq2x^{2}-5y^{2}}
	\end{equation}
\end{proof}

\begin{proof}
	$\mathbb{F}_{19}$ est un corps car 19 est premier. On a donc $\overline{x}^{3}=\overline{1}$ si et seulement si $(x-\overline{1})(x^{2}+x-\overline{1})=\overline{0}$. On a donc $x=\overline{1}$ ou $x^{2}+x+\overline{1}=\overline{0}$.
	On a 
	\begin{equation}
		x^{2}+x+\overline{1}=(x+\overline{2}^{-1})^{2}+\overline{3}\times\overline{4}^{-1}=(x+\overline{10})^{2}+\overline{3}\times\overline{5}\overline{0}
	\end{equation}
	Donc $(x+\overline{10})^{2}=\overline{4}$ d'où 
	\begin{equation}
		\boxed{x=\overline{-8}=\overline{11}\text{ ou }x=\overline{-12}=\overline{7}.}
	\end{equation}
\end{proof}

\begin{proof}
	\phantom{}
	\begin{enumerate}
		\item $m$ est inversible si et seulement si $m\wedge 2^{n}=1$ si et seulement si $m\wedge 2=1$ si et seulement si $m$ est impair. 
		\begin{equation}
			\boxed{\text{Il y a donc }2^{n-1}\text{ inversibles.}}
		\end{equation}
		\item On a $5^{2^{3-3}}=5\equiv1+2^{2}[2^{3}]$. Par récurrence, soit $n\geqslant3$. Il existe $k\in\Z$ avec $5^{2^{n-3}}=1+2^{n-1}+k2^{n}$ donc 
		\begin{equation}
			\boxed{
			5^{2^{n-1}}=1+2^{n}+k2^{n+1}+2^{2n-2}(1+2k)^{2}\equiv 1+2^{n}[2^{n+1}]}
		\end{equation}
		car $2n-2\geqslant n+1$ ($n\geqslant3)$.

		\item On a $5^{2^{n-2}}\equiv 1+2^{n}[2^{n+1}]\equiv 1[2^{n}]$ et $5^{2^{n-3}}\not\equiv1[2^{n}]$. 
		\begin{equation}
			\boxed{\text{Donc l'ordre de }\overline{5}\text{ est }2^{n-2}.}
		\end{equation}
		
		\item $gr\left\{\overline{-1}\right\}=\left\{\overline{-1},\overline{1}\right\}$. $\overline{5}$ n'engendre pas $\overline{-1}$ car si $\overline{5}^{k}=\overline{-1}$, on a $\overline{5}^{2k}=\overline{1}$ d'où $2^{n-2}\mid 2k$ donc $2^{n-3}\mid k$. Ainsi, $k\in\left\{2^{n-3},2^{n-2},2^{n-1}\right\}$. Mais $\overline{5}^{2^{n-2}}=\overline{1},\overline{5}^{2^{n-3}}=\overline{1+2^{n-1}}\neq\overline{-1}$ donc un tel $k$ n'existe pas.
		
		Posons \function{\varphi}{\left(\Z/2\Z\times\Z/2^{n-2}\Z, +\right)}{\left(\Z/2^{n}\Z^{\times},\times\right)}{(\widetilde{a},\dot{b})}{\overline{-1}^{a}\overline{5}^{b}}
		Elle est bien définie car $\omega(\overline{-1})=2$ et $\omega(\overline{5})=2^{n-2}$. C'est évidemment un morphisme, on a égalité des cardinaux des ensembles de départ et d'arrivée, et on vérifie qu'elle est injective, et donc 
		\begin{equation}
			\boxed{\text{c'est un isomorphisme.}}
		\end{equation}
	\end{enumerate}
\end{proof}

\begin{proof}
	Soit $(x,x')\in G^{2}$ tel que $x\cdot x'=e$. Alors 
	\begin{equation}
		e\cdot x=x\cdot x'\cdot x =x\cdot e\cdot x'\cdot x
	\end{equation}
	si et seulement si 
	\begin{equation}
		e\cdot x\cdot x'=e=x\cdot e\cdot x'\cdot x\cdot x'=x\cdot e \cdot	x'
	\end{equation}

	Soit $(x,x',x'')\in G^{3}$ tel que $x\cdot x'=e$ et $x'\cdot x''=e$. On a alors 
	\begin{equation}
		x\cdot x'\cdot x''=x\cdot e = x = e\cdot x''
	\end{equation}
	Donc $x=e\cdot x''$ et $e=e\cdot x''\cdot x'$. Si on prouve que $e\cdot x''=x''$, alors $x=x''$ et $x'\cdot x=e$.

	Montrons donc que pour tout $x\in G$, $e\cdot x=x$. Notons que s'il existe $e'\in G$ tel que pour tou t$x\in G$, $e'\cdot x=x$, alors $e'\cdot e=e'=e$.
	Il vient donc 
	\begin{equation}
		x'\cdot x=x'\cdot e\cdot x''=x'\cdot x''=e
	\end{equation}
	Donc pour tout $x\in G$, l'élément $x'$ est inverse à droite et à gauche: $x\cdot x'=e$.

	Donc 
	\begin{equation}
		x\cdot x'\cdot x=e\cdot x =x\cdot x'\cdot x=x\cdot e=x
	\end{equation}
	Et donc $e$ est neutre à gauche. Finalement, 
	\begin{equation}
		\boxed{(G,\cdot)\text{ est un groupe.}}
	\end{equation}
\end{proof}

\begin{remark}
	Si $f\colon\R\to\R$ est surjective, on peut définir \function{g}{\R}{\R}{y}{f(x)} pour un certain $x\in\R$. On a $f\circ g=id$. Si $f$ n'est pas injective: s'il existait $h\colon\R\to\R$ telle que $h\circ f=id$, soit $(x,x')\in \R^{2}$ telle que $f(x)=f(x')$. En composant par $h$, on aurait $x=x'$ donc $f$ serait injective ce qui n'est pas. 

	On peut donc avoir un inverse à droite mais pas à gauche.
\end{remark}

\begin{proof}
	Soit $n\in\N^{*}$.
	\begin{equation}
		\underbrace{1\dots 1}_{\text{n fois en base 10}}=1+10+\dots+10^{n-1}=\frac{10^{n}-1}{9}
	\end{equation}
	On a 
	\begin{align}
		21\Bigm|\frac{10^{n}-1}{9}
		&\Longleftrightarrow 3\Bigm|\frac{10^{n}-1}{9}\text{ et }7\Bigm|\frac{10^{n}-1}{9}\\
		&\Longleftrightarrow 27\bigm|10^{n}-1\text{ et }7\bigm| 10^{n}-1
	\end{align}
	car $7\wedge 9=1$.
	Dans $\Z/7\Z$, on a $\overline{10}=\overline{3}$ donc pour tout $k\in\N$, $\overline{10}^{6k}=\overline{1}$ d'après le petit théorème de Fermat. Dans $\Z/27\Z$, $\widetilde{10}$ est inversible car $10\wedge 27=1$. $\left((\Z/27\Z)^{\times},+,\times\right)$ comporte 18 éléments donc pour tout $k'\in\N$, on a $\widetilde{10}^{18k'}=\widetilde{1}$.

	Lorsque $81\mid n$, on a $21\mid 1\dots 1$. 
	
	Cherchons plus précisément les ordres de $\overline{10}$ dans $((\Z/7\Z)^{*},\times)$ et de $\widetilde{10}$ dans $((\Z/27\Z)^{\times},\times)$.
	Dans $(\Z/7\Z)^{*}$, groupe de cardinal 6, on vérifie que l'ordre de 10 est 6. Dans l'autre groupe, on vérifie que l'ordre de $\widetilde{10}$ est 3. Ainsi, $21\mid 1\dots 1$ si et seulement si $6\mid n$.

	\begin{equation}
		\boxed{\text{Il y a donc une infinité de multiples de 21 qui s'écrivent avec uniquement des 1 en base 10.}}
	\end{equation}
\end{proof}

\begin{remark}
	Il suffit de trouver l'ordre de 10 dans les deux ensembles et de prendre le ppcm.
\end{remark}

\begin{proof}
	\phantom{}
	\begin{enumerate}
		\item $X^{d}-1$ a au plus $d$ racines dans $\K$. Pour tout $k\in\left\llbracket 0,d-1\right\rrbracket$, $x_{0}^{k}$ est racine de $X^{d}-1_{\K}$ car $gr\left\{x_{0}\right\}$ a pour cardinal $d$. Donc les racines sont exactement les puissances de $x_{0}$.
		
		Soit $x\in\K^{*}$ d'ordre $d$. On a $x\in gr\left\{x_{0}\right\}$ car $x^{d}=1$ (racine du polynôme de $X^{d}=1_{\K}$). Or, dans le groupe cyclique engendré par $x_{0}$, 
		\begin{equation}
			\boxed{\text{il y a }\varphi(d)\text{ éléments.}}
		\end{equation}

		\item On a ou bien $\varphi(d)$ ou bien aucun élément d'ordre $d$ dans $\K$. Soit $d$ tel que $d\mid n$, on note $H_{d}=\{x\in K\bigm| \omega(x)=d\}$. On a 
		\begin{equation}
			\K^{*}=\bigcup_{d\mid n}H_{d}
		\end{equation}
		Alors
		\begin{equation}
			n=\sum_{d\mid n}\vert H_{d}\vert\leqslant\sum_{d\mid n}\varphi(d)=n
		\end{equation}

		Alors pour tout $d$ tel que $d\mid n$, on a $\vert H_{d}\vert=\varphi(d)$. En particulier, on a $\vert H_{n}\vert=\varphi(n)\geqslant1$ donc $H_{n}$ est non vide. Donc il existe (au moins) un élément d'ordre $n$, donc 
		\begin{equation}
			\boxed{(\K^{*},\times)\text{ est cyclique.}}
		\end{equation}
	\end{enumerate}
\end{proof}

\begin{proof}
	\phantom{}
	\begin{enumerate}
		\item Soit $x\in M$. On a $\overline{1}-\overline{x}^{-1}$ si et seulement si $\overline{x}=\overline{1}$ et $\overline{1}-\overline{x}^{-1}=\overline{1}$ si et seulement si $\overline{x}=\overline{0}$, ce qui n'est pas possible pour les deux cas. 
		\begin{equation}
			\boxed{\text{Donc f est bien définie.}}
		\end{equation}
		
		Soit $x\in M$, on a 
		\begin{align}
			f^{2}(x)
			&=f(\overline{1}-\overline{x}^{-1})\\
			&=\overline{1}-(\overline{1}-\overline{x}^{-1})^{-1}\\
			&=(\overline{1}-\overline{x}^{-1})^{-1}(\overline{1}-\overline{x}^{-1}-\overline{1})\\
			&=-\overline{x}^{-1}(\overline{1}-\overline{x}^{-1})^{-1}
		\end{align}
		Donc 
		\begin{align}
			f^{3}(x)
			&=\overline{1}-(\overline{1}-(\overline{1}-\overline{x}^{-1})^{-1})^{-1}\\
			&=\overline{1}-(-x\overline{x}^{-1}(\overline{1}-\overline{x}^{-1})^{-1})^{-1}\\
			&=\overline{1}+\overline{x}(\overline{1}-\overline{x}^{-1})\\
			&=\overline{1}+\overline{x}-\overline{1}\\
			&=\overline{x}
		\end{align}

		Donc
		\begin{equation}
			f^{3}=id_{M}
		\end{equation}

		\item Soit $x\in M$, on a 
		\begin{align}
			f(x)=x
			&\Longleftrightarrow \overline{1}-\overline{x}^{-1}=x\\
			&\Longleftrightarrow \overline{x}^{2}-\overline{x}+\overline{1}=\overline{0}\\
			&\Longleftrightarrow (\overline{x}-\overline{2}^{-1})^{2}+\overline{3}\times\overline{4}^{-1}=\overline{0}\\
			&\Longleftrightarrow \overline{-3}=(\overline{2}\overline{x}-\overline{1})^{2}
		\end{align}
		$f$ admet un point fixe si et seulement $\overline{-3}$ est un carré dans $\Z/p\Z$ car $\overline{y}=\overline{2}\overline{x}-\overline{1}$ si et seulement si $\overline{x}=\overline{2}^{-1}(\overline{y}+\overline{1})$.
		
		Donc 
		\begin{equation}
			\boxed{\overline{-3}\text{ est un carré dans }\Z/p\Z\text{si et seulement si f admet un point fixe.}}
		\end{equation}

		\item Comme $p$ est premier plus grand que 5, on a $p\equiv 1\text{ ou }2[3]$ donc $p-2\equiv 0\text{ ou }2[3]$ car $f^{3}=id_{M}$, les longueurs des cycles qui composent $f$ valent 1 ou 3. 
		
		Si $f$ n'a pas de point fixe, tous les cycles sont de longueur 3, donc $3\mid p-2$ donc $p\equiv 2[3]$. Si $p\equiv 2[3]$, alors $3\mid p-2$, le nombre de points fixes est un multiple de $3$ donc aussi du nombre de racine carrés de $\overline{-3}$. Et puisque l'on est dans un corps, il y a au plus 2 racines de $\overline{-3}$. Donc si $p\equiv2[3]$, il n'y a pas de point fixe.

		Donc 
		\begin{equation}
			\boxed{\overline{-3}\text{ est un carré dans }\Z/p\Z\text{ si et seulement si }p\equiv1[3].}
		\end{equation}
	\end{enumerate}
\end{proof}

\begin{proof}
	Soit $x\in\R$. Supposons que $x$ possède un développement décimal périodique. Alors il existe $(n_{0},T)\in\N\times\N^{*}$ tels que pour tout $n\geqslant n_{0}$, $a_{n+T}=a_{n}$. On a alors 
	\begin{equation}
		\vert x\vert=\underbrace{b_{m}\dots b_{0},a_{0}\dots a_{n_{0}-1}}_{\in\Q}+\frac{1}{10^{n_{0}-1}}\underbrace{(0,a_{n_{0}}\dots a_{n_{0}+T-1}a_{n_{0}}\dots)}_{=y}
	\end{equation}
	\begin{equation}
		10^{T}y-y=a_{n_{0}}\dots a_{n_{0}+T-1}\in\N
	\end{equation}
	et donc 
	\begin{equation}
		y=\frac{a_{n_{0}}\dots a_{n_{0}+T-1}}{10^{T}-1}\in\Q
	\end{equation}
	Donc $x\in Q$.

	Réciproquement, soit $x=\frac{p}{q}\in\Q$ avec $q\in\N^{*}$. Il existe $(a,b)\in\Z\times\N^{*}$ tel que $p=aq+b$ avec $b\in\left\llbracket 0,q-1\right\rrbracket$. Si $b=0$, on arrête. On a sinon 
	\begin{equation}
		x=a+\frac{1}{10^{k}}\frac{10^{k}b}{q}
	\end{equation}
	où $k=\min\{m\geqslant1\bigm| 10^{m}b>q\}$.
	On réitère l'algorithme avec $\frac{10^{k}b}{q}$ car on a $\left\lfloor\frac{10^{k}b}{q}\right\rfloor\in\left\llbracket 1,9\right\rrbracket$ par définition de $k$.

	Il y a $q$ restes possibles dans la division euclidienne par $q$. Ainsi, au bout d'au plus de $q+1$ itérations, on retrouve un reste précédent. Par unicité de la division euclidienne, on obtient un développement décimal périodique.

	Donc 
	\begin{equation}
		\boxed{x\in\Q\text{ si et seulement si }\exists n_{0}\in\N,\exists T\in\N^{*},\forall n\geqslant n_{0},a_{n+T}=a_{n}.}
	\end{equation}
\end{proof}

\begin{remark}
	On peut écrire $q=2^{a}5^{b}q'$ avec $q'\wedge 2=q'\wedge 5=1$. On se ramène alors à $q\wedge2=q\wedge5=1$. En reportant dans l'écriture décimale de $x$, on a 
	\begin{equation}
		\frac{\alpha}{q}=\frac{\beta}{10^{T}-1}
	\end{equation}
	avec $\alpha\wedge q=1$. On a donc $q\mid 10^{T}-1$ d'après le lemme de Gauss. $T$ revient donc à l'ordre de $\overline{10}$ dans $\left(\left(\Z/q\Z\right)^{\times},\times\right)$ qui contient $\varphi(q)$ éléments. Par défaut, on a donc $T=\varphi(q)$.
\end{remark}

\begin{proof}
	\phantom{}
	\begin{enumerate}
		\item Soit $m\in\Z$. Si $m\in\left\llbracket 0,n-1\right\rrbracket$, on a $H_{n}(m)=0\in\Z$.
		Si $m\geqslant n$, on a $H_{n}(m)=\binom{m}{n}\in\Z$. Si $m<0$, on a 
		\begin{equation}
			H_{n}(m)=\frac{m(m-1)\dots(m-n+1)}{n!}=(-1)^{n}\binom{-m+n-1}{-m-1}\in\Z
		\end{equation}
		Donc 
		\begin{equation}
			\boxed{H_{n}(\Z)\subset\Z}
		\end{equation}

		\item Supposons qu'il existe $n\in\N$ et $(a_{0},\dots,a_{n})\in\Z^{n+1}$ et $P=\sum_{k=0}^{n}a_{k}H_{k}$. On a $H_{k}(\Z)\subset\Z$ donc $P(\Z)\subset\Z$.
		Supposons $P(\Z)\subset\Z$. $(H_{k})_{k\in\N}$ est une base étagée en degré de $\C[X]$. Donc il existe $(a_{0},\dots,a_{n})\in\C^{n+1}$ tel que $P=\sum_{k=0}^{n}a_{k}H_{k}$. Par récurrence, on a $P(0)=a_{0}\in\Z$. Soit $k\in\left\llbracket 0,n-1\right\rrbracket$, supposons $(a_{0},\dots,a_{k})\in\Z^{k+1}$. On a alors 
		\begin{equation}
			P(k+1)=\underbrace{\sum_{i=0}^{k}\underbrace{a_{k}}_{\in\Z}H_{k}}_{\in\Z}+a_{k+1}\underbrace{H_{k+1}(k+1)}_{=1}
		\end{equation}
		Donc $a_{k+1}\in\Z$.

		Donc 
		\begin{equation}
			\boxed{P\left(\Z\right)\subset\Z\text{ si et seulement si }\exists n\in\N,\exists(a_{0},\dots,a_{n})\in\Z^{n+1},P=\sum_{k=0}^{n}a_{k}H_{k}.}
		\end{equation}
	\end{enumerate}
\end{proof}

\begin{remark}
	Les translation $X+\alpha$ sont les seules pour lesquelles on a $(X+\alpha)(\Z)=\Z$. En effet, si $P\in\C[X]$ est tel que $P(\Z)=\Z$, on a $P\in\Q[X]$ d'après ce qui précède. Si $\deg(P)\geqslant2$, quitte à remplacer $P$ par $-P$, on peut supposer le coefficient dominant de $P$ strictement positif. On a alors $\lim\limits_{x\to+\infty}P'(x)=+\infty$ donc il existe $A>0$ tel que $P$ est strictement croissant sur $[A,+\infty[$. De plus, $P(x+1)-P(x)\to+\infty$ quand $x\to+\infty$. Donc il existe $A'>0$ tel que $P(x+1)>P(x)+1$. Pour $n\geqslant\max(A,A')$, on a $P(n+1)\geqslant P(n)+2$ ce qui contredit $P(\Z)=\Z$. Donc le degré de $P$ est inférieur à 1.
\end{remark}

\begin{proof}
	Le coefficient en $X^{k}$ s'écrit $a_{k-1}-\alpha a_{k}\in\Q$. Si $a_{k}\in\Q$, on a donc $a_{k-1}\in\Q$. Il est donc impossible d'avoir deux coefficients consécutifs rationnels. Or $x_{n-1}\in\Q$ car c'est le coefficient dominant de $P$. Donc 
	\begin{equation}
		\boxed{\alpha\text{ est nécessairement racine simple.}}
	\end{equation}
\end{proof}

\begin{proof}
	Soit $\Delta=P\wedge P'=\Delta$. On a $\deg(\Delta)\in\left\{1,2,3,4\right\}$ car $\Delta\mid P'$.

	Si $\deg(\Delta)=4$, alors $\Delta=P'$ (car associé). Donc il existe $\beta\in\C$ d'où $\underbrace{P}_{\in\Q[X]}=(X-\beta)\underbrace{P'}_{\in\Q[X]}$. Par division euclidienne, $X-\beta\in\Q[X]$ et $\beta\in\Q$ d'après l'algorithme de la division euclidienne.

	Si $\deg(\Delta)=1$, on a $P=X-\beta$ avec $\beta\in\Q$ racine de $P$.

	Si $\deg(\Delta)=2$, si $\Delta=(X-\beta)^{2}$, on a $\Delta'=2(X-\beta)\in\Q[X]$ donc $\beta\in\Q$ racine de $\Delta$ donc de $P$.
	Si $\Delta=(X-\alpha_{1})(X-\alpha_{2})$ avec $\alpha_{1}\neq\alpha_{2}$. $\alpha_{1}$ et $\alpha_{2}$ sont racines doubles de $P$ donc $P=(X-\beta)\underbrace{(X-\alpha_{1})^{2}(X-\alpha_{2})^{2}}_{=\Delta^{2}\in\Q[X]}$
	Par division euclidienne, $X-\beta\in\Q[X]$ et donc $\beta\in\Q$.

	Si $\deg(\Delta)=3$, si $\Delta=(X-\beta)^{3}$, on a $\Delta^{(2)}=6(X-\beta)\in\Q[X]$ donc $\beta\in\Q$.
	Si $\Delta=(X-\alpha_{1})(X-\alpha_{2})(X-\alpha_{3})$ avec $\alpha_{1},\alpha_{2}$ et $\alpha_{3}$ distinctes. $\alpha_{1},\alpha_{2}$ et $\alpha_{3}$ seraient racines doubles de $P$ ce qui contredit $\deg(P)=5$.
	Si $\Delta=(X-\alpha)^{2}(X-\beta)$, $\alpha$ est racine triple de $P$ et $\beta$ racine double de $P$ donc $P=(X-\alpha)^{3}(X-\beta)^{2}\in\Q[X]$. Par division euclidienne, $(X-\alpha)(X-\beta)\in\Q[X]$ et 
	\begin{equation}
		X-\alpha=\frac{\Delta}{(X-\alpha)(X-\beta)}\in\Q[X]
	\end{equation}
	donc $\alpha\in\Q$.

	Donc
	\begin{equation}
		\boxed{\text{P admet au moins une racine rationnelle.}}
	\end{equation}
\end{proof}

\begin{proof}
	\phantom{}
	\begin{enumerate}
		\item $1\in\Z[\i],0\in\Z[\i],\i\in\Z[\i]$. Soit $(a,b,a',b')\in\Z^{4}$:
		\begin{equation}
			\left\{
				\begin{array}[]{l}
					(a+\i b)-(a'+\i b')=(a-a')+\i(b-b')\in\Z[\i]\\
					(a+\i b)\times (aa'-bb')+\i(ab'+ba')\in\Z[\i]
				\end{array}
			\right.
		\end{equation}
		
		Donc $\Z[\i]$ est un sous-anneau de $\C$ contenant $\i$.

		Soit $A$ un sous anneau de $\C$ contenant $\i$. $A$ est stable par $x$ donc $i^{4}=1\in A$. $A$ est stable par + donc $\Z\subset A$, puis $\i\Z\subset A$ donc $\Z[\i]\subset A$. 
		\begin{equation}
			\boxed{\Z[\i]\text{ est donc le plus petit sous anneau de }\C\text{ contenant }\i.}
		\end{equation}

		\item Si $\vert z\vert^{2}=1$ c'est-à-dire $a^{2}+b^{2}=1$, alors 
		\begin{equation}
			\frac{1}{z}=\frac{a-\i b}{\vert z\vert^{2}}=a-\i b\in\Z[\i]
		\end{equation}
		Si $z$ est inversible dans $\Z[\i]$, il existe $'\in\Z[\i]$ tel que $zz'=1$ donc $\vert z\vert^{2}\vert z'\vert^{2}=1$ donc $\vert z\vert^{2}=1$. Donc 
		\begin{equation}
			\boxed{z\text{ est inverse dans }\Z[\i]\text{ si et seulement si }\left\lvert z\right\rvert^{2}=1.}
		\end{equation}

		Soit $(a,b)\in\Z^{2}$. Si $\vert a\vert\geqslant2$ ou $\vert b\geqslant2$, alors $a^{2}+b^{2}\geqslant4$ donc si $\vert z\vert^{2}=1$, alors $a^{2}+b^{2}=1$ et $\left(\vert a\vert=1\text{ et }\vert b\vert=0\right)$ ou $\left(\vert a\vert=0\text{ et }\vert b\vert=1\right)$. Donc 
		\begin{equation}
			\boxed{U=\left\{1,-1,\i,-\i\right\}}
		\end{equation}

		\item 
		\begin{enumerate}
			\item Si $x\in\R$, il existe $n\in\Z$ tel que $\vert x-n\vert\leqslant\frac{1}{2}$ (faire un dessin et le montrer grâce aux parties entières). Soit alors $z_{0}=x_{0}+\i y_{0}\in\C$, on prend un $(a,b)\in\Z^{2}$ tel que $\vert x_{0}-a\vert\leqslant\frac{1}{2},\vert y_{0}-b\vert\leqslant\frac{1}{2}$. Et pour $z=a+\i b\in\Z[\i]$, on a 
			\begin{equation}
				\boxed{\vert z-z_{0}\vert^{2}=(x_{0}-a)^{2}+(y_{0}-b)^{2}\leqslant\frac{1}{2}}
			\end{equation}
			
			\item Soit $(q,r)\in\Z[\i]^{2}$, on a $z_{1}=qz_{2}+r$ si et seulement si $\frac{z_{1}}{z_{2}}-q=\frac{r}{z_{2}}$. On a $\vert r\vert<\vert z_{1}\vert$ si et seulement si $\left\vert\frac{z_{1}}{z_{2}}-q\right\vert<1$.
			On a $\frac{z_{1}}{z_{2}}\in\C$ donc d'après 3.(a), il existe $q\in\Z[\i]$ tel que $\left\lvert \frac{z_{1}}{z_{2}}-q\right\rvert\leqslant\frac{\sqrt{2}}{2}<1$. On pose alors $r=z_{1}-qz_{2}\in\Z[\i]$ par stabilité. Il vient donc $\vert r\vert<\vert z_{2}\vert$. Ainsi,
			\begin{equation}
				\boxed{\exists(q,r)\in\Z[\i]^{2},z_{1}=qz_{2}+r\text{ et }\left\lvert r\right\rvert<\left\lvert z_{1}\right\rvert.}
			\end{equation}

			Si $z_{2}=1$ et $z_{1}=\frac{1+\i}{2}$, on peut prendre $q\in\left\{0,1,\i,1+\i\right\}$.
			Donc 
			\begin{equation}
				\boxed{\text{il n'y a pas unicité.}}
			\end{equation}

			\item Soit $I\neq\left\{0\right\}$ un idéal de $\Z[\i]$. On note $n_{0}=\min\left\{\vert z\vert^{2}\bigm| z\in I\setminus\left\{0\right\}\right\}$ (partie non vide de $\N^{*}$). Soit $z_{0}\in I\setminus\left\{0\right\}$ tel que $\vert z_{0}\vert^{2}=n_{0}$. On a directement $z_{0}\Z[\i]\subset I$ ($I$ est un idéal). 
			
			Réciproquement, soit $z\in I$, d'après 3.(b), il existe $(q,r)\in\Z[\i]^{2}$ tel que 
			\begin{equation}
				r=\underbrace{z}_{\in I}-\underbrace{z_{0}}_{\in I}\underbrace{q}_{\in\Z[\i]}\in I
			\end{equation} 
			et $\vert r\vert^{2}<n_{0}$. Nécessairement, $r=0$ et $z=z_{0}q\in z_{0}\Z[\i]$. Donc $I=z_{0}\Z[\i]$. Finalement, 
			\begin{equation}
				\boxed{\Z[\i]\text{ est principal.}}
			\end{equation}
		\end{enumerate}

		
		\item Si $\vert z\vert^{2}=1$, alors $z\in U$ donc c'est bon. On travaille ensuite par récurrence sur $n\in\N^{*}$. Supposons que la décomposition existe pour $z\in\Z[\i]$ avec $\vert z\vert^{2}\leqslant n$. Soit $z\in\Z[\i]$ tel que $\vert z\vert^{2}=n+1$. On a $\vert z\vert^{2}\geqslant2$ donc $z\in U$. Si $z$ est irréductible, c'est bon. Sinon, il existe $(z_{1},z_{2})\in\Z[\i]^{2}$ tel que $z=z_{1}z_{2}$ et $z_{1}$ et $z_{2}$ non inversibles. Alors $\vert z_{1}\vert^{2}\geqslant2$ et $\vert z_{2}\vert^{2}\geqslant2$. Or $\vert z\vert^{2}=n+1=\vert z_{1}\vert^{2}\vert z_{2}\vert^{2}$ donc $\vert z_{1}\vert^{2}\leqslant n$ et $\vert z_{2}\vert^{2}\leqslant n$. Par hypothèse de récurrence, on peut décomposer $z_{1}$ et $z_{2}$, donc $z$ est décomposable
		\begin{equation}
			\boxed{\text{D'où le résultat par récurrence.}}
		\end{equation}
			
		Pour l'unicité, soit $z\in\Z[\i]\setminus\left\{0\right\}$ tel que $z=u\prod_{\rho\in\mathcal{P}_{0}}\rho^{\nu_{\rho}(z)}=v\prod_{\rho\in\mathcal{P}_{0}}\rho^{\mu_{\rho}(z)}$. Le théorème de Gauss est valable dans $\Z[\i]$, car c'est un anneau principal. S'il existe $\rho_{0}\in\mathcal{P}_{0}$ tel que $\nu_{\rho_{0}}(z)<\mu_{\rho_{0}}(z)$, alors 
		\begin{equation}
			\rho_{0}\Bigm|\prod_{p\in\mathcal{P}_{0}\setminus\{\rho_{0}\}}\rho^{\nu_{\rho}(z)}
		\end{equation}
		ce qui est proscrit par le théorème de Gauss. On a donc pour tout $\rho\in\mathcal{P}_{0}$, $\nu_{\rho}(z)=\mu_{\rho}(z)$. En reportant, on a $u=v$.
		\begin{equation}
			\boxed{\text{D'où l'unicité de la décomposition.}}
		\end{equation}
	\end{enumerate}
\end{proof}

\begin{proof}
	\phantom{}
	\begin{enumerate}
		\item On a $\overline{1}\in R$. Soit $(\overline{x_{1}},\overline{x_{2}})\in R^{2}$, il existe $(\overline{y_{1}},\overline{y_{2}})\in(\mathbb{F}_{p}^{*})^{2}$ tel que $\overline{x_{1}}=\overline{y_{1}}^{2}$ et $\overline{x_{2}}=\overline{y_{2}}^{2}$. On a  alors 
		\begin{equation}
			\overline{x_{1}}\overline{x_{2}}^{-1}=(\overline{y_{1}}\overline{y_{2}}^{-1})^{2}\in R
		\end{equation}
		donc 
		\begin{equation}
			\boxed{R\text{ est un sous groupe de }(\mathbb{F}_{p}^{*},\times).}
		\end{equation}
		
		Soit \function{\varphi}{\mathbb{F}_{p}^{*}}{\mathbb{F}_{p}^{*}}{\overline{y}}{\overline{y}^{2}}
		On a $\im(\varphi)=R$. Comme $\mathbb{F}_{p}$ est un corps, chaque éléments de $R$ a exactement 2 antécédents par $\varphi$. Donc $\vert R\vert=\frac{\vert\mathbb{F}_{p}^{*}\vert}{2}=\frac{p-1}{2}$.

		S'il existe $\overline{y}\in\mathbb{F}_{p}^{*}$ tel que $\overline{a}=\overline{y}^{2}$, on a $\overline{a}^{\frac{p-1}{2}}=\overline{y}^{p-1}=\overline{1}$ par le théorème de Fermat.

		Réciproquement, si $\overline{a}^{\frac{p-1}{2}}=\overline{1}$, $X^{\frac{p-1}{2}}-\overline{1}$ admet au plus $\frac{p-1}{2}$ racines dans $\mathbb{F}_{p}^{*}$. Tous les éléments de $R$ sont racines de ce polynôme, ce sont donc ses seules racines. Donc $a\in R$.

		\begin{equation}
			\boxed{\text{Donc }a\in R\text{ si et seulement si }a^{\frac{p-1}{2}}=1.}
		\end{equation}

		\item Si $p=a^{2}+b^{2}$, alors $\overline{0}=\overline{a}^{2}+\overline{b}^{2}$. Si $\overline{a}=\overline{b}=\overline{0}$, on a $p\mid a$ et $p\mid b$ donc $p^{2}\mid p$ ce qui est exclu. Par exemple, si $\overline{a}\neq\overline{0}$, on a $\overline{1}=-\overline{b}^{2}\overline{a}^{-2}$ donc $\overline{-1}=(\overline{a}^{-1}\overline{b})^{2}\in R$ d'après 1. On a donc $(\overline{-1})^{\frac{p-1}{2}}=\overline{1}$ si et seulement si $2\bigm|\frac{p-1}{2}$ (car $p$ est premier plus grand que 3) d'où $4\mid p-1$ donc 
		\begin{equation}
			\boxed{p\equiv 1[4]}
		\end{equation}
		
		\item On a $\vert\mathbb{F}_{p}\vert=p$, $E(\sqrt{p})\leqslant\sqrt{p}<E(\sqrt{p})+1$ et $\vert\{0,\dots,E(\sqrt{p})\}\vert^{2}=(E(\sqrt{p})+1)^{2}>p$ ($p$ est premier, ce n'est pas un carré) donc (cardinalité)
		\begin{equation}
			\boxed{f\text{ n'est pas injective.}}
		\end{equation}
		
		Donc il existe 
		\begin{equation}
			((a_{1},b_{1}),(a_{2},b_{2}))\in(\{0,\dots,E(\sqrt{p})\}^{2})^{2}
		\end{equation} avec $(a_{1},b_{1})\neq (a_{2},b_{2})$ et $f(a_{1},b_{1})=f(a_{2},b_{2})$. Donc 
		\begin{equation}
			\overline{a_{1}}-\overline{k}\overline{b_{1}}=\overline{a_{2}}-\overline{k}\overline{b_{2}}\Rightarrow \overline{a_{1}}-\overline{a_{2}}=\overline{k}(\overline{b_{1}}-\overline{b_{2}})
		\end{equation}
		
		Si $\overline{b_{1}}=\overline{b_{2}}$, alors $\overline{a_{1}}=\overline{a_{2}}$ donc $p\mid b_{1}-b_{2}$ et $p\mid a_{1}-a_{2}$ donc $(a_{1},b_{1})=(a_{2},b_{2})$ ce qui n'est pas vrai. Donc $\overline{b_{1}}\neq\overline{b_{2}}$. Posons $b_{0}=b_{1}-b_{2}$ et $a_{0}=a_{1}-a_{2}$. On a $\overline{b_{0}}\neq\overline{0}$. Il vient donc $(\vert a_{0}\vert,\vert b_{0}\vert)\in\left\llbracket 1,E(\sqrt{p})\right\rrbracket^{2}$, $\overline{a_{0}}=\overline{k}\overline{b_{0}}$ donc 
		\begin{equation}
			\boxed{\overline{k}=\overline{a_{0}}\overline{b_{0}}^{-1}}
		\end{equation}

		\item Si $p\equiv 1[4]$, en remontant les calculs, on a $(\overline{-1})^{\frac{p-1}{2}}=\overline{1}$ donc $\overline{-1}\in R$ et il existe $\overline{k}\in\mathbb{F}_{p}^{*}$ tel que $\overline{-1}=\overline{k}^{2}$. Alors d'après 3., il existe $(a_{0},b_{0})$ tels que $\overline{k}=\overline{a_{0}}\overline{b_{0}}^{-1}$. Il vient alors $\overline{-1}=\overline{a_{0}}^{2}(\overline{b_{0}}^{-1})^{2}$ donc $\overline{-b_{0}}^{2}=\overline{a_{0}}^{2}$. On a 
		\begin{equation}
			p\mid a_{0}^{2}+b_{0}^{2}\in\left\llbracket 2,2E(\sqrt{p})\right\rrbracket^{2}\subset\left\llbracket 2,2p-1\right\rrbracket
		\end{equation}
		Nécessairement, $a_{0}^{2}+b_{0}^{2}=p$ et 
		\begin{equation}
			\boxed{p\text{ est somme de deux carrés.}}
		\end{equation}
	\end{enumerate}
\end{proof}

\begin{proof}
	\phantom{}
	\begin{enumerate}
		\item Soit $(m,n)\in A^{2}$. Il existe $(a,b,c,d)\in\N^{4}$ tel que $m=a^{2}+b^{2}=\vert a+\i b\vert^{2}$ et $n=c^{2}+d^{2}=\vert c+\i d\vert^{2}$. Donc 
		\begin{equation}
			\boxed{m\times n=\vert ac-bd6\i(bc+ad)\vert^{2}=(ac-bd)^{2}+(bc+ad)^{2}\in A}
		\end{equation}
		\item On a 
		\begin{equation}
			\boxed{n=\underbrace{\prod_{p\in\mathcal{P}_{1}}p^{\nu_{p}(n)}}_{\in A\text{ car }\mathcal{P}_{1}\subset A}\times\underbrace{\prod_{p\in\mathcal{P}_{2}}p^{\nu_{p}(n)}}_{=\prod_{p\in\mathcal{P}_{2}}p^{2\alpha_{p}}\in A}\in A}
		\end{equation}

		\item Soit $n\in A$, il existe $(a,b)\in\N^{2}$ avec $n=a^{2}+b^{2}$. Soit $p\in \mathcal{P}_{1}\cup\mathcal{P}_{2}$, on a $p\mid a^{2}+b^{2}$ donc $\overline{a^{2}+b^{2}}=\overline{0}$ dans $\Z/p\Z$. Si $p\nmid a$ ou $p\nmid b$, alors $\overline{1+\frac{b^{2}}{a^{2}}}=\overline{0}$ donc $\overline{-1}\in R$ (résidus quadratiques, voir exercice précédent). Donc $p=2$ ou $p\equiv 1[4]$.
		
		Si $p\mid a$ et $p\mid b$, $a=p^{k}a', b=p^{l}b'$ avec $p\nmid a'$ et $p\nmid b'$. On suppose $1\leqslant k\leqslant l$ (quitte à échanger $a$ et $b$).
		On a 
		\begin{equation}
			a^{2}+b^{2}=p^{2k}(a'^{2}+p^{2(l-k)}b'^{2})=n
		\end{equation}
		donc 
		\begin{equation}
			p\Bigm| a'^{2}+p^{2(l-k)b'^{2}}
		\end{equation}
		et $\overline{a'}^{2}+\overline{p^{2(l-k)}}\overline{b'}^{2}=\overline{0}$. Nécessairement, $l=k$. De même $p\in\mathcal{P}_{1}$. Par contraposée, $\nu_{p}$ est pair.
		\begin{equation}
			\boxed{\text{D'où la réciproque.}}
		\end{equation}
	\end{enumerate}
\end{proof}
\documentclass[12pt]{article}
\usepackage{style/style_sol}

\begin{document}

\begin{titlepage}
	\centering
	\vspace*{\fill}
	\Huge \textit{\textbf{Solutions MP/MP$^*$\\ Séries numériques et familles sommables}}
	\vspace*{\fill}
\end{titlepage}

\begin{proof}
	\phantom{}
	\begin{enumerate}
		\item On a $b_{0}=a_{1}=5,b_{1}=a_{3}=13$ et pour $p\geqslant2$, $b_{p}=2b_{p-1}+3b_{p-2}$.
		
		On a donc l'équation caractéristique $x^{2}-2x-3=0$. Les deux solutions sont 3 et -1. Donc il existe $(\lambda,\mu)\in\R^{2}$, $b_{p}=\lambda 3^{p}+\mu(-1)^{p}$.

		On a alors $b_{0}=5=\lambda+\mu$ et $b_{1}=13=3\lambda-\mu$. On trouve alors 
		\begin{equation}
			\boxed{\lambda=\frac{9}{2} \text{ et } \mu=\frac{1}{2}}
		\end{equation}

		\item On le montre par récurrence sur $p\in\N$.
		
		\item Si $3^{p}\leqslant n<3^{p+1}$, on a $a_{n}=b_{p}=\frac{9}{2}3^{p}+\frac{1}{2}(-1)^{p}$.
		Alors 
		\begin{equation}\frac{3}{2}+\frac{1}{2}(-1)^{p}\frac{1}{3^{p+1}}<\frac{a_{n}}{n}\leqslant\frac{9}{2}+\frac{1}{2}(-1)^{p}\frac{1}{3^{p}}\end{equation}
		Soit $\sigma\colon\N\to\N$ strictement croissante telle que 
		\begin{equation}\frac{a_{\sigma(n)}}{\sigma(n)}\xrightarrow[n\to+\infty]{}\lambda\end{equation}
		Soit $p_{n}\in\N$ tel que $3^{p_{n}}\leqslant\sigma(n)<3^{p_{n}+1}$. On a 
		\begin{equation}p_{n}=\bigl\lfloor\log_{3}(\sigma(n))\bigr\rfloor\xrightarrow[n\to+\infty]{}+\infty\end{equation}
		En reportant, on a $\frac{3}{2}\leqslant\lambda\leqslant\frac{9}{2}$.

		Si $\sigma(n)=3^{n}$, on a 
		\begin{equation}\frac{a_{3^{n}}}{3^{n}}=\frac{b_{n}}{3^{n}}=\frac{9}{2}+\frac{1}{2}\frac{(-1)^{n}}{3^{n}}\xrightarrow[n\to+\infty]{}\frac{9}{2}\end{equation}
		Si $\sigma(n)=3^{n+1}-1$, on a 
		\begin{equation}\frac{a_{3^{n}}}{3^{n}}=\frac{b_{n}}{3^{n+1}-1}\xrightarrow[n\to+\infty]{}\frac{3}{2}\end{equation}

		Soit $\mu\in[1,3[$ et $\sigma(n)=\lfloor 3^{n}\mu\rfloor\underset{n\to+\infty}{\sim}3^{n}\mu$. Alors 
		\begin{equation}\frac{a_{\sigma(n)}}{\sigma(n)}=\frac{b_{n}}{\lfloor3^{n}\mu\rfloor}\underset{n\to+\infty}{\sim}\frac{b_{n}}{3^{n}\mu}=\frac{9}{2\mu}+\frac{1}{2\mu}\frac{(-1)^{n}}{3^{n}}\xrightarrow[n\to+\infty]{}\frac{9}{2\mu}\end{equation}
		
		\begin{equation}\boxed{\text{Donc tout réel compris dans } \Biggl[\frac{3}{2},\frac{9}{2}\Biggr] \text{ est valeur d'adhérence.}}\end{equation}
	\end{enumerate}
\end{proof}

\begin{proof}
	\phantom{}
	\begin{enumerate}
		\item \function{g}{[a,b]}{\R}{x}{f(x)-x}
		est continue, $g(a)\geqslant0$ et $g(b)\leqslant0$, donc le théorème des valeurs intermédiaires affirme qu'il existe $l\in[a,b]$ avec $g(l)=0$, d'où 
		\begin{equation}\boxed{f(l)=l}\end{equation}

		\item On note $A=\{\lambda\bigm| \lambda\text{ est valeur d'adhérence}\}$.
		Le théorème de Bolzano-Weierstrass indique que $A$ est non vide. De plus, $A$ est borné car $A\subset[a,b]$. Soit $\lambda=\inf(A)$ et $\mu=\sup(A)$. 
		
		Si $\lambda=b$, on a $\mu=b$ et $A=\{b\}=\{\lambda\}=\{\mu\}$.

		Si $\lambda<b$, soit $\varepsilon>0$. Si $\lambda\notin A$, $\{k\in\N\bigm| x_{k}\in]\lambda,\lambda+\varepsilon[\}$ est infini. Par définition, $\lambda$ est valeur d'adhérence. Donc $\lambda\in A$, et de même $\mu\in A$.

		Soit $\nu\in]\lambda,\mu[$ avec $\lambda<\mu$. Si $\nu\notin A$, il existe $\varepsilon_{0}>0$ tel que $\{k\in\N\bigm|\vert x_{k}-\nu\vert<\varepsilon_{0}\}$ est fini. Donc il existe $N_{0}\in\N$ tel que pour tout $n\geqslant N_{0}$, $x_{n}\notin]\nu-\varepsilon_{0},\nu+\varepsilon_{0}[$. Comme $\lim\limits_{n\to+\infty}\vert x_{n+1}-x_{n}=0$, il existe $N_{1}\in\N$ tel que pour tout $n\geqslant N_{1}$, $\vert x_{n+1}-x_{n}\vert<2\varepsilon_{0}$. 
		Soit alors $n\geqslant\max(N_{0},N_{1})$. Si $x_{n}\leqslant\nu-\varepsilon_{0}$, alors $x_{n+1}\leqslant\nu-\varepsilon_{0}$. Si $x_{n}\geqslant\nu+\varepsilon_{0}$, alors $x_{n+1}\geqslant\nu+\varepsilon_{0}$. Ceci contredit que $\lambda$ et $\mu$ sont valeur d'adhérence. 
		
		Ainsi, $\nu\in A$ et 
		\begin{equation}\boxed{[\lambda,\mu] \text{ est le segment des valeurs d'adhérence.}}\end{equation}

		\item Si $(x_{n})$ converge, alors $\lim\limits_{n\to+\infty}x_{n+1}-x_{n}=0$. Réciproquement, si $\lim\limits_{n\to+\infty}x_{n+1}-x_{n}=0$, d'après 2., on a $A=[\lambda,\mu]$. On suppose $\lambda<\nu$. Ainsi, $\frac{\lambda+\nu}{2}=\alpha$ est valeur d'adhérence. Donc il existe $\sigma\colon\N\to\N$ strictement croissante telle que $x_{\sigma(n)}\xrightarrow[n\to+\infty]{}\alpha$. Alors $\lim\limits_{n\to+\infty}x_{\sigma(n)+1}=f(\alpha)$ par continuité de $f$ et c'est aussi égale à $\lim\limits_{n\to+\infty}x_{\sigma(n)}=\alpha$ car $\lim\limits_{n\to+\infty}x_{n+1}-x_{n}=0$.
		Ainsi, \begin{equation}\boxed{f(\alpha)=\alpha}\end{equation}

		Par ailleurs, il existe $n_{0}\in\N$ tel que $x_{n_{0}}\in[\lambda,\mu]$ et $f(x_{n_{0}})=x_{n_{0}}\in A$, alors pour tout $n\geqslant n_{0}$, on a $x_{n}=x_{n_{0}}$. Donc $(x_{n})_{n\in\N}$ converge et $\lambda=\mu$: $(x_{n})_{n\in\N}$ est bornée et a une unique valeur d'adhérence. 
		\begin{equation}\boxed{\text{Donc }(x_{n})_{n\in\N}\text{ converge.}}\end{equation}
	\end{enumerate}
\end{proof}

\begin{proof}
	On a $u_{n}=e^{\i 2^{n}\theta}$ pour tout $n\in\N$.

	Si $(u_{n})_{n\in\N}$ converge vers $l$, alors $\lim\limits_{n\to+\infty}u_{n}=1$ car $l=l^{2}$ et $\vert l\vert=1$.

	Si $(u_{n})_{n\in\N}$ est périodique au-delà d'un certain rang, il existe $T\in\N^{*}$, il existe $N_{0}\in\N$ tel que pour tout $n\geqslant N_{0}$, $u_{n+T}=u_{n}$. En particulier, $u_{N_{0}+T}=u_{N_{0}}$. On veut alors $2^{N_{0}+T}\theta\equiv 2^{N_{0}}\theta[2\pi]$. D'où $2^{N_{0}+T}\theta=2\theta+2k\pi$ donc $2^{N_{0}}(2^{T}-1)\theta=2k\pi$. Donc $\frac{\theta}{2\pi}\in\Q$.

	Réciproquement, si $\frac{\theta}{2\pi}\in\Q$, son développement binaire est périodique à partir d'un certain rang, et donc $(u_{n})_{n\in\N}$ l'est aussi.

	Si $(u_{n})_{n\in\N}$ est stationnaire, il existe $N\in\N$ tel que pour tout $n\geqslant N$, $U_{N+1}=U_{N}=U_{N^{2}}$. Comme $\vert U_{N}\vert=1$, alors $2^{n}\theta\in 2\pi\N$ et $\frac{\theta}{2\pi}$ est dyadique. 

	Réciproquement, s'il existe $p\in\N$, $u_{0}\in\N$ tel que $\frac{\theta}{2\pi}=\frac{p}{2^{n_{0}}}$ (nombre dyadique). Alors pour tout $n\geqslant n_{0}$, $2^{n}\theta\in 2\pi\N$ et $u_{n}=u_{n_{0}}=1$.

	Pour la densité, on prend une suite $(a_{n})_{n\in\N}$ en écrivant successivement, pour tout $k\in\N^{*}$, tous les paquets de $k$ entiers sont dans $\{0,1\}^{k}$. Soit $x\in[0,1[$ tel que 
	\begin{equation}x=\sum_{n=1}^{+\infty}\frac{a_{n}}{2^{n}}\end{equation}
	Soit $N\in\N$, il existe $p_{N}\in\N$, 
	\begin{equation}2^{p_{N}}\theta=2\pi\underbrace{(\dots)}_{\in\N}+2\pi(\frac{a_{1}}{2}+\dots+\frac{a_{N}}{2^{N}}+\underbrace{\dots}_{\in[0,\frac{1}{2^{N}}[})\end{equation}
	On a alors 
	\begin{equation}e^{\i2^{p_{N}}\theta}=e^{\i2\pi(\frac{a_{1}}{2}+\dots+\frac{a_{N}}{2^{N}}+\dots)}\end{equation}
	et 
	\begin{equation}\Bigl\vert\frac{a_{1}}{2}+\dots+\frac{a_{N}}{2^{N}}-x\Bigr\vert\leqslant\frac{1}{2^{N}}\end{equation}
	D'où $\lim\limits_{N\to+\infty}u_{p_{N}}=e^{\i2\pi x}$ et $(u_{n})_{n\in\N}$ est dense dans $\U$.
\end{proof}

\begin{proof}
	Si $a=0$ et $b=0$, $u_{n}\xrightarrow[n\to+\infty]{}0$.

	Si $a=0$ et $b\neq0$ (ou inversement), $u_{n}\underset{n\to+\infty}{\sim}\Bigl(\frac{1}{2}\Bigr)^{n^{2}}\xrightarrow[n\to+\infty]{}0$.

	Si $a>0$ ou $b>0$, on a
	\begin{align}
		u_{n}
		&=\exp\Bigl(n^{2}\ln\Bigl(\frac{e^{\frac{1}{n}\ln(a)}+e^{\frac{1}{n}\ln(b)}}{2}\Bigr)\Bigr)\\
		&=\exp\Bigl(n^{2}\ln\Bigl(1+\frac{1}{2n}\ln(ab)+\frac{1}{4n^2}(\ln(a)^{2}+\ln(b)^{2})\Bigr)+o\Bigl(\frac{1}{n^{2}}\Bigr)\Bigr)\\
		&=\exp\Bigl(\frac{n}{2}\ln(ab)+\frac{1}{4}(\ln(a)^{2}+\ln(b)^{2}+o(1))\Bigr)
	\end{align}

	Si $ab>1$, on a 
	\begin{equation}\boxed{\lim\limits_{n\to+\infty} u_{n}=+\infty}\end{equation}
	
	Si $ab<1$, on a 
	\begin{equation}\boxed{\lim\limits_{n\to+\infty} u_{n}=0}\end{equation}

	Si $ab=1$, on a 
	\begin{equation}\boxed{\lim\limits_{n\to+\infty} u_{n}=e^{\frac{1}{2}\ln(a)^{2}}}\end{equation}
\end{proof}

\begin{proof}
	\phantom{}
	\begin{enumerate}
		\item Soit $M=\sup\limits_{n\in\N}x_{n}>0$ (car $\sum_{n\in\N}x_{n}=+\infty$). 
		\begin{equation}J=\Biggl\{k\in\N\bigm| x_{k}\geqslant\frac{M}{2}\Biggr\}\end{equation}
		est fini car $x_{n}\xrightarrow[n\to+\infty]{}0$ et est non vide. On définit 
		\begin{equation}\varphi(0)=\min\Biggl\{k\in J\Bigm| x_{k}=\max\{x_{n}\bigm|n\in J\}\Biggr\}\end{equation}
		Pour tout $n\in J$, $x_{\varphi(0)}\geqslant x_{n}$. Si $n\notin J$, $x_{n}\leqslant\frac{M}{2}<x_{\varphi(0)}$. Ainsi, 
		\begin{equation}\boxed{x_{\varphi(0)}=\max\{x_{n}\bigm| n\in\N\}}\end{equation}
		Puis on recommence avec 
		\begin{equation}\Bigl\{x_{n}\bigm| n\in\N\setminus\{\varphi(0)\}\Bigr\}\end{equation}

		\item Pour $l=0$, pour tout $\varepsilon>0$, il existe $n\in\N$ tel que $x_{N}<\varepsilon$. On pose 
		\begin{equation}\boxed{I=\{N\}}\end{equation}
		et on a bien 
		\begin{equation}\Biggl\lvert\sum_{k\in I}x_{k}-l\Biggr\rvert\leqslant\varepsilon\end{equation}

		Si $l=+\infty$, soit $A>0$. Il existe $N\in\N$ tel que $\sum_{k=0}^{N}x_{k}>A$ (car $\sum_{n\in\N}x_{n}=+\infty$). Donc on peut prendre 
		\begin{equation}\boxed{I=\{0,\dots,N\}}\end{equation}

		Si $l\in\R_{+}^{*}$. Soit $\varepsilon>0$, on peut supposer sans perte de généralité que $\varepsilon<l$. Il existe $N_{0}\in\N$ tel que pour tout $n\geqslant N_{0}$, on a $x_{n}<\varepsilon$ et $\sum_{n=N_{0}}^{+\infty}x_{n}=+\infty$. Donc il existe un plus petit entier $N_{1}$ tel que $\sum_{n=N_{0}}^{N_{1}}x_{n}\geqslant l-\varepsilon$. Comme $x_{N_{1}}<\varepsilon$, on a $\sum_{n=N_{0}}^{N_{1}}x_{n}\leqslant l+\varepsilon$. Donc 
		\begin{equation}\boxed{I=\{N_{0},\dots,N_{1}\}}\end{equation}
	\end{enumerate}
\end{proof}

\begin{proof}
	On pose 
	\begin{equation}S_{n}=\sum_{k=0}^{n}u_{k}^{2}\end{equation}
	Montrons que $S_{n}\xrightarrow[n\to+\infty]{}+\infty$. D'abord, il existe $n_{0}\in\N$ tel que $u_{n_{0}}>$ donc $\lim\limits_{n\to+\infty}S_{n}=l\in\overline{R}_{+}^{*}$. Si $l<+\infty$, on a $u_{n}\xrightarrow[n\to+\infty]{}\frac{1}{l}$ et donc $u_{n}^{2}\xrightarrow[n\to+\infty]{}\frac{1}{l^{2}}$ et la série diverge. Donc $l=+\infty$ et comme $u_{n}\underset{n\to+\infty}{\sim}\frac{1}{S_{n}}$, on a $u_{n}\xrightarrow[n\to+\infty]{}0$.

	On observe ensuite que $S_{n}-S_{n-1}=u_{n}^{2}=o(1)$ donc $S_{n-1}\underset{n\to+\infty}{\sim}S_{n}$. Ainsi, 
	\begin{equation}\underbrace{u_{n}^{2}S_{n}^{2}}_{=~(S_{n}-S_{n-1})S_{n}^{2}}\xrightarrow[n\to+\infty]{}1\end{equation}
	et on a 
	\begin{equation}\frac{S_{n}^{2}+S_{n}S_{n-1}+S_{n-1}^{2}}{S_{n}^{2}}=1+\frac{S_{n-1}}{S_{n}}+\frac{S_{n-1}^{2}}{S_{n}^{2}}\xrightarrow[n\to+\infty]{}3\end{equation}
	donc 
	\begin{equation}\underbrace{(S_{n}-S_{n-1})(S_{n}^{2}+S_{n}S_{n-1}+S_{n-1}^{2})}_{=~S_{n}^{3}-S_{n-1}^{3}}\xrightarrow[n\to+\infty]{}3\end{equation}

	On applique le théorème de Césaro à la suite $S_{n}^{3}-S_{n-1}^{3}$:
	\begin{equation}\frac{S_{n}^{3}-S_{0}^{3}}{n}\xrightarrow[n\to+\infty]{}3\end{equation}
	donc $S_{n}\underset{n\to+\infty}{\sim}\sqrt[3]{3n}$, et comme $u_{n}\underset{n\to+\infty}{\sim}\frac{1}{S_{n}}$, on a bien 
	\begin{equation}\boxed{u_{n}\underset{n\to+\infty}{\sim}\frac{1}{\sqrt[3]{3n}}}\end{equation}

	Réciproquement, soit $u_{n}=\frac{1}{\sqrt[3]{3n}}$ avec $u_{0}=1$. On a 
	\begin{equation}u_{n}^{2}=\frac{1}{(3n)^{\frac{2}{3}}}\end{equation}
	Par comparaison série-intégrale, on a 
	\begin{equation}\sum_{k=0}^{n}u_{k}^{2}\underset{n\to+\infty}{\sim}\frac{1}{3^{\frac{2}{3}}}\times 3n^{\frac{1}{3}}=(3n)^{\frac{1}{3}}\end{equation}
	et donc 
	\begin{equation}\boxed{u_{n}\times\sum_{k=0}^{n}u_{k}^{2}\underset{n\to+\infty}{\sim}\frac{\sqrt[3]{3n}}{\sqrt[3]{3n}}=1}\end{equation}
\end{proof}

\begin{remark}
	On rappelle que l'on a la comparaison série-intégrale, pour $\alpha<1$,
	\begin{equation}\sum_{k=1}^{N}\frac{1}{k^{\alpha}}\underset{n\to+\infty}{\sim}\int_{1}^{N}\frac{dt}{t^{\alpha}}\underset{n\to+\infty}{\sim}\frac{1}{1-\alpha}N^{1-\alpha}\end{equation}
\end{remark}

\begin{proof}
	Tout d'abord, on montre que pour tout $x\in[0,1]$,
	\begin{equation}0\leqslant\cosh(x)-1-\frac{x^{2}}{2}\leqslant x^{4}\end{equation}
	en posant \function{f}{[0,1]}{\R}{x}{\cosh(x)-1-\frac{x^{2}}{2}}
	de classe $\mathcal{C}^{\infty}$ sur $[0,1]$ et on a $f''(x)=\cosh(x)-1\geqslant0$ et $f'(0)=0$. Comme $f(0)=0$, on a pour tout $x\in[0,1],f(x)\geqslant0$.

	Avec l'inégalité de Taylor-Lagrange à l'ordre 4 sur $f$, on a
	\begin{equation}0\leqslant\cosh(x)-1-\frac{x^{2}}{2}\leqslant\frac{x^{4}}{24}\times\underbrace{\sup\limits_{t\in[0,1]}\vert\cosh^{(4)}(t)\vert}_{\leqslant\cosh(1)}\leqslant x^{4}\end{equation}

	\begin{figure}[ht!]
		\centering
		\begin{tikzpicture}
			\begin{axis}[
				xmin=-2, xmax=2,
				ymin=-0.5, ymax=1,
				axis lines=center,
				axis on top=true,
				xlabel=$x$,
				samples=100,
				legend pos=outer north east
			]
			\addplot[blue, ultra thick] {cosh(x)-1-x^2/2};
			\addplot[color=red, ultra thick, restrict y to domain=-1.5:1.5] {x^4};
			\legend{$f(x)$,$x^4$}
			\end{axis}
		\end{tikzpicture}
		\caption{$0\leqslant\cosh(x)-1-\frac{x^{2}}{2}\leqslant x^{4}$ pour $x\in\R$.}
	\end{figure}

	On a 
	\begin{equation}-x_{n}=\sum_{k=1}^{n}\Bigl[\cosh\Bigl(\frac{1}{\sqrt{k+n}}\Bigr)-1\Bigr]\end{equation}
	Ainsi, 
	\begin{equation}0\leqslant x_{n}-\sum_{k=1}^{n}\frac{1}{2}\frac{1}{n+k}\leqslant\sum_{k=1}^{n}\frac{1}{(n+k)^{2}}\leqslant\frac{n}{(n+1)^{2}}\xrightarrow[n\to+\infty]{}0\end{equation}

	On a 
	\begin{equation}\sum_{k=1}^{n}\frac{1}{n+k}=H_{2n}-H_{n}=\ln(2n)+\gamma+o(1)-\ln(n)-\gamma=\ln(2)+o(1)\end{equation}
	Donc 
	\begin{equation}\boxed{\lim\limits_{n\to+\infty}x_{n}=-\frac{\ln(2)}{2}}\end{equation}
\end{proof}

\begin{proof}
	$\varphi$ est dérivable sur $\R$ et on a pour tout $x\in\R$, $\varphi'(x)=e^{x}-1$.

	\begin{figure}[ht!]
		\centering
		\begin{tikzpicture}
			\begin{axis}[
				xlabel=x,
				samples=100,
				xmin=-5, xmax=5, ymin=-5, ymax=5,
				legend pos=outer north east,
				axis lines=center,
				axis on top=true,]
				\addplot[blue, ultra thick] {e^x-x-1};
				\addplot[red, ultra thick] {-x-1};
				\legend{$\varphi(x)$,$y=-x-1$}
			\end{axis}
		\end{tikzpicture}
		\caption{$e^{x}-x-1\geqslant -x-1$ pour $x\in\R$.}
	\end{figure}

	On a 
	\begin{equation}0\varphi(a_{n})\leqslant\varphi(a_{n})+\varphi(b_{n})+\varphi(c_{n})\xrightarrow[n\to+\infty]{}0\end{equation}
	donc \begin{equation}\lim\limits_{n\to+\infty}\varphi(a_{n})=0\end{equation}

	Par l'absurde, soit $\varepsilon>0$. Supposons qu'il existe une infinité d'entiers $k\in\N$ tel que $\vert a_{k}\vert>\varepsilon$. Cela implique alors
	\begin{equation}\varphi(a_{k})\geqslant\min(\varphi(\varepsilon),\varphi(-\varepsilon))>0\end{equation}
	ce qui contredit $\lim\limits_{n\to+\infty}\varphi(a_{n})=0$.
	Donc 
	\begin{equation}\boxed{\lim\limits_{n\to+\infty}a_{n}=0}\end{equation}
	et c'est pareil pour $b_{n}$ et $c_{n}$.
\end{proof}

\begin{proof}
	\phantom{}
	\begin{enumerate}
		\item Soit \function{f}{]0,1[}{\R}{x}{x(1-x)}
		On a $f(x)\in]0,\frac{1}{4}]$. Pour tout $n\in\geqslant1$, $u_{n}\in]0,\frac{1}{4}]$. Par récurrence, on a donc $u_{n+1}\leqslant u_{n}$ et $\lim\limits_{n\to+\infty}u_{n}=0$. 
		
		\begin{equation}\boxed{\text{Donc }v_{n}\text{ est bien définie.}}\end{equation}

		\begin{figure}[!ht]
			\centering
			\begin{tikzpicture}
				\begin{axis}
					[
						xlabel=$x$,
						xmin=-0.5, xmax=1.5,
						ymin=-0.5, ymax=1,
						legend pos=outer north east,
						axis lines=center,
						axis on top=true,
						samples=100,
						extra y ticks={0.25}
					]
					\addplot[blue, ultra thick] {x*(1-x)};
					\addplot[red, ultra thick] {x};
					\addplot[green, dashed, domain=0:0.5] {0.25};
					\addplot[green, dashed, domain=0:0.5] coordinates {(0.5,0)(0.5,0.25)};
					\legend{$f(x)$,$y=x$}
				\end{axis}
			\end{tikzpicture}
			\caption{$x(1-x)\in\bigl]0,\frac{1}{4}\bigr]$ pour $x\in]0,1[$.}
		\end{figure}

		\item On a 
		\begin{equation}\frac{1}{u_{n+1}}=\frac{1}{u_{n}}\times\frac{1}{1-u_{n}}=\frac{1}{u_{n}}(1+u_{n}+o(u_{n}))=\frac{1}{u_{n}}+1+o(1)\end{equation}
		Donc $v_{n+1}-v_{n}\xrightarrow[n\to+\infty]{}1$. D'après le théorème de Césaro, on a 
		\begin{equation}\frac{v_{n}-v_{0}}{n}\xrightarrow[n\to+\infty]{}1\end{equation}
		donc $v_{n}\underset{n\to+\infty}{\sim}n$ et $u_{n}\underset{n\to+\infty}{\sim}\frac{1}{n}$.

		On a 
		\begin{equation}\frac{1}{u_{n+1}}=\frac{1}{u_{n}}(1+u_{n}+u_{n}^{2}+O(u_{n}^{3}))=\frac{1}{u_{n}}+1+u_{n}+\underbrace{O(u_{n}^{2})}_{=~O\bigl(\frac{1}{n^{2}}\bigr)}\end{equation}
		donc 
		\begin{equation}\frac{1}{u_{n+1}}-\frac{1}{u_{n}}=1+u_{n}+O\Bigl(\frac{1}{n^{2}}\Bigr)\end{equation}
		et $u_{n}\underset{n\to+\infty}{\sim}\frac{1}{n}$ donc $\sum_{k=0}^{n}u_{k}\underset{n\to+\infty}{\sim}\ln(n)$. En sommant, on a donc 
		\begin{equation}v_{n}-v_{0}=n+\ln(n)+o\bigl(\ln(n)\bigr)\end{equation}

		On a alors 
		\begin{align}
			u_{n}
			&=\frac{1}{n+\ln(n)+o\bigl(\ln(n)\bigr)}\\
			&=\frac{1}{n}\times\frac{1}{1+\frac{\ln(n)}{n}+o(\frac{\ln(n)}{n})}\\
			&=\frac{1}{n}\Bigl(1-\frac{\ln(n)}{n}+o\Bigl(\frac{\ln(n)}{n}\Bigr)\Bigr)\\
			&=\frac{1}{n}-\underbrace{\frac{\ln(n)}{n^{2}}+o\Bigl(\frac{\ln(n)}{n^{2}}\Bigr)}_{=~\alpha_{n}}
		\end{align}
		$\alpha_{n}$ est le terme genéral d'une série à termes positifs convergentes car $\alpha_{n}=O\Bigl(\frac{1}{n^{\frac{3}{2}}}\Bigr)$. Donc 
		\begin{equation}v_{n+1}-v_{n}=1+\frac{1}{n}+\alpha_{n}+O\Bigl(\frac{1}{n^{2}}\Bigr)\end{equation}
		et en sommant,
		\begin{equation}\boxed{v_{n}=n+\ln(n)+O(1)}\end{equation}
		et comme montré auparavant,
		\begin{equation}\boxed{u_{n}=\frac{1}{n}-\frac{\ln(n)}{n^{2}}+o\Bigl(\frac{\ln(n)}{n^{2}}\Bigr)}\end{equation}
	\end{enumerate}
\end{proof}

\begin{proof}
	\phantom{}
	\begin{enumerate}
		\item Soit \function{f_{n}}{\R^{+}}{\R}{x}{x^{n}-x-n}
		On a $f_{n}'(x)=nx^{n-1}-1=0$ si et seulement si 
		\begin{equation}x=\Bigl(\frac{1}{n}\Bigr)^{\frac{1}{n-1}}=\alpha_{n}\end{equation}
		$f_{n}(0)=0$ et $f_{n}(x)\xrightarrow[x\to+\infty]{}+\infty$.
		$f_{n}$ est monotone strictement sur $]\alpha_{n},+\infty[$.
		
		\begin{equation}\boxed{\text{Donc il existe un unique }x_{n}\in\R^{+}\text{ tel que }f_{n}(x_{n})=0}\end{equation}

		On a $f_{n}(1)=-n<0$ donc $x_{n}>1$ et $f_{n}(2)=2^{n}-2-n>0$ pour $n\geqslant3$ (on a $x_{2}=2$). Donc pour $n\geqslant3$, $x_{n}\in]1,2[$.

		\begin{figure}[!ht]
			\centering
			\begin{tikzpicture}
				\begin{axis}
					[
						xlabel=$x$,
						xmin=-0.5, xmax=2.5,
						ymin=-4, ymax=1,
						legend pos=outer north east,
						axis lines=center,
						axis on top=true,
						samples=100,
						extra y ticks={-3.38490017946},
						extra y tick labels={\color{red}$f_{3}(\alpha_{3})$}
					]
					\addplot[blue, ultra thick, restrict y to domain=-4:1] {x^3-x-3};
					\addplot[red, dashed, domain=0:0.57735] {-3.38490017946};
					\addplot[red, dashed, domain=0:0.57735] coordinates {(0.57735,0)(0.57735,-3.38490017946)};
					\node[anchor=north west, color=blue] at (axis cs: 1.6717,0) {$x_{3}$};
					\node[anchor=south, color=red] at (axis cs: 0.57735,0) {$\alpha_{3}$};
					\legend{$f_{3}(x)$}
				\end{axis}
			\end{tikzpicture}
			\caption{$x\mapsto x^{3}-x-3$ a exactement un zéro sur $\R_{+}$.}
		\end{figure}

		\item On a $x_{n}^{n}=x_{n}+n\leqslant2+n$ donc 
		\begin{equation}1\leqslant x_{n}\leqslant(2+n)^{\frac{1}{n}}=e^{\frac{1}{n}\ln(2+n)}\xrightarrow[n\to+\infty]{}1\end{equation}
		Donc 
		\begin{equation}\boxed{\lim\limits_{n\to+\infty}x_{n}=1}\end{equation}

		\item On peut poser $x_{n}=1+\varepsilon_{n}$ avec $\varepsilon_{n}>0$ et $\lim\limits_{n\to+\infty}\varepsilon_{n}=0$. On a 
		\begin{equation}(1+\varepsilon_{n})^{n}=1+\varepsilon_{n}+n\end{equation}
		donc 
		\begin{equation}n\ln(1+\varepsilon_{n})=\ln(1+\varepsilon_{n}+n)=\ln(n)+\underbrace{\ln\Bigl(1+\frac{1+\varepsilon_{n}}{n}\Bigr)}_{\underset{n\to+\infty}{\sim}\frac{1}{n}}\end{equation}
		et donc 
		\begin{equation}\varepsilon_{n}\underset{n\to+\infty}{\sim}\frac{\ln(n)}{n}\end{equation}

		On a donc 
		\begin{equation}x_{n}=1+\frac{\ln(n)}{n}+o\Bigl(\frac{\ln(n)}{n}\Bigr)\end{equation}

		On a enfin 
		\begin{equation}(1+\varepsilon_{n})^{n}=1+\varepsilon_{n}+n=1+n+\frac{\ln(n)}{n}+o\Bigl(\frac{\ln(n)}{n}\Bigr)\end{equation}
		d'où
		\begin{align}
			\ln(1+\varepsilon_{n})
			&=\frac{1}{n}\ln(n+1+\frac{\ln(n)}{n}+o\Bigl(\frac{\ln(n)}{n}\Bigr))\\
			&=\frac{1}{n}\Bigl[\ln(n)+\ln\Bigl(1+\frac{1}{n}+\underbrace{\frac{\ln(n)}{n^{2}}+o\Bigl(\frac{\ln(n)}{n^{2}}\Bigr)}_{=~o\bigl(\frac{1}{n}\bigr)}\Bigr)\Bigr]\\
			&=\frac{\ln(n)}{n}+\frac{1}{n^{2}}+o\Bigl(\frac{1}{n^{2}}\Bigr)
		\end{align}
		donc 
		\begin{equation}1+\varepsilon_{n}=e^{\frac{\ln(n)}{n}+\frac{1}{n^{2}}+o\Bigl(\frac{1}{n^{2}}\Bigr)}=1+\frac{\ln(n)}{n}+\frac{\ln(n)^{2}}{2n^{2}}+o\Bigl(\frac{\ln(n)^{2}}{n^{2}}\Bigr)\end{equation}
		puis
		\begin{equation}\varepsilon_{n}=\frac{\ln(n)}{n}+\frac{\ln(n)^{2}}{2n^{2}}+o\Bigl(\frac{\ln(n)^{2}}{n^{2}}\Bigr)\end{equation}
		et ainsi
		\begin{equation}\boxed{x_{n}=1+\frac{\ln(n)}{n}+\frac{\ln(n)^{2}}{2n^{2}}+o\Bigl(\frac{\ln(n)^{2}}{n^{2}}\Bigr)}\end{equation}
	\end{enumerate}
\end{proof}

\begin{proof}
	On note 
	\begin{equation}v_{n}=\lim\limits_{n\to+\infty}\frac{u_{n}a_{0}+u_{n-1}a_{1}+\dots+u_{0}a_{n}}{u_{0}+\dots+u_{n}}\end{equation}

	Si pour tout $n\in\N$, $a_{n}=a$ alors $v_{n}=a\xrightarrow[n\to+\infty]{}a$. De manière générale, on a 
	\begin{equation}v_{n}-a=v_{n}-a\frac{u_{n}+\dots+u_{0}}{u_{0}+\dots+u_{n}}=\frac{\sum_{k=0}^{n}u_{n-k}(a_{k}-a)}{u_{0}+\dots+u_{n}}\end{equation}

	Ainsi,
	\begin{equation}\vert u_{n}-a\vert\leqslant\frac{\sum_{k=0}^{n}u_{n-k}\vert a_{k}-a\vert}{u_{0}+\dots+u_{n}}\end{equation}

	Soit $\varepsilon>0$. Il existe $N\in\N$ tel que pour tout $k\geqslant N$, $\vert a_{k}-a\vert\leqslant\frac{\varepsilon}{2}$. Comme $(a_{k})_{k\in\N}$ converge, on note $M=\sup\limits_{k\in\N}\vert a_{k}-a\vert$. Soit $n\geqslant N$, on a 
	\begin{align}
		\vert v_{n}-a\vert
		&\leqslant\frac{\sum_{k=0}^{N-1}u_{n-k}\vert a_{k}-a\vert+\sum_{k=N}^{n}\vert a_{k}-a\vert}{u_{0}+\dots+u_{n}}\\
		&\leqslant\frac{\sum_{k=n-N+1}^{n}u_{k}M}{u_{0}+\dots+u_{n}}+\underbrace{\frac{\sum_{k=N}^{n}u_{n-k}\frac{\varepsilon}{2}}{u_{0}+\dots+u_{n}}}_{\leqslant\frac{\varepsilon}{2}}
	\end{align}
	car les $u_{i}$ sont positifs.

	On remarque enfin que 
	\begin{equation}
	\begin{array}[]{rcl}
		u_{n} &=& o(u_{0}+\dots+u_{n})\\
		u_{n-1} &=& o(u_{0}+\dots+u_{n-1})=o(u_{0}+\dots+u_{n})\\
		\vdots & &\\
		u_{n-N+1}&=&o(u_{0}+\dots+u_{n})
	\end{array}
	\end{equation}

	Donc 
	\begin{equation}M\frac{\sum_{k=n-N+1}^{n}u_{k}}{u_{0}+\dots+u_{n}}\xrightarrow[n\to+\infty]{}0\end{equation}
	et il existe $N'\in\C$ tel que pour tout $n\geqslant N'$, on a 
	\begin{equation}M\frac{\sum_{k=n-N+1}^{n}u_{k}}{u_{0}+\dots+u_{n}}\leqslant\frac{\varepsilon}{2}\end{equation}
	et donc pour tout $n\geqslant\max(N,N')$, on a $\vert v_{n}-a\vert\leqslant\frac{\varepsilon}{2}$ et ainsi 
	\begin{equation}\boxed{\lim\limits_{n\to+\infty}v_{n}=a}\end{equation}
\end{proof}

\begin{proof}
	\phantom{}
	\begin{enumerate}
		\item Pour $n\geqslant2$, (iii) donne 
		\begin{equation}x-\frac{a_{2}}{2}-\dots-\frac{a_{n}}{n!}=\sum_{k=n+1}^{+\infty}\frac{a_{k}}{k!}\end{equation}
		Ainsi,
		\begin{equation}0\leqslant x-\frac{a_{2}}{2}-\dots-\frac{a_{n}}{n!}<\sum_{k=n+1}^{+\infty}\frac{k-1}{k!}=\sum_{k=n+1}^{+\infty}\frac{1}{(k-1)!}-\frac{1}{k!}=\frac{1}{n!}\end{equation}
		où l'inégalité est stricte d'après (ii). Pour $n\geqslant2$, on a 
		\begin{equation}x-\frac{a_{2}}{2}<\frac{1}{2!}\end{equation}
		donc 
		\begin{equation}0\leqslant 2x-\underbrace{a_{2}}_{\in\N}<1\end{equation}
		Donc $a_{2}=\lfloor 2x\rfloor$.
		On a ensuite 
		\begin{equation}0\leqslant n!\Bigl(x-\frac{a_{2}}{2}-\dots-\frac{a_{n-1}}{(n-1)!}\Bigr)-\underbrace{a_{n}}_{\in\N}<1\end{equation}
		donc 
		\begin{equation}\boxed{a_{n}=\Biggl\lfloor n!\Bigl(x-\frac{a_{2}}{2}-\dots-\frac{a_{n-1}}{(n-1)!}\Bigr)\Biggr\rfloor}\end{equation}

		On a donc bien unicité.

		Soit maintenant $(a_{n})_{n\in\N}$ définie comme ci-dessus. On a, pour tout $n\geqslant2$, on a 
		\begin{equation}0\leqslant n!\Bigl(x-\frac{a_{2}}{2}-\dots-\frac{a_{n-1}}{(n-1)!}\Bigr)-\underbrace{a_{n}}_{\in\N}<1\end{equation}
		Or 
		\begin{equation}0-\frac{a_{2}}{2}-\dots-\frac{a_{n-1}}{(n-1)!}\leqslant\frac{1}{(n-1)!}\end{equation}
		donc 
		\begin{equation}a_{n}\in\{0,\dots,n-1\}\end{equation}
		et (i) est vérifié.

		On a 
		\begin{equation}0\leqslant x-\sum_{k=2}^{n}\frac{a_{k}}{k!}<\frac{1}{n!}\xrightarrow[n\to+\infty]{}0\end{equation}
		donc (iii) est vérifié, et supposons qu'il existe $n_{0}\geqslant2$ tel que pour tout $m\geqslant n_{0}+1$, on a $a_{m}=m-1$.
		Alors 
		\begin{equation}x=\sum_{k=0}^{n_{0}}\frac{a_{k}}{k!}+\sum_{k=n_{0}+1}^{+\infty}\frac{k-1}{k!}\end{equation}
		et
		\begin{equation}x-\sum_{k=0}^{n_{0}}\frac{a_{k}}{k!} = \sum_{k=n_{0}+1}^{+\infty}\frac{k-1}{k!}=\frac{1}{n_{0}!}\end{equation}
		donc 
		\begin{equation}n_{0}!\Bigl(x-\sum_{k=0}^{n_{0}}\frac{a_{k}}{k!}\Bigr)=1\end{equation}
		et 
		\begin{equation}n_{0}!\Bigl(x-\sum_{k=0}^{n_{0}-1}\frac{a_{n_{0}-1}}{(n_{0}-1)!}\Bigr)-a_{n_{0}}=1\end{equation}
		En prenant la partie entière, on a donc $0=1$ ce qui est absurde.

		Donc (ii) est vérifié.
	
		\item S'il existe $n_{0}\in\N$ tel que pour tout $n\geqslant n_{0}$, $a_{n}=0$ alors $x\in\Q$.
		
		Si $x=\frac{p}{q}\in\Q$, on a 
		\begin{equation}x=\frac{a_{2}}{2}+\dots+\frac{a_{n}}{n!}\end{equation}
		si et seulement si 
		\begin{equation}a_{n}=n!\Bigl(x-\frac{a_{2}}{2}-\dots-\frac{a_{n-1}}{(n-1)!}\Bigr)\end{equation}
		si et seulement si 
		\begin{equation}n!\Bigl(x-\frac{a_{2}}{2}-\dots-\frac{a_{n-1}}{(n-1)!}\Bigr)\in\N\end{equation}
		ce qui est vrai dès que $n\geqslant q$. Donc pour tout $n > q$, on a $a_{n}=0$ par unicité.

		\item Soit $l\in[-1,1]$. Soit $x\in[0,1[$ avec 
		\begin{equation}x=\sum_{k=2}^{+\infty}\frac{a_{k}}{k!}\end{equation}
		On a alors 
		\begin{equation}n!2\pi x=\underbrace{\sum_{k=2}^{n}\frac{2\pi a_{k}n!}{k!}}_{\in 2\pi\Z}+\frac{2\pi a_{n+1}}{n+1}+\underbrace{\sum_{k\geqslant n+2}\frac{2\pi a_{k}n!}{k!}}_{=~\varepsilon_{n}}\end{equation}
		On a 
		\begin{equation}0\leqslant \varepsilon_{n} < \frac{2\pi n!}{(n+1)!}=\frac{2\pi}{n+1}\xrightarrow[n\to+\infty]{}0\end{equation}
		Donc 
		\begin{equation}\sin(n!2\pi x)=\sin\Bigl(\frac{2\pi a_{n+1}}{n+1}+\varepsilon_{n}\Bigr)\end{equation}
		et il suffit d'avoir, comme $\varepsilon_{n}\xrightarrow[n\to+\infty]{}0$,
		\begin{equation}\frac{a_{n}}{n}\xrightarrow[n\to+\infty]{}\frac{\arcsin(l)}{2\pi}\in\Bigl[0,\frac{1}{4}\Bigr]\end{equation}
		On pose alors 
		\begin{equation}\boxed{a_{n}=\Biggl\lfloor\frac{n\arcsin(l)}{2\pi}\Biggr\rfloor}\end{equation}
		pour $n\geqslant2$ et on a $0\leqslant a_{n}\leqslant \frac{n}{4}<n-1$ pour tout $n\geqslant2$. On a donc le résultat.
	\end{enumerate}
\end{proof}

\begin{remark}
	Il n'y a pas unicité. Par exemple, pour $l=0$, $x=0$ ou $x=\frac{1}{2}$ convient. Plus généralement, pour tout $\frac{p}{q}\in\Q$, pour tout $n\geqslant q$, on a
	\begin{equation}\sin\Biggl(n!2\pi\Bigl(x+\frac{p}{q}\Bigr)\Biggr)=\sin(n!2\pi x)\end{equation}
\end{remark}

\begin{proof}
	Par récurrence, on a $u_{n}>0$ pour tout $n\in\N$.
	Soit \function{g}{\R_{+}}{\R}{x}{2\ln(1+x)-x}
	et \function{f}{\R_{+}}{\R}{x}{2\ln(1+x)}
	$g$ est dérivable est 
	\begin{equation}g'(x)=\frac{1-x}{1+x}\end{equation}
	donc $g$ est croissante sur $[0,1]$ et décroissante sur $[1,+\infty[$. Comme $g(0)=0$ et $\lim\limits_{x\to+\infty}g(x)=-\infty$, d'après le théorème des valeurs intermédiaires, il existe un unique réel $l\in]0,+\infty[$ tel que $g(l)=0$ d'où $f(l)=l$.

	\begin{figure}[ht!]
		\centering
		\begin{tikzpicture}
			\begin{axis}[
				xmin=-0.5, xmax=5,
				ymin=-0.5, ymax=5,
				axis lines=center,
				axis on top=true,
				xlabel=$x$,
				samples=100,
				legend pos=outer north east
			]
			\addplot[blue, ultra thick] {2*ln(1+x)};
			\addplot[color=red, ultra thick] {x};
			\addplot[green, dashed, domain=0:2.5128] {2.5128};
			\addplot[green, dashed, domain=0:2.5128] coordinates {(2.5128,0)(2.5128,2.5128)};
			\node[anchor=north, color=green] at (axis cs: 2.5128,0) {$l$};
			\legend{$f(x)$,$y=x$}
			\end{axis}
		\end{tikzpicture}
		\caption{$x\mapsto 2\ln(1+x)$ admet un unique point fixe sur $\R_{+}^{*}$.}
	\end{figure}

	Pour tout $x\in]0,l]$, on a $x\leqslant f(x)\leqslant l$ et pour tout $x>l$, on a $l\leqslant f(x)\leqslant x$.

	Soit $n\geqslant1$. Si $u_{n}\geqslant l$ et $u_{n-1}\geqslant l$, on a $m_{n}=l$ et $M_{n}\in\{u_{n},u_{n-1}\}$. Il vient donc 
	\begin{equation}u_{n+1}=\frac{1}{2}(f(u_{n})+f(u_{n-1}))\geqslant f(l)=l\end{equation}
	et 
	\begin{equation}u_{n+1}\leqslant\frac{1}{2}(u_{n}+u_{n-1})\leqslant M_{n}\end{equation}
	Donc $m_{n+1}=l=m_{n}$ et $M_{n+1}\leqslant M_{n}$.

	Par récurrence, on a pour tout $k\geqslant n$, $u_{k}\geqslant l$ et $(M_{k})_{k\geqslant n}$ converge vers $\lambda\geqslant l$ (car décroissante et plus grande que $l$) et $m_{k}=l$ pour tout $k\geqslant n$.

	De plus pour tout $k\geqslant n$, on a 
	\begin{equation}u_{k+1}=\frac{1}{2}(f(u_{k})+f(u_{k-1}))\leqslant f(M_{k})\end{equation}
	car $f$ est croissante et donc 
	\begin{equation}u_{k+2}\leqslant f(M_{k+1})\leqslant f(M_{k})\end{equation}
	Par passage à la limite, on a $\lambda\leqslant f(\lambda)$ donc $\lambda=f(\lambda)$ et donc $\lambda=l$. Or pout tout $k\geqslant n$, on a 
	\begin{equation}\underbrace{m_{k}}_{=~l}\leqslant u_{k}\leqslant M_{k}\xrightarrow[k\to+\infty]{}l\end{equation}
	donc 
	\begin{equation}\boxed{u_{k}\xrightarrow[k\to+\infty]{}l}\end{equation}

	S'il existe $n_{0}\in\N^{*}$ tel que $u_{n_{0}-1}\geqslant l$ et $u_{n_{0}}\geqslant l$ alors $\lim\limits_{n\to+\infty}u_{n}=l$. Or même s'il existe $n_{1}\in\N^{*}$ tel que $u_{n_{1}-1}\leqslant l$ et $u_{n_{1}}\leqslant l$, alors on inverse les rôles de $M_{n_{1}}$ et $m_{n_{1}}$.

	Si pour tout $n\in\N$,
	\begin{equation}(u_{n}-l)(u_{n+1}-l)\leqslant 0\end{equation}
	Supposons par exemple $u_{0}\geqslant l$ et $u_{1}\leqslant l$. Alors 
	\begin{equation}0\leqslant u_{2}-l\leqslant\frac{u_{0}-l}{2}\end{equation}
	et par récurrence, pour tout $k\in\N$, on a $0\leqslant u_{2k}-l\leqslant\frac{u_{0}-l}{2^{k}}$.
	Donc $u_{2k}\xrightarrow[k\to+\infty]{}l$ et de même $u_{2k+1}\xrightarrow[k\to+\infty]{}l$ (par valeurs inférieures). Donc 
	\begin{equation}\boxed{u_{k}\xrightarrow[k\to+\infty]{}l}\end{equation}
\end{proof}

\begin{proof}
	Soit $(\theta,\theta')\in[2,2\pi[^{2}$ tel que 
	\begin{equation}\lim\limits_{k\to+\infty}e^{\i px_{n}}=e^{\i \theta}\end{equation} et 
	\begin{equation}\lim\limits_{k\to+\infty}e^{\i qx_{n}}=e^{\i \theta'}\end{equation}
	Soient $x,x'$ deux valeurs d'adhérence de $(x_{n})_{n\in\N}$ distinctes. On a 
	\begin{equation}
	\left\{
		\begin{array}[]{l}
			e^{\i px}=e^{\i \theta}=e^{\i px'}\\
			e^{\i qx}=e^{\i \theta'}=e^{\i qx'}
		\end{array}
	\right.
	\end{equation}

	Il existe $(k,k')\in\Z^{2}$ tel que 
	\begin{equation}
	\left\{
		\begin{array}[]{l}
			px=px'+2k\pi\\
			qx=qx'+2k\pi
		\end{array}
	\right.
	\end{equation}
	et donc $p(x-x')=2k\pi$ et $q(x-x')=2k'\pi$ et alors $\frac{p}{q}\in\Q$ ce qui contredit l'hypothèse. Donc $(u_{n})_{n\in\N}$ possède une unique valeur d'adhérence. Comme elle est bornée,
	\begin{equation}\boxed{(x_{n})_{n\in\N}\text{ converge.}}\end{equation}

	Si $(x_{n})_{n\in\N}$ n'est pas bornée, on peut prendre 
	\begin{equation}\boxed{x_{n}=n!}\end{equation}
	On a 
	\begin{equation}e^{2\i \pi n!}=1\end{equation}
	et 
	\begin{equation}n!e=n!\sum_{k=0}^{+\infty}\frac{1}{k!}=\underbrace{\sum_{k=0}^{n}\frac{n!}{k!}}_{\in\N}+\underbrace{\sum_{k=n+1}^{+\infty}\frac{n!}{k!}}_{\xrightarrow[k\to+\infty]{}0}\end{equation}
	
	Si on veut $x_{n}$ divergente dans $\overline{\R}$, on peut prendre 
	\begin{equation}\boxed{x_{n}=(-1)^{n}n!}\end{equation}
\end{proof}

\begin{proof}
	\phantom{}
	\begin{enumerate}
		\item On a \begin{equation}\binom{n}{k}=\frac{n!}{k!(n-k)!}=\frac{n(n-1)\dots(n-k+1)}{k!}\leqslant\boxed{\frac{n^{k}}{k!}}\end{equation}
		\item On a 
		\begin{equation}\left(1+\frac{z}{n}\right)^{n}=\sum_{k=0}^{n}\binom{n}{k}\left(\frac{z}{n}\right)^{k}\end{equation}
		donc 
		\begin{align}
			\left|\sum_{k=0}^{n}\frac{z^{k}}{k!}-\binom{n}{k}\frac{z^{k}}{n^{k}}\right|
			&\leqslant\sum_{k=0}^{n}\vert z\vert^{k}\underbrace{\left|\frac{1}{k!}-\binom{n}{k}\frac{1}{n^{k}}\right|}_{\geqslant0}\\
			&\leqslant\sum_{k=0}^{n}\frac{\left|z\right|^{k}}{k!}-\sum_{k=0}^{n}\binom{k}{n}\frac{\left|z\right|^{k}}{n^{k}}\\
			&\boxed{=\sum_{k=0}^{n}\frac{\left|z\right|^{k}}{k!}-\left(1+\frac{\left|z\right|}{n}\right)^{n}}
		\end{align}

		\item On sait que 
		\begin{equation}\sum_{k=0}^{n}\frac{\left|z\right|^{k}}{k!}\xrightarrow[k\to+\infty]{}e^{\left|z\right|}\end{equation}
		et 
		\begin{equation}\left(1+\frac{\left|z\right|}{n}\right)^{n}=e^{n\ln\left(1+\frac{\left|z\right|}{n}\right)}=e^{n\left(\frac{\left|z\right|}{n}+o\left(\frac{\left|z\right|}{n}\right)\right)}=e^{\left|z\right|}e^{o\left(1\right)}\xrightarrow[n\to+\infty]{}e^{\left|z\right|}\end{equation}
		
		En reportant dans la question précédente, on a donc 
		\begin{equation}\boxed{\lim\limits_{n\to+\infty}\left(1+\frac{z}{n}\right)^{n}=\sum_{k=0}^{+\infty}\frac{z^{k}}{k!}=e^{z}}\end{equation}
	\end{enumerate}
\end{proof}

\begin{remark}
	Une autre méthode est d'écrire, pour $z=a+\i b$, 
	\begin{equation}1+\frac{z+\i b}{n}=1+\frac{a}{n}+\i\frac{b}{n}=\rho_{n}e^{\i\theta_{n}}\end{equation}.
	On a alors 
	\begin{equation}\left|1+\frac{a+\i b}{n}\right|=\sqrt{\left(1+\frac{a}{n}\right)^{2}+\frac{b^{2}}{n^{2}}}=\rho_{n}\end{equation}
	et alors 
	\begin{align}
		\rho_{n}^{n}
		&=\left|\left(1+\frac{z}{n}\right)\right|^{n}\\
		&=e^{\frac{n}{2}\ln\left(\left(1+\frac{a}{n}\right)^{2}+\frac{b^{2}}{n^{2}}\right)}\\
		&=e^{\frac{n}{2}\ln\left(1+\frac{2a}{n}+o\left(\frac{1}{n}\right)\right)}\\
		&=e^{a+o\left(1\right)}\xrightarrow[n\to+\infty]{}e^{a}=\left|e^{z}\right|
	\end{align}

	On écrit ensuite 
	\begin{equation}1+\frac{a+\i b}{n}=\rho_{n}\Bigl(\underbrace{\frac{1+\frac{a}{n}}{\rho_{n}}}_{=~\cos(\theta_{n})}+\i\underbrace{\frac{b}{n\rho_{n}}}_{=~\sin(\theta_{n})}\Bigr)\end{equation}
	On a alors
	\begin{equation}\lim\limits_{n\to+\infty}\frac{b}{n\rho_{n}}=0\text{ et }\lim\limits_{n\to+\infty}\frac{1+\frac{a}{n}}{\rho_{n}}=1\end{equation}
	On peut imposer $\theta_{n}\in]-\pi,\pi]$ et il existe alors $N\in\N$ tel que pour tout $n\geqslant N$, $\cos(\theta_{n})\geqslant0$. Pour $n\geqslant N$, on a alors $\theta_{n}\in\left[-\frac{\pi}{2},\frac{\pi}{2}\right]$ donc 
	\begin{equation}\theta_{n}=\arcsin\left(\frac{b}{n\rho_{n}}\right)\end{equation}
	et $n\theta_{n}=n\arcsin\left(\frac{b}{n\rho_{n}}\right)\underset{n\to+\infty}{\sim}b$. Finalement, on a bien 
	\begin{equation}\left(1+\frac{z}{n}\right)^{n}=\rho_{n}^{n}e^{\i\theta_{n}}\xrightarrow[n\to+\infty]{}e^{a}e^{\i b}=e^{z}\end{equation}
\end{remark}

\begin{proof}
	Pour tout $n\geqslant2$, $u_{n}>0$. On a 
	\begin{equation}u_{n+1}=\underbrace{\frac{\sqrt{n+1}-1}{\sqrt{n+1}+1}}_{<1}u_{n}\end{equation}
	donc $(u_{n})_{n\in\N}$ est décroissante donc converge. On a 
	\begin{equation}\ln(u_{n})=\sum_{k=2}^{n}\underbrace{\ln\Bigl(1-\frac{1}{\sqrt{k}}\Bigr)-\ln\Bigl(1+\frac{1}{\sqrt{k}}\Bigr)}_{=~v_{k}}<0\end{equation}
	Ensuite, 
	\begin{equation}v_{k}=-\frac{1}{\sqrt{k}}-\frac{1}{\sqrt{k}}+o\left(\frac{1}{\sqrt{k}}\right)\underset{k\to+\infty}{\sim}-\frac{2}{\sqrt{k}}\end{equation}

	Comme $\sum_{k\geqslant2}\frac{1}{\sqrt{k}}$ diverge, on a $\lim\limits_{n\to+\infty}\ln(u_{n})=-\infty$.

	Ainsi, 
	\begin{equation}\boxed{\lim\limits_{n\to+\infty}u_{n}=0}\end{equation}

	On a ensuite 
	\begin{equation}u_{n}=\exp\left(\sum_{k=2}^{n}\left[\ln\left(1-\frac{1}{\sqrt{k}}\right)-\ln\left(1+\frac{1}{\sqrt{k}}\right)\right]\right)\end{equation}
	et 
	\begin{equation}\ln\left(1\pm\frac{1}{\sqrt{k}}\right)=\pm\frac{1}{\sqrt{k}}-\frac{1}{2k}+O\left(\frac{1}{k^{\frac{3}{2}}}\right)\end{equation}
	Donc 
	\begin{equation}v_{k}=-\frac{2}{\sqrt{k}}+O\left(\frac{1}{k^{\frac{3}{2}}}\right)\end{equation}
	Le terme dans le $O$ est le terme générale d'une série absolument convergent donc convergent, on note ce terme $\alpha_{k}$. On a alors 
	\begin{equation}\sum_{k=2}^{n}v_{k}=\sum_{k=2}^{n}\left(-\frac{2}{\sqrt{k}}+\alpha_{k}\right)=-2\sum_{k=2}^{n}\frac{1}{\sqrt{k}}+\underbrace{\sum_{k=2}^{+\infty}\alpha_{k}}_{=~C}+o\left(1\right)\end{equation}
	Par comparaison série-intégrale, on a 
	\begin{equation}\sum_{k=2}^{n}\frac{1}{\sqrt{k}}\underset{k\to+\infty}{\sim}\int_{2}^{n}\frac{dt}{\sqrt{t}}\underset{k\to+\infty}{\sim}2\sqrt{n}\end{equation}

	Posons 
	\begin{equation}w_{n}=\sum_{k=2}^{n}\frac{1}{\sqrt{k}}-2\sqrt{n}\end{equation}
	On étudie la série de terme général $w_{n}-w_{n-1}$. On a 
	\begin{align}
		w_{n}-w_{n-1}
		&=\frac{1}{\sqrt{n}}-2\left(\sqrt{n}-\sqrt{n-1}\right)\\
		&=\frac{1}{\sqrt{n}}-2\left(1-\sqrt{1-\frac{1}{n}}\right)\\
		&=\frac{1}{\sqrt{n}}-2\left(1-\left(1-\frac{1}{2n}+O\left(\frac{1}{n^{2}}\right)\right)\right)\\
		&=\frac{1}{\sqrt{n}}-\frac{\sqrt{n}}{n}+O\left(\frac{1}{n^{\frac{3}{2}}}\right)\\
		&=O\left(\frac{1}{n^{\frac{3}{2}}}\right)
	\end{align}

	Donc la série de terme général $w_{n}-w_{n-1}$ converge et ainsi $(w_{n})_{n\geqslant2}$ converge: il existe $C'\in\R$ tel que 
	\begin{equation}\sum_{k=2}^{n}\frac{1}{\sqrt{n}}=2\sqrt{n}+C'+o\left(1\right)\end{equation}

	On a donc 
	\begin{equation}\ln(u_{n})=\sum_{k=2}^{n}v_{k}=-4\sqrt{n}-2C'+C+o\left(1\right)\end{equation}
	Ainsi, 
	\begin{equation}u_{n}=\exp\left(-4\sqrt{n}-2C'+C+o\left(1\right)\right)\underset{n\to+\infty}{\sim}Ke^{-4\sqrt{n}}\end{equation}
	où $K=e^{-2C'+C}>0$.

	Donc 
	\begin{equation}u_{n}^{\alpha}\underset{n\to+\infty}{\sim}K^{\alpha}e^{-4\alpha\sqrt{n}}\end{equation}

	Si $\alpha\leqslant0$, $\lim\limits_{n\to+\infty}u_{n}^{\alpha}\not\to 0$
	donc 
	\begin{equation}\boxed{\sum u_{n}^{\alpha}\text{ diverge.}}\end{equation}

	Si $\alpha>0$, $u_{n}^{\alpha}=o\left(\frac{1}{n^{2}}\right)$ donc d'après le critère de Riemann, 
	\begin{equation}\boxed{\sum u_{n}^{\alpha}\text{ converge.}}\end{equation}
\end{proof}

\begin{proof}
	Soit $S_{n}=\sum_{k=0}^{n}u_{k}$. On a 
	\begin{equation}u_{n+1}+\dots+u_{2n}\geqslant nu_{2n}\geqslant0\end{equation}

	Si $(S_{n})$ converge alors $S_{2n}-S_{n}\xrightarrow[n\to+\infty]{}0$. Alors $\lim\limits_{n\to+\infty}nu_{2n}=0$ et $\lim\limits_{n\to+\infty}2n u_{2n}=0$.

	Comme on a $(2n+1)u_{2n}\geqslant (2n+1)u_{2n+1}\geqslant0$, on a aussi $\lim\limits_{n\to+\infty}(2n+1) u_{2n}=0$. Finalement, on a bien 
	\begin{equation}\boxed{\lim\limits_{n\to+\infty}nu_{n}=0\text{ et donc }u_{n}=o\left(\frac{1}{n}\right)}\end{equation}

	Si $\left\{p\in\N \middle|pu_{p}\geqslant1 \right\}$ est infini, alors $u_{p}\neq o\left(\frac{1}{p}\right)$ donc 
	\begin{equation}\boxed{\sum u_{p}\text{ diverge.}}\end{equation}
\end{proof}

\begin{remark}
	Ce n'est pas vrai si $(u_{n})_{n\in\N}$ n'est pas décroissante, par exemple si $u_{n}=\frac{1}{n}$ si $n$ est un carré et 0 sinon.
\end{remark}

\begin{proof}
	\begin{enumerate}
		\item C'est une série à termes positifs. On a 
		\begin{equation}n^{\frac{1}{n}}=e^{\frac{1}{n}\ln\left(n\right)}\xrightarrow[n\to+\infty]{}1\end{equation}
		Ainsi
		\begin{equation}n^{-1-\frac{1}{n}}\underset{n\to+\infty}{\sim}\frac{1}{n}\end{equation}
		et donc 
		\begin{equation}\boxed{\sum u_{n}\text{ diverge.}}\end{equation}

		\item C'est une série à termes positifs. On a 
		\begin{equation}u_{n}\geqslant \int_{1}^{\frac{\pi}{2}}t^{n}\sin(1)dt=\frac{\sin(1)}{n+1}\times\left(\left(\frac{\pi}{2}\right)^{n+1}-1\right)\xrightarrow[n\to+\infty]{}+\infty\end{equation}
		donc 
		\begin{equation}\boxed{\sum u_{n}\text{ diverge grossièrement.}}\end{equation}

		\item On écrit 
		\begin{equation}\frac{n!}{e}=\sum_{k=0}^{+\infty}\frac{n!}{k!}(-1)^{k}=\underbrace{\sum_{k=0}^{n}\frac{n!}{k!}(-1)^{k}}_{\in\Z}+\frac{(-1)^{n+1}}{n+1}+\sum_{k=n+2}^{+\infty}\frac{n!}{k!}(-1)^{k}\end{equation}
		et
		\begin{equation}\left\vert\sum_{k=n+2}^{+\infty}\frac{n!}{k!}(-1)^{k}\right\vert<\frac{1}{(n+1)(n+2)}\end{equation}
		d'après le critère spécial des séries alternées.

		Donc 
		\begin{align}
			\sin\left(2\pi\frac{n!}{e}\right)
			&=\sin\left(\frac{2\pi(-1)^{n+1}}{n+1}+O\left(\frac{1}{n^{2}}\right)\right)\\
			&=\underbrace{\frac{2\pi(-1)^{n+1}}{n+1}}_{\substack{\text{terme général}\\\text{d'une série alternée}\\\text{convergente}}}+\underbrace{O\left(\frac{1}{n^{2}}\right)}_{\substack{\text{terme général}\\\text{d'une série absolument}\\\text{convergente}}}
		\end{align}
		
		Donc
		\begin{equation}\boxed{\sum u_{n}\text{ converge.}}\end{equation}

		\item Si $\alpha\leqslant0$, $u_{n}\underset{n\to+\infty}{\sim}\frac{1}{\ln(n)}$ et comme $\frac{1}{n}=O\left(\frac{1}{\ln(n)}\right)$, 
		\begin{equation}\boxed{\sum u_{n}\text{ diverge.}}\end{equation}

		Si $\alpha>1$, $\vert u_{n}\vert\underset{n\to+\infty}{\sim}\frac{1}{n^{\alpha}}$ donc d'après le critère de Riemann,
		\begin{equation}\boxed{\sum u_{n}\text{ converge absolument donc converge.}}\end{equation}

		Si $\alpha\in]0,1]$, on écrit 
		\begin{align}
			u_{n}
			&=\frac{(-1)^{n}}{n^{\alpha}}\times\frac{1}{1+\underbrace{(-1)^{n}\frac{\ln(n)}{n^{\alpha}}}_{\xrightarrow[n\to+\infty]{}0}}\\
			&=\frac{(-1)^{n}}{n^{\alpha}}\left(1-(-1)^{n}\frac{\ln(n)}{n^{\alpha}}+o\left(\frac{\ln(n)}{n^{\alpha}}\right)\right)\\
			&=\underbrace{\frac{(-1)^{n}}{n^{\alpha}}}_{\substack{\text{terme général}\\\text{d'une série alternée}\\\text{convergente}}}\underbrace{-\frac{\ln(n)}{n^{2\alpha}}+o\left(\frac{\ln(n)}{n^{2\alpha}}\right)}_{\substack{\underset{n\to+\infty}{\sim}-\frac{\ln(n)}{n^{2\alpha}}<0\\\text{terme général}\\\text{d'une série convergente}\\\text{ssi }\alpha>\frac{1}{2}}}
		\end{align}

		\begin{equation}\boxed{\sum u_{n}\text{ converge si et seulement si }\alpha>\frac{1}{2}\text{.}}\end{equation}
	\end{enumerate}
\end{proof}

\begin{remark}
	Soit $\alpha\in[0,1]$ et 
	\begin{equation}u_{n}=\int_{0}^{\alpha}t^{n}\sin(t)dt\geqslant0\end{equation}
	Si $\alpha<1$, $u_{n}\leqslant\alpha^{n+1}$, terme général d'une série convergente donc $\sum u_{n}$ converge.

	Si $\alpha=1$, on utilise
	\begin{equation}\forall t\in\left[0,\frac{\pi}{2}\right],\sin(t)\geqslant\frac{2}{\pi}t\end{equation}
	Alors $u_{n}\geqslant\frac{2}{\pi (n+2)}$, terme générale d'une série divergente donc $\sum u_{n}$ diverge.
\end{remark}

\begin{figure}[ht!]
	\centering
	\begin{tikzpicture}
		\begin{axis}[
			xmin=-0.5, xmax=2,
			ymin=-0.5, ymax=2,
			axis lines=center,
			axis on top=true,
			xlabel=$t$,
			samples=100,
			legend pos=outer north east
		]
		\addplot[blue, ultra thick] {sin(deg(x))};
		\addplot[color=red, ultra thick, restrict y to domain=-0.5:2] {2*x/pi};
		\node[anchor=south west, color=green] at (axis cs: 1.5708,0) {$\frac{\pi}{2}$};
		\addplot[green, dashed, domain=0:2] coordinates {(1.5708,0)(1.5708,1)};
		\legend{$\sin(t)$,$\frac{2}{\pi}t$}
		\end{axis}
	\end{tikzpicture}
	\caption{$\sin(t)\geqslant\frac{2}{\pi}t$ pour $t\in\left[0,\frac{\pi}{2}\right]$.}
\end{figure}

\begin{proof}
	\item On a 
	\begin{equation}u_{n}=\sum_{k=n}^{+\infty}\frac{(-1)^{k}}{k}\end{equation}
	$u_{n}$ est le reste d'ordre $n$ d'une série alternée, donc $u_{n}$ est du signe de $\frac{(-1)^{n}}{n}$. Donc on a 
	\begin{equation}u_{n+1}\times u_{n}\leqslant0\end{equation}
	Par ailleurs,
	\begin{equation}\vert u_{n}\vert=\underbrace{\frac{1}{n}-\frac{1}{n+1}}_{=~\frac{1}{n(n+1)}}+\underbrace{\frac{1}{n+2}-\frac{1}{n+3}}_{=~\frac{1}{(n+2)(n+3)}}+\dots=\sum_{p=0}^{+\infty}\frac{1}{(n+2p)(n+2p+1)}\end{equation}
	Donc $(\vert u_{n}\vert)_{n\geqslant1}$ est décroissante.

	D'après le critère des séries alternées, 
	\begin{equation}\boxed{\sum u_{n}\text{ converge.}}\end{equation}

	Pour calculer la somme, on peut chercher si la famille $(u_{n,p})_{\substack{n\geqslant1\\p\in\N}}$ est sommable où
	\begin{equation}u_{n,p}=\frac{(-1)^{n}}{(n+2p)(n+2p+1)}\end{equation} 
	Soit $p\geqslant0$, on a 
	\begin{equation}\sum_{n=1}^{+\infty}\frac{1}{(n+2p)(n+2p+1)}=\sum_{n=1}^{+\infty}\left(\frac{1}{n+2p}-\frac{1}{n+2p+1}\right)=\frac{1}{2p+1}\end{equation}
	Donc 
	\begin{equation}\sum_{p\in\N}\sum_{n\geqslant1}\vert u_{n,p}\vert=+\infty\end{equation}
	Ainsi, cette famille n'est pas sommable. Essayons plutôt de calculer $u_{n}$ d'abord: soit $n\geqslant1$ fixé et $N\geqslant n$. On a 
	\begin{align}
		\sum_{k=n}^{N}\frac{(-1)^{k}}{k}
		&=\sum_{k=n}^{N}(-1)^{k}\int_{0}^{1}t^{k-1}dt\\
		&=-\sum_{k=n}^{N}\int_{0}^{1}(-t)^{k-1}dt\\
		&=-\int_{0}^{1}\sum_{k=n}^{N}(-t)^{k-1}dt\\
		&=\int_{0}^{1}(-t)^{n}\frac{1-(-t)^{N-n+1}}{1+t}dt
	\end{align}

	Ainsi, 
	\begin{equation}\sum_{k=n}^{N}\frac{(-1)^{k}}{k}=-\int_{0}^{1}\frac{(-t)^{n}}{1+t}dt+\int_{0}^{1}\frac{(-t)^{N+1}}{1+t}dt\end{equation}
	et 
	\begin{equation}\left\vert\int_{0}^{1}\frac{(-t)^{N+1}}{1+t}dt\right\vert\leqslant\int_{0}^{1}t^{N+1}dt=\frac{1}{N+2}\end{equation}

	Donc 
	\begin{equation}u_{n}=-\int_{0}^{1}\frac{(-t)^{n}}{1+t}dt\end{equation}
	Soit alors $M\geqslant1$. On a 
	\begin{align}
		\sum_{n=1}^{M}u_{n}
		&=\sum_{n=1}^{M}\left(-\int_{0}^{1}\frac{(-t)^{n}}{t+1}dt\right)\\
		&=-\int_{0}^{1}\frac{1}{1+t}\sum_{n=1}^{M}(-t)^{n}dt\\
		&=-\int_{0}^{1}\frac{-t}{1+t}\frac{1-(-t)^{M}}{1+t}dt\\
		&=\int_{0}^{1}\frac{t}{(1+t)^{2}}dt+\int_{0}^{1}\frac{(-t)^{M+1}}{(1+t)^{2}}dt
	\end{align}
	Comme 
	\begin{equation}\left\vert\int_{0}^{1}\frac{(-t)^{M+1}}{(1+t)^{2}}dt\right\vert\leqslant\int_{0}^{1}t^{M+1}dt=\frac{1}{M+2}\xrightarrow[M\to+\infty]{}0\end{equation}
	on a 
	\begin{align}
		\sum_{n=1}^{+\infty}u_{n}
		&=\int_{0}^{1}\frac{t}{(1+t)^{2}}dt\\
		&=\int_{0}^{1}\frac{(t+1)-1}{(1+t)^{2}}dt\\
		&=\int_{0}^{1}\frac{1}{1+t}dt-\int_{0}^{1}\frac{1}{(1+t)^{2}}dt\\
		&=\left[\ln\left(1+t\right)\right]_{0}^{1}+\left[\frac{1}{2}-1\right]\\
		&=\ln(2)-\frac{1}{2}
	\end{align}

	Finalement, 
	\begin{equation}\boxed{\sum_{n=1}^{+\infty} u_{n}=\ln(2)-\frac{1}{2}}\end{equation}

	\item \begin{equation}\frac{1}{(3n)!}=\left(\frac{1}{n^{2}}\right)\end{equation}
	donc d'après le critère de Riemann, 
	\begin{equation}\boxed{\sum u_{n}\text{ converge.}}\end{equation}
	Posons 
	\begin{equation}
	\left\{
		\begin{array}[]{rcl}
			S_{0} &= &\sum_{n=0}^{+\infty}\frac{1}{(3n)!}\\
			S_{1} &= &\sum_{n=0}^{+\infty}\frac{1}{(3n+1)!}\\
			S_{2} &= &\sum_{n=0}^{+\infty}\frac{1}{(3n+2)!}
		\end{array}
	\right.
	\end{equation}
	On a 
	\begin{equation}
	\left\{
		\begin{array}[]{rcl}
			S_{0} + S_{1} + S_{2} &= & e\\
			S_{0} + jS_{1} + j^{2}S_{2} &= &\exp(j)\\
			S_{0} + j^{2}S_{1} + jS_{2} &= &\exp(j^{2})
		\end{array}
	\right.
	\end{equation}
	où $j=\exp\left(\frac{2\i \pi}{3}\right)$.
	En sommant les trois lignes, on a 
	\begin{equation}3S_{0}=e+\exp(j)+\exp(j^{2})=e+e^{-\frac{1}{2}}\left(2\cos\left(\frac{\sqrt{3}}{2}\right)\right)\end{equation}

	Donc 
	\begin{equation}\boxed{\sum_{n=0}^{+\infty} u_{n}=\frac{1}{3}\left(e+e^{-\frac{1}{2}}\left(2\cos\left(\frac{\sqrt{3}}{2}\right)\right)\right)}\end{equation}

	\item S'il existe $p\geqslant0$ tel que $n=p^{3}$, alors 
	\begin{equation}\left\lfloor n^{\frac{1}{3}}\right\rfloor=p\end{equation}
	et 
	\begin{equation}\left\lfloor \left(n-1\right)^{\frac{1}{3}}\right\rfloor=\left\lfloor \left(p^{3}-1\right)^{\frac{1}{3}}\right\rfloor=p-1\end{equation}

	Sinon, $n^{\frac{1}{3}}\notin\N$. Soit $k=\left\lfloor n^{\frac{1}{3}}\right\rfloor$. Alors $k^{3}<n\leqslant (k+1)^{3}$ donc $k^{3}\leqslant n-1<(k+1)^{3}$ d'où $k\leqslant (n-1)^{\frac{1}{3}}<k+1$. Donc $\left\lfloor(n-1)^{\frac{1}{3}}\right\rfloor=k$.

	Donc $\sum u_{n}$ est une série lacunaire. Comme $u_{p^{3}}=O\left(\frac{1}{p^{3}}\right)$, d'après le critère de Riemann,
	\begin{equation}\boxed{\sum u_{n}\text{ converge.}}\end{equation}

	Sa somme vaut 
	\begin{equation}\sum_{n=1}^{+\infty}u_{n}=\sum_{p=1}^{+\infty}\frac{1}{4p^{3}-p}\end{equation}
	On décompose en éléments simples:
	\begin{equation}\frac{1}{4x^{3}-x}=\frac{1}{x(4x^{2}-1)}=\frac{-1}{x}+\frac{1}{2x-1}+\frac{1}{2x+1}\end{equation}

	Donc la somme partielle jusqu'au rang $n$ vaut
	\begin{align}
		S_{n}
		&=-\underbrace{\sum_{p=1}^{n}\frac{1}{p}}_{=~H_{n}}+\sum_{p=1}^{n}\frac{1}{2p-1}+\sum_{p=1}^{n}\frac{1}{2p+1}\\
		&=-H_{n}+1+\frac{1}{2n+1}+2\sum_{p=1}^{n-1}\frac{1}{2p+1}\\
		&=-H_{n}+1+\frac{1}{2n+1}+2\left(\underbrace{\sum_{k=1}^{2n-1}\frac{1}{k}}_{=~H_{2n-1}}-1-\underbrace{\sum_{k=1}^{n-1}\frac{1}{2k}}_{=~\frac{1}{2}H_{n-1}}\right)\\
		&=-H_{n}+2H_{2n-1}-H_{n-1}-1+\frac{1}{2n+1}\\
		&=-\ln(n)+2\ln(2n-1)-\ln(n-1)-1+\underbrace{\frac{1}{2n+1}}_{=~o\left(1\right)}+o\left(1\right)\\
		&=\ln\left(\frac{(2n-1)^{2}}{n(n-1)}\right)-1+o\left(1\right)\xrightarrow[n\to+\infty]{}\ln(4)-1
	\end{align}

	Donc 
	\begin{equation}\boxed{\sum_{n=1}^{+\infty}u_{n}=\ln(4)-1}\end{equation}
\end{proof}

\begin{proof}
	Soit $\varepsilon>0$ tel que $a+\varepsilon<0$. Il existe $A>0$ tel que pour tout $x>A$, 
	\begin{equation}a-\varepsilon\leqslant\frac{f'(x)}{f(x)}\leqslant a+\varepsilon\end{equation}
	Alors
	\begin{equation}(a-\varepsilon)f(x)\leqslant f'(x)\leqslant (+\varepsilon)f(x)\end{equation}

	On voit donc que 
	\begin{equation}f'(x)-f(x)(a+\varepsilon)\leqslant0\end{equation}
	On pose alors (sorte d'inéquation différentielle) 
	\begin{equation}g_{1}(x)=f(x)e^{-(a+\varepsilon)x}\end{equation}
	On a 
	\begin{equation}g_{1}'(x)=e^{-(a+\varepsilon)x}\left(f'(x)-f(x)(a+\varepsilon)\right)\leqslant 0\end{equation}
	pour tout $x\geqslant A$. Donc $g_{1}$ est décroissante sur $[A,+\infty[$. Alors 
	\begin{equation}0<g_{1}(x)\leqslant g_{1}(A)=f(A)e^{-(a+\varepsilon)A}\end{equation}
	Alors 
	\begin{equation}0<f(x)\leqslant \left(f(A)e^{-(a+\varepsilon)A}\right)e^{(a+\varepsilon)x}\end{equation}

	De même, pour $x\geqslant A$, 
	\begin{equation}\left(f(A)e^{-(a+\varepsilon)A}\right)e^{(a-\varepsilon)x}\leqslant f(x)\end{equation}
	car $g_{2}(x)=f(x)e^{-(a-\varepsilon)x}$ est croissante sur $[A,+\infty[$.

	Donc 
	\begin{equation}f(n)\leqslant\left(f(A)e^{-(a+\varepsilon)A}\right)e^{(a+\varepsilon)n}\end{equation}
	Comme $a+\varepsilon<0$, 
	\begin{equation}\boxed{\sum_{n\geqslant1}f(n)\text{ converge.}}\end{equation}
	
	De plus
	\begin{equation}f(A)e^{-(a-\varepsilon)A}\frac{e^{(a-\varepsilon)N}}{1-e^{a-\varepsilon}}\leqslant R_{N}=\sum_{n=N}^{+\infty}f(n)\leqslant f(A)e^{-(a+\varepsilon)A}\frac{e^{(a+\varepsilon)N}}{1+e^{a+\varepsilon}}\end{equation}

	Donc 
	\begin{equation}\boxed{R_{N}=\underset{n\to+\infty}{O}\left(e^{aN}\right)\text{ et }e^{aN}=\underset{n\to+\infty}{O}\left(R_{N}\right)}\end{equation}
\end{proof}

\begin{proof}
	On a 
	\begin{equation}S_{n}=\sum_{k=1}^{n}\underbrace{\frac{e^{k}}{k}}_{\xrightarrow[k\to+\infty]{}+\infty}\xrightarrow[n\to+\infty]{}+\infty\end{equation}
	On utilise la règle d'Abel: on écrit $e^{k}=B_{k}-B_{k-1}$ avec 
	\begin{equation}
	\left\{
		\begin{array}[]{ll}
			B_{k}=\sum_{j=0}^{k}e^{j}=\frac{e^{k+1}-1}{e-1}\\
			B_{-1}=0
		\end{array}	
	\right.
	\end{equation}
	Alors 
	\begin{equation}S_{n}=\sum_{k=1}^{n}\frac{B_{k}}{k}-\sum_{k=0}^{n-1}\frac{B_{k}}{k+1}=\underbrace{-1+\sum_{k=1}^{n-1}\underbrace{\frac{B_{k}}{k(k+1)}}_{\underset{k\to+\infty}{\sim}\frac{e^{k+1}}{(e-1)k^{2}}}}_{=~\underset{n\to+\infty}{o}(S_{n})}+\underbrace{\frac{B_{n}}{n}}_{\underset{n\to+\infty}{\sim}\frac{e^{n}e}{n(e-1)}}\end{equation}

	Donc 
	\begin{equation}\boxed{S_{n}\underset{n\to+\infty}{\sim}\frac{e^{n+1}}{n(e-1)}}\end{equation}
\end{proof}

\begin{proof}
	\phantom{}
	\begin{enumerate}
		\item $u_{n}>0$ et 
		\begin{equation}u_{n}=e^{n^{\alpha}\ln\left(1-\frac{1}{n}\right)}=e^{n^{\alpha}\left(-\frac{1}{n}+O\left(\frac{1}{n^{2}}\right)\right)}=e^{-n^{\alpha-1}}+O\left(n^{\alpha-2}\right)\end{equation}

		Si $\alpha<1$, $u_{n}\xrightarrow[n\to+\infty]{}1$ donc 
		\begin{equation}\boxed{\sum u_{n}\text{ diverge grossièrement.}}\end{equation}

		Si $\alpha=1$, $u_{n}\xrightarrow[n\to+\infty]{}\frac{1}{2}$ donc 
		\begin{equation}\boxed{\sum u_{n}\text{ diverge grossièrement.}}\end{equation}

		Si $\alpha>1$, on a 
		\begin{equation}-n^{\alpha-1}+O\left(n^{\alpha-2}\right)\underset{n\to+\infty}{\sim}-n^{\alpha-1}\end{equation}
		donc il existe $N_{0}\in\N$ tel que pour tout $n\geqslant N_{0}$,
		\begin{equation}-n^{\alpha-1}+O\left(n^{\alpha-1}\right)\leqslant\frac{-n^{\alpha-1}}{2}\end{equation}
		d'où 
		\begin{equation}u_{n}\leqslant e^{-\frac{n^{\alpha-1}}{2}}=\underset{n\to+\infty}{o}\left(\frac{1}{n^{2}}\right)\end{equation}
		donc 
		\begin{equation}\boxed{\sum u_{n}\text{ converge.}}\end{equation}

		\item On a $u_{n}>0$ et 
		\begin{equation}\left(\frac{1}{k}\right)^{\frac{1}{k}}e^{-\frac{1}{k}\ln(k)}\xrightarrow[k\to+\infty]{}1\end{equation}
		donc par comparaison des sommes partielles, on a 
		\begin{equation}\sum_{k=1}^{n}\left(\frac{1}{k}\right)^{\frac{1}{k}}\underset{n\to+\infty}{\sim}n\end{equation}
		Donc $u_{n}\underset{n\to+\infty}{\sim}\frac{1}{n}$ et 
		\begin{equation}\boxed{\sum u_{n}\text{ diverge.}}\end{equation}

		\item On écrit $n!e=\left\lfloor n!e\right\rfloor+\alpha_{n}$.
		Alors 
		\begin{equation}\sin(n!\pi e)=(-1)^{\left\lfloor n!e\right\rfloor}\sin(\alpha_{n}\pi)\end{equation}
		On écrit 
		\begin{equation}n!e=\sum_{k=0}^{n-2}\frac{n!}{k!}+n+1+\frac{1}{n+1}+\sum_{k=n+1}^{+\infty}\frac{n!}{k!}\end{equation}

		On pose $v_{n}=\sum_{k=0}^{n}\frac{1}{k!}$ et $w_{n}=v_{n}+\frac{1}{n\times n!}$. On a 
		\begin{equation}v_{n}\leqslant e\leqslant w_{n}\end{equation}
		donc 
		\begin{equation}0\leqslant e-v_{n}\leqslant\frac{1}{n\times n!}\end{equation}
		d'où
		\begin{equation}0\leqslant \sum_{k=n+1}^{+\infty}\frac{n!}{k!}\leqslant\frac{n!}{(n+1)(n+1)!}=\frac{1}{(n+1)^{2}}\end{equation}
		Donc 
		\begin{equation}n!e\pi = \underbrace{\sum_{k=0}^{n-2}\frac{n!}{k!}}_{\text{pair}}\pi+(n+1)\pi+\frac{\pi}{n+1}+O\left(\frac{1}{n^{2}}\right)\end{equation}

		Finalement,a
		\begin{equation}\frac{\sin(n!e\pi)}{\ln(n)}=(-1)^{n+1}\frac{\sin\left(\frac{\pi}{n+1}+O\left(\frac{1}{n^{2}}\right)\right)}{\ln(n)}=\underbrace{\frac{(-1)^{n+1}\pi}{\ln(n)(n+1)}}_{\substack{\text{terme général}\\\text{d'une série alternée}\\\text{convergente}}}+\underbrace{O\left(\frac{1}{n^{2}\ln(n)}\right)}_{\substack{\text{terme général}\\\text{d'une série absolument}\\\text{convergente}}}\end{equation}

		Donc 
		\begin{equation}\boxed{\sum u_{n}\text{ converge.}}\end{equation}
	\end{enumerate}
\end{proof}

\begin{proof}
	\phantom{}
	\begin{enumerate}
		\item On a 
		\begin{equation}u_{n}=(a+b+c)\ln(n)+b\ln\left(1+\frac{1}{n}\right)+c\ln\left(1+\frac{2}{n}\right)=(a+b+c)\ln(n)+\frac{b+2c}{n}+O\left(\frac{1}{n^{2}}\right)\end{equation}
		Donc $\sum u_{n}$ converge si et seulement si 
		\begin{equation}
		\left\{
			\begin{array}[]{lcr}
				a+b+c &= &0\\
				b+2c &= &0
			\end{array}
		\right.
		\end{equation}
		si et seulement si 
		\begin{equation}
		\left\{
			\begin{array}[]{lcr}
				a &= &c\\
				b &= &-2c
			\end{array}
		\right.
		\end{equation}

		Donc 
		\begin{equation}\boxed{\sum u_{n}\text{ converge si et seulement si} a=b\text{ et }b=-2c\text{ avec }c\in\R}\end{equation}

		Prenons $c=1$ pour calculer la somme. On a 
		\begin{align}
			\sum_{n=1}^{N}u_{n}
			&=\sum_{n=1}^{N}\ln(n)-2\ln(n+1)+\ln(n+2)\\
			&=\sum_{n=1}^{N}\ln(n)-\ln(n+1)+\sum_{n=1}^{N}\ln(n+2)-\ln(n+1)\\
			&=-\ln(N+1)-\ln(2)+\ln(N+2)\\
			&=\ln\left(\frac{N+2}{N+1}\right)-\ln(2)\xrightarrow[n\to+\infty]{}-\ln(2)
		\end{align}

		Donc 
		\begin{equation}\boxed{\sum_{n=1}^{+\infty}u_{n}=-\ln(2)}\end{equation}

		\item On a $u_{n}=\underset{n\to+\infty}{O}\left(\frac{1}{n^{2}}\right)$ donc d'après le critère de Riemann,
		\begin{equation}\boxed{\sum u_{n}\text{ converge.}}\end{equation}

		On écrit 
		\begin{equation}u_{n}=\frac{2^{n}\left(3^{2^{n-1}}-1\right)}{\left(3^{2^{-1}}+1\right)\left(3^{2^{n}}-1\right)}=\frac{2^{n}\left(3^{2^{n-1}}+1-2\right)}{3^{2^{n}}-1}=\underbrace{\frac{2^{n}}{3^{2^{n-1}}-1}}_{=~v_{n}}-\underbrace{\frac{2^{n+1}}{3^{2^{n}}-1}}_{=~v_{n+1}}\end{equation}

		Donc 
		\begin{equation}\boxed{\sum_{n=1}^{+\infty}u_{n}=v_{1}=1}\end{equation}

		\item On remarque que $k-n\left\lfloor\frac{k}{n}\right\rfloor$ est le reste de la division euclidienne de $k$ par $n$. Donc ce reste est borné par $k-1$. Donc $u_{n}=\underset{n\to+\infty}{O}\left(\frac{1}{n^{2}}\right)$. D'après le critère de Riemann,
		\begin{equation}\boxed{\sum u_{n}\text{ converge.}}\end{equation}

		On note alors 
		\begin{equation}J_{r}=\left\{n\in\N^{*}\middle| n\equiv r[k]\right\}\end{equation}
		$(J_{r})_{r\in\{0,\dots,k-1\}}$ forme une partition de $\N^{*}$. On a 
		\begin{equation}\sum_{n\in J_{r}}\frac{r}{n(n+1)}=0\end{equation}
		si $r=0$. Si $r\in\{1,\dots,k-1\}$, on a 
		\begin{equation}S_{r}=r\sum_{p=0}^{+\infty}\frac{1}{(kp+r)(kp+r+1)}\end{equation}
		et par sommabilité on a 
		\begin{equation}S=\sum_{k=1}^{+\infty}\frac{n-k\left\lfloor\frac{n}{k}\right\rfloor}{n(n+1)}=\sum_{r=1}^{k-1}S_{r}=\sum_{p=0}^{+\infty}\sum_{r=1}^{k-1}\frac{1}{(kp+r)(kp+r+1)}\end{equation}

		Soit $p\in\N$ fixé. On a 
		\begin{align}
			v_{p}
			&=\sum_{r=1}^{k-1}\frac{r}{(kp+r)(kp+r+1)}\\
			&=\sum_{r=1}^{k-1}\frac{r}{kp+r}-\sum_{r=1}^{k-1}\frac{r}{kp+r+1}\\
			&=\sum_{r=1}^{k-1}\frac{r}{kp+r}-\sum_{r=2}^{k}\frac{r-1}{kp+r}\\
			&=\frac{1}{kp+1}+\sum_{r=2}^{k-1}\frac{1}{kp+r}-\frac{k-1}{k(p+1)}\\
			&=\sum_{r=1}^{k}\frac{1}{kp+r}-\frac{1}{p+1}
		\end{align}

		Ainsi, 
		\begin{equation}\sum_{p=0}^{N}v_{p}=\sum_{n=1}^{k(N+1)}\frac{1}{n}-\sum_{n=1}^{N+1}\frac{1}{n}=\ln\left(\frac{k(N+1)}{N+1}\right)+\underset{n\to+\infty}{o}\left(1\right)=\ln(k)+\underset{n\to+\infty}{o}\left(1\right)\end{equation}

		Donc 
		\begin{equation}\boxed{\sum_{n=1}^{+\infty}u_{n}=\ln(k)}\end{equation}

		\item On a 
		\begin{equation}\arctan(u)+\arctan(v)=\arctan\left(\frac{u+v}{1-uv}\right)\end{equation}
		donc \begin{equation}\arctan\left(\frac{1}{n^{2}+n+1}\right)=\arctan(n+1)-\arctan(n)\end{equation}
		Ainsi,
		\begin{equation}\boxed{\sum_{n=0}^{+\infty}\arctan\left(\frac{1}{n^{2}+n+1}\right)=\frac{\pi}{2}}\end{equation}
	\end{enumerate}
\end{proof}

\begin{proof}
	On a 
	\begin{equation}\sum_{k=1}^{n}v_{k}=\sum_{k=1}^{n}ku_{k}-\sum_{k=2}^{n+1}(k-1)u_{k}=u_{1}-nu_{n+1}+\sum_{k=2}^{n}u_{k}=\sum_{k=1}^{n}u_{k}-nu_{n+1}\end{equation}

	Si $(nu_{n})_{n\geqslant1}$, on a donc évidemment d'après ce qui précède
	\begin{equation}\boxed{\sum_{k=1}^{+\infty}v_{k}=\sum_{k=1}^{+\infty}u_{k}}\end{equation}

	Si $(u_{n})_{n\geqslant1}$ décroît, $v_{n}\geqslant0$ et on a 
	\begin{equation}\frac{v_{k}}{k}=u_{k}-u_{k+1}\end{equation}
	et donc 
	\begin{equation}\sum_{k=n}^{+\infty}\frac{v_{k}}{k}=u_{n}=\sum_{k=1}^{+\infty}w_{k,n}\end{equation}
	en définissant $w_{n,k}=\frac{v_{k}}{k}$ si $k\geqslant n$ et 0 sinon. On a $w_{k,n}\geqslant0$ car $(u_{n})_{n\geqslant1}$ est décroissante.

	Ainsi, $\sum_{n\geqslant1}u_{n}$ converge si et seulement si $(u_{n})_{n\in\N^{*}}$ sommable si et seulement si $(w_{n,k})_{k\in\N^{*}}$ si et seulement si (d'après le théorème de Fubini) 
	\begin{equation}\sum_{k=1}^{+\infty}\underbrace{\sum_{n=1}^{+\infty}w_{n,k}}_{\sum_{n=1}^{k}\frac{v_{k}}{k}=v_{k}}<+\infty\end{equation}

	Et dans ce cas (toujours d'après le théorème de Fubini), 
	\begin{equation}\sum_{n=1}^{+\infty}\underbrace{\sum_{k=1}^{+\infty}w_{n,k}}_{=~u_{n}}=\sum_{k=1}^{+\infty}\underbrace{\sum_{n=1}^{+\infty}w_{n,k}}_{=~v_{k}}<+\infty\end{equation}

	donc 
	\begin{equation}\boxed{\sum_{k=1}^{+\infty}v_{k}=\sum_{k=1}^{+\infty}u_{k}}\end{equation}

	On pose
	\begin{equation}u_{n}=\frac{1}{n(n+1)\dots(n+p)}\end{equation}
	et 
	\begin{equation}v_{n}=\frac{1}{(n+1)\dots(n+p)}-\frac{n}{(n+1)\dots(n+p+1)}=\frac{p+1}{(n+1)\dots(n+p+1)}\end{equation}

	Soit 
	\begin{equation}S_{p}=\sum_{n=1}^{+\infty}\frac{1}{n(n+1)\dots(n+p)}=(p+1)\sum_{n=2}^{+\infty}\frac{1}{n(n+1)\dots(n+p)}=(p+1)\left(S_{p}-\frac{1}{p!}\right)\end{equation}
	Ainsi, 
	\begin{equation}\boxed{\sum_{n=1}^{+\infty}\frac{1}{n(n+1)\dots(n+p)}}=\frac{p+1}{p(p!)}\end{equation}
\end{proof}

\begin{proof}
	Montrons d'une manière générale que si $(u_{k})_{k\in\N}\in\left(\R_{+}^{*}\right)^{\N}$ est telle que 
	\begin{equation}
		u_{k}=\underset{k\to+\infty}{o}(u_{k+1})	
	\end{equation}
	, alors $\sum u_{k}$ diverge et $\sum_{k=0}^{n}u_{k}\underset{n\to+\infty}{\sim}u_{n}$.

	En effet, on a alors $\lim\limits_{k\to+\infty}\frac{u_{k+1}}{u_{k}}=+\infty$ et d'après la règle de d'Alembert, $\sum u_{k}$ diverge. Soit ensuite $\varepsilon>0$. Il existe $N\in\N$ tel que pour tout $k\geqslant N$, 
	\begin{equation}0\leqslant \frac{u_{k}}{u_{k+1}}\leqslant\varepsilon\end{equation}
	Soit $n\geqslant N$. Pour $k\geqslant N+1$, on a 
	\begin{equation}u_{k}\leqslant \varepsilon u_{k+1}\leqslant\dots\leqslant \varepsilon^{n-k}u_{n}\end{equation}
	pour $k\leqslant n-1$.

	Alors 
	\begin{equation}0\leqslant\sum_{k=0}^{n-1}u_{k}\leqslant\sum_{k=0}^{N}u_{k}+\sum_{k=N+1}^{n-1}u_{k}\leqslant\left(\varepsilon+\varepsilon^{2}+\dots+\varepsilon^{n-N-1}u_{n}\right)\end{equation}

	On peut supposer que $\varepsilon<\frac{1}{2}$ et alors 
	\begin{equation}0\leqslant\sum_{k=0}^{n-1}u_{k}\leqslant\frac{\varepsilon}{1-\varepsilon}u_{n}\leqslant2\varepsilon u_{n}\end{equation}
	Donc on a bien le résultat voulu.

	Pour revenir à l'exercice, on a alors 
	\begin{equation}v_{n}\underset{n\to+\infty}{\sim}\frac{n!}{(n+q)!}\underset{n\to+\infty}{\sim}\frac{1}{n^{q}}\end{equation}
	qui est le terme général d'une série absolument convergente.
	Donc 
	\begin{equation}\boxed{\sum v_{n}\text{ converge.}}\end{equation}
\end{proof}

\begin{proof}
	On a 
	\begin{equation}\left\vert\frac{z^{nb}}{z^{na+c}+1}\right\vert=\frac{\left\vert z\right\vert^{nb}}{\left\vert1+z^{na+c}\right\vert}\underset{n\to+\infty}{\sim}\left\vert z\right\vert^{nb}\end{equation}
	car $\vert z\vert<1$. $\vert z\vert^{nb}$ est le terme général d'une série absolument convergente.

	Pour $n$ fini, on a 
	\begin{equation}\frac{1}{1+z^{na+c}}=\sum_{k=0}^{+\infty}(-z^{na+c})^{k}\end{equation}

	Montrons donc que $\left(z^{nb}\left(\left(-z^{na+c}\right)^{k}\right)\right)_{(k,n)\in\N^{2}}$ est sommable. On a 
	\begin{equation}\sum_{n=0}^{+\infty}\sum_{k=0}^{+\infty}\vert z\vert^{nb}\vert z\vert^{k(na+c)}=\sum_{n=0}^{+\infty}\frac{\vert z\vert^{nb}}{1-\vert z\vert^{na+c}}<+\infty\end{equation}
	d'après ce qui précède. On a sommabilité, donc d'après le théorème de Fubini,
	\begin{align}
		\sum_{n=0}^{+\infty}\frac{z^{nb}}{1+z^{na+c}}
		&=\sum_{k=0}^{+\infty}\sum_{n=0}^{+\infty}z^{nb}(-z^{na+c})^{k}\\
		&=\sum_{k=0}^{+\infty}(-1)^{k}z^{ck}\left(\sum_{n=0}^{+\infty}z^{n(b+ak)}\right)\\
		&=\sum_{k=0}^{+\infty}\frac{(-1)^{k}z^{ck}}{1-z^{b+ak}}
	\end{align}

	Ainsi, on a bien 
	\begin{equation}\boxed{\sum_{n=0}^{+\infty}\frac{z^{nb}}{1+z^{na+c}}=\sum_{k=0}^{+\infty}\frac{(-1)^{k}z^{ck}}{1-z^{b+ak}}}\end{equation}
\end{proof}

\begin{proof}
	On a 
	\begin{equation}b_{q}=\sum_{n=1}^{q}u_{n,q}=\sum_{n=1}^{+\infty}u_{n,q}\end{equation}

	Montrons donc que la famille des $(u_{n,q})_{(n,q)\in(\N^{*})^{2}}$ est sommable. On a 
	\begin{align}
		\sum_{n=1}^{+\infty}\sum_{q=1}^{+\infty}\vert u_{n,q}\vert
		&=\sum_{n=1}^{+\infty}\sum_{q=n}^{+\infty}\frac{n\vert a_{n}\vert}{q(q+1)}\\
		&=\sum_{n=1}^{+\infty}n\vert a_{n}\vert\left(\sum_{q=n}^{+\infty}\frac{1}{q}-\frac{1}{q+1}\right)\\
		&=\sum_{n=1}^{+\infty}\vert a_{n}\vert<+\infty
	\end{align}

	Donc le théorème de Fubini s'applique et on a 
	\begin{equation}\sum_{q=1}^{+\infty}\sum_{n=1}^{+\infty}u_{n,q}=\sum_{q=1}^{+\infty}b_{q}=\sum_{n=1}^{+\infty}\sum_{q=1}^{+\infty}u_{n,q}=\sum_{n=1}^{+\infty}a_{n}\end{equation}

	Donc 
	\begin{equation}\boxed{\sum_{q=1}^{+\infty}b_{q}=\sum_{n=1}^{+\infty}a_{n}}\end{equation}
\end{proof}

\begin{proof}
	D'après l'exercice précédent, $\sum v_{n}$ converge et 
	\begin{equation}\sum_{n=1}^{+\infty}v_{n}=\sum_{n=1}^{+\infty}u_{n}\end{equation}
	On applique l'inégalité de la moyenne géométrique et arithmétique à $(u_{1},2u_{2},\dots,nu_{n})$:
	\begin{equation}\sqrt[n]{u_{1}\times 2u_{2}\times\dots\times nu_{n}}=w_{n}\sqrt[n]{n!}\leqslant\frac{1}{n}(u_{1}+2u_{2}+\dots+nu_{n})=(n+1)v_{n}\end{equation}
	Donc on a 
	\begin{equation}w_{n}\leqslant\frac{(n+1)v_{n}}{\sqrt[n]{n!}}\end{equation}

	On étudie donc $\sqrt[n]{n!}$:
	\begin{align}
		\sqrt[n]{n!}
		&=\exp\left(\frac{1}{n}\ln(n!)\right)\\
		&=\exp\left(\frac{1}{n}\ln\left(n^{n}e^{-n}\sqrt{2\pi n}\left(1+\underset{n\to+\infty}{o}\left(1\right)\right)\right)\right)\\
		&=\exp\left(\frac{1}{n}\left(n\ln\left(n\right)-n+\frac{1}{2}\ln\left(\pi n\right)+\ln\left(1+\underset{n\to+\infty}{o}\left(1\right)\right)\right)\right)\\
		&=n\exp\left(-1+\underset{n\to+\infty}{o}\left(1\right)\right)\\
		&\underset{n\to+\infty}{\sim}\frac{n}{e}
	\end{align}

	Donc \begin{equation}\lim\limits_{n\to+\infty}\frac{n+1}{\sqrt[n]{n!}}=e\end{equation}

	Ainsi, $w_{n}=\underset{n\to+\infty}{O}(v_{n})$ donc 
	\begin{equation}\boxed{\sum w_{n}\text{ converge.}}\end{equation}

	Montrons que pour tout $n\geqslant1$, 
	\begin{equation}\frac{n+1}{\sqrt[n]{n!}}\leqslant e\end{equation}
	Cela équivaut à $(n+1)^{n}\leqslant e^{n}n!$ si et seulement si 
	\begin{equation}\sum_{k=0}^{n}\binom{n}{k}n^{k}\leqslant n!e^{n}\end{equation}
	ce qui est vrai car pour tout $k\in\{0,\dots,n\}$ on a $\frac{1}{(n-k)!}\leqslant1$.
	Donc $w_{n}\leqslant ev_{n}$ pour tout $n\geqslant1$ et donc 
	\begin{equation}\boxed{\sum_{n=1}^{+\infty}w_{n}\leqslant e\sum_{n=1}^{+\infty}v_{n}=e\sum_{n=1}^{+\infty}u_{n}}\end{equation}

	Pour montrer que $e$ est la meilleure constante possible, on forme pour $N\in\N^{*}$, $u_{n,N}=\frac{1}{n}$ si $n\leqslant N$ et 0 sinon.
	On a 
	\begin{equation}\sum_{n=1}^{+\infty}=H_{n}<+\infty\end{equation}
	Dans ce cas, on a 
	\begin{equation}w_{n,N}=\sqrt[n]{u_{1,N}\dots w_{n,N}}=\frac{1}{\sqrt[n]{n!}}=\frac{n+1}{\sqrt[n]{n!}}v_{n}\end{equation}
	pour $n\leqslant N$ et 0 sinon.
	On a alors 
	\begin{equation}\sum_{n=1}^{+\infty}w_{n,N}=\sum_{n=1}^{N}w_{n,N}=\sum_{n=1}^{N}\frac{n+1}{\sqrt[n]{n!}}v_{n}\end{equation}
	En divisant par $\sum_{n=1}^{+\infty}u_{n}=\sum_{n=1}^{+\infty}v_{n}$, on a donc 
	\begin{equation}\frac{\sum_{n=1}^{+\infty}w_{n,N}}{\sum_{n=1}^{+\infty}u_{n,N}}=\frac{\sum_{n=1}^{N}v_{n,N}\times\frac{n+1}{\sqrt[n]{n!}}}{\sum_{n=1}^{+\infty}v_{n,N}}\xrightarrow[n\to+\infty]{}e\end{equation}
	d'après le théorème de Césaro.

	On a trouvé une suite donc la constante $C$ est égale à $e$. D'après ce qui précède, 
	\begin{equation}\boxed{e\text{ est la meilleure constante possible.}}\end{equation}
\end{proof}

\begin{remark}
	Pour la fin de l'exercice précédent, on peut utiliser le fait que $H_{N}\underset{N\to+\infty}{\sim}\ln(N)$ et alors 
	\begin{equation}\sum_{n=1}^{N}w_{n,N}\sum_{n=1}^{N}\underbrace{\frac{n+1}{\sqrt[n]{n!}}\times\frac{1}{n+1}}_{\underset{n\to+\infty}{\sim}\frac{e}{n}}\underset{N\to+\infty}{\sim}e\sum_{n=1}^{N}\frac{1}{n}\underset{N\to+\infty}{\sim}e\ln(N)\end{equation}
	par le théorème de sommation des relations de comparaison.
\end{remark}

\begin{proof}
	\phantom{}
	\begin{enumerate}
		\item Soit $n\in\N^{*}$ et 
		\begin{equation}I_{n}=\left\{(p,q)\in\N^{2}\setminus\{(0,0)\}\middle| p+q=n\right\}\end{equation}
		On a alors 
		\begin{equation}\Sigma_{n}=\sum_{(p,q)\in I_{n}}\frac{1}{(p+q)^{\alpha}}=\sum_{(p,q)\in I_{n}}\frac{1}{n^{\alpha}}=\frac{n+1}{n^{\alpha}}\underset{n\to+\infty}{\sim}\frac{1}{n^{\alpha-1}}\end{equation}

		\begin{equation}\boxed{\text{Donc la condition nécessaire et suffisante est }\alpha>2\text{.}}\end{equation}

		Dans ce cas, par le théorème des sommation par paquets, on a 
		\begin{equation}\boxed{\sum_{(p,q)\in\N^{2}\setminus\{(0,0)\}}\frac{1}{(p+q)^{\alpha}}=\sum_{n=0}^{+\infty}\frac{1}{n^{\alpha-1}}+\frac{1}{n^{\alpha}}=\zeta(\alpha-1)+\zeta(\alpha)}\end{equation}

		\item Pour tout $(p,q)\in\N^{2}\setminus\{(0,0)\}$, on a 
		\begin{equation}\frac{(p+q)^{2}}{2}\leqslant p^{2}+q^{2}\leqslant (p+q)^{2}\end{equation}

		Pour $\alpha\leqslant0$, il est clair que l'on a divergence. Pour $\alpha>0$, on a donc 
		\begin{equation}\frac{1}{(p+q)^{2\alpha}}\leqslant\frac{1}{(p^{2}+q^{2})^{\alpha}}\leqslant\frac{2^{\alpha}}{(p+q)^{2\alpha}}\end{equation}
		
		\begin{equation}\boxed{\text{Donc la condition nécessaire et suffisante est }\alpha>1\text{.}}\end{equation}
		d'après le 1.
	\end{enumerate}
\end{proof}

\begin{proof}
	On fixe $n\in\N^{*}$. On a 
	\begin{equation}\sum_{m=0}^{+\infty}\frac{1}{(m+n^{2})(m+n^{2}+1)}=\sum_{m=0}^{+\infty}\frac{1}{m+n^{2}}-\frac{1}{m+n^{2}-1}=\frac{1}{n^{2}}=\Sigma_{n}\end{equation}
	par téléscopage.
	$\sum_{n\geqslant1}\Sigma_{n}$ converge et 
	\begin{equation}\sum_{n\geqslant1}\Sigma_{n}=\frac{\pi^{2}}{6}\end{equation}

	\begin{equation}\boxed{\text{Donc }\left(\frac{1}{\left(m+n^{2}\right)\left(m+n^{2}+1\right)}\right)_{(m,n)\in\N\times\N^{*}}\text{ est sommable et la somme vaut }\frac{\pi^{2}}{6}\text{.}}\end{equation}

	Posons, pour $k\geqslant1$, 
	\begin{equation}I_{k}=\left\{(m,n)\in\N\times\N^{*}\middle| m+n^{2}=k\right\}\end{equation}
	On a $n^{2}\in\{1,\dots,k\}$ si et seulement si $n\in\left\{1,\dots,\left\lfloor \sqrt{k}\right\rfloor\right\}$ et $(m,n)\in I_{k}$ si et seulement si $m=k-n^{2}$.

	On a $\left\vert I_{k}\right\vert=\left\lfloor\sqrt{k}\right\rfloor$ et par sommation par paquets,
	\begin{equation}\boxed{\frac{\pi^{2}}{6}=\sum_{k=1}^{+\infty}\sum_{(m,n)\in I_{k}}\frac{1}{(m+n^{2})(m+n^{2}+1)}=\sum_{k=1}^{+\infty}\frac{\left\lfloor k\right\rfloor}{k(k+1)}}\end{equation}
\end{proof}

\begin{remark}
	Grâce à une transformation d'Abel, on a aussi, pour $N\geqslant1$,
	\begin{align}
		\sum_{k=1}^{N}\frac{\left\lfloor k\right\rfloor}{k(k+1)}
		&=\sum_{k=1}^{N}\frac{\left\lfloor k\right\rfloor}{k}-\sum_{k=1}^{N}\frac{\left\lfloor k\right\rfloor}{k+1}\\
		&=1+\sum_{k=2}^{N}\underbrace{\frac{\left\lfloor k\right\rfloor-\left\lfloor k-1\right\rfloor}{k}}_{\neq0\text{ ssi }k=p^{2}}+\underbrace{\frac{\left\lfloor N\right\rfloor}{N+1}}_{\xrightarrow[N\to+\infty]{}0}
	\end{align}
	et on retrouve le résultat.
\end{remark}

\begin{proof}
	\phantom{}
	\begin{enumerate}
		\item \begin{equation}\prod_{k\geqslant1}\frac{1}{1-\frac{1}{p_{k}}}\end{equation}
		converge si et seulement si 
		\begin{equation}\sum_{k\geqslant1}-\ln\left(\frac{1}{1-\frac{1}{p_{k}}}\right)\end{equation}
		converge si et seulement si 
		\begin{equation}\sum_{k\geqslant1}-\ln\left(1--\frac{1}{p_{k}}\right)\end{equation}
		converge si et seulement si (car $-\ln\left(1-\frac{1}{p_{k}}\right)\underset{k\to+\infty}{\sim}p_{k}>0$ vu que $p_{k}\geqslant k$ pour tout $k\geqslant1$)
		\begin{equation}\sum_{k\geqslant1}\frac{1}{p_{k}}\end{equation}
		converge.

		Donc 
		\begin{equation}\boxed{\prod_{k\geqslant1}\frac{1}{1-\frac{1}{p_{k}}}\text{ converge si et seulement si }\sum_{k\geqslant 1}\frac{1}{p_{k}}\text{ converge.}}\end{equation}

		Fixons alors $N\in\N^{*}$. On a 
		\begin{equation}\prod_{k=1}^{N}\frac{1}{1-\frac{1}{p_{k}}}=\prod_{k=1}^{N}\left(\sum_{n_{k}=0}^{+\infty}\frac{1}{p_{k}^{n_{k}}}\right)\end{equation}
		où la série est à termes positifs et est convergent. Par produit de Cauchy,
		\begin{equation}\left(\frac{1}{p_{1}^{n_{1}}\dots p_{N}^{n_{N}}}\right)_{n_{1},\dots,n_{N}\in\N^{N}}\end{equation}
		est sommable et on a
		\begin{align}
			\prod_{k=1}^{N}\frac{1}{1-\frac{1}{p_{k}}}
			&=\sum_{(n_{1},\dots,n_{N})\in\N^{N}}\frac{1}{p_{1}^{n_{1}}\dots p_{N}^{n_{N}}}\\
			&\geqslant \sum_{k=1}^{p_{N+1}-1}\frac{1}{k}\xrightarrow[N\to+\infty]{}+\infty
		\end{align}
		car dans la première somme, tous les inverses (et une seule fois) des nombres dont les facteurs premiers sont dans $\{p_{1},\dots,p_{N}\}$ apparaissent.

		Donc 
		\begin{equation}\boxed{\sum_{k\geqslant1}\frac{1}{p_{k}}\text{ diverge.}}\end{equation}

		\item Posons 
		\begin{equation}\Pi_{n}=\prod_{k=1}^{n}\frac{1}{1-\frac{1}{p_{k}^{s}}}\end{equation}
		On a 
		\begin{equation}\ln\left(\Pi_{n}\right)=\sum_{k=1}^{n}\underbrace{-\ln\left(1-\frac{1}{1-\frac{1}{p_{k}^{s}}}\right)}_{\underset{k\to+\infty}{\sim}\frac{1}{p_{k}^{s}}=\underset{k\to+\infty}{O}\left(\frac{1}{k^{s}}\right)}\end{equation}
		car $p_{k}\geqslant k$. 
		Donc 
		\begin{equation}\boxed{\left(\Pi_{n}\right)\text{ converge dans }\R_{+}^{*}\text{.}}\end{equation}
		
		Par produit de Cauchy,
		\begin{equation}\left(\frac{1}{\left(p_{1}^{s}\right)^{j_{1}}\dots\left(p_{n}^{s}\right)^{j_{n}}}\right)_{(j_{1},\dots,j_{n})\in\N^{n}}\end{equation}
		
		Ainsi, on a 
		\begin{align}
			\Pi_{n}=\sum_{(j_{1},\dots,j_{n})\in\N^{n}}\left(\frac{1}{p_{1}^{j_{1}}\dots p_{n}^{j_{n}}}\right)^{s}
			&\leqslant\sum_{k=1}^{+\infty}\frac{1}{k^{s}}\\
			&=\zeta(s)
		\end{align}
		car dans la première somme, par unicité de la décomposition en facteurs premiers, chaque $k$ n'apparaît qu'une unique fois.
		Comme on a 
		\begin{equation}\sum_{k=1}^{p_{n+1}-1}\frac{1}{k^{s}}\leqslant \Pi_{n}\end{equation}
		Donc $\Pi_{n}\xrightarrow[n\to+\infty]{}\zeta(s)$ et ainsi 
		\begin{equation}\boxed{\prod_{k=1}^{+\infty}\frac{1}{1-\frac{1}{p_{k}^{s}}}=\zeta(s)}\end{equation}

		\item Soit $z=a+\i b\in\C$. Si $a>1$, on a 
		\begin{equation}\left\vert\frac{1}{n^{z}}\right\vert=\frac{1}{n^{a}}\end{equation}
		Donc $\sum \frac{1}{n^{z}}$ converge absolument. On peut donc prolonger $\zeta$ à $\left\{z\in\C\middle|\Re(z)>1\right\}$.

		De même que précédemment, puisque 
		\begin{equation}\left\vert\left(\frac{1}{p_{1}^{j_{1}}\dots p_{n}^{j_{n}}}\right)^{z}\right\vert=\frac{1}{\left(p_{1}^{j_{1}}\dots p_{n}^{j_{n}}\right)^{a}}\end{equation}
		la famille 
		\begin{equation}\left(\left(\frac{1}{p_{1}^{j_{1}}\dots p_{n}^{j_{n}}}\right)^{z}\right)_{(j_{1},\dots,j_{n})\in\N^{n}}\end{equation}
		est sommable. On peut aussi développer et 
		\begin{equation}\Pi_{n}=\prod_{k=1}^{n}\frac{1}{1-\frac{1}{p_{k}^{z}}}=\sum_{(j_{1},\dots,j_{n})\in\N^{n}}\left(\frac{1}{p_{1}^{j_{1}}\dots p_{n}^{j_{n}}}\right)^{z}\end{equation}
		
		On a 
		\begin{align}
			\left\vert\Pi_{n}-\zeta(z)\right\vert
			&=\left\vert\sum_{(j_{1},\dots,j_{n})\in\N^{n}}\left(\frac{1}{p_{1}^{j_{1}}\dots p_{n}^{j_{n}}}\right)^{z}-\sum_{k=1}^{+\infty}\frac{1}{k^{z}}\right\vert\\
			&=\left\vert\sum_{k\in\N\setminus J_{n}}\frac{1}{k^{z}}\right\vert\\
			&\leqslant \sum_{k\in\N\setminus J_{n}}\frac{1}{k^{a}}\xrightarrow[n\to+\infty]{}0
		\end{align}
		où l'on a noté $J_{n}=\{k\geqslant 1|\text{ les facteurs premiers de }k\text{ sont dans }\{p_{1},\dots,p_{n}\}\}$ et où l'on a appliqué l'inégalité triangulaire et le résultat de 2. pour conclure.

		Ainsi, on a bien 
		\begin{equation}\boxed{\zeta(z)=\prod_{k=1}^{+\infty}\frac{1}{1-\frac{1}{p_{k}^{s}}}}\end{equation}
	\end{enumerate}
\end{proof}

\begin{proof}
	Pour $\alpha>2$, puisque $\varphi(n)\geqslant n$, on a 
	\begin{equation}\frac{\varphi(n)}{n^{\alpha}}\leqslant \frac{1}{n^{\alpha-1}}\end{equation}
	qui est le terme général d'une série absolument convergente. 

	Pour $\alpha=2$, si $n=p_{k}$ est premier, on a $\varphi(p_{k})=p_{k}-1$ et 
	\begin{equation}\frac{\varphi(p_{k})}{p_{k}^{2}}=\frac{p_{k}-1}{p_{k}^{2}}\underset{k\to+\infty}{\sim}\frac{1}{p_{k}}\end{equation}
	et $\sum_{k\geqslant1}\frac{1}{p_{k}}$ diverge.

	De même pour $\alpha<2$, $\sum\frac{\varphi(n)}{n^{\alpha}}$ diverge car $\frac{\varphi(n)}{n^{2}}=\underset{n\to+\infty}{O}\left(\frac{\varphi(n)}{n^{\alpha}}\right)$.

	Donc 
	\begin{equation}\boxed{\sum \frac{\varphi(n)}{n^{\alpha}}\text{ converge si et seulement si }\alpha>2\text{.}}\end{equation}

	Pour $\alpha>1$, on calcule 
	\begin{equation}S=\sum_{n_{1}=1}^{+\infty}\frac{\varphi(n_{1})}{n_{1}^{\alpha}}\times \sum_{n_{2}=1}^{+\infty}\frac{1}{n_{2}^{\alpha}}=\sum_{(n_{1},n_{2})\in\N^{2}}\frac{\varphi(n_{1})}{(n_{1}n_{2})^{\alpha}}\end{equation}
	ce qui est légitime car il s'agit de deux séries à termes positifs convergentes. Soit, pour $n\geqslant1$, $D_{n}=\left\{(n_{1},n_{2})\in\left(\N^{*}\right)^{2}\middle| n=n_{1}n_{2}\right\}$. Par sommation par paquets, on a, 
	\begin{equation}S=\sum_{n=1}^{+\infty}\sum_{(n_{1},n_{2})\in D_{n}}\frac{\varphi(n_{1})}{n^{\alpha}}=\sum_{n=1}^{+\infty}\frac{1}{n^{\alpha}}\left(\sum_{n_{1}\mid n}\varphi(n_{1})\right)\end{equation}
	et grâce à la formule d'Euler-Möbius, on a 
	\begin{equation}\sum_{n_{1}\mid n}\varphi(n_{1})=n\end{equation}
	Ainsi, $S=\zeta(\alpha-1)$ et donc 
	\begin{equation}\boxed{\sum_{n\geqslant1}\frac{\varphi(n)}{n^{\alpha}}=\frac{\zeta(\alpha-1)}{\zeta(\alpha)}}\end{equation}
\end{proof}

\begin{proof}
	Soit $A\in \C$ et $R>0$. S'il y a $n$ indices $k\in\N$ tels que $z_{k}\in B(A,R)$, alors pour ces indices $k$, on a $B(z_{k},\frac{1}{2})\subset B(A,R+\frac{1}{2})$. Donc (faire un dessin !), on a 
	\begin{equation}n\frac{\pi}{4}\leqslant \pi\left(R+\frac{1}{2}\right)^{2}\end{equation}

	On pose, pour tout $n\in\N$, $B_{n}=\left\{i\in\N\middle| z_{i}\in B(0,n)\right\}$. De l'inégalité précédente, pour tout $n\in\N$, $D_{n}$ est fini. Il existe $\sigma\colon\N\to\N$ bijective qui permet d'ordonner les $z_{n}$ par module croissante et à même module par indice croissant.

	Pour $n\in\N$ et $R=\left\vert z_{\sigma(n)}\right\vert$, on a pour tout $k\leqslant n$, $z_{\sigma(k)}\in B(0,R)$.

	Donc 
	\begin{equation}n\frac{\pi}{4}\leqslant\pi\left(\left\vert z_{\sigma(n)}\right\vert+\frac{1}{2}\right)^{2}\end{equation}
	d'où 
	\begin{equation}\left\vert z_{\sigma(n)}\right\vert\geqslant\left\vert z_{\sigma(n)}+\frac{1}{2}\right\vert-\frac{1}{2}\geqslant\frac{\sqrt{n}}{2}-\frac{1}{2}\end{equation}
	Donc 
	\begin{equation}\left\vert\frac{1}{z_{\sigma(n)}}\right\vert^{3}=\underset{n\to+\infty}{O}\left(\frac{1}{n^{\frac{3}{2}}}\right)\end{equation}

	Donc
	\begin{equation}\boxed{\sum \frac{1}{z_{\sigma(n)}^{3}}\text{ est absolument convergente.}}\end{equation}
\end{proof}

\begin{proof}
	On a $k=\left\lfloor n\right\rfloor$ si et seulement si $k^{2}\leqslant n<(k+1)^{2}$. Il y a $(k+1)^{2}-k^{2}=2k+1$ entiers.

	Posons
	\begin{equation}B_{p}=\sum_{n=1}^{p}(-1)^{\left\lfloor n\right\rfloor}\end{equation}
	et $B_{-1}=0$.
	Si $k^{2}\leqslant p\leqslant(k+1)^{2}$, on a 
	\begin{equation}B_{p}=\underbrace{B_{k}^{2}}_{\text{signe de }(-1)^{k}}+(-1)^{k}\underbrace{(p-k)^{2}}_{\left\vert \cdot\right\vert\leqslant 2k+1}\end{equation}

	Par récurrence, pour tout $p\in\N$,
	\begin{equation}\left\vert B_{p}\right\vert\leqslant2\left\lfloor p\right\rfloor+1\end{equation}
	Donc avec une transformation d'Abel, on a 
	\begin{align}
		\sum_{n=1}^{N}\frac{(-1)^{\left\lfloor n\right\rfloor}}{n}
		&=\sum_{n=1}^{N}\frac{\left(B_{n}-B_{n-1}\right)}{n}\\
		&=\sum_{n=1}^{N}\frac{B_{n}}{n}-\sum_{n=0}^{N-1}\frac{B_{n}}{n+1}\\
		&=\underbrace{\frac{B_{N}}{N}}_{=\underset{N\to+\infty}{O}\left(\frac{1}{\sqrt{N}}\right)}-B_{0}+\underbrace{\sum_{n=1}^{N-1}\frac{B_{n}}{n(n+1)}}_{=\underset{N\to+\infty}{O}\left(\frac{1}{n^{\frac{3}{2}}}\right)}
	\end{align}

	D'après le critère de Riemann, la dernière somme converge absolument et donc 
	\begin{equation}\boxed{\sum_{n\geqslant1}\frac{(-1)^{\left\lfloor n\right\rfloor}}{n}\text{ converge.}}\end{equation}
\end{proof}

\begin{proof}
	\phantom{}
	\begin{enumerate}
		\item Pour tout $n\in\N$, on a $u_{n}\neq 0$. Il existe $N_{0}\in\N$ tel que pour tout $n\geqslant N_{0}$, $u_{n}u_{n+1}>0$. On a 
		\begin{equation}\ln\left(\frac{u_{n}}{u_{N_{0}}}\right)=\sum_{k=N_{0}+1}^{n}\ln\left(\frac{a+k}{n+k}\right)=\sum_{k=N_{0}+1}\ln\left(1+\frac{a}{k}\right)-\ln\left(1+\frac{b}{k}\right)\end{equation}

		Alors 
		\begin{equation}\ln\left(\frac{u_{n}}{u_{N_{0}}}\right)=\sum_{k=N_{0}+1}^{n}\frac{a-b}{k}+\underbrace{\underset{k\to+\infty}{O}\left(\frac{1}{k^{2}}\right)}_{\text{terme général d'une série convergente}}=(a-b)\ln(n)+\underbrace{C}_{\in\R}+\underset{n\to+\infty}{o}(1)\end{equation}

		Ainsi,
		\begin{equation}u_{n}=u_{N_{0}}n^{a-b}\underbrace{k^{1+\underset{n\to+\infty}{o}(1)}}_{>0}\underset{n\to+\infty}{\sim}U_{N_{0}}n^{a-b}k\end{equation}

		Donc 
		\begin{equation}\boxed{\sum u_{n}\text{ converge si et seulement si }b-a>1}\end{equation}

		\item On a 
		\begin{equation}u_{n+1}(b+n+1)=u_{n}(a+n+1)\end{equation}
		donc 
		\begin{equation}(a+1)u_{n}=bu_{n+1}+(n+1)u_{n+1}-nu_{n}\end{equation}
		En sommant sur $\N$, on a 
		\begin{equation}(a+1)\sum_{n=0}^{+\infty}u_{n}=b\sum_{n=1}^{+\infty}u_{n}+u_{1}=b\sum_{n=1}^{+\infty}+\underbrace{u_{1}-bu_{0}}_{=~\frac{a(a+1)}{b(b+1)}-a}\end{equation}

		Ainsi, 
		\begin{equation}\boxed{\sum_{n=0}^{+\infty}u_{n}=\frac{a\left(a+1-b\left(b+1\right)\right)}{b\left(b+1\right)\left(a+1-b\right)}=a\left(\frac{1}{b(b+1)}-\frac{b}{(b+1)(a+1-b)}\right)}\end{equation}

		\item Pour $a=-\frac{1}{2}$ et $b=1$, on a 
		\begin{equation}\boxed{u_{n}=\frac{\left(-\frac{1}{2}\right)\left(\frac{1}{2}\right)\dots\left(n-\frac{1}{2}\right)}{(n+1)!}=\frac{-\frac{(2n)!}{2^{2n+1}n!}}{(n+1)!}=-\frac{1}{2^{2n+1}(n+1)}\binom{2n}{n}}\end{equation}
	\end{enumerate}
\end{proof}

\begin{proof}
	\phantom{}
	\begin{enumerate}
		\item $u_{n}$ est une série à termes positifs et 
		\begin{equation}\frac{1}{n}=\underset{n\to+\infty}{O}\left(\frac{\ln(n)}{n}\right)\end{equation}
		donc 
		\begin{equation}\boxed{\sum u_{n}\text{ diverge.}}\end{equation}

		$\sum v_{n}$ est une série alternée. On a $\lim\limits_{n\to+\infty}v_{n}=0$ et en formant \function{f}{[2,\infty[}{\R}{x}{\frac{\ln(x)}{x}}
		On a $f'(x)=\frac{1-\ln(x)}{x^{2}}$ qui est négatif dès que $x>e$. Donc $(v_{n})_{n\geqslant3}$ décroît. D'après le critère des séries alternées, 
		\begin{equation}\boxed{\sum v_{n}\text{ converge.}}\end{equation}

		\item $f$ décroît sur $[,+\infty[$ donc pour tout $k\geqslant4$, on a 
		\begin{equation}\int_{k}^{k+1}\frac{\ln(x)}{x}dx\leqslant\frac{\ln(k)}{k}\leqslant\int_{k-1}^{k}\frac{\ln(x)}{x}dx\end{equation}
		d'où 
		\begin{equation}\underbrace{\int_{4}^{N+1}\frac{\ln(x)}{x}dx}_{=~\frac{1}{2}\left[\ln^{2}\left(N+1\right)-\ln^{2}\left(4\right)\right]}\leqslant\sum_{k=4}^{N}\frac{\ln(k)}{k}\leqslant\underbrace{\int_{3}^{N}\frac{\ln(x)}{x}dx}_{=~\frac{1}{2}\left[\ln^{2}\left(N\right)-\ln^{2}\left(3\right)\right]}\end{equation}

		Donc 
		\begin{equation}\boxed{S_{N}\underset{N\to+\infty}{\sim}\frac{1}{2}\ln^{2}\left(N\right)}\end{equation}

		Formons $w_{n}=S_{n}-\frac{\ln^{2}(n)}{2}$. $(w_{n})_{n\in\N}$ converge si et seulement si $\sum_{n\in\N^{*}}w_{n}-w_{n-1}$ converge.
		On a 
		\begin{equation}w_{n}-w_{n-1}=\frac{\ln(n)}{n}-\frac{\ln^{2}(n)}{2}+\frac{\ln^{2}(n-1)}{2}\end{equation}

		On a 
		\begin{equation}\ln(n-1)=\ln(n)+\ln\left(1-\frac{1}{n}\right)=\ln(n)-\frac{1}{n}+\underset{n\to+\infty}{O}\left(\frac{1}{n^{2}}\right)\end{equation}
		et 
		\begin{equation}\ln^{2}(n-1)=\ln^{2}(n)-\frac{2\ln(n)}{n}+\underbrace{\underset{n\to+\infty}{O}\left(\frac{\ln(n)}{n^{2}}\right)}_{=\underset{n\to+\infty}{O}\left(\frac{1}{n^{\frac{3}{2}}}\right)}\end{equation}

		Donc 
		\begin{equation}w_{n}-w_{n-1}=\underbrace{\underset{n\to+\infty}{O}\left(\frac{1}{n^{\frac{3}{2}}}\right)}_{\text{terme général d'une série absolument convergente}}\end{equation}

		Donc il existe $L\in\R$ tel que 
		\begin{equation}\boxed{S_{n}=\frac{\ln^{2}(n)}{2}+L+\underset{n\to+\infty}{o}(1)}\end{equation}
		
		\item On a 
		\begin{equation}\sum_{n=2}^{2N}v_{n}=\underbrace{\sum_{k=1}^{N}\frac{\ln(2k)}{2k}}_{=~I_{N}}-\underbrace{\sum_{k=1}^{N-1}}\frac{\ln(2k+1)}{2k+1}_{=~J_{N}}\end{equation}

		Donc
		\begin{equation}\sum_{n=2}^{2N}v_{n}=I_{N}-(S_{2N}-I_{N})\end{equation}
		On a 
		\begin{equation}S_{2N}=\frac{\ln^{2}(2N)}{2}+L+\underset{N\to+\infty}{o}(1)=\frac{\ln^{2}(2)}{2}+\frac{\ln^{2}(N)}{2}+\ln(2)\ln(N)+L+\underset{N\to+\infty}{o}(1)\end{equation}
		De plus, 
		\begin{equation}I_{N}=\underbrace{\sum_{k=1}^{N}\frac{\ln(2)}{2k}}_{=\frac{\ln(2)}{2}\left(\ln(N)+\gamma+\underset{N\to+\infty}{o}(1)\right)}+\underbrace{\sum_{k=1}^{N}\frac{\ln(k)}{2k}}_{=~\frac{1}{2}S_{N}=\frac{\ln^{2}(N)}{4}+\frac{L}{2}+\underset{N\to+\infty}{o}(1)}\end{equation}

		Finalement, on a bien 
		\begin{align}
			\sum_{n=2}^{2N}v_{n}
			&=2I_{n}-S_{2N}\\
			&=\ln(2)\gamma-\frac{\ln^{2}(2)}{2}+\underset{N\to+\infty}{o}(1)
		\end{align}
		Donc 
		\begin{equation}\boxed{\sum_{n=2}^{+\infty}v_{n}=\ln(2)\gamma-\frac{\ln^{2}(2)}{2}}\end{equation}
	\end{enumerate}
\end{proof}

\begin{proof}
	Si $\alpha_{0}=2$ et $\alpha_{n+1}=10^{\alpha_{n}-1}$. Alors $q_{1}(\alpha_{n+1})=\alpha_{n}$, $q_{k}(\alpha_{n})=\alpha_{n-k}$, $q_{n}(\alpha_{n})=2$ et $q_{n+1}(\alpha_{n})=1$.

	Si $k<\alpha_{n}$, $q_{n}(k)=1$. Soit 
	\begin{equation}S_{n}=\sum_{k=\alpha_{n}}^{\alpha_{n+1}-1}u_{k}\end{equation}
	Comme c'est une série à termes positifs, $\sum_{k\geqslant1}u_{k}$ converge si et seulement $\sum_{n\geqslant0}S_{n}$ converge.

	Par définition, pour tout $k\in\{\alpha_{n},\dots,\alpha_{n+1}-1\}$, on a $q_{n+1}(k)=1$ et pour tout $p\geqslant n+1$, $q_{p}(k)=1$. Donc 
	\begin{equation}S_{n}=\sum_{k=\alpha_{n}}^{\alpha_{n+1}-1}\frac{1}{kq_{1}(k)\dots \underbrace{q_{n}(k)}_{\geqslant 2}}\end{equation}

	Posons \function{f}{\R_{+}^{*}}{\R}{t}{\log_{10}(t)=\frac{\ln(t)}{\ln(10)}}

	Il vient $q_{1}(t)=\left\lfloor f(t)\right\rfloor+1>f(t)$. Par récurrence, on a
	\begin{equation}q_{n}(t)\geqslant f^{n}(t)\end{equation}
	défini pour $t\geqslant \alpha_{n}$. On a donc 
	\begin{equation}S_{n}\sum_{k=\alpha_{n}}^{\alpha_{n+1}-1}	\frac{1}{k(f(k))\dots f^{n}(k)}\end{equation}

	On forme \function{g_n}{[\alpha_{n},\alpha_{n+1}-1}{\R}{t}{\frac{1}{tf(t)\dots f^{n}(t)}}
	qui est décroissante. Ainsi, pour tout $k\in\{\alpha_{n},\alpha_{n+1}-1\}$, on a 
	\begin{equation}\int_{k}^{k+1}g_{n}(t)\leqslant u_{k}\leqslant\int_{k-1}^{k}g_{n}(t)\end{equation}
	d'où en faisant le changement de variables $u=\log_{10}(t)$, on a
	\begin{equation}\int_{\alpha_{n-1}-1}^{\alpha_{n}-1}g_{n-1}(u)du(\ln(10))\leqslant S_{n}\leqslant\int_{\alpha_{n}-1}^{\alpha_{n+1}-1}g_{n}(t)dt\end{equation}

	On obtient donc une minoration par $C\times(\ln(10))^{n}$ donc 
	\begin{equation}\boxed{\text{la série diverge.}}\end{equation}
\end{proof}

\begin{proof}
	\phantom{}
	\begin{enumerate}
		\item Montrons le résultat par récurrence sur $n\in\N^{*}$. On a $P_{0}=1>0$ et $P_{1}(x)=1+x$ s'annule en -1. Soit $n\in\N^{*}$, supposons le résultat au rang $n$. On a $P_{2n+2}'(x)=P_{2n+1}(x)$, par hypothèse $P_{2n+1}$ s'annule uniquement en $\alpha_{2n+1}<0$. Donc $P_{2n+2}	(\alpha_{2n+1})=\frac{\left(\alpha_{2n+1}\right)^{2n+2}}{(2n+2)!}>0$ donc $P_{2n+2}>0$.
		Comme $P_{2n+3}'=P_{2n+2}>0$ donc $P_{2n+3}$ est strictement croissante sur $\R$. On a $\lim\limits_{x\to\pm\infty}P_{2n+3}=\pm\infty$. Donc il existe un unique $\alpha_{2n+3}\in\R$ tel que $P_{2n+3}(\alpha_{2n+3})=0$. Comme $P_{2n+3}(0)=1\geqslant1$, $\alpha_{2n+3}<0$.
		\begin{equation}\boxed{\text{D'où le résultat par récurrence.}}\end{equation}

		\item Soit $x<0$, on a $\lim\limits_{n\to+\infty}P_{n}(x)=e^{x}>0$. Donc il existe $n_{0}\in\N$ tel que pour tout $n\geqslant n_{0}$, $P_{2n+1}(x)>0$. En particulier, $\alpha_{2n+1}<x$ donc 
		\begin{equation}\boxed{\lim\limits_{n\to+\infty}\alpha_{2n+1}=-\infty}\end{equation}
	\end{enumerate}
\end{proof}

\begin{proof}
	On pose $f_{n}(x)=e^{x}-x-n$, on a $f'_{n}(x)=e^{x}-1$. Donc $x_{1}=0$ et ainsi 
	\begin{equation}\boxed{\forall n\geqslant 2,\exists! x_n\geqslant0\colon e^{x_{n}}=x_{n}+n}\end{equation}

	Pour tout $x\geqslant0$, on a $f_{n+1}(x)-f_{n}(x)=-1<0$ donc $f_{n+1}(x)<f_{n}(x)$ et ainsi $f_{n+1}(x_{n})<0$ et $x_{n}<x_{n+1}$.

	$(x_{n})_{n\in\N}$ est strictement croissante, de plus $e^{x_{n}}=x_{n}+n\geqslant n$ donc $x_{n}\geqslant \ln(n)$ et donc 
	\begin{equation}\boxed{\lim\limits_{n\to+\infty}x_{n}=+\infty}\end{equation}

	De plus, $x_{n}=\ln(x_{n}+n)$ et $f_{n}(n)=e^{n}-2n>0$ (par récurrence), donc $x_{n}<n$ par stricte croissante de $f_{n}$ donc 
	\begin{equation}x_{n}=\ln(x_{n}+n)\leqslant \ln(2n)=\ln(n)+\ln(2)\end{equation}
	Ainsi, $x_{n}=\underset{n\to+\infty}{O}(\ln(n))$. En reportant, on a 
	\begin{equation}x_{n}=\ln(n+\underset{n\to+\infty}{O}(\ln(n)))=\ln(n)+\ln(1+\underset{n\to+\infty}{O}\left(\frac{\ln(n)}{n}\right))=\ln(n)+\underset{n\to+\infty}{o}(1)\end{equation}
	donc 
	\begin{equation}\boxed{n\underset{n\to+\infty}{\sim}\ln(n)}\end{equation}

	En reportant, on a 
	\begin{equation}\boxed{x_{n}=\ln(n)+\frac{\ln(n)}{n}+\underset{n\to+\infty}{o}\left(\frac{\ln(n)}{n}\right)}\end{equation}
\end{proof}

\begin{proof}
	\phantom{}
	\begin{enumerate}
		\item Si $S_{n}\xrightarrow[n\to+\infty]{}S\in\R^{+}_{+}$, on a \begin{equation}v_{n}\underset{n\to+\infty}{\sim}\frac{u_{n}}{S^{\alpha}}\end{equation}
		Comme $u_{n}$ est le terme générale d'une série convergente donc 
		\begin{equation}\boxed{\sum v_{n}\text{ converge.}}\end{equation}

		\item On a $\alpha=1$ donc $v_{n}=\frac{u_{n}}{S_{n}}$, soit $(n,p)\in\N^{2}$. On a 
		\begin{equation}\sum_{i=n+1}^{n+p}v_{i}=\sum_{i=1}^{n+p}\frac{u_{i}}{S_{i}}\end{equation}
		où $(S_{i})_{i\in\N}$ est croissante donc pour tout $i\in\{n+1,n+p\}$, $S_{i}\leqslant S_{n+p}$
		donc 
		\begin{equation}\sum_{i=n+1}^{n+p}v_{i}\geqslant\frac{1}{S_{n+p}}\sum_{i=n+1}^{n+p}u_{i}=\frac{1}{S_{n+p}}\left(S_{n+p}-S_{n}\right)=1-\frac{S_{n}}{S_{p+n}}\end{equation}
		et ainsi,
		\begin{equation}\boxed{\sum_{i=n+1}^{n+p}v_{i}\geqslant 1-\frac{S_{n}}{S_{n+p}}}\end{equation}

		Supposons que $\sum v_{n}$ converge. Pour $n$ fixé, on a $\lim\limits_{p\to+\infty}S_{n+p}=+\infty$ (car $\sum u_{n}$ diverge). Donc lorsque $p\to+\infty$, on a pour tout $n\in\N$,
		\begin{equation}\sum_{i=n+1}^{+\infty}v_{i}\geqslant1\end{equation}
		ce qui est absurde puisque la limite en $+\infty$ du reste est 0.
		Ainsi,
		\begin{equation}\boxed{\sum v_{n}\text{ diverge.}}\end{equation}

		\item On a $\lim\limits_{n\to+\infty}S_{n}=+\infty$ et 
		\begin{equation}v_{n}=\frac{1}{\alpha-1}\left(S_{n-1}^{1-\alpha}-S_{n}^{1-\alpha}\right)\end{equation}
		avec $(S_{n}^{1-\alpha})_{n\in\N}$ tend vers $0$ quand $n\to+\infty$.
		Donc $\sum w_{n}$ est une série téléscopique convergente. Comme $t\mapsto\frac{1}{t^{\alpha}}$ est décroissante, on a 
		\begin{equation}\frac{u_{n}}{S_{n}^{\alpha}}\leqslant w_{n}\leqslant \frac{u_{n}}{S_{n-1}^{\alpha}}\end{equation}
		car $u_{n}=S_{n}-S_{n-1}$. Comme $\sum w_{n}$ converge, 
		\begin{equation}\boxed{\sum \frac{u_{n}}{S_{n}^{\alpha}}\text{ converge.}}\end{equation}

		Si $\alpha<1$, comme $\lim\limits_{n\to+\infty}S_{n}^{\alpha-1}=0$,
		\begin{equation}\frac{u_{n}}{S_{n}}=\underset{n\to+\infty}{o}\left(\frac{u_{n}}{S_{n}^{\alpha}}\right)\end{equation}
		donc 
		\begin{equation}\boxed{\sum v_{n}\text{ diverge.}}\end{equation}

		\item On a $\lim\limits_{n\to+\infty}R_{n}=0$ par convergence et $\lim\limits_{n\to+\infty}u_{n}=0$ et de plus $u_{n}=R_{n}-R_{n+1}$. On pose 
		\begin{equation}\alpha_{n}=\int_{R_{n+1}}^{R_{n}}\frac{dt}{t^{\alpha}}=\frac{1}{\alpha-1}\left(R_{n+1}^{1-\alpha}-R_{n}^{1-\alpha}\right)\end{equation}
		si $\alpha\neq1$.

		Si $0<\alpha<1$, $\lim\limits_{n\to+\infty}R_{n}^{1-\alpha}=0$ donc $\sum \alpha_{n}$ est une série téléscopique convergente et de même que précédemment, on a 
		\begin{equation}\frac{u_{n}}{R_{n}^{\alpha}}\leqslant\alpha_{n}\end{equation}
		donc $w_{n}\leqslant \alpha_{n}$ et 
		\begin{equation}\boxed{\sum w_{n}\text{ converge.}}\end{equation}

		Si $\alpha=1$, on a 
		\begin{equation}\alpha_{n}=\ln(R_{n})-\ln(R_{n+1})\end{equation}
		où $\ln(R_{n})\xrightarrow[n\to+\infty]{}-\infty$. Donc $\sum \alpha_{n}$ est une série téléscopique divergente. De plus 
		\begin{equation}\frac{u_{n}}{R_{n}}=\frac{R_{n}-R_{n+1}}{R_{n}}=1-\frac{R_{n+1}}{R_{n}}\end{equation}
		donc 
		\begin{equation}\ln\left(\frac{R_{n+1}}{R_{n}}\right)=\ln\left(1-\frac{u_{n}}{R_{n}}\right)\underset{n\to+\infty}{\sim}\frac{-u_{n}}{R_{n}}\end{equation}

		On a donc \begin{equation}\frac{u_{n}}{R_{n}}\underset{n\to+\infty}{\sim}\alpha_{n}\end{equation}
		donc 
		\begin{equation}\boxed{\sum w_{n}\text{ diverge.}}\end{equation}

		Si $\alpha>1$, on a \begin{equation}\frac{u_{n}}{R_{n}}=\underset{n\to+\infty}{o}\left(\frac{u_{n}}{R_{n}^{\alpha}}\right)\end{equation}
		donc 
		\begin{equation}\boxed{\sum w_{n}\text{ diverge.}}\end{equation}
	\end{enumerate}
\end{proof}

\begin{proof}
	\phantom{}
	\begin{enumerate}
		\item Pour tout $x\in[0,1[$ il existe un unique $q_{x}\in\{0,\dots,n-1\}$ tel que $x\in[\frac{q_{x}}{n},\frac{q_{x}+1}{n}]$ avec $q_{x}=\left\rfloor nx\right\rfloor$ et \function{h}{\{0,\dots,n\}}{\{0,\dots,n-1\}}{k}{q_{x_{k}}=\left\lfloor nx_{k}\right\rfloor}
		n'est pas injective donc il existe $k>k'$ tel que $\vert x_{k}-x_{k'}\vert<\frac{1}{n}$ avec $(k,k')\in\{0,\dots,n\}^{2}$ d'où
		\begin{equation}\left\vert kx-\left\lfloor kx\right\rfloor-\left(k'x-\left\lfloor k'x\right\rfloor\right)\right\vert<\frac{1}{n}\end{equation}
		d'où 
		\begin{equation}\left\vert (k-k')x-p\right\vert<\frac{1}{n}\end{equation}
		avec $p\in\Z$ et pour $q=(k-k')\in\{1,\dots,n\}$, on a 
		\begin{equation}\boxed{\left\vert x-\frac{p}{q}\right\vert<\frac{1}{qn}}\end{equation}

		\item D'après ce qui précède, pour tout $n\geqslant1$, il existe $(p_{n},q_{n})\in\Z\times\{1,\dots,n\}$ tels que 
		\begin{equation}\left\vert x-\frac{p_{n}}{q_{n}}\right\vert<\frac{1}{nq_{n}}\leqslant\frac{1}{q_{n}^{2}}\end{equation}
		car $n\geqslant q_{n}$.
		Donc 
		\begin{equation}\boxed{\left\vert x-\frac{p_{n}}{q_{n}}\right\vert<\frac{1}{q_{n}^{2}}}\end{equation}

		On a donc $\frac{p_{n}}{q_{n}}\xrightarrow[n\to+\infty]{}x\in\R\setminus\Q$. Si $q_{n}$ ne tend pas vers $+\infty$, il existe $A>0$ tel que pour tout $N\in\N$ il existe $n>N$ avec $q_{n}<A$. Donc $\{n\in\N|q_{n}<A\}$ est infini: on peut extraire $(q_{\sigma(n)})$ telle que pour tout $n\in\N$, on a $q_{\sigma(n)}<A$. D'après le théorème de Bolzano-Weierstrass, on peut extraire $(q_{\varphi(n)})$ qui converge vers $q\in\R$. Notons que toute suite d'entiers relatifs qui converge est stationnaire à partir d'un certain rang donc $q\in\N^{*}$. Or pour tout $n\in\N$, on a $p_{\varphi(n)}=\frac{p_{\varphi(n)}}{q_{\varphi(n)}}q_{\varphi(n)}\xrightarrow[n\to+\infty]{}\alpha q$. $(p_{\varphi(n)})_{n\in\N}$ est une suite convergente d'entiers relatifs stationnaire, donc $\alpha q\in\Z$ et $\alpha\in\Q$ ce qui est absurde.

		Donc 
		\begin{equation}\boxed{\lim\limits_{n\to+\infty}q_{n}=+\infty}\end{equation}

		\item On sait qu'il existe $\sigma\colon\N\to\N$ croissante telle que $\sin\left(\sigma(n)\right)\xrightarrow[n\to+\infty]{}1$ alors 
		\begin{equation}\lim\limits_{n\to+\infty}\frac{1}{\sigma(n)\sin(\sigma(n))}=0\end{equation}
		donc si la suite converge, alors elle converge vers 0.

		Appliquons ce qui précède à $\alpha=\frac{1}{\pi}\notin\Q$. Il existe $(p_{n},q_{n})\in\Z^{\N}\times(\N^{*})^{\N}$ avec $\lim\limits_{n\to+\infty}q_{n}=0$ et 
		\begin{equation}\left\vert\frac{1}{\pi}-\frac{p_{n}}{q_{n}}\right\vert<\frac{1}{q_{n}^{2}}\end{equation}
		Alors 
		\begin{equation}\left\vert q_{n}-\pi p_{n}\right\vert<\frac{\pi}{q_{n}}\leqslant\frac{\pi}{2}\end{equation}
		pour $n$ suffisamment grand. Quitte à extraire, on peut supposer que $(q_{n})_{n\in\N}$ est croissante. On a
		\begin{equation}\left\vert\sin(x)\right\vert=\left\vert\sin(q_n-\pi q_n)\right\vert\end{equation}
		donc 
		\begin{equation}\left\vert\sin(q_n)\right\vert\leqslant\left\vert\sin\left(\frac{\pi}{q_n}\right)\right\vert\leqslant\frac{\pi}{q_n}\end{equation}
		car $\sin$ est croissant sur $\left[0,\frac{\pi}{2}\right]$ et $\vert\sin(x)\vert\leqslant\vert x\vert$.

		Donc 
		\begin{equation}\underbrace{\frac{1}{\vert q_n\sin(q_n)\vert}}_{\xrightarrow[n\to+\infty]{}0}\geqslant\frac{1}{\pi}\end{equation}
		ce qui est absurde.

		Donc 
		\begin{equation}\boxed{\left(\frac{1}{n\sin(n)}\right)_{n\geqslant1}\text{ ne converge pas.}}\end{equation}
	\end{enumerate}
\end{proof}

\begin{proof}
	\phantom{}
	\begin{enumerate}
		\item On a 
		\begin{equation}\left\lvert\sum_{p=1}^{n}a_{n,p}-\sum_{p=1}^{+\infty}a_{p}\right\rvert=\left\lvert\sum_{p=1}^{n}\left(a_{n,p}-a_{p}\right)-\sum_{p=n+1}^{+\infty}a_{p}\right\rvert=\leqslant\sum_{p=1}^{n}\left\lvert a_{n,p}-a_{p}\right\rvert+\sum_{p=n+1}^{+\infty}\left\lvert a_{p}\right\rvert\end{equation}

		Soit $N\in\N^{*}$. Si $n\geqslant N$, on a 
		\begin{equation}\sum_{p=1}^{n}\left\lvert a_{n,p}-a_{p}\right\rvert=\sum_{p=1}^{N}\left\lvert a_{n,p}-a_{p}\right\rvert+\sum_{p=N+1}^{n}\left\lvert a_{n,p}-a_{p}\right\rvert\end{equation}

		Pour $p$ fixé, on a $\left\lvert a_{p}\right\rvert\leqslant b_{p}$ donc 
		\begin{equation}\sum_{p=N+1}^{n}\left\lvert a_{n,p}-a_{p}\right\rvert\leqslant 2\sum_{p=N+1}^{n} b_{p}\end{equation}

		Ainsi, 
		\begin{equation}\left\lvert\sum_{p=1}^{n}a_{n,p}-\sum_{p=1}^{+\infty}a_{p}\right\rvert\leqslant\sum_{p=1}^{N}\left\lvert a_{n,p}-a_{p}\right\rvert+3\sum_{p=N+1}^{+\infty}b_{p}\end{equation}

		Soit $\varepsilon>0$. Comme $\sum_{p\geqslant1}b_{p}$ converge, il existe $N_{1}\in\N$ tel que 
		\begin{equation}3\sum_{p=N_{1}+1}^{+\infty}b_{p}\leqslant\frac{\varepsilon}{2}\end{equation}
		donc pour tout $n\geqslant N_{1}$, on a 
		\begin{equation}\left\lvert\sum_{p=1}^{n}a_{n,p}-\sum_{p=1}^{+\infty}a_{p}\right\rvert<\frac{\varepsilon}{2}+\sum_{p=1}^{N_{1}}\left\lvert a_{n,p}-a_{p}\right\rvert\end{equation}

		$N_{1}$ étant fixé, il existe $N_{2}\in\N$ tel que pour tout $n\geqslant N_{2}$, on a 
		\begin{equation}\sum_{p=1}^{N_{1}}\left\lvert a_{n,p}-a_{p}\right\rvert<\frac{\varepsilon}{2}\end{equation}
		car 
		\begin{equation}\lim\limits_{n\to+\infty}\sum_{p=1}^{N_{1}}\left\lvert a_{n,p}-a_{p}\right\rvert=0\end{equation}

		Donc pour tout $n\geqslant\max\left(N_{1},N_{2}\right)$, on a 
		\begin{equation}\left\lvert\sum_{p=1}^{n}a_{n,p}-\sum_{p=1}^{+\infty}a_{p}\right\rvert<\varepsilon\end{equation}
		Ainsi,

		\begin{equation}\boxed{\lim\limits_{n\to+\infty}\sum_{p=1}^{+\infty}a_{n,p}=\sum_{p=1}^{+\infty}a_{p}}\end{equation}

		\item On fixe $p\in\N$, on a 
		\begin{equation}\lim\limits_{n\to+\infty}\left(1-\frac{p}{n}\right)^{n}=e^{-p}\end{equation}
		Pour $x\geqslant -1$, on a $\ln\left(1+x\right)\leqslant x$ donc $\ln\left(1-\frac{p}{n}\right)\leqslant -\frac{p}{n}$ et $a_{n,p}=e^{n\ln\left(1-\frac{p}{n}\right)}\leqslant e^{-p}=b_{p}$
		Donc d'après ce qui précède, 
		
		\begin{equation}\boxed{\lim\limits_{n\to+\infty\left(\left(\frac{1}{n}\right)^{n}+\dots+\left(\frac{n-1}{n}\right)^{n}\right)}=\frac{1}{e-1}}\end{equation}
		
		\begin{figure}[ht!]
			\centering
			\begin{tikzpicture}
				\begin{axis}[
					xmin=-2, xmax=2,
					ymin=-2, ymax=2,
					axis lines=center,
					axis on top=true,
					xlabel=$x$,
					samples=100,
					legend pos=outer north east
				]
				\addplot[blue, ultra thick, restrict y to domain=-2:2] {ln(1+x)};
				\addplot[color=red, ultra thick, restrict y to domain=-2:2] {x};
				\legend{$\ln\left(1+x\right)$,$y=x$}
				\end{axis}
			\end{tikzpicture}
			\caption{$\ln\left(1+x\right)\leqslant x$ pour $x>-1$.}
		\end{figure}
	\end{enumerate}
\end{proof}

\begin{remark}
	C'est faux si on n'a pas l'hypothèse (ii). Par exemple, 
	\begin{equation}a_{n,p}=\frac{1}{\sqrt{n}}\xrightarrow[n\to+\infty]{}0\end{equation}
	pour $p$ fixé mais 
	\begin{equation}\sum_{p=1}^{n}\frac{1}{\sqrt{n}}=\sqrt{n}\xrightarrow[n\to+\infty]{}+\infty\end{equation}
\end{remark}

\begin{proof}
	\phantom{}
	\begin{enumerate}
		\item Pour tout $k\geqslant1$, $\left(u_{kn}\right)_{n\geqslant1}$ est une sous-famille de $\left(u_{n}\right)_{n\geqslant1}$ sommable, donc $\left(u_{kn}\right)_{n\geqslant1}$ est sommable.
		\begin{equation}\boxed{\text{Donc }S_{k}\text{ existe.}}\end{equation}

		\item On a 
		\begin{equation}
		\left\{
			\begin{array}[]{lllll}
				S_{1}-S_{2} &= &u_{1}+u_{3}+\dots+u_{2n+1}+\dots &= &0\\
				S_{1}-S_{2}-S_{3}+S_{6} &= &u_{1}+u_{5}+u_{7}+u_{11}+\dots &= &0\\
				S_{1}-S_{2}-S_{3}-S_{5}+S_{6}+S_{10}+S_{15}-S_{10} &= &u_{1}+u_{7}+u_{11}+\dots &= &0
			\end{array}
		\right.
		\end{equation}
		A la première ligne on enlève les multiples de 2, à la deuxième ligne on enlève les multiples de 2 et 3, à la troisième ligne on enlève les multiples de 2, 3 et 5. Et ainsi de suite.

		Soient donc $p_{1},\dots,p_{N}$ les $N$ premiers nombres premiers. On a 
		\begin{equation}0=\sum_{k=1}^{p_{1}\dots p_{N}}\mu(k)S_{k}=\sum_{k\in\left\{p_{1}^{\alpha_{1}}\dots p_{N}^{\alpha_{N}}\middle|\left(\alpha_{1},\dots,\alpha_{N}\right)\in\left\{0,1\right\}^{N}\right\}}\mu(k)S_{k}\end{equation}
		où si $k=p_{1}^{\alpha_{1}}\dots p_{r}^{\alpha_{r}}$, $\mu(k)=0$ s'il existe $\alpha_{i}\geqslant 2$ et $\mu\left(p_{i_{1}}\dots p_{i_{s}}\right)=\left(-1\right)^{s}$ sinon (fonction de Möbius).

		Soit $n=p_{1}^{\beta_{1}}\dots p_{N}^{\beta_{N}}$. On cherche le coefficient en $u_{n}$ dans la somme. Si $n=1$, c'est 1. Si $n\geqslant 1$, on a 
		\begin{equation}\sum_{k\mid n}\mu(k)=0\end{equation}
		donc 
		\begin{equation}\sum_{k\in\left\{p_{1}^{\alpha_{1}}\dots p_{N}^{\alpha_{N}}\middle|\left(\alpha_{1},\dots,\alpha_{N}\right)\in\left\{0,1\right\}^{N}\right\}}\mu(k)S_{k}=u_{1}+\alpha_{N}\end{equation}
		avec 
		\begin{equation}\alpha_{N}=\sum_{k\in B_{N}}u_{k}\end{equation}
		où $B_{N}\subset\N^{*}$ est tel que $\min\left(B_{N}\right)=p_{N+1}$. On a 
		\begin{equation}\left\lvert \alpha_{N}\right\rvert\leqslant\sum_{k\geqslant p_{N+1}}\left\lvert u_{k}\right\rvert\xrightarrow[N\to+\infty]{}0\end{equation}
		car c'est le reste de $\sum_{n\geqslant 1}\left\lvert u_n\right\rvert$ convergente.

		Donc $u_{1}+\alpha_{N}=0\xrightarrow[N\to+\infty]{}u_{1}$ donc $u_{1}=0$.

		Avec $u_{1}=0$,
		\begin{equation}
		\left\{
			\begin{array}[]{lllll}
				S_{n} &= &u_{n}+u_{2n}+u_{3n}+\dots &= &0\\
				S_{2n} &= & u_{2n}+u_{4n}+u_{6n}+\dots &= &0
			\end{array}
		\right.
		\end{equation}
		et en recommençant avec $u_{n}$ pour tout $n\geqslant1$, on obtient bien 
		\begin{equation}\boxed{u_{n}=0}\end{equation}
	\end{enumerate}
\end{proof}

\begin{proof}
	\phantom{}
	\begin{enumerate}
		\item On prend $u_{n}=0$ pour tout $n\in\N$. Alors $\sum u_{n}=0$ converge donc $\sum f\left(u_{n}\right)=\sum f(0)$ converge. Donc 
		\begin{equation}\boxed{f(0)=0}\end{equation}

		Supposons que $f$ n'est pas continue en 0. Alors il existe $\varepsilon_{0}>$ tel que pour tout $\alpha>0$, il existe $x\in\left[-\alpha,\alpha\right]\colon\left\lvert f(x)\right\rvert\geqslant\varepsilon_{0}$. Pour $\alpha\equiv\alpha_{n}=\frac{1}{n^{2}}$, il existe $x_{n}\in\left[-\frac{1}{n^{2}},\frac{1}{n^{2}}\right]\colon\left\lvert f(x_{n})\right\rvert\geqslant\varepsilon_{0}$.
		$\sum x_{n}$ converge absolument mais $\sum f\left(x_{n}\right)$ diverge grossièrement ce qui est absurde.
		\begin{equation}\boxed{\text{f est continue en 0.}}\end{equation}

		\item Supposons que pour tout $\alpha>0$, il existe $x\in\left]-\alpha,\alpha\right[\colon f\left(-x\right)\neq-f\left(x\right)$.
		On définit $\left(x_{n}\right)_{n\in\N}\xrightarrow[n\to+\infty]{}0$ telle que $f\left(-x_{n}\right)+f\left(x_{n}\right)\neq 0$. Il existe $N_{n}\in\N^{*}$ tel que 
		\begin{equation}N_{n}\left\lvert f\left(-x_{n}\right)+f\left(x_{n}\right)\right\rvert\geqslant1\end{equation}
		(il suffit de prendre $N_{n}=\left\lfloor\frac{1}{\left\lvert f\left(x_{n}\right)+f\left(-x_{n}\right)\right\rvert}\right\rfloor+1$)

		On définit 
		\begin{equation}\left(u_{n}\right)_{n\in\N}=\left(x_{0},-x_{0},x_{0},-x_{0},\dots,\dots,x_{n},-x_{n},\dots,\dots\right)\end{equation}
		où $\left(x_{n},-x_{n}\right)$ apparaît $N_{n}$ fois. On a 
		$\sum_{k=0}^{2N}u_{k}=0$ et $\sum_{k=0}^{2N+1}u_{k}=x_{n}\xrightarrow[n\to+\infty]{}0$. Donc $\sum u_{n}$ converge.

		Si $\sum f\left(u_{n}\right)$ convergeait, alors il existerait $n_{0}\in\N$ tel que pour tout $n\geqslant n_{0}$ alors 
		\begin{equation}\left\lvert\sum_{k=n+1}^{+\infty}f\left(x_{k}\right)\right\rvert<\frac{1}{2}\end{equation}
		De plus, pour $n\geqslant n_{0}$, on a 
		\begin{equation}\left\lvert f\left(x_n\right)+f\left(-x_n\right)+\dots+f\left(x_n\right)+f\left(-x_n\right)+f\left(x_{n+1}\right)+f\left(-x_{n+1}\right)+\dots\right\rvert<\frac{1}{2}\end{equation}
		où $\left(f\left(x_{n}\right),f\left(-x_{n}\right)\right)$ apparaît $N_{n}$ fois.
		Comme 
		\begin{equation}\left\lvert f\left(x_{n+1}\right)+f\left(-x_{n+1}\right)+\dots\right\rvert<\frac{1}{2}\end{equation}
		on a 
		\begin{equation}\left\lvert f\left(x_n\right)+f\left(-x_n\right)+\dots+f\left(x_n\right)+f\left(-x_n\right)\right\rvert=N_{n}\left\lvert f\left(x_n\right)+f\left(-x_n\right)\right\rvert<1\end{equation}
		ce qui est absurde.

		\begin{equation}\boxed{\text{Donc f est impaire au voisinage de 0.}}\end{equation}

		\item Supposons que pour tout $\beta>0$, il existe $(x,y)\in\left]-\beta,\beta\right[^{2}$ avec $f\left(x+y\right)\neq f\left(x\right)+f\left(y\right)$. Alors il existe $\left(x_{n}\right)_{n\in\N}$ et $\left(y_{n}\right)_{n\in\N}$ qui tendent vers 0 telles que pour tout $n\in\N$, il existe $M_{n}\in\N$,
		\begin{equation}M_{n}\left\lvert f\left(x_{n}+y_{n}\right)-f\left(x_{n}\right)-f\left(y_{n}\right)\right\rvert\geqslant 1\end{equation}

		On définit alors 
		\begin{equation}\left(u_{n}\right)_{n\in\N}=\left(x_{0}+y_{0},-x_{0},-y_{0},\dots,x_{0}+y_{0},-x_{0},-y_{0},\dots,x_{n}+y_{n},-x_{n},-y_{n},\dots\right)\end{equation}
		où $\left(x_{n}+y_{n},-x_{n},-y_{n}\right)$ apparaît $M_{n}$ fois. On a 
		\begin{equation}\sum_{k=0}^{N}u_{k}=
		\left\{
			\begin{array}[]{ll}
				0 & \text{si }N\equiv0[3]\\
				x_{n}+y_{n} & \text{si }N\equiv1[3]\\
				y_{n} & \text{si }N\equiv2[3]
			\end{array}
		\right.\xrightarrow[N\to+\infty]{}0
		\end{equation}
		donc $\sum u_{n}$ converge.

		Si $\sum f\left(u_{n}\right)$ convergeait, alors il existerait $n_{0}\in\N$ tel que pour tout $n\geqslant n_{0}$, 
		\begin{equation}\left\lvert\sum_{k=n+1}^{+\infty}f\left(u_{k}\right)\right\rvert<\frac{1}{2}\end{equation}
		De plus, d'après 2., il existe $n_{1}\in\N$ tel que pour tout $n\geqslant n_{1}$, on a $f\left(-x_{n}\right)+f\left(-y_{n}\right)=-f\left(x_{n}\right)-f\left(y_{n}\right)$ donc pour tout $n\geqslant\max\left(n_{0},n_{1}\right)$, on a 
		\begin{equation}\left\lvert f\left(x_{n}+y_{n}\right)+f\left(-x_{n}\right)+f\left(-y_{n}\right)\right\rvert\times M_{n}=\left\lvert f\left(x_{n}+y_{n}\right)-f\left(x_{n}\right)-f\left(y_{n}\right)\right\rvert\times M_{n}<1\end{equation}
		ce qui est absurde.

		\begin{equation}\boxed{\text{Donc f est linéaire au voisinage de 0.}}\end{equation}

		\item Soit $k\in\Z^{*},x\in\R,\left\lvert x\right\rvert\leqslant\frac{\beta}{\left\lvert k\right\rvert}$. Par récurrence, on a $f\left(kx\right)=kf\left(x\right)$.
		
		Si $\left\lvert x\right\rvert<\beta$ et si $\frac{x}{\beta}\in\Q$, on a 
		\begin{equation}\frac{x}{\frac{\beta}{2}}=\frac{p}{q}\end{equation}
		donc en posant $\lambda=\frac{2}{\beta}f\left(\frac{\beta}{2}\right)$, on a
		\begin{equation}f\left(x\right)=f\left(\frac{p\beta}{2q}\right)=\frac{p}{q}f\left(\frac{\beta}{2}\right)=\frac{p}{q}\frac{\beta}{2}\lambda=\lambda x\end{equation}

		Si $\frac{x}{\beta}\notin\Q$, il existe une suite de rationnels $(r_{n})_{n\in\N}$ telle que $\lim\limits_{n\to+\infty}r_{n}=\frac{x}{\frac{\beta}{2}}$. On a alors 
		\begin{align}
			f\left(x\right)
			&=f\left(\left(x-\frac{r_{n}\beta}{2}\right)+r_{n}\frac{\beta}{2}\right)\\
			&=f\left(x-\frac{r_{n}\beta}{2}\right)+f\left(\frac{r_{n}\beta}{2}\right)
		\end{align}
		et $x-\frac{r_{n}\beta}{2}\xrightarrow[n\to+\infty]{}0$ et donc $f\left(x-\frac{r_{n}\beta}{2}\right)\xrightarrow[n\to+\infty]{}0$ d'après 1. et $r_{n}f\left(\frac{\beta}{2}\right)\xrightarrow[n\to+\infty]{}\lambda x$
		
		\begin{equation}\boxed{\text{Donc f est une homothétie au voisinage de 0.}}\end{equation}
	\end{enumerate}
\end{proof}

\end{document}
\documentclass[12pt]{article}
\usepackage{style/style_sol}

\begin{document}

\begin{titlepage}
	\centering
	\vspace*{\fill}
	\Huge \textit{\textbf{Solutions MP/MP$^*$\\ Probabilités sur un univers dénombrable}}
	\vspace*{\fill}
\end{titlepage}

\begin{proof}
    \phantom{}
    \begin{enumerate}
        \item On note P:`le lancer initial donne pile', F:`le lancer initial donne face', $B_{k}$:`la k-ième boule est blanche', $N_{k}$:`la k-ième boule est noire'.
        
        On a 
        \begin{equation}
            \P\left(B_{k}\right)=\P\left(P\right)\P_{P}\left(B_{k}\right)+\P\left(F\right)\P_{F}\left(B_{k}\right)=\frac{1}{2}\frac{k}{k+1}+\frac{1}{2}\frac{1}{k+1}
        \end{equation}
        donc 
        \begin{equation}
            \boxed{\P\left(B_{k}\right)=\frac{1}{2}}
        \end{equation}

        \item On a 
        \begin{equation}
            \boxed{\P_{B_{k}}\left(P\right)=\P_{P}\left(B_{k}\right)\frac{\P\left(P\right)}{\P\left(B_{k}\right)}=\frac{k}{k+1}\xrightarrow[k\to+\infty]{}1}
        \end{equation}

        \item On a 
        \begin{align}
            \P\left(B_{1}\bigcap\dots\bigcap B_{k}\right)
            &=\frac{1}{2}\P_{P}\left(B_{1}\bigcap\dots\bigcap B_{k}\right)+\frac{1}{2}\P_{F}\left(B_{1}\bigcap\dots\bigcap B_{k}\right)\\
            &=\frac{1}{2}\left(\prod_{j=1}^{k}\frac{j}{j+1}+\prod_{j=1}^{k}\frac{1}{j+1}\right)\\
            &\boxed{=\frac{1}{2}\left(\frac{1}{k+1}+\frac{1}{(k+1)!}\right)}
        \end{align}

        \item On a 
        \begin{align}
            \P\left(B_{k}\bigcap B_{k+1}\right)
            &=\frac{1}{2}\left(\frac{k}{k+1}\times\frac{k+1}{k+2}+\frac{1}{k+1}\times\frac{1}{k+2}\right)\\
            &=\frac{1}{2}\left(\frac{k\left(k+1\right)+1}{\left(k+1\right)\left(k+2\right)}\right)
        \end{align}

        Donc on a indépendance si et seulement si 
        \begin{align}
            \P\left(B_{k}\bigcap B_{k+1}\right)=\P\left(B_{k}\right)\P\left(B_{k+1}\right)=\frac{1}{4}
            &\Leftrightarrow \frac{k\left(k+1\right)+1}{\left(k+1\right)\left(k+2\right)}=\frac{1}{2}\\
            &\Leftrightarrow 2k\left(k+1\right)+2=\left(k+2\right)\left(k+2\right)\\
            &\Leftrightarrow 2k^{2}+2k=k^{2}+3k\\
            &\Leftrightarrow \boxed{k=1}
        \end{align}
        Ainsi, seuls les deux premiers tirages sont indépendants.
    \end{enumerate}
\end{proof}

\begin{remark}
    Seuls les deux premiers tirages sont indépendants car le premier tirage est indépendant du lancer de pièce.
\end{remark}

\begin{proof}
    \phantom{}
    \begin{enumerate}
        \item 
        \begin{equation}
            \boxed{p_{0}=1,q_{0}=0,p_{N}=0,q_{N}=1}
        \end{equation}
        \item Soit $a\in\left\llbracket 1,N-1\right\rrbracket$. Puisque les lancers de pièce sont indépendants, on peut partitionner selon le résultat du premier lancer. On a donc [probabilités conditionnelles]
        \begin{equation}
            \boxed{p_{a}=p\times p_{a+1}+q\times p_{a-1}}
        \end{equation}

        L'équation caractéristique est 
        \begin{equation}
            pX^{2}-x+q=0
        \end{equation}
        On a $\Delta=1-4pq=1-4\left(1-p\right)p=4p^{2}-4p+1=\left(1-2p\right)^{2}$.

        Ainsi, si $p\neq\frac{1}{2}$, il existe $\left(\alpha,\beta\right)\in\R^{2}$ tels que pour tout $a\in\left\llbracket 0,N\right\rrbracket$, on a 
        \begin{equation}
            p_{a}=\alpha+\beta\left(\frac{q}{p}\right)^{a}
        \end{equation}
        Grâce aux valeurs en $a=0,a=N$, on en déduit que 
        \begin{equation}
            \boxed{p_{a}=\frac{1}{1-\left(\frac{q}{p}\right)^{N}}\times\left(\left(\frac{q}{p}\right)^{a}-\left(\frac{q}{p}\right)^{N}\right)}
        \end{equation}

        Si $p=\frac{1}{2}$, il existe $\left(\alpha,\beta\right)\in\R^{2}$ tels que 
        \begin{equation}
            p_{a}=\alpha a+\beta
        \end{equation}
        Grâce aux valeurs en $a=0,a=N$, on en déduit que 
        \begin{equation}
            \boxed{p_{a}=\frac{1}{N}\left(N-a\right)}
        \end{equation}

        \item Pour tout $a\in\left\llbracket 1,N-1\right\rrbracket$, on a 
        \begin{equation}
            q_{a}=pq_{a+1}+qp_{a-1}
        \end{equation}
        donc pour tout $a\in\left\llbracket 1,N-1\right\rrbracket$, on a 
        \begin{equation}
            p_{a}+q_{a}=p\left(p_{a+1}+q_{a+1}\right)+q\left(p_{a-1}+q_{a-1}\right)
        \end{equation}
        Comme $p_{0}+q_{0}=p_{N}+q_{N}=1$, on a pour tout $a\in\left\llbracket 0,N\right\rrbracket$,
        \begin{equation}
            \boxed{p_{a}+q_{a}=1}
        \end{equation}

        Ainsi, le jeu s'arrête presque sûrement en temps fini.
    \end{enumerate}
\end{proof}

\begin{proof}
    \phantom{}
    \begin{enumerate}
        \item Les tirs sont indépendants donc 
        \begin{equation}
            \boxed{
                \begin{array}[]{l}
                    \P\left(A_{n}\right)=\left(1-a\right)^{n}\times\left(1-b\right)^{n}\times a\\
                    \P\left(B_{n}\right)=\left(1-a\right)^{n}\times\left(1-b\right)^{n}\times \left(1-a\right)\times b
                \end{array}
            }    
        \end{equation}
        
        \item On a 
        \begin{equation}
            G_{A}=\bigcup_{n\in\N}A_{n}
        \end{equation}
        réunion disjointe. Donc 
        \begin{equation}
            \boxed{
                \begin{array}[]{l}
                    \P\left(G_{A}\right)=\sum_{n=0}^{+\infty}\P\left(A_{n}\right)=\frac{a}{1-\left(1-a\right)\left(1-b\right)}=\frac{a}{a+b-ab}\\
                    \P\left(G_{B}\right)=\sum_{n=0}^{+\infty}\P\left(B_{n}\right)=\frac{b\left(1-a\right)}{a+b-ab}\\
                \end{array}
            }
        \end{equation}
        
        Ainsi, 
        \begin{equation}
            \boxed{\P\left(G_{A}\right)+\P\left(G_{B}\right)=1}
        \end{equation}

        \item On a $\P\left(G_{A}\right)=\P\left(G_{B}\right)$ si et seulement si 
        \begin{equation}
            \frac{a}{1-a}=b
        \end{equation}
        Cela implique que $\frac{a}{1-a}\in]0,1[$ ce qui est possible uniquement (après étude de fonction) si
        \begin{equation}
            \boxed{a\in\left]0,\frac{1}{2}\right[\text{ et }b=\frac{a}{1-a}}
        \end{equation}
    \end{enumerate}
\end{proof}

\begin{proof}
    \phantom{}
    \begin{enumerate}
        \item Pour $n\in\N^{*}$, on pose $E_{n}$:`Le joueur gagne au bout du n-ième lancer' (évènement disjoints) et G:`Le joueur gagne'. On a $G\cup_{n\in\N^{*}}E_{n}$. Donc 
        \begin{equation}
            \boxed{\P\left(G\right)=\sum_{n\in\N^{*}}\P\left(E_{n}\right)=\sum_{n\in\N^{*}}\left(\frac{1}{2}\right)^{n}\times\frac{1}{n}=\ln(2)}
        \end{equation}

        \item On note $P_{n}$:`le joueur obtient pile au n-ième lancer', P:`il obtient pile'. On a 
        \begin{equation}
            \P_{G}\left(P_{n}\right)=\frac{\P\left(G\bigcap P_{n}\right)}{\P\left(G\right)}=\frac{\P_{P_{n}}\left(G\right)\times\P\left(P_{n}\right)}{\P\left(G\right)}
        \end{equation}
        donc 
        \begin{equation}
            \boxed{\P_{G}\left(P_{n}\right)=\frac{\frac{1}{n}\left(\frac{1}{2}    \right)^{n}}{\ln\left(2\right)}}
        \end{equation}

        Puis 
        \begin{equation}
            \boxed{\P_{G}\left(P\right)=\sum_{n\in\N^{*}}\P_{G}\left(P_{n}\right)=1}
        \end{equation}
    \end{enumerate}
\end{proof}

\begin{remark}
    On a utilisé le résultat suivant: pour tout $x\in\left]0,1\right[$, 
    \begin{equation}
        \sum_{n=1}^{+\infty}\frac{x^{n}}{n}=-\ln\left(1-x\right)
    \end{equation}
    Soit on connaît le résultat avec les séries entières, soit on le redémontre à la main: pour $N\geqslant1$, on a 
    \begin{align}
        \sum_{n=1}^{N}\frac{x^{n}}{n}
        &=\int_{0}^{1}\sum_{n=1}^{N}x^{n}t^{n-1}dt\\
        &=x\int_{0}^{1}\frac{1-\left(xt\right)^{N}}{1-xt}dt\\
        &=\underbrace{\int_{0}^{1}\frac{x}{1-xt}dt}_{=\left[\ln\left(1-xt\right)\right]_{0}^{1}}+R_{N}
    \end{align}
    avec $\left\lvert R_{N}\right\rvert\leqslant\frac{x^{N+1}}{1-x}\xrightarrow[N\to+\infty]{}0$ d'où le résultat.
\end{remark}

\begin{proof}
    \phantom{}
    \begin{enumerate}
        \item On a 
        \begin{equation}
            \sum_{k=0}^{+\infty}p_{k}=p_{0}+p_{1}+\sum_{k=2}^{+\infty}\frac{1-2\alpha}{2^{k-1}}=2\alpha+\left(1-2\alpha\right)\times\sum_{k=1}^{+\infty}\frac{1}{2^{k}}=1
        \end{equation}
        donc 
        \begin{equation}
            \boxed{\text{c'est une probabilité sur }\N.}
        \end{equation}

        \item Pour tout $k\in\N$, on note $E_{k}$:`la famille a $k$ enfants et exactement 2 garçons', $E$:`la famille a exactement 2 garçons', $A_{k}$:`la famille a $k$ enfants'.
        
        On a alors 
        \begin{align}
            \P\left(E\right)
            &=\sum_{k=2}^{+\infty}\P_{A_{k}}\left(E_{k}\right)\times\P\left(A_{k}\right)\\
            &=\sum_{k=2}^{+\infty}\binom{k}{2}\left(\frac{1}{2}\right)^{k}\times p_{k}\\
            &=\sum_{k=2}^{+\infty}\frac{k\left(k-1\right)}{2^{k+1}}\times\frac{1-2\alpha}{2^{k-1}}\\
            &=\left(1-2\alpha\right)\sum_{k=2}^{+\infty}\frac{k\left(k-1\right)}{2^{2k}}\\
            &=\left(1-2\alpha\right)\sum_{k=0}^{+\infty}\frac{\left(k+1\right)\left(k+2\right)}{2^{2k+4}}\\
            &=\frac{1}{16}\left(1-2\alpha\right)\sum_{k=0}^{+\infty}\frac{\left(k+1\right)\left(k+2\right)}{4^{k}}=\frac{1}{16}\left(1-2\alpha\right)\times\frac{1}{\left(\frac{3}{4}\right)^{3}}\\
            &\boxed{=\frac{4\left(1-2\alpha\right)}{27}}
        \end{align}

        \item On note $F$:`la famille a au moins 2 filles', $F_{k}$:`la famille a exactement $k$ filles et au moins 4 enfants', $G$:`la famille a au moins 2 garçons', $G_{k}$:`la famille a exactement $k$ garçons et au moins 4 enfants'.
        
        On a 
        \begin{equation}
            \P_{G}\left(G\right)=\frac{\P\left(F\cap G\right)}{\P\left(G\right)}
        \end{equation}
        et $\overline{F\cap G}=\overline{F}\cup\overline{G}=F_{0}\cup F_{1}\cup G_{0}\cup G_{1}$.
        Donc, comme $\P\left(F_{0}\right)=\P\left(G_{0}\right)$ et $\P\left(F_{1}\right)=\P\left(G_{1}\right)$, on a $\P\left(F\cap G\right)=1-2\left(\P\left(G_{0}\right)+\P\left(G_{1}\right)\right)$.

        On a alors 
        \begin{align}
            \P\left(G_{0}\right)
            &=\sum_{k=4}^{+\infty}\binom{k}{0}\left(\frac{1}{2}\right)^{k}p_{k}\\
            &=\sum_{k=4}^{+\infty}\frac{1-2\alpha}{2^{2k-1}}\\
            &=2\left(1-2\alpha\right)\frac{1}{4^{4}}\times\frac{1}{1-\frac{1}{4}}\\
            &=2\left(1-2\alpha\right)\times\frac{1}{4^{3}}\times\frac{1}{3}
        \end{align}
        et 
        \begin{align}
            \P\left(G_{1}\right)
            &=\sum_{k=4}^{+\infty}\binom{k}{1}\left(\frac{1}{2}\right)^{k}p_{k}\\
            &=\sum_{k=4}^{+\infty}k\times\frac{1}{2^{k}}\times\frac{1-2\alpha}{2^{k-1}}\\
            &=\left(1-2\alpha\right)\sum_{k=4}^{+\infty}\frac{k}{2^{2k-1}}\\
            &=\left(1-2\alpha\right)\times\frac{2}{4}\sum_{k=4}^{+\infty}\frac{k}{4^{k-1}}\\
            &=\frac{1-2\alpha}{2}\sum_{k=3}^{+\infty}\frac{k+1}{4^{k}}\\
            &=\frac{1-2\alpha}{2}\times\left(\frac{1}{\left(1-\frac{1}{4}\right)^{2}}-1-\frac{2}{4}-\frac{3}{4^{2}}\right)
        \end{align}

        et on calcule enfin 
        \begin{equation}
            \boxed{\P\left(G\right)=1-\P\left(G_{0}\right)-\P\left(G_{1}\right)}
        \end{equation}
    \end{enumerate}
\end{proof}

\begin{proof}
    Pour tout $k\geqslant1$, on note $A_{k}$:`A gagne à son lancé $k$' et $B_{k}$ de manière équivalente pour le joueur $B$. On note $G_{A}$:`A gagne' et de même pour B. On a ainsi
    \begin{equation}
        G_{A}=\bigcup_{k\geqslant 1}A_{k}
    \end{equation}
    (réunion disjointe) et pareil pour $G_{B}$. On a 
    \begin{equation}
        \P\left(A_{k}\right)=\left(1-\frac{5}{36}\right)^{k-1}\times\left(1-\frac{1}{6}\right)^{k-1}\times\frac{5}{36}
    \end{equation}
    d'où 
    \begin{equation}
        \P\left(G_{A}\right)=\frac{5}{36}\times\frac{1}{1-\left(1-\frac{5}{36}\right)\left(1-\frac{1}{6}\right)}
    \end{equation}
    et pareil 
    \begin{equation}
        \boxed{\P\left(G_{B}\right)=\frac{1}{6}\times\left(1-\frac{5}{36}\right)\times\frac{1}{1-\left(1-\frac{5}{36}\right)\left(1-\frac{1}{6}\right)}>\P\left(G_{A}\right)}
    \end{equation}
    et 
    \begin{equation}
        \P\left(G_{A}\right)+\P\left(G_{B}\right)=1
    \end{equation}
    donc $G_{A}\cup G_{B}$ est presque sur.
\end{proof}

\begin{proof}
    Soit $k\in\left\llbracket0,\left\lfloor\frac{n}{2}\right\rfloor\right\rrbracket$. La probabilité que l'on tire $2k$ boules blanches est (loi binomiale):
    \begin{equation}
        \binom{n}{2k}\times\left(\frac{a}{a+b}\right)^{2k}\times\left(\frac{b}{a+b}\right)^{n-2k}
    \end{equation}
    donc la probabilité que le nombre de boules blanches tirées soit pair est 
    \begin{equation}
        \P_{P}=\sum_{0\leqslant 2k\leqslant n}\binom{n}{2k}\times\left(\frac{a}{a+b}\right)^{2k}\times\left(\frac{b}{a+b}\right)^{n-2k}
    \end{equation}
    
    De même, la probabilité que le nombre de boules blanches tirées soit impair est 
    \begin{equation}
        \P_{I}=\sum_{0\leqslant 2k+1\leqslant n}\binom{n}{2k+1}\times\left(\frac{a}{a+b}\right)^{2k+1}\times\left(\frac{b}{a+b}\right)^{n-2k-1}
    \end{equation}

    On a alors 
    \begin{equation}
        \P_{P}+\P_{I}=1
    \end{equation}
    et 
    \begin{equation}
        \P_{P}-\P_{I}=\sum_{k'=0}^{n}\binom{n}{k'}\left(-1\right)^{k'}\left(\frac{a}{a+b}\right)^{k'}\left(\frac{b}{a+b}\right)^{n-k'}=\left(\frac{b-a}{a+b}\right)^{n}
    \end{equation}

    On a donc 
    \begin{equation}
        \boxed{\P_{P}=\frac{1}{2}\left(1+\left(\frac{b-a}{a+b}\right)^{n}\right)}
    \end{equation}
\end{proof}

\begin{remark}
    Si on note $\P_{3}$ la probabilité que le nombre de boules blanches tirées soit multiple de 3:
    \begin{equation}
        \P_{3}=\sum_{0\leqslant 3k\leqslant n}\binom{n}{3k}\left(\frac{a}{a+b}\right)^{3k}\left(\frac{b}{a+b}\right)^{n-3k}
    \end{equation}
    On note $\P_{2}$ la probabilité pour que le nombre de boules blanches tirées soit congru à 2 module 3, et on définit $\P_{1}$ de même.
    Alors on a 
    \begin{equation}
        \left\{
            \begin{array}[]{rcl}
                \P_{1}+\P_{2}+\P_{3} &= &1\\
                \mathrm{j}\P_{1}+\mathrm{j}^{2}\P_{2}+\P_{3} &= &\left(\frac{b+\mathrm{j}a}{a+b}\right)^{n}\\
                \mathrm{j}^{2}\P_{1}+\mathrm{j}\P_{1}+\P_{3} &= &\left(\frac{b+\mathrm{j}^{2}a}{a+b}\right)^{n}
            \end{array}
        \right.
    \end{equation}
    et donc 
    \begin{equation}
        \P_{3}=\frac{1}{3}\left(1+\left(\frac{b+\mathrm{j}a}{a+b}\right)^{n}+\left(\frac{b+\mathrm{j}^{2}a}{a+b}\right)^{n}\right)
    \end{equation}
\end{remark}

\begin{proof}
    Soit pour $i\in\left\llbracket 1,n\right\rrbracket$, 
    \begin{equation}
        A_{i}=\left\{\sigma\in\Sigma_{n}\middle|\sigma(i)=i\right\}
    \end{equation}
    \begin{equation}
        A=\left\{\sigma\in\Sigma_{n}\middle|\sigma\text{ a un point fixe}\right\}
    \end{equation}
    On a 
    \begin{equation}
        A=\bigcup_{i=1}^{n}A_{i}
    \end{equation}

    On a 
    \begin{equation}
        \left\lvert A\right\rvert=\sum_{k=1}^{n}(-1)^{k-1}\sum_{\substack{J\subset\left\llbracket 1,n\right\rrbracket\\\left\lvert J\right\rvert=k}}\left\lvert\bigcap_{i\in J}A_{i}\right\rvert
    \end{equation}
    Il y a $\binom{n}{k}$ tels $J$, et on a 
    \begin{equation}
        \left\lvert\bigcap_{i\in J}A_{i}\right\rvert=\left\lvert\left\{\sigma\in\Sigma_{n}\middle|\forall i\in J,\sigma(i)=i\right\}\right\rvert=(n-k)!
    \end{equation}
    Ainsi, 
    \begin{equation}
        \left\lvert A\right\rvert=\sum_{k=1}^{n}(-1)^{k-1}\binom{n}{k}(n-k)!
    \end{equation}
    donc 
    \begin{equation}
        \boxed{p_{n}=\sum_{k=1}^{n}\frac{(-1)^{k-1}}{k!}\xrightarrow[n\to+\infty]{}\sum_{k=1}^{+\infty}\frac{(-1)^{k-1}}{k!}=-\left(\frac{1}{e}-1\right)=1-\frac{1}{e}}
    \end{equation}
\end{proof}

\begin{proof}
    \phantom{}
    \begin{enumerate}
        \item 
        \begin{equation}
            \boxed{p_{N}(0)=0,p_{N}(1)=1}    
        \end{equation}

        \item Pour tout $n\in\left\llbracket 1,N-1\right\rrbracket$, on a 
        \begin{equation}
            p_{N}(n)=p\times p_{N}(n+1)+(1-n)\times p_{N}(n-1)
        \end{equation}
        et l'équation caractéristique est $X^{2}-\frac{1}{p}X+\frac{1-p}{p}$ et le discriminant vaut 
        \begin{equation*}
            \Delta=\left(\frac{1}{p}-2\right)^{2}\geqslant0.
        \end{equation*}Donc les solutions sont $r_{1}=1$ et $r_{2}=\frac{q}{p}$. Ainsi, pour tout $n\in\left\llbracket 1,N-1\right\rrbracket$,
        \begin{equation}
            p_{N}(n)=\lambda+\mu\left(\frac{q}{p}\right)^{n}
        \end{equation}
        avec $\left(\lambda,\mu\right)\in\R^{2}$.
        
        Avec les conditions initiales, on trouve 
        \begin{equation}
            \left\{
                \begin{array}[]{rcl}
                    \mu &= &\frac{1}{\left(\frac{q}{p}\right)^{N}-1}\\
                    \lambda &= &\frac{1}{1-\left(\frac{q}{p}\right)^{N}}
                \end{array}
            \right.
        \end{equation}
        donc 
        \begin{equation}
            \boxed{
            p_{N}(n)=\frac{1-\left(\frac{q}{p}\right)^{n}}{1-\left(\frac{q}{p}\right)^{N}}\xrightarrow[N\to+\infty]{}
            \left\{
                \begin{array}[]{ll}
                    0 &\text{si }q>p\text{ i.e.~}p<\frac{1}{2}\\
                    1-\left(\frac{q}{p}\right)^{n} &\text{si }q<p\text{ i.e.~}p>\frac{1}{2}
                \end{array}
            \right.}
        \end{equation}
        On vérifie d'ailleurs que l'arrêt en temps fini est presque sûr: $p_{N}(n)+q_{N}(n)=1$ (utiliser la relation de récurrence et les conditions initiales).
    \end{enumerate}
\end{proof}

\begin{proof}
    \phantom{}
    \begin{enumerate}
        \item On note $A_{n}$:`la première boule blanche apparaît au $n$-ième tirage' et $B_{n}$:`on tire une boule noire au $n$-ième tirage'. On a 
        \begin{equation}
            A_{n}=\bigcap_{i=1}^{n-1}B_{i}\bigcap\overline{B_{n}}
        \end{equation}
        ce qui implique donc 
        \begin{align}
            \P\left(A_{n}\right)
            &=p_{n}\\
            &=\P\left(B_{1}\right)\P_{B_{1}}\left(B_{2}\right)\dots\P_{B_{1}\cap \dots \cap B_{n-1}}\left(\overline{B_{n}}\right)\\
            &=\frac{1}{2}\times\frac{2}{3}\times\dots\times\frac{n-1}{n}\times\frac{1}{n+1}\\
            &=\boxed{\frac{1}{n(n+1)}}
        \end{align}
        et par sommation téléscopique, on a 
        \begin{equation}
            \boxed{\sum_{n=1}^{+\infty}p_{n}=1}
        \end{equation}
        Donc on tire une boule blanche presque sûrement.

        \item On utilise le même principe: pour $n\geqslant1$,
        \begin{equation}
            \boxed{\P\left(A_{n}\right)=p_{n}=\frac{1}{2}\times\frac{c+1}{c+2}\times\frac{2c+1}{2c+2}}\times\dots\times\frac{\left(n-2\right)c+1}{\left(n-2\right)c+2}\times\frac{1}{\left(n-1\right)c+2}
        \end{equation}
        Comme les $\left(A_{n}\right)_{n\in\N^{*}}$ sont incompatibles, on a 
        \begin{equation}
            \sum_{n\geqslant1}\P\left(A_{n}\right)=\P\left(\bigcup_{n\geqslant1} A_{n}\right)\leqslant1
        \end{equation}
        donc 
        \begin{equation}
            \boxed{\text{la série converge.}}
        \end{equation}

        On peut montrer à nouveau que le tirage d'une boule blanche reste presque sûr. En effet, on a 
        \begin{equation}
            \frac{p_{n+1}}{p_{n}}=\frac{nc+2-c-1}{nc+2}=1-\frac{c+1}{nc+2}=a-\frac{c+1}{nc}+\underset{n\to+\infty}{O}\left(\frac{1}{n^{2}}\right)
        \end{equation}
        D'après la règle de Raabe-Duhamel, il existe $K>0$ tel que 
        \begin{equation}
            p_{n}\underset{n\to+\infty}{\sim}\frac{K}{n^{\frac{c+1}{c}}}
        \end{equation}
        avec $\frac{c+1}{c}>1$. Notamment, $\lim\limits_{n\to+\infty}np_{n}=0$. Comme 
        \begin{equation}
            \left(nc+2\right)p_{n+1}=\left(\left(n-1\right)c+1\right)p_{n}
        \end{equation}
        on a 
        \begin{align}
            \sum_{n=1}^{+\infty}ncp_{n+1}-\left(n-1\right)cp_{n}
            &=\sum_{n=1}^{+\infty}p_{n}-2p_{n+1}\\
            &=\sum_{n=1}^{+\infty}p_{n}-2\left(\sum_{n=1}^{+\infty}p_{n}-u_{1}\right)
        \end{align}
        La première somme est téléscopique et vaut 0, et $u_{1}=\frac{1}{2}$ donc on trouve bien 
        \begin{equation}
            \sum_{n=1}^{+\infty}p_{n}=1
        \end{equation}
    \end{enumerate}
\end{proof}

\begin{remark}
    On peut contourner la règle de Raabe-Duhamel. On écrit 
    \begin{align}
        \ln\left(p_{n+1}\right)
        &=-\ln\left(2\right)+\sum_{k=1}^{n-1}\ln\left(\frac{kc+1}{kc+2}\right)-\ln\left(nc+2\right)\\
        &=\sum_{k=1}^{n-1}\ln\left(1-\frac{1}{kc+2}\right)-\ln\left(n\right)+\ln\left(c\right)-\ln(2)+\underset{n\to+\infty}{o}\left(1\right)\\
        &=-\sum_{k=1}^{n-1}\left(\frac{1}{kc}+\underset{k\to+\infty}{O}\left(\frac{1}{k^{2}}\right)\right)-\ln\left(n\right)-A+\underset{n\to+\infty}{o}\left(1\right)\\
        &=-\frac{1}{c}\left(\ln\left(n\right)+\gamma+\underset{n\to+\infty}{o}\left(1\right)\right)-\ln\left(n\right)-A+\underset{n\to+\infty}{o}\left(1\right)\\
        &=-\ln\left(n\right)\left(1+\frac{1}{c}\right)+A'+\underset{n\to+\infty}{o}\left(1\right)
    \end{align}
    Ainsi, 
    \begin{equation}
        p_{n+1}\underset{n\to+\infty}{\sim}\frac{K}{n^{1+\frac{1}{c}}}
    \end{equation}
    donc la série converge.
\end{remark}

\begin{proof}
    On a 
    \begin{equation}
        u_{n+1}=q\times 1+p\times u_{n}^{2}
    \end{equation}
    car soit la bactérie meure au premier jour, soit les deux descendants n'ont plus de lignée au $n$-ième jour (on a $u_{n}^{2}$ car les lignées des deux descendants sont indépendantes).

    Soit \function{f}{[0,1]}{\R}{x}{q+px^{2}}
    Si $x\in[0,1]$, on a $f(x)\in[0,1]$ car $f(1)=q+p=1$. Soit $g(x)=f(x)-x$. On a 
    \begin{equation}
        g(x)=p\left(x-1\right)\left(x-\frac{p}{q}\right)
    \end{equation}
    \begin{itemize}
        \item Si $1\leqslant\frac{p}{q}$: on a pour tout $x\in[0,1[$, $g(x)>0$ et $g(1)=0$. Donc si 
        \begin{equation}
            \boxed{\lim\limits_{n\to+\infty}u_{n}=1}
        \end{equation}
        car c'est une suite croissante, majorée, convergente vers le point fixe 1.

        \item Si $1>\frac{q}{p}$: si $x\in\left[0,\frac{q}{p}\right[$, on a $g(x)>0$, si $x\in\left]\frac{q}{p},1\right[$, $g(x)<0$ et $g\left(\frac{q}{p}\right)=0$.
        
        Par récurrence, comme $u_{0}=0$, pour tout $n\in\N$, $u_{n}\in\left[0,\frac{q}{p}\right[$ donc (suite croissante majorée qui converge vers le point fixe $\frac{q}{p}$) donc 
        \begin{equation}
            \boxed{\lim\limits_{n\to+\infty}u_{n}=\frac{q}{p}}
        \end{equation}
    \end{itemize}

    On a bien 
    \begin{equation}
        \boxed{\lim\limits_{n\to+\infty}u_{n}=\min\left(1,\frac{q}{p}\right)}
    \end{equation}

    Ainsi, la lignée s'éteint presque sûrement si et seulement si $\frac{q}{p}\geqslant1$ i.e.~$p\leqslant\frac{1}{2}$. Sinon, la probabilité d'extinction est $\frac{q}{p}$.

    Si $p=\frac{1}{2}$, on pose $\varepsilon_{n}=1-u_{n}\xrightarrow[n\to+\infty]{}0$. On a 
    \begin{equation}
        u_{n+1}=\frac{1}{2}\left(1+u_{n}^{2}\right)
    \end{equation}
    d'où 
    \begin{equation}
        \varepsilon_{n+1}=1-u_{n+1}=\varepsilon_{n}\left(1-\frac{\varepsilon}{2}\right)
    \end{equation}

    Soit $\alpha\in\R$, on a 
    \begin{equation}
        \varepsilon_{n+1}^{\alpha}=\varepsilon_{n}^{\alpha}\left(1-\frac{\varepsilon_{n}}{2}\right)^{\alpha}=\varepsilon_{n}^{\alpha}-\frac{\alpha\varepsilon_{n}^{\alpha+1}}{2}+\underset{n\to+\infty}{o}\left(\varepsilon_{n}^{\alpha+1}\right)
    \end{equation}

    On choisit $\alpha=-1$, on a
    \begin{equation}
        \frac{1}{\varepsilon_{n+1}}-\frac{1}{\varepsilon_{n}}\xrightarrow[n\to+\infty]{}\frac{1}{2}
    \end{equation}

    D'après le lemme de Césaro, on a $\frac{1}{\varepsilon_{n}}\underset{n\to+\infty}{\sim}\frac{n}{2}$ d'où 
    \begin{equation}
        \boxed{\varepsilon_{n}\underset{n\to+\infty}{\sim}\frac{2}{n}}
    \end{equation}
\end{proof}

\begin{proof}
    On note $E_{n}$:`la puce est en $0$ à l'instant $2n$' et $B_{n}$:`la puce repasse pour la première fois en $0$ à l'instant $2n$'.

    Soit $E$:`la puce repasse par l'origine'. On a 
    \begin{equation}
        E=\bigcup_{n\in\N^{*}}E_{n}=\bigcup_{n\in\N^{*}}B_{n}
    \end{equation}
    où les $B_{n}$ sont disjoints donc $\P(E)=\sum_{n\in\N^{*}}\P(B_{n})$.

    On a 
    \begin{equation}
        \P(E_{n})=\binom{2n}{n}p^{n}q^{n} 
    \end{equation}
    On écrit alors 
    \begin{equation}
        E_{n}=\bigcup_{1\leqslant k\leqslant n}\left(E_{n}\cap B_{k}\right)
    \end{equation}
    où la réunion est disjointe (on partitionne selon le premier passage en 0). D'où 
    \begin{equation}
        u_{n}=\P(E_{n})=\sum_{k=1}^{n}\P(B_{k})\P_{B_{k}}(E_{n})
    \end{equation}
    On pose $b_{k}=\P(B_{k})$ et on a $\P_{B_{k}}(E_{n})=\P(E_{n-k})=u_{n-k}$: c'est comme si on repartait de 0 à l'étape $k$. On a donc $u_{0}=\P(E_{0})=1$ et pour tout $n\geqslant1$,
    \begin{equation}
        u_{n}=\sum_{k=1}^{n}b_{k}u_{n-k}=\sum_{k=0}^{n}b_{k}u_{n-k}
    \end{equation}
    en posant $b_{0}=0$.

    Or, on a 
    \begin{equation}
        u_{n}=\frac{(2n)!}{(n!)^{2}}(pq)^{n}\underset{n\to+\infty}{\sim}\frac{\sqrt{4\pi n}\left(\frac{2n}{e}\right)^{2n}}{2\pi n\left(\frac{n}{e}\right)^{2n}}(pq)^{n}
    \end{equation}
    d'où 
    \begin{equation}
        u_{n}\underset{n\to+\infty}{\sim}\frac{(4pq)^{n}}{\sqrt{\pi n}}
    \end{equation}
    et on a $4pq<1$ si et seulement si $p\neq\frac{1}{2}$ donc $\sum_{n\in\N}u_{n}$ converge si et seulement si $p\neq\frac{1}{2}$.

    Dans le cas $p\neq\frac{1}{2}$, on pose $S=\sum_{n=0}^{+\infty}u_{n}$. On a 
    \begin{align}
        \sum_{n=1}^{+\infty}u_{n}
        &=S-u_{0}\\
        &=S-1\\
        &=\sum_{n=1}^{+\infty}\sum_{k=0}^{n}b_{k}u_{n-k}\\
        &=\sum_{n=0}^{+\infty}\sum_{k=0}^{n}b_{k}u_{n-k}\\
        &=\left(\sum_{n=0}^{+\infty}b_{n}\right)\left(\sum_{l=0}^{+\infty}u_{l}\right)
        &=S\sum_{n=0}^{+\infty}b_{n}
    \end{align}
    donc 
    \begin{equation}
        \boxed{\sum_{n=0}^{+\infty}b_{n}=\P(E)=\frac{S-1}{S}<1}
    \end{equation}

    Comme dans ce cas, on a $\sum_{n\geqslant1}\P(E_{n})<\infty$, le lemme de Borel-Cantelli indique que le nombre de retours à l'origine est presque sûrement fini.
\end{proof}

\begin{remark}

    Avec les séries entières, on peut vérifier que 
    \begin{equation}
        S=\frac{1}{\sqrt{1-4pq}}
    \end{equation}
    d'où
    \begin{equation}
        \P(E)=1-\sqrt{1-4pq}
    \end{equation}

    Dans le cas $p=\frac{1}{2}$, $\sum_{n\in\N^{*}}u_{n}$ diverge. Comme on a pour $p\neq\frac{1}{2}$, on a 
    \begin{equation}
        \sum_{n=0}^{+\infty}b_{n}(p)=1-\sqrt{4p(1-p)}
    \end{equation}
    et $b_{n}(p)\leqslant b_{n}\left(\frac{1}{2}\right)$, on peut passer à la limite donc 
    \begin{equation}
        \sum_{n=0}^{+\infty}b_{n}\left(\frac{1}{2}\right)=1
    \end{equation}
    et la retour en 0 est presque sûr si $p=\frac{1}{2}$.
\end{remark}

\begin{remark}
    Pour montrer que 
    \begin{equation}
        l(x)=\sum_{n=0}^{+\infty}\binom{2n}{n}x^{n}=\frac{1}{1-4x}
    \end{equation}
    lorsque $0\leqslant x<\frac{1}{4}$. On effectue un produit de Cauchy
    \begin{equation}
        l(x)^{2}=\sum_{n=0}^{+\infty}\left(\sum_{k=0}^{n}\binom{2k}{k}\binom{2n-2k}{n-k}\right)x^{n}=\sum_{n=0}^{+\infty}4^{n}x^{n}=\frac{1}{1-4r}
    \end{equation}
    en dénombrant les parties d'un ensemble à $2n$ éléments séparées en $n$ éléments dans $A$ et $n$ éléments dans $B$.
\end{remark}

\begin{proof}
    On note $P_{n}$:`on obtient pile au $n$-ième lancer' et $F_{n}$:`on obtient face au $n$-ième lancer'.
    \begin{enumerate}
        \item On a 
        \begin{equation}
            \boxed{a_{1}=0,a_{2}=p^{2},a_{3}=qp^{2}}
        \end{equation}
        \item Pour $n\geqslant4$, on a 
        \begin{equation}
            A_{n}=\bigcap_{k=1}^{n-3}\overline{A_{k}}\bigcap F_{n-2}\bigcap P_{n-1}\bigcap P_{n}
        \end{equation}
        Comme les évènements concernant des lancers différents sont supposés indépendants, on a 
        \begin{equation}
            a_{n}=\P\left(\bigcap_{k=1}^{n-3}\overline{A_{k}}\right)qp^{2}
        \end{equation}
        On écrit 
        \begin{equation}
            \P\left(\bigcap_{k=1}^{n-3}\overline{A_{k}}\right)=1-\P\left(\bigcup_{k=1}^{n-3}A_{k}\right)=1-\sum_{k=1}^{n-3}a_{k}
        \end{equation}
        car les $A_{k}$ sont incompatibles. Ainsi, 
        \begin{equation}
            a_{n}=p^{2}q\left(1-\sum_{k=1}^{n-3}a_{k}\right)
        \end{equation}
        et $\sum_{k\geqslant1}a_{k}$ converge puisque 
        \begin{equation}
            \sum_{k=1}^{N}a_{k}\leqslant\P\left(\bigcup_{k\geqslant1}\right)\leqslant1
        \end{equation}

        Pour calculer $a_{n}$, on remarque que 
        \begin{equation}
            B_{n}=\bigcap_{k=1}^{n}\overline{A_{k}}
        \end{equation}
        est exactement l'évènement `on n'a pas deux piles consécutifs dans les lancers $\lbrace1,\dots,n\rbrace$'.

        Si $P_{n}$, on a nécessairement $F_{n-1}$ et $B_{n-2}$, si $F_{n}$ on a nécessairement $B_{n-1}$. Ainsi,
        \begin{equation}
            \P(B_{n})=qp\P(B_{n-2})+q\P(B_{n-1})
        \end{equation}

        On a l'équation caractéristique $X^{2}-qX-pq$, le discriminant est $\Delta=q^{2}+4pq>0$. On en déduit les racines $\lambda_{1}=\frac{q+\sqrt{\Delta}}{2}$ et $\lambda_{2}=\frac{q-\sqrt{\Delta}}{2}$, et on utilise les conditions aux limites $\P(B_{0})=\P(B_{1})=1$ et $\P(B_{n})=A\lambda_{1}^{n}+B\lambda_{2}^{n}$.
    \end{enumerate}
\end{proof}

\begin{remark}
    La probabilité d'obtenir une séquence fixée de longueur $N$ est égale à 1. En effet, on pose pour tout $n\in\N$, $A_{n}$:`la séquence apparaît entre les lancers $nN+1$ et $(n+1)N$'. Les $A_{n}$ sont clairement indépendants et on a $\P(A_{n})=\P(A_{1})=\alpha>0$ et $\sum_{n\geqslant1}\P(A_{n})$ diverge. Notamment,
    \begin{equation}
        \P\left(\overline{\bigcup_{n\in\N}A_{n}}\right)=\P\left(\bigcap_{n\in\N}\overline{A_{n}}\right)=\lim\limits_{k\to+\infty}\P\left(\bigcap_{n=0}^{k}\overline{A_{n}}\right)=\lim\limits_{k\to+\infty}\prod_{n=0}^{k}(1-\alpha)=0
    \end{equation}
    On a donc presque sûrement la séquence. D'après le lemme de Borel-Cantelli, on a presque sûrement une infinité de fois la séquence.
\end{remark}

\begin{proof}
    On note $N_{n}$:`on tire une boule noire au $n$-ième tirage', et $B_{n}$:`on tire une boule blanche au $n$-ième tirage'.
    
    On a 
        \begin{equation}
            \P_{N}(n)=\P_{B_{1}\cap\dots\cap B_{n}}(B_{n+1})=\frac{\P(B_{1}\cap\dots\cap B_{n+1})}{\P(B_{1}\cap\dots\cap B_{n})}
        \end{equation}
        Or 
        \begin{equation}
            \P(B_{1}\cap\dots\cap B_{n})=\sum_{k=0}^{N}\frac{1}{N+1}\left(\frac{k}{n}\right)^{n}\xrightarrow[N\to+\infty]{}\int_{0}^{1}x^{n}dx=\frac{1}{n+1}
        \end{equation}
        car pour $k$ fixé, $\frac{1}{N+1}$ est la probabilité d'avoir l'urne $k$ et $\left(\frac{k}{n}\right)^{n}$ est la probabilité d'avoir une blanche sachant qu'on a pris l'urne $k$, et la limite vient d'une somme de Riemann.
        
        Donc 
        \begin{equation}
            \boxed{\P_{N}(n)\xrightarrow[N\to+\infty]{}\frac{\frac{1}{n+2}}{\frac{1}{n+1}}=\frac{n+1}{n+2}}
        \end{equation}
\end{proof}

\begin{remark}
    Pour $n=0$, on a 
    \begin{equation}
        \P_{N}(0)=\frac{\frac{1}{N}\sum_{k=0}^{N}k}{N+1}=\frac{1}{2}
    \end{equation}
\end{remark}

\begin{proof}
    Si cette probabilité est définie, on note $p$ la probabilité recherchée. Soit $d\in\N^{*}$, pour tout $(n_{1},n_{2})\in\left(\N^{*}\right)^{2}$, on a $n_{1}\wedge n_{2}=d$ si et seulement si $n_{1}=dn_{1}'$ et $n_{2}=dn_{2}'$ avec $n_{1}'\wedge n_{2}'=1$. Ainsi, la probabilité pour que $n_{1}\wedge n_{2}=d$ est $\frac{p}{d^{2}}$ et 
    \begin{equation}
        \sum_{d\geqslant1}\frac{p}{d^{2}}=1
    \end{equation}
    d'où 
    \begin{equation}
        \boxed{p=\frac{6}{\pi^{2}}}
    \end{equation}
\end{proof}

\begin{remark}
    Pour justifier un peu plus précisément, on note que dans l'ensemble $\llbracket 1,dN\rrbracket$, la proportion de multiplies de $d$ est de $\frac{1}{d}$, donc sur $\llbracket 1,dN\rrbracket^{2}$, la proportion de couples de multiples de $d$ est  $\frac{1}{d^{2}}$.
\end{remark}

\begin{proof}

    On note $q_{n}=\P(X_{n}=1)$ (qui détermine la loi de $X_{n}$ car c'est une variable de Bernouilli). On a $q_{1}=p_{2}$ et pour $n\geqslant2$,
    \begin{equation}
        q_{n}=p_{1}q_{n-1}+p_{2}(1-q_{n-1})=(p_{1}-p_{2})q_{n-1}+p_{2}
    \end{equation}
    La relation est vraie pour $n=1$ en posant $q_{0}=0$.

    \begin{itemize}
        \item Si $p_{1}=1$ et $p_{2}=0$, on a $q_{n}=q_{n-1}+p_{2}$ d'où 
        \begin{equation}
            \boxed{q_{n}=0}
        \end{equation}

        \item Si $(p_{1},p_{2})\neq(1,0)$, on a $p_{1}-p_{2}\neq1$ donc 
        \begin{align}
            q_{n}
            &=(p_{1}-p_{2})^{n}\times \frac{-p_{2}}{1-(p_{1}-p_{2})}+\frac{p_{2}}{1-(p_{1}-p_{2})}\\
            &=\boxed{\frac{p_{2}}{1-(p_{1}-p_{2})}\left(1-\left(p_{1}-p_{2}\right)^{n}\right)=\E(X_{n})}
        \end{align}

        \item Si $p_{1}-p_{2}=-1$, i.e.~$p_{1}=0$ et $p_{2}=1$, 
        \begin{equation}
            \boxed{q_{n}\text{ n'a pas de limite.}}
        \end{equation}

        \item Si $p_{1}-p_{2}\neq-1$,
        \begin{equation}
            \boxed{q_{n}\xrightarrow[n\to+\infty]{}\frac{p_{2}}{1-(p_{2}-p_{1})}}
        \end{equation}
    \end{itemize}
\end{proof}

\begin{proof}
    \phantom{}
    \begin{enumerate}
        \item Pour tout $k\in\llbracket1,6\rrbracket$, on veut
        \begin{equation}
            \P(X\leqslant k)=\P\left(\left(D_{1}\leqslant k\right)\cap\left(D_{2}\leqslant k\right)\right)=\P(D_{1}\leqslant k)\P(D_{2}\leqslant k)=\frac{k^{2}}{36}
        \end{equation}

        Or on a (avec $P(X\leqslant 0)=0$)
        \begin{equation}
            P(X=k)=P(X\leqslant k)-\P(X\leqslant k-1)=\frac{2k-1}{36}
        \end{equation}
        De même, on a $P(Y\geqslant k)=\frac{(7-k)^{2}}{36}$ donc
        \begin{equation}
            \P(Y=k)=\frac{13-2k}{36}
        \end{equation}
        A chaque fois, on vérifie que $\sum_{k=1}^{6}\P(X=k)=\sum_{k=1}^{6}\P(Y=k)=1$.

        Pour les calculs de variance et d'espérance, on calcule $\sum_{k=1}^{6}k\P(X=k)$ et 
        \begin{equation*}
            \sum_{k=1}^{6}k^{2}\P(X=k)-\left(\sum_{k=1}^{6}k\P(X=k)\right),
        \end{equation*}de même pour $Y$.

        \item Soit $(i,j)\in\llbracket1,6\rrbracket^{2}$, si $i<j$ on a $\P((X=i)\cap (Y=j))=0$ mais $P(X=i)\P(Y=j)\neq0$, on n'a donc pas indépendance.
        
        \item Si $P(D_{i}=k)=p_{k,i}$, on a 
        \begin{equation}
            \P(X\leqslant k)=\left(\sum_{l=1}^{k}p_{l,1}\right)\left(\sum_{l=1}^{k}p_{l,2}\right)=\sum_{1\leqslant l,r\leqslant k}p_{l,1}\times p_{r,2}
        \end{equation}
        et on calcule ensuite $P(X=k)=\P(X\leqslant k)-\P(X\leqslant k-1)$ et cela vaut ce que cela vaut.
    \end{enumerate}
\end{proof}

\begin{proof}
    \phantom{}
    \begin{enumerate}
        \item On a 
        \begin{align}
            \sum_{(i,j)\in\N^{2}}
            &=\sum_{i=0}^{+\infty}\sum_{j=0}^{i}\frac{a^{j}(1-a)^{i-j}}{j!(i-j)!}(b^{i}e^{-b})\\
            &=\sum_{i=0}^{+\infty}\frac{b^{i}e^{-b}}{i!}\sum_{j=0}^{i}\binom{i}{j}a^{j}(1-a)^{i-j}\\
            &=\sum_{i=0}^{+\infty}\frac{b^{i}e^{-b}}{i!}\\
            &=e^{b}e^{-b}\\
            &=1
        \end{align}
        donc la définition est cohérente.

        \item On a 
        \begin{equation}
            \boxed{p_{i,\cdot}=\sum_{j=0}^{i}p_{i,j}=\frac{b^{i}e^{-b}}{i!}}
        \end{equation}
        et 
        \begin{align}
            p_{\cdot,j}
            &=\sum_{i=j}^{+\infty}\frac{e^{-b}a^{j}}{j!}\left(\frac{b^{i}(1-a)^{i-j}}{(i-j)!}\right)\\
            &=\frac{e^{-b}a^{j}b^{j}}{j!}\sum_{j=0}^{+\infty}\frac{b^{i}(1-a)^{i}}{i!}\\
            &=\boxed{\frac{e^{-ab}(ab)^{j}}{j!}}
        \end{align}
        On a $p_{i,j}\neq p_{i,\cdot}\neq p_{\cdot,j}$ donc les variables ne sont pas indépendantes.

        \item $Z$ est à valeurs dans $\N$ (car $p_{i,j}=0$ si $i<j$). On a 
        \begin{align}
            \P(X-Y=k)
            &=\sum_{j=0}^{+\infty}\P(X=j+k,Y=j)\\
            &=\sum_{j=0}^{+\infty}\frac{b^{j+k}e^{-b}a^{j}(1-a)^{k}}{j!k!}\\
            &=\frac{e^{-b}b^{k}(1-a)^{k}}{k!}\sum_{j=0}^{+\infty}\frac{(ba)^{j}}{j!}\\
            &=\boxed{\frac{e^{b(a-1)}(b(1-a))^{k}}{k!}}
        \end{align}

        De plus,
        \begin{equation}
            \P(Z=k,Y=j)=\P((X,Y)=(k+j,j))=p_{k+j,j}=\frac{b^{j+k}e^{-b}a^{j}(1-a)^{k}}{j!k!}
        \end{equation}
        et
        \begin{align}
            \P(Z=k)\P(Y=j))
            &=\frac{e^{b(a-1)b^{k}(1-a)^{k}}}{k!}\frac{e^{-ab}(ab^{j})}{j!}\\
            &=\frac{e^{-b}b^{k+j}a^{j}(1-a)^{k}}{k!j!}
        \end{align}
        donc $Z$ et $Y$ sont indépendantes.
    \end{enumerate}
\end{proof}

\begin{remark}
    On a $X\sim \mathcal{P}(b)$ et $Y\sim\mathcal{P}(ab)$ donc $X$ et $Y$ ont des espérances.
\end{remark}

\begin{proof}
    \phantom{}
    \begin{enumerate}
        \item On a $S_{n}-S_{n-1}=T_{n}$ pour tout $n\geqslant2$, donc 
        \begin{equation}
            \boxed{S_{n}=\sum_{i=1}^{n}T_{i}}
        \end{equation}

        \item Pour tout $k\in\N^{*}$, comme $T_{n}\sim\mathcal{G}(1-x)$, on a 
        \begin{equation}
            \boxed{\P(T_{n}=k)=x^{k-1}(1-x)}
        \end{equation}
        et 
        \begin{equation}
            \boxed{\E(T_{n})=\sum_{k=1}^{+\infty}kx^{k-1}(1-x)=\frac{1}{1-x}}
        \end{equation}
        et 
        \begin{equation}
            \boxed{\mathbb{V}(T_{n})=\frac{x}{(1-x)^{2}}}
        \end{equation}

        \item On a
        \begin{equation}
            \boxed{\E(S_{n})=\sum_{i=1}^{n}\E(T_{i})=\frac{n}{1-x}}
        \end{equation}
        Comme les $(T_{i})_{1\leqslant i\leqslant n}$ sont indépendants, on a 
        \begin{equation}
            \boxed{\mathbb{V}(S_{n})=\sum_{i=1}^{n}\mathbb{V}(T_{i})=\frac{nx}{(1-x)^{2}}}
        \end{equation}

        Pour $k<n$, on a $\P(S_{n}=k)=0$ et sinon, on a 
        \begin{equation}
            \P(S_{n}=k)=\binom{k-1}{n-1}(1-x)^{n}x^{k-n}
        \end{equation}
        (choisir les $n-1$ succès parmi $k-1$ épreuves).

        \item On a $\sum_{k=n}^{+\infty}\P(S_{n}=k)=1$ donc 
        \begin{equation}
            \boxed{\sum_{k=n}^{+\infty}\binom{k-1}{n-1}x^{k}=\frac{x^{n}}{(1-x)^{n}}}
        \end{equation}
    \end{enumerate}
\end{proof}

\begin{proof}
    On a $\P(X>n)=\sum_{k=n+1}^{+\infty}$. On pose 
    \begin{equation}
        u_{k,n}=
        \left\lbrace
            \begin{array}[]{l}
                \P(X=k)\text{ si k>n}\\
                0 \text{ sinon}
            \end{array}
        \right.
    \end{equation}

    Alors $\sum_{n\in\N}\P(X>n)$ converge si et seulement si $(u_{k,n})_{(k,n)\in\N^{2}}$ est sommable ($u_{k,n}\geqslant0$) si et seulement si $\sum_{k=1}^{+\infty}\sum_{n=0}^{+\infty}u_{k,n}$ converge (théorème de Fubini). Or 
    \begin{equation}
        \sum_{k=1}^{+\infty}\sum_{n=0}^{+\infty}u_{k,n}=\sum_{k=1}^{+\infty}\sum_{n=0}^{k-1}\P(X=k)=\sum_{k=1}^{+\infty}k\P(X=k)
    \end{equation}
\end{proof}

\begin{proof}
    On cherche $\sup\limits_{k\in\N}\left(\frac{\lambda^{k}}{k!}e^{-\lambda}\right)=\sup\limits_{k\in\N}u_{k}$ avec $u_{k}>$. On a 
    \begin{equation}
        \frac{u_{k+1}}{u_{k}}=\frac{\lambda}{k+1}\geqslant 1 
    \end{equation}
    si et seulement si $k\leqslant\lambda-1$. On a donc $u_{k}\leqslant u_{k+1}$ si et seulement si $k\leqslant \left\lfloor\lambda\right\rfloor-1$ et le maximum est donc atteint pour $k=\left\lfloor\lambda\right\rfloor$.

    Si $\lambda=n\in\N^{*}$, le maximum vaut 
    \begin{equation}
        \frac{n^{n}e^{-n}}{n!}\underset{n\to+\infty}{\sim}\frac{1}{\sqrt{2\pi n}}
    \end{equation}
\end{proof}

\begin{proof}
    \phantom{}
    \begin{enumerate}
        \item On a 
        \begin{equation}
            \P(X=k)=\frac{9}{10}\times\frac{8}{9}\times\dots\times\frac{10-(k-1)}{10-(k-2)}\times\frac{1}{10-(k-1)}=\frac{1}{10}
        \end{equation}
        donc $X\sim\mathcal{U}\left(\llbracket1,10\rrbracket\right)$ et $Y\sim\mathcal{G}\left(\frac{1}{10}\right)$ donc $\E(X)=\frac{11}{2}$, $\mathbb{V}(X)=\frac{10^{2}-1}{12}=\frac{33}{4}$, $\E(Y)=10$, $\mathbb{V}(Y)=\frac{9}{10}\times100=90$.

        \item Soit $S$ l'événement `le gardien est sobre' et $Z$ compte le nombre d'essais au bout desquels il a réussi. Alors 
        \begin{equation}
            \P_{Z\geqslant 9}(5)=\frac{\P(S)\P_{S}(Z\geqslant9)}{\P(Z\geqslant9)}=\frac{\P(S)\P(X\geqslant9)}{\frac{1}{3}\P(Y\geqslant9)+\frac{2}{3}\P(X\geqslant9)}
        \end{equation}
        On a $\P(X\geqslant9)=\frac{1}{5}$ et 
        \begin{equation}
            \P(Y\geqslant9)=\sum_{n=9}^{+\infty}\left(\frac{9}{10}\right)^{n-1}\times\frac{1}{10}=\left(\frac{9}{10}\right)^{8}
        \end{equation}
        d'où 
        \begin{equation}
            \boxed{\P_{Z\geqslant9}(5)=\frac{2}{5\times\left(\frac{9}{10}\right)^{8}+2}<\frac{1}{2}}
        \end{equation}
    \end{enumerate}
\end{proof}

\begin{proof}
    \phantom{}
    \begin{enumerate}
        \item On a 
        \begin{equation}
            \boxed{N=\frac{n(n+1)}{2}}
        \end{equation}

        \item Soit $(i,j)\in\llbracket1,n\rrbracket^{2}$, si $i<j$ on a $p_{i,j}=0$ (où $p_{i,j}$ est la loi conjointe). Si $j\leqslant i$, on a
        \begin{equation}
            p_{i,j}=\frac{1}{N}=\frac{2}{n(n+1)}
        \end{equation}

        On a ensuite 
        \begin{equation}
            \boxed{p_{i,\cdot}=\sum_{j=1}^{i}\frac{2}{n(n+1)}}=\frac{2i}{n(n+1)}
        \end{equation}
        et 
        \begin{equation}
            \boxed{p_{j,\cdot}}=\sum_{i=j}^{n}\frac{2}{n(n+1)}=\frac{2(n-j+1)}{n(n+1)}
        \end{equation}

        \item On calcule 
        \begin{equation}
            \boxed{
                \begin{array}[]{l}
                    \E(B)=\sum_{i=1}^{n}p_{i,\cdot}=\frac{2}{n(n+1)}\sum_{i=1}^{n}i^{2}=\frac{2n+1}{3}\\
                    \E(R)=\sum_{j=1}^{n}jp_{\cdot,j}=\frac{2}{n(n+1)}\sum_{j=1}^{n}j(n+1-j)=\frac{n(n+1)^{2}}{2}-\frac{n(n+1)(2n+1)}{6}
                \end{array}
            }
        \end{equation}

        On laisse le reste en calcul facile en utilisant $\mathbb{V}(G)=\mathbb{V}(R)+\mathbb{V}(B)-2\mathrm{cov}(B,R)$ et 
        \begin{equation}
            \E(BR)=\sum_{(i,j)\in\llbracket1,n\rrbracket^{2}}ijp_{i,j}
        \end{equation}
    \end{enumerate}
\end{proof}

\begin{proof}
    \phantom{}
    \begin{enumerate}
        \item On écrit 
        \begin{equation}
            \boxed{\P(X_{n+1}=k)=\P_{X_{n}=k-1}(X_{n+1}=k)\P(X_{n}=k-1)=\frac{k}{k+1}\P(X_{n}=k)}
        \end{equation}

        \item On a $\P(X_{n}=k)=0$ si $k>n$, sinon on écrit 
        \begin{align}
            \P(X_{n}=k)
            &=\frac{k}{k+1}\P(X_{n-1}=k-1)\\
            &=\frac{k}{k+1}\times\frac{k-1}{k}\times\frac{k-2}{k-1}\times\dots\times\frac{2}{3}\times\frac{1}{2}\times u_{n-k}\\
            &=\boxed{\frac{1}{k+1}u_{n-k}}
        \end{align}

        \item On a $\sum_{j=0}^{n}\P(X_{n}=j)=1$ donc 
        \begin{equation}
            \boxed{\sum_{j=0}^{n}\frac{u_{n}}{n-j+1}=1}
        \end{equation}
        et on a $u_{0}=1,u_{1}=\frac{1}{2},u_{2}=\frac{5}{12},u_{3}=\frac{3}{8}$ (en utilisant la formule précédente).

        \item On écrit 
        \begin{equation}
            (k+1)\P(X_{n+1}=k)=k\P(X_{n}=k-1)
        \end{equation}
        donc pour tout $k\in\llbracket1,n+1\rrbracket$, 
        \begin{equation}
            k\P(X_{n+1}=k)=(k-1)\P(X_{n}=k-1)+\P(X_{n}=k-1)-\P(X_{n+1}=k)
        \end{equation}
        En sommant sur $k\in\llbracket1,n+1\rrbracket$, on trouve donc 
        \begin{equation}
            \boxed{\E(X_{n+1})=\E(X_{n})+1-(1-u_{n+1})=\E(X_{n})+u_{n+1}}
        \end{equation}
        Par récurrence, on a directement 
        \begin{equation}
            \boxed{\E(X_{n})=u_{n}+\dots+u_{1}+\underbrace{\E(X_{0})}_{{=~0}}}
        \end{equation}

        \item On écrit 
        \begin{align}
            \P(T=n)
            &=\P\left(\left(X_{0}=0\right)\cap\left(X_{1}=1\right)\cap\dots\cap\left(X_{n-1}=n-1\right)\cap\left(X_{n}=0\right)\right)\\
            &=\P(X_{0}=0)\times\P_{X_{0}=0}(X_{1}=1)\times\dots\times\P_{\left(X_{0}=0\right)\cap\dots\cap\left(X_{n-1}=n-1\right)}(X_{n=0})\\
            &=\P(X_{0}=0)\times\P_{X_{0}=0}(X_{1}=1)\times\dots\times\P_{\left(X_{n-1}=n-1\right)}(X_{n=0})\\
            &=1\times\frac{1}{2}\times\frac{2}{3}\times\dots\times\frac{n-1}{n}\times\frac{1}{n+1}\\
            &=\boxed{\frac{1}{n(n+1)}}
        \end{align}

        Comme $\frac{1}{n(n+1)}=\frac{1}{n}-\frac{1}{n+1}$, on a 
        \begin{equation}
            \sum_{n=1}^{+\infty}\P(T=n)=1
        \end{equation}
        donc 
        \begin{equation}
            \boxed{\P(T=0)=0}
        \end{equation}
        Donc le retour en temps fini à l'origine est presque sûr.

        \item Non au vu de la formule donnée par $\P(T=n)$.
    \end{enumerate}
\end{proof}

\begin{proof}
    \phantom{}
    \begin{enumerate}
        \item Soit $k\in\llbracket0,n\rrbracket$, on a 
        \begin{equation}
            \boxed{\P_{N=n}(X_{1}=k)=\binom{n}{k}\left(\frac{1}{m}\right)^{k}\left(1-\frac{1}{m}\right)^{n-k}}
        \end{equation}
        (loi binomiale, car les $m$ caisses sont équiprobables).

        \item On a $\P_{N=n}(X_{1}=k)=0$ si $k>n$ donc si $k\in\N$,
        \begin{align}
            \P_{X_{1}=k}
            &=\sum_{n=0}^{+\infty}\P_{N=n}(X_{1}=k)\P(N=n)\\
            &=\sum_{n=k}\binom{n}{k}\left(\frac{1}{m}\right)^{k}\left(1-\frac{1}{m}\right)^{n-k}\frac{\lambda^{n}}{n!}e^{-\lambda}\\
            &=\frac{1}{k!}e^{-\lambda}\left(\frac{1}{m}\right)^{k}\lambda^{k}\sum_{n=k}^{+\infty}\lambda^{n-k}\left(1-\frac{1}{m}\right)^{n-k}\frac{1}{(n-k)!}
        \end{align}
        On reconnaît la série exponentielle, après un changement d'indice, appliquée en $\lambda\left(1-\frac{1}{m}\right)$. Ainsi,
        \begin{equation}
            \boxed{\P(X_{1}=k)=e^{-\frac{\lambda}{n}}\frac{\left(\frac{\lambda}{n}\right)^{k}}{k!}}
        \end{equation}
        et donc $X_{1}\sim\mathcal{P}\left(\frac{\lambda}{n}\right)$.
    \end{enumerate}
\end{proof}

\begin{proof}
    \phantom{}
    \begin{enumerate}
        \item Si $(i,j)\in\lbrace0,1,2\rbrace\times\lbrace-1,0,1\rbrace$, on a 
        \begin{equation}
            \boxed{\P\left((U,V)=(i,j)\right)=\P\left(X=\frac{i+j}{2},Y=\frac{-j}{2}\right)}
        \end{equation}
        Cette probabilité vaut 0 si $i$ et $j$ n'ont pas la même parité. Sinon, on a 
        \begin{equation}
            \boxed{
                \begin{array}[]{l}
                    \P\left((U,V)=(0,0)\right)=\P(X=0,Y=0)=q^{2}\\
                    \P\left((U,V)=(1,-11)\right)=\P(X=0,Y=1)=qp\\
                    \P\left((U,V)=(1,1)\right)=qp\\
                    \P\left((U,V)=(2,0)\right)=\P(X=1,Y=1)=p^{2}\\
                \end{array}
            }
        \end{equation}

        \item On a 
        \begin{align}
            \mathrm{cov}(U,V)
            &=\E\left(\left(U-\E(U)\right)\left(V-\E(V)\right)\right)\\
            &=\E(UV)-\E(U)\E(V)\\
            &=\E(X^{2}-Y^{2})-\left[\left(\E(X)+\E(Y)\right)\left(\E(X)-\E(Y)\right)\right]\\
            &=\E(X^{2}-Y^{2})-\left[\E(X)^{2}-\E(Y)^{2}\right]\\
            &=\mathbb{V}(X)-\mathbb{V}(Y)\\
            &=0
        \end{align}

        \item Les variables $U$ et $V$ ne sont pas indépendantes, il suffit de voir que 
        \begin{equation}
            \boxed{\P\left((U,V)=(1,0)\right)=0\neq\P(U=1)\P(V=0)=2pq\times\left(q^{2}+p^{2}\right)}
        \end{equation}
    \end{enumerate}
\end{proof}

\begin{proof}
    \phantom{}
    \begin{enumerate}
        \item $P\sim\mathcal{G}(p)$ et $F\sim\mathcal{G}(q)$ donc 
        \begin{equation}
            \boxed{
                \begin{array}[]{l}
                    \E(P)=\frac{1}{p}\\
                    \mathbb{V}(P)=\frac{q}{p^{2}}\\
                    \E(F)=\frac{1}{q}\\
                    \mathbb{V}(F)=\frac{p}{q^{2}}
                \end{array}
            }
        \end{equation}

        \item On a $\P\left((P=1)\cap(F=1)\right)=0\neq\P(P=1)\P(F=1)$ donc $P$ et $F$ ne sont pas indépendantes.
        
        \item Soit $(i,j)\in\left(\N^{*}\right)^{2}$, on note $p_{i,j}=\P(X=i,Y=j)$. On partitionne selon si $P=1$ ou $F=1$ et donc 
        \begin{equation}
            \boxed{p_{i,j}=p^{i+1}q^{j}+q^{i+1}p^{j}}
        \end{equation}
        On note 
        \begin{equation}
            \begin{array}[]{l}
                p_{i,\cdot}=\sum_{j=1}^{+\infty}p_{i,j}=p^{i+1}\frac{q}{1-q}+q^{i+1}\frac{p}{1-p}=p^{i}q+q^{i}p\\
                p_{\cdot,j}=\sum_{i=1}^{+\infty}p_{i,j}=\frac{p^{2}}{1-p}q^{j}+\frac{q^{2}}{1-q}p^{j}=q^{j-1}p^{2}+p^{j-1}q^{2}
            \end{array}
        \end{equation}

        De plus, on a $p_{1,1}=p^{2}q+q^{2}p=pq$ et $p_{1,\cdot}\times p_{\cdot,1}=(pq+qp)(p^{2}+q^{2})=2pq(p^{2}+q^{2})$ dp,c si $X$ et $Y$ sont indépendantes, on a $1=2(p^{2}+q^{2})$ d'où $p=\frac{1}{2}$. Réciproquement, si $p=\frac{1}{2}$, on a $p_{i,\cdot}=\frac{1}{2^{i}}$, $p_{\cdot, j}=\frac{1}{2^{j}}$ et $p_{i,j}=\frac{1}{2^{i+j}}=p_{i,\cdot}p_{\cdot,j}$. Ainsi, $X$ et $Y$ sont indépendantes si et seulement si $p=\frac{1}{2}$.

        \item On a 
        \begin{equation}
            \E(X)=\sum_{i=1}^{+\infty}ip_{i,\cdot}=q\sum_{i=1}^{+\infty}ip^{i}+p\sum_{i=1}^{+\infty}iq^{i}
        \end{equation}
        On utilise alors le fait que $\sum_{n=1}^{+\infty}nz^{n-1}=\frac{1}{(1-z)^{2}}$ si $\left\lvert z\right\rvert<1$ et donc 
        \begin{equation}
            \boxed{\E(X)=\frac{p^{2}+q^{2}}{pq}\geqslant2}
        \end{equation}
        car $(p-q)^{2}\geqslant0$.

        \item On a 
        \begin{equation}
            \boxed{\P(X=Y)=\sum_{i=1}^{+\infty}p_{i,i}=\sum_{i=1}^{+\infty}p^{i}q^{i}(p+q)=pq\times\frac{1}{1-pq}}
        \end{equation}

        \item Si $p=\frac{1}{2}$, $X$ et $Y$ sont indépendantes, donc par convolution,
        \begin{equation}
            \boxed{\P(X+Y=k)=\sum_{i=1}^{k-1}\P(X=i)\P(Y=k-i)=\sum_{i=1}^{k-1}p_{i,\cdot}p_{\cdot,k-i}=\sum_{i=1}^{k-1}\frac{1}{2^{k}}=\frac{k-1}{2^{k}}}
        \end{equation}
    \end{enumerate}
\end{proof}

\begin{proof}
    \phantom{}
    \begin{enumerate}
        \item On a 
        \begin{equation}
            \boxed{\e^{\lambda x}=\sum_{k=0}^{+\infty}\frac{(\lambda x)^{2k}}{(2k)!}+\sum_{k=0}^{+\infty}\frac{(\lambda x)^{2k+1}}{(2k+1)!}\leqslant\sum_{k=0}^{+\infty}\frac{\lambda^{2k}}{2^{k}k!}+x\sinh(\lambda)=\e^{\frac{\lambda^{2}}{2}}+x\sinh(\lambda)}
        \end{equation}
        car $\left\lvert x^{2}\right\rvert\leqslant1$ et pour tout $k\in\N^{*}$, $(2k)!\geqslant2^{k}k!$ (par récurrence).

        \item $\e^{\lambda X}$ admet une espérance car $\left\lvert\e^{\lambda X}\right\rvert\leqslant \e^{\lambda}$. Comme $X$ est centrée, on a d'après l'inégalité précédente, en prenant l'espérance,
        \begin{equation}
            \boxed{\E\left(\e^{\lambda X}\right)\leqslant\E\left(e^{\frac{\lambda^{2}}{2}}\right)+\sinh(\lambda)\E(X)=\e^{\frac{\lambda^{2}}{2}}}
        \end{equation}
        En appliquant l'inégalité à $-X$, on a l'autre inégalité.

        \item Grâce à l'inégalité de Markov, on en déduit que 
        \begin{equation}
            \boxed{\P(X\geqslant a)=\P\left(\e^{\lambda X}\geqslant \e^{\lambda a}\right)}\leqslant \frac{\E\left(\e^{\lambda X}\right)}{e^{\lambda a}}=\e^{-\lambda a}\E\left(\e^{\lambda X}\right)
        \end{equation}

        \item On pose $X=\frac{1}{n}\sum_{k=1}^{n}X_{i}$. $X$ est centrée dans $[-1,1]$ ainsi que $-X$. On a donc, pour tout $\lambda\geqslant0$,
        \begin{equation}
            \P\left(\left\lvert X\right\rvert\geqslant a\right)=\P(X\geqslant a)+\P(-X\geqslant a)\leqslant 2\e^{-\lambda a}\e^{\frac{\lambda^{2}}{2}}
        \end{equation}
        On optimise ensuite cette inégalité en $\lambda\geqslant0$ (le minimum est en $\lambda=a$) et on a bien 
        \begin{equation}
            \boxed{\P\left(\left\lvert\frac{1}{n}\sum_{i=1}^{n}X_{i}\right\rvert\geqslant a\right)\leqslant 2\e^{-\frac{a^{2}}{2}}}
        \end{equation}
    \end{enumerate}
\end{proof}

\begin{proof}
    \phantom{}
    \begin{enumerate}
        \item Comme $\E(Y)<+\infty$, on a d'après le théorème de Fubini et le fait que $\left(X=l\right)_{l\in\N}$ est un système complet d'événements:
        \begin{equation}
            \E(Y)=\sum_{l=0}^{+\infty}l\P(Y=l)\sum_{l=0}^{+\infty}\sum_{k=0}^{+\infty}\P_{(X=k)}(Y=l)\P(X=k)=\sum_{k=0}^{+\infty}\E_{(X=k)}(Y)\P(X=k).
        \end{equation}

        \item Pour $\lambda=X_{n+1}$ et $X=X_{n}$, et en utilisant le fait que les poules sont indépendantes, on a 
        \begin{equation}
            \E(X_{n+1})=\sum_{k=0}^{+\infty}\E_{(X_{n}=k)}\P(X_{n}=k)=\sum_{k=0}^{+\infty}k\lambda\P(X_{n}=k)=\lambda\E(X_{n})
        \end{equation}
        Par récurrence, on a 
        \begin{equation}
            \boxed{\E(X_{n})=\lambda^{n}\E(X_{0})=\lambda^{n}N}
        \end{equation}

        On note que si $\lambda>1$, $\E(X_{n})\xrightarrow[n\to+\infty]{}+\infty$ donc la descendance est assurée. Si $\lambda<1$, on a $\E(X_{n})\xrightarrow[n\to+\infty]{}0$.
    \end{enumerate}
\end{proof}

\begin{proof}
    Soit $k\in\N$, on a 
    \begin{align}
        \P(K=k)
        &=\sum_{n=0}^{+\infty}\P_{(N=n)}(K=k)\P(N=n)\\
        &=\sum_{n=k}^{+\infty}\binom{n}{k}p^{k}(1-p)^{n-k}\e^{-\lambda}\frac{\lambda^{n}}{n!}\\
        &=\e^{-\lambda p}\frac{\left(\lambda p\right)^{k}}{k!}
    \end{align}

    Donc $K\sim\mathcal{P}(\lambda p)$ et 
    \begin{equation}
        \boxed{\E(K)=\lambda p}
    \end{equation}
\end{proof}

\begin{proof}
    \phantom{}
    \begin{enumerate}
        \item On a $\chi_{A_{k}}\sim\mathcal{B}\left(\frac{1}{k}\right)$. Ainsi,
        \begin{equation}
            \boxed{\E(S_{n})=\sum_{k=1}^{n}\frac{1}{k}\underset{n\to+\infty}{\sim}\ln(n)}
        \end{equation}

        Comme les $(A_{n})_{n\geqslant1}$ sont indépendants, on a aussi 
        \begin{equation}
            \boxed{\mathbb{V}(S_{n})=\sum_{k=1}^{n}\mathbb{V}(\chi_{A_{k}})=\sum_{k=1}^{n}\frac{1}{k}\left(1-\frac{1}{k}\right)\underset{n\to+\infty}{\sim}\ln(n)}
        \end{equation}

        \item Soit $X_{n}=\frac{S_{n}}{\ln(n)}$. On a $\E(X_{n})\xrightarrow[n\to+\infty]{}1$ et $\mathbb{V}(X_{n})=\frac{1}{\ln^{2}(n)}\mathbb{V}(S_{n})\xrightarrow[n\to+\infty]{}0$.
        
        D'après l'inégalité de Bienaymé-Tchebychev,
        \begin{equation}
            \P\left(\left\lvert X_{n}-\E(X_{n})\right\rvert\geqslant\frac{\varepsilon}{2}\right)\leqslant4\frac{\mathbb{V}(X_{n})}{\varepsilon^{2}}\xrightarrow[n\to+\infty]{}0
        \end{equation}

        Or, si $\left\lvert X_{n}-\E(X_{n})\right\rvert<\frac{\varepsilon}{2}$ et $\left\lvert\E(X_{n})-1\right\rvert<\frac{\varepsilon}{2}$, alors $\left\lvert X_{n}-1\right\rvert<\varepsilon$.
        Par contraposée, si $\left\lvert X_{n}-1\right\rvert\geqslant\varepsilon$, alors ou bien $\left\lvert X_{n}-\E(X_{n})\right\rvert\geqslant\frac{\varepsilon}{2}$ ou $\left\lvert\E(X_{n})-1\right\rvert\geqslant\frac{\varepsilon}{2}$. Ainsi,
        \begin{equation}
            \P\left(\left\lvert X_{n}-1\right\rvert\geqslant\varepsilon\right)\leqslant4\frac{\mathbb{V}(X_{n})}{\varepsilon^{2}}+\P\left(\left\lvert\E(X_{n})-1\right\rvert\geqslant\frac{\varepsilon}{2}\right)
        \end{equation}
        A partir d'un certain rang $N_{0}\in\N$, on a $\left\lvert\E(X_{n})-1\right\rvert<\frac{\varepsilon}{2}$, donc pour tout $n\geqslant N_{0}$, on a $\P\left(\left\lvert\E(X_{n})-1\right\rvert\geqslant\frac{\varepsilon}{2}\right)=0$. Ainsi,
        \begin{equation}
            \boxed{\lim\limits_{n\to+\infty}\P\left(\left\lvert\frac{S_{n}}{\ln(n)}-1\right\rvert\geqslant\varepsilon\right)=0}
        \end{equation}
    \end{enumerate}
\end{proof}

\begin{proof}
    \phantom{}
    \begin{enumerate}
        \item Soit $k\in\N^{*}$. On a 
        \begin{equation}
            \P(U\geqslant k)=\P\left(\bigcap_{i=1}^{n}\left(X_{i}\geqslant k\right)\right)=\prod_{i=1}^{k}\P\left(X_{i}\geqslant k\right)=\prod_{i=1}^{n}\sum_{j=k}^{+\infty}q^{j-1}p=(q^{k-1})^{n}
        \end{equation}

        Ainsi, 
        \begin{equation}
            \boxed{\P(U=k)=\P(U\geqslant k)-\P(U\geqslant k+1)=q^{(k-1)n}(q^{n}-1)}
        \end{equation}

        $U$ possède une espérance car $0\leqslant U\leqslant X\sim\mathcal{G}(p)$ et on a 
        \begin{align}
            \E(U)
            &=\sum_{k=1}^{+\infty}k\P(U=k)\\
            &=\sum_{k=1}^{+\infty}k\left(\left(q^{n}\right)^{k-1}-\left(q^{n}\right)^{k}\right)\\
            &=\frac{1}{\left(1-q^{n}\right)^{2}}-\frac{q^{n}}{(1-q^{n})^{2}}\\
            &=\boxed{\frac{1}{1-q^{n}}}
        \end{align}
        où l'on a utilisé le fait que $\sum_{n=0}^{+\infty}nx^{n-1}=\frac{1}{(1-x)^{2}}$ si $\left\lvert x\right\rvert<1$.

        \item Soit $k\in\N^{*}$. On a $\P(V\leqslant k)=\left(1-q^{k}\right)^{n}$ donc 
        \begin{equation}
            \boxed{\P(V=k)=\left(1-q^{k}\right)^{n}-\left(1-q^{k-1}\right)^{n}\underset{n\to+\infty}{\sim}nq^{k-1}\left(1-q\right)\xrightarrow[k\to+\infty]{}0}
        \end{equation}

        Comme $k\P(V=k)=\underset{k\to+\infty}{O}\left(\frac{1}{k^{2}}\right)$, $V$ admet une espérance est 
        \begin{align}
            \E(V)
            &=\sum_{k=1}^{+\infty}k\P(V=k)\\
            &=\sum_{k=1}^{+\infty}k\left[\left(1-q^{k}\right)^{n}-\left(1-q^{k-1}\right)^{n}\right]\\
            &=\sum_{k=1}^{+\infty}k\sum_{i=1}^{n}\binom{n}{i}\left[(-1)^{i}q^{ki}-(-1)^{i}q^{(k-1)i}\right]\\
            &=\sum_{k=1}^{+\infty}k\sum_{i=1}^{n}(-1)^{i+1}\binom{n}{i}q^{(k-1)i}(1-q^{i})\\
            &=\sum_{k=1}^{+\infty}(-1)^{i+1}\binom{n}{i}(1-q^{i})\sum_{k=1}^{+\infty}k\left(q^{i}\right)^{k-1}\\
            &=\boxed{\sum_{i=1}^{n}\binom{n}{i}(-1)^{i+1}\frac{1}{1-q^{i}}}
        \end{align}
    \end{enumerate}
\end{proof}

\begin{proof}
    \phantom{}
    \begin{enumerate}
        \item $1-p^{N}$ correspond à la probabilité que le joueur perde au moins une partie sur $N$ consécutives donc c'est aussi la probabilité pour qu'il perde une partie entre la $nN+1$-ième et la $(n+1)N$-ième (inclus), car les parties sont indépendantes. On note $A_{nN+1,(n+1)N}$:`le joueur perd une partie entre la $nN+1$-ième et la $(n+1)N$-ième (au sens large)'. $\lbrace t_{k}>nN\rbrace$ et $A_{nN+1,(n+1)N}$ sont des événements indépendants car les différentes parties sont indépendantes. Ainsi,
        \begin{equation}
            \boxed{\P\left(t_{k}>nN\right)(1-p^{N})=\P\left(\left(t_{k}>nN\right)\cap A_{nN+1,(n+1)N}\right)\geqslant\P(t_{k}>n(N+1))}
        \end{equation}
        car s'il avait gagné toutes les parties entre $nN+1$ et $(n+1)N$ on aurait $t_{k}\leqslant n(N+1)$.

        On sait que si $X$ est une variable aléatoire discrète à valeurs dans $\N^{*}$, $X$ possède une espérance finie si et seulement si $\sum_{k\in\N}\P(X>k)$ converge et on a (théorème de Fubini) $\E(X)=\sum_{k\in\N}\P(X>k)$.

        On a donc 
        \begin{equation}
            \P(t_{k}>Nn)\leqslant (1-p^{n})\P(t_{k}>0)=1-p^{n}
        \end{equation}
        par récurrence sur $n$ d'après 1. Pour $l\in\N$, soit $n=\left\lfloor\frac{l}{N}\right\rfloor$, on a $nN\leqslant l<(n+1)N$ donc 
        \begin{equation}
            \P(t_{k}>l)\leqslant\P(t_{k}>nN)\leqslant(1-p^{N})^{\left\lfloor\frac{l}{N}\right\rfloor}\leqslant\left(1-p^{N}\right)^{\frac{l}{N}}
        \end{equation}
        et le membre de droite est le terme général d'une série converge car $\left(1-p^{N}\right)^{\frac{1}{N}}<1$. Donc $t_{k}$ admet une espérance.

        \item On note $b_{i}$:`le joueur gagne au $i$-ième coup'. Alors 
        \begin{align}
            T_{k}
            &=\sum_{l=0}^{+\infty}l\P(t_{k}=l)\\
            &=\sum_{l=0}^{+\infty}l\left(\P_{b_{i}}(t_{k}=l)\P(b_{i})+\P_{\overline{b_{i}}}(t_{k}=l)\P(\overline{b_{i}})\right)\\
            &=p\sum_{l=1}^{+\infty}l\P(t_{k+1}=l-1)+q\sum_{l=1}^{+\infty}l\P(t_{k-1}=l-1)\\
            &=p\left(\sum_{l=1}^{+\infty}(l-1)\P(t_{k+1}=l-1)+\sum_{l=1}^{+\infty}1\times\P(t_{k+1}=l-1)\right)\notag\\
            &~~~+q\left(\sum_{l=1}^{+\infty}(l-1)\P(t_{k-1}=l-1)+\sum_{l=1}^{+\infty}1\times\P(t_{k-1}=l-1)\right)\\
            &=p\left(T_{k-1}+1\right)+q\left(T_{k+1}+1\right)
        \end{align}

        \item On a $qT_{k+1}-T_{k}+pT_{k-1}=-1$. Comme $q\alpha(k+1)-\alpha k+p\alpha(k-1)=1$ si et seulement si $\alpha=\frac{1}{1-2p}$, on pose $U_{k}=T_{k}-\frac{k}{1-2p}$ si $p=\frac{1}{2}$. Alors 
        \begin{equation}
            qU_{k+1}-U_{k}+pU_{k-1}=-1-q\frac{k+1}{1-2p}+\frac{k}{1-2p}-p\frac{k-1}{1-2p}=0
        \end{equation}
        car $p+q=1$.

        L'équation caractéristique est $qr^{2}-r+p=0$, les racines sont $1$ et $\frac{p}{q}$ (qui est différent de 1 car $p\neq \frac{1}{2}$). Donc 
        \begin{equation}
            q\left(1-\frac{p}{q}\right)(1-1)=0
        \end{equation}
        On a $T_{0}=T_{N}=0$ donc 
        \begin{equation}
            \boxed{T_{k}=\frac{1}{q-p}\left(k-N\left(\frac{1-\left(\frac{q}{p}\right)^{k}}{1-\left(\frac{q}{p}\right)^{N}}\right)\right)}
        \end{equation}
        si $p\neq\frac{1}{2}$. Si $p=\frac{1}{2}$, on a 
        \begin{equation}
            T_{k+1}-2T_{k}+T_{k-1}=-2    
        \end{equation}
        Ainsi, si $V_{k}=T_{k-1}-T_{k}$, on a 
        \begin{equation}
            V_{k+1}=-2+V_{k}
        \end{equation}
        On en déduit grâce aux conditions aux limites $T_{0}=T_{N}=0$ que 
        \begin{equation}
            \boxed{T_{k}=k(N-k)}
        \end{equation}
    \end{enumerate}
\end{proof}

\begin{proof}
    \phantom{}
    \begin{enumerate}
        \item On a 
        \begin{equation}
            \boxed{\sum_{n\geqslant1}\P(\lbrace n\rbrace)=\sum_{n\geqslant1}\frac{1}{\zeta(s)n^{s}}=1}
        \end{equation}

        \item On a 
        \begin{equation}
            \boxed{\P(A_{n})=\sum_{k\in A_{n}}\P(\lbrace k\rbrace)=\sum_{r=1}^{+\infty}\frac{1}{\zeta(s)r^{s}n^{s}}=\frac{1}{n^{s}}}
        \end{equation}

        \item Soient $p_{1},\dots,p_{k}$ des nombres premiers distincts. On a $A_{p_{1}}\cap\dots\cap A_{p_{r}}=A_{p_{1}\times\dots\times p_{k}}$ donc 
        \begin{equation}
            \boxed{\P\left(A_{p_{1}}\bigcap\dots\bigcap A_{p_{k}}\right)}=\frac{1}{(p_{1}\dots p_{k})^{s}}=\prod_{i=1}^{k}\frac{1}{p_{i}^{s}}=\prod_{i=1}^{k}\P(A_{p_{i}})
        \end{equation}
        donc les $(A_{p})_{p\in\mathcal{P}}$ sont indépendants.

        On remarque que l'on a $\lbrace1\rbrace=\cap_{p\in\mathcal{P}}\overline{A_{p}}$. On pose $p_{k}$ le $k$-ième nombre premier et $B_{k}=\cap_{i=1}^{k}\overline{A_{p_{i}}}$ qui est une suite décroissante d'événements. Comme les $(A_{p_{i}})_{i}$ sont indépendants, c'est aussi le cas des $(\overline{A_{p_{i}}})_{i}$, et on a donc 
        \begin{equation}
            \P(\lbrace1\rbrace)=\lim\limits_{k\to+\infty}\P(B_{k})=\lim\limits_{k\to+\infty}\prod_{i=1}^{k}\left(1-\P(A_{p_{i}})\right)=\prod_{i=}^{+\infty}\left(1-\frac{1}{p_{i}^{s}}\right)
        \end{equation}
        Or $\P(\lbrace1\rbrace)=\frac{1}{\zeta(s)}$ donc 
        \begin{equation}
            \boxed{\zeta(s)=\left(\prod_{i=1}^{+\infty}\left(1-\frac{1}{p_{i}^{s}}\right)\right)^{-1}}
        \end{equation}
    \end{enumerate}
\end{proof}

\begin{proof}
    On a $\P(E_{1})=\frac{1}{2},\P(E_{2})=\frac{1}{2}(1-b)$ et pour tout $n\geqslant2$,
    \begin{equation}
        \boxed{\P(E_{n})=\frac{1}{2}b^{n-1}(1-b)}
    \end{equation}

    Les événements $E_{n}$ sont incompatibles donc 
    \begin{equation}
        \boxed{\P\left(\bigcup_{n=1}^{+\infty}E_{n}\right)=\sum_{n=1}^{+\infty}\P(E_{n})=1}
    \end{equation}
    Il est donc presque sûr qu'on finisse par utiliser $A$.

    On a 
    \begin{equation}
        \boxed{\P(U_{n})=\frac{1}{2}a^{n}+\frac{1}{2}b^{n}}
    \end{equation}
    $(U_{n})_{n\in\N}$ forme une suite décroissante d'événements, donc 
    \begin{equation}
        \boxed{\P\left(\bigcap_{n=1}^{+\infty}U_{n}\right)=\lim\limits_{n\to+\infty}\P(U_{n})=0}
    \end{equation}
\end{proof}

\begin{proof}
    \phantom{}
    \begin{enumerate}
        \item On a 
        \begin{equation}
            \boxed{\P(A_{n})=\sum_{k=1}^{n}p(1-p)^{k-1}=1-(1-p)^{n}}
        \end{equation}

        \item $B_{n}=\bigcap_{i=1}^{N}A_{i,n}$ et les $A_{i,n}$ sont indépendants donc 
        \begin{equation}
            \boxed{\P(B_{n})=\prod_{i=1}^{N}\P(A_{i,n})=\left(1-\left(1-p\right)^{n}\right)^{N}}
        \end{equation}

        \item On note $C_{n}=B_{n}\setminus B_{n-1}$ et $B_{n-1}\subset B_{n}$ donc 
        \begin{equation}
            \boxed{\P(C_{n})=\P(B_{n})-\P(B_{n-1})=\left(1-\left(1-p\right)^{n}\right)^{N}-\left(1-\left(1-p\right)^{n-1}\right)^{N}}
        \end{equation}
    \end{enumerate}
\end{proof}

\begin{proof}
    \phantom{}
    \begin{enumerate}
        \item On a 
        \begin{equation}
            \K_{<n}[X]=\left\lbrace a_{0}+\dots+a_{n-1}X^{n-1}\middle|(a_{0},\dots,a_{n-1})\in\K^{n}\right\rbrace    
        \end{equation}
        donc $\left\lvert\K_{<n}[X]\right\rvert=p^{n}$. De même, on a 
        \begin{equation}
            \K_{=n}[X]=\left\lbrace a_{0}+\dots+a_{n-1}X^{n-1}+a_{n}X^{n}\middle|(a_{0},\dots,a_{n})\in\K^{n},a_{n}\neq\overline{0}\right\rbrace
        \end{equation}
        donc $\left\lvert\K_{=n}[X]\right\rvert=p^{n}(p-1)$ d'où $\left\lvert\Omega\right\rvert=p^{2n}(p-1)$.

        On a $\P\left(\deg(Q)=-\infty\right)=\frac{1}{p^{n}}$ et si $k\in\llbracket0,n-1\rrbracket$,
        \begin{equation}
            \boxed{\P\left(\deg(Q)=k\right)=\frac{p^{k}(p-1)}{p^{n}}}
        \end{equation}

        \item On a $(Q,P)\in A$ si et seulement si $Q\mid P$ si et seulement si il existe $A\in\K_{\leqslant n}[X]$ tel que $P=AQ$ et $\deg(A)+\deg(Q)=n$. Ainsi, $\left(Q,\frac{P}{Q}\right)\in B$ et $f$ est bien définie. On a directement
        \function{f^{-1}}{B}{A}{(Q,A)}{(Q,AQ)}
        donc $f$ est bijective et $\left\lvert A\right\rvert =\left\lvert B\right\rvert$.

        On a 
        \begin{equation}
            B=\bigcup_{k=0}^{n-1}\left\lbrace(Q,A)\in\K_{=k}[X]\times\K_{=n-k}[X]\right\rbrace
        \end{equation}
        donc 
        \begin{equation}
            \left\lvert B\right\rvert=\sum_{k=0}^{n-1}p^{k}(p-1)\times p^{n-k}(p-1)=np^{n}(p-1)^{2}=\left\lvert A\right\rvert
        \end{equation}

        Ainsi, 
        \begin{equation}
            \boxed{\P(Q\mid P)=\frac{np^{n}(p-1)^{2}}{p^{2n}(p-1)}=\frac{n(p-1)}{p^{n}}}
        \end{equation}

        \item On a $R_{1}=R$ si et seulement si il existe $A\in\K[X]$ tel que $P=AQ+R$ avec $\deg(R)<\deg(Q)$ si et seulement si $Q\mid P-R$ et $\deg(R)<\deg(Q)$. Comme $\deg(Q)<\deg(P)$, $\deg(R)<\deg(P)$ implique $\deg(P-R)=\deg(P)$.
        
        Or \function{\varphi}{\K_{=n}[X]}{\K_{=n}[X]}{P}{P-R}
        est bijective donc les lois de $P-R$ et de $P$ sont les mêmes. En notant $r=\deg(R)$, on a donc 
        \begin{align}
            \P(R_{1}=R)
            &=\P\left((Q\mid P-R)\cap(\deg(Q)>\deg(R))\right)\\
            &=\sum_{q=r+1}^{n-1}\frac{p^{n-q}(p-1)}{(p-1)^{2}}\times\frac{p^{d}(p-1)}{p^{2n}}\\
            &=\boxed{\frac{1}{p^{n}}\times(n-r-1)}
        \end{align}

        et 
        \begin{align}
            \P_{\deg(Q)=q}(R_{1}=R)
            &=\frac{\P\left((R_{1}=R)\cap(\deg(Q)=q)\right)}{\P\left(\deg(Q)=s\right)}\\
            &=\frac{\frac{1}{p^{n}}}{\frac{p^{q}(p-1)}{p^{n}}}\\
            &=\boxed{\frac{1}{p^{q}(p-1)}}
        \end{align}
    \end{enumerate}
\end{proof}

\begin{proof}
    \phantom{}
    \begin{enumerate}
        \item $(X_{1},X_{2})$ prend ses valeurs dans $\llbracket1,n\rrbracket^{2}\setminus\left\lbrace(i,i)\middle| i\in\left\llbracket1,n\right\rrbracket\right\rbrace$. Soit $(i,j)\in\llbracket1,n\rrbracket^{2}$ avec $i\neq j$, alors 
        \begin{equation}
            \boxed{\P(X_{1}=i,X_{2}=j)=\frac{1}{n(n-1)}}
        \end{equation}
        La loi conjointe est uniforme.

        \item $X_{2}$ prend ses valeurs dans $\llbracket1,n\rrbracket$. On a 
        \begin{equation}
            \P(X_{2}=j)=\sum_{\substack{i=1\\i\neq j}^{n}}\P(X_{1}=i,X_{2}=j)=\frac{n-1}{n(n-1)}=\frac{1}{n}
        \end{equation}
        donc $X_{2}\sim\mathcal{U}\left(\llbracket1,n\rrbracket\right)$.

        $X_{1_{\mid X_{2}=j}}$ prend ses valeurs dans $\llbracket1,n\rrbracket\setminus\lbrace j\rbrace$ et 
        \begin{equation}
            \P(X_{1_{\mid X_{2}=j}}=i)=\frac{\P(X_{1}=i,X_{2}=j)}{\P(X_{2}=j)}=\frac{1}{n-1}
        \end{equation}
        donc $X_{1_{\mid X_{2}=j}}\sim\mathcal{U}\left(\left\llbracket1,n\right\rrbracket\setminus\left\lbrace j\right\rbrace\right)$.

        \item D'après ce qui précède, les lois de $X_{1}$ et $X_{2}$ sont différentes et $X_{1}$ et $X_{2}$ ne sont pas indépendantes.
        
        \item On écrit 
        \begin{equation}
            (X_{1},\dots,X_{k})=\left\lbrace(x_{1},\dots,x_{k})\in\llbracket1,n\rrbracket^{k}\middle| \forall i\neq j,x_{i}\neq x_{j}\right\rbrace
        \end{equation}
        ensemble que l'on note $A_{n,k}$. On a $\left\lvert A_{n,k}\right\rvert=n(n-1)\dots(n-k+1)$. Pour $(x_{1},\dots,x_{k})\in A_{n,k}$, on a donc 
        \begin{equation}
            \boxed{\P(X_{1}=x_{1},\dots,X_{k}=x_{k})=\frac{1}{n(n-1)}\dots(n-k+1)}
        \end{equation}
        et pour tout $i\in\llbracket1,n\rrbracket$, $X_{i}\sim\mathcal{U}(\llbracket1,n\rrbracket)$ et les $X_{i}$ ne sont pas indépendants.

        \item On a 
        \begin{align}
            \E(X_{1},X_{2})
            &=\sum_{\substack{(i,j)\in\llbracket1,n\rrbracket^{2}\\ i\neq j}}(i,j)\times\frac{1}{n(n-1)}\\
            &=\frac{1}{n(n-1)}\left(\sum_{i=1}^{n}\left(\sum_{j\neq i}i\right),\sum_{j=1}^{n}\left(\sum_{i\neq j}j\right)\right)\\
            &=\frac{1}{n(n-1)}\left(\frac{n(n-1)(n+1)}{2},\frac{n(n-1)(n+1)}{2}\right)\\
            &=\boxed{\frac{n+1}{2},\frac{n+1}{2}}
        \end{align}
    \end{enumerate}
\end{proof}

\begin{proof}
    \phantom{}
    \begin{enumerate}
        \item $R_n$="les $n$ premiers tirages sont rouges". Avec la formule des probabilités conditionnelles,
        \begin{equation*}
            \P(R_n)=\P(R_1)\P_{R_1}(R_2)\dots\P_{R_{n-1}}(R_n)=\frac{1}{2}\times\frac{3}{4}\times\dots\times\frac{2n-1}{2n}.
        \end{equation*}
        car $R_n\subset R_{n-1}$.
        \item $R=\cap_{n\in\N^{*}}R_n$, suite décroissante d'évènements. Donc 
        \begin{equation*}
            \P(R)=\lim\limits_{n\to+\infty}\P(R_n)=\prod_{n=1}^{+\infty}\left(\frac{2n-1}{2n}\right),
        \end{equation*}
        et $\ln(\P(R_n))=\sum_{k=1}^{n}\ln\left(1-\frac{1}{2k}\right)\xrightarrow[n\to+\infty]{}-\infty$ car le terme est équivalent à $\frac{-1}{2k}<0$ donc $\P(R)=0$.

        \item Par symétrie, la probabilité de tirer une boule rouge ou une blanche est la même, leur somme égale à 1, donc on a une probabilité de $\frac{1}{2}$.
        \item On refait les mêmes calculs qu'à la premier étape, si ce n'est que $\ln(\P(R_n))=\sum_{k=1}^{n}\ln\left(1-\frac{2}{4+k(k+1)}\right)$ converge donc $\P(R)>0$.
    \end{enumerate}
\end{proof}

\begin{remark}
    Soit une suite infinie (dénombrable) de tirages de pile ou face indépendantes suivant chacun une loi Bernoulli de paramètre $p\in(0,1)$. La probabilité qu'une séquence donnée de $N$ résultats apparaisse une infinité de fois vaut 1 d'après le lemme de Borel-Cantelli.
\end{remark}

\end{document}
\documentclass[12pt]{article}
\usepackage{style/style_sol}

\begin{document}

\begin{titlepage}
	\centering
	\vspace*{\fill}
	\Huge \textit{\textbf{Solutions MP/MP$^*$\\ Calcul matriciel}}
	\vspace*{\fill}
\end{titlepage}

\begin{proof}
    Soit $(k,m)\in\left(\N^{*}\right)^{2}$, on a 
    \begin{align}
        \left[M\overline{M}\right]_{k,m}
        &=\sum_{j=1}^{n}\omega^{(k-1)(j-1)}\overline{\omega}^{(j-1)(m-1)}\\
        &=\sum_{j=0}^{n-1}\left[\omega^{k-1}\overline{\omega}^{m-1}\right]^{j}\\
        &=\sum_{j=0}^{n-1}\left[\omega^{k-m}\right]^{j}
    \end{align}

    Or $\omega^{k-m}=1$ si et seulement si $n\mid k-m$ si et seulement si $k=m$ car $\left\lvert k-m\right\rvert\in\llbracket0,n-1\rrbracket$. Si $k=m$, on a $[M\overline{M}]_{k,m}=n$ et si $k\neq m$, on a 
    \begin{equation}
        [M\overline{M}]_{k,m}=\frac{1-\left(\omega^{k-m}\right)^{n}}{1-\omega^{k-m}}=0
    \end{equation}
    Donc $M\overline{M}=nI_{n}$. Ainsi, $M\in GL_{n}(\C)$ et 
    \begin{equation}
        \boxed{M^{-1}=\frac{1}{n}\overline{M}}
    \end{equation}

    On a $\det(M\overline{M})=\det(M)\det(\overline{M})=n^{n}=\det(M)\overline{\det(M)}=\left\lvert\det(M)\right\rvert^{2}$ donc $\left\lvert\det(M)\right\rvert=n^{\frac{n}{2}}$.

    On calcul $M^{2}$. On a 
    \begin{equation}
        [M^{2}]_{k,m}=\sum_{j=1}^{n}\omega^{(k-1)(j-1)+(j-1)(m-1)}=\sum_{j=0}^{n-1}\left[\omega^{k+m-2}\right]^{j}
    \end{equation}

    On a $k+m-2\in\llbracket0,2n-2\rrbracket$ donc $n\mid k+m-2$ si et seulement si $k+m=n+2$ ou $k+m=2$ si et seulement si $m=n+2-k$ ou $k=m=1$. Donc 
    \begin{equation}
        M^{2}=
        \begin{pmatrix}
            n       & 0         & \dots     & \dots & 0\\
            0       &           &           &       & n\\
            \vdots  &           &           & n     & 0\\
            \vdots  &           & \iddots   &       & \vdots\\
            0       & n         &\dots      &\dots  &0
        \end{pmatrix}
    \end{equation}

    En développant par rapport à la première ligne (ou colonne), on a 
    \begin{equation}
        \det(M^{2})=n^{n}(-1)^{\frac{n(n+1)}{2}}
    \end{equation}

    donc 
    \begin{equation}
        \boxed{\det(M)=
        \left\lbrace
            \begin{array}[]{ll}
                \pm n^{\frac{n}{2}} &\text{si }\frac{n(n+1)}{2}\text{ est pair i.e. }
                \left\lbrace
                \begin{array}[]{l}
                    n\equiv 0[4]\\
                    \text{ou}\\
                    n\equiv 3[4]
                \end{array}
                \right.\\
                \pm \i n^{\frac{n}{2}} &\text{si }\frac{n(n+1)}{2}\text{ est impair i.e. }
                \left\lbrace
                \begin{array}[]{l}
                    n\equiv 1[4]\\
                    \text{ou}\\
                    n\equiv 2[4]
                \end{array}
                \right.\\
            \end{array}
        \right.
        }
    \end{equation}
\end{proof}

\begin{proof}
    \phantom{}
    \begin{enumerate}
        \item Si $A\geqslant0$, soit $X\geqslant0$, on a 
        \begin{equation}
            [AX]_{i}=\sum_{j=1}^{n}a_{i,j}x_{j}\geqslant 0
        \end{equation}
        donc $AX\geqslant0$.

        Réciproquement, soit $j\in\llbracket1,n\rrbracket$, on prend 
        \begin{equation}
            X_{j}=
            \begin{pmatrix}
                0\\
                \vdots\\
                0\\
                1\\
                0\\
                \vdots\\
                0    
            \end{pmatrix}
        \end{equation}
        où le 1 est en $j$-ième position. $X_{j}\geqslant0$ et 
        \begin{equation}
            AX=
            \begin{pmatrix}
                a_{1,j}\\
                \vdots\\
                a_{n,j}
            \end{pmatrix}
            \geqslant0
        \end{equation}
        donc $A\geqslant0$.

        \item Soit $A\in GL_{n}(\R)$ avec $A=(A_{i,j})_{1\leqslant i,j\leqslant n}\geqslant0,A^{-1}=(A^{-1})_{1\leqslant i,j\leqslant n}\geqslant0$. Soit $(i,j)\in\llbracket1,n\rrbracket^{2}$ avec $i\neq j$. On a 
        \begin{equation}
            \sum_{k=1}^{n}A_{i,k}A^{-1}_{k,j}=0
        \end{equation}
        donc pour tout $k\in\llbracket1,n\rrbracket$ on a $A_{i,j}=0$ ou $A^{-1}_{k,j}=0$.

        $i$ étant fixé, comme $A\in Gl_{n}(\R)$, il existe $k_{0}\in\llbracket1,n\rrbracket$ tel que $A_{i,k_{0}}>0$. Alors pour tout $j\in\llbracket1,n\rrbracket\setminus\lbrace i\rbrace$, on a $A^{-1}_{k_{0},j}=0$ et $A^{-1}_{k_{0},i}>0$ (car $A^{-1}$ est inversible). Supposons qu'il existe $k_{1}\neq k_{0}$ tel que $A_{i,k_{1}}>0$. Alors pour tout $j\neq i$, on a $A^{-1}_{k,j}=0$ et $A^{-1}_{k_{1},i}>0$, mais alors les lignes $k_{0}$ et $k_{1}$ sont liées, ce qui est impossible. Donc il existe un unique $k_{i}\in\llbracket1,n\rrbracket$, $A_{i,k_{i}}>0$.

        Comme $A$ est inversible, pour $i\neq i'$, on a $k_{i}\neq k_{i'}$, sinon on aurait deux lignes proportionnelles. Donc \function{\Delta}{\llbracket1,n\rrbracket}{\llbracket1,n\rrbracket}{i}{k_{i}}
        Ainsi il existe une unique permutation $\sigma\in\Sigma_{n}$ telle que pour tout $i\in\llbracket1,n\rrbracket$, $A_{i,\sigma(i)}>0$ et pour tout $j\neq\sigma(i)$, $A_{ij}=0$. Donc 
        \begin{equation}
            \boxed{A=\diag(a_{1},\dots,a_{n})P_{\sigma}}
        \end{equation}
        avec $P_{\sigma}=(\delta_{i,\sigma(j)})_{i,j}$ et $a_{i}>0$.

        Réciproquement, si $A$ est de cette forme, on a $A\geqslant0$ et 
        \begin{equation}
            A^{-1}=P^{-1}_{\sigma}\diag\left(\frac{1}{a_{1}},\dots,\frac{1}{a_{n}}\right)=P_{\sigma^{-1}}\diag\left(\frac{1}{a_{1}},\dots,\frac{1}{a_{n}}\right)
        \end{equation}
        donc $A^{-1}\geqslant0$.
    \end{enumerate}
\end{proof}

\begin{remark}
    Soit 
    \begin{equation}
        A=
        \begin{pmatrix}
            2   & -1    & 0   &\dots &0\\
            -1  & \ddots&\ddots     &\ddots&\vdots\\
            0   & \ddots&\ddots     &\ddots&0\\
            \vdots&\ddots&\ddots&\ddots&-1\\
            0&\dots&0&-1&2
        \end{pmatrix}
    \end{equation}

    Si $AX\geqslant0$, en définissant $x_{0}=x_{n+1}=0$, on a pour tout $k\in\llbracket1,n\rrbracket$,
    \begin{equation}
        -x_{k-1}+2x_{k}-x_{k+1}\geqslant0
    \end{equation}
    Si $x_{i_{0}}=\min(x_{0},\dots,x_{n+1})$, on a 
    \begin{equation}
        2x_{i_{0}}\geqslant x_{i_{0}-1}+x_{i_{0}+1}\geqslant 2x_{i_{0}}
    \end{equation}
    donc $x_{i_{0}-1}=x_{i_{0}+1}=x_{i_{0}}$. De proche en proche, on a $x_{i_{0}}=x_{0}=0$. Donc $X\geqslant0$.

    Si $AX=0$, on a $AX\geqslant0$ et $A(-X)=0$ donc $X\geqslant0$ et $-X\geqslant0$ donc $X=0$ et $A\in GL_{n}(\R)$ et pour tout $Y=AX\geqslant0$, on a $A^{-1}Y=X\geqslant0$ donc $A^{-1}\geqslant0$.
\end{remark}

\begin{proof}
    Soit \function{u}{\R_{n-1}[X]}{\R_{n-1}[X]}{P}{P(X+1)}
    Pour tout $j\in\llbracket1,n\rrbracket$,
    \begin{equation}
        (X+1)^{j-1}=\sum_{i=0}^{j-1}\binom{j-1}{i}X^{i}=\sum_{i=1}^{j}\binom{j-1}{i-1}X^{i-1}
    \end{equation}
    On note $P_{i}=X^{i-1}$ et $\mathcal{B}=(P_{1},\dots,P_{n})$ la base canonique de $\R_{n-1}[X]$. On note $A=\mat_{\mathcal{B}}(u)$. $u^{-1}\colon P\mapsto P(X-1)$ donc $A$ est inversible et pour tout $k\in\llbracket1,n\rrbracket$,
    \begin{equation}
        (X-1)^{j-1}=\sum_{i=0}^{j-1}\binom{j-1}{i}X^{i}(-1)^{j-i-1}=\sum_{i=1}^{j}\binom{j-1}{i-1}X^{i-1}(-1)^{j-i}
    \end{equation}
    donc 
    \begin{equation}
        \boxed{A^{-1}=\left(\binom{j-1}{i-1}(-1)^{j-i}\right)_{1\leqslant i,j\leqslant n}}
    \end{equation}

    Pour tout $k\in\N$, on a $u^{k}\colon P\mapsto P(X+k)$ et pour tout $j\in\llbracket1,n\rrbracket$,
    \begin{equation}
        (X+k)^{j-1}=\sum_{i=0}^{j-1}\binom{j-1}{i}X^{i}k^{j-i-1}=\sum_{i=1}^{j}\binom{j-1}{i-1}X^{i-1}k^{j-i}
    \end{equation}
    donc 
    \begin{equation}
        \boxed{A^{k}=\left(\binom{j-1}{i-1}k^{j-i}\right)_{1\leqslant i,j\leqslant n}}
    \end{equation}
\end{proof}

\begin{proof}
    \label{sol:4.4}
    \phantom{}
    \begin{enumerate}
        \item Pour $n\in\N^{*}$, on note $H(n)$:`si $\dim(E)=n$ et si $u\in\L(E)$ vérifie $\Tr(u)=0$, alors il existe une base $\mathcal{B}$ de $E$ telle que 
        \begin{equation}
            \mat_{\mathcal{B}}(u)=
            \begin{pmatrix}
                0 & \star & \dots & \star\\
                \star & \ddots & \ddots & \vdots\\
                \vdots & \ddots & \ddots & \star\\
                \star & \dots & \star &0
            \end{pmatrix}
        \end{equation}'

        Pour $n=1$, on a $u=0$ si $\Tr(u)=0$. Pour $n\geqslant1$, on suppose $H(n)$, soit $E$ de dimension $n+1$ et $u\in\L(E)$ tel que $\Tr(u)=0$. S'il existe $\lambda\in\K$ tel que $u=\lambda id_{E}$, on a $\Tr(u)=(n+1)\lambda=0$ donc $\lambda=0$ donc $u=0$. 

        Sinon, il existe $e_{1}\neq0$ tel que $(e_{1},u(e_{1}))$ est libre (résultat classique, redémontré en remarque ci-dessous). On pose $e_{2}=u(e_{1})$ et on complète $(e_{1},e_{2})$ en une base de $E$: $(e_{1},e_{2},\dots,e_{n+1})=\mathcal{B}_{1}$. Alors $\mat_{\mathcal{B}_{1}}(u)$ est de la forme 
        \begin{equation}
            \begin{pmatrix}
                0 & 
                \begin{matrix}
                    \star & \dots & \dots & \star    
                \end{matrix}\\
                \begin{matrix}
                    1\\
                    0\\
                    \vdots\\
                    0
                \end{matrix}
                & \mbox{\Huge A'}
            \end{pmatrix}
        \end{equation}
        avec $\Tr(u)=\Tr(A')=0$. Posons $F=\Vect(e_{2},\dots,e_{n+1})$. On note $\Pi$ la projection sur $F$ parallèlement à $\Vect(e_{1})$. Alors si \function{u'}{F}{F}{x}{\Pi(u(x))}
        et $A'=\mat_{(e_{2},\dots,e_{n+1})}(u')$ donc $\Tr(u')=0$. D'après $H(n)$, il existe $(f_{2},\dots,f_{n+1})$ une base de $F$ telle que 
        \begin{equation}
            \mat_{(f_{2},\dots,f_{n+1})}(u')=
            \begin{pmatrix}
                0 & \star & \dots & \star\\
                \star & \ddots & \ddots & \vdots\\
                \vdots & \ddots & \ddots & \star\\
                \star & \dots & \star &0
            \end{pmatrix}
        \end{equation}

        Soit donc $\mathcal{B}_{2}=(e_{1},f_{2},\dots,f_{n+1})$ base de $E$. On a $u(e_{1})\in F$ donc 
        \begin{equation}
            \boxed{
            \mat_{\mathcal{B}_{2}}(u)=
            \begin{pmatrix}
                0 & \star & \dots & \star\\
                \star & \ddots & \ddots & \vdots\\
                \vdots & \ddots & \ddots & \star\\
                \star & \dots & \star &0
            \end{pmatrix}
            }
        \end{equation}

        \item Soit $M=(a_{i,j})_{1\leqslant i,j\leqslant n}$ et $D=(i\delta_{i,j})_{1\leqslant i,j\leqslant n}$. On a 
        \begin{equation}
            [DM]_{i,j}=\sum_{k=1}^{n}i\delta_{i,k}a_{k,j}=ia_{i,j}
        \end{equation}
        et 
        \begin{equation}
            [MD]_{i,j}=\sum_{k=1}^{n}a_{i,k}k\delta_{k,j}=ja_{i,j}
        \end{equation}

        On a $M\in\ker(\varphi)$ si et seulement si pour tout $i\neq j$, $a_{i,j}=0$ si et seulement si $M\in D_{n}(\K)$ (ensemble des matrices diagonales). Donc $\dim(\ker(\varphi))=n$ et $\dim(\im(\varphi))=n^{n}-n$. Or pour tout $M\in \M_{n}(\K), [MD-DM]_{i,i}=0$. Notons $\Delta_{n}$ l'ensemble des matrices de diagonale nulle. On a $\im\varphi\subset\Delta_{n}$ et $\dim(\Delta_{n})=n^{2}-n$ (une base de $\Delta_{n}$ est $(E_{i,j})_{i\neq j}$, matrices élémentaires) donc $\im(\varphi)=\Delta_{n}$.

        Soit alors $A\in\M_{n}(\K)$ telle que $\Tr(A)=0$. D'après 1. il existe $P\in GL_{n}(\K)$ telle que $P^{-1}AP\in\Delta_{n}=\im(\varphi)$ donc il existe $M\in\M_{n}(\K)$ telle que $P^{-1}AP=MD-DM$ donc 
        \begin{align}
            A
            &=P(MD-DM)P^{-1}\\
            &=PM DP^{-1}-PD MP^{-1}\\
            &=\boxed{XY-YX}
        \end{align}
        avec $X=PMP^{-1}$ et $Y=PDP^{-1}$.
    \end{enumerate}
\end{proof}

\begin{remark}
    Soit $E$ un $\K$-espace vectoriel et $u\in\L(E)$ tel que pour tout $x\in E\setminus\lbrace0\rbrace$, $(x,u(x))$ est liée i.e.~pour tout $x\in E\setminus\lbrace0\rbrace$, $u(x)=\lambda_{x}x$. Alors $u$ est une homothétie.

    En effet, soit $(x,y)\in\left(E\setminus\lbrace0\rbrace\right)^{2}$, si $(x,y)$ est liée, il existe $\mu\in\K^{*}$ tel que $y=\mu x$. On a alors 
    \begin{equation}
        u(y)=\lambda_{y}y=\mu u(x)=\mu\lambda_{x}x=\lambda_{x}y
    \end{equation}
    On a $y\neq0$ donc $\lambda_{x}=\lambda_{y}$.

    Si $(x,y)$ est libre, on a
    \begin{equation}
        u(x+y)=\lambda_{x+y}(x+y)=\lambda_{x}x+\lambda_{y}y
    \end{equation}
    Par liberté de $(x,y)$, on a $\lambda_{x+y}=\lambda_{x}=\lambda_{y}$. 
    
    Ainsi, $\lambda_{x}$ ne dépend pas de $x$: il existe $\lambda\in\K$ tel que pour tout $x\in E$, $u(x)=\lambda x$, i.e.~$u=\lambda id_{E}$.
\end{remark}

\begin{proof}
    \phantom{}
    \begin{enumerate}
        \item Si
        $X=\begin{pmatrix}
            x_{1} & \dots &x_{n}
        \end{pmatrix}^{\mathsf{T}}$ et $Y=\begin{pmatrix}
            x_{1} & \dots &x_{n}
        \end{pmatrix}^{\mathsf{T}}$, on a 
        \begin{equation}
            XY^{\mathsf{T}}=
            \begin{pmatrix}
                x_{1}y_{1}&\dots&\dots&x_{1}y_{n}\\
                \vdots & & & \vdots\\
                \vdots & & & \vdots\\
                x_{n}y_{1} & \dots & \dots & x_{n}y_{n}
            \end{pmatrix}
        \end{equation}
        est de rang 1. On a 
        \begin{equation}
            (XY^{\mathsf{T}})^{2}=X(Y^{\mathsf{T}}X)Y^{\mathsf{T}}=\left(\sum_{i=1}^{n}x_{i}y_{i}\right)XY^{\mathsf{T}}
        \end{equation}

        Si $\lambda=0$, c'est évident.

        Si $\lambda\neq0$ et $B=I_{n}+\lambda XY^{\mathsf{T}}$, on a
        \begin{equation}
            XY^{\mathsf{T}}=\frac{B-I_{n}}{\lambda}
        \end{equation}
        et 
        \begin{equation}
            (XY^{\mathsf{T}})^{2}=\frac{\left(B-I_{n}\right)^{2}}{\lambda^{2}}
        \end{equation}
        soit 
        \begin{equation}
            (XY^{\mathsf{T}})^{2}=\frac{B^{2}-2B+I_{n}}{\lambda^{2}}=\left(\sum_{i=1}^{n}x_{i}y_{i}\right)\left(\frac{B-I_{n}}{\lambda}\right)
        \end{equation}
        d'où 
        \begin{equation}
            \lambda\left(Y^{\mathsf{T}}X\right)\left(B-I_{n}\right)=B^{2}-2B+I_{n}
        \end{equation}
        d'où
        \begin{equation}
            B^{2}+\left(-2-\lambda\left(Y^{\mathsf{T}}X\right)\right)B+I_{n}\left(1+\lambda\left(Y^{\mathsf{T}}X\right)\right)=0
        \end{equation}

        Si $1+\lambda Y^{\mathsf{T}}X\neq0$, alors $B$ est inversible et 
        \begin{equation}
            \boxed{B^{-1}=-\frac{1}{1+\lambda Y^{\mathsf{T}}X}\left(B-\left(2+\lambda Y^{\mathsf{T}}X\right)I_{n}\right)}
        \end{equation}

        Si $1+\lambda Y^{\mathsf{T}}X=0$, on a 
        \begin{equation}
            B\left(B-I_{n}\right)=0
        \end{equation}
        Si $B$ est inversible, on aura $B=I_{n}$ et $\lambda XY^{\mathsf{T}}=O_{\M_{n}(\K)}$. Or $\lambda\neq0$ donc $X=Y=0$ et $1=0$: absurde. Donc $B\notin GL_{n}(\K)$.

        \item On a 
        \begin{equation}
            M=A+\lambda XY^{Y}=A\left(I_{n}+\lambda A^{-1}XY^{\mathsf{T}}\right)
        \end{equation}
        donc $M\in GL_{n}(\R)$ si et seulement si $\left(I_{n}+\lambda A^{-1}XY^{\mathsf{T}}\right)$ est inversible si et seulement si $1+\lambda Y^{\mathsf{T}}A^{-1}X$ est inversible d'après 1. Alors 
        \begin{equation}
            \boxed{M^{-1}=\left(I_{n}-\frac{\lambda A^{-1}XY^{\mathsf{T}}}{1+\lambda Y^{\mathsf{T}}A^{-1}X}\right)A^{-1}}
        \end{equation}
    \end{enumerate}
\end{proof}

\begin{proof}
    On a $\dim(\R_{n}[X])=n+1$ donc il faut montrer que $(S_{0},\dots, S_{n})$ est libre. Soit donc $\alpha=\left(\alpha_{0},\dots,\alpha_{n}\right)\in\R^{n+1}$ tel que 
    \begin{equation}
        \alpha_{0}S_{0}+\dots+\alpha_{n}S_{n}=0
    \end{equation}

    Si $\alpha\neq0$, on pose $k_{0}=\max\left(k\in\llbracket0,n\rrbracket\middle|\alpha_{k}\neq0\right)$. On a 
    \begin{equation}
        \alpha_{0}(1-X)^{n}+\dots+\alpha_{k_{0}}X^{k_{0}}(1-X)^{n-k_{0}}=0
    \end{equation}
    soit 
    \begin{equation}
        \alpha_{0}(1-X)^{k_{0}}+\dots+\alpha_{k_{0}}X^{k_{0}}=0
    \end{equation}
    En évaluant en 1, on a $\alpha_{k_{0}}=0$ ce qui est absurde. Donc $(S_{0},\dots,S_{n})$ est une base de $\R_{n}[X]=n+1$.

    Pour tout $k\in\llbracket0,n\rrbracket$, on a 
    \begin{align}
        S_{j}
        &=X^{j}(1-X)^{n-j}\\
        &=X^{j}\left(\sum_{k=0}^{n-j}\binom{n-j}{k}(-1)^{k}X^{k}\right)\\
        &=\sum_{k=0}^{n-j}\binom{n-j}{k}(-1)^{k}X^{k+j}\\
        &=\sum_{k=j}^{n}\binom{n-j}{k-j}(-1)^{k-j}X^{k}
    \end{align}
    donc 
    \begin{equation}
        \boxed{
            A=P_{(1,\dots,X^{n})\to(S_{0},\dots, S_{n})}=\left(\binom{n-j}{k-j}(-1)^{k-j}\right)_{0\leqslant k,j\leqslant n}
        }
    \end{equation}

    On considère $u\in\L\left(\R[X]\right)$ tel que $u(X^{j})=S_{j}$ pour tout $j\in\llbracket0,n\rrbracket$. On a $u(X^{j})=\left(\frac{X}{1-X}\right)^{j}(1-X)^{n}$. Pour tout $P\in\R_{n}[X]$, on a $u(P)=P\left(\frac{X}{1-X}\right)(1-X)^{n}$. Soit $(P,Q)\in\R_{n}[X]^{2}$, on a $u(P)=Q$ si et seulement si $P\left(\frac{X}{1-X}\right)(1-X)^{n}=Q(X)$ si et seulement si $P(Y)\left(\frac{1}{1+Y}\right)^{n}=Q\left(\frac{Y}{1+Y}\right)$ soit $u(P)=Q$ si et seulement si $P(Y)=Q\left(\frac{Y}{1+Y}\right)(1+Y)^{n}$. Ainsi $u^{-1}(X^{j})=X^{j}(1+X)^{n-j}$, donc 
    \begin{equation}
        \boxed{
            A^{-1}=\mat_{(1,\dots,X^{n})}(u^{-1})=\left(\binom{n-j}{k-j}\right)_{0\leqslant k,j\leqslant n}
        }
    \end{equation}
\end{proof}

\begin{proof}
    Si on a $H\cap GL_{n}(\K)=\emptyset$, on a $I_{n}\notin H$. On écrit donc 
    \begin{equation}
        \M_{n}(\K)=H\oplus \K I_{n}
    \end{equation}
    Soit $i\neq j$, on prend $E_{i,j}=M+\lambda I_{n}$ (décomposition précédente) avec $\lambda\in\K$. Si $\lambda\neq0$, on a
    \begin{equation}
        M=E_{i,j}-\lambda I_{n}\in GL_{n}(\K)
    \end{equation}
    donc $M\in GL_{n}(\K)\cap H$: absurde. Donc $\lambda=0$ et $E_{i,j}\in H$, d'où $\Vect(E_{i,j})_{,\neq i}\subset H$. Or
    \begin{equation}
        \begin{pmatrix}
            0 & \dots & \dots & 0 & 1\\
            1 & \ddots & & & 0\\
            0 & \ddots & \ddots& & \vdots\\
            \vdots & \ddots & \ddots&\ddots & \vdots\\
            0 & \dots& 0 & 1 & 0
        \end{pmatrix}\in \left(GL_{n}(\K)\cap\Vect(E_{i,j})_{i\neq j}\right)\subset \left(GL_{n}(\K)\cap H\right)
    \end{equation}
    donc $H\cap GL_{n}(\K)\neq\emptyset$: absurde.
\end{proof}

\begin{remark}
    Il existe une forme linéaire non nulle $\varphi\colon\M_{n}(\K)\to \K$ telle que $H=\ker(\varphi)$. 

    En effet, pour toute forme linéaire $\varphi$ sur $\M_{n}(\K)$, il existe une unique matrice $A\in\M_{n}(\K)$ telle que 
    \begin{equation}
        \varphi(M)=\Tr(AM)
    \end{equation}

    Pour le montrer: si $A$ existe, pour tout $(i,j)\in\llbracket1,n\rrbracket^{2}$, $\varphi(E_{i,j})=\Tr(AE_{i,j})=a_{j,i}$. Réciproquement, soit $A=\left(\varphi(E_{j,i})\right)_{1\leqslant i,j\leqslant n}$. On a pour tout $M\in\M_{n}(\K)$, $\varphi(M)=\Tr(AM)$ car cex deux formes linéaires coïncident sur les $(E_{i,j})_{1\leqslant i,j\leqslant n}$.

    Il existe donc $A\in \M_{n}(\K)\setminus\lbrace0\rbrace$,
    \begin{equation}
        H=\left\lbrace M\in\M_{n}(\K)\middle|\Tr(AM)=0\right\rbrace
    \end{equation}

    Si $r=\rg(A)$, il existe $(P,Q)\in GL_{n}(\K)^{2}$ telles que $A=Q^{-1}J_{n,n,r}P$ ($J_{n,n,r}$: matrice de taille $n\times n$ avec les $r$ premiers coefficients diagonaux valant 1). Alors pour tout $M\in \M_{n}(\K)$, on a 
    \begin{equation}
        \Tr(AM)=\Tr(J_{n,n,r}\underbrace{MPQ^{-1}}_{=~M'})
    \end{equation}
    et il suffit de prendre 
    \begin{equation}
        M'=
        \begin{pmatrix}
            0 & \dots & \dots & 0 & 1\\
            1 & \ddots & & & 0\\
            0 & \ddots & \ddots& & \vdots\\
            \vdots & \ddots & \ddots&\ddots & \vdots\\
            0 & \dots& 0 & 1 & 0
        \end{pmatrix}
    \end{equation}
\end{remark}

\begin{remark}
    Si $F$ est un sous-espace vectoriel de $\M_{n}(\K)$ vérifie $\dim(F)\geqslant n^{2}-n+1$ alors 
    \begin{equation}
        G\cap GL_{n}(\K)\neq\emptyset
    \end{equation}
\end{remark}

\begin{proof}
    \phantom{}
    \begin{enumerate}
        \item On prend $\lambda=0$ et $N(0\times 0)=0\times N(0)=0$ donc 
        \begin{equation}
            \boxed{N(0)=0}
        \end{equation}

        \item On a pour $j\neq i,$ $E_{i,j}\times E_{j,j}=E_{i,j}$ et $E_{j,j}E_{i,j}=0$ donc $N(E_{i,j})=N(E_{i,j}E_{j,j})=N(E_{j,j}E_{i,j})$ d'où 
        \begin{equation}
            \boxed{N(E_{i,j})=0}
        \end{equation}

        \item Déjà traité à l'Exercice~4.
        
        \item Si $\Tr(A)=0$, alors il existe $P\in GL_{n}(\C)$ telle que 
        \begin{equation}
            P^{-1}AP=\sum_{i\neq j}\alpha_{i,j}E_{i,j}
        \end{equation}
        donc 
        \begin{equation}
            \boxed{N(A)=N(P^{-1}AP)\leqslant\sum_{i\neq j}\alpha_{i,j}N(E_{i,j})=0}
        \end{equation}

        \item Soit $A'=A-\frac{\Tr(A)}{n}I_{n}$. On a $N(A')=0$ d'après ce qui précède. Montrons que pour tout $(A,B)\in\M_{n}(\C)^{2}$,
        \begin{equation}
            \left\lvert N(A)-N(B)\right\rvert\leqslant N(A-B)
        \end{equation}
        On écrit $A=A-B+B$ et $N(A)\leqslant N(A-B)+N(B)$ d'où $N(A)-N(B)\leqslant N(A-B)$ et on a le résultat par symétrie de $A$ et $B$.

        On a donc 
        \begin{equation}
            \left\lvert N(A)-N\left(\frac{\Tr(A)}{n}I_{n}\right)\right\rvert\leqslant N\left(A-\frac{\Tr(A)}{n}I_{n}\right)=0
        \end{equation}
        d'où 
        \begin{equation}
            \boxed{N(A)=N\left(\frac{\Tr(A)}{n}I_{n}\right)=\left\lvert\Tr(A)\right\rvert\times \underbrace{N\left(\frac{I_{n}}{n}\right)}_{=~a\geqslant0}}
        \end{equation}
    \end{enumerate}
\end{proof}

\begin{proof}
    On écrit 
    \begin{equation}
        f+g=f\circ\left(id+f^{-1}\circ g\right)
    \end{equation}
    avec $f^{-1}\circ g$ de rang 1. Il existe une base $\mathcal{B}
    $ de $E$ telle que 
    \begin{equation}
        \mat_{\mathcal{B}}(f^{-1}\circ g)=
        \begin{pmatrix}
            0 & \dots & \dots & 0 & \star\\
            \vdots & & &\vdots &\vdots\\
            \vdots & & &\vdots & \vdots\\
            0 & \dots & \dots & 0 & \alpha
        \end{pmatrix}
    \end{equation}
    avec $\alpha=\Tr(f^{-1}\circ g)$ et donc $\mat_{\mathcal{B}}(id+g^{-1}\circ g)$ est inversible si et seulement si $1+\alpha\neq0$ si et seulement si $\Tr(f^{-1}\circ g)\neq 1$.
\end{proof}

\begin{proof}
    Par symétrie du problème, il suffit de déterminer les
    \begin{equation}
        a_{n,j}=\left\lvert\left\lbrace\text{chemins de longueur }n\text{ de }1\text{ vers }j\in\left\lbrace2,3,4\right\rbrace\right\rbrace\right\rvert
    \end{equation}

    On pose 
    \begin{equation}
        X_{n}=
        \begin{pmatrix}
            a(n,1)\\
            a(n,2)\\
            a(n,3)\\
            a(n,4)
        \end{pmatrix}
    \end{equation}
    On a alors 
    \begin{equation}
        X_{n+1}=
        \underbrace{
        \begin{pmatrix}
            0 &1 &0 &1\\
            1 &0 &1 &0\\
            0 &1 &0 &1\\
            1 &0 &1 &0
        \end{pmatrix}}_{=~A}
        X_{n}
    \end{equation}
    car (en raisonnant modulo 4) il y a autant de chemins de longueur n+1 reliant 1 à j que de chemins de longueur n reliant 1 à j-1 + chemins de longueur n reliant 1 à j+1.
    d'où $X_{n}=A^{n}X_{0}$ avec 
    \begin{equation}
        X_{0}=
        \begin{pmatrix}
            1\\0\\0\\0
        \end{pmatrix}
    \end{equation}
    On a
    \begin{equation}
        A=
        \begin{pmatrix}
            B&B\\B&B
        \end{pmatrix}
    \end{equation}
    avec
    \begin{equation}
        B=
        \begin{pmatrix}
            0&1\\1&0
        \end{pmatrix}
    \end{equation}
    On a $B^{2}=I_{2}$ et on montre par récurrence
    \begin{equation}
        \left\lbrace
            \begin{array}[]{ll}
                A^{2p}=2^{2p-1}
                \begin{pmatrix}
                    I_{2} &I_{2}\\ I_{2}&I_{2}
                \end{pmatrix} &p\geqslant1\\
                A^{2p+1}=2^{2p}
                \begin{pmatrix}
                    B & B\\B &B
                \end{pmatrix}&p\geqslant0
            \end{array}
        \right.
    \end{equation}

    Ainsi,
    \begin{equation}
        \boxed{
            \begin{array}[]{l}
                a(2p,1)=2^{2p-1}=a(2p,3)\\
                a(2p,2)=0=a(2p,4)\\
                a(2p+1,1)=0=a(2p+1,3)\\
                a(2p+1,4)=2^{2p}=a(2p+1,4)\\
            \end{array}
        }
    \end{equation}

    \item Ici, on a 
    \begin{equation}
        A=
        \begin{pmatrix}
            0 &1 &0 &1 &0 &1 &0 &0\\
            1 &0 &1 &0 &0 &0 &1 &0\\
            0 &1 &0 &1 &0 &0 &0 &1\\
            1 &0 &1 &0 &1 &0 &0 &0\\
            0 &0 &0 &1 &0 &1 &0 &1\\
            1 &0 &0 &0 &1 &0 &1 &0\\
            0 &1 &0 &0 &0 &1 &0 &1\\
            0 &0 &1 &0 &1 &0 &1 &0
        \end{pmatrix}
    \end{equation}

    En deux itérations, il y a chaque fois deux possibilités pour relier deux sommets différents de même partié, et 3 pour revenir au même sommet. On a donc 
    \begin{equation}
        A^{2}=
        \begin{pmatrix}
            3 &0 &2 &0 &2 &0 &2 &0\\
            0 &3 &0 &2 &0 &2 &0 &2\\
            2 &0 &3 &0 &2 &0 &2 &0\\
            0 &2 &0 &3 &0 &2 &0 &2\\
            2 &0 &2 &0 &3 &0 &2 &0\\
            0 &2 &0 &2 &0 &3 &0 &2\\
            2 &0 &2 &0 &2 &0 &3 &0\\
            0 &2 &0 &2 &0 &2 &0 &3
        \end{pmatrix}
        =I_{8}+2
        \begin{pmatrix}
            B &B &B &B\\
            B &B &B &B\\
            B &B &B &B\\
            B &B &B &B
        \end{pmatrix}
    \end{equation}

    On applique le binôme de Newton pour calculer les puissances paires de $A$, puis on déduit les puissances impaires en multipliant par $A$.
\end{proof}

\begin{proof}
    Soit $X=\begin{pmatrix}
        x_{1}&\dots&x_{n}
    \end{pmatrix}^{\mathsf{T}}\in\M_{n,1}(\C)$. Supposons $AX=0$. Alors pour tout $i\in\llbracket1,n\rrbracket$,
    \begin{equation}
        \sum_{j=1}^{n}a_{i,j}x_{j}=0\Rightarrow -a_{i,i}x_{i}=\sum_{\substack{j=1\\j\neq i}}^{n}a_{i,j}x_{j}
    \end{equation}
    donc 
    \begin{equation}
        \left\lvert\sum_{\substack{j=1\\j\neq i}}^{n}a_{i,j}x_{j}\right\rvert=\left\lvert a_{i,i}x_{i}\right\rvert\leqslant\sum_{\substack{j=1\\j\neq i}}^{n}\left\lvert a_{i,j}x_{j}\right\rvert
    \end{equation}
    Soit $i_{0}\in\llbracket1,n\rrbracket$ tel que 
    \begin{equation}
        x_{i_{0}}=\max\left\lbrace\left\lvert x_{i}\right\rvert,i\in\llbracket1,n\rrbracket\right\rbrace
    \end{equation}

    On a alors 
    \begin{equation}
        \left\lvert a_{j,i_{0}}\right\rvert\left\lvert x_{i_{0}}\right\rvert\leqslant \left\lvert x_{i_{0}}\right\rvert\sum_{\substack{j=1\\j\neq i}}^{n}\left\lvert a_{i_{0},j}\right\rvert
    \end{equation}
    D'après l'hypothèse, on a $\left\lvert x_{i_{0}}\right\rvert=0$ donc $X=0$ et $A$ est inversible.

    Il faut l'inégalité stricte, un contre-exemple est donnée par une ligne nulle.
\end{proof}

\begin{remark}
    Si pour tout $j\in\llbracket1,n\rrbracket$, $\left\lvert a_{j,j}\right\rvert>\sum_{i\neq j}\left\lvert a_{i,j}\right\rvert$ alors $A^{\mathsf{T}}\in GL_{n}(\C)$ et donc $A\in GL_{n}(\C)$.
\end{remark}

\begin{proof}
    On écrit, pour tout $(i,j)\in\llbracket1,n\rrbracket^{2}$,
    \begin{align}
        i\wedge j
        &=\sum_{k\mid i\wedge j}\varphi(k)\\
        &=\sum_{\substack{k\mid i\\ k\mid j}}\varphi(k)\\
        &=\sum_{k=1}^{n}b_{k,i}b_{k,j}\varphi(k)
    \end{align}
    avec $b_{k,i}=1$ si $k\mid i$ et 0 sinon. On a alors, si $A=\left(i\wedge j\right)_{1\leqslant i,j\leqslant n}$, $A=B^{\mathsf{T}}C$ avec $B=(b_{k,i})_{1\leqslant i,k\leqslant n}$ (triangulaire supérieure) et $C=\left(\varphi(k)b_{k,j}\right)_{1\leqslant k,j\leqslant n}$ (triangulaire supérieure). Donc 
    \begin{equation}
        \boxed{\det(A)=\prod_{i=1}^{n}\varphi(i)}
    \end{equation}
\end{proof}

\begin{proof}
    Pour l'unicité, si $A=L_{1}U_{1}=L_{2}U_{2}$ telles que proposées. Comme $A$ est inversible, on a $\det(A)=\det(L_{i})\det(U_{i})\neq0$ pour $i\lbrace1,2\rbrace$ et donc $L_{i}$ et $U_{i}$ sont inversibles. Ainsi,
    \begin{equation}
        L_{2}^{-1}L_{1}=U_{2}U_{1}^{-1}\in \mathcal{T}_{n}^{-}(\C)\cap\mathcal{T}_{n}^{+}(\C)
    \end{equation}
    avec des 1 sur la diagonale, c'est donc $I_{n}$, d'où l'unicité.

    Pour l'existence, on travaille par récurrence sur $n\in\N$: pour $n=1$ on a $A=(1)\times(a_{1,1})$. Soit $A_{n+1}\in \M_{n+1}(\C)$ vérifiant l'hypothèse, alors $A_{n}$ vérifie l'hypothèse $A_{n}=L_{n}U_{n}$ avec 
    \begin{equation}
        A_{n+1}=
        \begin{pmatrix}
            A_{n} & Y\\
            X^{\mathsf{T}} & a_{n+1,n+1}
        \end{pmatrix}
    \end{equation}

    On veut 
    \begin{equation}
        A_{n+1}=
        \begin{pmatrix}
            L' &\begin{matrix}
                0\\\vdots\\0
            \end{matrix}\\
            X_{1}^{\mathsf{T}} &1
        \end{pmatrix}\times\begin{pmatrix}
            U' & Y_{1}\\
            \begin{matrix}
                0&\dots&0
            \end{matrix}&u_{n+1,n+1}
        \end{pmatrix}
    \end{equation}
    On a $(X,Y)\in\M_{n+1}(\C)$, par produits par blocs, on a $A_{n}=L'U'=L_{n}U_{n}$ et par unicité, $L'=L_{n}$ et $U'=U_{n}$. On a $X^{\mathsf{T}}=X_{1}^{\mathsf{T}}U'$ et donc $X_{1}^{\mathsf{T}}=X^{\mathsf{T}}U_{n}^{-1}$ et $Y=L_{n}Y_{1}$ donc $Y_{1}=L_{n}^{-1}Y$.

    Enfin, $a_{n+1,n+1}=X_{1}^{\mathsf{T}}Y_{1}+u_{n+1,n+1}$ et donc 
    \begin{equation}
        u_{n+1,n+1}=a_{n+1,n+1}-X_{1}^{\mathsf{T}}Y_{1}=a_{n+1,n+1}-X^{\mathsf{T}}U_{n}^{-1}L_{n}^{-1}Y
    \end{equation}

    Réciproquement, en définissant ainsi $U$ et $L$, on a bien $A=Lu$ en remontant les calculs.
\end{proof}

\begin{proof}
    On a $\sum_{k\in A_{i}}a_{k}-\sum_{k\in B_{i}}a_{k}=0$ (combinaison linéaire des $a_{k}$ avec des coefficients $\pm1$), donc 
    \begin{equation}
        \underbrace{
            \begin{pmatrix}
                0 & \pm 1&\dots&\dots&\pm1\\
                \pm 1&\ddots&\ddots&&\vdots\\
                \vdots&\ddots&\ddots&\ddots&\vdots\\
                \vdots&&\ddots&\ddots&\pm1\\
                \pm1&\dots&\dots&\pm1&0
            \end{pmatrix}
        }_{=~A}
        \underbrace{
            \begin{pmatrix}
                a_{1}\\
                \vdots\\
                a_{2n+1}
            \end{pmatrix}
        }_{=~X}=0
    \end{equation}

    Sur chaque ligne, il y a $n$ fois 1 et $n$ fois -1 (car les $A_{i}$ et $B_{i}$ sont disjoints). On veut montrer que $X=\alpha\bm{1}$. On a $X\in\ker(A)$ et $\bm{1}\in\ker(A)$ (car il y a $n$ 1 et $n$ -1 par ligne). On veut donc montrer que $\dim(\ker(A))=1$, soit $\rg(A)=2n$.

    On doit donc montrer qu'il existe une sous-matrice de taille $2n$ inversible car $\dim(\ker(A))\geqslant1$. Comme on est bloqué par les $\pm1$, on se place dans $\Z/2\Z$. Soit donc 
    \begin{equation}
        \overline{B_{n}}=
        \begin{pmatrix}
            \overline{0} & \overline{1}&\dots&\dots&\overline{1}1\\
            \overline{1}&\ddots&\ddots&&\vdots\\
            \vdots&\ddots&\ddots&\ddots&\vdots\\
            \vdots&&\ddots&\ddots&\overline{1}1\\
            \overline{1}1&\dots&\dots&\overline{1}1&\overline{0}
        \end{pmatrix}\in\M_{n}\left(\Z/2\Z\right)
    \end{equation}
    Si $\det(\overline{B_{n}})\neq0$, on a $\det(B_{n})\neq 2k$ pour tout $k\in\N$ où $B_{n}$ est obtenue en enlevant à $A$ sa dernière ligne et sa dernière colonne, et donc $\det(A)\neq0$.

    On cherche un polynôme annulateur de $\overline{B_{n}}$. On a 
    \begin{equation}
        \left(\overline{B_{n}}+\overline{I_{2n}}\right)^{2}=\overline{B_{n}}^{2}+2\overline{B_{n}}+I_{2n}=\begin{pmatrix}
            \overline{1}&\dots&\overline{1}\\
            \vdots & & \vdots\\
            \overline{1}&\dots&\overline{1}
        \end{pmatrix}^{2}=2n
        \begin{pmatrix}
            \overline{1}&\dots&\overline{1}\\
            \vdots & & \vdots\\
            \overline{1}&\dots&\overline{1}
        \end{pmatrix}=\begin{pmatrix}
            \overline{0}
        \end{pmatrix}
    \end{equation}

    Ainsi,
    \begin{equation}
        \overline{B_{n}}\left(\overline{B_{n}}+2\overline{I_{2n}}\right)=-\overline{I_{2n}}=\overline{I_{2n}}
    \end{equation}
    donc $\overline{B_{n}}\in GL_{n}\left(\Z/2\Z\right)$ et donc $B_{n}\in GL_{2n}(\R)$, ce qui démontre bien que $\rg(A)=2n$ et $\ker(A)=\Vect(\bm{1})$, d'où
    \begin{equation}
        \boxed{a_{1}=\dots=a_{2n+1}}
    \end{equation}
\end{proof}

\begin{proof}
    On note $T_{i,j}(\lambda)=I_{n}+\lambda E_{i,j}$ pour $i<j$. On rappelle que la multiplication à gauche par $T_{i,j}(\lambda)$ remplace la $i$-ième ligne de la matrice $L_{i}$ par $L_{i}+\lambda L_{j}$: on ajoute à une ligne $\lambda$ fois une ligne d'indice supérieur.
    La multiplication à droite par $T_{i,j}(\lambda)$ remplace la $j$-ième colonne de la matrice $C_{j}$ par $C_{j}+\lambda C_{i}$: on ajoute à une colonne $\lambda$ foi une colonne d'indice inférieur. Ces matrices sont des matrices de transvection.

    On note aussi $D_{i}(\lambda)$ la matrice de dilatation qui contient des 1 sur la diagonale sauf en $i$ position où il y a un $\lambda$. On rappelle que la multiplication à gauche par $D_{i}(\lambda)$ revient à multiplier $L_{i}$ par $\lambda$ et la multiplication à droite revient à multiplier $C_{i}$ par $\lambda$.

    Sur la première colonne de $M$, il y a au moins un coefficient non nul car $M\in GL_{n}(\C)$. Soit $i_{1}=\max\lbrace i\in\llbracket,n\rrbracket,m_{i,1}\neq0\rbrace$. On effectue alors 
    \begin{equation}
        D_{i_{1}}\left(\frac{1}{m_{i_{1},1}}\right)M=
        \begin{pmatrix}
            \star & \star & \dots & \dots & \star\\
            \vdots & \vdots & & &\vdots\\
            \star & \vdots & & & \vdots\\
            1 & \vdots & & & \vdots\\
            0 & \vdots & & &\vdots\\
            \vdots & \vdots & & & \vdots\\
            0 & \star & \dots & \dots & \star
        \end{pmatrix}
    \end{equation}

    Par produite de transvections (qui sont des matrices triangulaires supérieures, i.e.~dans $\mathcal{T}_{n}^{+}$) à gauche, on obtient 
    \begin{equation}
        \begin{pmatrix}
            0 & \star & \dots & \dots & \star\\
            \vdots & \vdots & & &\vdots\\
            0 & \vdots & & & \vdots\\
            1 & \vdots & & & \vdots\\
            0 & \vdots & & &\vdots\\
            \vdots & \vdots & & & \vdots\\
            0 & \star & \dots & \dots & \star
        \end{pmatrix}
    \end{equation}

    Par produite de transvections $\in\mathcal{T}_{n}^{+}$ à droite, on obtient
    \begin{equation}
        \begin{pmatrix}
            0 & \star & \dots & \dots & \star\\
            \vdots & \vdots & & &\vdots\\
            0 & \star &\dots &\dots & \star\\
            1 & 0 &\dots & \dots& 0\\
            0 & \star & \dots& \dots&\star\\
            \vdots & \vdots & & & \vdots\\
            0 & \star & \dots & \dots & \star
        \end{pmatrix}
    \end{equation}

    Soit $M'\in GL_{n}(\C)$ la matrice extraite de $M$ en ôtant la première colonne et la $i_{1}$-ième ligne. On procède par récurrence avec $M'$. Donc il existe $\sigma\in\Sigma_{n},(T,T')\in\left(\mathcal{T}_{n}^{+}\right)^{2}$ telle que 
    \begin{equation}
        \boxed{M=TP_{\sigma}T'}
    \end{equation}

    Montrons que tout matrice de $\mathcal{T}_{n}^{+}$ inversible est produit de matrices de transvections dans $\mathcal{T}_{n}^{+}$ et de dilatations.

    Soit $T\in\mathcal{T}_{n}^{+}\cap GL_{n}(\C)$ avec 
    \begin{equation}
        T=
        \begin{pmatrix}
            t_{1,1}&\star &\dots & \dots& \star\\
            0 & \ddots & \ddots & & \vdots\\
            \vdots & \ddots & \ddots & \ddots & \vdots\\
            \vdots & & \ddots & \ddots & \star\\
            0 &\dots & \dots & 0 & t_{n,n}
        \end{pmatrix}
    \end{equation}

    On a $t_{1,1}\neq0$ car sinon la colonne 1 est nulle. On a donc 
    \begin{equation}
        TD_{1}\left(\frac{1}{t_{1,1}}\right)=
        \begin{pmatrix}
            1&\star &\dots & \dots& \star\\
            0 & \star & \ddots & & \vdots\\
            \vdots & \ddots & \ddots & \ddots & \vdots\\
            \vdots & & \ddots & \ddots & \star\\
            0 &\dots & \dots & 0 & \star
        \end{pmatrix}
    \end{equation}

    Puis, par produit de transvections à droite, on a 
    \begin{equation}
        \begin{pmatrix}
            1&0 &\dots & \dots& 0\\
            0 & \star & \dots &\dots & \star\\
            \vdots & \ddots & \ddots &  & \vdots\\
            \vdots & & \ddots & \ddots & \star\\
            0 &\dots & \dots & 0 & \star
        \end{pmatrix}
    \end{equation}

    On procède ensuite par récurrence sur $n$, et on a 
    \begin{equation}
        T\times B_{1}\times\dots\times B_{l}=I_{n}
    \end{equation}
    donc 
    \begin{equation}
        T=B_{l}^{-1}\times\dots\times B_{1}^{-1}
    \end{equation}
    où $B_{i}\mathcal{T}_{n}^{+}$ transvection ou dilatation.

    Soit donc $(T,T',P_{\sigma})$ vérifiant les hypothèses telles que $M=TP_{\sigma}T'$, alors on a 
    \begin{equation}
        T^{-1}MT'^{-1}=P_{\sigma}=\underbrace{B_{l}^{-1}\times\dots\times B_{1}^{-1}}_{\text{transvections ou dilatations}}\times M\times\underbrace{B_{l}'^{-1}\times\dots\times B_{1}'^{-1}}_{\text{transvections ou dilatations}}
    \end{equation}

    Nécessairement, on a  $\sigma(1)=i$ défini plus haut. Donc de proche en proche, $\sigma$ est univoquement déterminée.

    Cependant, on peut écrire
    \begin{equation}
        I_{2}=
        \begin{pmatrix}
        \frac{1}{2}&0\\
        0&\frac{1}{2}    
        \end{pmatrix}\times I_{2}\times
        \begin{pmatrix}
            2&0\\
            0&2    
        \end{pmatrix}
        =
        I_{2}=
        \begin{pmatrix}
        \frac{1}{3}&0\\
        0&\frac{1}{3}    
        \end{pmatrix}\times I_{2}\times
        \begin{pmatrix}
            3&0\\
            0&3    
        \end{pmatrix}
    \end{equation}
    donc il n'y a pas unicité de $T$ et $T'$.
\end{proof}

\begin{proof}
    \phantom{}
    \begin{enumerate}
        \item Soit $A\in J\cap GL_{n}(\K)$, on a pour tout $M\in\M_{n}(\K)$, on a 
        \begin{equation}
            M=\underbrace{\underbrace{M\times A}_{\in J}\times A^{-1}}_{\in J}\in J
        \end{equation}

        \item Soit $A_{0}\in J\setminus\lbrace0\rbrace$ de rang $r\neq0$. Il existe $(P,Q)\in GL_{n}(\K)$ telle que $Q^{-1}A_{0}P=J_{r}\in J$, on a alors 
        \begin{equation}
            \boxed{J_{r}\times J_{1}=J_{1}\in J}
        \end{equation}

        \item Deux matrices de rang 1 sont équivalentes donc toutes les matrices de rang 1 son dans $J$. Or si $A=\left(a_{i,j}\right)_{1\leqslant i,j\leqslant n}\in\M_{n}(\K)$ s'écrit 
        \begin{equation}
            \boxed{A=\sum_{1\leqslant i,j\leqslant n}\underbrace{a_{i,j}E_{i,j}}_{\text{de rang 1 ou 0}}\in J}
        \end{equation}
    \end{enumerate}
\end{proof}

\begin{proof}
    On a 
    \begin{equation}
        (\lambda A+I_{n})(\lambda B+I_{n})=\lambda^{2} AB+\lambda A+\lambda B+I_{n}=I_{n}
    \end{equation}
    donc $\lambda B+I_{n}$ est inversible. De plus, $A(\lambda B+I_{n})=-B$ donc 
    \begin{equation}
        A=-\left(\lambda B+I_{n}\right)^{-1}B
    \end{equation}

    Or $(\lambda B+I_{n})^{-1}$ et $B$ commutent. En effet, comme $\lambda\neq0$, on a 
    \begin{equation}
        B(\lambda B+I_{n})^{-1}=\left(\left[B+\frac{1}{\lambda I_{n}}\right]-\frac{1}{\lambda}I_{n}\right)(\lambda B+I_{n})^{-1}=\frac{1}{\lambda}I_{n}-\frac{1}{\lambda}(\lambda B+I_{n})^{-1}
    \end{equation}
    et on montre de même que 
    \begin{equation}
        (\lambda B+I_{n})^{-1}B=\frac{1}{\lambda}I_{n}-\frac{1}{\lambda}(\lambda B+I_{n})^{-1}
    \end{equation}

    Ainsi, 
    \begin{equation}
        \boxed{BA=-B(\lambda B+I_{n})^{-1}B=-(\lambda B+I_{n})^{-1}BB=AB}
    \end{equation}
\end{proof}

\begin{proof}
    Soit $X=\begin{pmatrix}
        x_{1} &\dots&x_{n}
    \end{pmatrix}^{\mathsf{T}}$ et $Y=\begin{pmatrix}
        y_{1}&\dots& y_{n}
    \end{pmatrix}^{\mathsf{T}}$

    On a $AX=Y$ si et seulement si 
    \begin{equation}
        \left\lbrace
        \begin{array}[]{lcll}
            x_{1}-a_{2}x_{2}-\dots- a_{n}x_{n} &= &y_{1} &[1]\\
            a_{2}x_{1}+x_{2} &= &y_{2} &[2]\\
            \vdots\\
            a_{n}x_{1}+x_{n}&=&y_{n}&[n]
        \end{array}
        \right.
    \end{equation}
    si et seulement si ($L_{1}\leftarrow L_{1}+\sum_{i=2}^{n}a_{i}L_{i}$)
    \begin{equation}
        \left\lbrace
        \begin{array}[]{lcll}
            \left(1+\sum_{i=2}^{n}a_{i}^{2}\right)x_{1} &= &y_{1}+\sum_{i=2}^{n}a_{i}y_{i} &[1]\\
            a_{2}x_{1}+x_{2} &= &y_{2} &[2]\\
            \vdots\\
            a_{n}x_{1}+x_{n}&=&y_{n}&[n]
        \end{array}
        \right.
    \end{equation}
    si et seulement si 
    \begin{equation}
        \left\lbrace
        \begin{array}[]{lcll}
            x_{1} &= &\dfrac{y_{1}+a_{2}y_{2}+\dots+a_{n}y_{n}}{1+\sum_{i=2}^{n}a_{i}^{2}}\\
            x_{j}&=&y_{j}-a_{j}x_{1} &\forall j\in\llbracket2,n\rrbracket
        \end{array}
        \right.
    \end{equation}
    En posant 
    \begin{equation}
        \lambda=\frac{1}{1+\sum_{i=2}^{n}a_{i}^{2}}
    \end{equation}
    cela équivaut à (en posant $a_{1}=1$)
    \begin{equation}
        \left\lbrace
        \begin{array}[]{lcll}
            x_{1} &= &\lambda(y_{1}+a_{2}y_{2}+\dots+a_{n}y_{n})\\
            x_{j}&=&\lambda\left[\sum_{\substack{i=1\\i\neq j}}a_{i}y_{i}-\left(1+\sum_{\substack{i=1\\i\neq j}}a_{i}^{2}\right)y_{j}\right] &\forall j\in\llbracket2,n\rrbracket
        \end{array}
        \right.
    \end{equation}

    Donc $A\in GL_{n}(\R)$.
\end{proof}

\begin{remark}
    On pourrait se poser la question si $A\in GL_{n}(\C)$ ? Si $1+a_{2}^{2}+\dots+a_{n}^{2}\neq0$, on sait que $A\in GL_{n}(\C)$. Cependant, on vérifie que si $X=\begin{pmatrix}
        1 & -a_{2}&\dots&-a_{n}
    \end{pmatrix}^{\mathsf{T}}\neq0$, on a $AX=0$ et donc $A\notin GL_{n}(\C)$.
\end{remark}

\begin{proof}
    \phantom{}
    \begin{enumerate}
        \item Soit $X=\begin{pmatrix}
            x_{1}&\dots &x_{n}
        \end{pmatrix}\in \ker(A)\cap H$, on a $\sum_{i=1}^{n}x_{i}=0$ et $AX=0$. Notons que l'on a 
        \begin{equation}
            A+A^{\mathsf{T}}=
            \begin{pmatrix}
                0&1 &\dots & \dots& 1\\
                1 & \ddots & \ddots & & \vdots\\
                \vdots & \ddots & \ddots & \ddots & \vdots\\
                \vdots & & \ddots & \ddots & 1\\
                1 &\dots & \dots & 1 & 0
            \end{pmatrix}=N
        \end{equation}
        On a 
        \begin{equation}
            NX=-X
        \end{equation}
        et $N^{\mathsf{T}}=N$.

        On a alors 
        \begin{equation}
            X^{\mathsf{T}}AX+X^{\mathsf{T}}A^{\mathsf{T}}X=X^{\mathsf{T}}NX=-X^{\mathsf{T}}X=-\sum_{i=1}^{n}x_{i}^{2}
        \end{equation}
        Comme $AX=0$, on a aussi $X^{\mathsf{T}}AX=0$ et $X^{\mathsf{T}}A^{\mathsf{T}}X=(AX)^{\mathsf{T}}X=0$ donc on a $\sum_{i=1}^{n}x_{i}^{2}=0$ d'où $x_{i}=0$ et $X=0$. Donc 
        \begin{equation}
            \boxed{\ker(u)\cap H}=\lbrace 0\rbrace
        \end{equation}

        Donc $\dim(\ker(u))\in\lbrace0,1\rbrace$ et le théorème du rang assure alors que $\rg(A)\in\lbrace n-1,n\rbrace$.

        \item Comme $A+A^{\mathsf{T}}=N$, on a $A=\frac{1}{2}N+S$ avec $S\in\mathcal{A}_{n}(\R)$. Or, pour $S=0$, on a 
        \begin{equation}
            N=\begin{pmatrix}
                1&\dots&1\\
                \vdots & &1\\
                1 & \dots &1
            \end{pmatrix}
            -I_{n}
        \end{equation}
        et $(N+I_{n})^{2}=n(M+I_{n})$ donc $N\in GL_{n}(\R)$. De même, pour 
        \begin{equation}
            S=\frac{1}{2}
            \begin{pmatrix}
                0&-1 &\dots & \dots& -1\\
                1 & \ddots & \ddots & & \vdots\\
                \vdots & \ddots & \ddots & \ddots & \vdots\\
                \vdots & & \ddots & \ddots & -1\\
                1 &\dots & \dots & 1 & 0
            \end{pmatrix}
        \end{equation}
        on a
        \begin{equation}
            A=
            \begin{pmatrix}
                0&0 &\dots & \dots& 0\\
                1 & \ddots & \ddots & & \vdots\\
                \vdots & \ddots & \ddots & \ddots & \vdots\\
                \vdots & & \ddots & \ddots & 0\\
                1 &\dots & \dots & 1 & 0
            \end{pmatrix}
        \end{equation}
        et $\rg(A)=n-1$.

        Donc on peut avoir les deux possibilités.
    \end{enumerate}
\end{proof}

\begin{proof}
    Soit $u\in\L(\C^{n})$ de range 1 telle que $\Tr(u)=\lambda$. On a $\dim(\ker(u))=n-1$

    En prenant une base de $\ker(u)$ $(e_{1},\dots,e_{n-1})$ que l'on complète en $\mathcal{B}=(e_{1},\dots,e_{n-1},e_{n})$ une base de $\C^{n}$, on a 
    \begin{equation}
        \mat_{\mathcal{B}}(u)
        =
        \begin{pmatrix}
            0 & \dots & 0 & \alpha_{1}\\
            \vdots & & \vdots &\vdots\\
            \vdots & & \vdots &\alpha_{n-1}\\
            0 & \dots & 0& \lambda
        \end{pmatrix}
    \end{equation}

    Si $\lambda\neq0$, posons $f_{n}=\beta_{1}e_{1}+\dots+\beta_{n-1}e_{n}+e_{n}$, on a 
    \begin{equation}
        u(f_{n})=\lambda f_{n}
    \end{equation}
    si et seulement si
    \begin{equation}
        \alpha_{1}e_{1}+\dots+\alpha_{n-1}e_{n-1}+\lambda e_{n}=\lambda f_{n}
    \end{equation}
    On pose $\beta_{1}=\frac{\alpha}{\lambda},\dots,\beta_{n-1}=\frac{\alpha_{n-1}}{\lambda}$ et si $\mathcal{B}'=(e_{1},\dots,e_{n-1},f_{n})$, on a 
    \begin{equation}
        \mat_{\mathcal{B}'}(u)=
        \begin{pmatrix}
            0 & \dots & 0 & 0\\
            \vdots & & \vdots &\vdots\\
            \vdots & & \vdots &0\\
            0 & \dots & 0& \lambda
        \end{pmatrix}
    \end{equation}

    Si $\lambda=0$, il existe $i_{0}\in\llbracket 1,n-1\rrbracket$ tel que $\alpha_{i_{0}}\neq0$ (sinon $\rg(u)=0$). On pose $f_{n}=e_{n}$ et $f_{1}=\alpha_{1}e_{1}+\dots+\alpha_{n-1}e_{n-1}\in\ker(u)\setminus\lbrace0\rbrace$ et on complète $(f_{1},\dots,f_{n-1})$ en une base de $\ker(u)$. On pose $\mathcal{B}'=(f_{1},\dots,f_{n})$ base de $\C^{n}$ et on a alors 
    \begin{equation}
        \mat_{\mathcal{B}'}(u)=
        \begin{pmatrix}
            0 & \dots & 0 & 1\\
            \vdots & & 0 &0\\
            \vdots & & \vdots &\vdots\\
            0 & \dots & 0& 0
        \end{pmatrix}
    \end{equation}

    Ainsi, dans les deux cas, deux matrices sont de rang 1 et de même trace si et seulement si elles sont semblables.
\end{proof}

\begin{proof}
    \phantom{}
    \begin{enumerate}
        \item Soit $M\in F$, telle que $\rg(M)=r$. $M$ est équivalente à $J_{r}$, donc il existe $(P_{0},Q_{0})\in GL_{n}(\R)^{2}$ telle que $P_{0}^{-1}J_{r}Q_{0}=M\in F$.
        
        \item Soit \function{\varphi}{F}{F_0}{M}{P_0 MQ_{0}^{-1}}
        est linéaire surjective par définition de $F_{0}$ de réciproque $\varphi^{-1}\colon M_{0}\to P_{0}^{-1}M_{0}Q_{0}$ donc $F$ et $F_{0}$ sont isomorphes.

        Pour tout $M\in F$, $\rg(M)=\rg(\varphi(M))$: $\varphi$ étant bijective, on a $r=\max\left\lbrace\rg(M_{0})\middle| M_{0}\in F_{0}\right\rbrace$

        \item Il suffit de choisir les coefficients de $B$ et $C$ donc 
        \begin{equation}
            \boxed{\dim(G_{0})=n(n-r)}
        \end{equation}

        \item On écrit 
        \begin{equation}
            \begin{pmatrix}
                \lambda I_{r} & B^{\mathsf{T}}\\
                B & C
            \end{pmatrix}
            =\lambda J_{r}+M_{0}\in F_{0}
        \end{equation}

        Si on avait 
        \begin{equation}
            \det\left(
				\begin{array}{@{}c|c@{}}
				\lambda
				I_{r} &
				\begin{matrix}
				b_{j,1}\\
						\vdots\\
						b_{j,r}
						\end{matrix}
						\\
					\hline
					\begin{matrix}
						b_{i,1} &
						\dots
						& b_{i,r}
						\end{matrix}
						& c_{i,j}
				\end{array}
			\right)\neq0
        \end{equation}
        (déterminant d'une sous-matrice de taille $r+1$ de la matrice précédente), on aurait 
        \begin{equation}
            \rg\begin{pmatrix}
                \lambda I_{r} & B^{\mathsf{T}}\\
                B & C
            \end{pmatrix}\geqslant r+1>r
        \end{equation}
        ce qui est exclu d'après 2.

        \item En effectuant $L_{r+1}\leftarrow L_{r+1}-\frac{b_{i,1}}{\lambda}L_{1}-\dots-\frac{b_{i,r}}{\lambda}L_{r}$, en notant $f(\lambda)=c_{i,j}-\sum_{k=1}^{r}\frac{b_{i,k}}{\lambda}b_{j,k}$, on obtient 
        \begin{equation}
            \det\left(
				\begin{array}{@{}c|c@{}}
				\lambda
				I_{r} &
				\begin{matrix}
				b_{j,1}\\
						\vdots\\
						b_{j,r}
						\end{matrix}
						\\
					\hline
					\begin{matrix}
						0 &
						\dots
						& 0
						\end{matrix}
						& f(\lambda)
				\end{array}
			\right)=0
        \end{equation}

        D'où $f(\lambda)=0$ et comme $\lambda\neq0$, on a
        \begin{equation}
            \lambda f(\lambda)=0=\lambda c_{i,j}-\sum_{k=1}^{r}b_{i,k}b_{j,k}
        \end{equation}
        qui est nulle sur $\R^{*}$ donc $c_{i,j}=0$ et $\sum_{k=1}^{r}b_{i,k}b_{j,k}=0$. Ceci implique $C=0$ et pour $i=j$, on a $\sum_{k=1}^{r}b_{j,k}^{2}=0$ donc $B=0$.

        \item On a donc $G_{0}\cap F_{0}=\lbrace0\rbrace$ ($\dim(G_{0})=n(n-r)$). $G_{0}$ et $F_{0}$ sont en somme directe, donc 
        \begin{equation}
            \dim(G_{0}\oplus F_{0})=\dim(G_{0})+\dim(F_{0})\leqslant n^{2}
        \end{equation}
        donc 
        \begin{equation}
            \boxed{\dim(F)=\dim(F_{0})\leqslant n^{2}-n(n-r)=nr}
        \end{equation}

        \item Si $F\cap GL_{n}(\R)=\emptyset$, on a $r\leqslant n-1$ et $\dim(F)\leqslant n(n-1)$. Par contraposée, si $\dim(F)\geqslant n^{2}-n+1$, on a $F\cap GL_{n}(\R)\neq\emptyset$.
        
        \item Soit 
        \begin{equation}
            G_{1}=\left\lbrace\begin{pmatrix}
                0 & B^{\mathsf{T}}\\
                B & C
            \end{pmatrix}\middle| B\in\M_{n-r,r}(\C),C\in\M_{r}(\C)\right\rbrace
        \end{equation}
        sous-$\R$-espace-vectoriel de $\M_{n}(\C)$. Par les mêmes arguments que précédemment, on a $G_{1}\cap F_{0}=\lbrace0\rbrace$ et $\dim_{\R}(G_{1})=2n(n-r)$ et $\dim_{\R}\M_{n}(\C)=2n^{2}$ donc 
        \begin{equation}
            \boxed{\dim_{\R}F_{0}=2\dim_{\C}F_{0}\leqslant 2nr}
        \end{equation}

        Le résultat est donc encore valable.
    \end{enumerate}
\end{proof}

\begin{proof}
    On a $f(I_{n})=f(I_{n})^{2}$ donc $f(I_{n})\in\lbrace0,1\rbrace$. Si $f(I_{n})=0$, alors $f=0$ ce qui est exclu.

    Si $M$ est inversible, on a 
    \begin{equation}
        f(M\times M^{-1})=f(M)\times f(M^{-1})=1
    \end{equation}
    donc $f(M)\neq0$.

    Si $M$ n'est pas inversible, posons $r=\rg(M)\leqslant n-1$. $M$ est équivalente la matrice nilpotente
    \begin{equation}
        M'=
        \begin{pmatrix}
            0 & 1 & 0 &\dots & \dots&\dots & 0\\
            \vdots & \ddots & \ddots & \ddots& & &\vdots\\
            \vdots & &\ddots & 1 & \ddots &&\vdots\\
            \vdots & && \ddots & 0 &\ddots &\vdots\\
            \vdots & & & &\ddots & \ddots & 0\\
            0 &\dots & \dots & \dots & \dots &0&0
        \end{pmatrix}
    \end{equation}

    Donc il existe $(P,Q)\in (GL_{n}(\C))^{2}$ telles que $M=P^{-1}M'Q$. On a
    \begin{equation}
        f(M'^{n})=\left(f(M')\right)^{n}=f(0)
    \end{equation}
    Comme $f(0)=f(0)^{2}$, on a aussi $f(0)\in\lbrace0,1\rbrace$. Si $f(0)=1$, pour tout $A\in\M_{n}(\C)$, on a $f(A\times 0)=f(A)\times f(0)=1$ ce qui est impossible car $f$ n'est pas constante. Donc $f(0)=0$. Ainsi, $f(M')=0$ et donc $f(M)=0$.
\end{proof}

\begin{remark}
    $f$ induit donc un morphisme de $\left(GL_{n}(\C),\times\right)\to(\C^{*},\times)$.
\end{remark}

\begin{remark}
    On peut montrer que pour $n\geqslant 2$, pour tout $i\neq j\in\lbrace1,n\rbrace^{2}$, $\forall \lambda \in \C$, il existe $(A,B)\in GL_{n}(\C)^{2}$,
    \begin{equation}
        T_{i,j}(\lambda)=ABA^{-1}B^{-1}
    \end{equation}
    en écrivant 
    \begin{align}
        T_{i,k}(\alpha)T_{k,j}(\beta)T_{i,k}(-\alpha)T_{k,j}(-\beta)
        =&\left(I_{n}+\alpha E_{i,k}+\beta E_{k,j}+\alpha \beta E_{i,j}\right)\notag\\
        ~~~~~~&\times \left(I_{n}-\alpha E_{i,k}-\beta E_{k,j}+\alpha\beta E_{i,j}\right)\\
        =&I_{n}+\alpha\beta E_{i,j}
    \end{align}

    Il vient 
    \begin{equation}
        f(T_{i,j}(\lambda))=f(A)f(B)f(A)^{-1}f(B)^{-1}=1
    \end{equation}
    Si $M\in GL_{n}(\C)$ s'écrit comme produit de transvections $T_{i,j}(\lambda)$ et de dilatations $D_{n}(\det(M))$.

    Il vient $f(M)=f(D_{n}(\det(M)))$. Or 
    \function{\varphi}{(\C^{*},\times)}{(\C^{*},\times)}{\alpha}{f(D_{n}(\alpha))}
    est un morphisme de groupe (car $D_{n}(\alpha\beta)=D_{n}(\alpha)D_{n}(\beta)$).

    Finalement, $f(M)=\varphi(\det(M))$.

    Si de plus $f$ est continue, $\varphi$ aussi et on peut montrer qu'il existe $k\in\R$ tel que pour tout $z\in\C^{*}$, $\varphi(z)=z^{k}$.
\end{remark}

\begin{proof}
    Rappelons que $F_1,\dots,F_k$ $k$ sous-espaces vectoriels de $E$ sont en somme directe si et seulement si 
    \function{\varphi}{F_1\times\dots\times F_k}{E}{(x_1,\dots,x_k)}{\sum_{i=1}^{k}x_i} est injective si et seulement si (théorème du rang) $\dim(F_1\times\dots\times F_k)=\dim(\im(\varphi))$ et $\im\varphi=\sum_{i=1}^{k}F_i$ si et seulement si $\dim(\sum_{i=1}^{k}F_i)=\sum_{i=1}^{k}\dim(F_i)$.

    Dans notre cas, $\im\left(\sum_{i=1}^{k}A_i\right)=\K^{n}\subset\sum_{i=1}^{n}\im(A_i)\subset\K^{n}$ donc $\dim\left(\sum_{i=1}^{k}A_i\right)=n$. De plus, $\sum_{i=1}^{k}\dim(\im(A_i))=\sum_{i=1}^{k}\rg(A_i)=\sum_{i=1}^{k}\Tr(A_i)=\Tr\left(\sum_{i=1}^{k}A_i\right)=n$. Donc $\left(\im(A_i)\right)_{1\leqslant i\leqslant k}$ sont en somme directe et $\oplus_{i=1}^{k}\im(A_i)=\K^{n}$.

    On a $A_iA_j=0$ si et seulement si $\im(A_j)\subset\ker(A_i)$. La matrice $I_n-A_i=\sum_{\substack{j=1\\j\neq i}}^{k}A_j$ représente le projecteur sur $\ker(A_i)$ parallèlement à $\im(A_i)$ donc $\ker(A_i)=\im(I_n-A)=\im\left(\sum_{j\neq i}A_j\right)=\oplus_{j\neq i}\im(A_j)$. Ainsi, pour tout $i\neq j$, $\im(A_j)\subset\ker(A_i)$ et $A_iA_j=0$.
\end{proof}

\end{document}
\documentclass[12pt]{article}
\usepackage{style/style_sol}

\begin{document}

\begin{titlepage}
	\centering
	\vspace*{\fill}
	\Huge \textit{\textbf{Solutions MP/MP$^*$\\ Réduction des endomorphismes}}
	\vspace*{\fill}
\end{titlepage}

\begin{proof}
	\phantom{}
	\begin{enumerate}
		\item On a \function{f^k}{E}{E}{M}{A^{k}M} donc pour tout polynôme $P$, on a $P(f)=P(A)M$ par combinaison linéaire. Si $P(A)=0$, alors $P(f)=0$. Donc si $A$ est diagonalisable, $f$ l'est aussi. Si $P(f)=0$ alors avec $M=I_{n}$, on a $P(A)=0$ et $A$ est diagonalisable si $f$ l'est.
		
		Même résultat avec $g$ et $B$.

		\item Soit $(\lambda_{i,j})_{1\leqslant i,j\leqslant n}$ tel que $\sum_{(i,j)\in\llbracket1,n\rrbracket^{2}}\lambda_{i,j}X_{i}Y_{j}^{\mathsf{T}}=0$. Alors on a 
		\begin{equation}
			\sum_{j=1}^{n}\left(\sum_{i=1}^{n}\lambda_{i,j}X_{i}\right)Y_{j}^{\mathsf{T}}=0
		\end{equation}

		Soit $k\in\llbracket1,n\rrbracket$, la $k$-ième ligne de notre matrice est 
		\begin{equation}
			\sum_{j=1}^{n}\left(\sum_{i=1}^{n}\lambda_{i,j}X_{i,k}\right)Y_{j}^{\mathsf{T}}=0
		\end{equation}
		Puisque $(Y_{j}^{\mathsf{T}})_{1\leqslant j\leqslant n}$ est libre, on a pour tout $j\in\llbracket1,n\rrbracket$,
		\begin{equation}
			\sum_{i=1}^{n}\lambda_{i,j}X_{i,k}=0
		\end{equation}
		Puisque $(X_{i})_{1\leqslant i\leqslant n}$ est libre, pour tout $(i,j)\in\llbracket1,n\rrbracket^{2}$, $\lambda_{i,j}=0$, d'où le résultat.

		\item Puisque $B$ est diagonalisable, $B^{\mathsf{T}}$ l'est aussi. On prend $(X_{i})_{1\leqslant i\leqslant n}$ une base de vecteurs propres de $A$ avec pour tout $i\in\llbracket1,n\rrbracket$, $AX_{i}=\lambda_{i}X_{i}$. Prenons $(Y_{j})_{1\leqslant j\leqslant n}$ une base de vecteurs propres de $B^{\mathsf{T}}$ avec pour tout $j\in\llbracket1,n\rrbracket$, $B^{\mathsf{T}}Y_{j}=\mu_{j}Y_{j}$ et $Y_{j}B^{\mathsf{T}}=\mu_{j}Y_{j}^{\mathsf{T}}$. Ainsi,
		\begin{equation}
			h\left(X_{i}Y_{j}^{\mathsf{T}}\right)=AX_{i}Y_{j}^{\mathsf{T}}B=\mu_{j}AX_{i}Y_{j}^{\mathsf{T}}=\mu_{j}\lambda_{i}X_{i}Y_{j}^{\mathsf{T}}
		\end{equation}
		et les $(X_{i}Y_{j}^{\mathsf{T}})_{1\leqslant i,j\leqslant n}$ forment une base de $E$ d'après ce qui précède. Donc $h$ est diagonalisable.

		Réciproquement, on a le contre-exemple $A=0$ et $B$ non diagonalisable: $h$ est l'endomorphisme nul.
	\end{enumerate}	
\end{proof}

\begin{remark}
	Généralement, soit $A\in\M_{n}(\K)$ et $B\in\M_{p}(\K)$, on définit \function{h_{A,B}}{\M_{n,p}(\K)}{\M_{n,p}(\K)}{M}{AMB}
	La matrice de $h_{A,B}$ dans la base canonique de $\M_{n,p}(\K)$ s'appelle le produit tensoriel de $A$ et $B$ noté 
	\begin{equation}
		A\otimes B=
		\begin{pmatrix}
			a_{1,1}B & \dots & a_{1,n}B\\
			\vdots & &\vdots\\
			a_{n,1}B&\dots & a_{n,n}B
		\end{pmatrix}
	\end{equation}
	On a toujours 
	\begin{equation}
		\Tr(A\otimes B)=\sum_{i=1}^{n}a_{i,i}\Tr(B)=\Tr(A)\Tr(B)
	\end{equation}
	Si $A$ et $B$ sont diagonalisables, $h_{A,B}$ l'est.
\end{remark}

\begin{proof}
	On pose $P=DP_{1}$ et $Q=DQ_{1}$ avec $P_{1}\wedge Q_{1}=1$. Il existe $(U,V)\in\K[X]^{2}$ telles que $UP_{1}+VQ_{1}=1$. On a $MD=PQ$ donc $M=DP_{1}Q_{1}=PQ_{1}=P_{1}Q$.

	\begin{enumerate}
		\item Soit $x\in\ker(D(f))$. On a 
		\begin{equation}
			P(f)(x)=DP_{1}(f)(x)=P_{1}(f)\circ D(f)(x)=0	
		\end{equation}
		De même pour $Q(f)(x)=0$, donc 
		\begin{equation}
			\ker(D(f))\subset\ker(P(f))\cap\ker(Q(f))
		\end{equation}

		Soit $x\in\ker(P(f))\cap\ker(Q(f))$. On a
		\begin{equation}
			DUP_{1}+DVQ_{1}=0
		\end{equation}
		d'où 
		\begin{equation}
			UP+VQ=0
		\end{equation}
		et
		\begin{equation}
			D(f)(x)=UP(f)(x)+VQ(f)(x)=0
		\end{equation}

		Donc 
		\begin{equation}
			\boxed{\ker(D(f))=\ker(P(f))\cap\ker(M(f))}	
		\end{equation}

		\item On a $P\mid M$ donc $\ker(P(f))\subset\ker(M(f))$. De même, $\ker(Q(f))\subset\ker(M(f))$ donc 
		\begin{equation}
			\ker(P(f))+\ker(Q(f))\subset\ker(M(f))
		\end{equation}

		Si $x\in\ker(M(f))$, on a 
		\begin{equation}
			x=\underbrace{UP_{1}(f)(x)}_{\in\ker(Q(f))}+\underbrace{VQ_{1}(f)(x)}_{\in\ker(P(f))}
		\end{equation}
		car $M=P_{1}Q=Q_{1}P$. Donc 
		\begin{equation}
			\boxed{\ker(M(f))=\ker(P(f))+\ker(Q(f))}
		\end{equation}

		\item Si $i\in\im(P(f))$, il existe $x\in E$ tel que $y=P(f)(x)=D(f)\circ P_{1}(f)(x)\in\im(D(f))$. De même pour $\im(Q(f))\subset\im(D(f))$. Donc 
		\begin{equation}
			\im(P(f))+\im(Q(f))\subset\im(D(f))
		\end{equation}

		Soit $y\in\im(D(f))$, alors il existe $x\in E$ tel que 
		\begin{equation}
			y=D(f)(x)=\underbrace{UP(f)(x)}_{\in\im(P(f))}+\underbrace{VQ(f)(x)}_{\in\im(Q(f))}
		\end{equation}
		Donc 
		\begin{equation}
			\boxed{\im(D(f))=\im(P(f))+\im(Q(f))}
		\end{equation}

		\item On a $P\mid M$ d'où $\im(M(f))\subset\im P(f)$ et $\im(M(f))\subset\im Q(f)$. Ainsi,
		\begin{equation}
			\im(M(f))\subset\im(Q(f))\cap\im\im(Q(f))
		\end{equation}

		Si $y\in\im(P(f))\cap\im(Q(f))$ alors il existe $(x,x')\in E^{2}$ tels que 
		\begin{equation}
			y=P(f)(x)=P(f)(x')
		\end{equation}
		Or $M=P_{1}Q=PQ_{1}$ donc 
		\begin{equation}
			y=UP_{1}(f)(y)+VQ_{1}(f)(y)=UP_{1}Q(f)(x')+VQ_{1}P(f)(x)\in\im(M(f))
		\end{equation}
		donc 
		\begin{equation}
			\boxed{\im(M(f))=\im(P(f))\cap\im(Q(f))}
		\end{equation}
	\end{enumerate}
\end{proof}

\begin{proof}
	On a 
	\begin{equation}
		A\left(\frac{-1}{5}A+\frac{4}{5}I_{n}\right)=I_{n}
	\end{equation}
	donc $A$ est inversible.
	\begin{equation}
		X^{2}-4X+5=(X-2+\i)(X-2-\i)
	\end{equation}
	est scindé à racines simples sur $\C$. Donc $A$ est diagonalisable sur $\C$, semblable sur $\C$ à
	\begin{equation}
		\begin{pmatrix}
			\lambda_{1}I_{n_{1}} &0\\
			0 & \lambda_{2}I_{n_{2}}
		\end{pmatrix}
	\end{equation}
	où $\lambda_{1}=2+\i$ et $\lambda_{2}=2-\i$. $A\in\M_{n}(\R)$ donc $\Tr(A)=n_{1}\lambda_{1}+n_{2}\lambda_{2}\in\R$

	Donc 
	\begin{equation}
		\Im(n_{1}\lambda_{1}+n_{2}\lambda_{2})=0=n_{1}-n_{2}
	\end{equation}

	Ainsi $n_{1}=n_{2}$ donc $n$ est pair.

	$A$ est semblable sur $\C$ à 
	\begin{equation}
		\begin{pmatrix}
			\lambda_{1}&0&\dots&\dots&0\\
			0&\overline{\lambda_{1}}&\ddots&&\vdots\\
			\vdots &\ddots & \ddots &\ddots&\vdots\\
			\vdots & &\ddots & \lambda_{1}&0\\
			0 &\dots&\dots&0&\overline{\lambda_{1}}
		\end{pmatrix}
	\end{equation}

	Soit 
	\begin{equation}
		A_{0}=
		\begin{pmatrix}
			0&-5\\
			1&4
		\end{pmatrix}
	\end{equation}
	On a $\chi_{A_{0}}=X^{2}-4X+5$. $A_{0}$ est diagonalisable sur $\C$ et est semblable à 
	\begin{equation}
		\begin{pmatrix}
			\lambda_{1}&0\\
			0&\overline{\lambda_{1}}
		\end{pmatrix}
	\end{equation}
	Donc $A$ est semblable sur $\C$ à 
	\begin{equation}
		\begin{pmatrix}
			A_{0}&&\\
			&\ddots&\\
			&&A_{0}
		\end{pmatrix}
	\end{equation}
	donc $A$ est semblable sur $\R$ à cette même matrice.

	Soit $l\in\N$, on a 
	\begin{equation}
		X^{l}=Q_{p}(X^{2}-4X+5)+\alpha_{l}X+\beta_{l}
	\end{equation}
	par division euclidienne. Donc 
	\begin{equation}
		A^{l}=\alpha_{l}A+\beta_{l}I_{n}
	\end{equation}
	On a notamment 
	\begin{equation}
		\left\lbrace
			\begin{array}[]{l}
				(2+\i)^{l}=\alpha_{l}(2+\i)+\beta_{l}\\
				(2-\i)^{l}=\alpha_{l}(2-\i)+\beta_{l}
			\end{array}
		\right.
	\end{equation}

	On a donc 
	\begin{equation}
		\boxed{
			\left\lbrace
			\begin{array}[]{l}
				\alpha_{l}=\frac{(2+\i)^{l}-(2-\i)^{l}}{2\i}\\
				\beta_{l}=(2+\i)^{l}-\frac{(2+\i)}{2\i}\left[(2+\i)^{l}-(2-\i)^{l}\right]
			\end{array}
		\right.
		}
	\end{equation}
\end{proof}

\begin{remark}
	On a $2+\i=\sqrt{5}\e^{\i\theta}$ avec $\theta=\arccos\left(\frac{2}{\sqrt{5}}\right)\in]0,\pi[$. Donc $\alpha_{l}=\left(\sqrt{5}\right)^{l}\sin(l\theta)$.
\end{remark}

\begin{remark}
	On a 
	\begin{equation}
		I_{n}-4A^{-1}+5A^{-2}=0
	\end{equation}
	De même, $\left(X-\frac{1}{2-\i}\right)\left(X-\frac{1}{2+\i}\right)$ annule $A^{-1}$ et on a pour tout $l\in-\N^{*}$,
	\begin{equation}
		A^{l}=\alpha_{l}A+\beta_{l}I_{n}
	\end{equation}
\end{remark}

\begin{remark}
	$(A-2I_{n})^{2}=-I_{n}$ donc $\det(-I_{n})=(-1)^{n}>0$ donc $n$ est pair.
\end{remark}

\begin{proof}
	\phantom{}
	\begin{enumerate}
		\item On a 
		\begin{equation}
			A\begin{pmatrix}
				1\\\vdots\\1
			\end{pmatrix}=\begin{pmatrix}
				1\\\vdots\\1
			\end{pmatrix}
		\end{equation}
		et $\begin{pmatrix}
			1&\dots&1
		\end{pmatrix}^{\mathsf{T}}\neq0$ donc 
		\begin{equation}
			\boxed{1\in\Sp_{\R}(A)}
		\end{equation}

		\item Soit $X=\begin{pmatrix}
			x_{1}&\dots&x_{n}
		\end{pmatrix}^{\mathsf{T}}\neq0$ associé à $\lambda$. Pour tout $i\in\llbracket1,n\rrbracket$, on a 
		\begin{equation}
			\lambda x_{i}=\sum_{j=1}^{n}a_{i,j}x_{j}
		\end{equation}
		Soit $i_{0}\in\llbracket1,n\rrbracket$ tel que $\left\lvert x_{i_{0}}\right\rvert=\max\limits_{i\in\llbracket1,n\rrbracket}\left\lvert x_{i}\right\rvert>0$ car $X\neq0$. On a alors 
		\begin{equation}
			\left\lvert\lambda\right\rvert\left\lvert x_{i_{0}}\right\rvert=\left\lvert\sum_{j=1}^{n}a_{i_{0},j}x_{j}\right\rvert\leqslant\sum_{j=1}^{n}a_{i_{0},j}\left\lvert x_{j}\right\rvert\leqslant\left(\sum_{j=1}^{n}a_{i_{0},j}\right)\left\lvert x_{i_{0}}\right\rvert
		\end{equation}
		donc 
		\begin{equation}
			\boxed{\left\lvert\lambda\right\rvert\leqslant1}
		\end{equation}

		\item Soit $J_{i}=\left\lbrace j\in\llbracket1,n\rrbracket\middle| a_{i,j}>0\right\rbrace$. On a 
		\begin{equation}
			\left\lvert\lambda\right\rvert\left\lvert x_{i_{0}}\right\rvert=\left\lvert\sum_{j\in J_{i_{0}}}^{n}a_{i_{0},j}x_{j}\right\rvert\leqslant\sum_{j\in J_{i_{0}}}^{n}a_{i_{0},j}\left\lvert x_{j}\right\rvert\leqslant\left(\sum_{j\in J_{i_{0}}}^{n}a_{i_{0},j}\right)\left\lvert x_{i_{0}}\right\rvert=\left\lvert x_{i_{0}}\right\rvert
		\end{equation}

		On a égalité partout donc pour tout $j\in J_{i_{0}}$, $\left\lvert x_{j}\right\rvert=\left\lvert x_{i_{0}}\right\rvert$ et $x_{j}=\left\lvert x_{i_{0}}\right\rvert\e^{\i \theta}$. En reportant, on a 
		\begin{equation}
			\lambda\left\lvert x_{i_{0}}\right\rvert=\sum_{j\in J_{i_{0}}}a_{i_{0},j}\left\lvert x_{i_{0}}\right\rvert
		\end{equation}
		donc 
		\begin{equation}
			\boxed{\lambda=1}
		\end{equation}

		\item Si $\left\lvert\lambda\right\rvert=1$ et $\lambda\neq1$, on a $i_{0}\notin J_{i_{0}}$ car sinon $\lambda=1$. Donc il existe $i_{1}\in J_{i_{0}}\setminus\lbrace i_{0}\rbrace$ tel que $x_{i_{1}}=\left\lvert x_{i_{0}}\right\rvert\e^{\i\theta}=\lambda x_{i_{0}}$. Ainsi, il existe $i_{2}\neq i_{1}$ tel que $x_{i_{2}}=\lambda x_{i_{1}}$. De proche en proche, il existe $i_{q}\neq i_{q-1}$ tel que $x_{i_{q}}=\lambda x_{i_{q-1}}$ (avec $q\geqslant1$) et $x_{i_{q}}=\lambda^{q}x_{i_{0}}$. Or \function{\varphi}{\N}{\llbracket1,n\rrbracket}{k}{i_{k}} n'est pas injective. Donc il existe $k>l$ tel que $i_{k}=i_{l}$ et $x_{i_{k}}=\lambda^{k-k}x_{i_{k}}$ et $k-l>1$ donc 
		\begin{equation}
			\boxed{
				\lambda\in\U_{k-l}
			}
		\end{equation}

		\item L'identité convient, les matrices de permutation aussi. En effet, si $\sigma\in\Sigma_{n}$, on a $P_{\sigma}^{n!}=I_{n}$ donc les valeurs propres sont racines de $X^{n!}-1$ donc $\Sp_{\C}(P_{\sigma})\subset\U_{n!}$.
		
		Réciproquement, soit $A$ stochastique telle que $\Sp_{\C}(A)\subset(\U)$. Soit $i\in\llbracket1,n\rrbracket$, supposons $\left\lvert J_{i_{0}}\right\rvert\geqslant2$. D'après la décomposition de Dunford, il existe $D$ diagonale et $N$ nilpotente qui commutent telles que $A=D+N$ et $\Sp_{\C}(D)=\Sp_{\C}(A)$. Si $N$ est nilpotente d'indice $r\geqslant2$, on a pour tout $k\in\N^{*}$ avec $k\geqslant r$, on a 
		\begin{equation}
			A^{k}=\sum_{j=1}^{k}\binom{k}{j}N^{j}D^{k-j}=\sum_{j=1}^{r}\binom{k}{j}N^{j}D^{k-j}
		\end{equation}

		Pour tout $j\in\llbracket1,r\rrbracket$, on a 
		\begin{equation}
			\binom{k}{j}=\frac{k(k-1)\dots(k-j+1)}{j!}\underset{k\to+\infty}{\sim}\frac{k^{j}}{j!}
		\end{equation}
		Comme $N^{r-1}\neq0$, on a 
		\begin{equation}
			A^{k}\underset{k\to+\infty}{sim}\frac{k^{r-1}}{(r-1)!}N^{r-1}D^{k-r+1}
		\end{equation}
		et les coefficients de $D^{k-r+1}$ sont bornés car $\Sp(D)\subset\U$.

		Or, notons que si $A$ et $B$ sont stochastiques, $AB$ l'est aussi ($\bm{1}$ est toujours valeur propre). Par récurrence, $A^{k}$ l'est. Donc $A^{k}\in\M_{n}([0,1])$, et l'équivalent est impossible si $r\geqslant2$. Donc $r=1$ donc $N=0$ et $A=D$ est diagonalisable.

		Les valeurs propres de $A$ sont des racines de l'unité, soit $m$ le $\ppcm$ des ordres de ces racines (dans $(\U,\times)$). On a alors 
		\begin{equation}
			A=P\diag(\lambda_{1},\dots,\lambda_{n})P^{-1}
		\end{equation}
		d'où 
		\begin{equation}
			A^{m}=P\diag(\lambda_{1}^{m},\dots,\lambda_{n}^{m})P^{-1}
		\end{equation}

		Notons $M=\max\limits_{j\in J_{i_{0}}}\left\lvert a_{i_{0},j}\right\rvert<1$ (car $\left\lvert J_{i_{0}}\right\rvert\geqslant2$ donc pour tout $j\in J_{i_{0}}$, $a_{i_{0},j}\neq1$). On note $a_{i_{0},i_{0}}^{(m)}$ le coefficient $(i_{0},i_{0})$ de $A^{m}$. On a alors 
		\begin{equation}
			a_{i_{0},i_{0}}^{(m)}=1=\sum_{j\in J_{i_{0}}}a_{i_{0},j}a_{j,i_{0}}^{(m-1)}\leqslant M\sum_{j\in J_{i_{0}}}a_{j,i_{0}}^{(m-1)}\leqslant M\sum_{j=1}^{n}a_{j,i_{0}}^{(m-1)}=M
		\end{equation}
		car $A^{m-1}$ est stochastique. Donc $M=1$ ce qui n'est pas possible (par définition de $M$). Ainsi, pour tout $i\in\llbracket1,n\rrbracket$, on a $\lvert J_{i}\rvert=1$ donc il existe un unique $j_{i}\in\llbracket1,n\rrbracket$ avec $a_{i,j_{i}}=1$ et pour tout $j\neq j_{i}$, $a_{i,j}=0$.

		$i\mapsto j_{i}$ est injective, sinon $\rg(A)\leqslant n-1$ et $0\in\Sp(A)$.
	\end{enumerate}
\end{proof}

\begin{remark}
	On peut avoir $\left\lvert \lambda\right\rvert<1$ pour la question 2, par exemple 
	\begin{equation}
		A=
		\begin{pmatrix}
			\frac{1}{n}&\dots&\frac{1}{n}\\
			\vdots&&\vdots\\
			\frac{1}{n}&\dots&\frac{1}{n}
		\end{pmatrix}
	\end{equation}
	On a $A^{2}=A$ et $\rg(A)=1$, $\Sp(A)=\lbrace0,1\rbrace$.
\end{remark}

\begin{remark}
	Par exemple, pour 4, on a 
	\begin{equation}
		A=
		\begin{pmatrix}
			0&1\\
			1&0
		\end{pmatrix}
	\end{equation}
	On a $\chi_{A}=X^{2}-1$ et $\Sp(A)=\lbrace-1,1\rbrace$.
\end{remark}

\begin{remark}
	Si pour tout $(i,j)\in\llbracket1,n\rrbracket$, $a_{i,j}>0$ (i.e.~pour tout $i\in\llbracket1,n\rrbracket$, $J_{i}=\llbracket1,n\rrbracket$). D'après 3, on a $\Sp_{\C}(A)\cap\U=\lbrace1\rbrace$. De plus, si $X=\begin{pmatrix}
		x_{1}&\dots&x_{n}
	\end{pmatrix}^{\mathsf{T}}\in\M_{n,1}(\C)\setminus\lbrace0\rbrace$ vérifie $AX=X$, d'après ce qui précède, on a $x_{1}=\dots=x_{n}$ et le sous-espace propre associé à 1 est de dimension 1.
\end{remark}

\begin{proof}
	\phantom{}
	\begin{enumerate}
		\item Soit $(\lambda,\mu)\in\Sp_{\C}(A)\times\Sp_{\C}(B)$. On a $\mu\in\Sp_{\C}(B^{\mathsf{T}})$. Soit $(X,Y)\in\M_{n-1}(\C)\setminus\lbrace0\rbrace$ vecteurs propres associés respectivement à $\lambda$ et à $\mu$. On pose $M=XY^{\mathsf{T}}$. Alors
		\begin{equation}
			\Phi_{A,B}(M)=AXY^{\mathsf{T}}-XY^{\mathsf{T}}B=(\lambda-\mu)XY^{\mathsf{T}}=(\lambda-\mu)M
		\end{equation}
		donc 
		\begin{equation}
			\boxed{\lambda-\mu\in\Sp(\Phi_{A,B})}
		\end{equation}

		Réciproquement, soit $\alpha\in\Sp(\Phi_{A,B})$. Il existe $M\in\M_{n}(\C)\setminus\lbrace0\rbrace$ tel que l'on ait $AM-MB=\alpha M$ d'où $AM=M(\alpha I_{n}+B)$. Par récurrence, $A^{k}M=M(\alpha I_{n}+B)^{k}$ et par combinaison linéaire, pour tout $P\in\C[X]$ on a $P(A)M=MP(\alpha I_{n}+B)$. En particulier, on prend $P=\chi_{A}$. D'après le théorème de Cayley-Hamilton, on a 
		\begin{equation}
			0=M\chi_{A}(\alpha I_{n}+B)
		\end{equation}
		On a $M\neq0$ donc $\chi_{A}(\alpha I_{n}+B)$ n'est pas inversible. On écrit 
		\begin{equation}
			\chi_{A}(X)=\prod_{k=1}^{n}(X-\lambda_{k})
		\end{equation}
		d'où 
		\begin{equation}
			\chi_{A}(\alpha I_{n}+B)=\prod_{k=1}^{n}(B+(\alpha-\lambda_{k})I_{n})
		\end{equation}
		donc il existe $k_{0}\in\llbracket1,n\rrbracket$ tel que $B+(\alpha-\lambda_{k_{0}})I_{n}$ est non inversible. Donc $\lambda_{k_{0}}-\alpha\in\Sp(B)$ et donc $\alpha$ est une différence d'un élément de $\Sp(A)$ et de $\Sp(B)$.

		\item On forme \function{f_A}{\M_n(\C)}{\M_n(\C)}{M}{AM} et \function{g_B}{\M_n(\C)}{\M_n(\C)}{M}{MB}
		Toujours par récurrence et combinaison linéaires, pour tout $P\in\C[X]$,
		\begin{equation}
			P(f_{A})M=P(A)M
		\end{equation}
		Si $P(A)=0$, on a $P(f_{A})=0$. Si $P(f_{A})=0$, pour $M=I_{n}$, on a $P(A)=0$. De même pour $B$. Donc $\Pi_{A}=\Pi_{f_{A}}$ (polynômes minimaux) et $A$ est diagonalisable si et seulement si $f_{A}(M)$ est diagonalisable. $f_{A}$ et $g_{B}$ commutent car 
		\begin{equation}
			(f_{A}\circ g_{B})(M)=AMB=(g_{B}\circ f_{A})(M)
		\end{equation}
		Donc $f_{A}$ et $g_{B}$ sont codiagonalisables et donc $\Phi_{A,B}$ l'est.
	\end{enumerate}
\end{proof}

\begin{remark}
	Si $(X_{1},\dots,X_{n})$ (respectivement $(Y_{1},\dots,Y_{n})$) est une base de vecteurs\\propres de $A$ (respectivement de $B^{\mathsf{T}}$), alors $(X_{i}Y_{j}^{\mathsf{T}})_{1\leqslant i,j\leqslant n}$ est une base de vecteurs propres pour $\Phi_{A,B}$.
\end{remark}

\begin{remark}
	C'est faux sur $\R$, par exemple 
	\begin{equation}
		A=B=
		\begin{pmatrix}
			0 & -1\\
			1 &0	
		\end{pmatrix}
	\end{equation}
	On a $\Sp_{\R}=\emptyset$ et $\Phi_{A,A}(I_{2})=0$ donc $0\in\Sp_{\Phi_{A,A}}$.
\end{remark}

\begin{remark}
	Si $\Phi_{A,B}$ est diagonalisable, soit $(M_{i,j})_{1\leqslant i,j\leqslant n}$ une base de vecteurs propres de $\Phi_{A,B}$. Soit $\lambda\in\Sp_{\C}(B)$ et $X\in\M_{n,1}(\C)\setminus\lbrace0\rbrace$ tel que $BX=\lambda X$. On a 
	\begin{equation}
		AM_{i,j}=M_{i,j}(B+\lambda_{i,j}I_{n})
	\end{equation}
	avec $\Phi_{A,B}(M_{i,j})=\lambda_{i,j}M_{i,j}$. Donc 
	\begin{equation}
		AM_{i,j}X=(\lambda+\lambda_{i,j})M_{i,j}X
	\end{equation}
	Pour tout $X_{0}\in\M_{n,1}(\C)$, il existe $M\in\M_{n}(\C)$ tel que $X_{0}=MX$. $M\in\Vect(M_{i,j})_{1\leqslant i,j\leqslant n}$ donc 
	\begin{equation}
		\Vect(M_{i,j}X)_{1\leqslant i,j\leqslant n}=M_{n,1}(\C)
	\end{equation}
	On peut donc en extraire une base: c'est une base de vecteurs propres de $A$.
\end{remark}

\begin{proof}
	\phantom{}
	\begin{enumerate}
		\item Par récurrence, pour tout $k\in\N$, on a $A^{k}M=\theta^{k}MA^{k}$, or $F$ est un sous-espace vectoriel donc par combinaisons linéaires, pour tout $P\in\K[X]$, on a $P(A)M=MP(\theta A)$.
		
		\item Soit $X\in\ker(A-\lambda I_{n})$. On a $AMX=\theta MAX=\lambda\theta MX$. On a donc $MX\in\ker(A-\lambda\theta I_{n})$.
		
		Si pour tout $\lambda\in\Sp_{\C}(A)$, on a $\theta\lambda\notin\Sp_{\C}(A)$, alors si $\lambda\in\Sp_{\C}(A)$ et $X\in\ker(A-\lambda I_{n})$, alors $\ker(A-\lambda\theta I_{n})=\lbrace0\rbrace$. Donc $MX=0$. Or les vecteurs propres forment une famille génératrice donc $M=0$ et $F=\lbrace0\rbrace$.

		S'il existe $\lambda_{0}\in\Sp_{\C}(A)$ tel que $\theta\lambda_{0}\in\Sp_{\C}(A)$. Soit $X_{1}$ un vecteur propre de $A$ associé à $\lambda_{0}$. On complète $(X_{1})$ en $\mathcal{B}=(X_{1},\dots,X_{n})$ base de $\C^{n}$ formé de vecteurs propres de $A$. On définit $MX_{1}=Y_{1}\in\ker(A-\lambda_{0}\theta I_{n})\setminus\lbrace0\rbrace$ et pour tout $i\in\llbracket2,n\rrbracket$, on a $MX_{i}=0$. Ainsi, pour tout $i\in\llbracket2,n\rrbracket$, on a 
		\begin{equation}
			AMX_{i}=0=\theta MAX_{i}=\theta\lambda_{i}MX_{i}
		\end{equation}
		et 
		\begin{equation}
			AMX_{1}=AY_{1}=\lambda_{0}\theta Y_{1}=\theta MAX_{1}=\theta\lambda_{0}X_{1}
		\end{equation}
		Donc $M\neq0$ et $M\in F$. Finalement, on a $F=\lbrace0\rbrace$ si et seulement si pour tout $\lambda\in\Sp_{\C}(A),\theta\lambda\notin\Sp_{\C}(A)$.

		\item On écrit $\chi_{A}=\prod_{j=1}^{r}(X-\lambda_{j})^{m_{j}}$ avec $\lambda_{j}$ distincts et $m_{j}\geqslant1$. D'après le théorème de Cayley-Hamilton et le lemme des noyaux, on a 
		\begin{equation}
			\C^{n}=\bigotimes_{j=1}^{r}\ker(A-\lambda_{j}I_{n})^{m_{j}}
		\end{equation}

		Supposons $\theta\neq0$. Si $M\in F$ et si $x\in \ker(A-\lambda_{j}I_{n})^{m_{j}}$. On a 
		\begin{equation}
			\left(\left(\frac{X}{\theta}-\lambda_{j}\right)^{m_{j}}\right)(A)(Mx)=M\left(A-\lambda_{j}I_{n}\right)^{m_{j}}(x)=0
		\end{equation}
		Donc 
		\begin{equation}
			Mx\in\ker\left(\frac{1}{\theta}A-\lambda_{j}I_{n}\right)^{m_{j}}=\ker\left(A-\theta\lambda_{j}I_{n}\right)^{m_{j}}
		\end{equation}
		car $\theta\neq0$.

		De plus, $\ker(A-\theta\lambda_{j} I_{n})^{m_{j}}\neq\lbrace0\rbrace0$ si et seulement si $\ker(A-\theta\lambda_{j}I_{n})\neq\lbrace0\rbrace$ car \begin{equation}
			\det\left[(A-\theta\lambda_{j}I_{n})^{m_{j}}\right]=\det\left[(A-\theta\lambda_{j}I_{n})\right]^{m_{j}}
		\end{equation}

		Si pour tout $\lambda\in\Sp_{\C}(A)$, $\lambda\theta\notin\Sp_{\C}(A)$, soit $x\in\ker(A-\lambda_{j}I_{n})^{m_{j}}$. On a 
		\begin{equation}
			Mx\in\ker(A-\theta\lambda_{j}I_{n})^{m_{j}}=\lbrace0\rbrace
		\end{equation}
		donc $M=0$ car $\C^{n}=\bigotimes_{j=1}^{r}\ker(A-\lambda_{j}I_{n})^{m_{j}}$.

		S'il existe $\lambda_{0}\in\Sp_{\C}(A)$ tel que $\lambda_{0}\theta\in\Sp_{\C}(A)$, soit $x_{1}\in\ker(A-\lambda_{0}I_{n})\neq\lbrace0\rbrace$. On pose 
		\begin{equation}
			Mx_{1}=y_{1}\in\ker(A-\lambda_{0}\theta I_{n})\setminus\lbrace0\rbrace
		\end{equation}
		On complète $(x_{1})$ en $\mathcal{B}=(x_{1},\dots,x_{n})$ base de $\C^{n}$ formée de vecteurs appartenant à 
		\begin{equation}
			\bigcup_{j=1}^{r}\ker(A-\lambda_{j}I_{n})^{m_j}	
		\end{equation}
		On a pour tout $i\in\llbracket2,n\rrbracket$, $Mx_{i}=0$. On a $M\neq0$ et 
		\begin{equation}
			AMx_{1}=Ay_{1}=\theta\lambda_{0}y_{1}=\theta\lambda_{0}Mx_{1}
		\end{equation}
		Pour tout $i\in\llbracket2,n\rrbracket$, on a $AMx_{i}=0$ si $x_{i}\in\ker(A-\lambda_{j_{i}}I_{n})^{m_{j_{i}}}$ et si $\lambda_{j_{i}}\neq\lambda_{0}$. On a $Ax_{i}\in\ker(A-\lambda_{j_{i}}I_{n})^{m_{j_{i}}}$ donc 
		\begin{equation}
			Ax_{i}\in\Vect(x_{2},\dots,x_{n})
		\end{equation}
		et $MAx_{i}=0$ donc $AMx_{i}=\theta MAx_{i}$.

		Si $F\neq\lbrace0\rbrace$, il existe $M\neq0$ tel que $AM=\theta MA$. Pour tout $P\in\C[X]$, on a $P(A)M=MP(\theta A)$. En particulier, pour $P=\chi_{A}$, on a
		\begin{equation}
			M\chi_{A}(\theta A)=0
		\end{equation}
		$M\neq0$ et donc $\chi_{A}(\theta A)$ n'est pas inversible. Si $\chi_{A}=\prod_{k=1}^{n}(X-\lambda_{k})$, il existe $k\in\llbracket1,n\rrbracket$, $(\theta A-\lambda_{k}I_{n})$ est non inversible, d'où 
		\begin{equation}
			\boxed{\lambda_{k}\in\Sp_{\C}(A)\cap\Sp_{\C}(\theta A)}
		\end{equation}
	\end{enumerate}
\end{proof}

\begin{proof}
	On a 
	\begin{align}
		\chi_{A}(\lambda)
		&=
		\begin{vmatrix}
			\lambda-1	&-1			&0			&-1\\
			-1			&\lambda-1	&-1			&0\\
			-1			&0			&\lambda-1	&-1\\
			0			&-1			&-1			&\lambda-1
		\end{vmatrix}\\
		&=(\lambda-3)
		\begin{vmatrix}
			1	&1			&1			&1\\
			-1			&\lambda-1	&-1			&0\\
			-1			&0			&\lambda-1	&-1\\
			0			&-1			&-1			&\lambda-1
		\end{vmatrix}\\
		&=(\lambda-3)
		\begin{vmatrix}
			1	&0			&0			&0\\
			-1			&\lambda-1	&-1			&0\\
			-1			&0			&\lambda-1	&-1\\
			0			&-1			&-1			&\lambda-1
		\end{vmatrix}\\
		&=(\lambda-3)
		\begin{vmatrix}
			\lambda &0 &1\\
			1 &\lambda &0\\
			-1 &-1 &\lambda-1
		\end{vmatrix}\\
		&=(\lambda-3)
		\begin{vmatrix}
			\lambda-1 &0 &1\\
			1-\lambda &\lambda &0\\
			1-\lambda &-1 &\lambda-1
		\end{vmatrix}\\
		&=(\lambda-3)(\lambda-1)
		\begin{vmatrix}
			1 &0 &1\\
			-1 &\lambda &0\\
			-1 &-1 &\lambda-1
		\end{vmatrix}\\
		&=(\lambda-3)(\lambda-1)
		\begin{vmatrix}
			1 &0 &1\\
			0 &\lambda &1\\
			0 &-1 &\lambda
		\end{vmatrix}\\
		&=(\lambda-3)(\lambda-1)(\lambda^{2}+1)
	\end{align}
	où l'on a fait successivement les opérations suivantes: $L_{1}\leftarrow L_{1}+L_{2}+L_{3}+L_{4}$, $C_{i}\leftarrow C_{i}-C_{1}$ pour $i\in\lbrace2,3,4\rbrace$, développement selon la première ligne, $C_{1}\leftarrow C_{1}-C_{2}-C_{3}$, $L_{i}\leftarrow L_{i}+L_{1}$ pour $i\in\lbrace2,3\rbrace$, développement selon la première colonne.

	$\chi_{A}$ est scindé à racines simples sur $\C$ donc $A$ est diagonalisable. On trouve ensuite un vecteur propre dans chaque sous-espace propre (qui sont de dimension un).
\end{proof}

\begin{proof}
	\phantom{}
	\begin{enumerate}
		\item On a $\lambda\in\Sp_{\R}(A)$ si et seulement s'il existe $X\in\M_{n,1}(\R)\setminus\lbrace0\rbrace$ telle que $AX=\lambda X$ si et seulement si 
		\begin{equation}
			\left\lbrace
				\begin{array}[]{lll}
					\sum_{i\neq 1}a_{i}x_{i}&=&\lambda x_{1}\\
					\vdots\\
					\sum_{i\neq j}a_{i}x_{i}&=&\lambda x_{1}\\
					\vdots\\
					\sum_{i\neq n}a_{i}x_{i}&=&\lambda x_{1}\\
				\end{array}
			\right.
		\end{equation}

		Soit $S=\sum_{i=1}^{n}a_{i}x_{i}$. Ce système équivaut à 
		\begin{equation}
			S=(\lambda+a_{1})x_{1}=\dots=(\lambda+a_{n})x_{n}
		\end{equation}

		Si $S=0$, pour tout $i\in\llbracket1,n\rrbracket$, on a $\lambda=-a_{i}$ ou $x_{i}=0$ (et $X\neq0$). Les $(a_{i})_{1\leqslant i\leqslant n}$, il existe un unique $i_{0}\in\llbracket1,n\rrbracket$ tel que $\lambda=-a_{i_{0}}$ et pour tout $i\neq i_{0}$, on a $x_{i}=0$. En reportant, on a $S=0=\lambda x_{i_{0}}$ donc $\lambda=0$ ce qui est impossible car $0=\lambda=-a_{i_{0}}>0$.

		Donc $S\neq0$ et pour tout $i\in\llbracket1,n\rrbracket$, $\lambda+a_{i}\neq0$ et pour tout $i\in\llbracket1,n\rrbracket$, $x_{i}=\frac{S}{\lambda+a_{i}}$. On a alors 
		\begin{equation}
			S=\sum_{i=1}^{n}a_{i}x_{i}=\sum_{i=1}^{n}\frac{a_{i}S}{\lambda+a_{i}}
		\end{equation}
		donc 
		\begin{equation}
			\boxed{
				\sum_{i=1}^{n}\frac{a_{i}}{\lambda+a_{i}}=1
			}
		\end{equation}

		Réciproquement, on prend $x_{i}=\frac{1}{\lambda+a_{i}}$ et on a bien $AX=\lambda X$.

		\item On définit \function{f}{\R\setminus\lbrace -a_{n},\dots,-a_{1}\rbrace}{\R}{x}{\sum_{i=1}^{n}\frac{a_{i}}{x+a_{i}}}
		\item Posons $-a_{n+1}=-\infty$ et $-a_{0}=+\infty$. Sur $]-a_{k+1},-a_{k}[$, on a 
		\begin{equation}
			f'(x)=\sum_{i=1}^{n}\frac{-a_{i}}{(x+a_{i})^{2}}	
		\end{equation}

		Les $(a_{i})_{1\leqslant i\leqslant n}$ étant positifs, on a $\lim\limits_{x\to-a_{k+1}^{+}}f(x)=+\infty$ et $\lim\limits_{x\to-a_{k}^{-}}f(x)=-\infty$ (si $k\neq n$) (et $\lim\limits_{x\to-\infty}f(x)=\lim\limits_{x\to+\infty}f(x)=0$).

		D'après le théorème des valeurs intermédiaires, pour tout $k\in\llbracket0,n-1\rrbracket$, il existe un unique $\lambda_{k}\in]-a_{k+1},-a_{k}[$ tel que $f(\lambda_{k})=1$. Donc $A$ admet exactement $n$ valeurs propres réelles distinctes. Donc $A$ est diagonalisable sur $\R$.
	\end{enumerate}
\end{proof}

\begin{remark}
	Soit 
	\begin{equation}
		F(X)=-\sum_{k=1}^{n}\frac{a_{k}}{X+a_{k}}+1=\frac{P(X)}{(X+a_{1}\dots(X+a_{n}))}
	\end{equation}
	avec $P=(X+a_{1})\dots(X+a_{n})-\sum_{k=1}^{n}a_{k}P_{k}$ où $P_{k}=\prod_{\substack{i=1\\i\neq k}}(X+a_{i})$ de degré $n-1$. On a $\deg(P)=n$ et son coefficient dominant est 1. De plus, pour tout $\lambda\in\R$, on a $P(\lambda)=0$ si et seulement si $\sum_{k=1}^{n}\frac{a_{k}}{\lambda+a_{k}}=1$ si et seulement si $\lambda\in\Sp(A)$ donc $P=\chi_{A}$.
\end{remark}

\begin{proof}
	On a 
	\begin{equation}
		\begin{pmatrix}
			1 &0 &\dots&\dots&\dots&\dots&0\\
			0&\frac{1}{2}&\ddots&&&&\vdots\\
			\vdots&\ddots & \ddots &\ddots &&\frac{\lambda}{j}&\vdots\\
			\vdots&& \ddots & \ddots &\ddots&&\vdots\\
			\vdots&&&\ddots&\ddots&\ddots&\vdots\\
			\vdots&&&&\ddots&\ddots&0\\
			0&\dots&\dots&\dots&\dots&0&\frac{1}{n}\\
		\end{pmatrix}
		\times\diag(1,2,\dots,n)
		=
		\begin{pmatrix}
			1 &0 &\dots&\dots&\dots&\dots&0\\
			0&1&\ddots&&&&\vdots\\
			\vdots&\ddots & \ddots &\ddots &&\lambda&\vdots\\
			\vdots&& \ddots & \ddots &\ddots&&\vdots\\
			\vdots&&&\ddots&\ddots&\ddots&\vdots\\
			\vdots&&&&\ddots&\ddots&0\\
			0&\dots&\dots&\dots&\dots&0&1\\
		\end{pmatrix}
	\end{equation}
	où le coefficient est à la $i$-ième ligne et la $j$-ième colonne. La matrice à gauche est diagonalisable car son polynôme caractéristique est scindé à racines simples. Donc les matrices de transvections sont dans $G$. De plus, les matrices de dilatations sont aussi dans $G$. Donc $G=GL_{n}(\R)$.
\end{proof}

\begin{proof}
	Supposons $u$ diagonalisable, il existe un base $\mathcal{B}$ telle que 
	\begin{equation}
		\mat_{\mathcal{B}}(u)=A=\diag(0,\dots,0,\lambda_{1},\dots,\lambda_{r})
	\end{equation}
	avec $\lambda_{i}\neq0$. Donc $\mat_{\mathcal{B}}(u^{p})=A^{p})\diag(0,\dots,0,\lambda_{1}^{p},\dots,\lambda_{r}^{p})$ donc $u^{p}$ est diagonalisable. On a toujours $\ker(u)\subset\ker(u^{2})$ et la forme diagonale implique $\ker(u)=\ker(u^{2})$.

	Supposons $u^{p}$ diagonalisable, on écrit $\Pi_{u^{p}}=(X-\lambda_{0})\dots(X-\lambda_{r})=R$ (avec $\lambda_{k}\neq0$ pour tout $k\geqslant k$) qui est scindé à racines simples. On a 
	\begin{equation}
		P(u^{p})=0=(u^{p}-\lambda_{0}id_{E})\circ\dots\circ(u^{p}-\lambda_{r}id_{E})=Q(u)
	\end{equation}
	avec $Q(X)=P(X^{p})$. 

	Si $\lambda_{0}\neq0$, chaque $\lambda_{k}$ admet $p$ racines $p$-ièmes distinctes et si $\mu_{k}$ est l'une de ses racines, on a 
	\begin{equation}
		X^{p}-\lambda_{k}=\prod_{j=1}^{p}\left(X-\mu_{k}\e^{\i\frac{2j\pi}{p}}\right)
	\end{equation}
	De plus, les racines $p$-ièmes des $(\lambda_{k})_{kk\in\llbracket1,r\rrbracket}$ sont deux à deux distinctes. Donc $Q$ est scindé à racines simples, et donc $u$ est diagonalisable.

	Si $\lambda_{0}=0$, on a $Q=X^{p}A(X)$ avec $A$ scindé à racines simples non nulles et $X^{p}\wedge A=1$. D'après le lemme des noyaux, on a 
	\begin{equation}
		\ker(Q(u))=\C^{n}=\ker(u^{p})\otimes\ker(A(u))=\ker(u^{p})\otimes_{i\in I}\ker(u-\mu_{i}id)
	\end{equation}
	car $A$ est scindé à racines simples.
	Montrons que $\ker(u)=\ker(u^{p})$. L'inclusion directe est évidente. Réciproquement, montrons que pour tout $k\in\N$, on a $\ker(u^{k})\subset\ker(u^{k+1})$ et si $\ker(u^{k})=\ker(u^{k+1})$, alors $\ker(u^{k+1})=\ker(u^{k+2})$. L'inclusion est évidente, et si on a l'égalité, si $x\in\ker(u^{k+2})$, on a $u(x)\in\ker(u^{k+1})=\ker(u^{k})$ donc $x\in\ker(u^{k+1})$.
	Comme $\ker(u)=\ker(u^{2})$, d'après ce qui précède, par récurrence, on a $\ker(u)=\ker(u^{p})$, donc $u$ est diagonalisable.
\end{proof}

\begin{proof}
	Soit $(e_{1},\dots,e_{n})$ la base canonique de $\C^{n}$, $u$ canoniquement associée à 
	\begin{equation}
		J_{n}=
		\begin{pmatrix}
			0 & 1 & 0&\dots&0\\
			\vdots & \ddots & \ddots &\ddots& \vdots\\
			\vdots&&\ddots&\ddots&0\\
			0 &&&\ddots&1\\
			1 & 0&\dots &\dots&0
		\end{pmatrix}
	\end{equation}. On a 
	\begin{equation}
		\left\lbrace
			\begin{array}[]{lll}
				u(e_{1})&=&e_{n}\\
				u(e_{2})&=&e_{1}\\
				\vdots\\
				u(e_{n})&=&e_{n-1}
			\end{array}
		\right.
	\end{equation}
	d'où 
	\begin{equation}
		\left\lbrace
			\begin{array}[]{lll}
				u^{k}(e_{1})&=&e_{n+1-k}\\
				\vdots
				u^{k}(e_{k-1})&=&e_{n-1}\\
				\vdots\\
				u^{k}(e_{n})&=&e_{n-k}
			\end{array}
		\right.
	\end{equation}
	et donc 
	\begin{equation}
		J_{n}^{k}=
		\begin{pmatrix}
			0 & \dots & \dots&0 &1 &0&\dots &0\\
			\vdots&\ddots&&&\ddots&\ddots&\ddots&\vdots\\
			\vdots&&\ddots&&&\ddots&\ddots&0\\
			0&&&\ddots&&&\ddots&1\\
			1&\ddots&&&\ddots&&&0\\
			0&\ddots&\ddots&&&\ddots&&\vdots\\
			\vdots&\ddots&\ddots&\ddots&&&\ddots&\vdots\\
			0&\dots&0&1&0&\dots&\dots&0
		\end{pmatrix}
	\end{equation}
	où les $1$ commencent à la $k+1$-ième colonne sur la première ligne et à la $n-k+1$-ième ligne sur la première colonne. Notamment, le 1 sur la dernière colonne est à la $n-k$-ième ligne.

	On a $A(a_{0},\dots,a_{n})=\sum_{k=0}^{n-1}a_{k}J_{n}^{k}$. En développant par rapport à la première ligne, on a 
	\begin{equation}
		\chi_{J_{n}}(X)=X
		\begin{vmatrix}
			X&-1&0&\dots&\dots&0\\
			0&\ddots&\ddots&\ddots&&\vdots\\
			\vdots&\ddots&\ddots&\ddots&\ddots&\vdots\\
			\vdots&&\ddots&\ddots&\ddots&0\\
			\vdots&&&\ddots&\ddots&-1\\
			0&\dots&\dots&\dots&0&X
		\end{vmatrix}+
		\begin{vmatrix}
			0&-1&0&\dots&\dots&0\\
			0&X&\ddots&\ddots&&\vdots\\
			\vdots&\ddots&\ddots&\ddots&\ddots&\vdots\\
			\vdots&&\ddots&\ddots&\ddots&0\\
			0&&&\ddots&\ddots&-1\\
			-1 &0&\dots&\dots&0&X
		\end{vmatrix}
	\end{equation}
	Le premier déterminant vaut $X^{n-1}$ et le deuxième vaut $-(-1)^{n}\times(-1)^{n-2}=-1$ donc $\chi_{J_{n}}(X)=X^{n}-1$.
	Ainsi, $\chi_{J_{n}}$ est scindé à racines simples sur $\C$ donc $J_{n}$ est diagonalisable avec des sous-espaces propres de dimension 1. Soit $\omega=\e^{\frac{2\i\pi}{n}}$, on a $\Sp(J_{n})=\left\lbrace\omega^{k},0\leqslant k\leqslant n-1\right\rbrace$. On a $J_{n}X=\omega^{k}X$ si et seulement si 
	\begin{equation}
		\left\lbrace
			\begin{array}[]{lll}
				x_{2}&=&\omega^{k}x_{1}\\
				\vdots\\
				x_{n}&=&=\omega^{k}x_{n-1}\\
				x_{1}&=&\omega^{k}x_{n}
			\end{array}
		\right.
	\end{equation}
	si et seulement si 
	\begin{equation}
		X=x_{1}\begin{pmatrix}
			1\\
			\omega^{k}\\
			\omega^{2k}\\
			\vdots\\
			(\omega^{k})^{n-1}
		\end{pmatrix}=x_{1}X_{k}
	\end{equation}
	avec $X_{k}$ vecteur propre de $J_{n}$ associé à $\omega^{k}$. Posons 
	\begin{equation}
		P=
		\begin{pmatrix}
			1&1&\dots&1\\
			\vdots&\omega&&\omega^{n-1}\\
			\vdots&\vdots&&\vdots\\
			1&\omega^{n-1}&\dots&(\omega^{n-1})^{n-1}
		\end{pmatrix}
	\end{equation}
	et $P^{-1}J_{n}P=\diag(1,\omega,\dots,\omega^{n-1})$. On a donc 
	\begin{equation*}
		P^{-1}A(a_{0},\dots,a_{n})P=\diag(Q(1),Q(\omega),\dots,Q(\omega^{n-1})),
	\end{equation*} où $Q=\sum_{k=0}^{n-1}a_{k}X^{k}$.
	Donc $A$ est diagonalisable de valeurs propres $Q(1),\dots,Q(\omega^{n-1})$ et donc
	\begin{equation}
		\boxed{\det(A)=\prod_{k=0}^{n-1}Q(\omega^{k})}
	\end{equation}
\end{proof}

\begin{remark}
	On a 
	\begin{equation}
		\begin{vmatrix}
			a&b&c\\
			c&a&b\\
			b&c&a
		\end{vmatrix}=(a+b+c)(a+\j b+\j^{2}c)(a+\j^{2}b+\j c)=(a+b+c)(a^{2}+b^{2}+c^{2}-ab-bc-ac)
	\end{equation}

	Si $a,b,c\in\R_{+}$ vérifient $a+b+c=1$, on a 
	\begin{equation}
		\left\lvert a+\j b+\j^{2}c\right\rvert=\left\lvert a+\j^{2}b+\j c\right\rvert\leqslant a+b+c=1
	\end{equation}
	si et seulement si $a,\j b,\j^{2}c$ ont même argument si et seulement si $\lbrace a,b,c\rbrace=\lbrace1,0,0\rbrace$.
\end{remark}

\begin{proof}
	On sait que que $f^{n}=0$ d'après le théorème de Cayley-Hamilton et que pour tout $k\in\N$, $\ker(f^{k})\subset\ker(f^{k+1})$ et si $\ker(f^{k})=\ker(f^{k+1})$, alors $\ker(f^{k})=\ker(f^{m})$ pour tout $m\geqslant k$.

	Soit $k\in\llbracket0,n-1\rrbracket$ et \function{u}{\ker(f^{+1})}{\ker(f^{k})}{x}{u(x)} est bien définie car si $x\in\ker(f^{k+1}),f(x)\in\ker(f^{k})$. Comme $\ker(f)\subset\ker(f^{k+1})$, $\ker(u)=\ker(f)$ et $\dim(\ker(u))=1$. D'après le théorème du rang, on a $\dim(\ker(f^{k+1}))=\rg(u)+1\leqslant\dim(\ker(f^{k}))+1$. Par récurrence, on a pour tout $k\in\N$, $\dim(\ker(f^{k}))\leqslant k$ (car on ne peut croître au lus de 1 à chaque itération).

	Si $f^{n-1}=0$, on a $\dim(\ker(f^{n-1}))=n\leqslant n-1$ ce qui est absurde. Donc 
	\begin{equation}
		\boxed{f^{n-1}\neq0}
	\end{equation}

	Soit $x\notin\ker(f^{n-1})$. Soit $(\alpha_{0},\dots,\alpha_{n-1})\in\K^{n}$. Si $\alpha_{0}x+\dots+\alpha_{n-1}f^{n-1}(x)=0$, en appliquant $f^{n-1}$, on a $\alpha_{0}f^{n-1}(x)=0$ donc $\alpha_{0}=0$. Puis on applique $f^{n-2}$, etc. De proche en proche, $\alpha_{0}=\alpha_{1}=\dots=\alpha_{n-1}=0$. Ainsi, $\mathcal{B}=(x,f(x),\dots,f^{n-1}(x))$ est libre en dimension $n$, c'est donc une base et on a 
	\begin{equation}
		\mat_{\mathcal{B}}(f)=
		\begin{pmatrix}
			0&\dots&\dots&\dots&0\\
			1&\ddots&&&\vdots\\
			0&\ddots&\ddots&&\vdots\\
			\vdots&\ddots&\ddots&\ddots&\vdots\\
			0&\dots&0&1&0
		\end{pmatrix}
	\end{equation}
	qui est une matrice nilpotente d'indice $n$. Matriciellement, on a 
	\begin{equation*}
		\ker(f^{k})=\Vect(e_{n-k+1},\dots,e_{n}).
	\end{equation*}
\end{proof}

\begin{proof}
	Supposons qu'il existe $x\in V$, $(x,u(x),\dots,u^{n-1}(x))$ soit une base de $V$. Notons $u^{n}(x)=a_{0}x+\dots+a_{n-1}u^{n-1}(x)$. Soit $y\in V$ tel que $u(y)=\lambda y$. Pour $y=\sum_{i=0}^{n-1}y_{i}u^{i}(x)$. On a donc 
	\begin{equation}
		u(y)=\sum_{i=0}^{n-1}y_{i}u^{i+1}(x)=\sum_{i=0}^{n-1}\lambda y_{i}u^{i}(x)=\sum_{i=1}^{n-1}y_{i-1}u^{i}(x)+y_{n-1}\sum_{i=0}^{n-1}a_{i}u^{i}(x)
	\end{equation}
	Donc $u(y)=\sum_{i=1}^{n-1}u^{i}(x)(y_{i-1}+y_{n-1}a_{i})+y_{n-1}a_{0}x$ donc 
	\begin{equation}
		\left\lbrace
			\begin{array}[]{lll}
				\lambda y_{0} &= &y_{n-1}a_{0}\\
				\lambda y_{1} &= &y_{0}+a_{1}y_{n-1}\\
				\vdots\\
				\lambda y_{n-2} &= &y_{n-3}+a_{n-2}y_{n-1}\\
				\lambda y_{n-1} &= &y_{n-2}+a_{n-1}y_{n-1}
			\end{array}
		\right.
	\end{equation}
	donc par récurrence 
	\begin{equation}
		\left\lbrace
			\begin{array}[]{lll}
				\lambda y_{n-2} &= &(\lambda -a_{n-1})y_{n-1}\\
				\lambda y_{n-3} &= &(\lambda(\lambda-a_{n-1})-a_{n-2})y_{n-1}\\
				\vdots\\
				\lambda y_{0} &= &(\lambda^{n-1}-a_{n-1}\lambda^{n-2}-\dots-a_{1})y_{n-1}
			\end{array}
		\right.
	\end{equation}
	Donc les sous-espaces propres sont de dimension 1.

	Supposons que les sous-espaces propres de $u$ sont de dimension 1. On écrit $\chi_{u}=\prod_{i=1}^{r}(X-\lambda_{i})^{n_{i}}$. D'après le théorème de Cayley-Hamilton et le lemme des noyaux, on a 
	\begin{equation}
		V=\bigotimes_{i=1}^{r}\underbrace{\ker(u-\lambda_{i}id_{V})^{n_{i}}}_{F_{i}}
	\end{equation}
	et les sous-espaces caractéristiques $F_{i}$ sont stables par $u$. Soit $v_{i}=u_{\mid F_{i}}-\lambda_{i}id_{F_{i}}$. On a $\chi_{u}=\prod_{i=1}^{r}\chi_{u_{\mid F_{i}}}$ (matrice diagonale par blocs dans un base adaptée). $(X-\lambda_{i})^{n}$ annule $u_{\mid F_{i}}$ et $\Sp_{F_{i}}(u_{\mid F_{i}})=\lbrace\lambda_{i}\rbrace$. Alors $\chi_{u_{\mid F_{i}}}=(X-\lambda_{i})^{\dim(F_{i})}$. En reportant, on a $\dim(F_{i})=n_{i}$. De plus, $V_{i}^{n_i}=0$ donc $v_{i}$ est nilpotent. On a donc $\dim(\ker(v_{i}))=\dim(\ker(u-\lambda_{i}id_{E}))=1$. Donc il existe $x_{i}\in F_{i}$ tel que $(x_{i},v_{i}(x_{i}),\dots,v_{i}^{n_{i}-1}(x_{i}))$ soit une base de $F_{i}$.

	On forme $x=\sum_{i=1}^{r}x_{i}$. Soit $(\alpha_{0},\dots,\alpha_{r-1})$ tel que 
	\begin{equation*}
		\sum_{j=0}^{n-1}\alpha_{j}u^{j}(x)=0=\sum_{i=1}^{r}\left(\sum_{j=0}^{n-1}\alpha_{j}u^{j}(x_{i})\right).	
	\end{equation*}
	Les $F_{i}$ sont en somme directe donc 
	\begin{equation}
		\sum_{j=0}^{n-1}\alpha_{j}u^{j}(x_{i})=0
	\end{equation}

	Soit $P(X)=\sum_{j=0}^{n-1}\alpha_{j}X^{j}$. $I_{x_{i}}=\left\lbrace A\in\C[X]\middle| A(u)(x_{i})=0\right\rbrace$ est un idéal de $\C[X]$ donc est principal et il existe $\Pi_{i}\in I_{x_{i}}$ minimal et 
	\begin{equation}
		\Pi_{i}\mid P
	\end{equation}
	On a $(X-\lambda_{i})^{n_{i}}(u)(x_{i})=0$ et $(x_{i},u(x_{i}),\dots,u^{n_{i}-1}(x))$ est libre, donc si $P\in I_{x_{i}}$, $\deg(P)\geqslant n_{i}$ donc $\deg(\Pi_{i})=n_{i}$ et $\Pi_{i}=(X-\lambda_{i})^{n_{i}}$. Ainsi, pour tout $i\in\llbracket1,r\rrbracket$, $\Pi_{i}\mid P$ et donc 
	\begin{equation}
		\prod_{i=1}^{r}(X-\lambda_{i})^{n_{i}}\mid P
	\end{equation}
	Mais $P$ est de degré $\leqslant n-1$, nécessairement $P=0$ et $(x,u(x),\dots,u^{n-1}(x))$ est libre.
\end{proof}

\begin{remark}
	Autre méthode pour le sens direct: on a 
	\begin{equation}
		\mat_{(x,u(x),\dots,u^{n-1}(x))}(u)=
		\begin{pmatrix}
			0&\dots&\dots&0&a_{0}\\
			1&\ddots&&\vdots&\vdots\\
			0&\ddots&\ddots&\vdots&\vdots\\
			\vdots&\ddots&\ddots&0&\vdots\\
			0&\dots&0&1&a_{n-1}
		\end{pmatrix}=A
	\end{equation}

	Si $\lambda\in\Sp(u)$, on a 
	\begin{equation}
		A-\lambda I_{n}=
		\mat_{(x,u(x),\dots,u^{n-1}(x))}(u)=
		\begin{pmatrix}
			-\lambda&\dots&\dots&0&a_{0}\\
			1&\ddots&&\vdots&\vdots\\
			0&\ddots&\ddots&\vdots&\vdots\\
			\vdots&\ddots&\ddots&-\lambda&\vdots\\
			0&\dots&0&1&a_{n-1}-\lambda
		\end{pmatrix}
	\end{equation}
	qui est non inversible, mais donc les $(n-1)$ première colonnes sont libres, donc est de rang $n-1$.
\end{remark}

\begin{proof}
	\phantom{}
	\begin{enumerate}
		\item On utilise le fait que pour tout $k\in\N$ tel que $\im(f^{k+1})\subset\im(f^{k})$. S'il existe $k\in\N$, $\im(f^{k+1})=\im(f^{k})$ alors pour tout $l\geqslant k$, $\im(f^{k})=\im(f^{l})$. 

		En effet, si $x=f^{k+1}(x')\in\im(f^{k+1})$,, on a $x=f^{k}(f(x))\in\im(f^{k})$. Si on a égalité des espaces, soit $x=f^{k+1}(x')=f(f^{k}(x'))\in\im(f^{k+1})$. Alors $f^{k}(x')\in\im(f^{k})=\im(f^{k+1})$ donc il existe $x''$ tel que $f^{k}(x')=f^{k+1}(x'')$, mais alors $x=f^{k+2}(x'')\in\im(f^{k+2})$. On a donc le résultat en itérant.

		Ainsi, pour tout $n\geqslant d$, on a $\rg(f^{n})=\rg(f^{d})$ donc $(\rg(f^{n}))_{n\in\N}$ est stationnaire au moins à partir de $d$ et $r(f)=\rg(f^{d})$.

		\item Comme $f$ et $g$ commutent, on a 
		\begin{equation}
			(f+g)^{2d}=\sum_{k=0}^{2d}\binom{2d}{k}f^{k}g^{2d-k}
		\end{equation}
		Pour tot $k\in\llbracket0,2d\rrbracket$, on a $k\geqslant d$ ou $2d-k\geqslant d$ donc 
		\begin{equation}
			\left\lbrace
				\begin{array}[]{l}
					\im(f^{k}g^{2d-k})\subset\im(f^{d})\\
					\text{ou}\\
					\im(f^{k}g^{2d-k})\subset\im(g^{d})
				\end{array}
			\right.
		\end{equation}
		et donc $\im(f^{k}g^{2d-k})\subset\im(f^{d})+\im(g^{d})$. Finalement, $\im(f+g)^{2d}\subset\im(f^{d})+\im(g^{d})$. On a donc 
		\begin{align}
			r(f+g)
			&=\dim(\im(f+g)^{2d})\\
			&\leqslant \dim(\im(f^{d})+\im(g^{d}))\\
			&\leqslant \dim(\im(f^{d}))+\im(g^{d})\\
			&\leqslant r(f)+r(g)
		\end{align}

		Pour un contre-exemple, on utilise $A=\begin{pmatrix}
			0&0\\1&0
		\end{pmatrix}$ et $B=A^{\mathsf{T}}$. On a $A^{2}=B^{2}$ donc $r(A^{2})=r(B^{2})=0$ et $A+B$ inversible donc $r(A+B)=2>r(A)+r(B)$.

		\item On a $\chi_{f}=X^{m_{0}}Q$ avec $\deg(Q)=d-m_{0}$ et $Q(0)=0$. D'après le lemme des noyaux, on a 
		\begin{equation}
			V=\ker(f^{m_{0}})\otimes\ker(Q(f))
		\end{equation}
		Dans une base adaptée $\mathcal{B}$, on a $\mat_{\mathcal{B}}(f)=\begin{pmatrix}
			A&0\\0&B
		\end{pmatrix}$ avec $A^{m_{0}}=0$ et $B$ inversible. Alors pour tout $k\geqslant m_{0}$, $\mat_{\mathcal{B}}(f^{k})=\begin{pmatrix}
			0&0\\0&B^{k}
		\end{pmatrix}$ et $\rg(f^{k})=\rg(B^{k})=d-m_{0}=r(f)$.
	\end{enumerate}
\end{proof}


\begin{proof}
	On munit $\M_{n}(\C)$ de la norme $\vertiii{A}=\sup\limits_{\left\lVert X\right\rVert_{\infty}=1}\left\lVert AX\right\rVert_{\infty}$. Notons que si $\lambda\in\Sp_{\C}(A)$, alors $\left\lvert\lambda\right\rvert\leqslant\vertiii{A}$. En effet, si $X$ est un vecteur propre associé à $\lambda$, on a 
	\begin{equation}
		\left\lVert AX\right\rVert_{\infty}=\left\lvert\lambda\right\rvert\left\lVert X\right\rVert_{\infty}\leqslant\vertiii{A}\left\lVert X\right\rVert_{\infty}
	\end{equation}
	et $X\neq0$ donc $\left\lVert X\right\rVert_{\infty}\neq0$.

	Soit $A=\e^{\frac{2\i k\pi}{q}}I_{n}$, soit $B\in \mathcal{G}_{q}$ telle que $\vertiii{B-A}\leqslant\sin\left(\frac{\pi}{q}\right)$. Soit $\mu\in\Sp_{\C}(B)$, on a $\mu\in\U_{q}$ car $B^{q}=I_{n}$. Donc $\mu-\e^{\frac{2\i k\pi}{q}}\in\Sp_{\C}(B-A)$ et 
	\begin{equation}
		\left\lvert\mu-\e^{\frac{2\i k\pi}{q}}\right\rvert\leqslant\sin\left(\frac{\pi}{q}\right)
	\end{equation}
	Si $\mu=\e^{\frac{2\i l\pi}{q}}$, on a 
	\begin{equation}
		\left\lvert\mu-\e^{\frac{2\i k\pi}{q}}\right\rvert=2\left\lvert\sin\left(\frac{\left(l-k\right)\pi}{q}\right)\right\rvert>\sin\left(\frac{\pi}{q}\right)
	\end{equation}
	si $l\neq k$. Nécessairement, on a $\mu=\e^{\frac{2\i k\pi}{q}}$, donc $B=A$ car $B$ est diagonalisable et $\Sp_{\C}(B)=\left\lbrace\e^{\frac{2\i k\pi}{q}}\right\rbrace$. Donc $A$ est un point isolé de $\mathcal{G}_{q}$.

	Soit maintenant $A\in\mathcal{G}_{q}$, on suppose que $A$ n'est pas une matrice scalaire, donc $\left\lvert\Sp_{\C}(A)\right\rvert\geqslant2$. Soit $\lambda\in\left(\Sp_{\C}(A)\right)$.
	Il existe $P\in GL_{n}(\C)$ telle que 
	\begin{equation*}
		P^{-1}AP=\diag(\lambda,\dots,\lambda,\mu_{1},\dots,\mu_{r}),
	\end{equation*}avec $\mu_{1},\dots,\mu_{r}\neq\lambda$. Soit $\varepsilon>0$, posons 
	\begin{equation}
		A_{\varepsilon}=P
		\begin{pmatrix}
			\lambda &0 &\dots&\dots &\dots&\dots&\dots&\dots&0\\
			0 & \ddots & \ddots &&&&&&\vdots\\
			\vdots &\ddots&\ddots&0&\dots&\dots&\dots&\dots&\vdots\\
			\vdots&&\ddots&\lambda & \varepsilon&0&\dots&\dots&0\\
			\vdots&&& \ddots & \mu_{1}&0&\dots&\dots&0\\
			\vdots&&&&\ddots&\ddots&\ddots&&\vdots\\
			\vdots&&&&&\ddots&\ddots&\ddots&\vdots\\
			\vdots&&&&&&\ddots&\ddots&0\\
			0&\dots&\dots&\dots&\dots&\dots&\dots&0&\mu_{r}
		\end{pmatrix}P^{-1}
	\end{equation}

	On a $A_{\varepsilon}\xrightarrow[\varepsilon\to0]{}A$ et $A_{\varepsilon}\neq A$. Montrons que $A_{\varepsilon}\in \mathcal{G}_{q}$. On a $\chi_{A_{\varepsilon}}=\chi_{A}$, $\rg\left(A_{\varepsilon}-\lambda I_{n}\right)=\rg(A-\lambda I_{n})$ (observer les colonnes) et pour $\mu_{l}\in\Sp(A)$, $\mu_{l}\neq \lambda$, on a $\rg\left(A_{\varepsilon}-\mu_{l}I_{n}\right)=\rg\left(A-\mu_{l}I_{n}\right)$ (observer les lignes). La dimension des sous-espaces propres de $A$ et $A_{\varepsilon}$ sont les mêmes donc $A_{\varepsilon}$ est diagonalisable. De plus, $\Sp(A_{\varepsilon})\subset\Sp(A)\subset\U_{q}$ donc $A_{\varepsilon}\in\mathcal{G}_{q}$. Ainsi, $A$ n'est pas isolé dans $\mathcal{G}_{q}$.
\end{proof}

\begin{proof}
	On a 
	\begin{equation}
		\chi_{M}(\lambda)=
		\begin{vmatrix}
			\lambda-1 &1&0\\
			1&\lambda-2&-1\\
			-1&0&\lambda-1
		\end{vmatrix}=(\lambda-1)(\lambda-2)(\lambda-1)-\left((\lambda-1)-1\right)=\lambda(\lambda-2)^{2}
	\end{equation}

	Soit $X=\begin{pmatrix}
		x&y&z
	\end{pmatrix}^{\mathsf{T}}$. On a $MX=0$ si et seulement si $y=x$ et $z=-x$ donc $E_{0}=\Vect\begin{pmatrix}
		1&1&-1
	\end{pmatrix}=\Vect(\varepsilon_{1})$.

	On a $(M-2I_{3})X=0$ si et seulement si $y=z=-x$ donc $E_{2}=\Vect\begin{pmatrix}
		1&-1&-1
	\end{pmatrix}=\Vect(\varepsilon_{2})$.

	$M$ n'est pas diagonalisable sur $\R$ mais trigonalisable. D'après le théorème de Cayley-Hamilton et le lemme des noyaux, on a 
	\begin{equation}
		\R^{3}=\ker(u)\otimes\ker(u-2 id)^{2}
	\end{equation}
	Soit $P\in GL_{n}(\C)$ tel que 
	\begin{equation}
		P^{-1}MP=
		\begin{pmatrix}
			0&0&0\\
			0&2&\star\\
			0&0&2
		\end{pmatrix}
	\end{equation}
	avec $\varepsilon_{3}\in\ker(u-2id)^{2}$ et $\varepsilon_{3}\notin\Vect(\varepsilon_{2})$. On a 
	\begin{equation}
		(M-2I_{3})^{2}=
		\begin{pmatrix}
			2&1&-1\\
			2&1&-1\\
			-2&-1&1
		\end{pmatrix}
	\end{equation}
	donc $(M-2I_{3})^{2}X=0$ si et seulement si $2x-y+z=0$. On pose $\varepsilon_{3}=\begin{pmatrix}
		0\\1\\1
	\end{pmatrix}$. On a $M\varepsilon_{3}=-\varepsilon_{2}+2\varepsilon_{3}$ donc si 
	\begin{equation}
		P=
		\begin{pmatrix}
			1&1&0\\
			1&-1&1\\
			-1&1&1
		\end{pmatrix}
	\end{equation}
	on a 
	\begin{equation}
		P^{-1}MP=
		\begin{pmatrix}
			0&0&0\\
			0&2&-1\\
			0&0&2
		\end{pmatrix}
	\end{equation}

	Les sous-espaces stables de dimension $0$: $\lbrace0\rbrace$. Les sous-espaces stables de dimension 1: ils sont engendrés par les vecteurs propres, ce sont donc $\Vect(\varepsilon_{1})$ et $\Vect(\varepsilon_{2})$. Si maintenant $F$ est un sous-espace stable de dimension $2$, montrons que l'on a 
	\begin{equation}
		F=\left(F\cap\ker(u)\right)\otimes\left(F\cap\ker(u-2id)^{2}\right)=F_{0}\otimes F_{2}
	\end{equation}
	En effet, on a $F_{0}\otimes F_{2}\subset F$. Si maintenant $x\in F$, a priori on a $x=x_{0}+x_{2}$ avec $x_{0}\in\ker(u)$ et $x_{2}\in\ker(u-2id)^{2}$. On a $u(x)=u(x_{2})\in F$ par stabilité, $u^{2}(x)=u^{2}(x_{2})\in F$, et $(u-2id)^{2}(x_{2})=0$ donc $x_{2}=\frac{1}{4}\left(-u^{2}(x)+4u(x_{2})\right)\in F$ et $x_{0}=x-x_{2}\in F$.

	On a $F_{0}=\lbrace0\rbrace$ ou $\ker(u)$. Si $F_{0}=\lbrace0\rbrace$, on a $F=F_{2}$. Si $F_{0}=\ker(u)$, on a $\dim(F_{2})=1$ donc $F_{2}=\Vect(\varepsilon_{2})$.

	Donc les sous-espaces stables de dimension 2 sont $\ker(u-2id)^{2}$ et $\Vect(\varepsilon_{1},\varepsilon_{2})$. 

	Enfin, les sous-espaces stables de dimension 3: $\R^{3}$.
\end{proof}

\begin{remark}
	Plus généralement, si $\chi_{u}=\prod_{i=1}^{r}(X-\lambda_{i})^{m_{i}}$. On écrit 
	\begin{equation}
		E=\bigotimes_{i=1}^{r}\ker(u-\lambda_{i}id_{E})^{m_{i}}=\bigotimes_{i=1}^{r}F_{i}
	\end{equation}
	Si $F$ est stable, on note $\Pi_{i}$ le projecteur sur $F_{i}$ parallèlement à $\bigotimes_{\substack{j=1\\j\neq i}}^{r}F_{i}\in\K[u]$. On a pour tout $x\in F$, $\Pi_{i}(x)\in F$ par stabilité, il s'ensuit que 
	\begin{equation}
		\boxed{F=\bigotimes_{i=1}^{r}\left(F\cap F_{i}\right)}
	\end{equation}
\end{remark}

\begin{proof}
	\phantom{}
	\begin{enumerate}
		\item Si $(a_{1},\dots,a_{n})\neq(0,\dots,0)$, on a 
		\begin{equation}
			A-I_{n+1}=
			\left(
			\begin{array}{@{}c|c@{}}
			0_{n} &
			\begin{matrix}
				a_{1}\\
				\vdots\\
				a_{n}
				\end{matrix}
				\\
			\hline
			\begin{matrix}
				a_{1} &
				\dots
				& a_{n}
				\end{matrix}
				& 0
			\end{array}
			\right)
		\end{equation}
		donc $\rg(A-I_{n+1})=2$ et $\chi_{A-I_{n+1}}=X^{n-1}(X-\lambda)(X-\mu)$ (sur $\C$). On a $\Tr(A-I_{n+1})=0=\mu+\lambda$ et $\Tr(A-I_{n+1})^{2}=2\sum_{i=1}^{n}a_{i}^{2}=\lambda^{2}+\mu^{2}$ donc $\lbrace\lambda,\mu\rbrace\in\left\lbrace\pm\sqrt{\sum_{i=1}^{n}a_{i}^{2}}\right\rbrace$ et $A-I_{n+1}$ est semblable à 
		\begin{equation}
			\begin{pmatrix}
				0&\star&\dots&\dots&\star\\
				0&\ddots&\ddots&&\vdots\\
				\vdots&\ddots&0&\ddots&\vdots\\
				\vdots&&\ddots&\lambda&\star\\
				0&\dots&\dots&0&\mu
			\end{pmatrix}
		\end{equation}

		\item On note $\lambda=\sqrt{\sum_{i=1}^{n}a_{i}^{2}}$. Soit $X$ tel que $A'X=\pm\lambda$ où $A'=A-I_{n+1}$. Alors en écrivant le système, on vérifie que l'on peut prendre 
		\begin{equation}
			f_{\pm}=\begin{pmatrix}
				a_{1}\\
				\vdots\\
				a_{n}\\
				\lambda
			\end{pmatrix}	
		\end{equation}
		et si $X$ est tel que $A'X=0$, si $i_{0}$ est tel que $a_{i_{0}}\neq0$, on récupère une bas de $\ker(A')$ avec 
		\begin{equation}
			f_{i}=\begin{pmatrix}
				0\\
				\vdots\\
				0\\
				1\\
				0\\
				\dots\\
				0\\
				-\frac{a_{i}}{a_{i_{0}}}\\
				0\\
				\vdots\\
				0
			\end{pmatrix}
		\end{equation}
		où $i\in\llbracket1,n\rrbracket\setminus\lbrace i_{0}\rbrace$. Le 1 est à la $i$-ième ligne, $-\frac{a_{i}}{a_{i_{0}}}$ est à la ligne $i_{0}$.
	\end{enumerate}
\end{proof}


\begin{proof}
	On pose $\vertiii{A}=\sup\limits_{\left\lVert X\right\rVert_{\infty}=1}\left\lVert AX\right\rVert_{\infty}$. On montre d'abord que si $A\in\M_{n}(\C)$, alors pour tout $\lambda\in\Sp_{\C}(A)$ $\left\lvert\lambda\right\rvert\leqslant\vertiii{A}$. En effet, si $X$ non nul est tel que $AX=\lambda X$, on a $\left\lVert AX\right\rVert_{\infty}\leqslant\vertiii{A}\left\lVert X\right\rVert_{\infty}$ donc $\left\lvert\lambda\right\rvert\left\lVert X\right\rVert_{\infty}\leqslant\vertiii{A}\left\lVert X\right\rVert_{\infty}$, d'où le résultat.

	Soit alors $m=\sup\left\lbrace\vertiii{M}\middle| M\in G\right\rbrace$. Soit $M\in G$ et $\lambda\in\Sp_{\C}(M)$. On a $\left\lvert\lambda\right\rvert\leqslant m$. Comme $M^{k}\in G$ pour tout $k\in\Z$, on a aussi $\left\lvert\lambda\right\rvert^{k}\leqslant m$ donc $\left\lvert\lambda\right\rvert=1$ (faire tendre $k$ vers $-\infty$ et $+\infty$).

	Grâce à la décomposition de Dunford, on a $M=D+N$ avec $D$ diagonalisable, $N$ nilpotente et $D$ et $N$ qui commutent. Soit $r\in\N^{*}$ telle que $N^{r-1}\neq0$ et $N^{r}=0$. On a $\Sp_{\C}(D)=\Sp_{\C}(M)\subset\U$ donc $G\in GL_{n}(\C)$ et $MD^{-1}=I_{n}+ND^{-1}$ avec $ND^{-1}$ est nilpotente d'indice $r$. On a pour tout $k\geqslant r$, on a 
	\begin{align}
		(MD^{-1})^{k}
		&=M^{k}(D^{-1})^{k}\\
		&=\sum_{i=0}^{k}\binom{k}{i}(ND^{-1})^{i}\\
		&=\sum_{i=0}^{r-1}\binom{k}{i}(ND^{-1})^{i}\\
		&\underset{k\to+\infty}{\sim}\frac{k^{r-1}}{(r-1)!}N^{r-1}D^{k-r+1}
	\end{align}

	Notons que pour tout $(A,B)\in\M_{n}(\C)$, pour tout $X\in\M_{n,1}(\C)$ si $\left\lVert X\right\rVert_{\infty}=1$, on a 
	\begin{equation}
		\left\lVert (AB)X\right\rVert_{\infty}\leqslant\vertiii{A}\left\lVert BX\right\rVert_{\infty}\leqslant\vertiii{A}\vertiii{B}
	\end{equation}
	donc $\vertiii{AB}\leqslant\vertiii{A}\vertiii{B}$.

	La suite $(MD^{-1})^{k}$ est bornée, donc $r=1$ et $N=0$, donc $M$ est diagonalisable.

	Prenons ensuite $\alpha=\sqrt{3}$. Soit $\lambda\in\Sp_{\C}(M)$, on a $\lambda-1\in\Sp_{\C}(M-I_{n})$. Si $\lambda=\e^{\i\theta}$ avec $\theta\in]-\pi,\pi[$, on a $\left\lvert\e^{\i\theta}-1\right\rvert<\sqrt{3}$ si et seulement si $2\sin\left(\frac{\theta}{2}\right)<\sqrt{3}$ si et seulement si $\theta\in\left]-\frac{2\pi}{3},\frac{2\pi}{3}\right[$.

	Pour tout $k\in\R$, on a $\left(\e^{\i\theta}\right)^{k}\in\Sp(M^{k})$. Donc on a aussi $\left\lvert\sin\left(\frac{k\theta}{2}\right)\right\rvert<\frac{\sqrt{3}}{2}$. Quitte à changer $\theta$ en $-\theta$, on peut supposer $\theta\in\left[0,\frac{2\pi}{3}\right[$. Si $\theta\geqslant0$, posons l'unique $k\in\N$ tel que $k\theta\geqslant\frac{2\pi}{3}$ et $(k-1)\theta\in\left[0,\frac{2\pi}{3}\right[$. On a alors 
	\begin{equation}
		k\theta=(k-1)\theta+\theta<\frac{4\pi}{3}
	\end{equation}
	ce qui est absurde si et seulement si $\left\lvert\sin\left(\frac{k\theta}{2}\right)\right\rvert\geqslant\frac{\sqrt{3}}{2}$.

	Ainsi, $\theta=0$ et $\Sp(M)=\lbrace1\rbrace$, et puisque $M$ est diagonalisable, $M=I_{n}$ et $G=\lbrace I_{n}\rbrace$.
\end{proof}

\begin{remark}
	Soit $\alpha>\sqrt{3}$ et $G=\left\lbrace I_{n},\j I_{n},\j^{2} I_{n}\right\rbrace$. Pour tout $M\in G$, $\vertiii{M-I_{n}}<\alpha$ et $G\neq\lbrace I_{n}\rbrace$.
\end{remark}

\begin{proof}
	\phantom{}
	\begin{enumerate}
		\item On vérifie que $\chi_{A}(\lambda)=\lambda^{3}$. On a $AX=0$ si et seulement si $x_{1}=x_{3}$ et $x_{2}=-2x_{1}$. On prend $\varepsilon_{1}=\begin{pmatrix}
			1\\-1\\1
		\end{pmatrix}$. $u$ est nilpotente et $\dim(\ker(u))=1$. On a $u^{3}=0$ et on a
		\begin{equation}
			A^{2}=\begin{pmatrix}
				1&1&1\\
				-2&-2&-2\\
				1&1&1
			\end{pmatrix}\neq0
		\end{equation}
		Soit $e_{1}=\begin{pmatrix}
			1\\0\\0
		\end{pmatrix}$, on a $u^{2}(e_{1})\neq0$ donc $(u^{2}(e_{1}),u(e_{1}),e_{1})$ est une base de $\R^{3}$. On a
		\begin{equation}
			\mat_{\mathcal{B}}(u)=\begin{pmatrix}
				0&1&0\\
				0&0&1\\
				0&0&0
			\end{pmatrix}
		\end{equation}
		
		$\dim(\ker(u^{k}))=k$ pour $k\in\lbrace0,1,2,3\rbrace$, notamment car $\rg(u^{2})=1$ pour justifier que $\dim(\ker(u^{2}))=2$.

		Soit $F$ stable par $u$ de dimension $i\in\lbrace0,1,2,3\rbrace$. $u_{\mid F}$ est nimpotente et $u_{\mid F}^{i}=0$. Donc $F\subset\ker(u^{i})$ qui est de dimension $i$. Donc $F=\ker(u^{i})$.

		\item Si $B^{2}=A$, $B^{6}=0$ donc $B^{3}=0$. Alors $B^{4}=0=A^{2}$ ce qui n'est pas vrai. Donc il n'y a pas de $B$ tel que $B^{2}=A$.
	\end{enumerate}
\end{proof}

\begin{proof}
	Soit $u\in\L(\R^{3})$ canoniquement associé à $A$. On a $X^{3}+X^{2}+X+1=(X+1)(X^{2}+1)$. D'après le lemme des noyaux, on a 
	\begin{equation}
		\R^{3}=\ker(u+id)\otimes\underbrace{\ker(u^{2}+id)}_{F}
	\end{equation}
	On a $F\neq\lbrace0\rbrace$ car $u\neq-id$. On note $v=u_{\mid F}$. On a $v^{2}=-id_{F}$ et $\det(v^{2})=(\det(v))^{2}=(-1)^{\dim(F)}>0$ donc $\dim(F)$ est pair. Nécessairement, on a $\dim(F)=2$ et $\dim(\ker(u+id))=1$. Soit $\varepsilon_{3}$ vecteur propre associé à -1. Soit $x\in F\setminus\lbrace0\rbrace$. Si $(x,u(x))$ est lié, $x$ est vecteur propre de $v$ et $v^{2}+id_{F}=0$ ce qui est impossible car il n'y a pas de valeur propre réelle. Donc on pose $\mathcal{B}=(x,u(x),\varepsilon_{3})$ base de $\R^{3}$. On a $u^{2}(x)=-x$, donc 
	\begin{equation}
		\boxed{
			\mat_{\mathcal{B}}(u)=
			\begin{pmatrix}
				0&-1&0\\
				1&0&0\\
				0&0&-1
			\end{pmatrix}
		}
	\end{equation}
\end{proof}

\begin{remark}
	Sur $\C$, on peut prendre 
	\begin{equation}
		A=
		\begin{pmatrix}
			\i&0&0\\
			0&-1&0\\
			0&0&-1
		\end{pmatrix}
	\end{equation}
	ou $\i I_{3}$.
\end{remark}

\begin{proof}
	\phantom{}
	\begin{enumerate}
		\item $I_{x}=\left\lbrace A\in\K[X]\middle| A(f)(x)=0\right\rbrace$ est un idéal de $\K[X]$ non vide car $\mu_{f}\in I_{x}$. Donc il existe un unique $P_{x}$ unitaire tel que $I_{x}=P_{x}\K[X]$.
		\item Soit $A\in\K[X]$, on a $A(f)=0$ si et seulement si pour tout $x\in E$, $P_{x}\mid A$ si et seulement si $\vee_{x\in E}P_{x}\mid A$ donc $\mu_{f}=\vee_{x\in E}P_{x}$.
		\item On a 
		\begin{align}
			P_{x}P_{y}(f)(x+y)
			&=P_{x}P_{y}(f)(x)+P_{x}P_{y}(f)(y)\\
			&=P_{f}(f)\left(P_{x}(f)(x)\right)+P_{x}(f)\left(P_{y}(f)(y)\right)\\
			&=0
		\end{align}
		Donc $P_{x+y}\mid P_{x}P_{y}$.

		Soit $A\in\K[X]$. Supposons $A(f)(x+y)=0$. On a $P_{x}(f)\left(A(f)(x+y)\right)=0$, $P_{x}(f)\left(A(f)(x)\right)=A(f)\left(P_{x}(f)(x)\right)=0=-AP_{x}(f)(y)$. Donc $P_{y}\mid AP_{x}$. D'après le théorème de Gauss, $P_{y}\mid A$. De même, $P_{x}\mid A$. On prend $A=P_{x+y}$. Comme $P_{x}\wedge P_{y}=1$ et $P_{x}P_{y}\mid P_{x+y}$, on a 
		\begin{equation}
			\boxed{P_{x}P_{y}=P_{x+y}}
		\end{equation}

		\item On décompose $\mu_{f}=\prod_{i=1}^{r}A_{i}^{\alpha_{i}}$ avec pour tout $i\in\llbracket1,r\rrbracket$, $A_{i}$ irréductible sur $\K[X]$ et $\alpha_{i}\geqslant1$.
		Comme $\mu_{f}=\vee_{x\in E}P_{x}$, pour tout $i\in\llbracket1,r\rrbracket$, il existe $y_{i}\in E$ tel que $P_{y_{i}}=A_{i}^{\alpha_{i}}Q_{i}$. On pose $x_{i}=Q_{i}(f)(y_{i})$. Pour tout $A\in\K[X]$, on a $A(f)(x_{i})=0$ si et seulement si $AQ_{i}(f)(y_{i})=0$ si et seulement si $A_{i}^{\alpha_{i}}Q_{i}\mid AQ_{i}$ si et seulement si $A_{i}^{\alpha_{i}}\mid A$. Ainsi, $P_{x_{i}}=A_{i}^{\alpha_{i}}$. En utilisant le point précédent par récurrence, on a $\mu_{f}=P_{\sum_{i=1}^{r}x_{i}}$ et on pose donc $x=\sum_{i=1}^{r}x_{i}$.

		\item Supposons que ce $v$ existe. D'après le théorème de Cayley-Hamilton, $\deg(\mu_{f})\leqslant n$. Soit $(\alpha_{0},\dots,\alpha_{n-1})\in\K^{n}$. Si $\alpha_{0}id+\alpha_{1}f+\dots+\alpha_{n-1}f^{n-1}=0$. En appliquant en $v$, comme la famille est libre, on a de proche en proche $\alpha_{0}=\dots=\alpha_{n-1}=0$. Donc pour tout $A\in\K_{n-1}[X]$, si $A(f)=0$, alors $A=0$. Donc $\deg(\mu_{f})\geqslant n$, donc $\deg(\mu_{f})=n$.
		
		Réciproquement, soit $v\in E$ tel que $P_{v}=\mu_{f}$ qui existe d'après le point précédent. Soit $(\alpha_{0},\dots,\alpha_{n-1})\in\K^{n}$ tel que $\alpha_{0}v+\dots+\alpha_{n-1}f^{n-1}(v)=0$. On forme $A=\alpha_{0}+\dots+\alpha_{n-1}X^{n-1}$. On a $A(f)(v)=0$, donc $P_{v}\mid A$ mais $P_{v}$ est de degré $n$ donc $A=0$. Donc la famille est libre et de cardinal $n$: c'est une base.
	\end{enumerate}
\end{proof}

\begin{proof}
	\phantom{}
	\begin{enumerate}
		\item $\varphi$ est linéaire, soit $(s,t)\in S^{2}$. On a $\varphi(s)=t$ si et seulement si $\frac{1}{n}\sum_{k=1}^{n}s_{k}=t_{n}$ pour tout $n\geqslant1$ si et seulement si 
		\begin{equation}
			\left\lbrace
				\begin{array}[]{lll}
					s_{1}&=&t_{1}\\
					s_{1}+s_{2}&=&2t_{2}\\
					\vdots\\
					s_{1}+\dots+s_{n-1}&=&(n-1)t_{n-1}\\
					s_{1}+\dots+s_{n}&=&nt_{n}\\
					\vdots
				\end{array}
			\right.
		\end{equation}
		si et seulement si 
		\begin{equation}
			\left\lbrace
				\begin{array}[]{lll}
					s_{1}&=&t_{1}\\
					s_{2}&=&2t_{2}-t_{1}\\
					\vdots\\
					s_{n}=nt_{n}-(n-1)t_{n-1}\\
					\vdots
				\end{array}
			\right.
		\end{equation}
		donc $\varphi$ est bijective.

		\item Soit $\lambda\in\R$, il existe $s\in S\setminus\lbrace0\rbrace$ tel que $\varphi(s)=\lambda s$ si et seulement si $\frac{1}{n}\sum_{k=1}^{n}s_{k}=\lambda s_{n}$ pour tout $n\geqslant1$ si et seulement si $s\in S\setminus\lbrace0\rbrace$ tel que $s=\lambda\varphi^{-1}(s)$ si et seulement si $s\in S\setminus\lbrace0\rbrace$ tel que $s_{1}=\lambda s_{1}$ et pour tout $n\geqslant2$, $s_{n}=\lambda(ns_{n}-(n-1)s_{n-1})$ i.e.~$(\lambda n-1)s_{n}=\lambda(n-1)s_{n-1}$.
		
		Si c'est le cas, si $s_{1}\neq0$, on a $\lambda=1$ et pour tout $n\geqslant2$, $s_{n}=s_{n-1}$ donc $s$ est constante.

		Sinon, soit $n_{0}=\min\left\lbrace n\in\N^{*}\middle| s_{n_{0}}\neq0\right\rbrace$. On a $(\lambda n_{0}-1)s_{n_{0}}=0$ donc $\lambda=\frac{1}{n_{0}}$ et pour tout $n> n_{0}$, on a $s_{n}=\frac{\frac{1}{n_{0}}(n-1)}{\frac{n}{n_{0}}-1}s_{n-1}=\frac{n-1}{n-n_{0}}s_{n-1}$. Ainsi, 
		\begin{equation}
			s_{n}=\frac{(n-1)!}{(n_{0}-1)!(n-n_{0})!}s_{n_{0}}=\binom{n-1}{n_{0}-1}s_{n_{0}}
		\end{equation}

		Réciproquement, en posant $s_{n_{0}}=1$ et en définissant pour tout $n>n_{0}$, $s_{n}=\binom{n-1}{n_{0}-1}$ et pour tout $n\leqslant n_{0}-1$, $s_{n}=0$, alors on a $\varphi(s)=\frac{1}{n_{0}}s$. Ainsi, 
		\begin{equation}
			\boxed{
				\Sp(\varphi)=\left\lbrace\frac{1}{n}\middle| n\in\N^{*}\right\rbrace
			}
		\end{equation}
		et les sous-espaces propres sont de dimension 1.
	\end{enumerate}
\end{proof}

\begin{remark}
	Si on se limite à $\R^{p}$, en définissant 
	\begin{equation*}
		\varphi_{p}(s_{1},\dots,s_{p})=(s_{1},\frac{s_{1}+s_{2}}{2},\dots,\frac{s_{1}+\dots+s_{p}}{p}),	
	\end{equation*}
	alors en écrivant la matrice de $\varphi_{p}$ dans la base canonique, on a 
	\begin{equation*}
		\chi_{\varphi_{p}}=(X-1)(X-\frac{1}{2})\dots(X-\frac{1}{p}).
	\end{equation*}
\end{remark}

\begin{proof}
	\phantom{}
	\begin{enumerate}
		\item Soit $\lambda\in\Sp(A)$. Supposons que pour tout $i\in\llbracket1,n\rrbracket$, $\left\lvert\lambda-a_{i,i}\right\rvert >L_{i}$. $\lambda I_{n}-A$ est une matrice à diagonale strictement dominante donc inversible: absurde. Donc il existe $i\in\llbracket1,n\rrbracket$ tel que $\lambda\in D_{i}$. Comme $\lambda\in\Sp(A^{\mathsf{T}})$, il existe $i\in\llbracket1,n\rrbracket$ tel que $\lambda\in S_{i}$. D'où le résultat.
		\item Soit $X$ non nul (dans $\C^{n}$) tel que $AX=\lambda X$. Soit $i_{1}\in\llbracket1,n\rrbracket$ tel que $\left\lvert x_{i}\right\rvert=\left\lVert X\right\rVert_{\infty}>0$. On a, pour tout $i\in\llbracket1,n\rrbracket$, $(\lambda-a_{i,i})x_{i}=\sum_{j\neq i}a_{i,j}x_{j}$. 
		
		Soit $i_{2}\in\llbracket1,n\rrbracket$ tel que $\left\lvert x_{i_{2}}\right\rvert=\max_{i\neq i_{1}}\left\lvert x_{i}\right\rvert$. 
		
		Si $x_{i_{2}}=0$, on a $\lambda=a_{i_{1},i_{1}}$ et $\left\lvert \lambda-a_{i_{1},a_{1}}\right\rvert=0$ et $\left\lvert\lambda-a_{i_{1},a_{1}}\right\rvert\left\lvert\lambda-a_{i_{2},i_{2}}\right\rvert=0\leqslant L_{i_{1}}L_{i_{2}}$.

		Sinon, on a $\left\lvert\lambda-a_{i_{1},i_{1}}\right\rvert\left\lvert x_{i_{1}}\right\rvert\leqslant\left\lvert x_{i_{2}}\right\rvert L_{i_{1}}$ et de même $\left\lvert\lambda-a_{i_{2},i_{2}}\right\rvert\left\lvert x_{i_{2}}\right\rvert\leqslant\left\lvert x_{i_{1}}\right\rvert L_{i_{2}}$ d'où le résultat.
	\end{enumerate}
\end{proof}

\begin{remark}
	On peut avoir égalité, par exemple avec la matrice nulle.
\end{remark}

\begin{proof}
	$A$ est symétrique réelle, donc diagonalisable. Si pour tout $i\in\llbracket1,n\rrbracket$, $a_{i}=0$ alors $A=0$.

	Sinon, $\rg(A)=2$. Soit $u\in\L(\R^{n+1})$ canoniquement associée à $A$. On a $\dim(\ker(u))=n-1$ et $\chi_{A}=X^{n-1}(X-\lambda)(X-\mu)$ sur $\C$. On a $\Tr(A)=\lambda+\mu=0$ donc $\mu=-\lambda$ et $\Tr(A^{2})=\lambda^{2}+\mu^{2}=2\sum_{i=1}^{n}a_{i}^{2}$ donc 
	\begin{equation}
		\left\lbrace\lambda,\mu\right\rbrace=\left\lbrace\pm\sqrt{\sum_{i=1}^{n}a_{i}^{2}}\right\rbrace
	\end{equation}
	Les deux valeurs propres sont de multiplicité 1 dans $\chi_{A}$: les sous-espaces propres sont de dimension 1.

	$A$ est diagonalisable sur $\R$, semblable à $\diag(0,\dots,0,\sqrt{\sum_{i=1}^{n}a_{i}^{2}},-\sqrt{\sum_{i=1}^{n}a_{i}^{2}})$.

	On a $AX=0$ si et seulement si 
	\begin{equation}
		\left\lbrace
			\begin{array}[]{lll}
				a_{1}x_{n+1}&=&0\\
				\vdots\\
				a_{n}x_{n+1}&=&0\\
				a_{1}x_{1}+\dots+a_{n}x_{n}&=&0
			\end{array}
		\right.
	\end{equation}
	si et seulement si 
	\begin{equation}
		\left\lbrace
			\begin{array}[]{lll}
				x_{n+1}&=&0\\
				x_{i_{0}}&=&\sum_{\substack{i=1\\ i\neq i_{0}}}a_{i}x_{i}
			\end{array}
		\right.
	\end{equation}
	avec $i_{0}$ indice tel que $a_{i_{0}}\neq0$. Une base de $\ker(u)$ est donc 
	\begin{equation}
		f_{i}=\begin{pmatrix}
			0\\
			\vdots\\
			0\\
			1\\
			0\\
			\dots\\
			0\\
			-\frac{a_{i}}{a_{i_{0}}}\\
			0\\
			\vdots\\
			0
		\end{pmatrix}
	\end{equation}
	où $i\in\llbracket1,n\rrbracket\setminus\lbrace i_{0}\rbrace$. Le 1 est à la $i$-ième ligne, $-\frac{a_{i}}{a_{i_{0}}}$ est à la ligne $i_{0}$.

	On a $AX=\sqrt{\sum_{i=1}^{n}a_{i}^{2}}X=\alpha X$ si et seulement si 
	\begin{equation}
		\left\lbrace
			\begin{array}[]{lll}
				a_{1}x_{n+1}&=&\alpha x_{1}\\
				\vdots\\
				a_{n}x_{n+1}&=&\alpha x_{n}\\
				\sum_{i=1}^{n}a_{i}x_{i}&=&\alpha x_{n+1}
			\end{array}
		\right.
	\end{equation}
	si et seulement si
	\begin{equation}
		\left\lbrace
			\begin{array}[]{lll}
				x_{1}&=&\frac{a_{1}}{\alpha}x_{n+1}\\
				\vdots\\
				x_{n}&=&\frac{a_{n}}{\alpha}x_{n+1}\\
			\end{array}
		\right.
	\end{equation}
	Une base de $ker(u-\alpha id)$ est donc 
	\begin{equation}
		\begin{pmatrix}
			\frac{a_{1}}{\alpha}\\
			\dots\\
			\frac{a_{n}}{\alpha}\\
			1
		\end{pmatrix}
	\end{equation}

	De même pour $-\alpha$, on vérifie qu'une base de $\ker(u+\alpha id)$ est 
	\begin{equation}
		\begin{pmatrix}
			-\frac{a_{1}}{\alpha}\\
			\dots\\
			-\frac{a_{n}}{\alpha}\\
			1
		\end{pmatrix}
	\end{equation}
\end{proof}

\begin{proof}
	Pour le sens direct, $f$ et $g$ ont les mêmes espaces propres distincts. Pour tout $i\in\llbracket1,r\rrbracket$, $E_{i}=\ker(f-\lambda_{i}id)=\ker(g-\mu_{i}id)$ avec $\lambda_{1},\dots,\lambda_{r}$ valeurs propres distinctes deux à deux de $f$ et $\mu_{1},\dots,\mu_{r}$ pour $g$.

	On pose 
	\begin{equation}
		\begin{array}[]{lll}
			Q&=&\sum_{i=1}^{r}\mu_{i}\prod_{\substack{j=1\\ j\neq i}}^{r}\frac{X-\lambda_{j}}{\lambda_{i}-\lambda_{j}}\\
			P&=&\sum_{i=1}^{r}\lambda_{i}\prod_{\substack{j=1\\ j\neq i}}^{r}\frac{X-\mu_{j}}{\mu_{i}-\mu_{j}}
		\end{array}
	\end{equation}

	Soit $i\in\llbracket1,r\rrbracket$ et $x\in E_{i}$. On a $Q(f)(x)=Q(\lambda_{i})x=\mu_{i}x=g(x)$. $Q(f)$ et $g$ coïncident sur les vecteurs d'une base, donc ils sont égaux, donc $Q(f)=g$. De même, $f=P(g)$.

	Réciproquement, s'il existe $(P,Q)\in\K_{n-1}[X]^{2}$ tel que $f=P(g)$ et $g=Q(f)$. On prend $\lambda\in\Sp(f)$, soit $x\in\ker(f-\lambda id)$. On a $g(x)=Q(f)(x)=Q(\lambda)x$ donc $x\in\ker(g-Q(\lambda)id)$. On a 
	\begin{equation}
		\ker(f-\lambda id)\subset\ker(g-Q(\lambda)id)\subset\ker(f-P(Q(\lambda))id)
	\end{equation}
	Or $P(Q(\lambda))=\lambda$ car pour $x\in\ker(f-\lambda id)\setminus\lbrace0\rbrace$, on a $\lambda x=P(Q(\lambda))x$. Donc $\ker(f-\lambda id)=\ker(g-Q(\lambda)id)$ donc $f$ et $g$ ont les mêmes sous-espaces propres.
\end{proof}

\begin{remark}
	C'est faux si $f$ et $g$ ne sont pas diagonalisables, par exemple 
	\begin{equation}
		A=\begin{pmatrix}
			0&1&0&0\\
			0&0&0&0\\
			0&0&0&0\\
			0&0&1&0
		\end{pmatrix},~~~B=\begin{pmatrix}
			0&1&0&0\\
			0&0&1&0\\
			0&0&0&0\\
			0&0&0&0
		\end{pmatrix}
	\end{equation}
	$A$ et $B$ ont les mêmes sous-espaces propres  (un seul: $\Vect(e_{1},e_{4})$). On a $A^{2}=0$ donc pour tout $P\in\C[X]$, il existe $(\alpha,\beta)\in\C^{2}$,
	\begin{equation}
		P(A)=\alpha I_{4}+\beta A=\begin{pmatrix}
			\alpha&\beta&0&0\\
			0&\alpha&0&0\\
			0&0&\alpha&0\\
			0&0&\beta&\alpha
		\end{pmatrix}\neq B
	\end{equation}
\end{remark}

\begin{proof}
	Soit $m=\left\lvert G\right\rvert$. On a pour tout $M\in G$, on a $M^{m}=I_{2}$ donc $M$ est diagonalisable sur $\C$ et $\Sp_{\C}(M)\subset\U_{m}$. Notons que $G$ étant abélien, toutes les matrices sont co-diagonalisables.

	Si $\Sp_{\C}(M)=\lbrace1\rbrace$, alors $M=I_{2}$. Si $\Sp_{\C}(M)=\lbrace-1\rbrace$, alors $M=-I_{2}$. Dans ces deux cas, on a $\det(M)=1$ et $\Tr(M)=\pm2$. On note ce cas 1.
	
	Si $\Sp_{\C}(M)=\lbrace-1,1\rbrace$, $M$ est semblable à $\begin{pmatrix}
		-1&0\\
		0&1
	\end{pmatrix}$ et $M^{2}=I_{2}$. Dans ce cas, on a $\det(M)=-1$ et $\Tr(M)=0$. On note ce cas 2.

	Notons que l'on a $\chi_{M}=X^{2}-\Tr(M)X+\det(M)$ et $\Delta=(\Tr(M))^{2}-4\det(M)$. Comme $\chi_{M}$ est un polynôme réel, si $\delta<0$, on écrit $\chi_{M}(X-\e^{\i\theta})(X-\e^{-\i\theta})$. Comme $\Tr(M)=2\cos(\theta)\in\Z$, on a $\theta\in\left\lbrace\frac{2\pi}{3},\frac{\pi}{2},\frac{\pi}{3}\right\rbrace$.

	Si $\theta=\frac{2\pi}{3}$, $M$ est semblable à 
	\begin{equation}
		\begin{pmatrix}
			\e^{\i\frac{2\pi}{3}}&0\\
			0&\e^{-\i\frac{2\pi}{3}}
		\end{pmatrix}
	\end{equation}
	et $M^{3}=I_{2}$. On a $\det(M)=1$ et $\Tr(M)=-1$. On note ce cas 3.

	Si $\theta=\frac{\pi}{2}$, $M$ est semblable à 
	\begin{equation}
		\begin{pmatrix}
			\e^{\i\frac{\pi}{2}}&0\\
			0&\e^{-\i\frac{\pi}{2}}
		\end{pmatrix}
	\end{equation}
	et $M^{4}=I_{2}$. On a $\det(M)=1$ et $\Tr(M)=0$. On note ce cas 4.

	Si $\theta=\frac{\pi}{3}$, $M$ est semblable à 
	\begin{equation}
		\begin{pmatrix}
			\e^{\i\frac{\pi}{3}}&0\\
			0&\e^{-\i\frac{\pi}{3}}
		\end{pmatrix}
	\end{equation}
	et $M^{6}=I_{2}$. On a $\det(M)=1$ et $\Tr(M)=1$. On note ce cas 5.

	Par ailleurs, il existe $P\in GL_{2}(\C)$ telle que pour tout $M\in G$, $P^{-1}MP=I_{2}$.

	S'il existe $M\in G$ du type 2, alors les types 3 et 5 sont exclus car on obtiendrait par produit $\begin{pmatrix}
		\e^{\i\theta}&0\\0&-\e^{\i\theta}
	\end{pmatrix}$ avec $\Tr(M)=0$ car $\chi_{m}$ est un polynôme réel, et $\theta\in\left\lbrace\frac{\pi}{3},\frac{2\pi}{3}\right\rbrace$, ce qui n'est pas.

	Ainsi, 
	\begin{equation}
		P^{-1}GP\subset\left\lbrace I_{2},-I_{2},\begin{pmatrix}
			-1&0\\0&1
		\end{pmatrix},\begin{pmatrix}
			1&0\\0&-1
		\end{pmatrix},\begin{pmatrix}
			\i&0\\0&-\i
		\end{pmatrix},\begin{pmatrix}
			-\i&0\\0&\i
		\end{pmatrix}\right\rbrace
	\end{equation}

	Ainsi, 
	\begin{equation}
		G=
		\left\lbrace
			\begin{array}[]{ll}
				\left\lbrace I_{2}\right\rbrace\\
				\left\lbrace -I_{2},I_{2}\right\rbrace\\
				\left\lbrace I_{2},B\right\rbrace& B\text{ matrice de type 2}\\
				\left\lbrace I_{2}, A,A^{2},A^{3}\right\rbrace\cong\Z/4\Z& A\text{ matrice de type 4}\\
				\left\lbrace I_{2},A,B,A^{2},A^{3},-B\right\rbrace\cong \Z/3\Z\times\Z/2\Z& A,B\text{ matrices de type 4, 2}\\
				\left\lbrace I_{2},B,-B,-I_{2}\right\rbrace\cong\left(\Z/2\Z\right)^{2} & B\text{ matrice de type 2}
			\end{array}
		\right.
	\end{equation}

	S'il n'y a pas de matrice de type 2, on a 
	\begin{equation}
		P^{-1}GP\subset\left\lbrace I_{2},-I_{2},\begin{pmatrix}
			\j&0\\0&\j^{2}
		\end{pmatrix},\begin{pmatrix}
			\i&0\\0&-\i
		\end{pmatrix},\begin{pmatrix}
			-\i&0\\0&\i
		\end{pmatrix},\begin{pmatrix}
			-\j&0\\0&-\j^{2}
		\end{pmatrix},\begin{pmatrix}
			-\j^{2}&0\\0&-\j
		\end{pmatrix}\right\rbrace
	\end{equation}

	On ne peut pas avoir une matrice de type 3 ou 5 car $\pm\i\j$ et $\pm\i\j^{2}$ ne sont pas des valeurs propres possibles. Donc 
	\begin{equation}
		G=
		\left\lbrace
		\begin{array}[]{ll}
			\left\lbrace I_{2},C,C^{2}\right\rbrace\cong\Z/3\Z & C\text{ matrice de type 3}\\
			\left\lbrace I_{2},D,D^{2},D^{3},D^{4},D^{5}\right\rbrace\cong\Z/6\Z& D\text{ matrice de type 5}
		\end{array}
		\right.
	\end{equation}
	Notons que dans le deuxième cas, $D^{2}$ est de type 3.

	On a donc bien $\left\lvert G\right\rvert\in\left\lbrace1,2,3,4,6\right\rbrace$.
\end{proof}

\begin{proof}
	Si $u$ est diagonalisable, la famille des vecteurs propres est génératrice. En prenant un sous-espace de $E$ de base $\mathcal{B}$, on complète avec des vecteurs propres, ce qui forme un sous-espace stable par u.

	Réciproquement, soit 
	\begin{equation}
		F=\bigotimes_{\lambda\in\Sp(u)}\ker(u-\lambda id)
	\end{equation}
	stable par $u$. $F$ admet un supplémentaire stable qu'on nommera $G$. Si $G\neq\lbrace0\rbrace$, $u_{\mid G}$ admet nécessairement un vecteur propre, or les vecteurs propres sont dans $F\setminus\lbrace0\rbrace$; absurde. Donc $G=\left\lbrace0\right\rbrace$ et $E=F$ donc $u$ est diagonalisable.
\end{proof}

\begin{proof}
	Plus généralement, soit $A=(a_{i,j})\in\M_{n}(\K)$ et $B=(b_{k,l})\in\M_{p}(\K)$. On définit 
	\begin{equation}
		M=B\otimes A=
		\begin{pmatrix}
			b_{1,1}A&\dots&b_{1,p}A\\
			\vdots&&\vdots\\
			b_{p,1}A&\dots&b_{p,p}A
		\end{pmatrix}
	\end{equation}

	Si $B$ est diagonalisable, il existe $Q\in GL_{p}(\K)$ tel que $Q^{-1}BQ=\diag(\mu_{1},dots,\mu_{p})$. On note $Q=(q_{i,j})$ et $Q^{-1}=(q'_{i,j})$. Par produits par blocs, on a 
	\begin{equation}
		\underbrace{\begin{pmatrix}
			q'_{1,1}I_{n}&\dots&q'_{1,p}I_{n}\\
			\vdots&&\vdots\\
			q'_{p,1}I_{n}&\dots&q'_{p,p}I_{n}
		\end{pmatrix}}_{=Q^{-1}\otimes I_{n}=(Q\otimes I_{n})^{-1}}M
		\underbrace{\begin{pmatrix}
			q_{1,1}I_{n}&\dots&q_{1,p}I_{n}\\
			\vdots&&\vdots\\
			q_{p,1}I_{n}&\dots&q_{p,p}I_{n}
		\end{pmatrix}}_{Q\otimes I_{n}}
		=\diag(\mu_{1}A,\dots,\mu_{p}A)=M_{1}
	\end{equation}

	Si de plus $A$ est diagonalisable, il existe $P\in GL_{n}(\K)$ tel que $P^{-1}AP=\diag(\lambda_{1},\dots,\lambda_{n})$. Alors 
	\begin{equation}
		\diag(P^{-1},\dots,P^{-1})M_{1}\diag(P,\dots,P)=\diag(\mu_{1},\lambda_{1},\dots,\mu_{1},\lambda_{n},\dots,\mu_{p}\lambda_{1},\dots,\mu_{p}\lambda_{n})
	\end{equation}

	Donc $B\otimes A$ est diagonalisable et $\Sp(B\otimes A)=\left\lbrace \lambda_{i}\mu_{j}\middle| 1\leqslant i\leqslant n,1\leqslant j\leqslant p\right\rbrace$.
\end{proof}

\begin{remark}
	On a $\Tr(B\otimes A)=\Tr(B)\times\Tr(A)$ et $\det(B\otimes A)=\det(B)^{n}\det(A)^{p}$.
\end{remark}
\begin{remark}
	Si $B$ est diagonalisable et non nulle, si $B\otimes A$ est diagonalisable, il existe $i\in\llbracket1,p\rrbracket$ tel que $\mu_{i}\neq0$ et $\mu_{i}A$ est diagonalisable (restriction à un sous-espace stable d'un endomorphisme diagonalisable) donc $A$ est diagonalisable.
\end{remark}

\begin{proof}
	\phantom{}
	\begin{enumerate}
		\item On note $(e_{1},\dots,e_{n})$ la base canonique de $\C^{n}$ et $u\in\L(\C^{n})$ canoniquement associé à $A$. 
		
		Si $n=2m$ avec $m\in\N^{*}$, on pose pour $k\in\llbracket1,m\rrbracket$, $F_{k}=\Vect(e_{k},\dots,e_{n-k+1})$. On a $u(e_{k})=x_{k}e_{n-k+1}$ et $u(e_{n-k+1})=x_{n-k+1}e_{k}$ donc $F_{k}$ est stable par $u$. Ainsi, $\mat_{(e_{k},e_{n-k+1})}(u_{\mid F_{k}})=\begin{pmatrix}
			0&x_{n-k+1}\\ x_{k}&0
		\end{pmatrix}$. 

		Étudions, pour tout $(a,b)\in\C^{2}$, $M=\begin{pmatrix}
			0&a\\b&0
		\end{pmatrix}$. On a $\chi_{M}=X^{2}-ab$. 
		
		Si $ab\neq0$, on note $\lambda$ une racine carrée de $ab$ sur $\C$. On a $\chi_{M}(X-\lambda)(X+\lambda)$ qui est scindé à racines simples donc $M$ est diagonalisable et semblable à $\diag(\lambda,-\lambda)$.

		Si $ab=0$: si $a=b=0$ alors $M=0$. Si $a$ ou $b\neq0$, alors $M$ n'est pas diagonalisable puisque si elle l'était, comme sa seule valeur propre est 0, elle serait semblable donc égale à 0 ce qui n'est pas. Donc $A$ est diagonalisable si et seulement si pour tout $k\in\llbracket1,m\rrbracket$, $x_{k}$ et $x_{n-k+1}$ non nuls ou $x_{k}=0=x_{n-k+1}$.

		Si $n=2m+1$, on fait le même raisonnement avec $u(e_{m})=x_{m}e_{m}$.

		\item $(A^{p})_{p\in\N}$ converge si et seulement si pour tout $k\in\llbracket1,m\rrbracket$, $\left(u_{\mid F_{k}}^{p}\right)_{p\in\N}$ converge. 
		
		Soit $k\in\llbracket1,m\rrbracket$. Si $x_{k}x_{n+1-k}\neq0$, soit $\lambda\in\C$ tel que $\lambda^{2}=x_{k}x_{n+1-k}$. Dans une base de vecteurs propres $\mathcal{B}$, on a 
		\begin{equation}
			\mat_{\mathcal{B}}(u_{\mid F_{k}}^{p})=\begin{pmatrix}
				\lambda^{p}&0\\0&(-\lambda)^{p}
			\end{pmatrix}
		\end{equation}
		Cela converge si et seulement si $\lvert\lambda\rvert<1$ si et seulement si $\left\lvert x_{k}x_{n+1-k}\right\rvert<1$.

		Si $x_{k}=x_{n+1-k}=0$, alors pour tout $p\geqslant2$, $u_{\mid F_{k}}^{p}=0$ donc on a convergence.
	\end{enumerate}
\end{proof}

\begin{proof}
	\phantom{}
	\begin{enumerate}
		\item On a 
		\begin{equation}
			\mat_{(1,X,\dots,X^{n})}(\varphi)=\diag(-n,1-n,\dots,0)
		\end{equation}
		donc les valeurs propres sont $(k-n)_{k\in\llbracket0,n\rrbracket}$. On a $n+1$ valeurs propres distinctes donc $\varphi$ est diagonalisable et les sous-espaces propres sont de dimension 1. Les vecteurs propres sont les $(X^{k})$ pour $k\in\llbracket1,n\rrbracket$.

		\item Si $\varphi(P)=kP$, alors $\deg(P)=k$. Si $P=\sum_{i=0}^{k}a_{i}X^{i}$ donc 
		\begin{equation}
			XP'-nP''-kP=0=\sum_{i=0}^{k-2}\left(ia_{i}-n(i+1)(i+2)-ka_{i}\right)X^{i}-a_{k-1}X^{k-1}
		\end{equation}

		Ainsi, par récurrence, pour tout $p\in\left\lbrace0,\dot,\left\lfloor\frac{k-1}{2}\right\rfloor\right\rbrace$, $a_{k-(2p+1)}=0$ et pour tout $p\in\left\lbrace0,\dots,\left\lfloor\frac{k}{2}\right\rfloor\right\rbrace$,
		\begin{equation}
			a_{k-2p}=\frac{n^{p}(k-2p+1)\dots(k-1)k}{(-2p)\dots(-4)(-2)}a_{k}=\frac{n^{p}(-1)^{p}}{2^{p}p!}\times \frac{k!}{(k-2p)!}a_{k}
		\end{equation}
		donc les vecteurs propres correspondent aux polynômes ayant ces coefficients.
	\end{enumerate}
\end{proof}

\begin{proof}
	\phantom{}
	\begin{enumerate}
		\item On a 
		\begin{equation}
			A=aI_{3}+c\underbrace{\begin{pmatrix}
				0&0&1\\1&0&0\\0&1&0
			\end{pmatrix}}_{C}+b\underbrace{\begin{pmatrix}
				0&1&0\\0&0&1\\1&0&0
			\end{pmatrix}}_{B}
		\end{equation}
		avec $B=C^{2}$ et $C$ est la matrice compagnon du polynôme $X^{3}-1$, donc $\chi_{C}=X^{3}-1$. Ainsi, $C$ est diagonalisable sur $\C$ et $\Sp_{\C}(C)=\left\lbrace1,\j,\j^{2}\right\rbrace$. On a 
		\begin{equation}
			\begin{array}[]{l}
				C\begin{pmatrix}
					1\\1\\1
				\end{pmatrix}=\begin{pmatrix}
					1\\1\\1
				\end{pmatrix}\\
				C\begin{pmatrix}
					1\\\j^{2}\\\j
				\end{pmatrix}=\j\begin{pmatrix}
					1\\\j^{2}\\\j
				\end{pmatrix}\\
				C\begin{pmatrix}
					1\\\j\\\j^{2}
				\end{pmatrix}=\j^{2}\begin{pmatrix}
					1\\\j\\\j^{2}
				\end{pmatrix}\\
			\end{array}
		\end{equation}
		donc si 
		\begin{equation}
			P=\begin{pmatrix}
				1&1&1\\ 1&\j^{2}&\j\\1&\j&\j^{2}
			\end{pmatrix}
		\end{equation}
		alors on a 
		\begin{equation}
			P^{-1}AP=\begin{pmatrix}
				1&0&0\\
				0&a+b\j^{2}+c\j&0\\
				0&0&a+b\j+c\j^{2}
			\end{pmatrix}
		\end{equation}
		et 
		\begin{equation}
			\boxed{
				\Sp_{\C}(A)=\left\lbrace1,a+c\j+b\j^{2},a+b\j+c\j^{2}\right\rbrace
			}
		\end{equation}

		\item On a 
		\begin{equation}
			A^{n}=P\begin{pmatrix}
				1&0&0\\
				0&(a+b\j^{2}+c\j)^{n}&0\\
				0&0&(a+b\j+c\j^{2})^{n}
			\end{pmatrix}P^{-1}
		\end{equation}

		Tout d'abord, on a $\left\lvert a+c\j+b\j^{2}\right\rvert\leqslant\left\lvert a\right\rvert+\left\lvert b\right\rvert+\left\lvert c\right\rvert=1$ et on a égalité si et seulement si $a,c\j$ et $b\j^{2}$ ont le même argument si et seulement si 
		\begin{equation*}
			\left\lbrace a,b,c\right\rbrace\in\left\lbrace\left\lbrace1,0,0\right\rbrace,\left\lbrace0,1,0\right\rbrace,\left\lbrace0,0,1\right\rbrace\right\rbrace.
		\end{equation*}
		Si $a=1$ et $b=c=0$, la suite $(A^{n})_{n\in\N}$ converge.
		Si $(b=1$ et $a=c=0$) ou ($c=1$ et $a=b=0$), $(A^{n})_{n\in\N}$ diverge.
		Sinon,
		\begin{equation}
			\boxed{
			A^{n}\xrightarrow[n\to+\infty]{}P\begin{pmatrix}
				1&0&0\\0&0&0\\0&0&0
			\end{pmatrix}P^{-1}}
		\end{equation}
	\end{enumerate}
\end{proof}

\begin{proof}
	\phantom{}
	\begin{enumerate}
		\item On vérifie en calculant les premiers termes puis par récurrence que 
		\begin{equation}
			\boxed{A^{k}=\begin{pmatrix}
				B^{k}&\sum_{i=1}^{k}B^{i-1}CD^{k-i}\\0&D^{k}
			\end{pmatrix}}
		\end{equation}

		\item On a 
		\begin{equation}
			\mu_{A}(A)=0=\begin{pmatrix}
				\mu_{A}(B)&\star\\0&\mu_{A}(D)
			\end{pmatrix}
		\end{equation}
		donc $\mu_{A}(B)=\mu_{A}(D)=0$. Ainsi, $\mu_{B}\mid \mu_{A}$ et $\mu_{D}\mid\mu_{A}$ donc $\mu_{B}\vee\mu_{D}\mid \mu_{A}$.

		Si $B$ et $D$ sont de tailles 1, $\chi_{A}(X)=(X-b)(X-d)$ d'où $\mu_{B}=X-b$ et $\mu_{D}=X-d$.

		Si $b\neq d$, $X-b$ et $X-d$ divise $\mu_{A}$ donc $\mu_{A}\mid \chi_{A}$ d'après le théorème de Cayley-Hamilton.

		Si $b=d$, si $c=0$ on a $\mu_{A}=X-b$ donc $\mu_{A}\mid\mu_{B}\mu_{D}=(X-b)^{2}$. Si $c\neq0$, $\mu_{A}=X-b$ ou $\mu_{A}=(X-b)^{2}$ et $A-bI_{n}\neq0$ donc $\mu_{A}=(X-b)^{2}$.

		\item Si $C=0$, on a $\mu_{A}=\mu_{B}\vee\mu_{D}$.
		\item Si $B=D$ et $C=I_{n_{1}}$, on a $A^{0}=I_{n_{1}+n_{2}}$ et pour $k\geqslant1$,
		\begin{equation}
			A^{k}=\begin{pmatrix}
				B^{k}&kB^{k-1}\\
				0&B^{k}
			\end{pmatrix}
		\end{equation}
		Ainsi
		\begin{equation}
			P(A)=\begin{pmatrix}
				P(B)&P'(B)\\
				0&P(B)
			\end{pmatrix}=0
		\end{equation}
		si et seulement si $P(B)=P'(B)=0$ donc $\mu_{B}\mid P$ et $\mu_{B}\mid P'$ donc $\mu_{B}\mid P\vee P'$.

		On a 
		\begin{equation}
			\mu_{B}^{2}(A)=\begin{pmatrix}
				0&2\mu_{B}\mu_{B}'(B)\\
				0&0
			\end{pmatrix}=0
		\end{equation}
		donc $\mu_{A}\mid\mu_{B}^{2}$.

		On décompose $\mu_{B}=(X-\lambda_{1})^{m_{1}}\times\dots\times(X-\lambda_{r})^{m_{r}}$. On a $P(A)=0$ si et seulement si pour tout $i\in\left\lbrace1,\dots,r\right\rbrace$, $\lambda_{i}$ est racine de $P$ d'ordre plus grand que $m_{i}+1$.

		\item On prend 
		\begin{equation}
			A=\begin{pmatrix}
				0&1&0&0\\
				0&0&0&1\\
				0&0&0&1\\
				0&0&0&0
			\end{pmatrix}
		\end{equation}
		On a 
		\begin{equation}
			B=D=\begin{pmatrix}
				0&1\\0&0
			\end{pmatrix}
		\end{equation}
		$\mu_{B}=\mu_{D}=X^{2}$, et $\mu_{A}\mid X^{3}$. Or $u^{2}(e_{4})\neq0$ donc $\mu_{A}=X^{3}\neq X^{2}=\mu_{B}\vee\mu_{D}\neq X^{4}=\mu_{B}\mu_{D}$.
	\end{enumerate}
\end{proof}

\begin{proof}
	On décompose sur $\C$: $P(X)-\lambda=\alpha\prod_{i=1}^{m}(X-\mu_{i})$ avec $\alpha\neq0$. On a 
	\begin{equation}
		\underbrace{g-\lambda id}_{\substack{\text{non inversible}\\\text{(respectivement non injectif)}}}=\alpha\prod_{i=1}^{m}(f-\mu_{i}id)
	\end{equation}
	donc il existe $i\in\llbracket1,m\rrbracket$, $f-\mu_{i}id$ est non inversible (respectivement non injectif), d'où le résultat.
\end{proof}

\begin{proof}
	\phantom{}
	\begin{enumerate}
		\item Soit $(f_{1},\dots,f_{r})$ une base de $V$ et $x\in E\setminus\lbrace0\rbrace$. Soit $(\alpha_{1},\dots,\alpha_{r})\in\K^{r}$ tels que $\alpha_{1}f_{1}(x)+\dots+\alpha_{r}f_{r}(x)=0$. Or $V$ est un sous-espace, donc $\alpha_{1}f_{1}+\dots+\alpha_{r}f_{r}\in V$ et $\alpha_{1}f_{1}+\dots+\alpha_{r}f_{r}\notin GL(E)$ car $x\neq0$. D'où $\alpha_{1}f_{1}+\dots+\alpha_{r}f_{r}=0$ et donc $\alpha_{1}=\dots=\alpha_{r}=0$ car $(f_{1},\dots,f_{r})$ est une base de $V$.
		
		Ainsi $(f_{1}(x),\dots,f_{r}(x))$ est libre donc $r=\dim(V)\leqslant\dim(E)$.

		\item Si $\dim(V)=1$, alors $V=\C f$ avec $f\in GL(E)$. Si $\dim(V)\geqslant2$, soient $(f,g)\in V^{2}$ tels que $(f,g)$ soit libre alors pour tout $\alpha\in\C$, $f+\alpha g\neq0$ et $f+\alpha g\in V\setminus\lbrace0\rbrace$. Or $f+\alpha g=g(g^{-1}\circ f+\alpha id)$. Pour $\alpha\in \Sp(-g^{-1}\circ f)$ (existe car on est dans $\C$), on obtient une contradiction. Donc de même, $V=\C f$ avec $f\in GL(E)$.
		
		\item Comme $\dim(E)=2$, on a $\dim(V)\leqslant2$. Si $\dim(V)=1$, comme précédemment, on a $V=\R f$ avec $f\in GL(E)$. Si $\dim(V)=2$, soit $(f,g)$ une base de $V$. D'après ce qui précède, on a $\Sp_{\R}(g^{-1}\circ f)=\emptyset$. Soit $\mathcal{B}$ une base de $E$ et $A=\mat_{\mathcal{B}}(f)$ et $B=\mat_{\mathcal{B}}(g)$. On écrit $\chi_{A^{-1}B}=(X-\lambda)(X-\overline{\lambda})$ avec $\lambda=\alpha+\i \beta$, $\beta>0$. $A^{-1}B$ est diagonalisable sur $\C$ et semblable à $\diag(\lambda,\overline{\lambda})$ et $\frac{A^{-1}B-\alpha I_{2}}{\beta}$ est semblable sur $\C$ à $\diag(\i,-\i)$ semblable sur $\R$ à $\begin{pmatrix}
			0&-1\\1&0
		\end{pmatrix}$ donc $\frac{A^{-1}B-\alpha I_{2}}{\beta}$ est semblable sur $\R$ à $\begin{pmatrix}
			0&-1\\1&0
		\end{pmatrix}$. Il existe $P\in GL_{2}(\R)$ tel que 
		\begin{equation}
			A^{-1}B=P\begin{pmatrix}
				\alpha&-\beta\\\beta&\alpha
			\end{pmatrix}P^{-1}
		\end{equation}

		Pour tout $(\lambda,\mu)\in\R^{2}$, 
		\begin{equation}
			\lambda A+\mu B=A(\lambda I_{2}+\mu A^{-1}B)=\underbrace{AP}_{\in GL_{2}(\R)}\begin{pmatrix}
				\lambda+\alpha\mu&-\beta\\\beta&\lambda+\alpha\mu
			\end{pmatrix}\underbrace{P^{-1}}_{\in GL_{2}(\R)}
		\end{equation}
		avec $\beta>0$,$\alpha\in\R$.

		\item Avec les notations précédentes, si $(A,B)\in V^{2}$ est libre, on a 
		\begin{equation}
			\i\in\Sp\left(\frac{A^{-1}B-\alpha I_{2}}{\beta}\right)=\Sp\left((BA)^{-1}(B-\alpha A)\right)
		\end{equation}
		et on pose $A'=\beta A\in V$ et $B'=B-\alpha A\in V$.
	\end{enumerate}
\end{proof}

\begin{proof}
	\phantom{}
	\begin{enumerate}
		\item En notant $c_{i,j}$ les cofacteurs d'indice $(j,i)$, on a 
		\begin{equation}
			\left[\com(\lambda I_{n}-A)^{\mathsf{T}}\right]_{i,j}=(-1)^{i+j}c_{j,i}(\lambda-I_{n})
		\end{equation}

		En développant, on obtient des polynômes en $\lambda$ de degré plus petit que $n-1$. En regroupant selon les puissances de $\lambda$, on a 
		\begin{equation}
			\boxed{\com(\lambda I_{n}-A)^{\mathsf{T}}=M_{0}+M_{1}\lambda+\dots+\lambda^{n-1}M_{n-1}}
		\end{equation}

		\item On a $(\lambda I_{n}-A)\com(\lambda I_{n}-A)^{\mathsf{T}}=\det(\lambda I_{n}-A)I_{n}=\chi_{A}(\lambda)I_{n}$
		
		En identifiant les coefficients, on a 
		\begin{equation}
			\left\lbrace
				\begin{array}[]{lll}
					M_{n-1}&=&I_{n}\\
					M_{n-2}&=&A+a_{n-1}I_{n}\\
					M_{n-3}&=&A^{2}+a_{n-1}A+a_{n-2}I_{n}\\
					\vdots\\
					M_{n-k}&=&A^{k-1}+a_{n-1}A^{k-2}+\dots+a_{n-k+1}I_{n}\\
					\vdots\\
					M_{0}&=&A^{n-1}+a_{n-1}A^{n-2}+\dots+a_{1}I_{n}
				\end{array}
			\right.
		\end{equation}
		et $-AM_{0}=a_{0}I_{n}$. En reportant, on a bien $\chi_{A}(A)=O_{\M_{n}(\K)}$: on a une preuve du théorème de Cayley-Hamilton.

		\item Soit $\lambda\in\K$. On forme $\lambda I_{n}-A=(c_{1}(\lambda),\dots,c_{n}(\lambda))$ avec 
		\begin{equation}
			c_{j}(\lambda)=\begin{pmatrix}
				-a_{1,j}\\\dots\\-a_{j-1,j}\\\lambda-a_{j,j}\\-a_{j+1,j}\\\dots\\-a_{n,j}
			\end{pmatrix}
		\end{equation}

		On a $\chi_{A}(\lambda)=\det(c_{1}(\lambda),\dots,c_{n}(\lambda))$. $\det$ étant une forme $n$-linéaire, on a 
		\begin{equation}
			\chi_{A}'(\lambda)=\sum_{k=1}^{n}\det(c_{1}(\lambda),\dots,c_{k-1}(\lambda),c_{k}'(\lambda),c_{k+1}(\lambda),\dots,c_{n}(\lambda))
		\end{equation}
		En développant le terme $k$ par rapport à la $k$-ième colonne, on trouve qu'il vaut $_{k,k}(\lambda I_{n}-A)$. Ainsi,
		\begin{equation}
			\boxed{\chi_{A}'(\lambda)=\Tr(\com(\lambda I_{n}-A)^{\mathsf{T}})}
		\end{equation}

		\item On a donc $a_{1}+2a_{2}\lambda+\dots+(n-1)a_{n-1}\lambda^{n-2}+n\lambda^{n-1}=\sum_{k=0}^{n-1}\lambda^{k}\Tr(M_{k})$ pour tout $\lambda\in\K$ (par linéarité de $\Tr$). Donc pour tout $k\in\llbracket0,n-2\rrbracket$, $\Tr(M_{k})$, $\Tr(M_{k})=(k+1)a_{k+1}$ (et $\Tr(M_{n-1})=\Tr(I_{n})=n$). On a $\Tr(M_{n-2})=(n-1)a_{n-1}=\Tr(A)+na_{n-1}$ donc $a_{n-1}=\Tr(A)$. Puis $\Tr(M_{n-3})=(n-2)a_{n-2}=\Tr(A^{2})+a_{n-1}\Tr(A)+a_{n-2}n$ donc $a_{n-2}=-\frac{\Tr(A^{2})}{2}+\frac{\Tr(A)^{2}}{2}$. De proche en proche, on a $a_{n-k}=f_{k}(\Tr(A),\dots,\Tr(A^{k}))$ avec $f_{k}$ indépendante de $A$.
		
		\item D'après ce qui précède, pour tout $k\in\llbracket0,n-1\rrbracket$, $a_{k}=b_{k}$ car $f$ est indépendante de $A$. Donc $\chi_{A}=\chi_{B}$.
	\end{enumerate}
\end{proof}

\begin{remark}
	Si $\Tr(A)=\Tr(A^{2})=\dots=\Tr(A^{n})=0$, alors $\chi_{A}=\chi_{0}=X^{n}$ et $A$ est nilpotente. On peut le vérifier à la main sur $\C$: si $(\lambda_{1},\dots,\lambda_{r})$ sont les valeurs propres non nulles distinctes de $A$ et $m_{i}$ la multiplicité de $\lambda_{i}$ dans $\chi_{A}$, alors on a le système
	\begin{equation}
		\left\lbrace
			\begin{array}[]{lllll}
				\Tr(A)&=&m_{1}\lambda_{1}+\dots+m_{r}\lambda_{r}&=&0\\
				\Tr(A^{2})&=&m_{1}\lambda_{1}^{2}+\dots+m_{r}\lambda_{r}^{2}&=&0\\
				\vdots\\
				\Tr(A^{r})&=&m_{1}\lambda_{1}^{r}+\dots+m_{r}\lambda_{r}^{r}&=&0
			\end{array}
		\right.
	\end{equation}
	donc 
	\begin{equation}
		\begin{pmatrix}
			\lambda_{1}&\dots&\lambda_{r}\\
			\vdots&&\vdots\\
			\lambda_{1}^{r}&\dots&\lambda_{r}^{r}
		\end{pmatrix}\begin{pmatrix}
			m_{1}\\\vdots\\m_{r}
		\end{pmatrix}=0
	\end{equation}
	et la matrice est inversible car les $\lambda_{i}$ sont distincts non nuls. Donc $m_{1}=\dots=m_{r}=0$ et $\Sp_{\C}(A)=\lbrace0\rbrace$ et $\chi_{A}=X^{n}$.
\end{remark}

\begin{proof}
	On définit, pour $A=(a_{i,j})\in\M_{n}(\Z)$, $\overline{A}=(\overline{a_{i,j}})\in\M_{n}\left(\Z/p\Z\right)$. Comme $p$ est premier, $\Z/p\Z$ est un corps. On a $\chi_{A}\in\Z/p\Z[X]$ donc il existe $\mathbb{L}$ un sur-corps de $\Z/p\Z$ où $\chi_{A}$ est scindé sur $\mathbb{L}$. On écrit $\chi_{A}=(X-\lambda_{1})\dots(X-\lambda_{n})$ avec $\lambda_{1},\dots,\lambda_{n}\in\mathbb{L}$. On peut trigonaliser $\overline{A}$ sur $\mathbb{L}$ et on a $\Tr(\overline{A}^{p})=\sum_{i=1}^{n}\lambda_{i}^{p}$. Or la caractéristique de $\mathbb{L}$ vaut $p$ donc on a $(x+y)^{p}=x^{p}+y^{p}$ (binôme de Newton et utiliser le fait que $p\mid\binom{p}{k}$ pour $k\in\llbracket1,p-1\rrbracket$). Ainsi,
	\begin{equation}
		\Tr(\overline{A}^{p})=\left(\sum_{i=1}^{n}\lambda_{i}\right)^{p}=\Tr(\overline{A})^{p}
	\end{equation}
	et on peut appliquer le petit théorème de Fermat: on a bien $\Tr(\overline{A}^{p})=\Tr(\overline{A})$ et en remontant dans $\Z$, 
	\begin{equation}
		\boxed{\Tr(A^{p})\equiv\Tr(A)[p]}
	\end{equation}
\end{proof}

\begin{proof}
	Si on a (i), soit $x$ un vecteur propre associé à $\rho(u)=\rho e^{\mathrm{i}\theta}$. On a $\Vert u(x)\Vert=\Vert\rho(u) x\Vert=\rho(u)\Vert x\Vert$ et comme $x\neq0$, on a $\rho(u)\leqslant\vertiii{\rho(u)}<1$ d'où (ii).

	Si (ii), on utilise la décomposition de Dunford $u=n+d$ avec $n$ nilpotent, $d$ diagonalisable et $dn=nd$. Soit $m=\dim(E)$. Pour tout $p\geqslant m$, on a 
	\begin{equation}u^{p}=\sum_{k=0}^{p}\binom{p}{k}n^{k}d^{p-k}=\sum_{k=0}^{m-1}\binom{p}{k}n^{k}\underbrace{d^{p-k}}_{\xrightarrow[p\to+\infty]{}0}\end{equation}
	En effet, on a $k\geqslant m-1$ fixé, il existe une base $\mathcal{B}$ de $E$ telle que 
	\begin{equation}\binom{p}{k}\mat\limits_{\mathcal{B}}(d^{p})=\binom{p}{k}\diag(\lambda_{1}^{p},\dots,\lambda_{m}^{p})\xrightarrow[p\to+\infty]{}0\end{equation}
	car $\vert\lambda_{i}\vert<1$ pour tout $i\in\{1,\dots,m\}$ et 
	\begin{equation}\binom{p}{k}\underset{p\to+\infty}{\sim}\frac{p^{k}}{k!}=\underset{p\to+\infty}{o}\Bigl(\frac{1}{\rho(u)^{p}}\Bigr)\end{equation}
	donc on a (iii).

	Si (iii), soit $x$ un vecteur propré associé à $\lambda\in\C$, on a $u^{p}\xrightarrow[p\to+\infty]{}0$ donc en particulier, $u^{p}(x)=\lambda^{p}\xrightarrow[p\to+\infty]{}0$, donc $\rho(u)^{p}\xrightarrow[p\to+\infty]{}0$ et $\rho(u)\geqslant0$ donc $\rho(u)<1$. Posons encore $u=d+n$ la décomposition de Dunford de $u$. Soit $\varepsilon>0$, il existe $\mathcal{B}_{0}=(e_{1},\dots,e_{n})$ base de $E$ dans laquelle les coefficients de $\mat\limits_{\mathcal{B}_{0}}(n)$ sont en module $\leqslant\varepsilon$. Définissons sur $E$ 
	\begin{equation}\Biggl\Vert\sum_{i=1}^{m}x_{i}e_{i}\Biggr\Vert_{\infty}=\max\limits_{1\leqslant i\leqslant m}\vert x_{i}\vert\end{equation}
	Soit $M=\mat\limits_{\mathcal{B}_{0}}(u)=(m_{i,j})_{1\leqslant i,j\leqslant m}$ triangulaire supérieure avec $m_{ii}=\lambda_{i}$ et pour tout $j\neq i$, $\vert m_{i,j}\vert<\varepsilon$. Soit donc $x=\sum_{i=1}^{m}x_{i}e_{i}\in\C^{m}$, on a 
	\begin{equation}
	\Vert Mx\Vert_{\infty}=\max\limits_{1\leqslant i\leqslant n}\underbrace{\Biggl\vert\sum_{j=1}^{m}m_{i,j}x_{j}\Biggr\vert}_{(\vert\lambda_{i}\vert+(m-1)\varepsilon)\Vert x\Vert_{\infty}}
	\end{equation}
	donc 
	\begin{equation}\vertiii{u}\leqslant\underbrace{\rho(u)}_{<1}+(m-1)\varepsilon\end{equation}
	et on choisit
	\begin{equation}\varepsilon<\frac{1-\rho(u)}{\underbrace{m-1}_{>0}}\end{equation}
	d'où $\vertiii{u}<1$ et donc on a (i) et finalement on a bien l'équivalence.
\end{proof}

\begin{remark}
	$u\mapsto\rho(u)$ n'est pas une norme car pour $u$ nilpotente non nulle, $\rho(u)=0$.
\end{remark}

\begin{proof}
	Supposons (i), soit $Y$ un vecteur propre de $A$ avec $AY=\lambda Y$ pour $\lambda\in\C$. Pour tout $k\in\N,BA^{k}Y=\lambda^{k}BY$ et il existe $k_{0}\in\N$ tel que $\lambda^{k_{0}}BY\neq0$ et $BY\neq0$ donc on a (ii).

	Si (ii), supposons qu'il existe $Y\in\C^{n}\setminus\{0\}$ tel que $\varphi=0$. On note 
	\begin{equation}\chi_{A}=\prod_{i=1}^{r}(X-\lambda_{i})^{m_{i}}\end{equation} avec les $\lambda_{i}$ distincts. Alors $Y=\sum_{i=1}^{r}Y_{i}$ où $Y_{i}\in\ker(A-\lambda_{i}I_{n})$. Il existe $i_{0}\in\{1,\dots,n\}$ tel que $Y_{i_{0}}\neq0$ car $Y\neq0$. On a alors, pour $t\in\R$,
	\begin{equation}B\exp(tA)Y=\sum_{i=1}^{r}B\exp(t\lambda_{i})Y_{i}=0\end{equation}
	Pour tout $k\in\{0,\dots,r-1\}$, on a $\varphi^{(k)}(t)=\sum_{i=1}^{r}B\lambda_{i}^{k}\exp(t\lambda_{i})Y_{i}=0$. Pour $t=0$ on a $\sum_{i=1}^{r}\lambda_{i}^{k}BY_{i}=0$ ce qui, pour $t=0$, donne le système 
	\begin{equation}
	\left\{
		\begin{array}[]{lll}
			BY_{1}+\dots+BY_{r} &= &0\\
			\lambda_{1}BY_{1}+\dots+\lambda_{r}BY_{r} &=& 0\\
			&\vdots&\\
			\lambda_{1}^{r-1}BY_{1}+\dots+\lambda_{r}^{r-1}BY_{r} &= &0
		\end{array}
	\right.
	\end{equation}
	Pour tout $P\in\C_{r-1}[X]$, on a donc $\sum_{i=1}^{r}P(\lambda_{i})BY_{i}=0$. Pour $i\in\{0,\dots,r-1\}$ et $P=\prod_{i\neq j}\frac{(X-\lambda_{j})}{\lambda_{i}-\lambda_{j}}$, on obtient pour tout $i\in\{1,\dots, r\}, BY_{i}=0$. En particulier, $BY_{i_{0}}=0$ et $Y_{i_{0}}$ est un vecteur propre de $A$ car non nul. C'est une contradiction. On a donc (iii).

	\item Soit $Y\in\C^{n}\setminus\{0\}$, supposons que pour tout $k\in\{0,\dots,n-1\}$, $BA^{k}Y=0$. Soit $k\geqslant n$, il existe $(Q_{k},R_{k})\in\C[X]\times\C_{n-1}[X]$ tel que 
	\begin{equation}X^{k}=Q_{k}\chi_{A}+R_{k}\end{equation}
	et le théorème de Cayley-Hamilton donne donc $A^{k}=R_{k}(A)$ d'où $BA^{k}Y=BR_{k}(A)Y=0$. Alors pour tout $t\in\R$,
	\begin{align}
		B\exp(tA)Y
		&=B\sum_{k=0}^{+\infty}\frac{t^{k}A^{k}}{k!}Y\\
		&=\sum_{k=0}^{+\infty}\frac{t^{k}(BA^{k}Y)}{k!}\\
		&=0
	\end{align}
	Par contraposée, on a bien ce qu'il faut, d'où l'équivalence.
\end{proof}

\begin{proof}
	Noter d'abord que le résultat ne dépend pas de la norme (équivalente).
	Si $A=\lambda I_n$ on vérifie que $\left\lVert A^{p}\right\rVert^{\frac{1}{p}}\xrightarrow[p\to+\infty]{}\left\lvert\lambda\right\rvert$. On choisit $\left\lVert A\right\rVert=\sup_{\left\lVert X\right\rVert_{\infty}=1}\left\lVert AX\right\rVert_{\infty}$ et on vérifie que si $A$ est diagonalisable avec $\left\lvert \lambda_{1}\right\rvert\leqslant\left\lvert\lambda_{2}\right\rvert$ ses valeurs propres (complexes), alors le résultat est $\left\lvert \lambda_{2}\right\rvert$. Si $A$ est juste trigonalisable (son spectre contient juste $\lambda\in\C$), on écrit $A$ comme somme d'une matrice scalaire et d'une matrice nilpotente, et on calcule explicitement $\left\lVert A^p\right\rVert$. Le résultat est alors $\left\lvert\lambda\right\rvert$, et de manière générale, il s'agit du rayon spectral. En dimension quelconque, on utilise la décomposition de Dunford.
\end{proof}

\begin{proof}\phantom{}
	\begin{enumerate}
		\item $M$ est la matrice compagnon de $P$, donc $\chi_{M}=P$. Si $\lambda\in\Sp_{\K}(M)$, $\rg(M-\lambda I_n)=n-1$ car les $n-1$ premières lignes sont indépendantes. Ainsi, $\dim(E_{\lambda})=1$. Pour la condition nécessaire et suffisante, $M$ est diagonalisable et si et seulement si $P$ est scindé à racines simples.
		\item On pose $Y=\begin{pmatrix}
			y\\y'\\y''
		\end{pmatrix}$ et 
		\begin{equation*}
			A=\begin{pmatrix}
				0&1&0\\
				0&0&1\\
				-1&1&1
			\end{pmatrix}.
		\end{equation*}

		La solution générale est alors $Y(t)=\e^{t A}Y_{0}$. $\chi_{A}=(X-1)^{2}(X+1)$, donc $A$ n'est pas diagonalisable sur $\R$ mais trigonalisable. On trigonalise, par exemple
		\begin{equation*}
			\begin{pmatrix}
				1&1&-1\\-1&1&0\\1&1&1
			\end{pmatrix},
		\end{equation*}
		et on a $P^{-1}AP=A_1=\begin{pmatrix}
			-1&0&0\\0&1&1\\0&0&1
		\end{pmatrix}$. On peut alors calculer $\exp(tA_{1})$.
	\end{enumerate}
\end{proof}

\begin{proof}
	\phantom{}
	\begin{enumerate}
		\item On utilise le fait que $\mathcal{B}$ est une base.
		\item On calcule le polynôme caractéristique qui vaut $P$. Comme $\mathcal{B}$ est une base, $P$ est nécessairement le polynôme minimal de $f$. Comme $X^{n}-1$ annule $f$, $P\mid X^{n}-1$. Comme $X^{n}-1$ est scindé à racines simples sur $\C$, $f$ est diagonalisable sur $\C$, et les valeurs propres réelles de $f$ sont dans $\left\lbrace-1,1\right\rbrace$. Le polynôme minimal de $f$ est de degré 3, donc il n'est pas scindé à racines simples sur $\R$. Donc $f$ n'est pas diagonalisable sur $\R$.
		\item $f^{n}=\mathrm{id}_{\R^{3}}$ donc $f$ est inversible. Comme $\chi_{f}=P=X^{3}-cX^{2}-bX-a$, $\det(f)=a\neq0$ donc $f^{-1}=\frac{1}{a}(f^{2}-cf-b\mathrm{id}_{\R^{3}})$.
		\item $\deg(\chi_{f})=3$ donc d'après le théorème des valeurs intermédiaires, il existe $\lambda\in\R$ telle que $\chi_{f}(\lambda)=0$. S'il existe $\mu\in\R$ avec $\mu\neq\lambda$ tel que $\chi_{f}(\mu)=0$, alors la troisième racine est réelle (car si elle était complexe, on aurait également son conjugué). Mais Le spectre de $f$ sur $\R$ est contenu dans $\left\lbrace-1,1\right\rbrace$, c'est absurde. Donc $\Sp_{\R}(f)=\left\lbrace\lambda\right\rbrace$.
		\item Les racines de $\chi_{f}$ sur $\C$ sont $\left\lbrace\lambda,\e^{\i\theta},\e^{-\i\theta}\right\rbrace$ avec $\theta\in]0,\pi[$ et $\e^{\i\theta}\in\U_{n}$ (racine de $X^{n}-1$). Donc $\det(f)=\lambda>0$ (respectivement $<0$). Donc $\lambda=1$ (respectivement -1). Si $\det(f)>0$, on a 
		\begin{equation*}
			\chi_{f}=(X-1)(X^{2}-2\cos\theta X+1)=X^{3}-cX^{2}-bX-a,
		\end{equation*}
		d'où $a=1$, $b=-(1+2\cos\theta)$, $c=(1+2\cos\theta)$.
	\end{enumerate}
\end{proof}

\begin{remark}
	Pour $n\geqslant3$, $r=\begin{pmatrix}
		1&\\&R_{\frac{2\pi}{n}}
	\end{pmatrix}$ vérifie l'hypothèse.
\end{remark}
\end{document}
\documentclass[12pt]{article}
\usepackage{style/style_sol}

\begin{document}

\begin{titlepage}
	\centering
	\vspace*{\fill}
	\Huge \textit{\textbf{Solutions MP/MP$^*$\\ Espaces vectoriels normés}}
	\vspace*{\fill}
\end{titlepage}

\begin{proof}
	\phantom{}
	\begin{enumerate}
		\item A $(x,y)\in\R^{2}$ fixé, la fonction \function{\varphi}{\R}{\R}{t}{x\cos(t)+y\sin(2t)}
		est bornée, donc le $\sup$ sur $\R$ existe. Pour la séparation, prendre $t=0$ et $t=\frac{\pi}{4}$. Pour l'inégalité triangulaire, montrer l'inégalité à $t$ fixé puis passer au $\sup$ sur $\R$.
		
		\item Si $\vert x\vert+\vert y\vert\leqslant1$, alors $N(x,y)\leqslant 1$ donc on a la première inclusion. 
		
		Si $N(x,y)\leqslant 1$, utiliser $t=0$ pour avoir $\vert x\vert\leqslant1$ et $t=\frac{\pi}{4}$ puis $t=-\frac{\pi}{4}$ pour pouvoir justifier
		\begin{equation}\vert 2y\vert\leqslant \Biggl\vert x\frac{\sqrt{2}}{2}+y\Biggr\vert+\Biggl\vert y-x\frac{\sqrt{2}}{2}\Biggr\vert\leqslant 2\end{equation}
		et donc $\vert y\vert\leqslant1$. D'où la deuxième inclusion. 

		\item On fixe $(x,y)\in S_{N}(0,1)\cap(\R_{+})^{2}$. $\varphi$ est $2\pi$-périodique, $\varphi(\pi-t)=\varphi(t)$ et $\sup\limits_{t\in\R}\vert\varphi(t)\vert=1$. On peut donc se limite à un intervalle de longueur $2\pi$ pour l'étude de $\varphi$. 
		
		On note que si $t\in[-\pi,0]$, $\cos(t)$ et $\sin(2t)$ sont de signes opposés. Donc
		\begin{equation}\vert\varphi(t)\vert\leqslant x\vert\cos(t)\vert+y\vert\sin(2t)\vert=\vert\varphi(-t)\vert\end{equation}
		et $-t\in[0,\pi]$. Donc le $\sup$ est atteint sur $[0,\pi]$.

		On note maintenant, comme $\vert\varphi(\pi-t)\vert=\vert\varphi(t)\vert$ sur $[0,\frac{\pi}{2}]$, que si $t\in[\frac{\pi}{4},\frac{\pi}{2}]$,
		\begin{equation}0\leqslant\varphi(t)=x\underbrace{\cos(t)}_{\in[0,\frac{\sqrt{2}}{2}]}+y\sin(2t)\leqslant x\underbrace{\cos(\frac{\pi}{2}-t)}_{\in[\frac{\sqrt{2}}{2},1]}+y\sin(2\times (\frac{\pi}{2}-t))=\varphi(\frac{\pi}{2}-t)\end{equation}

		Donc le $\sup$ est atteint sur $[0,\frac{\pi}{4}]$. Soit maintenant $t_{0}\in[0,\frac{\pi}{4}]$ tel que $\varphi(t_{0})$ réalise le $\sup$ (existe car $\varphi$ est continue sur un compact). Comme c'est aussi le $\sup$ sur $\R$ qui est ouvert, on a la condition d'Euler du premier ordre: $\varphi'(t_{0})=0$.

		On a donc $x\cos(t_{0})+y\sin(2t_{0})=1$ et $-x\sin(t_{0})+2y\cos(2t_{0})=0$. On en déduit les valeurs de $x$ et $y$ en fonction de $t_{0}$, en faisant attention que $\cos(t_{0})\neq0$ sinon $\sin(t_{0})=0$ aussi ce qui n'est pas le cas, et au cas où $t_{0}=0$.

		Réciproquement, s'il existe $t_{0}\in[0,\frac{\pi}{4}]$ tel que $x$ et $y$ s'écrivent de la façon demandée, alors $t_{0}$ est l'unique point satisfaisant $\varphi(t_{0})=1$ et $\varphi'(t_{0})=0$. Mais alors le $\sup$ de $\varphi$ sur $[0,\frac{\pi}{4}]$ est atteint en un point $t_{1}$ qui vérifie les mêmes choses, donc $t_{1}=t_{0}$ d'où $N(x,y)=1$.
	\end{enumerate}
\end{proof}

\begin{proof}
	\phantom{}
	\begin{enumerate}
		\item Pour l'inégalité triangulaire, introduire la forme bilinéaire symétrique positive sur $E$ \function{\varphi}{E\times E}{\R}{(f,g)}{f(0)g(0)+\int_{0}^{1}f'(t)g'(t)dt}
		Alors $N(f)=\sqrt{\varphi(f,f)}$ et on utilise l'inégalité de Minkowski.
		\item Pour $x\in[0,1]$, écrire $\vert f(x)\vert=\vert f(0)+f(x)-f(0)\vert$, $f(x)-f(0)=\int_{0}^{x}f'(t)dt$, utiliser Cauchy-Schwarz avec $f'$ et $1$ puis que $\sqrt{a}+\sqrt{b}\leqslant\sqrt{2}\sqrt{a+b}$, pour enfin passer au $\sup$ sur $x$.
		\item Utiliser, pour $n\in\N^{*}$, la fonction \function{f_n}{[0,1]}{\R}{t}{t^n}
	\end{enumerate}
\end{proof}

\begin{proof}
	Si $f$ est ouverte, $f(\R^{n})$ est un sous-espace vectoriel ouvert de $R^{p}$. Donc $f$ est surjective.

	Si $f$ est surjective, on prend $F$ un supplémentaire de $\ker(f)$ dans $\R^{n}$ avec $\dim(\ker(f))=n-p$ et $\dim(F)=p$. Soit $(e_{1},\dots,e_{p})$ une base de $F$ et $(e_{p+1},\dots,e_{n})$ une base de $\ker(f)$. On vérifie que $(f(e_{1},\dots,f(e_{p}))$ est une base de $\R^{p}$. On définit \function{N_1}{\R^n}{\R}{\sum_{i}^{n}x_{i}e_{i}}{\max\limits_{1\leqslant i\leqslant n}\vert x_{i}\vert}
	norme sur $\R^{n}$ et \function{N_2}{\R^p}{\R}{\sum_{i}^{p}y_{i}f(e_{i})}{\max\limits_{1\leqslant i\leqslant p}\vert y_{i}\vert}
	norme sur $\R^{p}$.

	Soit $\Theta$ un ouvert de $\R^{n}$, soit $y_{0}\in f(\Theta)$, il existe $x_{0}\in\Theta\colon y_{0}=f(x_{0})$. Si $x_{0}=\sum_{i=1}^{n}\alpha_{i}e_{i}$, alors $y_{0}=\sum_{i=1}^{p}\alpha_{i}f(e_{i})$. Comme $\Theta$ est un ouvert, il existe $r_{0}>0$ tel que 
	\begin{equation}B_{N_{1}}(x_{0},r_{0})\subset\Theta\end{equation}
	Soit $y=\sum_{i=}^{p}\beta_{i}f(e_{i})\in\R^{p}$, si $N_{2}(y-y_{0})<r_{0}$, pour tout $i\in\{1,\dots,p\}$, $\vert\beta_{i}-\alpha_{i}\vert<r_{0}$ et 
	\begin{equation}y=f\Biggl(\sum_{i=1}^{p}\beta_{i}e_{i}+\sum_{i=p+1}^{n}\alpha_{i}e_{i}\Biggr)\overset{\text{def}}{=}f(x)\end{equation}
	avec $N_{1}(x-x_{0})=\max\limits_{1\leqslant i\leqslant p}\vert\beta_{i}-\alpha_{i}\vert<r_{0}$. Ainsi $x\in\Theta$ et $y\in f(\Theta)$, donc $B_{N_{2}}(y_{0},r_{0})\subset f(\Theta)$ et $f(\Theta)$ est un ouvert.
\end{proof}

\begin{proof}
	\phantom{}
	\begin{enumerate}
		\item Classique.
		\item \begin{equation}\vert f(x)\vert\leqslant\vert f(0)\vert+\vert f(x)-f(0)\vert\leqslant\vert f(0)\vert+\kappa(f)x\leqslant N(f)\end{equation}
		car $x\leqslant 1$, donc $N_{\infty}\leqslant N$. Pour la non-équivalence, prendre \function{f_n}{[0,1]}{\R}{t}{t^n}
		\item On a $\vert f(0)\vert\leqslant N_{\infty}(f)$ donc $N(f)\leqslant N'(f)$. Ensuite, $N_{\infty}\leqslant N$ donne $N'\leqslant N+\kappa\leqslant 2N$. Donc $N$ est $N'$ sont équivalentes.
	\end{enumerate}
\end{proof}

\begin{remark}
	Exemple de normes qui, en dimension infinie, ne se dominent pas mutuellement. On prend $(e_{i})_{i\in I}$ une base (de Hamel), $J=(i_{n})_{n\in\N}\subset I$ dénombrable. Si $x=\sum_{i\in I}x_{i}e_{i}$, on peut vérifier que 
	\begin{equation}N_{1}(x)=\sum_{n\in\N}\vert x_{i_{n}}\vert+\sum_{i\in I\setminus J}\vert x_{i}\vert\end{equation}
	et
	\begin{equation}N_{2}(x)=\sum_{n\in\N}n\vert x_{i_{2n}}\vert+\sum_{n\in\N}\frac{1}{n+1}\bigl\lvert x_{i_{2n+1}}\bigr\rvert+\sum_{i\in I\setminus J}\vert x_{i}\vert\end{equation}
	ne se dominent pas.
\end{remark}

\begin{proof}
	Il existe $\alpha>0$ tel que $B_{\Vert\cdot\Vert_{\infty}}(I_{n},\alpha)\subset G$. Soient $i\neq j$ et $\lambda\in\C$. Il existe $p\in\N^{*}$ tel que $\frac{\vert\lambda\vert}{p}<\alpha$. Alors 
	\begin{equation}\Biggl\lVert T_{i,j}\Biggl(\frac{\lambda}{p}\Biggr)-I_{n}\Biggr\rVert_{\infty}=\Biggl\lvert\frac{\lambda}{p}\Biggr\rvert<\alpha\end{equation}
	donc $T_{i,j}(\lambda)\in G$ ($T_{i,j}$ est la matrice de transvection: $T_{i,j}(\lambda)=I_{n}+\lambda E_{i,j}$).

	Ainsi,
	\begin{equation}T_{i,j}(\lambda)=\Biggl(T_{i,j}\Biggl(\frac{\lambda}{p}\Biggr)\Biggr)^{p}\in G\end{equation}

	Soit $\delta=\rho e^{\mathrm{i}\theta}\in\C^{*}$. On a $\lim\limits_{n\to+\infty}\rho^{\frac{1}{p}}e^{\mathrm{i}\frac{\theta}{p}}=1$ donc il existe $p\in\N^{*}$ tel que $\vert\rho^{\frac{1}{p}}e^{\mathrm{i}\frac{\theta}{p}}-1\vert<\alpha$.
	
	On a alors
	\begin{equation}\Biggl\lVert D_{n}\Bigl(\rho^{\frac{1}{p}}e^{\mathrm{i}\frac{\theta}{p}}\Bigr)-I_{n}\Biggr\rVert_{\infty}<\alpha\end{equation}
	donc $D_{n}(\delta)=D_{n}(\rho^{\frac{1}{p}}e^{\mathrm{i}\frac{\theta}{p}})^{p}\in G$ (matrice de dilatation).

	Comme les matrices de transvection et de dilatation engendrent $GL_{n}(\C)$, on a bien $G=GL_{n}(\C)$.
\end{proof}

\begin{remark}
	C'est faux sur $\R$. Contre-exemple: matrices de déterminant positif.
\end{remark}

\begin{proof}
	Si $f$ n'est pas continue en 0, il existe $\varepsilon_{0}>0$ tel que pour tout $\alpha>0$, il existe $h\in E$ avec $\Vert h\Vert\leqslant\alpha$ et $\Vert f(h)\Vert>\varepsilon_{0}$. On prends $\alpha_{n}=\frac{1}{n+1}$, d'où $\Vert nh_{n}\Vert\leqslant1$ mais $\underbrace{\Vert f(nh_{n})\Vert}_{\leqslant M}>n\varepsilon_{0}\xrightarrow[n\to+\infty]{}+\infty$. Donc $f$ est continue en $0$. Comme $f$ est linéaire, pour tout $x\in E$,
	\begin{equation}\lim\limits_{\Vert h\Vert\to0}f(x+h)=\lim\limits_{\Vert h\Vert\to0}f(x)+f(h)=f(x)\end{equation}
	donc $f$ est continue.

	On a $f(px)=p(fx)$ pour tout $p\in\Z$ puis $qf(\frac{p}{q}x)=f(px)=pf(x)$ pour tout $(p,q)\in\Z\times\N^{*}$ donc pour tout $r\in\Q$, $f(rx)=rf(x)$.
	Soit $\lambda\in\E$, il existe une suite de rationnels telle que $\lim\limits_{n\to+\infty} r_{n}=\lambda$. Comme $f$ est continue, on a 
	\begin{align}
		f(\lambda x)
		&=\lim\limits_{n\to+\infty}f(r_{n}x)\\
		&=\lim\limits_{n\to+\infty}r_{n}f(x)\\
		&=\lambda f(x)
	\end{align}
	Donc $f$ est linéaire.
\end{proof}

\begin{remark}
	Soit $e_{0}=1$ et $e_{1}=\sqrt{2}$ et $(e_{i})_{i\in I}$ une $\Q$-base de $\R$ ($0\in I$). On définie 
	\begin{equation}f\Bigl(\sum_{i\in I}\lambda_{i} e_{i}\Bigr)=\lambda_{0}e_{0}+\sqrt{2}\sum_{i\in I\setminus\{0\}}\lambda_{i}e_{i}\end{equation}
	$f$ vérifie $f(x+y)=f(x)+f(y)$, mais si $(r_{n})_{n\in\N}$ est une suite de rationnels tendant vers $\sqrt{2}$, $f(r_{n})=r_{n}\to\sqrt{2}\neq f(\sqrt{2})=2$.
\end{remark}

\begin{proof}
	\phantom{}
	\begin{enumerate}
		\item On a $\alpha(A)\subset \overline{A}$ donc $\overline{\mathring{\overline{A}}}\subset\overline{A}$ donc $\alpha(\alpha(A))\subset\alpha(A)$. Comme $\alpha(A)$ est un ouvert inclus dans $\overline{\mathring{\overline{A}}}\subset\overline{A}$ donc $\alpha(A)\subset\alpha(\alpha(A))$.

		\item Si $\beta(A)=\overline{\mathring{A}}$, on montre aussi que $\beta(\beta(A))=\beta(A)$. On a donc $A,\overline{A},\mathring{A},\overline{\mathring{A}},\mathring{\overline{A}},\overline{\mathring{\overline{A}}}$ et $\mathring{\overline{\mathring{A}}}$ et c'est tout.
	\end{enumerate}
\end{proof}

\begin{proof}
	\phantom{}
	\begin{enumerate}
		\item Si $d_{A}=d_{B}$, 
		\begin{equation}\overline{A}=\{x\in E\bigm| d_{A}(x)=0\}=\{x\in E\bigm| d_{B}(x)=0\}=\overline{B}\end{equation}
		Réciproquement, soit $x\in E$ et $\varepsilon>0$, il existe $a_{1}\in\overline{A}$, $\Vert x-a_{i}\Vert\leqslant d_{\overline{A}}(x)+\frac{\varepsilon}{2}$ (par définition de l'inf). Il existe $a_{2}\in A$, $\Vert a_{1}-a_{2}\Vert\leqslant\frac{\varepsilon}{2}$ (par définition de la fermeture). Ainsi,
		\begin{equation}d_{A}(x)\leqslant\Vert x-a_{2}\Vert\leqslant\Vert x-a_{1}\Vert+\Vert a_{1}-a_{2}\Vert\leqslant d_{\overline{A}}(x)+\varepsilon\end{equation}
		Ceci valant pour tout $\varepsilon>0$, $d_{A}(x)\leqslant d_{\overline{A}}(x)$. Comme $A\subset\overline{A}$, $d_{\overline{A}}\leqslant d_{A}$, on a $d_{A}=d_{\overline{A}}=d_{\overline{B}}=d_{B}$.

		\item Soit $x\in A$, on a $d_{B}(x)=\vert d_{B}(x)-d_{A}(x)\vert\leqslant\rho(A,B)$ donc $\sup\limits_{x\in A}d_{B}(x)\leqslant\rho(A,B)$, de même pour $\sup\limits_{y\in B}d_{A}(y)$ donc on on a un première inégalité.
		
		Réciproquement, soit $x\in E$ et $\varepsilon>0$, il existe $a\in A$ et $b\in B$ tel que $\Vert x-a\Vert\leqslant d_{A}(x)+\varepsilon$ et $\Vert x-b\Vert\leqslant d_{B}(x)+\varepsilon$.
		On a alors
		\begin{equation}d_{A}(x)\leqslant\Vert x-a\Vert\leqslant\Vert a-b\Vert+\Vert x-b\Vert\leqslant d_{B}(x)+\varepsilon+\alpha(A,B)\end{equation}
		Ceci vaut pour tout $\varepsilon>0$, donc $d_{A}(x)\leqslant d_{B}(x)+\alpha(A,B)$. De même, $d_{B}(x)\leqslant d_{A}(x)+\alpha(A,B)$ donc $\rho(A,B)\leqslant\alpha(A,B)$.
	\end{enumerate}
\end{proof}

\begin{proof}
	\phantom{}
	\begin{enumerate}
		\item Soit $(y_{n})_{n\in\N}\in P(F)^{\N}$ qui converge vers $y\in\C$ donc il existe $(x_{n})\in F^{\N}$ telle que l'on ait pour tout $n\in\N$, $P(x_{n})=y_{n}$. $(x_{n})_{n\in\N}$ est bornée car $\lim\limits_{z\to+\infty}\vert P(z)\vert=+\infty$ (car $P$ est non constant), donc on peut extraire (Bolzano-Weierstrass) $x_{\sigma(n)}\to x$ et $x\in F$ car $F$ est fermé. Par continuité de $z\mapsto P(z)$ sur $\C$, on a $y=P(x)\in P(F)$.
		
		\item Soit $\Theta$ un ouvert de $\C$, soit $y\in P(\Theta),\exists x\in\Theta$ tel que $P(x)=y$ et il existe $r>0$, $B(x,r)\subset\Theta$. Soit $y'\in\C$, supposons que pour tout $x'\in\C$ tel que $P(x')=y'$, on a $\vert x-x'\vert>r$. Soit $Q(X)=P(X)-y'=a\prod_{i=1}^{n}(X-x_{i})$ non constant où $a$ est le coefficient dominatrice de $P$. Par hypothèse, pour tout $i\in\{1,\dots,n\}\colon\vert x_{i}-x\vert>r$ (car $P(x_{i})=y'$), ainsi 
		\begin{equation}\vert Q(x)\vert=\vert y-y'\vert\geqslant\vert a\vert r^{n}\end{equation}
		Par contraposée, si $\vert y-y'\vert\leqslant\frac{\vert a\vert r^{n}}{2}$, alors il existe $x'\in\C$ tel que $P(x')=y'$ et $\vert x'-x\vert<r$.Ainsi, $x'\in B(x,r)\subset\Theta$ et $y'\in P(\Theta)$. Donc $B(y,\vert a\vert r^{n})\subset P(\Theta)$ et $P(\Theta)$ est un ouvert.
	\end{enumerate}
\end{proof}

\begin{proof}
	\phantom{}
	\begin{enumerate}
		\item Si $P\notin\mathcal{S}$, il existe $z_{0}\in\C\setminus\R$ tel que $P(z_{0})=0$ et $\vert\Im(z_{0})\vert^{n}>0=P(z_{0})$. Par contraposée, si pour tout $z\in\C$, $\vert P(z)\vert\geqslant\vert\Im(z_{})\vert^{n}$,alors $P\in\mathcal{S}$.

		Réciproquement, si $P=\prod_{i=1}^{n}(X-\lambda_{i})\in\mathcal{S}$ avec $(\lambda_{i})_{1\leqslant i\leqslant n}$ réels, soit $z=a+ib\in\C$. On a
		\begin{equation}\vert P(z)\vert=\prod_{i=1}^{n}\vert a-\lambda_{i}+ib\vert\geqslant\vert b\vert^{n}\end{equation}
		
		\item Soit $(P_{p})_{p\in\N}\in\mathcal{S}^{\N}$ telle que $P_{p}\xrightarrow[p\to+\infty]{}P\in F$. Soit $z\in\C$, on a pour tout $p\in\N$, $\vert P_{p}(z)\vert\geqslant\vert\Im(z)\vert^{n}$ donc quand $p\to+\infty$, $\vert P(z)\vert\geqslant\vert\Im(z)\vert^{n}$ donc $P\in\mathcal{S}$ et $S$ est fermé.
		
		\item Soit $(M_{p})_{p\in\N}$ une suite de matrice trigonalisable sur $\R$ qui converge vers $M\in\M_{n}(\R)$. Ib bite $\chi_{p}$ le polynôme caractéristique de $M_{p}$. Pour tout $p\in\N$, $\chi_{p}\in\mathcal{S}$ et $\chi_{p}\xrightarrow[p\to+\infty]{}\chi_{M}$. Comme $\mathcal{S}$ est fermé, $\chi_{M}\in \mathcal{S}$ et $M$ est trigonalisable sur $\R$.
	\end{enumerate}
\end{proof}

\begin{proof}
	\phantom{}
	\begin{enumerate}
		\item $\varphi$ est linéaire et $\dim(\K_{m-1}[X]\times\K_{n-1}[X])=m+n+=\dim(\K_{n+m-1}[X])$.
		
		Si $\varphi$ est bijective, elle est surjective et il existe $(U,V)\in\K[X]^{2}$ tel que $UA+BV=1$ et d'après le théorème de Bézout, on a $A\wedge B=1$.

		Réciproquement, si $\varphi$ n'est pas surjective, il existe $(U,V)\in(\K_{m-1}[X]\times\K_{n-1}[X])\setminus\{(0,0)\}$ tel que $\varphi(U,V)=0$ d'où $AU=-BV$. Soit $\delta=A\wedge B$, on écrit $A=\delta A_{1}$ et $B=\delta B_{1}$ avec $A_{1}\wedge B_{1}=1$ et on a $A_{1}U=-B_{1}V$. D'après le théorème de Gauss, on a $A_{1}\mid V$ et $B_{1}\mid U$. Si $U=0$, on a $V=0$ et de même si $V=0$, on a $U=0$. On peut donc supposer $U\neq0$ et $V\neq 0$, et on a alors $\deg(A_{1})\leqslant\deg(V)\leqslant n-1<n=\deg(A)$ mais $A=\delta A_{1}$ donc $\deg(\delta)\geqslant1$ et $A\wedge B\neq 1$.

		\item $\Phi$ est continue car $R_{A,B}$ est un polynôme en les coefficients de $A$ et $B$.
		
		\item Comme on est dans $\C$, $\Delta=\{P\in\C_{p}[X]\bigm| P\wedge P'=1\}=\{P\in\C_p[X]\bigm| R_{P,P'}\neq0\}$. $\Phi_{P,P'}$ est continue d'après la question précédente, $\delta=\Phi_{P,P'}^{-1}(\C^{*})$ donc $\Delta$ est ouvert.
		
		Sur $\R$, on n'a pas la caractérisation de scindé à racines simples si et seulement si $P\wedge P'=1$ (contre-exemple: $P=X^{2}+1$). Dans $\R_{3}[X]$, $X$ est scindé à racines simples et $X(1+\varepsilon X)^{2}\xrightarrow[\varepsilon\to0]{}X$ et $-\frac{1}{\varepsilon}$ est racine double, donc $\Delta$ n'est pas ouvert.
	\end{enumerate}
\end{proof}

\begin{remark}
	On peut cependant considérer 
	\begin{equation}\Delta_{n}=\{P\in\C_{p}[X]\bigm| P\text{ scindé à racines simples sur }\R\text{ et }\deg(P)=n\}\end{equation}
	Si $\lambda_{1}<\lambda_{2}<\dots<\lambda_{n}$ sont les racines (distinctes) de $R$ sur $\R$, on choisit $\alpha_{0}\in]-\infty,\lambda_{1}$, $\alpha_{n}\in]\lambda_{n},+\infty[$ et $\alpha_{i}\in]\lambda_{i},\lambda_{i+1}[$ si $i=1,\dots,n-1$. 

	Pour tout $k\in\{0,\dots,n-1\}$, on a $P(\alpha_{k})P(\alpha_{k+1})<0$ (car les racines de $P$ provoquent des changements de signe). Soit \function{\Psi}{\R_n[X]}{\R^n}{Q}{(Q(\alpha_{k})Q(\alpha_{k+1}))_{0\leqslant k\leqslant n-1}}
	$\Psi$ est continue sur $\R_{n}[X]$ et $\Psi(P)\in(\R_{-}^{*})^{n}$ qui est ouvert, donc il existe $r>0$ tel que si $\Vert P-Q\Vert<r$, alors $\Psi(Q)\in(\R_{-}^{*})^{n}$. Donc $Q$ change $n$ fois de signe, et admet au moins $n$ racines. Mais $\deg(Q)=n$, donc $Q$ est scindé à racines simples sur $\R$, donc $\Delta_{n}$ est ouvert dans $\{P\in\R[X]\bigm|\deg(P)=n\}$.
\end{remark}

\begin{remark}
	$\{M\in\M_{n}(\C)\bigm|M\text{ diagonalisable à racines simples}\}$ est exactement égal à $\{M\in\M_{n}(\C)\bigm|\chi_M\text{ scindé à racines simples}\}$. C'est donc un ouvert de $\M_{n}(\C)$ car $M\mapsto\chi_{M}$ est continue sur $\M_{n}(\C)$, et c'est aussi vrai sur $\R$.
\end{remark}

\begin{proof}
	\phantom{}
	\begin{enumerate}
		\item Soit \function{f}{\M_n(\R)}{\M_n(\R)}{A}{A^{n}}
		$f$ est continue et $F=f^{-1}(\{0\})$ donc $F=\overline{F}$.

		Soit $M_{0}\in F$, $X^{n}$ annule $M_{0}$ donc $M_{0}$ est trigonalisable: on écrit $M_{0}$ dans une base où les coefficients diagonaux sont tous nuls. Soit alors $M_{\varepsilon}$ la même matrice dans la même base en rajoutant simplement $\varepsilon$ en première position de la diagonale. Alors $M_{\varepsilon}\xrightarrow[\varepsilon\to0]{}M_{0}$ et $M_{\varepsilon}\notin F$ donc $\mathring{F}=\emptyset$. Notons que cela signifie que $F$ est dense.

		\item La norme dérive du produit scalaire $(A|B)\mapsto\Tr(A^{\mathsf{T}}B)$. Soit $M\in F$, on a $\Vert M-I_{n}\Vert^{2}=\Vert M\Vert^{2}+\Vert I_{n}\Vert^{2}-2(M|I_{n})$. On a $(M|I_{n})=\Tr(M)=0$ car $M$ est nilpotente. Donc $\Vert M-I_{n}\Vert^{2}$ est minimale pour $\Vert M\Vert^{2}$ minimale, donc pour $M=0\in F$. Donc $d(I_{n},F)=\Vert I_{n}\Vert=\sqrt{n}$ (et la distance est atteinte pour $0_{\M_n(\R)}$).
	\end{enumerate}
\end{proof}

\begin{proof}
	\phantom{}
	\begin{enumerate}
		\item $A\mapsto\det(A)$ est continue et $GL_{n}(\K)=\det^{-1}(\K^{*})$ est donc ouvert. Si $A\in\M_{n}(\K)$, pour $p\in\N$, on pose $A_{p}=A-\frac{1}{p+1}I_{n}$. Comme $\Sp(A)$ est fini, il existe $N\in\N$, tel que pour tout $p\geqslant N$, $\frac{1}{p+1}\notin\Sp(A)$. Donc pour tout $p\geqslant N$, $A_{p}\in GL_{n}(\K)$, et $A_{p}\xrightarrow[p\to+\infty]{}A$ donc $GL_{n}(\K)$ est dense dans $\M_{n}(\K)$.
		\item On fixe $B\in\M_{n}(\K)$. Soit $A\in GL_{n}(\K)$. On écrit $BA=A^{-1}(AB)A$ donc $AB$ et $BA$ sont semblables donc $\chi_{AB}=\chi_{BA}$. Comme, à $B$ fixé, $A\mapsto\chi_{AB}$ et $A\mapsto\chi_{BA}$ sont continues sur $\M_{n}(\K)$, on a le résultat par densité.
	\end{enumerate}
\end{proof}

\begin{proof}
	\phantom{}
	\begin{enumerate}
		\item On a $v_{p}\circ(id_{E}-u)=(id_{E}-u)\circ v_{p}=\frac{1}{p}(id_{E}-u^{p})$, donc $\Vert v_{p}\circ(id_{E}-u)\Vert\leqslant\frac{1}{p}(\Vert id_{E}\Vert+\Vert u^{p}\Vert)\xrightarrow[p\to+\infty]{}0$.
		
		Soit $x\in\ker(u-id_{E})\cap\im(u-id_{E})$, on a $u(x)=x$ et il existe $y\in E$, $x=(u-id_{E})(y)$. On a $v_{p}(x)=\frac{1}{p}(px)=x$ et $v_{p}(x)=v_{p}\circ(u-id_{E})(y)\xrightarrow[p\to+\infty]{}0$ d'où $x=0$. Le théorème du rang permet de conclure.

		\item Soit $x\in E$, on écrit $x=x_{1}+x_{2}$ avec $\Pi(x)=x_{1}$ et $x_{2}=(u-id_{E})(y_{2})$. Alors $v_{p}(x)=x_{1}+v_{p}\circ(u-id_{E})(y_{2})\xrightarrow[p\to+\infty]{}x_{1}=\Pi(x)$.
	\end{enumerate}
\end{proof}

\begin{proof}
	\phantom{}
	\begin{enumerate}
		\item Pour tout $x\in A$, $f_{n}(x)\in A$ car $A$ est convexe. Soit $(x,y)\in A^{2}$, on a
		\begin{equation}\Vert f_{n}(x)-f_{n}(y)\Vert=\Bigl(1-\frac{1}{n}\Bigr)\Vert f(x)-f(y)\Vert\leqslant\Bigl(1-\frac{1}{n}\Bigr)\Vert x-y\Vert\end{equation}
		Donc $f_{n}$ est $(1-\frac{1}{n})$-lipschitzienne. On forme \function{g_n}{A}{\R}{x}{\Vert f_n(x)-x\Vert}
		qui est continue. Soit $x_{n}\in A$ telle que $g_{n}(x_{n})=\min\limits_{x\in A}g_{n}(x)$ (existe car $A$ est compact et $g_{n}$ continue). On a $x_{n}\in A$, d'où $f_{n}(x_{n})\in A$ et 
		\begin{equation}g_{n}(f_{n}(x_{n}))=\Vert f_{n}(f_{n}(x_{n}))-f_{n}(x_{n})\Vert\leqslant\Bigl(1-\frac{1}{n}\Bigr)\Vert f_{n}(x_{n})-x_{n}\Vert=\Bigl(1-\frac{1}{n}\Bigr)g_{n}(x_{n})\end{equation}
		Si $g_{n}(x_{n})\neq0$, alors on aurait $g_{n}(f(x_{n}))<g_{n}(x_{n})$ ce qui n'est pas possible. Donc $g_{n}(x_{n})=0$ et $f_{n}(x_{n})=x_{n}$.

		Soit $y_{n}$ un autre point fixe, on a 
		\begin{equation}\Vert f_{n}(x_{n})-f_{n}(y_{n})\Vert=\Vert x_{n}-y_{n}\Vert\leqslant\Bigl(1-\frac{1}{n}\Bigr)\Vert x_{n}-y_{n}\Vert\end{equation}
		donc $x_{n}=y_{n}$.

		\item On a $(x_{n})_{n\in\N}\in A^{\N}$ et on extrait (car $A$ est compact) et on a 
		\begin{equation}x_{\sigma(n)}\xrightarrow[n\to+\infty]{}x\in A\end{equation}
		On a 
		\begin{equation}f_{\sigma(n)}(x_{\sigma(n)})=x_{\sigma(n)}=\underbrace{\frac{1}{\sigma(n)}f(x_{0})}_{\xrightarrow[n\to+\infty]{}0}+\underbrace{\Bigl(1-\frac{1}{\sigma(n)}\Bigr)f(x_{\sigma(n)})}_{\xrightarrow[n\to+\infty]{}f(x)}\end{equation}
		par continuité de $f$. Donc $f(x)=x$.

		\item Soit $(x,y)\in A^{2}$, points fixes de $f$, et $t\in[0,1]$, on pose $z=tx+(1-t)y$. On a 
		\begin{align}
			\Vert x-y\Vert
			&=\Vert f(x)-f(y)\Vert\\
			&\leqslant \Vert f(x)-f(z)\Vert+\Vert f(z)-f(y)\Vert\\
			&\leqslant\Vert x-z\Vert+\Vert z-y\Vert\\
			&=(1-t)\Vert x-y\Vert+t\Vert x-y\Vert\\
			&=\Vert x-y\Vert
		\end{align}
		On a donc égalité partout: $\Vert f(x)-f(y)\Vert=\Vert f(x)-f(z)\Vert+\Vert f(z)-f(y)\Vert$ et $\Vert f(x)-f(z)\Vert=\Vert x-z\Vert$, $\Vert f(z)-f(y)\Vert=\Vert z-y\Vert$ car $f$ est $1$-lipschitzienne.

		Comme la norme est euclidienne, il existe $\lambda\in\R_{+}$ tel que $f(x)-f(z)=\lambda(f(z)-f(y))$ d'où $f(x)+\lambda f(y)=(\lambda+1)f(z)$ d'où $f(z)=\frac{x+\lambda y}{\lambda+1}=t'x+(1-t')y$ avec $t'=\frac{1}{\lambda+1}\in[0,1]$. En reportant, on a 
		\begin{equation}\Vert f(x)-f(z))\Vert=\Vert x-t'x-(1-t')y\Vert=(1-t')\Vert x-y\Vert=\Vert x-z\Vert=(1-t)\Vert x-y\Vert\end{equation}
		Si $x\neq y$, alors $t=t'$ et $f(z)=tx+(1-t)y=z$.

		\item Soit dans $\R^{2}$, $\overline{B_{\Vert\cdot\Vert}(0,1)}=[-1,1]^{2}=A$. Soit \function{f}{A}{A}{(x,y)}{(x,\vert x\vert)}
		On a 
		\begin{align}
			\Vert f(x_{1},y_{1})-f(x_{2},y_{2})\Vert_{\infty}
			&= \Vert (x_{1},\vert x_{1}\vert)(x_{2},\vert x_{2}\vert)\Vert_{\infty}\\
			&=\max\{\vert x_{1}-x_{2}\vert, \bigl\vert\vert x_{1}\vert-\vert x_{2}\vert\bigr\vert\}\\
			&=\vert x_{1}-x_{2}\vert\\
			&\leqslant\Vert (x_{1},y_{1})-(x_{2},y_{2})\Vert_{\infty}
		\end{align}
		Donc $f$ est 1-lipschitzienne, on a $f(x,y)=(y,x)$ si et seulement si $y=\vert x\vert$. Donc ici, $F$ n'est pas convexe.
	\end{enumerate}
\end{proof}

\begin{proof}
	\phantom{}
	\begin{enumerate}
		\item On a pour tout $(x,y)\in E^{2}$, $f(x+y)=f(x)+f(y)$ et par récurrence, pour tout $n\in\Z$, $f(nx)=nf(x)$. Pour $r=\frac{p}{q}\in\Q$, on a $f(qrx)=qf(rx)=f(px)=pf(x)$ donc $f(rx)=rf(x)$. Par densité de $\Q$ dans $\R$ et continuité de $f$, on a pour tout $\lambda\in\R$, $f(\lambda x)=\lambda f(x)$. Donc $f$ est linéaire.
		
		Pour $\K=\C$, cela ne marche pas. Contre-exemple: la conjugaison dans $\C$.

		\item On étudie la série, pour $x$ fixé de terme général 
		\begin{equation}\Vert v_{n+1}(x)-v_{n}(x)\Vert=\frac{1}{2^{n}}\Vert f(2^{n+1}x)-2f(2^{n}x)\Vert\leqslant\frac{M}{2^{n+1}}\end{equation}
		qui est donc convergente. Donc $(v_{n})_{n\in\N}$ converge.

		\item On a $v_{0}(x)=f(x)$, donc $\sum_{n=0}^{+\infty}v_{n+1}(x)-v_{n}(x)=g(x)-f(x)$. $f$ étant continue, $v_{n}$ l'est aussi, et pour tout $n\in\N$, comme pour tout $x\in E$, $\Vert (v_{n+1}-v_{n})(x)\Vert\leqslant\frac{M}{2^{n+1}}$, donc $g$ est continue.
		
		\item On a, pour tout $(x,y)\in E^{2}$,
		\begin{equation}\Vert v_{n}(x+y)-v_{n}(x)-v_{n}(y)\Vert=\Vert \frac{1}{2^{n}}f(2^{n}(x+y))-\frac{1}{2^{n}}(f(2^{n}x)+f(2^{n}y))\Vert\leqslant\frac{M}{2^{n}}\end{equation}
		Donc quand $n\to+\infty$, $g(x+y)=g(x)+g(y)$.

		On a pour tout $x\in E$, 
		\begin{equation}\Vert g(x)-f(x)\Vert=\Bigl\lVert\sum_{n=0}^{+\infty}v_{n+1}(x)-v_{n}(x)\Bigr\Vert\rVert\leqslant\sum_{n=0}^{+\infty}\Vert v_{n+1}(x)-v_{n}(x)\Vert\leqslant\sum_{n=0}^{\infty}\frac{M}{2^{n}}=M\end{equation}

		Soit maintenant $h$ linéaire continue telle que $h-f$ soit bornée, soit $M'=\sup\limits_{x\in E}\Vert h(x)-f(x)\Vert$. On a donc 
		\begin{equation}\Vert v_{n}(x)-h(x)\Vert=\Bigl\Vert\frac{1}{2^{n}}f(2^{n}x)-\frac{1}{2^{n}}h(2^{n}x)\Bigr\Vert\leqslant\frac{M'}{2^{n}}\end{equation}
		car $h$ est linéaire. Donc quand $n\to+\infty$, $g(x)=h(x)$ car $\lim\limits_{n\to+\infty}v_{n}(x)=g(x)$.
	\end{enumerate}
\end{proof}

\begin{proof}
	En particulier, pour $t=f(0)$, $f^{-1}(\{f(0)\})=\{x\in E\bigm| f(x)=f(0)\}$ est borné (car compact). Donc il existe $A$ tel que $f^{-1}(\{f(0)\})\subset\overline{B(0,A)}$. Par contraposée, pour tout $x\in E$, si $\Vert x\Vert>A$, alors $f(x)\neq f(0)$.

	On montre alors que $E\setminus\overline{B(0,A)}$ est connexe par arcs (faire le tour de la boule par l'extérieur).

	$f$ étant continue, d'après le théorème des valeurs intermédiaires, on a soit pour tout $x\in E\setminus\overline{B(0,A)}$, $f(x)>f(0)$ soit $f(x)<f(0)$. Quitte à remplacer $f$ par $-f$, on se place dans le cas $f(x)>f(0)$. Comme on est en dimension finie sur $\overline{B(0,A)}$ compact, $f$ atteint son minimum et ce minimum est plus petit que $f(0)$, c'est donc un minimum global.
\end{proof}

\begin{remark}
	C'est faux pour $n=1$. Contre-exemple: $f=id_{\R}$.
\end{remark}

\begin{proof}
	Si c'était le cas, on prend un cercle $\mathcal{C}$ compact (et connexe par arcs). $f(\mathcal{C})$ est compact connexe par arc dans $\R$. On note $f(\mathcal{C})=[a,b]$ (avec $a<b$ car $f$ injective). Si $x\in\mathcal{C}$ est tel que $f(x)=\frac{a+b}{2}$, on $\underbrace{f(\mathcal{C}\setminus\{x\})}_{\text{connexe par arc}}=\underbrace{[a,b]\setminus\Bigl\{\frac{a+b}{2}\Bigr\}}_{\text{pas connexe par arc}}$ donc une telle fonction n'existe pas.
\end{proof}

\begin{proof}
	\phantom{}
	\begin{enumerate}
		\item Pour tout $n\in\N$, $\Vert e_{n}\Vert_{l^{1}}=1$ et $\vert K_{n}\vert=\vert\varphi(e_{n})\vert\leqslant\vertiii{\varphi}$ donc $(K_{n})_{n\in\N}$ est bornée. On note $M=\sup\vert K_{n}\vert\leqslant\vertiii{\varphi}$.
		
		Soit maintenant $u=(u_{n})_{n\in\N}\in l^{1}$. On a, pour $N\in\N$, 
		\begin{equation}\Biggl\lVert u-\sum_{n=0}^{N}u_{n}e_{n}\Biggr\rVert_{1}\leqslant\sum_{n=N+1}^{\infty}\vert u_{n}\vert\xrightarrow[N\to+\infty]{}0\end{equation}
		(reste d'une série convergente). Par continuité de $\varphi$, on a donc 
		\begin{equation}\vert \varphi(u)\vert\leqslant\sum_{n=0}^{\infty}\vert u_{n}\vert \vert K_{n}\vert\leqslant M\Vert u\Vert_{1}\end{equation}

		Ainsi, $\vertiii{\varphi}\leqslant M$ et donc $\vertiii{\varphi}=M$.

		\item $F$ est linéaire et une isométrie d'après la question précédente, donc injective. 
		
		Soit $(K_{n})_{n\in\N}\in l^{\infty}$. On définit \function{\varphi}{l^1}{\R}{u=(u_n)_{n\in\N}}{\sum_{n=0}^{\infty}u_{n}K_{n}}
		Elle est bien définie car $\sum_{n=0}^{+\infty}\vert u_{n}\vert<+\infty$ et $(K_{n})_{n\in\N}$ est bornée. Elle est linéaire, et continue car $\vert\varphi(u)\vert\leqslant\Vert(K_{n})_{n\in\N}\Vert_{\infty}\Vert u\Vert_{1}$.

		Enfin, pour tout $n\in\N,\varphi(e_{n})=K_{n}$. Donc $F(\varphi)=(K_{n})_{n\in\N}$ et $F$ est surjective. Donc $F$ est une isométrie bijective et le dual topologique de $l^{1}$ est équivalent à $l^{\infty}$.
	\end{enumerate}
\end{proof}

\begin{proof}
	\phantom{}
	\begin{enumerate}
		\item Soit $\varphi$ une forme linéaire non nulle telle que $K=\ker(\varphi)$/ Si $F$ est dense, $\varphi$ est discontinue. Soit $(a,b)\in(E\setminus H)^{2}$ et $(x_{n})_{n\in\N}\in H^{\N}$ qui converge vers $b-a$ (existe car $H$ est dense). La suite $(a+x_{n})_{n\in\N}$ converge vers $b$. Pour $n\in\N$, on a $\varphi(a+x_{n})=\varphi(a)\neq0$, et pour $t\in[0,1]$, $\varphi(t(a+x_{n})+(1-t)(a+x_{n+1}))=\varphi(a)\neq0$. Donc $[a+x_{n},a+x_{n+1}]\subset E\setminus H$.
		
		Soit $\gamma:[0,1]\to E\setminus H$ telle que 
		\begin{equation}
		\left\{
			\begin{array}[]{rcll}
				\gamma(t) & = &\alpha_{n}t+\beta_{n}\in[a+x_{n},a+x_{n+1}]\subset E\setminus H &\text{si }t\in[1-\frac{1}{n},1-\frac{1}{n+1}]\\
				\gamma(1) & = &b&\\
				\gamma(t) &= & a+tx_{0}&\text{si }t\in[0,\frac{1}{2}]
			\end{array}
		\right.
		\end{equation}

		On cherche à définir $\alpha_{n}$ et $\beta_{n}$: on veut $\gamma(1-\frac{1}{n})=a+x_{n}$ et $\gamma(1-\frac{1}{n+1})=a+x_{n+1}$ (pour la continuité en se raccordant au $x_{n}$). En résolvant le système, on trouve $\alpha_{n}=n(n+1)(x_{n}-x_{n+1})$ et $\beta_{n}=a+x_{n}-(n-1)(n+1)(x_{n}-x_{n+1})$.

		Soit alors $\varepsilon>0$, il existe $N\in\N$ tel que pour tout $n\geqslant N\colon\Vert x_{n}+a-b\Vert<\varepsilon$ et pour tout $n\geqslant N$, pour tout $t\in[1-\frac{1}{n},1-\frac{1}{n+1}[$, $\gamma(t)\in[a+x_{n},a+x_{n+1}]\subset B(b,\varepsilon)$ par convexité de la boule. Donc $\lim\limits_{t\to 1}\gamma(t)=b$ et $\gamma$ est continue. Donc $E\setminus H$ est connexe par arcs.

		\item Soit $\varphi$ une forme linéaire telle que $\ker(f)=H$ est fermé. Alors $\varphi$ est continue (à redémontrer). Soit $x\in E\setminus H$, on a $\varphi(x)\varphi(-x)<0$ et d'après le théorème des valeurs intermédiaires, si $E\setminus H$ était connexe par arcs, $\varphi$ s'annulerait sur $E\setminus H$ ce qui n'est pas vrai. Donc $E\setminus H$ n'est pas connexe par arcs.
		
		\item Si $\K=\C$, si $H$ est dense alors $E\setminus H$ est connexe par arc d'après la première question. Si $H$ est fermé, soit $\varphi$ une forme linéaire continue telle que $\ker(f)=H$. Soit $(x_{1},x_{2})\in(E\setminus H)^{2}$. 
		
		\begin{itemize}
			\item Si $\frac{\varphi(x_{1})}{\varphi(x_{2})}\notin\R_{-}^{*}$, alors pour tout $t\in[0,1]$, $\varphi(\underbrace{tx_{1}+(1-t)x_{2}}_{\in E\setminus H})\neq0$ et on peut relier directement $x_{1}$ et $x_{2}$.
			\item Sinon, il existe $\theta\in\R,(\rho,\rho')\in(\R_{+}^{*})^{2}$ avec $\varphi(x_{1})=\rho e^{\mathrm{i}\theta}$ et $\varphi(x_{2})=\rho'e^{\mathrm{i}(\theta+\pi)}$. Alors $x_{3}=ix_{1}$ est tel que $[x_{1},x_{3}]\subset E\setminus H$ et $[x_{2},x_{3}]\subset E\setminus H$ (on contourne l'origine par une rotation de l'angle $\frac{\pi}{2}$). Par conséquent, on peut utiliser $x_{3}$ pour relier $x_{1}$ et $x_{2}$ donc $E\setminus H$ est connexe par arcs.
		\end{itemize}
	\end{enumerate}
\end{proof}

\begin{proof}
	Soit \function{\varphi}{\R_{+}^{*}}{\R}{x}{((x,\sin(\frac{1}{x})))}
	$\varphi$ est continue et $\Gamma)\varphi(\R_{+}^{*})$ est connexe par arcs.

	On a $\overline{\Gamma}=\Gamma\cup\Gamma'$ avec $\Gamma'=\{(0,y)\bigm| y\in[-1,1]\}$. En effet, pour tout $y\in[-1,1]$, on pose $x_{k}=\frac{1}{\arcsin(y)+2k\pi}$. On a $\sin(\frac{1}{x_{k}})=y\xrightarrow[k\to+\infty]{}y$ donc $(0,y)=\lim\limits_{k\to=+\infty}(x_{k},\sin(\frac{1}{x_{k}}))\in\overline{\Gamma}$.

	Réciproquement, si $(x,y)\in\overline{\Gamma}$, il existe $(x_{k})\in(\R_{+}^{*})^{\N}$ avec $x=\lim\limits_{k\to+\infty}x_{k}$ et $y=\lim\limits_{k\to+\infty}\sin(\frac{1}{x_{k}})$. Si $x>0$, par continuité, $y=\sin(\frac{1}{x})$ et $(x,y)\in\Gamma$. Si $x=0$, $y\in[-1,1]$ donc $(x,y)\in\Gamma'$.

	Si $\overline{\Gamma}$ est connexe par arcs, il existe \function{\gamma}{[0,1]}{\overline{\Gamma}}{t}{(x(t),y(t))}
	continue telle que $\gamma(0)=(0,0)$ et $\gamma(1)=(\frac{1}{\pi},0)$. La première projection $t\mapsto x(t)$ est continue avec $x(0)=0$ et $x(1)=\frac{1}{\pi}$. On définit maintenant $t_{1}=\sup\{t\in[0,1]\bigm| x(t)=0\}$. Par continuité, $x(t_{1})=0$ et donc $t_{1}<1$. Donc pour tout $t>t_{1}$, $x(t)>0$ et $\gamma(t)=(x(t),\sin(\frac{1}{x(t)}))$ pour $t>t_{1}$ et $\gamma(t_{1})=(0,y_{1})$ avec $y_{1}\in[-1,1]$.

	Or, -1 et 1 n'appartiennent pas simultanément à $]y_{1}-\frac{1}{2},y_{1}+\frac{1}{2}[$. On peut supposer que $1\notin]y_{1}-\frac{1}{2},y_{1}+\frac{1}{2}[$. Comme $\gamma$ est continue, il existe $t_{2}>t_{1}$ tel que pour tout $t\in]t_{1},t_{2}]$, $\sin(\frac{1}{x(t)})\in]y_{1}-\frac{1}{2},y_{1}+\frac{1}{2}[$. Or $x(t_{2})>0$ et $x(t_{1})=0$ donc il existe $k\in\N^{*}$, $t_{0}\in]t_{1},t_{2}]$ tel que $x(t_{0})=\frac{1}{2k\pi+\frac{\pi}{2}}$ (théorème des valeurs intermédiaires). Mais alors $\sin(\frac{1}{x(t_{0})})=1\notin]y_{1}-\frac{1}{2},y_{1}+\frac{1}{2}[$ ce qui contredit ce qui précède.

	Donc $\overline{\Gamma}$ n'est pas connexe par arcs.
\end{proof}

\begin{proof}
	\phantom{}
	\begin{enumerate}
		\item Pour tout $n\in\N$, $u_{n}\in K$ car $u_{n}$ est le barycentre de $(a,T(a),\dots,T^{n}(a))$ et $K$ est convexe. Comme $K$ est compact, on peut extraire $u_{\sigma(n)}\xrightarrow[n\to+\infty]{}u\in K$. Alors
		\begin{equation}(id_{E}-T)(u_{\sigma(n)})=\frac{1}{\sigma(n)+1}(id_{E}-T^{\sigma(n)+1})(a)\end{equation}
		d'où 
		\begin{equation}\rVert(id_{E}-T)(u_{\sigma(n)})\lVert\leqslant\frac{1}{\sigma(n)+1}\times 2M\xrightarrow[n\to+\infty]{}0\end{equation}
		avec $M=\sup\limits_{x\in K}\Vert x\Vert$ (existe car $K$ est compact donc borné). Par continuité de $T$, on a $T(u)=u$.

		\item Posons $F'=\{u\in K\bigm| T(u)=u\}$ fermé car $K'=K\cap\Bigl((\underbrace{id_{E}-T}_{\text{continu}})^{-1}\{0\}\Bigr)$.
		Donc $K'$ est compact et non vide d'après la première question. De plus, pour tout $(u_{1},u_{2})\in K'^{2}$, pour tout $t\in[0,1]$, par linéarité de $T$, on a 
		\begin{equation}T(tu_{1}+(1-t)u_{2})=tu_{1}+(1-t)u_{2}\end{equation}
		donc $K'$ convexe. De plus, comme $U\circ T=T\circ U$, pour tout $u\in K'$, on a $T(U(u))=U(T(u))=U(u)$ donc $U(u)\in K'$. On applique alors la question 1 à $K'$ est il existe $y\in K'\colon U(y)=y$ et $T(y)=y$.
	\end{enumerate}
\end{proof}

\begin{proof}
	\phantom{}
	\begin{enumerate}
		\item C'est le théorème du rang car $\rg(u)\leqslant n\leqslant p-2$, et $H=\{(\alpha_{1},\dots,\alpha_{p})\bigm|\sum_{i=1}^{p}\alpha_{i}=0\}$ est de dimension $p-1$ donc $H\cap\ker(u)\neq\{0\}$ (formule de Grassmann).
		
		\item On a 
		\begin{equation}\sum_{i=1}^{p}(\lambda_{i}+t\alpha_{i})x_{i}=\sum_{i=1}^{p}\lambda_{i}x_{i}+t\sum_{i=1}^{p}\alpha_{i}x_{i}=x\end{equation}
		et
		\begin{equation}\sum_{i=1}^{p}(\lambda_{i}+t\alpha_{i})=\sum_{i=1}^{p}\lambda_{i}+t\sum_{i=1}^{p}\alpha_{i}=1\end{equation}

		Soit $I_{+}=\{i\in\{1,\dots,p\}\bigm|\alpha_{i}>0\}$ et $I_{-}=\{i\in\{1,\dots,p\}\bigm|\alpha_{i}<0\}$. On a $I_{+}\neq\emptyset$ et $I_{-}\neq\emptyset$ car $\sum_{i=1}^{p}\alpha_{i}=0$ et $(\alpha_{1},\dots,\alpha_{p})\neq(0,\dots,0)$. Soit $t\geqslant0$. Pour tout $i\in I_{+}$, $\lambda_{i}+t\alpha_{i}\geqslant0$. Pour $i\in I_{-}$, $\lambda_{i}+t\underbrace{\alpha_{i}}_{<0}\geqslant 0$ si et seulement si $t\leqslant-\frac{\lambda_{i}}{\alpha_{i}}$. Prenons alors 
		\begin{equation}t=\min\limits_{i\in I_{-}}\Bigl(-\frac{\lambda_{i}}{\alpha_{i}}\Bigr)\end{equation}
		On au aussi pour tout $i\in I_{-}$, $\lambda_{i}+t\alpha_{i}\geqslant 0$ et il existe $i_{0}\in I_{-}$ tel que $\lambda_{i_{0}}+t\alpha_{i_{0}}=0$.

		\item Par récurrence descendante, on se ramène à n+1 points car si $x$ est barycentre de $p$ points avec $p\geqslant n+2$, alors il est barycentre de $p-1$ points.
		
		\item Soit $A=\{(\lambda_{1},\dots,\lambda_{n+1})\in\R_{+}^{n+1}\bigm|\sum_{i=1}^{n+1}\lambda_{i}=1\}$ fermé et borné en dimension finie donc compact. Soit \function{f}{A\times K^{n+1}}{\conv(K)}{((\lambda_{1},\dots,\lambda_{n}),(x_{1},\dots,x_{n+1}))}{\sum_{i=1}^{n+1}\lambda_{i}x_{i}}
		$f$ est surjective et continue, donc $\conv(K)$ est l'image continue d'un compact donc $\conv(K)$ est compact.
	\end{enumerate}
\end{proof}

\begin{proof}
	Pour tout $u\in A_{p}$, $\Sp(u)\subset\{\alpha_{1},\dots,\alpha_{r}\}$ distincts et $u$ est diagonalisable. Réciproquement, si $u$ est diagonalisable et $\Sp(u)\subset\{\alpha_{1},\dots,\alpha_{r}\}$ alors dans une base la matrice de $u$ est diagonale avec des $\alpha_{i}$ (éventuellement plusieurs selon leur multiplicités), donc $u\in A_{p}$.

	Si $u\in A_{p}$, on écrit donc le polynôme caractéristique de $u$
	\begin{equation}\chi_{u}=\prod_{i=1}^{r}(X-\alpha_{i})^{m_{i}}\end{equation}
	avec $0\leqslant m_{i}\leqslant\dim(E)=n$ et $\sum_{i=1}^{r}m_{i}=n$.
	$u\mapsto\chi_{u}$ est continue. Pour $(m_{1},\dots,m_{r})\in\{0,\dots,n\}^{r}$ tel que $\sum_{i=1}^{r}m_{i}=n$, notons 
	\begin{equation}A_{m_{1},\dots,m_{r}}=\Biggl\{u\in A_{p}\Bigm|\chi_{u}=\prod_{i=1}^{r}(X-\alpha_{i})^{m_{i}}\Biggr\}\end{equation}
	et 
	\begin{equation}\Bigl[u\mapsto\chi_u(A_{p})\Bigr]=\Biggl\{\bigcup_{(m_{1},\dots,m_{r})\in D_{n,r}}\Bigr\{\prod_{i=1}^{r}(X-\alpha_{i})^{m_{i}}\Bigr\}\Biggr\}\end{equation}
	où
	\begin{equation}D_{n,r}=\Bigl\{(m_{1},\dots,m_{r})\in\{0,\dots,n\}^{r}\Bigm|\sum_{i=1}^{r}m_{i}=n\Bigr\}\end{equation}

	Donc d'après la contraposée du théorème des valeurs intermédiaires,\\si $(m_{1},\dots,m_{r})\neq(m'_{1},\dots,m'_{r})$, alors $A_{m_{1},\dots,m_{r}}$ et $A_{m'_{1},\dots,m'_{r}}$ ne sont pas dans la même composante connexe par arcs car
	\begin{equation}\Bigl[u\mapsto\chi_u\Bigl(A_{m_{1},\dots,m_{p}}\bigcup A_{m'_{1},\dots,m'_{r}}\Bigr)\Bigr]=\underbrace{\Biggr\{\prod_{i=1}^{r}(X-\alpha_{i})^{m_{i}}\Bigr\}\Biggr\}\bigcup\Biggr\{\prod_{i=1}^{r}(X-\alpha_{i})^{m'_{i}}\Bigr\}\Biggr\}}_{\text{pas connexe par arcs}}\end{equation}
	
	Si $\gamma\colon[0,1]\to A_{p}$ est continue, $t\mapsto\chi_{\gamma(t)}=a_{0}(t)+a_{1}(t)X+\dots+a_{n-1}(t)X^{n-1}+X^{n}$ est continue sur $[0,1]$ et prend un nombre fini de valeurs donc est constante. $a_{i}\colon[0,1]\to\R$ continues et prend un nombre fini de valeurs donc est constante.

	Soit $u_{0}\in A_{m_{1},\dots,m_{r}}$, soit $u\in A_{m_{1},\dots,m_{r}}$, alors il existe une base $\mathcal{B}_{0}$ base de $E$ telle que $\mat\limits_{\mathcal{B}_{0}}(u_{0})=M_{0}$ soit diagonale avec des $\alpha_{1}$ sur les $m_{1}$ premières lignes de la diagonale, $\alpha_{2}$ sur les $m_{2}$ lignes suivantes, etc. Soit $M=\mat\limits_{\mathcal{B}_{0}}(u)$. $M$ est semblable à $M_{0}$ donc il existe $P\in GL_{n}(\C)$ telle que $M=PM_{0}P^{-1}$.

	Or $GL_{n}(\C)$ est connexe par arcs, donc il existe $\varphi\colon[0,1]\to GL_{n}(\C)$ continue telle que $\varphi(0)=P$ et $\varphi(1)=I_{n}$. On pose alors \function{\Phi}{[0,1]}{A_{m_{1},\dots,m_{r}}}{t}{\varphi(t)M_{0}\varphi^{-1}(t)}
	Alors $A_{m_{1},\dots,m_{r}}$ est connexe par arcs.

	Le nombre de composantes est donc égal au cardinal de 
	\begin{equation}D_{n,r}=\Bigl\{(m_{1},\dots,m_{r})\in\{0,\dots,n\}^{r}\Bigm|\sum_{i=1}^{r}m_{i}=n\Bigr\}\end{equation}
	qui vaut $\binom{m+r-1}{r-1}$ possibilités (place $n$ points sur une droite et les séparer avec $r-1$ barres: le nombre de points dans chaque segment donne un $m_{i}$, il y a $m+r-1$ possibilités pour placer les $r-1$ barres).
\end{proof}

\begin{proof}
	\phantom{}
	\begin{enumerate}
		\item Pour tout $i\in\{1,\dots,n\}$, $\vert AX\vert_{i}=\sum_{j=1}^{n}\underbrace{a_{i,j}x_{j}}_{>0}\geqslant0$. Si $\vert AX\vert_{i}=0$ alors pour tout $j\in\{1,\dots,n\}$, $\underbrace{a_{i,j}}_{>0}x_{j}=0$ donc $x_{j}=0$, impossible car $X\neq 0$.
		
		\item Si $\vert AX\vert=A\vert X\vert$. On a pour tout $i\in\{1,\dots,n\}$,
		\begin{equation}\Bigl\lvert\sum_{j=1}^{n}a_{i,j}x_{j}\Bigr\rvert=\sum_{j=1}^{n}a_{i,j}\lvert x_{j}\rvert\end{equation}
		donc les $(a_{i,j}x_{j})_{1\leqslant j\leqslant n}$ ont tous même argument. On prend $\theta=\arg(x_{j})$.

		\item $K$ est fermé et borné en dimension finie: c'est un compact. On a $I_{x}\neq\emptyset$ car $AX\geqslant0$ donc $0\in I_{x}$. Soit $(t_{n})_{n\in\N}\in I_{x}^{\N}$ convergeant vers $t\in\R$. Pour tout $k\in\N$, $AX-t_{k}X\geqslant0$ donc pour tout $i\in\{1,\dots,n\}$, $(AX-t_{k}X)_{i}\geqslant0$ et par passage à la limite, $AX-tX\geqslant0$ donc $I_{x}$ est fermé.
		
		Si $t\in I_{x}$, 
		\begin{equation}\vert tX\vert_{1}=t=\sum_{i=1}^{n}t\underbrace{x_{i}}_{\geqslant0}\leqslant\sum_{i=1}^{n}\underbrace{\sum_{j=1}^{n}a_{i,j}x_{j}}_{=(AX)_{i}}\leqslant n\max\limits_{1\leqslant i,j\leqslant n}\vert a_{i,j}\vert\end{equation}
		car $\sum_{j=1}^{n}x_{j}=1$.
		On note $M=n\max\limits_{1\leqslant i,j\leqslant n}\vert a_{i,j}\vert$.

		\item Pour tout $x\in K$, $\theta(X)\leqslant M$ donc $\theta$ est bien borné sur $K$. Par définition de $r_{0}$, il existe $(X_{k})_{k\in\N}\in K^{\N}$ tel que $\lim\limits_{k\to+\infty}\theta(X_k)=r_{0}$. On note $\theta(X_{k})=t_{k}$. Comme $K$ est compact, il existe $\sigma\colon\N\to\N$ strictement croissante telle que $X_{\sigma(k)}$ converge vers $X^{+}\in K$. A priori, $\theta(X^{+})\leqslant r_{0}$. On a $AX_{\sigma(k)}-t_{\sigma(k)}X_{\sigma(k)}\geqslant0$ pour tout $k\in\N$ donc par passage à la limite, $AX^{+}-r_{0}X^{+}\geqslant0$ et donc $r_{0}\leqslant\theta(X^{+})$ donc $r_{0}=\theta(X^{+})$.
		
		\item Soit $Y=A^{+}-r_{0}X^{+}\geqslant0$. Si $Y\neq0$, alors $AY>0$ d'après la question 1 donc 
		\begin{equation}AY=A\underbrace{(AX^{+})}_{>0}-r_{0}\underbrace{(AX^{+})_{>0}}>0\end{equation}
		On a $AY>\varepsilon AX^{+}$ si et seulement si pour tout $i\in\{1,\dots,n\}$, $\vert AY\vert_{i}>\varepsilon\vert AX^{+}\vert_{i}$ (car $AY>0$). On pose alors 
		\begin{equation}\varepsilon=\frac{1}{2}\min\limits_{1\leqslant i\leqslant n}\frac{\vert AY\vert_{i}}{\vert AX^{+}\vert_{i}}\end{equation}
		On a alors $AY-\varepsilon AX^{+}>0$ d'où 
		\begin{equation}A\underbrace{\frac{AX^{+}}{\Vert AX^{+}\Vert_{1}}}_{\in K}-(r_{0}+\varepsilon)\frac{AX^{+}}{\Vert AX^{+}\Vert_{1}}>0\end{equation}
		donc $r_{0}+\varepsilon\in I_{\frac{AX^{+}}{\Vert AX^{+}\Vert_{1}}}$ c'est-à-dire 
		\begin{equation}r_{0}+\varepsilon\leqslant\theta\Bigl(\frac{AX^{+}}{\Vert AX^{+}\Vert_{1}}\Bigr)\leqslant r_{0}\end{equation}
		ce qui est impossible. Nécessairement $Y=0$.

		\item Pour tout $i\in\{1,\dots,n\}$, on a 
		\begin{equation}\vert AV\vert_{i}=\Bigl\lvert\sum_{j=1}^{n}a_{i,j}v_{j}\Bigr\rvert\leqslant\sum_{i=1}^{n}a_{i,j}\vert v_{j}\vert=(A\vert V\vert)_{i}\end{equation}
		donc $\vert\lambda\vert=\vert AV\vert\leqslant A\vert V\vert$. De plus, $\vert V\vert\in K$ donc $\vert\lambda\vert\leqslant\theta(\vert V\vert)\leqslant r_{0}$. Notons que cela implique que le rayon spectral de $A$ est $\rho(A)$ est plus petit que $r_{0}$ et que l'on a même égalité.

		\item Si $\vert\lambda\vert=r_{0}$, on a $\vert\lambda\vert=\theta(\vert V\vert)=r_{0}$ et d'après la question 5 on a $A\vert V\vert=r_{0}\vert V\vert=\vert AV\vert$.
		
		D'après la question 2, il existe $\theta\in\R$ tel que $V=e^{\mathrm{i}\theta}\vert V\vert$. Or 
		\begin{equation}AV=\lambda V=e^{\mathrm{i}\theta}A\vert V\vert=e^{\mathrm{i}\theta}r_{0}\vert V\vert\end{equation}
		et comme $\vert K\vert\in K, \vert V\vert\neq0$ et on a donc $\lambda=r_{0}$.

		\item Soit $V\in\M_{n,1}(\C)$ tel que $\Vert V\Vert_{1}=1$ et $AV=r_{0}V$. D'après la question précédente, on a $V=e^{\mathrm{i}\theta}\vert V\vert$ et $A\vert V\vert=r_{0}\vert V\vert$. Soit alors $t\in\R$, on a 
		\begin{equation}A(X^{+}+t\vert V\vert)=r_{0}(X^{+}+t\vert V\vert)\end{equation}
		Notons maintenant que si $Y\geqslant0$ avec $Y\neq0$ vérifie $AY=r_{0}Y$, alors $Y>0$. En effet, d'après la première question, $AY>0$. On a $r_{0}\neq0$ car sinon $\Sp_{\C}=\{0\}$ et $A^{n}=0$ ce qui est impossible car ses coefficients sont strictement positifs. D'où $Y>0$.

		Ainsi, par définition de $X^{+}$, on a $X^{+}>0$ et $\vert V\vert>0$. On a alors 
		\begin{equation}(X^{+})_{i}+t\vert v_{i}\vert\geqslant0\end{equation}
		si et seulement si
		\begin{equation}t\geqslant -\frac{\vert X^{+}\vert_{i}}{\vert v_{i}\vert}\end{equation}
		On prend 
		\begin{equation}t=\min\limits_{1\leqslant i\leqslant n}-\frac{\vert X^{+}\vert_{i}}{\vert v_{i}\vert}\end{equation}
		Finalement, on a $X^{+}+t\vert V\vert\geqslant0$ et une de ses coordonnées vaut 0 (car on a pris le minimum sur les $i$). Nécessairement, $X^{+}+t\vert V\vert=0$ (car $A(X^{+}+t\vert V\vert)=r_{0}(X^{+}+t\vert V\vert)$) et donc $\vert V\vert\in\R X^{+}$. Donc $V=e^{\mathrm{i}\theta}\vert V\vert\in\C X^{+}$ et ainsi 
		\begin{equation}\dim(\ker(A-r_{0}I_{n}))=1\end{equation}
	\end{enumerate}
\end{proof}

\begin{proof}
	Soit \function{\varphi}{U\times V}{\R}{(x,y)}{\Vert x-y\Vert}
	On a 
	\begin{equation}\vert\varphi(x,y)-\varphi(x',y')=\vert\Vert x-y\Vert-\Vert x'-y'\Vert\vert\leqslant\Vert (x-y)-(x'-y')\Vert\leqslant\Vert x-x'\Vert+\Vert y-y'\Vert\leqslant2\Vert(x,y)-(x',y')\Vert_{\infty}\end{equation}
	donc $\varphi$ est continue.

	$U\times V$ est compact, donc il existe $(x_{1},y_{1})\in(U\times V)$ telle que $\varphi(x_{1},y_{1})=\min\limits_{(x,y)\in U\times V}\varphi(x,y)$. Comme $U$ et $V$ sont disjoints, $x_{1}\neq y_{1}$ et $\varphi(x_{1},y_{1}))d(U,V)>0$.

	Soit $\alpha=\frac{d(U,V)}{3}$. On pose $U'=\{x\in E\Bigm|d(x,U)<\alpha\}$ et $V'=\{x\in E\Bigm|d(x,V)<\alpha\}$. $x\mapsto\Vert x\Vert$ est continue car $1$-lipschitzienne donc $U'$ est $V'$ sont des ouverts et on a bien $U\subset U'$ et $V\subset V'$. Soit ensuite $x\in U'\cap V'$, on a $d(x,U)<\alpha$ et $d(x,V)<\alpha$ donc il existe $(u,v)\in U\times V$, $d(x,u)<\alpha$ et $d(x,v)<\alpha$. Alors $d(u,v)\leqslant2\alpha$ ce qui est absurde. Donc $U'\cap V'=\emptyset$.
\end{proof}

\begin{proof}
	\phantom{}
	\begin{enumerate}
		\item $f$ est 1-lipschitzienne donc est continue. On forme \function{g}{K}{\R}{x}{\Vert x-f(x)\Vert}
		$g$ est continue, $K$ est compact donc il existe $a\in K$ tel que $g(a)=\min_{x\in K}g(x)$. Si $a\neq f(a)$, alors $\Vert f(a)-f^{2}(a)\Vert=g(f(a))<\Vert a-f(a)\Vert=g(a)$ ce qui est impossible par définition de $a$. Donc $f(a)=a$. S'il existe $a'\neq a$ tel que $f(a')=a'$, alors $\Vert f(a)-f(a')\Vert=\Vert a-a'\Vert<\Vert a-a'\Vert$ ce qui est impossible. Donc $a$ est unique.

		\item S'il existe $n_{0}\in\N$ tel que $u_{n_{0}}=a$ alors pour tout $n\geqslant n_{0}$, $u_{n}=a$ et $\lim\limits_{n\to+\infty}u_{n}=a$. Si pour tout $n\in\N$, $u_{n}\neq a$, alors pour tout $n\in\N$, on a
		\begin{equation}\Vert u_{n+1}-a\Vert=\Vert f(u_{n})-f(a)\Vert<\Vert u_{n}-a\Vert\end{equation}
		donc la suite $(\Vert u_{n}-a\Vert)_{n\in\N}$ est strictement décroissante dans $\R_{+}$ donc elle converge vers $l\geqslant0$. Par compacité de $K$, il existe une extraction $\sigma$ telle que $\lim\limits_{n\to+\infty}u_{\sigma(n)}=\alpha\in K$. Par continuité, \begin{equation}\lim\limits_{n\to+\infty}\Vert u_{\sigma(n)}-a\Vert=\Vert\alpha-a\Vert=l\end{equation} 
		et
		\begin{equation}\lim\limits_{n\to+\infty}\Vert \underbrace{u_{\sigma(n)+1}}_{f(u_{\sigma(n)}}-f(a)\Vert=\Vert f(\alpha)-f(a)\Vert=l=\Vert\alpha-a\Vert\end{equation}
		par continuité de $f$.
		Ainsi, on a $\alpha=a$ et $l=0$ donc $\lim\limits_{n\to+\infty}u_{n}=a$.

		\item $f$ est $\mathcal{C}^{1}$ sur $\R$. Soit $x<y\in\R^{2}$, il existe $z\in]x,y[$ tel que (égalité des accroissements finis)
		\begin{equation}\Bigl\lvert\frac{f(x)-f(y)}{x-y}\Bigr\rvert=\vert f'(z)\vert=\Bigl\lvert\frac{z}{\sqrt{z^{2}+1}}\Bigr\rvert<1\end{equation}
		donc $f$ vérifie bien l'hypothèse de contraction. Cependant, pour tout $a\in\R$, on a $\sqrt{a^{2}+1}>a$ donc pas de point fixe. La démonstration tombe en défaut car $\R$ n'est pas compact.
	\end{enumerate}
\end{proof}

\begin{proof}
	La condition est équivalente à pour tout $(M_{1},M_{2},M_{3})\in K_{1}\times K_{2}\times K_{3}$, $M_{1},M_{2}$ et $M_{3}$ ne sont pas alignés.\\
	On forme alors \function{f}{K_1\times K_2\times K_3}{\R_+}{(M_1,M_2,M_3)}{R(M_1,M_2,M_3)}
	où $R(M_{1},R_{2},M_{3})$ est le rayon du cercle circonscrit au triangle formé par $M_{1},M_{2}$ et $M_{3}$.

	On note $M_{i}=(x_{i},y_{i})$ et $\Delta_{i}$ la médiatrice de $[M_{j}M_{k}]$. Établissons une équation de $\Delta_{i}$. On a $M=(x,y)\in\Delta_{i}$ si et seulement si $\Vert \vec{MM_{j}}\Vert_{2}^{2}=\Vert\vec{MM_{k}}\Vert_{2}^{2}$ si et seulement si $(\vec{MM_{j}}+\vec{MM_{k}}\bigm|\vec{MM_{j}}-\vec{MM_{k}})=0$ (produit scalaire), si et seulement si $(\vec{MC_{i}}\bigm|\vec{M_{j}M_{k}})=0$ où $C_{i}$ est le milieu de $[M_{j}M_{k}]$, si et seulement si (calculer le produit scalaire)
	\begin{equation}\Bigl(\frac{x_{j}+x_{k}}{2}-x\Bigr)(x_{k}-x_{j})+\Bigl(\frac{y_{j}+y_{k}}{2}-y\Bigr)(y_{k}-y_{j})=0\end{equation}
	Soit alors $M_{0}=(x_{0},y_{0})$ le centre du cercle circonscrit. $M_{0}\in\Delta_{i}\cap\Delta_{j}$ avec $i\neq j$. Par exemple, $M_{0}\in\Delta_{3}\cap\Delta_{1}$ si et seulement si
	\begin{equation}
	\left\{
		\begin{array}[]{rcl}
			\Bigl(\dfrac{x_{2}+x_{1}}{2}-x_{0}\Bigr)(x_{2}-x_{1})+\Bigl(\dfrac{y_{2}+y_{1}}{2}-y_{0}\Bigr)(y_{2}-y_{1}) &= &0\\[0.5cm]
			\Bigl(\dfrac{x_{3}+x_{2}}{2}-x_{0}\Bigr)(x_{3}-x_{2})+\Bigl(\dfrac{y_{3}+y_{2}}{2}-y_{0}\Bigr)(y_{3}-y_{2}) &= &0
		\end{array}	
	\right.
	\end{equation}
	si et seulement si ($L_{2}\leftarrow L_{1}(x_{3}-x_{2})+L_{2}(x_{1}-x_{2})$)
	\begin{equation}
	\left\{
		\begin{array}[]{rcl}
			x_{0}(x_{1}-x_{2})+y_{0}(y_{1}-y_{2})&=&\dfrac{x_{1}^{2}-x_{2}^{2}+y_{1}^{2}-y_{2}^{2}}{2}\\[0.5cm]
			x_{0}(x_{2}-x_{3})+y_{0}(y_{2}-y_{3})&=&\dfrac{x_{2}^{2}-x_{3}^{2}+y_{2}^{2}-y_{3}^{2}}{2}
		\end{array}	
	\right.
	\end{equation}
	si et seulement si ($L_{1}\leftarrow L_{2}(y_{2}-y_{1})+L_{1}(y_{2}-y_{3})$)
	\begin{equation}
	\left\{
		\begin{array}[]{rcl}
			x_{0} &= & \dfrac{\frac{x_{1}^{2}-x_{2}^{2}+y_{1}^{2}-y_{2}^{2}}{2}(y_{2}-y_{3})-(y_{1}-y_{2})\frac{x_{2}^{2}-x_{3}^{2}+y_{2}^{2}-y_{3}^{2}}{2}}{(x_{1}-x_{2})(y_{2}-y_{3})-(x_{2}-x_{3})(y_{1}-y_{2})}\\[0.5cm]
			y_{0} &= & \dfrac{\frac{x_{2}^{2}-x_{3}^{2}+y_{2}^{2}-y_{3}^{2}}{2}(x_{1}-x_{2})-(x_{2}-x_{3})\frac{x_{1}^{2}-x_{2}^{2}+y_{1}^{2}-y_{2}^{2}}{2}}{(x_{1}-x_{2})(y_{2}-y_{3})-(x_{2}-x_{3})(y_{1}-y_{2})}
		\end{array}	
	\right.
	\end{equation}
	et $R(M_{1},M_{2},M_{3})=\sqrt{(x_{0}-x_{3})^{2}+(y_{0}-y_{3})^{2}}$. En reportant, $f$ est continue sur $K_{1}\times K_{2}\times K_{3}$ compact donc $f$ atteint son minimum.
\end{proof}

\begin{proof}
	\phantom{}
	\begin{enumerate}
		\item Pour tout $f\in E$, $T(f)$ est $\mathcal{C}^{1}$ et $(T(f))'=f$, $T(f)(0)=0$. $T$ est clairement linéaire, soit ensuite $x\in[0,1]$, on a 
		\begin{equation}\vert T(f)(x)\vert=\Bigl\lvert\int_{0}^{x}f(t)dt\Bigr\rvert\leqslant\int_{0}^{x}\vert f(t)\vert dt\leqslant x\Vert f\Vert_{\infty}\leqslant\Vert f\Vert_{\infty}\end{equation}
		Donc $\Vert T(f)\Vert_{\infty}\leqslant\Vert f\Vert_{\infty}$ donc $T$ est continue et $\vertiii{T}\leqslant1$. Pour $f=1$, on a $\Vert f\Vert_{\infty}=1$ et pour tout $x\in[0,1]$, $T(f)(x)=x$ donc $\Vert T(1)\Vert_{\infty}=1$. Ainsi, $\vertiii{T}=1$.

		\item $id_{E}-T$ est continue. Soit $(f,g)\in E^{2}$, on a $g=f-T(f)$ si et seulement si $g=y'-y$ et $y(0)=0$. 
		On a $g(x)e^{-x}=\underbrace{e^{-x}(y'(x)-y(x))}_{(e^{-x}y(x))'}$ donc en intégrant de 0 à $x$ on a 
		\begin{equation}y(x)=e^{x}\int_{0}^{x}e^{-t}g(t)dt\end{equation}
		Donc $T(f)$ vérifie le problème de Cauchy si et seulement si pour tout $x\in\R$, $T(f)(x)=e^{x}\int_{0}^{x}e^{-t}g(t)dt$ si et seulement si pour tout $x\in[0,1]$, 
		\begin{equation}f(x)=g(x)+e^{x}\int_{0}^{x}e^{-t}g(t)dt\end{equation}
		Donc $id_{E}-T$ est bijective. 
		Enfin, on a pour tout $x\in[0,1]$, 
		\begin{equation}\vert f(x)\vert\leqslant\vert g(x)\vert+\Bigl\lvert\int_{0}^{x}g(t)e^{x-t}dt\Bigr\rvert\leqslant\Vert g\Vert_{\infty}(1+xe^{x})\leqslant\Vert g\Vert_{\infty}(1+e)\end{equation}
		Ainsi, 
		\begin{equation}\Vert f\Vert_{\infty}=\Vert(id_{E}-T)^{-1}(g)\Vert_{\infty}\leqslant\Vert g\Vert_{\infty}(1+e)\end{equation}
		donc $(id_{E}-T)^{-1}$ est continue. Ainsi, $id_{E}-T$ est un homéomorphisme.
	\end{enumerate}
\end{proof}

\begin{proof}
	\phantom{}
	\begin{enumerate}
		\item [(i) $\Rightarrow$ (ii)] $f^{-1}(K)$ est fermé car $f$ est continue. $K$ est borné, donc il existe $M>0$, tel que pour tout $y\in K$, $\Vert y\Vert\leqslant M$. Donc pour tout $x\in f^{-1}(K)$, $\Vert f(x)\Vert\leqslant M$. Par contraposée de (i) pour $A=M+1$, il existe $B>0$ tel que $\Vert f(x)\Vert<A\Rightarrow\Vert x\Vert<B$. Donc pour $x\in f^{-1}(K)$, $\Vert x\Vert<B$ donc $f^{-1}(K)$ est borné. C'est donc un compact.
		\item [(ii) $\Rightarrow$ (i)] Soit $A\geqslant0$. Soit $K=\overline{B(0,A)}$ compact car fermé et borné en dimension finie. D'après (ii), $f^{-1}(K)$ est compact donc borné: il existe $B>0$ tel que pour tout $x\in f^{-1}(K)$, $\Vert x\Vert\leqslant B$. Par contraposée, si $\Vert x\Vert>B$ alors $x\notin f^{-1}(K)$ et $f(x)\notin K$ donc $\Vert f(x)\Vert >A$. Ainsi, $\lim\limits_{\Vert x\Vert\to+\infty}\Vert f(x)\Vert=+\infty$.
	\end{enumerate}
\end{proof}

\begin{remark}
	Exemple pour l'exercice précédent: les fonctions polynômiales non constantes. Contre-exemple: l'exponentielle, cf $\exp([0,1])=\R_{-}$ non compact.
\end{remark}

\begin{proof}
	\phantom{}
	\begin{enumerate}
		\item 
		Soit $(x,y)\in K^{2}$ compact. Soit $\sigma$ un extraction telle que 
		\begin{equation}(f^{\sigma(n)}(x),f^{\sigma(n)}(y))\xrightarrow[n\to+\infty]{}(l,l')\in K^{2}\end{equation}
		On a 
		\begin{equation}f^{\sigma(n+1)}(x)-f^{\sigma(n)}(x)\xrightarrow[n\to+\infty]{}0\end{equation}
		de même pour $y$. Soit $\varepsilon>0$,
		\begin{equation}
		\left\{
			\begin{array}[]{l}
				\exists N_{1}\in\N,\forall n\geqslant N_{1},\Vert f^{\sigma(n+1)}(x)-f^{\sigma(n)}(x)\Vert\leqslant\varepsilon\\
			\exists N_{1}\in\N,\forall n\geqslant N_{1},\Vert f^{\sigma(n+1)}(y)-f^{\sigma(n)}(y)\Vert\leqslant\varepsilon
		\end{array}
		\right.
		\end{equation}
		Pour $N=\max(N_{1},N_{2})$ et $p=\sigma(N+1)-\sigma(N)\in\N^{*}$, on a 
		\begin{equation*}
			d(x,f^{p}(x))\leqslant d(f^{\sigma(n+1)}(x),f^{\sigma(n)}(x))\leqslant\varepsilon
		\end{equation*}
		et de même pour $y$ avec le même $p$.
	
		\item On a 
		\begin{align}
			d(x,y)
			&\leqslant d(f(x),f(y))\\
			&\leqslant d(f^{p}(x),f^{p}(y))\\
			&\leqslant d(f^{p}(x),x)+d(x,y)+d(y,f^{p}(y))\\
			&\leqslant 2\varepsilon+d(x,y)
		\end{align}

		Ceci valant pour tout $\varepsilon>0$, on a égalité tout du long. On a donc notamment, $\Vert x-y\Vert=\Vert f(x)-f(y)\Vert$ et donc $f$ est une isométrie.

		\item $f$ est 1-lipschitzienne donc continue. Donc $f(K)$ est compact donc fermé. Il suffit donc de montrer que $f(K)$ est dense dans $K$. Soit $x\in K$ et $\varepsilon>0$, il existe $p\in\N^{*}$ tel que $\Vert x-\underbrace{f^{p}(x)}_{\in f(K)}\Vert\leqslant\varepsilon$ d'après la première question. Donc $f(K)$ est dense dans $K$ et $f(K)=\overline{f(K)}=K$.
	\end{enumerate}
\end{proof}

\begin{remark}
	Exemple pour l'exercice précédent: une rotation sur la sphère unité.
\end{remark}

\begin{proof}
	Soit \function{f}{K}{\R}{M}{f(M)=\text{rayon du cercle circonscrit au triangle MAB}}
	On a $F=f(K)$. Soit $(C,i,j)$ un repère orthonormé où $C$ est le milieu de $[AB]$ et $A(-\alpha,0)$ et $B(\alpha,0)$ avec $\alpha>0$. La médiatrice $\Delta$ de $[A,B]$ a pour équation $x=0$. Si $M(x,y)$, soit $\varphi(M)$ le centre du cercle circonscrit. On a $\varphi(M)\in\Delta$ donc $\varphi(M)(0,y_{1})$ et $\varphi(M)$ appartient à la médiatrice de $[MA]$. On a $y_{1}\neq0$ car $M\notin(AB)$.

	Notons $M'$ le milieu de $[MA]$. On a $M'(\frac{x-\alpha}{2},\frac{y}{2})$ d'où $\vec{M'\varphi(M)}\cdot\vec{MA}=0$ d'où (en développant le produit scalaire),
	\begin{equation}y_{1}=\Bigl((\alpha+x)\Bigl(\frac{\alpha-x}{2}\Bigr)-\frac{y^{2}}{2}\Bigr)\Bigl(-\frac{1}{y}\Bigr)\end{equation}
	$\varphi$ est donc continue donc $f$ également et $f(K)=F$ est compact.
\end{proof}

\begin{proof}
	\phantom{}
	\begin{enumerate}
		\item Soit $\lambda\in\Sp(\tau)$ et $P\in\R[X]\setminus\{0\}$ avec $\tau(P)=\lambda P$. Si $P$ n'est pas constant, notons $\alpha\in\C$ alors $P(\alpha)=0$. Alors $P(\alpha+1)=0$. En itérant, pour tout $n\in\N$, $P(\alpha+n)=0$, impossible car $P$ n'est pas constant donc pas nul. Finalement, $P$ est constant et $\lambda=1$: $\Sp(\tau)=\{1\}$.
		\item $f\colon x\mapsto P(x)e^{-x}$ est continue et $\lim\limits_{x\to+\infty}f(x)=0$ donc le $\sup$ est bien défini. Il est ensuite facile de vérifier que $\Vert P\Vert$ est une norme.
		\item On a 
		\begin{equation}\Vert\tau(P)\Vert=\sup\limits_{x\geqslant0}\vert P(x+1)e^{-x}\vert=\sup\limits_{x'\geqslant1}\vert P(x')e^{-x'}e\vert\leqslant\sup\limits_{x'\geqslant0}\vert P(x')e^{-x'}e\vert\leqslant e\Vert P\Vert\end{equation}
		\item Utiliser $P=X$.
	\end{enumerate}
\end{proof}

\begin{proof}
	\phantom{}
	\begin{enumerate}
		\item Pour $x$ fixé, $\min(x,\varphi(t))=\frac{x+\varphi(t)-\vert x-\varphi(t)\vert}{2}$ est continue. Donc $T(f)$ est définie.
		
		Si $x\leqslant\varphi(0)$,
		\begin{equation}T(f)(x)=\int_{0}^{1}xf(t)dt=x\int_{0}^{1}f(t)dt\end{equation}
		et si $x\geqslant\varphi(1)$,
		\begin{equation}T(f)(x)=\int_{0}^{1}\varphi(t)f(t)dt\end{equation} 
		et si $\varphi(0)\leqslant x\leqslant\varphi(1)$, il existe un unique $t_{1}=\varphi^{-1}(x)$ (car $\varphi$ induit un homéomorphisme de $[0,1]$ dans $\varphi([0,1])$). 
		
		Si $t\leqslant t_{1}$, on a $\varphi(t)\leqslant x$, donc $\min(x,\varphi(t))=\varphi(t)$. Si $t\geqslant t_{1}$, on a $\min(x,\varphi(t))=x$. On a donc 
		\begin{align}
			T(f)(x)
			&=\int_{0}^{t_{1}}\varphi(t)f(t)dt+\int_{t_{1}}^{1}xf(t)dt\\
			&=\underbrace{\int_{0}^{\varphi^{-1}(x)}\varphi(t)f(t)dt}_{=F_{1}(\varphi^{-1}(x))}+x\underbrace{\int_{\varphi^{-1}(x)}^{1}f(t)dt}_{=F_{2}(\varphi^{-1}(x))}
		\end{align}
		et $f$ et $\varphi$ étant continues, $F_{1}$ et $F_{2}$ sont continues.

		Donc $T(f)$ continue et $T$ linéaire, c'est un endomorphisme de $E$.

		\item On a 
		\begin{equation*}
			\vert T(f)(x)\vert\leqslant\Vert f\Vert_{\infty}\underbrace{\int_{0}^{1}\min(x,\varphi(t))dt}_{=A(x)}
		\end{equation*}
		donc 
		\begin{equation}\Vert T(f)\Vert_{\infty}\leqslant\Vert f\Vert_{\infty}\Vert A\Vert_{\infty}\end{equation}
		donc $T$ est continue et $\vertiii{T}\leqslant\Vert A\Vert_{\infty}$. De plus pour $f=1$, on a $\vertiii{T}=\Vert A\Vert_{\infty}$.

		\item On a 
		\begin{equation}
		A(x)=\int_{0}^{1}\min(x,\varphi(t))dt=
		\left\{
			\begin{array}[]{lll}
				x & \text{si} & x\leqslant\varphi(0)\\
				\int_{0}^{1}\varphi(t)dt & \text{si} & x\geqslant\varphi(1)
			\end{array}
		\right.
		\end{equation}
		Dans tous les cas, 
		\begin{equation}\Vert A\Vert_{\infty}\leqslant\int_{0}^{1}\varphi(t)dt\end{equation}
		donc 
		\begin{equation}\Vert A\Vert_{\infty}=\int_{0}^{1}\varphi(t)dt\end{equation}
	\end{enumerate}
\end{proof}

\begin{proof}
	\phantom{}
	\begin{enumerate}
		\item $\varphi$ est une forme linéaire. et on a 
		\begin{equation}\vert\varphi(P)\vert\leqslant\sum_{k\in\N}\Bigl\vert\frac{a_k}{2^{k}}\Bigr\vert\leqslant2\Vert P\vert_{\infty}\end{equation}
		donc $\varphi$ est continue et $\vertiii{\varphi}\leqslant2$. Pour $p\neq0$, $\vert\varphi(P)\vert<2\Vert P\Vert_{\infty}$ : pour avoir égalité, il faudrait pour tout $k\in\N$, $a_{k}=\text{constante}\neq0$ ce qui n'est pas possible. Pour $P_{n}=\sum_{k=0}^{n}X^{k}$, on a $\Vert P_{n}\Vert_{\infty}=1$ et $\lim\limits_{n\to+\infty}\vert\varphi(P_{n})\vert\xrightarrow[n\to+\infty]{}2$ donc $\vertiii{\varphi}=2$. De plus, $\ker(\varphi)=\varphi^{-1}(\{0\})$ est fermé.

		\item Soit $P=\sum_{k\in\N}a_{k}X^{k}\in\ker(\varphi)$. On a $\varphi(P)=0$ d'où $a_{0}=-\sum_{k=1}^{+\infty}\frac{a_{k}}{2^{k}}$ (et il existe $N_{0}\in\N,\forall n\geqslant N_{0},a_{n}=0$). On a donc 
		\begin{equation}P(X)-1=(a_{0}-1)+\sum_{k\in\N^{*}}a_{k}X^{k}\end{equation}
		et si $\Vert P-1\Vert_{\infty}\leqslant\frac{1}{2}$, on a 
		\begin{equation}
		\left\{
			\begin{array}[]{l}
				\vert a_{0}-1\vert\leqslant\frac{1}{2}\\
				\forall k\in\N^{*},\vert a_{k}\vert\leqslant\frac{1}{2}
			\end{array}
		\right.
		\end{equation}
		et 
		\begin{equation}\vert a_{0}\vert=\Biggl\vert\sum_{k=1}^{+\infty}\frac{a_{k}}{2^{k}}\Biggr\vert\leqslant\sum_{k=1}^{+\infty}\frac{\vert a_{k}\vert}{2^{k}}\leqslant\sum_{k=1}^{+\infty}\frac{1}{2^{k+1}}=\frac{1}{2}\end{equation}

		Et $\frac{1}{2}\leqslant 1-\vert a_{0}\vert\leqslant\vert 1-a_{0}\vert\leqslant\frac{1}{2}$. Donc $\vert a_{0}\vert=\frac{1}{2}$ et $\vert 1-a_{0}\vert=\frac{1}{2}$.
		\begin{align}
			a_{0}=\frac{1}{2}e^{\mathrm{i}\theta}
			&\Rightarrow \Bigl\vert 1-\frac{1}{2}e^{\mathrm{i}\theta}\Bigr\vert^{2}=\frac{1}{4}\\
			&\Rightarrow \Bigl(1-\frac{1}{2}\cos(\theta)\Bigr)^{2}+\Bigl(\frac{1}{2}\sin(\theta)\Bigr)^{2}=\frac{1}{4}\\
			&\Rightarrow 1-\cos(\theta)+\frac{1}{4}=\frac{1}{4}\\
			&\Rightarrow \cos(\theta)=1
		\end{align}
		et donc $a_{0}=\frac{1}{2}$.

		Par ailleurs, on a 
		\begin{equation}\frac{1}{2}=\sum_{k=1}^{+\infty}\frac{\vert a_{k}\vert}{2^{k}}=\sum_{k=1}^{+\infty}\frac{1}{2^{k+1}}\end{equation}
		Donc pour tout $k\in\N$, $\vert a_{k}\vert=\frac{1}{2}$, impossible car $P\in\C[X]$, ainsi $\Vert P-1\Vert_{\infty}>\frac{1}{2}$.

		\item On définit, pour $n\geqslant1$, $P_{n}=\frac{1}{2}+\sum_{k=1}^{n}(-\frac{1}{2}+\varepsilon_{n})X^{k}$ avec $\varepsilon_{n}\in\R$ tel que $P_{n}\in\ker(\varphi)$. On a 
		\begin{align}
			P_{n}\in\ker(\varphi)
			&\Rightarrow\frac{1}{2}+\sum_{k=1}^{n}\Bigl(-\frac{1}{2}+\varepsilon_{n}\Bigr)\frac{1}{2^{k}}=0\\
			&\Rightarrow\varepsilon_{n}=-\frac{1}{2^{n+1}}\times \frac{1}{1-\frac{1}{2^{n}}}
		\end{align}
		et donc $\varepsilon_{n}\xrightarrow[n\to+\infty]{}0$ (et $\varepsilon_{n}<0$). On a donc $\Vert P_{n}-1\Vert_{\infty}=\frac{1}{2}-\varepsilon_{n}\xrightarrow[n\to+\infty]{}\frac{1}{2}$.

		Donc $d(1,\ker(\varphi))=\frac{1}{2}$ et cette distance n'est pas atteinte.
	\end{enumerate}
\end{proof}

\begin{proof}
	Prouvons d'abord l'existence. Soit $M\in\R^{n}$, on définit $r(M)=\sup\{\Vert M-A\Vert\bigm| A\in K\}$ et $\varphi\colon A\mapsto\Vert M-A\Vert$ est continue sur $K$ compact donc le sup est en fait un max. On a notamment $r(M)=\{R>0\bigm| K\subset B(M,R)\}$. Soit \function{r}{\R^n}{\R}{M}{r(M)}
	Soit $(M,M')\in(\R^{n})^{2}$. Pour tout $A\in K$, on a 
	\begin{equation}\Vert M-A\Vert\leqslant\Vert M-M'\Vert+\Vert M'-A\Vert\leqslant\Vert M-M'\Vert +r(M')\end{equation}
	En particulier, on a
	\begin{equation}r(M)\leqslant\Vert M-M'\Vert+r(M')\end{equation}
	et en échangeant $M$ et $M'$, on a $\vert r(M)-r(M')\vert\leqslant\Vert M-M'\Vert$. Donc $r$ est 1-lipschitzienne donc continue. Soit $A_{0}\in K$, $R(M)\geqslant\Vert M-A_{0}\Vert\geqslant\Vert M\Vert-\Vert A_{0}\Vert\xrightarrow[\Vert M\Vert\to+\infty]{}+\infty$. Donc il existe $M_{0}\in\R^{n}$ tel que $r(M_{0})=\min\limits_{M\in\R^{n}}r(M)=r_{0}$, d'où l'existence d'une boule fermée de rayon minimal.

	Pour l'unicité, soit $(M_{1},M_{2})\in(\R^{n})^{2}$ tel que $r(M_{1})=r(M_{2})=r_{0}$. On suppose que $\Vert M_{1}-M_{2}\Vert=\varepsilon>0$. Soit $M_{3}$ le milieu de $[M_{1}M_{2}]$. On a $K\subset B_{M_{1},r_{0}}\cap B_{M_{2},r_{0}}$. On prend $r^{2}+\bigl(\frac{\varepsilon}{2}\bigr)^{2}=r_{0}^{2}$ d'où 
	\begin{equation}r=\sqrt{r_{0}^{2}-\frac{\varepsilon^{2}}{4}}<r_{0}\end{equation}
	Soit $M\in B(M_{1},r_{0})\cap B(M_{2},r_{0})$, on a 
	\begin{align}
		\Vert M-M_{3}\Vert^{2}
		&=\frac{1}{4}\Bigl(\Vert M-M_{1}+M-M_{2}\Vert^{2}\Bigr)\\
		&=\frac{1}{4}\Bigl(2\Vert M-M_{1}\Vert^{2}+2\Vert M-M_{2}\Vert^{1}-\underbrace{\Vert M_{1}-M_{2}\Vert^{2}}_{=\varepsilon^{2}}\Bigr)\\
		&\leqslant\frac{1}{4}(2r_{0}^{2}+2r_{0}^{2}-\varepsilon^{2})\\
		&\leqslant r_{0}^{2}-\frac{\varepsilon^{2}}{4}=r^{2}
	\end{align}
	Donc $B_{1}\cap B_{2}\subset\overline{B(M_{3},r)}$ d'où $K\subset\overline{B(M_{3},r)}$, ce qui est absurde car $r<r_{0}$. Donc $M_{1}=M_{2}$.
\end{proof}

\begin{proof}
	$\varphi$ est évidemment définie et linéaire. Soit $f\in\mathcal{C}^{0}([0,1],\R)$.
	\begin{align}
		\vert\varphi(f)\vert
		&=\Biggl\vert\int_{0}^{\frac{1}{2}}f-\int_{\frac{1}{2}}^{1}f\Biggr\vert\\
		&\leqslant\Biggl\vert\int_{0}^{\frac{1}{2}}f\Biggr\vert+\Biggl\vert\int_{\frac{1}{2}}^{1}f\Biggr\vert\\
		&\leqslant\int_{0}^{\frac{1}{2}}\vert f\vert+\int_{\frac{1}{2}}^{1}\vert f\vert\\
		&\leqslant\int_{0}^{1}\Vert f\Vert_{\infty}=\Vert f\Vert_{\infty}
	\end{align}
\end{proof}

Donc $\varphi$ est continue et $\vertiii{\varphi}\leqslant1$. Notons que si l'on a $\vert\varphi(f)\vert=\Vert f\Vert_{\infty}$, alors on a égalité partout au-dessus et pour tout $t\in[0,1]$, $\vert f(t)\vert=\Vert f\Vert_{\infty}$ et comme $\Bigl\vert\int f\Bigr\vert=\int\vert f\vert$ implique que $f$ est de signe constant sur l'intervalle d'intégration, si l'on a $\vert\varphi(f)\vert=\Vert f\Vert_{\infty}$, alors $f$ est de signe constant sur $[0,\frac{1}{2}]$ et sur $[\frac{1}{2},1]$.  Or $\vert\int_{0}^{\frac{1}{2}}f-\int_{\frac{1}{2}}^{1}f\vert=\vert\int_{0}^{\frac{1}{2}}f\vert+\vert\int_{\frac{1}{2}}^{1}f\vert$, $f$ est de signe opposé sur les deux segments. Or $f$ est continue en $\frac{1}{2}$, donc $f$ est nulle. Donc pour $f$ non nulle, on a $\vert\varphi(f)\vert<\Vert f\Vert_{\infty}$ donc la norme triple n'est pas atteinte. Enfin, pour montrer que $\vertiii{\varphi}=1$, on utilise pour $n\geqslant1$,
\begin{equation}
f_{n}(t)=
\left\{
	\begin{array}[]{lll}
		1 & \text{si} & t\in[0,\frac{1}{2}-\frac{1}{n}]\\[0.3cm]
		(\frac{1}{2}-t)n & \text{si} & t\in[\frac{1}{2}-\frac{1}{n},\frac{1}{2}+\frac{1}{n}]\\[0.3cm]
		-1 & \text{si} & t\in[\frac{1}{2}+\frac{1}{n},1]
	\end{array}
\right.
\end{equation}
On a bien $\Vert f_{n}\Vert_{\infty}=1$.

\begin{proof}
	\phantom{}
	\begin{enumerate}
		\item Non car on applique l'application trace.
		\item On a le résultat par récurrence.
		\item On a 
		\begin{equation}(n+1)\vertiii{v^{n}}=\vertiii{u\circ v^{n}\circ v-v^{n}\circ v\circ r}\leqslant 2\vertiii{u}\vertiii{v}\vertiii{v^{n}}\end{equation}
		Si pour tout $n\in\N$, on a $v^{n}=0$, alors pour tout $n\in\N$,
		\begin{equation}n+1\leqslant 2\vertiii{u}\vertiii{v}\end{equation}
		ce qui est impossible. Donc il existe $n\in\N^{*}$ tel que $v^{n}=0$. Alors $u\circ v^{n}-v^{n}\circ u=nv^{n-1}=0$ donc $v^{n-1}=0$ et de proche en proche $v=0$: contradiction.
		\item Pour tout $P\in\R[X]$, 
		\begin{equation}(D\circ T-T\circ D)(P)=(XP)'-XP'=P\end{equation}
		donc $D\circ T-T\circ D=id$. D'après ce qui précède, $T$ et $D$ ne peuvent pas être continus simultanément.
	\end{enumerate}
\end{proof}

\begin{proof}
	\phantom{}
	\begin{enumerate}
		\item $\sum_{k\geqslant0}(A-I_{n})^{k}$ converge absolument car $\vertiii{A-I_{n}}^{k}\leqslant\alpha_{k}$ et $\alpha<$.
		
		Si $AX=0$, $\Vert (A-I_{n})X\Vert=\Vert X\Vert\leqslant\alpha\Vert X\Vert$ donc $\Vert X\Vert=0$ et $X=0$ donc $A\in GL_{n}(\C)$, idem pour $B$. On a alors
		\begin{equation*}
			A\sum_{k=0}^{+\infty}(I_{n}-A)^{k}=((A-I_{n})+I_{n})\sum_{k=0}^{+\infty}(I_{n}-A)^{k}=I_{n}
		\end{equation*}
		par téléscopage. Donc 
		\begin{equation}A^{-1}=\sum_{k=0}^{+\infty}(I_{n}-A)^{k}\end{equation}
		et
		\begin{equation}\vertiii{A^{-1}}\leqslant\sum_{k=0}^{+\infty}\alpha^{k}=\frac{1}{1-\alpha}\end{equation}
		et de même pour $B$. On écrit alors
		\begin{equation}ABA^{-1}B^{-1}-I_{n}=(AB-BA)A^{-1}B^{-1}=((A-I_{n})(B-I_{n})-(B-I_{n})(A-I_{n}))A^{-1}B^{-1})\end{equation}
		d'où
		\begin{equation}\vertiii{ABA^{-1}B^{-1}-I_{n}}\leqslant\frac{2\vertiii{A-I_{n}}\vertiii{B-I_{n}}}{(1-\alpha)(1-\beta)}\end{equation}

		\item On prend $\alpha=\beta=\frac{1}{4}$.
		\item Pour tout $M\in G$, il existe $r>0$ tel que $B(M,r)\cap G=\{M\}$. Montrons que $G$ est discret si et seulement si $I_{n}$ est isolé. En effet, si $I_{n}$ est isolé, il existe $r_{0}>0$ tel que $B(I_{n},r_{0})\cap G=\{I_{n}\}$. Soit $M\in G$, alors pour tout $M'\in G$, $M-M'=M(I_{n}-M^{-1}M')$ d'où $I_{n}-M^{-1}M'=M^{-1}(M-M')$. Si 
		\begin{equation}\vertiii{M-M'}<\frac{r_{0}}{\vertiii{M^{-1}}}\end{equation}
		on a $\vertiii{I_{n}-M^{-1}M'}<r_{0}$ et donc $M'=M$ et $M$ est isolé. Ainsi $G$ est isolé. La réciproque est évidente.

		$C$ est dans le commutant si et seulement si $C$ commute avec $A$ et $B$ si et seulement si
		\begin{equation}
		\left\{
			\begin{array}[]{l}
				ACA^{-1}C^{-1}=I_{n}\\
				BCB^{-1}C^{-1}=I_{n}
			\end{array}
		\right.
		\end{equation}

		Notons maintenant que 
		\begin{equation}\overline{B_{\Vert\cdot\Vert}(I_{n},\frac{1}{4})}\cap G=\mathcal{A}\end{equation}
		est fini. En effet, si cet ensemble était infini, il existerait $(M_{p})_{p\in\N}$ une suite injective dans $\mathcal{A}$. La suite étant bornée, on peut extraite $(M_{\sigma(p)})_{p\in\N}$ qui converge et alors pour tout $p\in I_{n}$
		\begin{equation}\underbrace{M_{\sigma(p)}M_{\sigma(p+1)}^{-1}}_{\xrightarrow[pto+\infty]{}I_{n}}\in G\setminus\{I_{n}\}\end{equation}
		ce qui est impossible car $I_{n}$ est isolé.

		Comme $A\in \mathcal{A}\setminus\{I_{n}\}$, il existe $C\in\mathcal{A}\setminus\{I_{n}\}$ telle que $\vertiii{C-I_{n}}$ soit minimale et $\vertiii{c-I_{n}}\leqslant\frac{1}{4}$. D'après la question 2 on a 
		\begin{equation}\vertiii{ACA^{-1}C^{-1}-I_{n}}<\vertiii{C-I_{n}}\end{equation}
		et même chose pour $B$. Donc nécessairement, $ACA^{-1}C^{-1}=I_{n}$ et de même pour $B$. Ainsi, $C$ commute avec toutes les matrices de $G$.
	\end{enumerate}
\end{proof}

\begin{proof}
	\phantom{}
	\begin{enumerate}
		\item $\C_{n-1}[A]$ est un sous-espace vectoriel de dimension finie donc c'est un fermé. Par division euclidienne par $\chi_{A}$, d'après le théorème de Cayley-Hamilton, $\C[A]=\C_{n-1}[A]$. Comme 
		\begin{equation}\exp(A)=\lim\limits_{n\to+\infty}\sum_{k=0}^{n}\frac{A^{k}}{k!}\end{equation}
		$\exp(A)\in \C[A]=\C_{n-1}[A]$.

		\item Si $A$ est diagonalisable, il existe $P\in GL_{n}(\C)$ tel que 
		\begin{equation}A=P^{-1}\diag(\lambda_{1},\dots,\lambda_{n})P\end{equation}
		et donc 
		\begin{equation}\exp(A)=P^{-1}\diag(e^{\lambda_{1}},\dots,e^{\lambda_{n}})P\end{equation}
		et $\exp(A)$ est diagonalisable.

		Si $\exp(A)$ est diagonalisable, on utilise la décomposition de Dunford: $A=D+N$ avec $DN=ND$, $D$ diagonalisable et $N$ nilpotente. On a donc 
		\begin{equation}\exp(A)=\exp(D)\underbrace{\exp(N)}_{=\sum_{k=0}^{n-1}\frac{N^{k}}{k!}}=\exp(D)+\exp(D)\Bigl(\sum_{k=1}^{n-1}\frac{N^{k}}{k!}\Bigr)=\exp(D)+N'\end{equation}
		avec $N'$ nilpotente et $\exp(D)$ est diagonalisable d'après le sens direct. $N'$ commute avec $\exp(D)$. Par unicité de la décomposition de Dunford, $\exp(A)$ étant diagonalisable, on a $N'=0$. Comme $\exp(D)$ est inversible, 
		\begin{equation}N\times\underbrace{\sum_{k=1}^{n-1}\frac{N^{k-1}}{k!}}_{=I_{n}+N''}=0\end{equation}
		avec $N''$ nilpotente. $I_{n}+N''$ est donc inversible et ainsi $N=0$ et $A$ est diagonalisable.

		\item D'après ce qui précède, $\exp(A)=I_{n}$ est diagonalisable et 
		\begin{equation}\Sp_{\C}(\exp(A))=\{e^{\lambda}\bigm|\lambda\in\Sp_{\lambda}(\C)\}=\{I_{n}\}\end{equation}
		Donc $\Sp_{\C}(A)\subset 2i\pi\Z$.

		Réciproquement, si $A$ est diagonalisable avec $\Sp(A)\subset 2i\pi\Z$, en diagonalisant, on a bien $\exp(A)=I_{n}$.

		\item Sur $\R$, si $A$ est diagonalisable, $\exp(A)$ l'est aussi. Cependant, la réciproque n'est pas vrai, par exemple
		\begin{equation}M=\begin{pmatrix}
			2\mathrm{i}\pi & 0\\
			0 & -2\mathrm{i}\pi
		\end{pmatrix}\text{  semblable à }
		\begin{pmatrix}
			0 & -4\pi^{2}\\
			1 & 0
		\end{pmatrix}=A\end{equation}
		On a $\chi_{M}=X^{2}+4\pi^{2}$, $\exp(A)=I_{2}$ et $A$ n'est pas diagonalisable sur $\C$.
	\end{enumerate}
\end{proof}

\begin{proof}
	\phantom{}
	\begin{enumerate}
		\item On a $\ln(1-x)=P(x)+x^{2}O(1)$ et $\exp(y)=Q(y)+y^{n}O(1)$ d'où 
		\begin{equation}\exp(\ln(1+x))=1+x=Q(\ln(1+x))+\underbrace{\ln(1+x)^{n}O(1)}_{O(x^{n})}\end{equation}
		alors $1+x=Q(P(x)+O(x^{n}))+O(x^{n})=Q(P(x))+O(x^{n})$. Soit $B(X)=Q(P(X))+O(x^{n})\in\R[X]$, on a $\frac{B(x)}{x^{n}}=O(1)$ donc $X^{n}\mid B$ et \begin{equation}Q(P(X))=1+X+B(X)=1+X+X^{n}A(X)\end{equation}

		\item On a $N^{n}=0$ donc $P(N)$ est aussi nilpotente et on a 
		\begin{equation}\exp(P(N))=\sum_{k=0}^{n-1}\frac{P(N)^{k}}{k!}=Q(P(N))=I_{n}+N+0\end{equation}

		\item Soit $M\in GL_{n}(\C)$ et sa décomposition de Dunford: $M=D+N$ avec $D$ diagonalisable, $N$ nilpotente et $DN=ND$. On a $\Sp(D)=\Sp(M)\subset\C^{*}$ et on écrit
		\begin{equation}M=D\underbrace{(I_{n}+\underbrace{D^{-1}N}_{\text{nilpotente}})}_{=\exp(P(D^{-1}N))}\end{equation}
		si $D=P_{1}\diag(\lambda_{1},\dots,\lambda_{n})P_{1}^{-1}$, pour tout $k\in\{1,\dots,n\}$ il existe $\mu_{k}\in\C$ tel que $\lambda_{k}=\exp(\mu_{k})$ (car $\exp$ est surjectif sur $\C^{*}$). Alors 
		\begin{equation}
		D=\exp(P_{1}\diag(\mu_{1},\dots,\mu_{n})P_{1}^{-1})\in\C[D]
		\end{equation}
		puis 
		\begin{align}
			M
			&=\exp\Bigl(P_{1}\diag(\mu_{1},\dots,\mu_{n})P_{1}^{-1}\Bigr)\exp\Bigl(P(D^{-1}N)\Bigr)\\
			&=\exp\Bigl(P_{1}\diag(\mu_{1},\dots,\mu_{n})P_{1}^{-1}+P(D^{-1}N)\Bigr)
		\end{align}
		car les matrices commutent.

		Donc $\exp$ est surjective.
	\end{enumerate}
\end{proof}

\begin{proof}
	On a $A\subset\overline{A}$, $0=\lim\limits_{n\to+\infty}(\frac{2}{n})^{2n}\in\overline{A}$ et $e=\lim\limits_{n\to+\infty}(1+\frac{1}{n})^{n+1}\in\overline{A}$.

	Si $n\geqslant2$ et $p\geqslant2$, $(\frac{1}{n}+\frac{1}{p})^{n+p}\leqslant1$. Donc si $(\frac{1}{n}+\frac{1}{p})^{n+p}\geqslant1$, alors $n=1$ ou $p=1$.

	Si $x>e$, à partir d'un certain rang, on a $(1+\frac{1}{n})^{n+1}\leqslant\frac{e+x}{2}$ et si $x\notin A$, $x\notin\overline{A}$.
	Si $1\leqslant x<e$, à partir d'un certain rang, on a $(1+\frac{1}{n})^{n+1}>x$ donc si $x\notin A$, $x\notin\overline{A}$.

	Soit $x<1$, si $n\geqslant2$ et $p\geqslant3$ ou $n\geqslant3$ et $p\geqslant2$, on a $\frac{1}{n}+\frac{1}{p}\leqslant\frac{5}{6}$ et 
	\begin{align}
		\Biggl(\frac{1}{n}+\frac{1}{p}\Biggr)^{n+p}
		&=\exp\Biggl((n+p)\ln\Bigl(\frac{1}{n}+\frac{1}{p}\Bigr)\Biggr)\\
		&\leqslant\exp\Biggl((n+p)\ln\Bigl(\frac{5}{6}\Bigr)\Biggr)\\
		&\leqslant\max\Biggl(\underbrace{\Bigl(\frac{5}{6}\Bigr)^{n}}_{\xrightarrow[n\to+\infty]{}0},\underbrace{\Bigl(\frac{5}{6}\Bigr)^{p}}_{\xrightarrow[p\to+\infty]{}0}\Biggr)
	\end{align}
	Il existe $N_{0}$ tel que pour tout $n\geqslant N_{0}$, $(\frac{5}{6})^{n}\leqslant\frac{x}{2}$. Si $n$ ou $p$ est plus grand que $N_{0}$, on a donc 
	\begin{equation}\Biggl(\frac{1}{n}+\frac{1}{p}\Biggr)^{n+p}\leqslant\frac{x}{2}\end{equation}
	Donc il n'y a qu'un nombre fini d'éléments de $A$ plus grand que $\frac{x}{2}$. Ainsi,
	\begin{equation}\overline{A}=A\cup\{e,0\}\end{equation}
\end{proof}

\begin{proof}
	On note 
	\begin{equation}\mathbb{V}=\bigcup_{m\geqslant1}\U_{m}=\Biggl\{e^{\frac{2\mathrm{i}k\pi}{m}}\Biggm| m\geqslant1,k\in\{0,\dots,m-1\}\Biggr\}\end{equation}
	Soit $M\in H$. $X^{m}-1$ est scindé à racines simples sur $\C$ donc $M$ est diagonalisable sur $\C$ avec ses valeurs propres dans $\mathbb{V}$. Réciproquement, si $M$ est diagonalisable sur $\C$ et $\Sp_{\C}(M)\subset\mathbb{V}$. Alors pour tout $\lambda\in\Sp_{\C}(M),\exists m_{\lambda}\in\N^{*},\lambda\in\U_{m_{\lambda}}$ et soit $m=\ppcm\limits_{\lambda\in\Sp_{\C}(M)}(m_{\lambda})$. Alors $M^{m}=I_{n}$.

	Soit $A\in\overline{H}$, il existe $(M_{p})_{p\in\N}\in H^{\N}$ telle que $\lim\limits_{p\to+\infty}M_{p}=A$. Comme le polynôme caractéristique est une fonction continue des coefficients, pour tout $\lambda\in\Sp_{\C}(A)$, on a 
	\begin{equation}\lim\limits_{p\to+\infty}\chi_{M_{p}}(\lambda)=\chi_{A}(\lambda)=0\end{equation}
	Or 
	\begin{equation}\vert\chi_{M_{p}}(\lambda)\vert=\vert\lambda-\lambda_{1,p}\vert\dots\vert\lambda-\lambda_{n,p}\vert\geqslant d(\lambda,\U)^{n}\end{equation}
	avec $\lambda_{i,p}\in\mathbb{V}$ pour tout $i\in\{1,\dots,n\}$. Donc $d(\lambda,\U)=0$ et comme $\U$ est fermé, $\lambda\in\U$.

	Réciproquement, soit $A\in\M_{n}(\C)$ tel que $\Sp_{\C}(A)\subset\U$. Soit 
	\begin{equation}\bigl\{e^{\mathrm{i}\theta_1},\dots,e^{\mathrm{i}\theta_r}\bigr\}\end{equation}
	les valeurs propres distinctes de $A$ de multiplicités $m_{1},\dots,m_{r}$. Il existe $Q\in GL_{n}(\C)$ tel que 
	\begin{equation}A=Q\diag(\underbrace{e^{\mathrm{i}\theta_{1}},\dots,e^{\mathrm{i}\theta_1}}_{m_{1}\text{ fois}},\dots,\underbrace{e^{\mathrm{i}\theta_r},\dots,e^{\mathrm{i}\theta_r}}_{m_{r}\text{ fois}})Q^{-1}\end{equation}
	On a 
	\begin{equation}\theta=\lim\limits_{k\to+\infty}\frac{2\pi}{k}\lfloor k\frac{\theta}{2\pi}\rfloor\end{equation}
	donc on peut former, pour $p\in\N^{*}$,
	\begin{equation}A=Q\diag(\underbrace{e^{\mathrm{i}\theta_{1,p}},\dots,e^{\mathrm{i}\theta_{1,p}}}_{m_{1}\text{ fois}},\dots,\underbrace{e^{\mathrm{i}\theta_{r,p}},\dots,e^{\mathrm{i}\theta_{r,p}}}_{m_{r}\text{ fois}})Q^{-1}\end{equation}
	avec $\theta_{i,p}=\frac{2\pi}{p}\lfloor p\frac{\theta_{j}}{2\pi}\rfloor+\frac{2 j\pi}{p}$. Pour $p$ suffisamment gand, les $(\theta_{j,p})$ sont deux à deux distincts donc $A_{p}$ est diagonalisable et $A_{p}\in H$, et donc $A\in \overline{H}$.
\end{proof}

\begin{proof}
	\phantom{}
	\begin{enumerate}
		\item On a l'inégalité triangulaire et l'homogénéité. On a cependant $N_{a}(X^{k})=\vert a_{k}\vert$ et pour tout $k\in\N$, $X^{k}\neq0$. Donc $N_{a}$ est une norme implique que $a$ ne s'annule pas sur $\N$. Réciproquement, si pour tout $k\in\N$, $a_{k}\neq0$, si $P\neq0$, il existe $k\in\N$ avec $p_{k}$ et donc $N_{a}(P)>0$. Donc $N_{a}$ est une norme si et seulement si pour tout $k\in\N$, $a_{k}\neq0$.
		
		\item Si $N_{a}$ et $N_{b}$ sont équivalentes, alors il existe $(\alpha,\beta)\in(\R_{+}^{*})^{2}$ tel que pour tout $k\in\N$,
		\begin{equation}\beta N_{b}(X^{k})\leqslant N_{a}(X^{k})\leqslant\alpha N_{b}(X^{k})\end{equation}
		d'où
		\begin{equation}\beta \vert b_{k}\vert\leqslant N_{a}(X^{k})\leqslant\alpha \vert b_{k}\vert\end{equation}
		Donc $a=O(b)$ et $b=O(a)$.

		Réciproquement, si $a=O(b)$ et $b=O(a)$, alors on a l'inégalité précédente sur les $a_{k}$ et $b_{k}$, d'où
		\begin{equation}\beta\sum_{k=0}^{+\infty}\vert p_{k}b_{k}\vert\leqslant\sum_{k=0}^{+\infty}\vert p_{k}a_{k}\vert\leqslant\alpha\sum_{k=0}^{+\infty}\vert p_{k} b_{k}\vert\end{equation}
		et donc pour tout $P\in\C[X]$
		\begin{equation}\beta N_{b}(P)\leqslant N_{a}(P)\leqslant\alpha N_{b}(P)\end{equation}
		et $N_{a}$ et $N_{b}$ sont équivalentes.

		\item $\Delta$ est continue pour $N_{a}$ si et seulement s'il existe $c\geqslant0$ tel que pour tout $P\in\C[X]$, $N_{a}(\Delta P)\leqslant CN_{a}(P)$. Si $\Delta$ est continue alors il existe $c\geqslant0$ tel que $N_{a}(kX^{k})\leqslant cN_{a}(X^{k})$ alors pour tout $k\in\N^{*}$,
		\begin{equation}
			\label{eq:6.1}
			\vert ka_{k-1}\vert\leqslant c\vert a_{k}\vert
		\end{equation}
		Réciproquement, si on a \eqref{eq:6.1}, pour tout $P\in\C[X]=N_{a}(\Delta P)\leqslant cN_{a}(P)$. Pour tout $k\in\N,a_{k}=k!$, \eqref{eq:6.1} est vérifiée pour $c=1$. Si $b_{k}=1$ pour tout $k\in\N$, \eqref{eq:6.1} n'est pas vérifiée donc $\Delta$ n'est pas continue pour $N_{b}$.
	\end{enumerate}
\end{proof}

\begin{proof}
	\phantom{}
	\begin{enumerate}
		\item On a $d(x,A)=0$ si et seulement si $\inf\limits_{a\in A}\Vert x-a\Vert=0$ si et seulement si $\varepsilon>0,\exists a\in A\colon\Vert x-a\Vert<\varepsilon$ si et seulement si $x\in\overline{A}$.

		On a $A\subset\overline{A}$ donc $d(x,\overline{A})\leqslant d(x,A)$. Soit $\varepsilon>0$, il existe $a'\in \overline{A}$ tel que $\Vert x-a'\Vert<d(x,\overline{A})+\varepsilon$ et il existe $a\in A$ tel que $\Vert a-a'\Vert<\varepsilon$. Ainsi, 
		\begin{equation}d(x,A)\leqslant\Vert x-a\Vert\leqslant d(x,\overline{A})+2\varepsilon\end{equation}
		Ceci calant pour tout $\varepsilon>0$, on a $d(x,A)\leqslant d(x,\overline{A})$ et donc on a égalité.

		\item $A\times B\subset\overline{A}\times\overline{B}$ donc $d(A,B)\geqslant d(\overline{A},\overline{B})$. De plus, pour tout $\varepsilon>0$, il existe $(a',b')\in\overline{A}\times\overline{B}$ tel que $\Vert a'-b'\Vert<d(\overline{A},\overline{B})+\varepsilon$ et il existe $(a,b)\in A\times B$ tel que $\Vert a-a'\Vert<\varepsilon$ et $\Vert b-b'\Vert\varepsilon$. En utilisant l'inégalité triangulaire, on a donc 
		\begin{equation}d(A,B)\leqslant\Vert a-b\Vert<d(\overline{A},\overline{B})+3\varepsilon\end{equation}
		Ceci valant pour tout $\varepsilon>0$, on a bien l'égalité.
	\end{enumerate}
\end{proof}

\begin{proof}
	$\varphi_{x_{0}}$ est une forme linéaire. Elle est continue si et seulement $C>0$ tel que pour tout $P\in\C[X]$,
	\begin{equation}\vert P(x_{0})\vert\leqslant C\Vert P\Vert_{\infty}\end{equation}
	Si $P=\sum_{k=0}^{n}a_{k}X^{k}$, on a 
	\begin{equation}\vert P(x_{0})\vert\leqslant \Vert P\Vert_{\infty}\sum_{k=0}^{n}\vert x_{0}\vert^{k}\end{equation}
	Si $\vert x_{0}\vert<1$, on a 
	\begin{equation}\vert P(x_{0})\vert\leqslant \Vert P\Vert_{\infty}\frac{1}{1-\vert x_{0}\vert}\end{equation}
	donc $\varphi_{x_{0}}$ est continue et si $x_{0}=\vert x_{0}\vert e^{\mathrm{i}\theta_{0}}$, soit $n\in\N$ et $P_{n}=\sum_{k=0}^{n}e^{-\mathrm{i}k\theta_{0}}X^{k}$, on a $\Vert P_{n}\Vert_{\infty}=1$ et 
	\begin{equation}\vert \varphi_{x_{0}}(P_{n})=\sum_{k=0}^{n}\vert x_{0}\vert^{k}\xrightarrow[n\to+\infty]{}\frac{1}{1-\vert x_{0}\vert}\end{equation}
	donc $\vertiii{\varphi_{x_{0}}}=\frac{1}{1-\vert x_{0}\vert}$.

	Si $\vert x_{0}\vert\geqslant1$, 
	\begin{equation}\vert\varphi_{x_0}(P_{n})\vert=\sum_{k=0}^{n}\vert x_{0}\vert^{k}\xrightarrow[n\to+\infty]{}+\infty\end{equation}
	donc $\varphi_{x_{0}}$ n'est pas continue.
\end{proof}

\begin{proof}
	Pour le sens indirect, soit $\lambda\in\Sp_{\C}(M)$. Pour tout $p\in\N$, $\lambda\in\Sp_{\C}(M_{p})$ donc $\det(M_{p}-\lambda I_{n})=0$. Par continuité du déterminant, on a $0=\det(M_{p}-\lambda I_{n})\xrightarrow[p\to+\infty]{}\det(-\lambda I_{n})$. Donc $\lambda=0$ et $\Sp_{\C}(M)=\{0\}$ donc $M$ est nilpotente.

	Pour le sens direct, soit $u\in\L(\C^{n})$ canoniquement associée à $M$. On trigonalise $u$ sur une base $\mathcal{B}=(\varepsilon_{1},\dots,\varepsilon_{n})$ avec $u(\varepsilon_{1})=0,u(\varepsilon_{2})=a_{1,2}\varepsilon_{1},\dots,u(\varepsilon_{n})=a_{1,n}\varepsilon_{1}+\dots+a_{n-1,n}\varepsilon_{n-1}$. Posons pour $i\in\{1,\dots,n\}$, $\varepsilon_{i,p}=\frac{\varepsilon_{i}}{p^{i-1}}$. On pose $\mathcal{B}_{p}=(\varepsilon_{1,p},\dots,\varepsilon_{n,p})$ et $M_{p}=\mat\limits_{B_{p}}(u)$, semblable à $M$ et $M_{p}\xrightarrow[p\to+\infty]{}0$ car $\Vert M_{p}\Vert\leqslant\frac{1}{p}\Vert M_{1}\Vert$.
\end{proof}

\begin{proof}
	On pose $u\in\L(\C^{n})$ canoniquement associée à $M$. 

	Pour le sens indirect, si $M$ n'est pas diagonalisable, il existe une base $B=(\varepsilon_{1},\dots,\varepsilon_{n})$ de $\C^{n}$ telle que 
	\begin{equation}\mat\limits_{\mathcal{B}}(u)=D+N\end{equation}
	où $D$ est diagonale et $N$ est nilpotente (décomposition de Dunford). En reprenant les bases $\mathcal{B}_{p}$ définies à l'exercice précédent, on a
	\begin{equation}\mat\limits_{\mathcal{B}_{p}}(u)=D+N_{p}\xrightarrow[p\to+\infty]{}D\end{equation}
	Si $D\in S_{M}$, alors $M$ est diagonalisable ce qui est exclu par hypothèse. Donc $S_{M}$ n'est pas fermé.

	Pour le sens direct, si $M$ est diagonalisable, soit $(M_{p})_{p\in\N}\in(S_{M})^{\N}$ avec $M_{p}\xrightarrow[p\to+\infty]{}M'$. Soit $\lambda\in\C$. On a $\chi_{M_{p}}(\lambda)=\det(\lambda I_{n}-M_{p})=\chi_{M}(\lambda)$ car $M$ et $M_{p}$ sont semblables. Par continuité du déterminant, on a $\chi_{M'}(\lambda)=\chi_{M}(\lambda)$, donc $\chi_{M'}=\chi_{M}$. De plus, $A\mapsto\Pi_{M}(A)$ (polynôme minimal) est continue sur $\M_{n}(\C)$ et pour tout $p\in\N$, on a $\Pi_{M}(M_{p})=0$ donc $\Pi_{M}(M')=0$. $M'$ est donc annulée par $\Pi_{M}$, donc $M'$ est diagonalisable et comme $\chi_{M}=\chi_{M'}$, $M$ et $M'$ ont les mêmes valeurs propres avec les mêmes multiplicités. Donc $M'\in S_{M}$.
\end{proof}

\begin{remark}
	Le polynôme caractéristique est une fonction continue de la matrice, mais c'est faux pour le polynôme minimal, par exemple pour 
	\begin{equation}M_{p}=\begin{pmatrix}
		\frac{1}{p} &0\\
		0 & \frac{2}{p}
	\end{pmatrix}\end{equation}
	On a $M_{p}\xrightarrow[p\to+\infty]{}0$ et $\Pi_{M_{p}}=(X-\frac{1}{p})(X-\frac{2}{p})\xrightarrow[p\to+\infty]{} X^{2}\neq X=\Pi_{M_{\infty}}$ donc $\lim\limits_{p\to+\infty}\Pi_{M_p}\neq\Pi_{\lim\limits_{p\to+\infty}M_{p}}$.
\end{remark}

\begin{proof}
	On note $A_{h}=\{\vert\varphi(x)-\varphi(y)\vert\bigm|(x,y)\in I^{2}\text{ et }\vert x-y\vert\leqslant h\}$.
	\begin{enumerate}
		\item $\omega_{\varphi}$ est bien défini car $\vert\varphi(x)-\varphi(y)\vert\leqslant 2\Vert\varphi\Vert_{\infty}$). Si $0<h\leqslant h'$, alors $A_{h}\subset A_{h'}$ donc $\sup(A_{h})\leqslant\sup(A_{h'})$ donc $\omega_{\varphi}(h)\leqslant\omega_{\varphi}(h')$.
		\item Soit $(h,h')\in(\R_{+}^{*})^{2}$, soit $(x,y)\in I^{2}$ tel que $\vert x-y\vert\leqslant h+h'$ (où on peut supposer que $x\leqslant y$).
		\begin{itemize}
			\item Si $y\in[x,x+h]$, alors $\vert x-y\vert\leqslant h$ donc $\vert\varphi(x)-\varphi(y)\vert\leqslant\omega_{\varphi}(h)\leqslant\omega_\varphi(h)+\omega_{\varphi}(h')$
			\item Si $y\in[x+h,x+h+h']$, $\vert\varphi(x)-\varphi(y)\vert\leqslant\vert\varphi(x)-\varphi(x+h)\vert+\vert\varphi(x+h)-\varphi(y)\vert\leqslant\omega_\varphi(h)+\omega_{\varphi}(h')$ car $\vert x-(x+h)\vert\leqslant h$ et $\vert x+h-y\vert\leqslant h'$.
		\end{itemize}
		Donc $\omega_{\varphi}(h+h')\leqslant\omega_\varphi(h)+\omega_\varphi(h')$.
		\item Par récurrence sur $n\in\N$, on a $\omega_\varphi(nh)=n\omega_\varphi(h)$. Si $\lambda\in\R_{+}^{*}$, on a $\lambda h\leqslant(\lfloor \lambda\rfloor+1)h$ et par croissance et ce qui précède, on a 
		\begin{equation}\omega_\varphi(\lambda h)\leqslant(\lfloor\lambda\rfloor+1)\omega_\varphi(h)\leqslant(\lambda+1)\omega_\varphi(h)\end{equation}
		\item Soit $\varepsilon>0$. $\varphi$ étant uniformément continue, il existe $\alpha>0$ tel que pour tout $(x,y)\in I^{2}$, si $\vert x-y\vert\alpha$ on a $\vert\varphi(x)-\varphi(y)\vert\leqslant\varepsilon$ et on a pour $h\leqslant\alpha$, $\omega_\varphi(h)\leqslant\varepsilon$ d'où $\lim\limits_{h\to0}\omega_\varphi(h)=0$.
		
		Soit alors $h_{0}>0$ fixé et $h>0$,
		\begin{itemize}
			\item si $h_{0}\leqslant h$, on a $0\leqslant\omega_\varphi(h)-\omega_\varphi(h_0)\leqslant\omega_\varphi(h-h_0)$.
			\item si $h\leqslant h_{0}$, on a $0\leqslant\omega_\varphi(h_0)-\omega_\varphi(h)\leqslant\omega_\varphi(h_0-h)$.
		\end{itemize}
		Dans tous les cas, on a $\vert\omega_\varphi(h)-\omega_\varphi(h_{0})\vert\leqslant\omega_\varphi(\vert h_{0}-h\vert)$. Donc on a bien $\lim\limits_{h\to h_{0}}\omega_\varphi(h)=\omega_\varphi(h_{0})$. Donc $\omega_{\varphi}$ est continue (et même uniformément).
	\end{enumerate}
\end{proof}

\begin{proof}
	$G$ est borné car si $M\in G$, $\vertiii{M}\leqslant \vertiii{I_{n}}+\mu=1+\mu$. Montrons donc que si $G_{0}$ est un sous-groupe borné de $GL_{n}(\C)$, alors les valeurs propres de ses éléments sont de module 1, et ceux-ci sont diagonalisables.

	En effet, soit $M\in G$ et $\lambda\in\Sp(M)$, soit $X$ un vecteur propre associé. On a 
	$\Vert MX\Vert=\vert\lambda\vert\Vert X\Vert\leqslant\vertiii{M}\Vert X\Vert$ donc $\vert\lambda\vert\leqslant\vertiii{M}\leqslant\sup\limits_{M\in G}\vertiii{M}$. Pour tout $k\in\Z$, $M^{k}\in G$ et $\lambda^{k}\in\Sp(M^{k})$, donc si $\vert\lambda\vert>1$, on a $\lim\limits_{k\to+\infty}\vert\lambda\vert^{k}=+\infty$, et si $\vert\lambda\vert^{\lambda}<1$, on a $\lim\limits_{k\to-\infty}\vert\lambda\vert^{k}=+\infty$. Comme 
	G est borné, $\vert\lambda\vert=1$.

	On utilise ensuite la décomposition de Dunford pour $M$: $M=D+N$ avec $DN=ND$, $D$ diagonalisable et $N$ nilpotente. Grâce au binôme de Newton, pour $k\geqslant r$ p* $r$ est l'indice de nilpotence de $N$, on a
	\begin{equation}M^{k}=\sum_{p=0}^{k}\binom{k}{p}N^{p}D^{k-p}=\underbrace{D^{k}}_{\text{borné}}+kND+\sum_{p=2}^{r-1}\underbrace{\binom{k}{p}}_{\underset{k\to+\infty}{\sim}\frac{k^{p}}{p!}}N^{p}\underbrace{D^{k-p}}_{\text{borné car }\Sp(D)\subset\U}\end{equation}
	Donc
	\begin{equation}M^{k}\underset{k\to+\infty}{\sim}\underbrace{\frac{k^{r-1}}{(r-1)!}\underbrace{N^{r-1}}_{\neq0}D^{k-r+1}}_{\text{non borné si }N\neq0}\end{equation}
	Donc $N=0$ et $M=D$ est diagonalisable.

	Revenons donc à l'exercice. Soit $M\in G$ et $\lambda=e^{\mathrm{i}\theta}\in\Sp(M)$ avec $\theta\in]-\pi,pi]$. Si $X$ est un vecteur propre associé à $\lambda$, on a 
	\begin{equation}(\lambda-1)\Vert X\Vert=\Vert(M-I_{n})X\Vert\leqslant\mu\Vert X\Vert\end{equation}
	donc $\vert\lambda-1\vert=2\vert\underbrace{\sin(\frac{\theta}{2})}_{\geqslant0}\vert\leqslant\mu$.
	Donc $\theta\in[-\theta_{0},\theta_{0}]$ où $\theta_{0}=\arcsin(\frac{\mu}{2})\in[0,\pi[$.

	Si $\frac{\theta}{\pi}\notin\Q$, $e^{\mathrm{i}k\pi}\in\Sp(M^{k})$, $\vert e^{\mathrm{i}k\theta}-1\vert\leqslant\mu$. Alors $\{k\theta+2l\pi\bigm| (k,l)\in\Z^{2}\}$ est un sous-groupe de $(\R,+)$ non monogène et donc dense, et alors $(e^{\mathrm{i}k\theta})_{k\in\Z}$ est dense dans $\U$, donc il existe $k_{0}\in\Z$ tel que $\vert e^{\mathrm{i}k_{0}\theta}+1\vert=\vert 2-(1-e^{\mathrm{i}k_{0}\theta_{0}})\vert<2-\mu$, ce qui est impossible car $\vert 2-(1-e^{\mathrm{i}k_{0\theta}})\vert\geqslant2-\vert 1-e^{\mathrm{i}k_{0}\theta_{0}}\vert\geqslant2-\mu$.

	Ainsi, $\frac{\theta}{\pi}\in\Q$ et il existe $m\in\N^{*}$ tel que $\lambda=e^{\mathrm{i}\theta}\in\U_{m}$. Ce n'est pas forcément le même $m$ pour tout les M dans G. Notons alors pour 
	\begin{equation}\lambda\in\bigcup_{M\in G}\Sp(M)=\mathcal{A}\end{equation}
	$\omega(\lambda)$ l'ordre (multiplicatif) de $\lambda$ dans $\U$.

	Si $\omega(\lambda)=m$, on a $gr(\lambda)=\U_{m}$ donc il existe $k\in\Z$ tel que $\lambda^{k}=e^{\frac{2\mathrm{i}\pi}{m}}\in\mathcal{A}$ (car $\lambda^{k}\in\Sp(M^{k})$). Supposons que $\{\omega(\lambda)\bigm| \lambda\in\mathcal{A}\}$ non borné. Alors il existe $(m_{k})_{k\in\N}$ tel que $m_{k}\xrightarrow[k\to+\infty]{}+\infty$ et $e^{\frac{2\mathrm{i}\pi}{m_{k}}}\in\mathcal{A}$. Alors 
	\begin{equation}\underbrace{e^{2\mathrm{i}\lfloor\frac{m_{k}}{2}\rfloor \frac{\pi}{m_{k}}}}_{\xrightarrow[k\to+\infty]{} e^{i\pi}=-1}\in\mathcal{A}\end{equation}
	ce qui est impossible car $\vert\lambda+1\vert\geqslant2-\mu>0$. On peut donc noter
	\begin{equation}m=\underset{\lambda\in\mathcal{A}}{\vee}\omega(\lambda)\end{equation}
	et pour tout $M\in G$, pour tout $\lambda\in\Sp(M)$, $\lambda^{m}=1$. Or $M$ est diagonalisable, donc $M^{m}=I_{n}$.
\end{proof}

\begin{proof}
	Si $M\in\mathcal{G}_{q}$, $P(X)=X^{q}-1$ annule $M$ donc $M$ est diagonalisable à valeurs propres dans $\U_{q}$. Réciproquement, si $M$ est diagonalisable et $\Sp_{\C}(M)\subset\U_{q}$ alors il existe $P\in GL_{n}(\C)$ avec 
	\begin{equation}M=P\diag(\lambda_{1},\dots,\lambda_{n})P^{-1}\end{equation}
	et donc 
	\begin{equation}M^{q}=P\diag(\lambda_{1}^{q},\dots,\lambda_{n}^{q})P^{-1}=I_{n}\end{equation}

	Si $M\in\mathcal{G}_{q}$ n'est pas une homothétie, il existe $\lambda\neq\mu\in\Sp_{\C}(M)^{2}$ et $P\in GL_{n}(\C)$ tel que 
	\begin{equation}M=P\begin{pmatrix}
		\lambda & &\\
		& \mu & &\\
		& & \ddots
	\end{pmatrix}P^{-1}\end{equation}
	Soit $k\in\N^{*}$ tel que 
	\begin{equation}M=P\begin{pmatrix}
		\lambda & \frac{1}{k}&\\
		& \mu & &\\
		& & \ddots
	\end{pmatrix}P^{-1}\xrightarrow[k\to+\infty]{}M\end{equation}
	Or 
	\begin{equation}\begin{pmatrix}
		\lambda & \frac{1}{k}\\
		0 & \lambda
	\end{pmatrix}\text{  est semblable }\begin{pmatrix}
		\lambda & 0\\
		0 & \mu
	\end{pmatrix}\end{equation}
	car $\chi_{A}=(X-\lambda)(X-\mu)$ donc est diagonalisable. Donc $M_{k}\sim M$ et $M_{k}\in\mathcal{G}_{q}$ et $M$ n'est pas isolé.

	Montrons le petit lemme suivante: soit $\Vert\cdot\Vert$ une norme sur $\C^{n}$ et $\vertiii{\cdot}$ la norme subordonnée, soit $\lambda\in\C$ et $M\in\M_{n}(\C)$ et $\varepsilon>0$. Si $\vertiii{M-\lambda I_{n}}\leqslant\varepsilon$ alors $\Sp_{\C}(M)\subset\overline{B(\lambda,\varepsilon)}$. En effet, soit $X$ un vecteur propre de $M$ associé à $\mu\in\Sp_{\C}(M)$. On a 
	\begin{equation}\Vert (M-\lambda I_{n})X\Vert=\vert\mu-\lambda \vert \Vert X\Vert\leqslant\vertiii{M-\lambda I_{n}}\Vert X\Vert\leqslant\varepsilon\Vert X\Vert\end{equation}
	donc $\vert\mu-\lambda\vert\leqslant\varepsilon$.

	Pour $\varepsilon=\sin(\frac{\pi}{q})>0$ et $\lambda\in\U_{q}$; si $M\in B_{\vertiii{\cdot}}(\lambda I_{n},\varepsilon)\cap \mathcal{G}_{q}$ alors pour tout $\mu\in\Sp_{\C}(M)$, on a $\vert\lambda-\mu\vert\leqslant\sin(\frac{\pi}{q})$ donc $\lambda=\mu$. Donc si $M=\lambda I_{n}$
	 alors $M$ est isolé (avec $\lambda\in\U_{q}$). Donc les matrices scalaires sont isolées.
\end{proof}

\end{document}
\documentclass[12pt]{article}
\usepackage{style/style_sol}

\begin{document}

\begin{titlepage}
	\centering
	\vspace*{\fill}
	\Huge \textit{\textbf{Solutions MP/MP$^*$\\ Fonction d'une variable réelle}}
	\vspace*{\fill}
\end{titlepage}

\begin{proof}
	Tout d'abord, $\deg(L_{n})=n$ et  son coefficient dominant et $\frac{(2n)!}{2^{n}(n!)^{2}}$.
	\begin{enumerate}
		\item Soit $f\in\mathcal{C}^{0}([0,1],\R)$. $-1$ et $1$ sont racines d'ordre $n$ de $P_{n}$ donc pour tout $k\in\{0,\dots,n-1\}$ $P_{n}^{(k)}(-1)=P_{n}^{(k)}(-1)=0$. Ainsi, on a par intégrations par parties successives:
		\begin{equation}(f|L_{n})=(-1)^{n}\int_{-1}^{1}f^{(n)}(t)P_{n}(t)dt\end{equation}
		Notamment, si $P\in\R_{n-1}[X]$, $P^{(n)}=0$ et $(P|L_{n})=0$. En particulier, pour tout $m<n$, $\deg(L_{m})\leqslant n-1$ et $(L_{m}|L_{n})=0$ donc $(L_{n})_{n\in\N}$ est orthogonale. Notons dès maintenant que l'on peut calculer la norme de $L_{n}$ grâce aux intégrales de Wallis:
		
		\begin{align}
			\Vert L_{n}\Vert_{2}^{2}
			&=(L_{n}|L_{n})\\
			&=(-1)^{n}\int_{-1}^{1}L_{n}^{(n)}(t^{2}-1)^{n}dt\\
			&=\frac{(2n)!}{2^{2n}(n!)^{2}}\int_{-1}^{1}(1-t^{2})^{n}dt
		\end{align}
		On pose $t=\cos(\theta)$ d'où $dt=-\sin(\theta)d\theta$, d'où
		\begin{align}
			\int_{-1}^{1}(1-t^{2})^{n}dt
			&=\int_{0}^{\pi}\sin(\theta)^{2n+1}d\theta\\
			&=2I_{2n+1}\text{ [Wallis] }
		\end{align}
		On a classiquement $I_{n+2}=\frac{n+1}{n+2}I_{n}$.
		D'où
		\begin{align}
			I_{2n+1}
			&=\frac{2n}{2n+1}\times\frac{2n-2}{2n-1}\times\dots\times\frac{2}{3}\times \underbrace{I_{1}}{=1}\\
			&=\frac{2^{2n}(n!)^{2}}{(2n+1)!}
		\end{align}
		d'où
		\begin{equation}\Vert L_{n}\Vert_{2}^{2}=\frac{(2n)!}{2^{2n}(n!)^{2}}\times 2\times \frac{2^{2n}(n!)^{2}}{(2n+1)!}=\frac{2}{2n+1}\end{equation}

		\item On utilise la formule de Leibniz en écrivant $X^{2}-1=(X+1)(X-1)$.
		\item On montre le résultat par récurrence sur $k\in\{0,\dots,n\}$ en invoquant le théorème de Rolle. On trouve donc que $L_{n}=P_{n}^{(n)}$ s'annule au moins $n$ fois sur $]-1,1[$. Or $\deg(L_{n})=n$, donc ces zéros sont simples et ce sont les seuls.
		\item $(L_{0},\dots,L_{n})$ est une base de $\R_{n}[X]$ (étagée en degré). Donc il existe $(\alpha_{n,0},\dots,\alpha_{n,k})\in\R^{k+1}$ tel que $XL_{n-1}=\sum_{k=0}^{n}\alpha_{n,k}L_{k}$. Si $k\leqslant n-3$, on a
		\begin{equation}(XL_{n-1} L_{k})=\alpha_{n,k}\Vert L_{k}\Vert_{2}^{2}=(L_{n-1}XL_{k})=0\end{equation}
		car $\deg(XL_{k})=k+1\leqslant n-2$. Donc 
		\begin{equation}XL_{n-1}=\alpha_{n,n-2}L_{n-2}+\alpha_{n,n-1}L_{n-1}+\alpha_{n,n}L_{n}\end{equation}
		Pour calculer les coefficients, on fait tout simplement les produits scalaires:
		\begin{equation}(Xl_{n-1}|L_{n-1})=\int_{-1}^{1}tL_{n-1}(t)^{2}dt\end{equation}
		Or $P_{n}$ est paire, donc $L_{n}$ est de la parité de $n$ et donc $L_{n}^{2}$ est paire puis $XL_{n}^{2}$ est impaire. Donc $\alpha_{n,n-1}=0$.

		\begin{align}
			(XL_{n-1}|L_{n-2})
			&=\alpha_{n,n-2}\underbrace{\Vert L_{n-2}\Vert_{2}^{2}}_{=\frac{2}{2n-3}}\\
			&=(-1)^{n}\int_{-1}^{1}P_{n-1}(t)\underbrace{(XL_{n-2})^{(n-1)}(t)}_{\frac{(2n-4)!(n-1)}{2^{n-2}(n-2)!}}
		\end{align}
		Par ailleurs,
		\begin{align}
			(-1)^{n-1}\int_{-1}^{1}P_{n-1}(t)dt
			&=\frac{1}{2^{n-1}(n-1)!}\underbrace{\int_{-1}^{1}(1-t^{2})^{n-1}dt}_{2I_{2n-1}}\\
			&=\frac{1}{2^{n-1}(n-1)!}\times 2\times\frac{2^{2n-2}(n-1)!)^{2}}{(2n-1)!}\\
			&=\frac{2^{n}(n-1)!}{(2n-1)!}
		\end{align}
		donc $\frac{\alpha_{n,n-2}}{\alpha_{n,n}}=\frac{n-1}{n}$. D'où le résultat.
	\end{enumerate}
\end{proof}

\begin{proof}
	On forme \function{g}{[a,b]}{\R}{x}{\underbrace{\Delta f(x_{0},\dots,x_{n-1},x)}_{\varphi(x)}-\underbrace{\prod_{i=0}^{n-1}(x-x_{i})A}_{P(x)}}
	On a $g(x_{n})=0$. On suppose les $(x_{i})_{1\leqslant i\leqslant n}$ distincts, et on pose 
	\begin{equation}A=\frac{V(x_{0},\dots,x_{n})}{\prod_{i=0}^{n-1}(x_{n}-x_{i})}\end{equation}
	$g$ est de classe $\mathcal{C}^{n}$ et pour tout $i\in\{0,\dots,n\}$, on a $g(x_{i})=0$.
	Donc il existe $\xi\in]a,b[$ tel que $g^{(n)}(\xi)=0$ (théorème de Rolle appliqué $n$ fois. $\deg(P)=n$ et son coefficient dominant est $A$ donc $P^{(n)}(\xi)=An!=\varphi^{(n)}(\xi)$.

	On développe maintenant $\varphi(x)$ par rapport à la dernière colonne:
	\begin{equation}\varphi(x)=f(x)\times V_{n}(x_{0},\dots,x_{n-1})+Q(X)\end{equation}
	avec $\deg(Q)\leqslant n-1$ et $V_{n}(x_{0},\dots,x_{n-1})=\prod_{0\leqslant j<i\leqslant n-1}(x_{i}-x_{j})$ (déterminant de Vandermonde). On a donc 
	\begin{equation}\varphi^{(n)}(x)=f^{(n)}(x)\prod_{0\leqslant j<i\leqslant n-1}(x_{j}-x_{i})\end{equation}
	et en reportant, on a 
	\begin{equation}\frac{f^{(n)}(\xi)}{n!}=\frac{A}{\prod_{0\leqslant i<j\leqslant n-1}(x_{j}-x_{i})}=\Delta f(x_{0},\dots,x_{n})\end{equation}
\end{proof}

\begin{proof}
	On utilise le développement de Taylor avec reste intégral.
	\begin{equation}f(0)=f\Bigl(\frac{1}{2}\Bigr)-\frac{1}{2}f'\Bigl(\frac{1}{2}\Bigr)+\int_{\frac{1}{2}}^{0}-tf''(t)dt\end{equation}
	et de même
	\begin{equation}f(1)=f\Bigl(\frac{1}{2}\Bigr)-\frac{1}{2}f'\Bigl(\frac{1}{2}\Bigr)+\int_{\frac{1}{2}}^{1}(1-t)f''(t)dt\end{equation}
	D'où
	\begin{align}
		A(f)
		&=f(0)-f\Bigl(\frac{1}{2}\Bigr)+f(1)-f\Bigl(\frac{1}{2}\Bigr)\\
		&=\int_{0}^{\frac{1}{2}}tf''(t)dt+\int_{\frac{1}{2}}^{1}(1-t)f''(t)dt\\
		&\leqslant\int_{0}^{\frac{1}{2}}tdt+\int_{\frac{1}{2}}^{1}(1-t)dt\\
		&=\frac{1}{4}
	\end{align}
	Et c'est atteint pour $f(t)=\frac{t^{2}}{4}$.
\end{proof}

\begin{proof}
	Pour tout $(x,h)\in\R^{2}$, $f(x+h)-f(x-h)=2hf'(x)$ donc 
	\begin{equation}
		\label{eq:7.1}
		f'(x)=\frac{1}{2}(f(x+1)-f(x-1))
	\end{equation}
	donc $f'$ est $\mathcal{C}^{1}$ et donc $f$ est $\mathcal{C}^{2}$. On fixe alors $x$ et on dérive deux fois~\eqref{eq:7.1} en fonction de $h$. On a alors
	\begin{equation}f''(x+h)=f''(x-h)\end{equation}
	pour tout $(x,h)\in\R^{2}$ donc $f''$ est constante et $f$ est polynômiale de degré 2.

	Réciproquement, si $f(x)=ax^{2}+bx+c$, on a bien la relation de l'énoncé.
\end{proof}

\begin{proof}
	\phantom{}
	\begin{enumerate}
		\item Soit $a>0$, \function{\tau_{a}}{\R}{]a,+\infty[}{x}{\frac{f(x)-f(a)}{x-a}}
		est croissante. Donc il existe $l=\lim\limits_{x\to+\infty}\tau_{a}(x)\in\overline{\R}$. On écrit alors 
		\begin{equation}\frac{f(x)}{x}=\frac{f(x)-f(a)}{x-a}\times \frac{x-a}{x}+\frac{f(a)}{x}\xrightarrow[x\to+\infty]{}l\end{equation}

		\item S'il existe $a<b\in(\R_{+}^{*})^{2}$ tel que $f(a)<f(b)$, alors $\tau_{a}(b)>0$. Comme $\tau_{a}$ est croissante, $l\geqslant\tau_{a}(b)>0$. Par contraposée, si $l\geqslant0$, $f$ est décroissante.
		\item Posons pour tout $x\in\R_{+}^{*}$, $\varphi(x)=f(x)-lx$. Pour $x<y$, on a 
		\begin{equation}\frac{\varphi(y)-\varphi(x)}{y-x}=\frac{f(y)-f(x)}{y-x}-l\leqslant0\end{equation}
		Donc $\varphi$ est décroissante et $\lim\limits_{x\to+\infty}\varphi(x)\in\overline{\R}$ existe.
	\end{enumerate}
\end{proof}

\begin{proof}
	\phantom{}
	\begin{enumerate}
		\item On forme \function{g}{[0,1]}{\R}{x}{\frac{1}{\frac{1}{p}+x}}
		Alors 
		\begin{equation}\sum_{k=0}^{np}\frac{1}{n+k}=\frac{1}{np}\sum_{k=0}^{np}\frac{1}{\frac{1}{p}+\frac{k}{np}}\xrightarrow[n\to+\infty]{}\int_{0}^{1}\frac{dx}{\frac{1}{p}+x}=\ln(p+1)=l_{p}\end{equation}

		\item On note $f(x)=f(0)+xf'(0)+x\varepsilon(x)$ avec $\varepsilon(x)\xrightarrow[\varepsilon\to0]{}0$. 
		
		Soit $\varepsilon_{0}>0$. Il existe $\alpha_{0}>0$ tel que si $0<x<\alpha_{0}$, alors $\vert\varepsilon(x_{0})\vert\leqslant\varepsilon_{0}$, et il existe $N_{0}\in\N$ tel que pour tout $n\geqslant N_{0}$, $\frac{1}{n}\leqslant\alpha_{0}$. Alors pour tout $n\geqslant N_{0}$, pour tout $k\in\{0,\dots,np\}$, 
		\begin{equation}\frac{1}{k+n}\Rightarrow \Biggl\vert\varepsilon\Bigl(\frac{1}{k+n}\Bigr)\Biggr\vert\leqslant\frac{\varepsilon_{0}}{p}\end{equation}
		et
		\begin{equation}\Biggl\vert\sum_{k=0}^{np}\frac{\varepsilon(\frac{1}{k+n})}{k+n}\Biggr\vert\leqslant\sum_{k=0}^{np}\frac{\frac{\varepsilon_{0}}{p}}{k+n}\leqslant\frac{\varepsilon_{0}}{p}\frac{np+1}{n+1}\leqslant\varepsilon_{0}\end{equation}

		On a donc
		\begin{equation}v_{n}=\sum_{k=0}^{np}\frac{1}{n+k}f'(0)+\sum_{k=0}^{np}\frac{\varepsilon(\frac{1}{n+k})}{n+k}\xrightarrow[n\to+\infty]{}\ln(p+1)f'(0)\end{equation}

		\item On peut penser à $f\colon x\mapsto\sqrt{x}$ continue et $f(0)=0$. De plus,
		\begin{equation}\sum_{k=0}^{np}\frac{1}{\sqrt{n+k}}\geqslant\frac{np+1}{\sqrt{n(p+1)}}\xrightarrow[n\to+\infty]{}+\infty\end{equation}
		donc $v_{n}$ diverge.

		\item On écrit $f(x)=f(0)+xf'(0)+\frac{x^{2}}{2!}f''(0)+x^{2}\varepsilon(x)$ avec $\varepsilon(x)\xrightarrow[\varepsilon\to+\infty]{}0$. Ainsi, 
		\begin{equation}v_{n}=\sum_{k=0}^{np}\frac{f''(0)}{2(n+k)^{2}}+\sum_{k=0}^{bp}\frac{\varepsilon(\frac{1}{k+n})}{(k+n)^{2}}\end{equation}
		Soit $\varepsilon>0$, il existe $N\in\N$ tel que pour tout $n\geqslant N$, pour tout $k\in\{0,\dots,np\}$, $\vert\varepsilon(\frac{1}{n+k})\vert\leqslant\varepsilon$ et donc 
		\begin{equation}\Biggl\vert\sum_{k=0}^{np}\frac{\varepsilon(\frac{1}{n+k})}{(n+k)^{2}}\Biggr\vert\leqslant\sum_{k=0}^{np}\frac{\varepsilon}{(n+k)^{2}}\end{equation}
		donc 
		\begin{equation}\sum_{k=0}^{np}\frac{\varepsilon(\frac{1}{n+k})}{(n+k)^{2}}=O\Biggl(\sum_{k=0}^{np}\frac{f''(0)}{2}\times\frac{1}{(n+k)^{2}}\Biggr)\end{equation}
		puis
		\begin{equation}v_{n}\underset{n\to+\infty}{\sim}\sum_{k=0}^{np}\frac{f''(0)}{2(n+k)^{2}}\end{equation}
		Or 
		\begin{align}
			\sum_{k=0}^{np}\frac{1}{(n+k)^{2}}
			&=\frac{1}{(np)^{2}}\sum_{k=0}^{np}\frac{1}{(\frac{1}{p}+\frac{k}{np})^{2}}\\
			&=\frac{1}{np}\times \underbrace{\frac{1}{np}\sum_{k=0}^{np}\frac{1}{(\frac{1}{p}+\frac{k}{np})^{2}}}_{\xrightarrow[n\to+\infty]{}\int_{0}^{1}\frac{dx}{(\frac{1}{p}+x)^{2}}}
		\end{align}
		donc 
		\begin{equation}v_{n}\underset{n\to+\infty}{\sim}\frac{f''(0)p}{n(p+1)}\end{equation}
	\end{enumerate}
\end{proof}

\begin{proof}
	Supposons que $f'$ ne tend pa vers 0 en $+\infty$: il existe $\varepsilon_{0}>0,\forall A>0,\exists x_{A}\geqslant A,\vert f'(x_{A})\vert\geqslant\varepsilon_{0}>0$. Par continuité uniforme, il existe $\alpha_{0}\geqslant0$, $\forall(x,y)\in(\R_{+})^{2}$, si $\vert x-y\vert\leqslant\alpha_{0}$ alors $\vert f'(x)-f'(y)\vert\leqslant\frac{\varepsilon_{0}}{2}$. Alors pour tout $t\in[x_{A}-\alpha,x_{A}+\alpha]$, on a 
	\begin{equation}\vert f'(t)\vert\geqslant \vert f'(x_{A})\vert-\vert f'(x_{A})-f'(t)\vert\geqslant\varepsilon_{0}-\frac{\varepsilon_{0}}{2}\geqslant\frac{\varepsilon_{0}}{2}\end{equation}
	et pour $A=n$, pour tout $n\in\N,\exists x_{n}\geqslant n,\forall t\in[x_{n}-\alpha,x_{n}+\alpha],\vert f'(t)\vert\geqslant\frac{\varepsilon_{0}}{n}$. D'après le théorème des valeurs intermédiaires, $f'$ est de signe constant sur $[x_{n}-\alpha,x_{n}+\alpha]$. Quitte à changer $f$ en $-f$, on peut supposer qu'il existe une infinité de $n\in\N$ tels que $f'>0$ sur les $[x_{n}-\alpha,x_{n}+\alpha]$. Alors
	\begin{equation}f(x_{n}+\alpha_{0})-f(x_{n}-\alpha_{0})=\int_{x_{n}-\alpha_{0}}^{x_{n}+\alpha_{0}}f'(t)dt\geqslant\varepsilon_{0}\alpha_{0}>0\end{equation}
	mais comme $\lim\limits_{x\to+\infty}f(x)\in\R$, on a 
	\begin{equation}\lim\limits_{n\to+\infty}f(x_{n}+\alpha_{0})-f(x_{n}-\alpha_{0})=0\end{equation}
	d'où la contradiction.

	Si $f\in\mathcal{C}^{1}(\R_{+},\C)$, on applique ce qui précède à $\Im(f)$ et $\Re(f)$. 

	Si $f'$ n'est pas uniformément continue, ce n'est plus valable, par exemple 
	\begin{equation}f(x)=\frac{\sin(x^{2})}{x}\xrightarrow[x\to+\infty]{}0\end{equation}
	car $\vert f(x)\vert\leqslant\frac{1}{x}$ et 
	\begin{equation}f'(x)=\underbrace{-\frac{1}{x^{2}}\sin(x^{2})}_{\xrightarrow[x\to+\infty]{}0}+\underbrace{\frac{2x\cos(x^{2})}{x}}_{\text{n'a pas de limite en }+\infty}\end{equation}
\end{proof}

\begin{proof}
	Soit $x\in\R$ et $h\neq0$, on a 
	\begin{equation}\frac{f(x+h)-f(x)}{h}=g(x+\frac{h}{2})\xrightarrow[h\to0]{}g(x)\end{equation}
	par continuité de $g$. Donc $f$ est dérivable et $f'=g$. Par ailleurs, pour $y=\frac{1}{2}$, on a 
	\begin{equation}f'(x)=f(x+\frac{1}{2})-f(x-\frac{1}{2})\end{equation}
	par récurrence $f$ est $\mathcal{C}^{\infty}$.

	En outre, en fixant $x$ et en dérivant la relation de départ deux fois par rapport à $y$, on a 
	\begin{equation}f''(x+y)-f''(x-y)=0\end{equation}
	Donc $f''$ est constante donc $f$ est un polynôme de degré plus petit que 2.

	Réciproquement, on vérifie que ces fonctions marchent (avec $f'=g$).
\end{proof}

\begin{proof}
	On a 
	\begin{equation}S_{n}=\sum_{k=1}^{n-1}\frac{1}{2}(f(k)+f(k+1))-\int_{k}^{k+1}f(t)dt\end{equation}
	On note $F(x)=\int_{1}^{x}f(t)dt$ de classe $\mathcal{C}^{2}$.

	On a
	\begin{equation}F(b)=F(a)+F'(a)(b-a)+\int_{a}^{b}F''(t)(b-t)dt\end{equation}
	Pour $a=k$ et $b=k+\frac{1}{2}$, on a 
	\begin{equation}F(k+\frac{1}{2})=F(k)+\frac{1}{2}F'(k)+\int_{k}^{k+\frac{1}{2}}(k+\frac{1}{2}-t)f'(t)dt=F(k)+\frac{1}{2}F'(k)+\int_{0}^{\frac{1}{2}}uf'(k+\frac{1}{2}-u)du\end{equation}
	et pour $a=k+1,b=k+\frac{1}{2}$,
	\begin{equation}F(k+\frac{1}{2})=F(k+1)-\frac{1}{2}F'(k+1)+\int_{k+1}^{k+\frac{1}{2}}(k+\frac{1}{2}-t)f'(t)dt=F(k+1)-\frac{1}{2}F'(k+1)+\int_{0}^{\frac{1}{2}}uf'(k+\frac{1}{2}+u)du\end{equation}

	On a donc
	\begin{equation}\frac{1}{2}(f(k)-f(k+1))-\int_{k}^{k+1}f(t)dt=\int_{0}^{\frac{1}{2}}u(f'(k+\frac{1}{2}+u)-f'(k+\frac{1}{2}-u))du\end{equation}
	d'où
	\begin{equation}S_{n}=\int_{0}^{\frac{1}{2}}u\sum_{k=1}^{n-1}\underbrace{f'(k+\frac{1}{2}+u)-f'(k+\frac{1}{2}-u)}_{\geqslant0\text{ car }u\geqslant0\text{ et }f'\text{ croissante}}du\end{equation}
	et 
	$f'(k+\frac{1}{2}+u)-f'(k+\frac{1}{2}-u)\leqslant f'(k+1)-f'(k)$ d'où 
	\begin{equation}S_{n}\leqslant\underbrace{\int_{0}^{\frac{1}{2}}udu}_{=\frac{1}{8}}(f'(n)-f'(1))\end{equation}
\end{proof}

\begin{proof}
	\phantom{}
	\begin{enumerate}
		\item D'après l'inégalité de Taylor-Lagrange, on a 
		\begin{equation}
		\left\{
			\begin{array}[]{l}
				\Vert A\Vert\leqslant\frac{h^{2}}{2}M_{2}\\
				\Vert B\Vert\leqslant\frac{h^{2}}{2}M_{2}
			\end{array}
		\right.
		\end{equation}
		On a $B-A-f(x-h)+f(x+h)=2hf'(x)$ d'où 
		\begin{equation}\Vert f'(x)\Vert\leqslant\frac{hM_{2}}{2}+\frac{M_{0}}{h}\end{equation}
		Donc $f'$ est bornée sur $\R$. On a ensuite un majorant qui dépend de $h$ que l'on peut optimiser, et on trouve la borne demandée.

		\item L'inégalité de Taylor-Lagrange donne à nouveau
		\begin{equation}\forall k\in\{1,\dots,n-1\},\Vert A_{k}\Vert\leqslant\frac{k^{n}}{n!}M_{n}\end{equation}
		On forme alors
		\begin{equation}
		\begin{pmatrix}
			A_{1}-f(x+1)\\
			\vdots\\
			A_{k}-f(x+k)\\
			\vdots\\
			A_{n}-f(x+n)
		\end{pmatrix}
		=
		\underbrace{
		\begin{pmatrix}
			-1 & -1 & \dots & \frac{-1}{(n-1)!}\\
			\vdots & \vdots & & \vdots\\
			-1 & -k & \dots & \frac{-k^{n-1}}{(n-1)!}\\
			\vdots & \vdots & & \vdots\\
			-1 & -n & \dots & \frac{-n^{n-1}}{(n-1)!}\\
		\end{pmatrix}}_{=M}
		\begin{pmatrix}
			f(x)\\
			\vdots\\
			f^{(k)}(x)\\
			\vdots\\
			f^{(n-1)}(x)
		\end{pmatrix}
		\end{equation}

		On a 
		\begin{equation}\det(M)=\frac{(-1)^{n}}{1!\times 2!\times\dots\times (n-1)!}V(1,\dots,n)\end{equation}
		où $V$ est le déterminant de Vandermonde. Donc $\det(M)\neq0$. On peut former les $f^{(j)}(x)$ en fonction des $(A_{i}-f(x+i))_{1\leqslant i\leqslant n}$: il existe $(\alpha_{1},\dots,\alpha_{n})\in\R^{n}$ tel que pour tout $x\in\R$, $f^{(j)}(x)=\sum_{i=1}^{n}\alpha_{i}(A_{i}-f(x+i))$. Donc 
		\begin{equation}\Vert f^{(j)}(x)\Vert\leqslant\sum_{i=1}^{n}\vert\alpha_{i}\vert\Bigl(\frac{n}{n!}M_{n}+M_{0}\Bigr)\end{equation}
		Donc $f^{(j)}$ est bornée pour tout $j\in\{1,\dots,n-1\}$.
	\end{enumerate}
\end{proof}

\begin{proof}
	\phantom{}
	\begin{enumerate}
		\item 
		\begin{equation}l_{\sigma,\gamma}=\sum_{i=0}^{n-1}\Bigl\Vert\int_{a_{i}}^{a_{i+1}}\gamma'(t)dt\Bigr\Vert\leqslant\sum_{i=0}^{n-1}\int_{a_{i}}^{a_{i+1}}\Vert\gamma'(t)\Vert dt=\int_{a}^{b}\Vert\gamma'(t)\Vert dt\end{equation}

		\item On a 
		\begin{align}
			\Bigl\vert l_{\sigma,\gamma}-\sum_{i=0}^{n-1}\Vert\gamma'(a_{i})\Vert(a_{i+1}-a_{i})\Bigr\vert
			&=\Bigl\vert\sum_{i=0}^{n-1}\Vert\gamma(a_{i+1})-\gamma(a_{i})\Vert-\Vert\underbrace{(a_{i+1}-a_{i})}_{>0}\gamma'(a_{i})\Vert\Bigr\vert\\
			&\leqslant\sum_{i=0}^{n-1}\Vert\gamma(a_{i+1})-\gamma(a_{i})-(a_{i+1}-a_{i})\gamma'(a_{i})\Vert\\
			&\leqslant\sum_{i=0}^{n-1}\int_{a_{i}}^{a_{i+1}}\Vert\gamma'(t)-\gamma'(a_{i})\Vert dt
		\end{align}

		\item $\Vert\gamma'\Vert$ est continue donc 
		\begin{equation}\int_{a}^{b}\Vert\gamma'(t)\Vert dt=\lim\limits_{\delta(\sigma)\to0}\sum_{i=0}^{n-1}\Vert\gamma'(a_{i})\Vert(a_{i+1}-a_{i})\end{equation}
		Donc $\alpha_{0}$ existe.

		$\gamma'$ est continue sur $[a,b]$ donc uniformément continue sur $[a,b]$, et il existe $\alpha_{1}>0$ tel que pour tout $(x,y)\in[a,b]^{2}$, on a 
		\begin{equation}\vert x-y\vert\leqslant\alpha_{}\Rightarrow\Vert\gamma'(x)-\gamma'(y)\Vert\leqslant\frac{\varepsilon}{2(b-a)}\end{equation}
		Alors si $\delta(\sigma)\leqslant\alpha_{1}$, pour tout $i\in\{0,\dots,n-1\}$, pour tout $t\in[a_{i},a_{i+1}]$, on a
		\begin{equation}\vert t-a_{i}\vert\leqslant(a_{i+1}-a_{i})\leqslant\alpha_{1}\end{equation}
		d'où 
		\begin{equation}\Vert \gamma'(a_{i})-\gamma'(t)\Vert\leqslant\frac{\varepsilon}{2(b-a)}\end{equation}
		et d'après la question 2, on a donc 
		\begin{equation}\Bigl\vert l_{\sigma,\gamma}-\sum_{i=0}^{n-1}\Vert\gamma'(a_{i})\Vert(a_{i+1}-a_{i})\Bigr\vert\leqslant\frac{\varepsilon}{2}\end{equation}

		Finalement, si $@d(\sigma)\leqslant\min(\alpha_{0},\alpha_{1})$, on a 
		\begin{equation}\Bigl\vert l_{\sigma,\gamma}-\int_{a}^{b}\Vert\gamma'(t)\Vert dt\Bigr\vert\leqslant\varepsilon\end{equation}
		Donc 
		\begin{equation}l(\gamma)=\int_{a}^{b}\Vert\gamma'(t)\Vert dt\end{equation}

		\item On a 
		\begin{equation}\gamma'(t)=\begin{pmatrix}
			-R\sin(t)\\
			R\cos(t)
		\end{pmatrix}\end{equation}
		donc $\Vert\gamma'(t)\Vert=R$ et $l(\gamma)=2\pi R$.
	\end{enumerate}
\end{proof}

\begin{proof}
	\phantom{}
	\begin{enumerate}
		\item Pour tout $t\in I$, on a 
		\begin{equation}\gamma(t)=\vert\gamma(t)\vert e^{\mathrm{i}\theta_{1}(t)}=\vert\gamma(t)\vert e^{\mathrm{i}\theta_{2}(t)}\end{equation}
		donc 
		\begin{equation}e^{\mathrm{i}(\theta_{1}(t)-\theta_{2}(t))}=1\end{equation}

		Ainsi, pour tout $t\in I$, il existe $k(t)\in\Z$ telle que $\theta_{2}(t)-\theta_{1}(t)=2k(t)\pi$. On a 
		\begin{equation}k(t)=\frac{\theta_{2}(t)-\theta_{1}(t)}{2\pi}\end{equation}
		qui est continue et à valeurs entières, donc constante égale à $k_{0}$ d'après le théorème des valeurs intermédiaires.

		\item Si $\gamma(t)=x(t)+\mathrm{i}y(t)$, 
		\begin{equation}\vert\gamma(t)\vert=\sqrt{x(t)^{2}+y(t)^{2}}\end{equation}
		Comme $\sqrt{\cdot}$ est $\mathcal{C}^{\infty}$ sur $\R_{+}^{*}$, par composition, $f$ est $\mathcal{C}^{k}$. On a alors
		\begin{equation}f(t)=e^{\mathrm{i}\theta(t)}\Rightarrow f'(t)=\mathrm{i}\theta'(t)e^{\mathrm{i\theta(t)}}=\mathrm{i}\theta'(t)f(t)\end{equation}
		Donc 
		\begin{equation}\theta(t)=-\mathrm{i}\frac{f'(t)}{f(t)}\end{equation}

		De plus, on a 
		\begin{equation}\theta(t)=\theta(t_{0})-\mathrm{i}\int_{t_{0}}^{t}\frac{f'(u)}{f(u)}du\end{equation}
		pour $t_{0}\in I$.

		\item On fixe $t_{0}\in I$. Soit $\theta_{0}$ un argument de $\gamma(t_{0})$, on pose 
		\begin{equation}\theta(t)=\theta_{0}-\mathrm{i}\int_{t_{0}}^{t}\frac{f'(u)}{f(u)}du\end{equation}
		
		Comme $\frac{f'}{f}$ est $\mathcal{C}^{k-1}$, $\theta$ est bien $\mathcal{C}^{k}$. On forme $g(t)=e^{\mathrm{i}\theta(t)}$ qui est de classe $\mathcal{C}^{k}$. On a 
		\begin{equation}g'(t)=\mathrm{i}\theta'(t)g(t)=\frac{f'(t)}{f(t)}g(t)\end{equation}
		donc $\Bigl(\frac{g}{f}\Bigr)'=0$, donc $\frac{g}{f}$ est constante sur $I$ et $g(t_{0})=e^{\mathrm{i}\theta_{0}}=f(t_{0})$ donc $g=f$ sur $I$. Ainsi, pour tout $t\in I$, on a $\vert f(t)\vert=\vert e^{\mathrm{i}\theta(t)}\vert=1$ et si $\theta(t)=a(t)+\mathrm{i}(t)$, on a donc 
		\begin{equation}e^{\mathrm{i}\theta(t)}=e^{-b(t)}e^{\mathrm{i}a(t)}\end{equation}
		donc $b(t)=0$ et $\theta(t)\in\R$.
	\end{enumerate}
\end{proof}

\end{document}
\documentclass[12pt]{article}
\usepackage{style/style_sol}

\begin{document}

\begin{titlepage}
	\centering
	\vspace*{\fill}
	\Huge \textit{\textbf{Solutions MP/MP$^*$\\ Suites et séries de fonctions}}
	\vspace*{\fill}
\end{titlepage}

\begin{proof}
    Pour $x\geqslant0$, on a $F_{n}(x)>0$, on a 
    \begin{equation*}
        \ln\left(F_{n}(x)\right)=\frac{1}{n}\sum_{k=1}^{n}\ln\left(1+\frac{kx}{n}\right)\xrightarrow[n\to+\infty]{}\int_{0}^{1}\ln\left(1+tx\right)dt=G(x)
    \end{equation*}

    On a $G(0)=0$ et pour $x>0$, on a
    \begin{align*}
        G(x)
        &= \left[\left(t+\frac{1}{x}\right)\ln(1+tx)\right]_{0}^{1}-\int_{0}^{1}\frac{x}{1+tx}\left(t+\frac{1}{x}\right)dt\\
        &=\frac{x+1}{x}\ln(1+x)-1
    \end{align*}
    (utiliser le fait que $G$ est continue sur $[0,1]$ et que $\ln(1+x)\underset{x\to0}{\sim}x$).

    Ainsi, $\lim\limits_{n\to+\infty}F_{n}(0)=1=F(0)$.
    Pour $x>0$, $\lim\limits_{n\to+\infty}F_{n}(x)=\left(1+x\right)^{\frac{x+1}{x}}\times\frac{1}{e}=F(x)$.

    $F$ est continue sur $[0,1]$. Soit $x\geqslant0$. On écrit 
    \begin{equation*}
        \left\lvert F_{n}(x)-F(x)\right\rvert=\left\lvert\e^{G_{n}(x)}-\e^{G(x)}\right\rvert
    \end{equation*}

    
    On a d'après l'inégalité des accroissements finis: 
    \begin{equation*}
        \left\lvert F_{n}(x)-F(x)\right\rvert\leqslant\e^{G_{n}(x)}\left\lvert G_{n}(x)-G(x)\right\rvert\leqslant\e^{G_{n}(x)}\times\frac{x}{2n}
    \end{equation*}
    
    Si $f(t)=\ln\left(1+tx\right)$, on a $f'(t)=\frac{x}{1+tx}\geqslant0$. Donc $f$ est croissante et $G_{n}(x)=\frac{1}{n}\sum_{k=1}^{n}f\left(\frac{k}{n}\right)\leqslant\ln(1+x)$. Finalement, 
    \begin{equation*}
        \left\lvert F_{n}(x)-F(x)\right\rvert\leqslant\frac{x(1+x)}{2n}
    \end{equation*}
    On a donc convergence uniforme sur $[0,A]$ pour tout $A\geqslant0$.
\end{proof}

\begin{proof}
    \phantom{}
    \begin{enumerate}
        \item Si $x=0$, on a $u_{n}(0)=0$ donc $\sum u_{n}(0)$ converge.
        Si $x\neq0$, on a 
        \begin{equation*}
            \left\lvert\frac{U_{n+1}(x)}{U_{n}(x)}\right\rvert=\frac{2n+1}{2n+1+\alpha}\left\lvert x\right\rvert\xrightarrow[n\to+\infty]{}\left\lvert x\right\rvert
        \end{equation*}
        Ainsi, si $\left\lvert x\right\rvert<1$, d'après la règle de d'Alembert, $\sum u_{n}(x)$ converge absolument. Si $\left\lvert x\right\rvert>1$, il existe un rang $n_{0}$ à partir duquel $\left\lvert U_{n+1}(x)\right\rvert>\left\lvert U_{n}(x)\right\rvert$, donc $(U_{n}(x))_{n\in\N}$ ne converge pas vers 0: $\sum u_{n}(x)$ diverge.

        Si $x=1$, il existe $N_{0}\in\N$ tel que pour tout $n\geqslant N_{0}$, on a $\frac{U_{n+1}(1)}{U_{n}(1)}>0$ donc $(u_{n})_{n\geqslant N_{0}}$ gare un signe constant. On a 
        \begin{equation*}
            \frac{u_{n+1}(1)}{u_{n}(1)}=\frac{2n+1}{2n+1+\alpha}=1-\frac{\alpha}{2n+1+\alpha}=1-\frac{\alpha}{2n}+\underset{n\to+\infty}{O}\left(\frac{1}{n^{2}}\right)
        \end{equation*}
        Ainsi, d'après la règle de Raabe-Duhamel, on a 
        \begin{equation*}
            \left\lvert U_{n}(1)\right\rvert\underset{n\to+\infty}{\sim}\frac{C}{n^{\frac{\alpha}{2}}}
        \end{equation*}
        Ainsi, on a convergence si et seulement si $\alpha>2$.

        Si $x=-1$, on a toujours $\left\lvert U_{n}(-1)\right\rvert=\left\lvert U_{n}(1)\right\rvert\underset{n\to+\infty}{\sim}\frac{C}{n^{\frac{\alpha}{2}}}$. Si $\sum u_{n}(-1)$ converge, on a $\alpha>0$. Réciproquement, si $\alpha>0$, on a $\left\lvert U_{n}(-1)\right\rvert\xrightarrow[n\to+\infty]{}0$ et $\sum u_{n}(-1)$ est une série alternée. On a donc 
        \begin{equation*}
            \left\lvert\frac{u_{n+1}(-1)}{u_{n}(-1)}\right\vert=\frac{2n+1}{2n+1+\alpha}<1
        \end{equation*}
        donc $\left(\left\lvert u_{n}(-1)\right\vert\right)_{n\in\N}$ est décroissante: d'après le critère spéciale des séries alternées, $\sum u_{n}(-1)$ converge. Ainsi, $\sum u_{n}(-1)$ converge si et seulement si $\alpha>0$.

        \item Supposons la convergence uniforme sur $[0,1[$. Comme pour tout $n\geqslant1$, $\lim\limits_{x\to1^{-}}u_{n}(x)=u_{n}(1)$, d'après le théorème d'interversion des limites, comme il ya convergence uniforme au voisinage de 1, $\sum u_{n}(1)$ converge. Donc d'après ce qui précède, on a $\alpha>2$.
        
        Réciproquement, si $\alpha>2$, pour tout $x\in[0,1]$, on a $\left\lvert u_{n}(x)\right\rvert\leqslant\left\lvert u_{n}(1)\right\rvert$ (terme général d'une série à termes positifs convergente). Donc on a convergence normale sur $[0,1]$.

        \item Supposons convergence uniforme sur $]-1,0]$. Comme pour tout $n\geqslant1$, $\lim\limits_{x\to-1^{+}}u_{n}(x)=u_{n}(-1)$. D'après le théorème d'interversion des limites, comme il y a convergence uniforme au voisinage de $-1$, $\sum u_{n}(-1)$ converge. D'après ce qui précède, on a $\alpha>0$. 
        
        Réciproquement, si $\alpha>0$, soit $x\in[-1,0]$, $\sum u_{n}(x)$ est alternée dont le terme général décroît en valeur absolue (et tend vers 0). Donc pour tout $N\geqslant1$, on a 
        \begin{equation*}
            \left\lvert\sum_{n=N}^{+\infty}u_{n}(x)\right\rvert\leqslant\left\lvert u_{N}(x)\right\rvert\leqslant\left\lvert u_{N}(-1)\right\rvert\xrightarrow[N\to+\infty]{}0
        \end{equation*}

        On a donc convergence uniforme de $\sum u_{n}(x)$ sur $[-1,0]$
        \end{enumerate}
\end{proof}

\begin{remark}
    Pour rappel, on redonne la règle de Raabe-Duhamel: si $(v_{n})_{n\in\N}\in\left(\R_{+}\right)^{\N}$ et 
    \begin{equation*}
        \frac{v_{n+1}}{v_{n}}=1-\frac{\beta}{n}+\underset{n\to+\infty}{\frac{1}{n^{2}}}
    \end{equation*}
    alors il existe $C>0$ telle que $v_{n}\underset{n\to+\infty}{\sim}\frac{C}{n^{\beta}}$. En effet, on écrit 
    \begin{equation*}
        \ln\left(\left(n+1\right)^{\beta}v_{n+1}\right)-\ln\left(n^{\beta}v_{n}\right)=\beta\ln\left(1+\frac{1}{n}\right)+\ln\left(\frac{v_{n+1}}{v_{n}}\right)=\underset{n\to+\infty}{O}\left(\frac{1}{n^{2}}\right)
    \end{equation*}
    donc $(n^{\beta}v_{n})_{n\in\N}$ converge dans $\R_{+}^{*}$.
\end{remark}

\begin{remark}
    On peut aussi éviter la règle de Raabe-Duhamel. On forme 
    \begin{align*}
        \ln\left(\left\lvert u_{n}(1)\right\rvert\right)
        &=
        \sum_{k=1}^{n}\ln\left\lvert\frac{2k-1}{2k-1+\alpha}\right\rvert,\\
        &=
        -\frac{\alpha}{2}\sum_{k=1}^{n}\frac{1}{k}+\sum_{k=1}^{n}\underset{k\to+\infty}{O}\left(\frac{1}{k^{2}}\right),\\
        &=-\frac{\alpha}{2}\ln(n)-\frac{\gamma\alpha}{2}+K+\underset{n\to+\infty}{o}\left(1\right)
    \end{align*}
    donc $\left\lvert u_{n}(1)\right\rvert\underset{n\to+\infty}{\sim}\frac{C}{n^{\frac{\alpha}{2}}}$ avec $C>0$.
\end{remark}

\begin{proof}
    Pour $k\geqslant \left\lfloor x\right\rfloor$, on a 
    \begin{equation*}
        \arctan(k+x)-\arctan(k)\in\left]\frac{-\pi}{2},\frac{\pi}{2}\right[
    \end{equation*}
    On a 
    \begin{equation*}
        f_{k}(x)=\arctan\left(\frac{x}{1+k(k+x)}\right)=\arctan\left(\frac{x}{k^{2}}+\underset{k\to+\infty}{o}\left(\frac{1}{k^{2}}\right)\right)\underset{k\to+\infty}{\sim}\frac{x}{k^{2}}
    \end{equation*}

    Pour tout $x\in\R$, $\sum_{k\in\N}f_{k}(x)$ converge absolument et $f$ définie sur $\R$. Pour tout $k\in\N$, $f_{k}$ est $\mathcal{C}^{1}$ sur $\R$ et 
    \begin{equation*}
        f_{k}'(x)=\frac{1}{1+(k+x)^{2}}
    \end{equation*}

    On fixe $[a,b]\subset\R$, pour tout $x\in[a,b]$, pour tout $k\in\N$, on a \begin{equation*}
        \left\lvert k+x\right\rvert\geqslant k-\left\lvert x\right\rvert \geqslant k-\underbrace{\max\left(\left\lvert a\right\rvert,\left\lvert b\right\rvert\right)}_{=M}\geqslant 0    
    \end{equation*}
    pour $k\geqslant \left\lfloor M+1\right\rfloor$.

    On a de plus $0\leqslant f_{k}'(x)\leqslant \frac{1}{1+(k-M)^{2}}$ (terme général d'une série à termes positifs convergente). On $\sum_{k\geqslant\left\lfloor M\right\rfloor+1}f_{k}'$ converge normalement sur $[a,b]$. Enfin, 
    \begin{equation*}
        f-\sum_{k=1}^{\left\lfloor M\right\rfloor}f_{k}=\sum_{k=\left\lfloor M\right\rfloor+1}^{+\infty}f_{k}
    \end{equation*}
    est donc $\mathcal{C}^{1}$ sur $[a,b]$ d'après le théorème de dérivation terme à terme, donc $f$ est $\mathcal{C}^{1}$ sur $[a,b]$ (car $\sum_{k=1}^{\left\lfloor M\right\rfloor}f_{k}$ est une somme finie de fonctions $\mathcal{C}^{1}$ donc est $\mathcal{C}^{1}$ sur $\R$). Ainsi $f$ est $\mathcal{C}^{1}$ sur $\R$.

    Soit $n\in\N$ et $N\in\N$, on a 
    \begin{align*}
        \sum_{k=0}^{N}f_{k}(n)
        &=\sum_{k=0}^{N}\arctan(k+n)-\arctan(k)\\
        &=\sum_{k=n}^{n+N}\arctan(k)-\sum_{k=0}^{N}\arctan(k)\\
        &=\sum_{k=N+1}^{n+N}\arctan(k)-\sum_{k=0}^{n-1}\arctan(k)\\
        &\xrightarrow[N\to+\infty]{}n\frac{\pi}{2}-\sum_{k=0}^{n-1}\arctan(k)=\frac{\pi}{2}+\sum_{k=1}^{n-1}\arctan(\frac{1}{k})=f(n)
    \end{align*}

    On a $\arctan\left(\frac{1}{k}\right)\underset{k\to+\infty}{\sim}\frac{1}{k}>0$, d'après le théorème de comparaison des sommes partielles de séries à termes positifs divergente, donc $f(n)\underset{n\to+\infty}{\sim}\ln(n)$. Par ailleurs, $f$ est croissante sur $\R$, donc pour tout $x\geqslant0$, on a 
    \begin{equation*}
        \ln\left\lvert x\right\rvert\underset{x\to+\infty}{\sim}f\left(\left\lfloor x\right\rfloor\right)\leqslant f(x)\leqslant f\left(\left\lfloor x\right\rfloor+1\right)\underset{x\to+\infty}{\sim}\ln(x)
    \end{equation*}
    donc $f(x)\underset{x\to+\infty}{\sim}\ln(x)$.
\end{proof}

\begin{proof}
    Soit $t>0$, on a $\ln\left(1-\e^{-nt}\right)\underset{n\to+\infty}{\sim}-\e^{-nt}$ car $\lim\limits_{n\to+\infty}-\e^{nt}=0$ (terme général d'une série à termes positifs convergente car $t>0$).

    On définit \function{g_{+}}{\R_{+}^{*}}{\R}{x}{-\ln\left(1-\e^{-xt}\right)\geqslant0}

    On a $-f(t)=\sum_{n=1}^{+\infty}g_{t}(x)$. De plus, $g_{t}'(x)=-\frac{t\e^{-xt}}{1-\e^{-xt}}\leqslant0$. $g_{+}$ est décroissante, et on a 
    \begin{equation*}
        \int_{n}^{n+1}g_{+}(x)dx\leqslant g_{+}(x)\leqslant\int_{n-1}^{n}g_{+}(x)dx
    \end{equation*}

    On somme de $n=1$ à $+\infty$ (on admet l'existence pour $n=0$). On obtient
    \begin{equation*}
        -\ln\left(1-\e^{-xt}\right)=\int_{1}^{+\infty}g_{+}(x)dx\leqslant -f(t)\leqslant\int_{0}^{+\infty}-\ln\left(1-\e^{-xt}\right)dx
    \end{equation*}
    On pose $u=xt$ et $dx=\frac{du}{t}$ car $t>0$. On a 
    \begin{equation*}
        \int_{1}^{+\infty}-\ln\left(1-\e^{-xt}\right)dx=\frac{1}{t}\int_{t}^{+\infty}-\ln\left(1-\e^{u}\right)du\underset{t\to0}{\sim}\frac{1}{t}I
    \end{equation*}
    et 
    \begin{equation*}
        \int_{0}^{+\infty}-\ln\left(1-\e^{-xt}\right)dx=\frac{1}{t}\int_{0}^{+\infty}-\ln\left(1-\e^{-u}\right)du=\frac{I}{t}
    \end{equation*}
    donc 
    \begin{equation*}
        \boxed{
            f(t)\underset{t\to+0^{+}}{\sim}-\frac{I}{t}
        }
    \end{equation*}
\end{proof}

\begin{proof}
    \phantom{}
    \begin{enumerate}
        \item On a $\left\lVert f_{n}\right\rVert_{\infty}=\frac{1}{2}$ donc 
        \begin{equation*}
            \left\lvert\frac{1}{2^{p}}f_{n}(x-a_{p})\right\rvert\leqslant\frac{1}{2^{p+1}},
        \end{equation*}
        et $g_{n}(x)$ est définie. Soit $x\in\R$, on pose \function{F_p}{\N}{\R}{n}{\frac{1}{2^{p}}f_n(x-a_p)}
        On a $\left\lvert F_p(n)\right\rvert\leqslant\frac{1}{2^{p+1}}$ et pour $p$ fixé, $\lim\limits_{n\to+\infty}F_p(n)=0$. Donc $\sum_{p\geqslant0}F_p$ converge normalement sur $\N$. D'après le théorème d'interversion des limites, on a 
        \begin{equation*}
            \boxed{
                \lim\limits_{x\to+\infty}f_n(x)=0.
                }
        \end{equation*}

        \item S'il existe $p_0\in\N$ tel que $a_{p_{0}}\in[a,b]$, alors il existe $N\in\N$ tel que pour tout $n\geqslant N$, $a_{p_{0}}+\frac{1}{n}$ ou $a_{p_{0}}-\frac{1}{n}\in[a,b]$ et $g_{n}(a_{p_{0}}\pm \frac{1}{n})\geqslant \frac{1}{2^{p_{0}+1}}$ (série à termes positifs).
        
        Si pour tout $p\in\N$, $a_{p}\notin[a,b]$, soit $\varepsilon>0$. Il existe $N_{0}\in\N$ tel que $\sum_{p=N_{0}+1}^{+\infty}\frac{1}{2^{p+1}}\leqslant\frac{\varepsilon}{2}$. Alors pour tout $x\in[a,b]$, on a 
        \begin{equation*}
            0\leqslant\sum_{p=N_{0}+1}^{+\infty}\frac{1}{2^{p}}f_{n}(x-a_{p})\leqslant\frac{\varepsilon}{2}.
        \end{equation*}

        Notons $\alpha)\min_{\substack{0\leqslant p\leqslant N_{0}\\ x\in[a,b]}}\left\lvert x-a_{p}\right\rvert>0$. Pour tout $x\in[a,b]$, pour tout $p\in\left\llbracket0,N_{0}\right\rrbracket$, $\left\lvert x-a_{p}\right\rvert\geqslant\alpha$ et il existe $N_{1}\in\N$ tel que pour tout $n\geqslant N_{1}$, $\frac{1}{n}\leqslant\alpha$. Alors pour tout $x\in[a,b]$, pour tout $p\in\left\llbracket 0,N_{0}\right\rrbracket$, $f_{n}(x-a_{p})\leqslant f_{n}(\alpha)$ et 
        \begin{equation*}
            0\leqslant\sum_{p=0}^{N_{0}}\frac{1}{2^{p}}f_{n}(x-a_{p})\leqslant\sum_{p=0}^{N_{0}}\frac{1}{2^{p}}f_{n}(\alpha)\xrightarrow[n\to+\infty]{}0.
        \end{equation*}
        Ainsi, il existe $N_{2}\in\N$ tel que pour tout $n\geqslant N_{2}$, pour tout $x\in[a,b]$, $\sum_{p=0}^{N_{0}}\frac{1}{2^{p}}f_{n}(x-a_{p})\leqslant\frac{\varepsilon}{2}$. Donc il existe $N\in\N$ tel que pour tout $n\geqslant N$, pour tout $x\in[a,b]$, $0\leqslant g_{n}(x)\leqslant \varepsilon$. D'où le résultat.
    \end{enumerate}
\end{proof}

\begin{proof}
    $f_{n}$ est définie car $\frac{1}{x^{2}+n^{2}}\leqslant\frac{1}{n^{2}}$. Soit $a>0$. Sur $[-a,a]$, $\left\lvert f_{n}(x)\right\rvert\leqslant\frac{\left\lvert a\right\rvert}{n^{2}}$, terme général d'une série à termes positifs convergente. Il y a donc convergence normale sur $[-a,a]$, et $f_{n}$ est continue pour tout $n\in\N$ donc $f$ l'est aussi. Soit $g_{n}(x)=\frac{1}{x^{2}+n^{2}}$. On a $g_{n}'(x)=-\frac{2x}{\left(x^{2}+n^{2}\right)^{2}}$ et pour tout $x\in[-a,a]$, $\left\lvert g_{n}'(x)\right\rvert\leqslant\frac{2\left\lvert a\right\rvert}{n^{4}}$. Il y a à nouveau convergence normale sur $[-a,a]$ pour tout $a\in\R$ et donc $g(x)=\sum_{n=1}^{+\infty}g_{n}(x)$ est $\mathcal{C}^{1}$ et donc $f$ aussi.

    Sur $[-1,1]$, on peut intervertir les limites:
    \begin{equation*}
        \boxed{\lim\limits_{x\to0}f(x)=\sum_{n=1}^{+\infty}\lim\limits_{x\to0}f_{n}(x)=0.}
    \end{equation*}

    Fixons $x>0$, on pose $\psi_{x}(t)=\frac{x}{x^{2}+t^{2}}$. $\psi_{x}$ est positive décroissante. Ainsi, pour tout $n\geqslant1$,
    \begin{equation*}
        \int_{n}^{n+1}\psi_{x}(t)\mathrm{d}t\leqslant\psi_{x}(n)\leqslant\int_{n-1}^{n}\psi_{x}(t)\mathrm{d}t.
    \end{equation*}
    On a 
    \begin{equation*}
        \int_{A}^{X}\frac{x\mathrm{d}t}{x^{2}+t^{2}}=\int_{A}^{X}\frac{\frac{\mathrm{d}t}{x}}{1+\left(\frac{t}{x}\right)^{2}}\xrightarrow[X\to+\infty]{}\frac{\pi}{2}-\arctan\left(\frac{A}{x}\right).
    \end{equation*}

    Ainsi, en sommant pour $n\in\N^{*}$, on a 
    \begin{equation*}
        \int_{1}^{+\infty}\psi_{x}(t)\mathrm{d}t\leqslant\sum_{n=1}^{+\infty}\psi_{x}(n)\leqslant\int_{0}^{+\infty}\psi_{x}(t)\mathrm{d}t.
    \end{equation*}
    Donc $\lim\limits_{x\to+\infty}f(x)=\frac{\pi}{2}$.

    En 0, on a $f(x)=xg(x)$ avec convergence normale sur $\R$ pour $g$, $g$ continue sur $\R$ et $g(0)=\frac{\pi^{2}}{6}$. Ainsi,
    \begin{equation*}
        \boxed{
            f(x)\underset{x\to0}{\sim}x\frac{\pi^{2}}{6}.
        }
    \end{equation*}
\end{proof}

\begin{proof}
    Les $f_{n}$ sont $M$-Lipschitziennes. Soient $x,y\in[a,b]$. On a $\left\lvert f_{n}(x)-f_{n}(y)\right\rvert\leqslant M\left\lvert x-y\right\rvert$ donc par passage à la limite, $f$ est $M$-Lipschitzienne. 
    
    Soit $\varepsilon>0$, on considère la subdivision $(a_{1},\dots,a_{N})$ de $[a,b]$ de pas $\delta$. Soit $x\in[a,b]$ et $K\in\left\llbracket0,N-1\right\rrbracket$ tel que $x\in[a_{K},a_{K+1}]$. Pour tout $n\in\N$, on a 
    \begin{align*}
        \left\lvert f_{n}(x)-f(x)\right\rvert
        &\leqslant\left\lvert f_{n}(x)-f_{n}(a_{K})\right\rvert+\left\lvert f_{n}(a_{K})-f(a_{K})\right\rvert+\left\lvert f(a_{K})-f(x)\right\rvert,\\
        &\leqslant M\delta+\left\lvert f_{n}(a_{k})-f(a_{k})\right\rvert+M\delta.
    \end{align*}
    On s'impose $\delta\leqslant\frac{\varepsilon}{3M}$. Il existe $N_{1}\in\N$ tel que pour tout $n\geqslant N_{1}$, on a pour tout $k\in\left\llbracket0,N\right\rrbracket$, $\left\lvert f_{n}(a_{k})-f(a_{k})\right\rvert\leqslant\frac{\varepsilon}{3}$. Alors pour tout $x\in[a,b]$, pour tout $n\geqslant N_{1}$, $\left\lvert f_{n}(x)-f(x)\right\rvert\leqslant\varepsilon$.
\end{proof}

\begin{remark}
    L'existence de $M$ est nécessaire, cf $f_{n}\colon[0,1]\to\R$ avec $f_n(x)=x^{n}$.
\end{remark}

\begin{remark}
    $f$ n'est pas nécessairement dérivable, cf $f_{n}\colon[-1,1]\to\R$ avec $f_n(x)=\sqrt{x^{2}+\frac{1}{n}}$ converge uniformément vers $x\mapsto\left\lvert x\right\rvert$ et 
    \begin{equation*}
        \left\lvert f_n'(x)\right\rvert=\left\lvert\frac{x}{\sqrt{x^{2}+\frac{1}{n}}}\right\rvert\leqslant1.
    \end{equation*}
\end{remark}

\begin{proof}
    Si $x=2$, on a 
    \begin{equation*}
        f_{n}(2)=\frac{1}{n}\sum_{p=1}^{n}\frac{1}{\sqrt{1+\left(\frac{p}{n}\right)^{2}}}\xrightarrow[n\to+\infty]{}\int_{0}^{1}\frac{\mathrm{d}t}{\sqrt{1+t^{2}}}=\left[\ln\left(t+\sqrt{1+t^{2}}\right)\right]_{0}^{1}=\ln(2).
    \end{equation*}
    Si $x<2$, on a pour tout $n\geqslant1$, pour tout $p\in\left\llbracket1,n\right\rrbracket$,
    \begin{equation*}
        \frac{1}{\sqrt{n^{2}+n^{x}}}\leqslant\frac{1}{\sqrt{n^{2}+p^{x}}}\leqslant\frac{1}{n}.
    \end{equation*}
    On somme pour obtenir
    \begin{equation*}
        \frac{n}{\sqrt{n^{2}+n^{x}}}\leqslant f_{n}(x)\leqslant1
    \end{equation*}
    et donc 
    \begin{equation*}
        \boxed{\lim\limits_{n\to+\infty}f_{n}(x)=1.}
    \end{equation*}

    De plus, soit $a<2$, pour tout $x\in]-\infty,a]$, on a 
    \begin{equation*}
        0\leqslant1-f_{n}(x)\leqslant1-\frac{n}{\sqrt{n^{2}+n^{x}}}\leqslant1-\frac{n}{\sqrt{n+n^{a}}}\xrightarrow[n\to+\infty]{}0.
    \end{equation*}
    Donc $(f_{n})_{n\geqslant1}$ converge uniformément vers 1 sur $]-\infty,a]$.

    Si $x>2$, soit $\alpha\in[0,1]$, on a 
    \begin{equation*}
        f_{n}(x)=\sum_{p=1}^{\left\lfloor n^{\alpha}\right\rfloor}\frac{1}{\sqrt{n^{2}+p^{x}}}+\sum_{p=\left\lfloor n^{\alpha}\right\rfloor+1}^{n}\frac{1}{\sqrt{n^{2}+p^{x}}}.
    \end{equation*}
    On a
    \begin{equation*}
        \sum_{p=1}^{\left\lfloor n^{\alpha}\right\rfloor}\frac{1}{\sqrt{n^{2}+p^{x}}}\leqslant\frac{n^{\alpha}}{\sqrt{1+n^{2}}}\underset{n\to+\infty}{n^{\alpha-1}},
    \end{equation*}
    et
    \begin{equation*}
        \sum_{p=\left\lfloor n^{\alpha}\right\rfloor+1}^{n}\frac{1}{\sqrt{n^{2}+p^{x}}}\leqslant\frac{n}{\sqrt{n^{2}+n^{x\alpha}}}=\frac{1}{\sqrt{1+n^{x\alpha-2}}}.
    \end{equation*}

    On choisit $\alpha$ tel que $\alpha<1$ et $x\alpha-2>0$ (possible car $x>2$). Si $a>2$, pour $\alpha=\left(1+\frac{2}{a}\right)\times\frac{1}{2}$, si $x\geqslant a$, on a $\frac{2}{x}\leqslant\frac{2}{a}<\alpha<1$ donc 
    \begin{equation*}
        0\leqslant f_{n}(x)\leqslant\frac{n^{\alpha}}{\sqrt{n^{2}+1}}+\frac{1}{\sqrt{1+n^{\alpha x-2}}}\xrightarrow[n\to+\infty]{}0.
    \end{equation*}
    Il y a donc convergence uniforme vers 0 sur $[a,+\infty[$.
\end{proof}

\begin{proof}
    \phantom{}
    \begin{enumerate}
        \item 
    Pour tout $(n,k)\in\N\times\left\llbracket0,n\right\rrbracket$,
    \begin{equation*}
        \frac{1}{n^{k}}\binom{n}{k}=\frac{1}{k!}\frac{n(n-1)\dots(n-k+1)}{n\times n\times\dots\times n}\leqslant\frac{1}{k!}.
    \end{equation*}

    Ainsi,
    \begin{align*}
        \left\lVert\sum_{k=0}^{n}\frac{a^{k}}{k!}-f_{n}(a)\right\rVert
        &=\left\lVert\sum_{k=0}^{n}\frac{a^{k}}{k!}-\sum_{k=0}^{n}\binom{n}{k}\frac{a^{k}}{n^{k}}\right\rVert,\\
        &\leqslant\sum_{k=0}^{n}\left(\frac{1}{k!}-\binom{n}{k}\frac{1}{n^{k}}\right)\left\lVert a\right\rVert^{k},\\
        &=\sum_{k=0}^{n}\frac{\left\lVert a\right\rVert^{k}}{k!}-\left(1+\frac{\left\lVert a\right\rVert}{n}\right)^{n}\xrightarrow[n\to+\infty]{}0.
    \end{align*}

    On a bien $\lim\limits_{n\to+\infty}f_{n}(a)=\exp(a)$.

    Soit $R\geqslant0$, pour tout $a\in\overline{B(0,R)}$,
    \begin{align*}
        \left\lVert\exp(a)-f_{n}(a)\right\rVert
        &\leqslant\left\lVert\sum_{k=0}^{n}\frac{a^{k}}{k!}-\sum_{k=0}^{n}\binom{n}{k}\frac{a^{k}}{n^{k}}\right\rVert+\left\lVert\sum_{k=n+1}^{+\infty}\frac{a^{k}}{k!}\right\rVert\\
        &\leqslant\sum_{k=0}^{n}\left(\frac{1}{k!}-\binom{n}{k}\frac{1}{n^{k}}\right)R^{n}\xrightarrow[n\to+\infty]{}0.
    \end{align*}
    $(f_{n})_{n\in\N}$ converge uniformément vers $\exp(a)$ sur les compacts.

    \item D'après ce qui précède, $(P_{n})_{n\in\N}$ converge simplement vers $z\mapsto\frac{\e^{\i z}-\e^{-\i z}}{2\i}=\sin(z)$. Et on a convergence sur les compacts.
    
    \item On peut déjà dire que $\deg(P_{n})\leqslant2n+1$. Le coefficient en $X^{2n+1}$ de $P_{n}$ est 
    \begin{equation*}
        \alpha=\frac{\left(\frac{\i}{2n+1}\right)^{2n+1}-\left(\frac{-\i}{2n+1}\right)^{2n+1}}{2\i}=\frac{1}{\left(2n+1\right)^{2n+1}}\times\frac{(-1)^{n}}{2\i}[\i-(-\i)]\neq0
    \end{equation*}
    et donc $\deg(P_{n})=2n+1$.

    Le coefficient en $X$ est $\frac{(2+1)\left(\frac{\i}{2n+1}-\left(\frac{-\i}{2n+1}\right)\right)}{2\i}=1$. 

    Soit $z\in\C$, on a 
    \begin{align*}
        P_{n}(z)=0
        &\Longleftrightarrow \left(1+\frac{\i z}{2n+1}\right)^{2n+1}=\left(1-\frac{\i z}{2n+1}\right)^{2n+1},\\
        &\Longleftrightarrow \exists k\in\left\llbracket0,2n\right\rrbracket,~ 1-\frac{\i z}{2n+1}=\left(1+\frac{\i z}{2n+1}\right)\exp\left(\frac{2\i k\pi}{2n+1}\right),\\
        &\Longleftrightarrow \exists k\in\left\llbracket0,2n\right\rrbracket,~1-\exp\left(\frac{2\i k\pi}{2n+1}\right)=\frac{\i z}{2n+1}\left(1+\exp\left(\frac{2\i k\pi}{2n+1}\right)\right),\\
        &\Longleftrightarrow \exists k\in\left\llbracket0,2n\right\rrbracket,~z=(2n+1)\times (-\i)\times \frac{(-2\i)\sin\left(\frac{k\pi}{2n+1}\right)}{2\cos\left(\frac{k\pi}{2n+1}\right)},\\
        &\Longleftrightarrow \exists k\in\left\llbracket0,2n\right\rrbracket,~=-(2n+1)\tan\left(\frac{k\pi}{2n+1}\right).
    \end{align*}

    On a 
    \begin{align*}
        P_{n}
        &=aX\times\prod_{k=1}^{2n}\left(X+\left(2n+1\right)\tan\left(\frac{k\pi}{2n+1}\right)\right),\\
        &=aX\prod_{k=1}^{n}\left(X+\left(2n+1\right)\tan\left(\frac{k\pi}{2n+1}\right)\right)\prod_{k=n+1}^{2n}\left(X+\left(2n+1\right)\tan\left(\frac{k\pi}{2n+1}\right)\right),\\
        &=aX\prod_{k=1}^{n}\left(X^{2}-\left(2n+1\right)^{2}\tan^{2}\left(\frac{k\pi}{2n+1}\right)\right),\\
        &=a'X\prod_{k=1}^{n}\left(1-\frac{X^{2}}{\left(2n+1\right)^{2}\tan^{2}\left(\frac{k\pi}{2n+1}\right)}\right).
    \end{align*}
    Comme le coefficient de $X$ vaut 1, on a $a'=1$, d'où le résultat.

    \item Soit $f_{n}\colon\N\to\C$ telle que $f(p)=a_{n,p}$. D'après (i), $\sum f_{n}$ converge normalement sur $\N$, et d'après (ii), on peut intervertir et $\lim\limits_{p\to+\infty}\sum_{n=0}^{+\infty}a_{n,p}=\sum_{n=0}^{+\infty}\beta_{n}$.
    
    \item $\tan$ est impaire, et $\tan''=2\tan(1+\tan^{2})>0$ sur $\left]0,\frac{\pi}{2}\right[$, donc $\tan$ est convexe et $\tan(t)>t$ sur $\left]0,\frac{\pi}{2}\right[$, et c'est bon par imparité.
    
    Pour tout $k\in\left\llbracket1,n\right\rrbracket$, $0\leqslant\frac{x^{2}}{(2n+1)^{2}\tan^{2}\left(\frac{k\pi}{2n+1}\right)}\leqslant\frac{x^{2}}{k^{2}\pi^{2}}$. Il existe $k_{0}\in\N$ tel que pour tout $k\geqslant k_{0}$, $\frac{x^{2}}{k^{2}\pi^{2}}\leqslant\frac{1}{2}$, alors pour tout $n\geqslant k_{0}$, pour tout $k\in\left\llbracket k_{0},n\right\rrbracket$, $1-\frac{x^{2}}{(2n+1)^{2}\tan^{2}\left(\frac{k\pi}{2n+1}\right)}\geqslant\frac{1}{2}>0$. Alors 
    \begin{align*}
        0&
        \leqslant -\ln\left(\prod_{k=k_{0}}^{n}\left(1-\frac{x^{2}}{\left(2n+1\right)^{2}\tan^{2}\left(\frac{k\pi}{3n+1}\right)}\right)\right),\\
        &=\sum_{k=k_{0}}^{n}-\ln\left(1-\frac{x^{2}}{(2n+1)^{2}\tan^{2}\left(\frac{k\pi}{2n+1}\right)}\right).
    \end{align*}
    On a 
    \begin{equation*}
        0\leqslant-\ln\left(1-\frac{x^{2}}{(2n+1)^{2}\tan^{2}\left(\frac{k\pi}{2n+1}\right)}\right)\leqslant-\ln\left(1-\frac{x^{2}}{k^{2}\pi^{2}}\right)=\underset{k\to+\infty}{O}\left(\frac{1}{k^{2}}\right),
    \end{equation*}
    terme général d'une série à termes positifs convergente.

    Si $g_{n}(x)=-\ln\left(\prod_{k=k_{0}}^{n}\left(1-\frac{x^{2}}{\left(2n+1\right)^{2}\tan^{2}\left(\frac{k\pi}{3n+1}\right)}\right)\right)$, alors $g_{n}(x)=\sum_{k=k_{0}}^{+\infty}a_{n,k}$ où l'on définit pour tout $k\geqslant k_{0},n\geqslant k_{0}$,
    \begin{equation*}
        a_{n,k}=-\ln\left(1-\frac{x^{2}}{(2n+1)^{2}\tan^{2}\left(\frac{k\pi}{2n+1}\right)}\right)
    \end{equation*}
    si $k\leqslant n$, et 0 sinon. On pose aussi $\alpha_{k}=-\ln\left(1-\frac{x^{2}}{k^{2}\pi^{2}}\right)$. On a bien $\left\lvert a_{n,k}\right\rvert\leqslant\alpha_{k}$ terme général d'une série à termes positifs convergente.

    Pour $k\geqslant k_{0}$ fixé, pour $n\geqslant k$, on a 
    \begin{equation*}
        a_{n,k}\xrightarrow[n\to+\infty]{}\alpha_{k}.
    \end{equation*}
    On peut donc appliquer ce qui précède, et on a 
    \begin{equation*}
        \lim\limits_{n\to+\infty}g_{n}(x)=\sum_{k=k_{0}}^{+\infty}\alpha_{k},
    \end{equation*}
    d'où
    \begin{equation*}
        \lim\limits_{n\to+\infty}\prod_{k=k_{0}}\left(1-\frac{x^{2}}{(2n+1)^{2}\tan^{2}\left(\frac{k\pi}{2n+1}\right)}\right)=\prod_{k=k_{0}}^{+\infty}\left(1-\frac{x^{2}}{k^{2}\pi^{2}}\right).
    \end{equation*}

    Soit $R_{n}(x)=x\prod_{k=1}^{k_{0}}\left(1-\frac{x^{2}}{(2n+1)^{2}\tan^{2}\left(\frac{k\pi}{2n+1}\right)}\right)\xrightarrow[n\to+\infty]{}x\prod_{k=1}^{k_{0}}\left(1-\frac{x^{2}}{k^{2}\pi^{2}}\right)$. Finalement, on a bien 
    \begin{equation*}
        \boxed{
            \sin(x)=x\prod_{k=1}^{+\infty}\left(1-\frac{x^{2}}{k^{2}\pi^{2}}\right).
        }
    \end{equation*}
    \end{enumerate}
\end{proof}

\begin{remark}
    En identifiant le coefficient en $x^{3}$, on obtient 
    \begin{equation*}
        -\frac{1}{6}=-\sum_{k=1}^{+\infty}\frac{1}{k^{2}\pi^{2}},
    \end{equation*}
    d'où 
    \begin{equation*}
        \zeta(2)=\frac{\pi^{2}}{6}.
    \end{equation*}

    De même, en identifiant le coefficient en $x^{5}$, on obtient
    \begin{equation*}
        \frac{1}{120}=\sum_{\substack{(k_{1},k_{2})\in(\N^{*})^{2}\\k_{1}\neq k_{2}}}\frac{1}{k_{1}^{2}k_{2}^{2}\pi^{4}}=\sum_{(k_1,k_2)\in(\N^{*})^{2}}\frac{1}{k_{1}^{2}k_{2}^{2}\pi^{4}}-\sum_{k=1}^{+\infty}\frac{1}{k^{4}\pi^{4}}=\zeta(2)^{2}-\sum_{k=1}^{+\infty}\frac{1}{k^{4}\pi^{4}}.
    \end{equation*}
    On trouve donc 
    \begin{equation*}
        \zeta(4)=\frac{\pi^{4}}{90}.
    \end{equation*}

    De la même façon, on montre de manière générale que
    \begin{equation*}
        \zeta(2p)=a_{p}\pi^{2p},
    \end{equation*}
    avec $a_p\in\Q$.
\end{remark}

\begin{proof}
    \phantom{}
    \begin{enumerate}
        \item Soit $\alpha\in[a,b]$. $f$ est strictement croissante sur $\left[0,\frac{1}{2}\right]$. On a $f\left(\left[0,1\right]\right)\subset\left[0,\frac{1}{2}\right]$. Pour tout $x\in\left[0,\frac{1}{2}\right]$, $f(x)\geqslant x$. $(f_{n}(x))_{n\geqslant1}$ est strictement croissante, majorée par $\frac{1}{2}$, donc converge vers $\frac{1}{2}$ seul point fixe de $f$ (continue). Ainsi $(f_{n})$ converge simplement vers $\frac{1}{2}$ sur $[a,b]$.
        
        Pour tout $n\geqslant1$,
        \begin{equation*}
            \left\lvert f_{n}(x)-\frac{1}{2}\right\rvert=\frac{1}{2}-f_{n}(x)\leqslant\max\left(\frac{1}{2}-f_{n}(a),\frac{1}{2}-f_{n}(b)\right)\xrightarrow[n\to+\infty]{}0.
        \end{equation*}
        Donc $(f_{n})_{n\geqslant0}$ converge uniformément sur $[a,b]$.

        On a $f_{n}(0)=f_{n}(1)=0\neq\frac{1}{2}$, on n'a donc pas la continuité de la limite simple. Donc il ne peut y avoir convergence uniforme sur $[0,1]$ (même sur $]0,1[$).

        \item Soit $P=\sum_{k=0}^{n}a_{k}X^{k}\in\R[X]$. $Q_{2}$ est dense dans $\R$, donc pour tout $k\in\left\llbracket0,n\right\rrbracket$, il existe $(\alpha_{k,m})_{m\in\N}\in\Q_{2}^{\N}$ telle que $\lim\limits_{n}\alpha_{k,m}=a_{k}$. Soit $Q_{n}=\sum_{k=0}^{n}\alpha_{k,m}X^{k}\in\Q_{2}[X]$. Pour tout $x\in[a,b]$ on a 
        \begin{equation*}
            \left\lvert P(x)-Q(x)\right\rvert\leqslant\sum_{k=0}^{n}\left\lvert a_{k}-\alpha_{k,m}\right\rvert\left\lvert x\right\rvert^{k}\leqslant\sum_{k=0}^{n}\left\lvert a_{k}-\alpha_{k,m}\right\rvert\xrightarrow[m\to+\infty]{}0    
        \end{equation*}
        donc il existe $M\in\N$, si $Q=Q_{M}$, alors $\left\lVert P-Q\right\rVert_{\infty}\leqslant\varepsilon$.

        \item Soit $f\in\mathcal{C}^{0}\left([a,b],\R\right)$, soit $\varepsilon>0$. D'après le théorème de Weierstrass, il existe $P\in\R[X]$, tel que $\left\lVert f-P\right\rVert_{\infty,[a,b]}\leqslant\frac{\varepsilon}{3}$. Si $Q=\sum_{k=0}^{n}\frac{p_{k}}{2^{n_{k}}}X^{k}$, soit pour $m\in\N$, $Q_{m}=\sum_{k=0}^{n}p_{k}\left(f_{m}\right)^{n_{k}}X^{k}$ converge uniformément vers $Q$ sur $[a,b]$ ($n$ est fixé), et $Q_{m}\in\Z[X]$ car pour tout $n\in\N$, $f_{n}\in\Z[X]$.
        Il existe $n_{0}\in\N$ tel que $\left\lVert Q_{n_{0}}-Q\right\rVert_{\infty,[a,b]}\leqslant\frac{\varepsilon}{3}$. Si $A=Q_{n_{0}}\in\Z[X]$, on a bien $\left\lVert f-A\right\rVert_{\infty,[a,b]}\leqslant\varepsilon$.

        Sur $[0,1]$, on n'a pas de suite de polynômes dans $\Z[X]$ qui converge uniformément sur $[0,1]$ vers $f=\frac{1}{2}$ car pour tout $P\in\Z[X]$, $P(0)\in\Z$.
    \end{enumerate}
\end{proof}

\begin{proof}
    \phantom{}
    \begin{enumerate}
        \item Par croissance des taux d'accroissements (en un point fixé):
        \begin{equation*}
            \frac{u_n(y)-u_n(x)}{y-x}\leqslant\frac{u_n(y)-u_n(b)}{y-b}\leqslant\frac{u_n(\beta)-u_n(b)}{\beta-b},
        \end{equation*}
        et de même
        \begin{equation*}
            \frac{u_n(y)-u_n(x)}{y-x}\geqslant\frac{u_n(\alpha)-u_n(a)}{\alpha-a}.
        \end{equation*}
        Finalement, on a 
        \begin{equation*}
            \left\lvert\frac{u_n(x)-u_(y)}{x-y}\right\rvert\leqslant\max\left(\left\lvert\frac{u_n(\alpha)-u_n(a)}{\alpha-a}\right\rvert,\left\lvert\frac{u_n(\beta)-u_n(b)}{\beta-b}\right\rvert\right),
        \end{equation*}
        qui sont des suites bornées car convergent. D'où l'existence de $A$.

        \item Par passage à la limite (simple), $u$ est $A$-Lipschitzienne sur $[a,b]$. Soit $\varepsilon>0$. Soit $(a_{k})_{1\leqslant k\leqslant N}$ une subdivision de pas $d$ de $[a,b]$. Pour tout $x\in[a,b]$, il existe $k\in\left\llbracket1,N-1\right\rrbracket$ tel que $x\in[a_{k},a_{k+1}]$.
        Alors pour tout $n\in\N$, on a 
        \begin{align*}
            \left\lvert u_n(x)-u(x)\right\rvert
            &\leqslant\left\lvert u_n(x)-u_n(a_k)\right\rvert+\left\lvert u_n(a_k)-u(a_k)\right\rvert+\left\lvert u(a_k)-u(x)\right\rvert,\\
            &\leqslant2Ad+\left\lvert u_n(a_k)-u(a_k)\right\rvert.
        \end{align*}
        On choisit $d$ tel que $2Ad\leqslant\frac{\varepsilon}{2}$. Par convergence simple, il existe $N_{0}\in\N$ tel que pour tout $n\geqslant N_{0}$, pour tout $k\in\left\llbracket1,N\right\rrbracket$, $\left\lvert u_n(a_k)-u(a_k)\right\rvert\leqslant\frac{\varepsilon}{2}$. Ainsi, pour tout $n\geqslant N_{0}$, pour tout $x\in[a,b]$, $\left\lvert u_n(x)-u(x)\right\rvert\leqslant\varepsilon$. Donc $(u_n)$ converge uniformément vers $u$ sur $[a,b]$.
    \end{enumerate}
\end{proof}

\begin{remark}
    C'est faux si $I=[a,b]$, cf $f_n\colon[0,1]\to\R$ donnée par $f_n(x)=x^{n}$.
\end{remark}

\begin{proof}
    Soit $f\in E$ et $(f_{n})_{n\in\N}$ qui converge uniformément vers $f$. Si $\varphi$ est une fonction polynômiale, $\varphi=\sum_{k=0}^{N}\alpha_{k}X^{k}$. Pour tout $k\in\left\llbracket0,N\right\rrbracket$, $(f_{n}^{k})$ converge uniformément vers $f^{k}$ sur $[a,b]$. Par combinaison linéaire, $(\varphi\circ f_{n})_{n\in\N}$ converge uniformément vers $\varphi\circ f$ sur $[a,b]$. $\left(\left\lVert f_n\right\rVert_{\infty}\right)$ est bornée (car converge), donc il existe $A\geqslant0$ telle que pour tout $n\in\N$, $\left\lVert f_n\right\rVert_{\infty}\leqslant A$ et $\left\lVert f\right\rVert_{\infty}\leqslant A$. 

    Soit $\varepsilon>0$, il existe $P\in\R[X]$ telle que $\left\lVert\varphi-P\right\rVert_{\infty,[-A,A]}\leqslant\frac{\varepsilon}{3}$ d'après le théorème de Weierstrass. Ainsi, pour tout $x\in[a,b]$,
    \begin{align*}
        \left\lvert\left(\varphi\circ f_n\right)(x)-\left(\varphi\circ f\right)(x)\right\rvert 
        &\leqslant\left\lvert\left(\varphi\circ f_n\right)(x)-\left(P\circ f_n\right)(x)\right\rvert\\
        &\qquad+\left\lvert P\circ f_n(x)-P\circ f(x)\right\rvert+\left\lvert P\circ f(x)-\varphi\circ f(x)\right\rvert,\\
        &\leqslant2\frac{\varepsilon}{3}+\left\lVert P\circ f_n-P\circ f\right\rVert_{\infty,[a,b]}
    \end{align*}
    et le dernier terme tend vers 0 donc est plus petit que $\frac{\varepsilon}{3}$ pour $n$ suffisamment grand. D'où le résultat.
\end{proof}

\begin{remark}
    Pour la deuxième partie du raisonnement, on peut aussi invoquer la continuité uniforme de $\varphi$ sur $[-A,A]$.
\end{remark}

\begin{proof}
    \phantom{}
    \begin{enumerate}
        \item Pour $t\geqslant0$, on a $0\leqslant f_{n}(t)\leqslant\frac{1}{1+n^{2}}$ donc on a convergence normale sur $\R^{+}$. Pour $t<0$, $\lim\limits_{n\to+\infty}(t)=+\infty$ donc la série diverge grossièrement. Ainsi, $E=\R_{+}$.
        \item Pour tout $n\in\N$, $f_{n}$ est continue et on a convergence normale donc $f$ est continue sur $E$. Pour tout $n\geqslant1$, $\lim\limits_{t\to+\infty}(t)=0$ et $\lim\limits_{t\to+\infty}f_{0}(t)=1$. On peut intervertir par convergence normale, donc $\lim\limits_{t\to+\infty}f_{n}(t)=1$.
        \item Pour tout $n\in\N$, $f_{n}$ est $\mathcal{C}^{\infty}$ sur $E$. Soit $k\in\N$. Pour tout $t\geqslant0$ et $n\in\N$, on a 
        \begin{equation*}
            f_{n}^{(k)}(t)=\frac{(-n)^{k}\e^{-nt}}{1+n^{2}}.
        \end{equation*}
        Soit $\alpha>0$. Pour $t\geqslant\alpha$, on a 
        \begin{equation*}
            \left\lvert f_{n}^{(k)}(t)\right\rvert\leqslant\frac{\e^{-n\alpha}}{1+n^{2}}=\underset{n\to+\infty}{O}\left(\frac{1}{n^{2}}\right).
        \end{equation*}
        Ainsi, $\sum_{n\geqslant0}f_{n}^{(k)}$ converge normalement sur $[\alpha,+\infty[$ pour tout $\alpha>0$, donc $f$ est $\mathcal{C}^{\infty}$ sur $\R_{+}^{*}$.
        On a pour tout $t>0$,
        \begin{equation*}
            \boxed{f''(t)+f(t)=\sum_{n=0}^{+\infty}\e^{-nt}=\frac{1}{1-\e^{-t}}.}
        \end{equation*}
    \end{enumerate}
\end{proof}

\begin{proof}
    \phantom{}
    \begin{enumerate}
        \item On a $u_n(0)=0$. Soit $x>0$, on a $\left\lvert u_n(x)\right\rvert=\underset{n\to+\infty}{O}\left(\frac{1}{n^{2}}\right)$ donc on a bien convergence simple sur $[0,1]$.
        \item On a 
        \begin{equation*}
            u_n(x)=\frac{(1-nx)\e^{-nx}}{n^{a}}.
        \end{equation*}
        Ainsi, $u_n$ est croissante de $0$ à $\frac{1}{n}$ et décroît de $\frac{1}{n}$ à 1, et on a $u_n(0)=0$, $u_n(1)=\frac{\e^{-n}}{a}$ et $u_n\left(\frac{1}{n}\right)=\frac{1}{\e n^{a+1}}=\left\lVert u_n\right\rVert_{\infty,[0,1]}$. On a donc convergence normale si et seulement si $a>0$.
        \item Pour $a=1$ (respectivement $a=2$), $S$ est continue par convergence normale car $u_n$ est $\mathcal{C}^{\infty}$ pour tout $n\geqslant1$. Soit $x>0$, si $g_n(x)=\frac{\e^{-nx}}{n}$ (respectivement $h_n(x)=\frac{\e^{-nx}}{n^{2}}$) et $g(x)=\frac{S(x)}{x}$ (respectivement $h(x)=\frac{S(x)}{x}$), soit $\alpha\in]0,1]$ et $x\in[\alpha,1]$, on a 
        \begin{equation*}
            \left\lvert g_n'(x)\right\rvert=\left\lvert\e^{-nx}\right\rvert\leqslant\e^{-n\alpha}
        \end{equation*}
        (respectivement
        \begin{equation*}
            \left\lvert h_n'(x)\right\rvert\leqslant\left\lvert\frac{\e^{-n\alpha}}{n}\right\rvert
        \end{equation*}
        et $\left\lvert h_n''(x)\right\rvert\leqslant\e^{-n\alpha}$). Donc $\sum g_n'$ (respectivement $\sum h_n'$ et $\sum h_n''$) converge normalement sur $[\alpha,1]$ pour tout $\alpha\in]0,1]$ donc $g$ est $\mathcal{C}^{1}$ (respectivement $h$ est $\mathcal{C}^{2}$) sur $]0,1]$.

        Pour tout $x\in]0,1]$, on a 
        \begin{equation*}
            g'(x)=\sum_{n=1}^{+\infty}-\e^{-nx}=\frac{-\e^{-x}}{1-\e^{-x}},
        \end{equation*}
        et donc (par changement de variable dans l'intégrale)
        \begin{equation*}
            g(x)=g(1)+\ln\left(1-\e^{-1}\right)-\ln\left(1-\e^{-x}\right).
        \end{equation*}
        puis $S(x)=xg(x)$. On fait de même pour $a=2$.
    \end{enumerate}
\end{proof}

\begin{proof}
    Si $x>0$, on a $f_{n}(x)=\frac{1}{\sqrt{n}}\times\frac{1}{+nx^{\frac{3}{2}}}=\underset{n\to+\infty}{O}\left(\frac{1}{n^{\frac{3}{2}}}\right)$. Ainsi, le domaine de $f$ est $]0,+\infty[$.

    Chaque $f_n$ est continue sur $\R_{+}^{*}$. Soit $a>0$, pour $x\geqslant a$, pour $n\geqslant1$, on a 
    \begin{equation*}
        0\leqslant f_{n}(x)\leqslant\frac{1}{\sqrt{n}}\times\frac{1}{1+na^{\frac{3}{2}}},
    \end{equation*}
    et le terme de droite est le terme général d'un série à termes positifs convergente indépendante de $x$. Donc $\sum_{n\geqslant1}f_{n}$ converge normalement ssur $[a,+\infty[$ pour tout $a>0$. Donc $f$ est continue sur $\R_{+}^{*}$.

    On a 
    \begin{equation*}
        0\leqslant f(x)\leqslant\sum_{n=1}^{+\infty}\frac{1}{(nx)^{\frac{3}{2}}}=\frac{1}{x^{\frac{3}{2}}}\times\zeta\left(\frac{3}{2}\right),
    \end{equation*}
    donc $f$ est intégrable sur $[1,+\infty[$.

    Fixons $x>0$, soit \function{g_x}{\R_{+}^{*}}{\R}{t}{\frac{1}{\sqrt{t}}\times\frac{1}{1+tx^{\frac{3}{2}}}}
    $g$ est continue positive et décroissante. Elle est intégrable sur $\R_{+}^{*}$ car est équivalente à $\frac{1}{\sqrt{t}}$ quand $t\to0$, et est un $O\left(\frac{1}{t^{\frac{3}{2}}}\right)$ quand $t\to+\infty$. Pour tout $n\geqslant1$, on a 
    \begin{equation*}
        g_{x}(n+1)\leqslant\int_{n}^{n+1}g_x(t)\mathrm{d}t\leqslant g_{x}(n)
    \end{equation*}
    Ainsi, en sommant, on obtient 
    \begin{equation*}
        f(x)-\frac{1}{1+x^{\frac{3}{2}}}=\sum_{n=1}^{+\infty}\frac{1}{\sqrt{n+1}}\times\frac{1}{1+(n+1)x^{\frac{3}{2}}}\leqslant I(x)=\int_{1}^{+\infty}g_x(t)\mathrm{d}t\leqslant f(x).
    \end{equation*}

    Pour calculer $I(x)$, on fait les changements de variables $u=\sqrt{t}$ puis $v=ux^{\frac{3}{4}}$ pour avoir 
    \begin{equation*}
        I(x)=\frac{2}{x^{\frac{3}{4}}}\int_{x^{\frac{3}{4}}}^{+\infty}\frac{\mathrm{d}v}{1+v^{2}}\underset{x\to0}{\sim}\frac{\pi}{x^{\frac{3}{4}}}.
    \end{equation*}

    Ainsi, $f(x)\underset{x\to0}{\sim}\frac{\pi}{x^{\frac{3}{4}}}$. Donc $f$ est intégrale sur $]0,1]$. Finalement, $f$ est intégrable sur $\R_{+}^{*}$.
\end{proof}

\begin{remark}
    On peut aussi former, pour $n\geqslant1$,
    \begin{equation*}
        u_{n}=\int_{0}^{+\infty}\frac{\mathrm{d}x}{\sqrt{n}(1+nx^{\frac{3}{2}})}=\frac{1}{n^{\frac{7}{6}}}\int_{0}^{+\infty}\frac{\mathrm{d}u}{1+u^{\frac{3}{2}}},
    \end{equation*}
    en faisant le changement de variables $u=n^{\frac{2}{3}}x$. $u_n$ est alors le terme général d'une série à termes positifs convergente, et on peut intervertir les signes $\sum$ et $\int$.
\end{remark}

\begin{proof}
    \phantom{}
    \begin{enumerate}
        \item Si $x<0$, on a 
        \begin{equation*}
            \lim\limits_{n\to+\infty}\frac{x\e^{-nx}}{\ln(n)}=+\infty.
        \end{equation*}
        Si $x=0$, on a $S(0)=0$. Si $x>0$, on a $\frac{x\e^{-nx}}{\ln(n)}=\underset{n\to+\infty}{\left(\frac{1}{n^{2}}\right)}$, donc on a convergence simple sur $\R_{+}$.

        \item On cherche $\sup\limits_{x\geqslant0}\left\lvert f_n(x)\right\rvert$. Pour tout $x\geqslant0$, on a 
        \begin{equation*}
            f_{n}'(x)=\frac{(1-nx)}{\ln(n)}\e^{-nx}.
        \end{equation*}
        Ainsi, le sup est atteint en $x=\frac{1}{n}$. Comme $f_{n}\left(\frac{1}{n}\right)=\frac{1}{n\e\ln(n)}$ est le terme général d'une série à termes positifs divergente (série de Bertrand), on n'a pas convergence normale sur $\R_{+}$.

        Soit $N\geqslant2$, $x\geqslant0$. On a 
        \begin{align*}
            \sum_{n=N}^{+\infty}f_{n}(x)
            &\leqslant\frac{x}{\ln(N)}\sum_{n=N}^{+\infty}\e^{-nx},\\
            &\leqslant\frac{x}{\ln(N)}\frac{\e^{-Nx}}{1-\e^{-x}},\\
            &\leqslant\frac{x\e^{-x}}{\ln(N)(1-\e^{-x})}\times\frac{\e^{x}}{\e^{x}},\\
            &\leqslant\frac{x}{\ln(N)(\e^{x}-1)}.
        \end{align*}

        $x\mapsto\frac{x}{\e^{x}-1}\in\mathcal{C}^{0}\left(\R_{+}^{*}\right)$, tend vers 1 quand $x\to0$ et tend vers $0$ quand $x\to+\infty$. Donc cette fonction est bornée par $M\geqslant\e$ et 
        \begin{equation*}
            \sum_{n=N}^{+\infty}f_{n}(x)\leqslant\frac{M}{\ln(N)}\xrightarrow[N\to+\infty]{}0.
        \end{equation*}
        On a donc convergence uniforme sur $\R_{+}$.

        \item On a $S(x)=x\times\sum_{n=2}^{+\infty}\frac{\e^{-nx}}{\ln(n)}=x\times\sum_{n=2}^{+\infty}g_n(x)$. $g_n$ est $\mathcal{C}^{1}$, et pour $a>0$, $x\geqslant a$ et $n\geqslant 2$, on a $g_n'(x)=-\frac{n\e^{-nx}}{\ln(n)}$ d'où $\left\lvert g_n'(x)\right\rvert\leqslant \frac{n\e^{-nx}}{\ln(n)}\leqslant\frac{n\e^{-na}}{\ln(n)}$ qui est le terme général d'une série à termes positifs convergente car $a>0$. Donc $\sum_{n=2}^{+\infty}g_{n}'$ converge normalement sur $[a,\infty[$ et $\sum_{n=2}^{+\infty}g_n$ convergen simplement sur $\R_{+}$ donc $\sum_{n\geqslant2}g_{n}$ est $\mathcal{C}^{1}$ sur $\R_{+}^{*}$.
        
        On a $\frac{S(x)-S(0)}{x-0}=\sum_{n=2}^{+\infty}g_n(x)=\tau(x)$. $\tau$ est décroissante car les $g_n$ le sont, donc $\tau(x)\xrightarrow[x\to0]{}l\in\overline{\R_{+}}$. Comme $g_{n}\geqslant0$, pour tout $N\geqslant2$ et $x>0$, on a $\sum_{n=2}^{N}g_{n}(x)\leqslant\tau(x)$. Quand $x\to0$, on a donc 
        \begin{equation*}
            \sum_{n=2}^{N}\frac{1}{\ln(n)}\leqslant l,
        \end{equation*}
        et quand $N\to+\infty$, on a $l=+\infty$. Ainsi, $S$ n 'est pas dérivable à droite en 0.

        \item On a $x^{k}S(x)=\sum_{n=2}^{+\infty}\frac{x^{k+1}\e^{-nx}}{\ln(n)}=\sum_{n=2}^{+\infty}k_n(x)$. On a 
        \begin{equation*}
            k_n'(x)=\frac{x^{k}\e^{-nx}}{\ln(n)}\left(k+1-nx\right),
        \end{equation*}
        donc le sup est atteinte en $x=\frac{k+1}{n}$. Pour tout $x\geqslant K+1$, on a $\left\lvert k_n(x)\right\rvert\leqslant\left\lvert k_n(K+1)\right\rvert\leqslant\frac{(k+1)^{k+1}\e^{-(k+1)n}}{\ln(n)}=\underset{n\to+\infty}{O}\left(\frac{1}{n^{2}}\right)$. Donc $\sum_{n\geqslant2}k_n$ converge normalement sur $[k+1,+\infty[$ et on peut intervertir les limites. Ainsi,
        \begin{equation*}
            \boxed{
                \lim\limits_{x\to+\infty}x^{k}S(x)=0.
            }
        \end{equation*}
    \end{enumerate}
\end{proof}

\begin{proof}
    \phantom{}
    \begin{enumerate}
        \item $Q_{k}$ est $\mathcal{C}^{\infty}$, $2\pi$-périodique, paire, $0\leqslant Q_{k}(t)\leqslant c_{k}$, $Q_{k}(-\pi)=Q_{k}(\pi)=0$, $Q_{k}(0)=c_{k}$. $Q_{k}$ est décroissante sur $[0,\pi]$. Pour tout $t\in[\delta,\pi]$, $0\leqslant Q_{k}(t)\leqslant c_{k}\left(\frac{1+\cos(\delta)}{2}\right)^{k}$.
        On a $I_{k}=\int_{-\pi}^{\pi}\left(\frac{1+\cos(\delta)}{2}\right)^{k}\mathrm{d}t=4\int_{0}^{\frac{\pi}{2}}\sin^{2k}(u)\mathrm{d}u\underset{k\to+\infty}{\sim}2\sqrt{\frac{\pi}{k}}$ (via les intégrales de Wallis). Ainsi, $c_{k}$ est équivalent à $\sqrt{k\pi}$ quand $k\to+\infty$ et $\lim\limits_{k\to+\infty}c_{k}\left(\frac{1+\cos(\delta)}{2}\right)^{k}=0$. Ainsi, on a bien 
        \begin{equation*}
            \boxed{
                \lim\limits_{k\to+\infty}\sup\limits_{\delta\leqslant\left\lvert t\right\rvert\leqslant\pi}Q_{k}(t)=0.
            }
        \end{equation*}

        \item Soit $t\in\R$, 
        \begin{align*}
            \left\lvert P_{k}(t)-f(t)\right\rvert
            &=\frac{1}{2\pi}\left\lvert\int_{-\pi}^{\pi}f(t-s)Q_{k}(s)\mathrm{d}s-\int_{-\pi}^{\pi}f(t)Q_{k}(s)\mathrm{d}s\right\rvert,\\
            &\leqslant\frac{1}{2\pi}\int_{-\pi}^{\pi}\left\lvert f(t-s)-f(t)\right\rvert Q_{k}(s)\mathrm{d}s,
        \end{align*}
        car $\frac{1}{2\pi}\int_{-\pi}^{\pi}Q_{k}(s)\mathrm{d}s=1$.
        $f$ est $\mathcal{C}^{0}$ sur $[0,4\pi]$ donc $f$ est uniformément continue sur $[0,4\pi]$. Soit $\varepsilon>0$, il existe $\delta_{1}>0$ tel que pour tout $(t,t')\in[0,4\pi]^{2}$, $\left\lvert t-t'\right\rvert\leqslant\delta_{1}\Rightarrow\left\lvert f(t)-f(t')\right\rvert\leqslant\varepsilon$. Alors pour tout $(t,t')\in\R^{2}$, si $\left\lvert t-t'\right\rvert\leqslant\min\left(\delta_{1},2\pi\right)$, alors $\left\lvert f(t)-f(t')\right\rvert\leqslant\varepsilon$. Donc $f$ est uniformément continue sur $\R$ et bornée sur $\R$ car continue $2\pi$-périodique.

        Soit $\varepsilon>0$ et $\delta>0$ ($\delta<\pi$) tel que pour tout $(t,t')\in[0,4\pi]^{2}$, $\left\lvert t-t'\right\rvert\leqslant\delta\Rightarrow\left\lvert f(t)-f(t')\right\rvert\leqslant\frac{\varepsilon}{2}$. Alors on a 
        \begin{equation*}
            \left\lvert P_{k}(t)-f(t)\right\rvert\leqslant\underbrace{\frac{1}{2\pi}\int_{[-\pi,\pi]\setminus[-\delta,\delta]}2\left\lVert f\right\rVert_{\infty}Q_{k}(s)\mathrm{d}s}_{\leqslant2\left\lVert f\right\rVert_{\infty}\sup\limits_{\delta\leqslant\left\lvert t\right\rvert \leqslant\pi}Q_{k}(t)\xrightarrow[k\to+\infty]{}0}+\underbrace{\frac{1}{2\pi}\int_{-\delta}^{\delta}\frac{\varepsilon}{2}Q_{k}(s)\mathrm{d}s}_{\leqslant\frac{\varepsilon}{2}},
        \end{equation*}
        donc il existe $N\in\N$ tel que pour tout $k\geqslant N$, pour tout $t\in\R$, $\left\lvert P_{k}(t)-f(t)\right\rvert\leqslant\varepsilon$. Donc $P_{k}$ converge uniformément vers $f$ sur $\R$.

        \item Montrons que $P_{k}\in F$. On a 
        \begin{align*}
            P_{k}(t)
            &=\frac{1}{2\pi}\int_{-\pi}^{\pi}f(t-s)Q_{k}(s)\mathrm{d}s,\\
            &=\frac{1}{2\pi}\int_{-\pi}^{\pi}f(u)Q_{k}(t-u)\mathrm{d}u,\\
            &=\frac{1}{2\pi}\frac{c_{k}}{2^{k}}\int_{-\pi}^{\pi}f(u)\left(1+\cos(t-u)\right)^{k}\mathrm{d}u,\\
            &=\frac{c_{k}}{2^{k+1}\pi}\int_{-\pi}^{\pi}f(u)\sum_{m=-k}^{k}\alpha_{m}\e^{\im(t-u)}\mathrm{d}u,
        \end{align*}
        où la dernière ligne est obtenue en développant $\cos(t-u)=\frac{\e^{\i(t-u)}+\e^{\i(u-t)}}{2}$.
        Ainsi, 
        \begin{equation*}
            P_{k}(t)=\sum_{m=-k}^{k}\left(\frac{c_{k}}{2^{k+1}\pi}\alpha_{m}\int_{-\pi}^{\pi}f(u)\e^{-\i mu}\mathrm{d}u\right)\e^{\i mt}\in F.
        \end{equation*}
        Donc $F$ est dense dans $E$.
    \end{enumerate}
\end{proof}

\begin{remark}
    Plus généralement, on peut remplacer la suite $Q_{k}$ par une \og approximation de l'unité\fg. Il faut une suite $(f_{k})_{k\in\N}$ telle que
    \begin{enumerate}
        \item [i)] $\forall k\in\N$, $f_{k}$ est continue et positive,
        \item [ii)] $\int_{\R}f_{k}=1$,
        \item [iii)] $\forall\delta >0$, $\lim\limits_{k+\infty}\int_{\R\setminus[-\delta,\delta]}f_{k}=0$.
    \end{enumerate}
    Alors si $f$ est uniformément continue et bornée de $\R$ dans $\C$, $(f\star f_{k})_{k\in\N}$ converge uniformément vers $f$ sur $\R$.
\end{remark}

\begin{remark}
    Soit \function{f}{[-1,1]}{\C}{x}{f(x)}
    $f$ est continue, on lui associe $g=f\circ \cos$, qui est continue $2\pi$-périodique. Ainsi, $(P_{k}\star g)_{k\in\N}$ converge uniformément vers $g$ sur $\R$ et 
    \begin{equation*}
        (Q_{k}\star g)(t)=\frac{c_{k}}{2^{k+1}\pi}\int_{-\pi}^{\pi}\left(1+\cos(t-u)\right)^{k}\mathrm{d}u,
    \end{equation*}
    qui est une fonction de $t$ parie car $g$ l'est. On a 
    \begin{equation*}
        (1+\cos(t-u))^{k}=\left(1+\cos(t)\cos(u)+\sin(t)\sin(u)\right)^{k}    
    \end{equation*}
    En développant, on a
    \begin{equation*}
        (Q_{k}\star g)(t)=A_{k}(\cos(t),\sin(t))=B_{k}(\cos(t)),
    \end{equation*}
    où $A_{k}\in\C[X,Y]$ (polynôme à deux variables) et $B\in\C[X]$ par parité. Ainsi, $(B_{k})_{k\in\N}$ converge uniformément vers $f$ sur $[-1,1]$: on vient de redémontrer le théorème de Weierstrass.
\end{remark}

\begin{proof}
    \phantom{}
    \begin{enumerate}
        \item Si $(u_n)$ est croissante, on pose $f_n=u-u_n$. Sinon, on pose $f_n=u_n-u$.
        \item $f_n$ est continue et $F_{n,\varepsilon}=f_{n}^{-1}\left([\varepsilon,+\infty[\right)$ donc $F_{n,\varepsilon}$ est fermé dans $K$, donc fermé. 
        
        Si $x\in F_{n+1,\varepsilon}$, on a $f_n(x)\geqslant f_{n+1}(x)\geqslant\varepsilon$ donc $x\in F_{n,\varepsilon}$. 
        
        Soit $x\in K$. Pour tout $\varepsilon>0$, il existe $N\in\N$ tel que $0\leqslant f_{N}(x)\leqslant\frac{\varepsilon}{2}$ donc $x\notin F_{n,\varepsilon}$ et $\cap_{n\in\N}F_{n,\varepsilon}=\emptyset$. Si pour tout $n\in\N$, on a $F_{n,\varepsilon}\neq\emptyset$, alors soit $(x_n)_{n\in\N}\in K^{\N}$ telle que pour tout $n\in\N,x_n\in F_{n,\varepsilon}$. $K$ étant compact, il existe $\varphi\colon\N\to\N$ strictement croissante telle que $\lim\limits_{n\to+\infty}x_{\varphi(n)}=x\in K$. Soit $n\in\N$ et $k\in\N$ tel que $\varphi(k)\geqslant n$. $x_{\varphi(k)}\in F_{\varphi(k),\varepsilon}\subset F_{n,\varepsilon}$. Alors, quand $k\to+\infty$, $x\in F_{n,\varepsilon}$ (fermé) donc $x\in\cap_{n\in\N}F_{n,\varepsilon}=\emptyset$ ce qui est absurde.

        Donc il existe $N\in\N$ tel que $F_{N,\varepsilon}=\emptyset$ et pour tout $n\geqslant N$, $F_{n,\varepsilon}=\emptyset$. Donc pour tout $n\geqslant N$, pour tout $x\in K$, $f(x)<\varepsilon$. Donc $(f_n)_{n\in\N}$ converge uniformément sur $K$.

        \item $f$ est continue sur un compact donc son maximum est atteint et $x_{n}$ existe. La suite $(\left\lVert f_n\right\rVert_{\infty})_{n\in\N}$ est décroissante positive, donc $\left\lVert f_n\right\rVert_{\infty}\xrightarrow[n\to+\infty]{}l\geqslant0$. Soit $\varphi\colon\N\to\N$ strictement croissante telle que $x_{\varphi(n)}\xrightarrow[n\to+\infty]{}x\in K$. Si $l>0$, alors il existe $N_{0}\in\N$ tel que $f_{\varphi(N_0)}(x)\leqslant\frac{l}{2}$. Par continuité de $f_{\varphi(N_0)}$, il existe $\alpha>0$ tel que pour tout $y\in\overline{B(x,\alpha)}$, $f_{\varphi(N_0)}(y)<l$. Il existe $N_{1}\in\N$ tel que pour tout $n\geqslant N_{1}$, $x_{\varphi(n)}\in\overline{B(x,\alpha)}$ donc pour tout $n\geqslant N_{1}$, $f_{\varphi(N_0)}(x_{\varphi(n)})<l$. Pour $n=\max(N_0,N_1)$, on a 
        \begin{equation*}
            l\leqslant f_{\varphi(n)}(x_{\varphi(n)})\leqslant f_{\varphi(N_{0})}(x_{\varphi(n)})<l,
        \end{equation*}
        ce qui est absurde. Donc $l=0$ et $(f_n)$ converge uniformément sur $K$.
    \end{enumerate}
\end{proof}

\begin{proof}
    \phantom{}
    \begin{enumerate}
        \item Soit $X\in[0,1]$, soit $(u_n)_{n\in\N}$ définie par $u_0=0$ et $u_{n+1}=u_n+\frac{1}{2}(x-u_{n}^{2})$. Soit $\varphi\colon t\mapsto t+]/  1   2   (x-t^{2})$. On a $\varphi'(t)=1-t\geqslant0$ sur $[0,1]$ et $\varphi(t)=t$ si et seulement si $t=\sqrt[]{x}$. Si $t<\sqrt[]{x}$, $\varphi(t)\geqslant t$. Donc pour tout $n\in\N$, $u_n\in[0,\sqrt[]{x}]$, $(u_n)_{n\in\N}$ est croissante marorée et converge vers $\sqrt[]{x}$ donc $(f_n)_{n\in\N}$ converge simplement. On a 
        \begin{align*}
            0&\leqslant
            \sqrt{f}_{n+1}(x),\\
            &=
            \left(\sqrt{x}-f_n(x)\right)\left(1-\frac{1}{2}\left(\sqrt{x}+f_n(x)\right)\right).
        \end{align*}

        Soit $\varepsilon>0$. Si $x\in[0,\varepsilon^{2}]$, pour tout $n\in\N$, $0\leqslant\sqrt{x}-f_n(x)\leqslant\sqrt{x}\leqslant\varepsilon$. Si $x>\varepsilon^{2}$, pour tout $n\in\N$ on a $\sqrt{x}+f_n(x)\geqslant \varepsilon$ et  par récurrence, pour tout $n\in\N$, pour tout $x\geqslant\varepsilon^{2}$, on a
        \begin{equation*}
            0\leqslant\sqrt{x}-f_{n+1}(x)\leqslant\left(1-\frac{1}{2}\varepsilon\right)^{n}\left(\sqrt{x}-f_0(x)\right).
        \end{equation*}
        Comme $\left(1-\frac{1}{2}\varepsilon\right)^{n}\xrightarrow[n\to+\infty]{}0$ et $\sqrt{x}-f_0(x)\leqslant 1$, il existe $N\in\N$ tel que pour tout $n\geqslant N$, $\left(1-\frac{1}{2}\varepsilon\right)^{n}\leqslant \varepsilon$, et $(f_n)_{n\in\N}$ converge uniformément vers $\sqrt{}$ sur $[0,1]$.

        \item Par récurrence, $f_n$ est polynomiale et on pose pour tout $n\in\N$, \function{P_n}{[-1,1]}{\R}{x}{f_n(x^{2})}
        qui converge uniformément vers $\left\lvert~\right\rvert$ sur $[-1,1]$.

        \item Soit $I=[a,b]$ avec $a<b$, $\varphi$ affine par morceaux de la forme 
        \begin{equation*}
            \varphi(x)=\sum_{i=1}^{n}b_i\left\lvert x-a_i\right\rvert.    
        \end{equation*}
        $\varphi$ est limite uniforme d'une suite de polynômes d'après la question précédente. Or l'espace des fonctions affines par morceaux est dense dans $\mathcal{C}^{0}([a,b],\R)$. D'où le théorème de Weierstrass.
    \end{enumerate}
\end{proof}

\end{document}
\documentclass[12pt]{article}
\usepackage{style/style_sol}

\begin{document}

\begin{titlepage}
	\centering
	\vspace*{\fill}
	\Huge \textit{\textbf{Solutions MP/MP$^*$\\ Intégration}}
	\vspace*{\fill}
\end{titlepage}

\begin{proof}
    $S$ est de classe $\mathcal{C}^{1}$ sur $[a,b]$ avec $S'=f>0$. Donc $S$ définit un $\mathcal{C}^{1}$-difféomorphisme de $[a,b]$ dans $[S(a)=0,S(b)]$. COmme pour tout $n\geqslant1$, pour tout $k\in\left\llbracket1,n\right\rrbracket$, $k\frac{S(b)}{n}\in[0,S(b)]$, il existe un unique $x_{k}\in[a,b]$ tel que $S(x_{k})=k\frac{S(b)}{n}$ qui est simplement donné par 
    \begin{equation}
        \boxed{
            x_{k}=S^{-1}\left(k\frac{S(b)}{n}\right).
        }
    \end{equation}

    On a 
    \begin{equation}
        \frac{1}{n}\sum_{k=1}^{n}f(x_{k})=\frac{1}{S(b)}\left(\frac{S(b)}{n}\sum_{k=1}^{n}f\left(S^{-1}\left(\frac{k}{n}S(b)\right)\right)\right)\xrightarrow[n\to+\infty]{}\frac{1}{S(b)}\int_{0}^{S(b)}f\left(S^{-1}(t)\right)\mathrm{d}t=I.
    \end{equation}
    On effectue le changement de variable $u=S^{-1}(t)$ pour obtenir
    \begin{equation}
        \boxed{
            I=\frac{1}{S(b)}\int_{0}^{b}f(u)^{2}\mathrm{d}u.
        }
    \end{equation}
\end{proof}

\begin{remark}
    On peut se demander si cela reste vrai si $f\geqslant0$ (mais $f\neq 0$). On définit \function{\varphi}{[0,S(b)]}{[a,b]}{y}{\min\left(\left\lbrace x\in[a,b]\middle|y=S(x)\right\rbrace\right)}
    On a $x_{k}=\varphi\left(k\frac{S(b)}{n}\right)$, $f\circ \varphi$ continue par morceaux sur $[0,S(b)]$ et 
    \begin{equation}
        \frac{1}{n}\sum_{k=1}^{n}f\left(\varphi\left(k\frac{S(b)}{n}\right)\right)\xrightarrow[n\to+\infty]{}\frac{1}{S(b)}\int_{0}^{S(b)}\left(f\circ \varphi\right)(t)\mathrm{d}t.
    \end{equation}

    Cela marche aussi si $\left\lbrace t\in[a,b], f(t)=0\right\rbrace$ est discret car $S$ reste strictement croissante. Cela marche aussi si $\int f>0$ (poser $f_p=f+\frac{1}{p}>0$ et passer à la limite).
\end{remark}

\begin{proof}
    \phantom{}
    \begin{enumerate}
        \item Pour tout $x>0$, on a $g(x)\leqslant\left\lVert f\right\rVert_{\infty}$. Soit $t_{0}\in[0,1]$ tel que $\left\lvert f(t_{0})\right\rvert=\left\lVert f\right\rVert_{\infty}$. Soit $\varepsilon>0$. Par continuité de $\left\lvert f\right\rvert$, il existe $[a,b]\subset[0,1]$ avec $a<b$ tel que pour tout $t\in[a,b]$, $0<\left\lvert f(t_{0})\right\rvert-\frac{\varepsilon}{2}<\left\lvert f(t)\right\rvert$.

        D'où 
        \begin{equation}
            \left(\int_{0}^{1}\left\lvert f(t)\right\rvert^{x}\mathrm{d}t\right)^{\frac{1}{x}}\geqslant\left(\int_{a}^{b}\left\lvert f(t)\right\rvert^{x}\mathrm{d}t\right)^{\frac{1}{x}}\geqslant(b-a)^{\frac{1}{x}}\left(\left\lvert f(t_{0})\right\rvert-\frac{\varepsilon}{2}\right)\xrightarrow[x\to+\infty]{}\left\lvert f(t_{0})\right\rvert-\frac{\varepsilon}{2}.
        \end{equation}
        Alors il existe $X_{1}>0$ tel que pour tout $x\geqslant X_{1}$, $\left\lvert f(t_{}0)\right\rvert-\frac{\varepsilon}{2}-\frac{\varepsilon}{2}=\left\lVert f\right\rVert_{\infty}-\varepsilon\leqslant g(x)$. D'où le résultat.

        \item On pose $h_{x}(t)=\exp(x\ln(\left\lvert f(t)\right\rvert))$ pour tout $t\in[0,1]$. Alors pour tout $t\in[0,1]$, on a $\lim\limits_{x\to0}h_{x}(t)=1$ et pour tout $x>0$, pour tout $t\in[0,1]$ $h_{x}(t)\leqslant\max\left(1,\left\lVert f\right\rVert_{\infty}\right)$ qui est intégrable sur $[0,1]$. D'après le théorème de convergence dominé, on a 
        \begin{equation}
            \lim\limits_{x\to0}\int_{0}^{1}\left\lvert f(t)\right\rvert^{x}\mathrm{d}t=1.
        \end{equation}

        Malheureusement, ce n'est pas suffisant.

        Posons donc $k_{x}(t)=\frac{\left\lvert f(t)\right\rvert^{x}-1}{x}$. Pour $t$ fixé, on a $\lim\limits_{x\to0}k_{x}(t)=\ln\left(\left\lvert f(t)\right\rvert\right)$. De plus, pour tout $0<x\leqslant1$, pour tout $t\in[0,1]$, on a 
        \begin{equation}
            \left\lvert\left\lvert f(t)\right\rvert^{x}-1\right\rvert=\left\lvert\e^{x\ln(\left\lvert f(t)\right\rvert)}-\e^{0}\right\rvert 
            \left\lbrace
                \begin{array}[]{ll}
                    \leqslant x\ln(\left\lvert f(t)\right\rvert),&\text{si }\left\lvert f(t)\right\rvert\leqslant1,\\\hline
                    \leqslant x\ln(\left\lvert f(t)\right\rvert)\e^{x\ln(\left\lvert f(t)\right\rvert)}\\
                    \leqslant x\ln(\left\lvert f(t)\right\rvert)\e^{\ln(\left\lvert f(t)\right\rvert)},&\text{si }\left\lvert f(t)\right\rvert>1.
                \end{array}
            \right.
        \end{equation}

        Ainsi $k_{x}(t)\leqslant\max\left(\ln\left(\left\lVert f\right\rVert_{\infty}\right), \left\lVert f\right\rVert_{\infty}\ln\left(\left\lVert f\right\rVert_{\infty}\right)\right)$ qui est intégrable sur $[0,1]$. D'après le théorème de convergence dominé,
        \begin{equation}
            \lim\limits_{x\to0}\int_{0}^{1}\frac{f(t)^{x}-1}{x}\mathrm{d}t=\int_{0}^{1}\ln\left(\left\lvert f(t)\right\rvert\right)\mathrm{d}t.
        \end{equation}

        Ainsi,
        \begin{align}
            g(x)
            &=\exp\left(\frac{1}{x}\ln\left(1+x\int_{0}^{1}k_{x}(t)\mathrm{d}t\right)\right),\\
            &=\exp\left(\frac{1}{x}\left(x\int_{0}^{1}\ln\left(\left\lvert f(t)\right\rvert\right)+\underset{x\to0}{o}(x)\right)\right),\\
            &=\exp\left(\int_{0}^{1}\ln\left(\left\lvert f(t)\right\rvert\right)\mathrm{d}t+\underset{x\to0}{o}(1)\right)\xrightarrow[x\to0]{}\exp\left(\int_{0}^{1}\ln\left(\left\lvert f(t)\right\rvert\right)\mathrm{d}t\right).
        \end{align}
    \end{enumerate}
\end{proof}

\begin{proof}
    On fixe $y\in[0,f(a)]$. On pose \function{\varphi}{[0,a]}{\R}{x}{\int_{0}^{x}f+\int_{0}^{y}g-xy}
    $\varphi$ est $\mathcal{C}^{1}$ et $\varphi'(x)=f(x)-y$ donc $\varphi$ décroît de $0$ à $g(y)$ puis croît jusqu'en $x=a$. Son minimum vaut alors $\varphi(g(y))=\int_{0}^{x}+\int_{0}^{f(x)}g-xf(x)$ avec $x=g(y)$.

    Si $f$ est $\mathcal{C}^{1}$, alors $g$ l'est aussi car $f$ définit un $\mathcal{C}^{1}$-difféomorphisme de $[0,a]$ dans $[0,f(a)]$. On effectue le changement de variable $u=f(t)$ et on obtient $\varphi(g(y))=\int_{0}^{x}(tf'(t)+f(t))\mathrm{d}t-xf(x)=[tf(t)]_{0}^{x}-xf(x)=0$. De même si $f$ est $\mathcal{C}^{1}$ par morceaux (utiliser la relation de Chasles).

    Plus généralement, on a le lemme
    \begin{lemma}
        Soit pour $n\geqslant1$, $f_n:[0,a]\to\R$ affine par morceaux continue telle que pour tout $k\in\left\llbracket0,n\right\rrbracket$, $f_{n}\left(\frac{k}{n}a\right)=f\left(\frac{k}{n}a\right)$. Alors $(f_{n})_{n\geqslant1}$ converge uniformément vers $f$ sur $[0,a]$ et $(f_{n}^{-1})_{n\geqslant1}$ converge uniformément vers $f$ sur $[0,f(a)]$.
    \end{lemma}
    \begin{proof}[Preuve du lemme]
        Soit $\varepsilon>0$. Par continuité uniforme de $f$, il existe $N_{0}\in\N$ tel que pour tout $n\geqslant N_{0}$, pour tout $k\in\left\llbracket0,n-1\right\rrbracket$, pour tout $x\in\left[\frac{ka}{n},\frac{k+1}{n}a\right]$, on a $\left\lvert f(x)-f\left(\frac{k}{n}a\right)\right\rvert\leqslant\frac{\varepsilon}{2}$. Alors 
        \begin{equation}
            \left\lvert f(x)-f_n(x)\right\rvert\leqslant\left\lvert f(x)-f\left(\frac{ka}{n}\right)\right\rvert+\left\lvert f_n\left(\frac{k}{n}a\right)-f_n(x)\right\rvert\leqslant\varepsilon.
        \end{equation}
        On fait de même pour $(f_n^{-1})_{n\geqslant1}$.
    \end{proof}
    $f_n$ et $f_n^{-1}$ sont $\mathcal{C}^{1}$ par morceaux continues et $g_n=f_{n}^{-1}$. On a $\int_{0}^{x}f_{n}+\int_{0}^{f_n(x)}f_n=xf_n(x)$. Quand $n\to+\infty$, par convergence uniforme, on a $\int_{0}^{f_n(x)}g_n=\int_{0}^{f(x)}g_n+\int_{f_n(x)}^{f(x)}g_n$ et le dernier terme est uniformément borné par $\left\lVert f^{-1}\right\rVert_{\infty}\left\lvert f(x)-f_n(x)\right\rvert\xrightarrow[n\to+\infty]{}0$. Ainsi, le cas d'égalité est quand 
    \begin{equation}
        \boxed{
            \int_{0}^{x}f+\int_{0}^{f(x)}g=xf(x).
        }
    \end{equation}
\end{proof}

\begin{proof}
    On pose $f(x)=\frac{\ln(x)}{(1+x)\sqrt{1-x^{2}}}$. $f$ est continue sur $\left[\frac{1}{2},1\right[$ et $f(x)\underset{x\to1^{-}}{\sim}\frac{x-1}{2\sqrt{2(1-x)}}\xrightarrow[x\to^{1^{-}}]{}0$. On effectue le changement de variable $x=\cos(t)$ d'où $\mathrm{d}t=-\frac{\mathrm{d}x}{\sqrt{1-x^{2}}}$. On a alors 
    \begin{equation}
        I=-\int_{\frac{\pi}{3}}^{0}\frac{\ln(\cos(t))}{1+\cos(t)}\mathrm{d}t=\int_{0}^{\frac{\pi}{3}}\frac{\ln(\cos(t))}{2\cos^{2}\left(\frac{t}{2}\right)}\mathrm{d}t.
    \end{equation}
    Or $\tan'\left(\frac{t}{2}\right)=\frac{1}{2\cos^{2}\left(\frac{t}{2}\right)}$ donc par intégrations par parties,
    \begin{equation}
        I=\left[\ln(\cos(t))\tan\left(\frac{t}{2}\right)\right]_{0}^{\frac{\pi}{3}}-\int_{0}^{\frac{\pi}{3}}\frac{-\sin(t)}{\cos(t)}\tan\left(\frac{t}{2}\right)\mathrm{d}t.
    \end{equation}
    Le premier terme vaut $\frac{\ln(\frac{1}{2})}{\sqrt{3}}$.
    Pour le deuxième terme, on utilise la formule d'addition $\tan\left(\frac{t}{2}+\frac{t}{2}\right)=\frac{2\tan\left(\frac{t}{2}\right)}{1-\tan^{2}\left(\frac{t}{2}\right)}$. Ainsi,
    \begin{equation}
        \int_{0}^{\frac{\pi}{3}}\tan(t)\tan\left(\frac{t}{2}\right)\mathrm{d}t=\int_{0}^{\frac{\pi}{3}}\frac{2\tan^{2}\left(\frac{t}{2}\right)}{1-\tan^{2}\left(\frac{t}{2}\right)}\mathrm{d}t=\int_{0}^{\frac{1}{\sqrt{3}}}\frac{4u^{2}}{1-u^{2}}\frac{\mathrm{d}u}{1+u},
    \end{equation}
    en ayant effectué le changement de variables $u=\tan\left(\frac{t}{2}\right)$, d'où $\mathrm{d}t=\frac{2\mathrm{d}u}{1+u^{2}}$. Il ne reste plus qu'à faire une décomposition en éléments simples.
\end{proof}

\begin{proof}
    \phantom{}
    \begin{enumerate}
        \item $I_{n}$ est bien définie. On a 
        \begin{align}
            I_{n}+I_{n+2}
            &=\int_{0}^{\frac{\pi}{4}}\tan^{n}(x)(1+\tan^{2}(x))\mathrm{d}x,\\
            &=[\tan^{n+1}(x)]_{0}^{\frac{\pi}{4}}-n\int_{0}^{\frac{\pi}{4}}\tan^{n}(x)(1+\tan^{2}(x))\mathrm{d}x,\\
            =1-n(I_{n}+I_{n+2}).
        \end{align}
        Donc $I_{n}+I_{n+2}=\frac{1}{n+1}$. On a $I_{0}=\frac{\pi}{4}$. On en déduit que 
        \begin{equation}
            I_{2p}=\frac{1}{2p-1}-I_{2p-2}=\dots=(-1)^{p}\left(\frac{\pi}{4}-1+\frac{1}{3}-\frac{1}{5}+\dots+\frac{(-1)^{p}}{2p-1}\right).
        \end{equation}

        On a $I_{1}=\int_{0}^{\frac{\pi}{4}}\tan(x)\mathrm{d}x=[-\ln(\cos(x))]_{0}^{\frac{\pi}{4}}=\frac{1}{2}\ln(2)$. Ainsi,
        \begin{equation}
            I_{2p+1}=(-1)^{p}\left(\frac{\ln(2)}{2}-\frac{1}{2}+\frac{1}{4}-\frac{1}{6}+\dots+\frac{(-1)^{p}}{2p}\right).
        \end{equation}

        \item On pose $f_n(x)=\tan^{n}(x)$. Si $x\in\left[0,\frac{\pi}{4}\right[$, on a $\lim\limits_{n\to+\infty}f_n(x)=0$. Si $x=\frac{\pi}{4}$, on a $\lim\limits_{n\to+\infty}f_n(x)=1$. Donc $(f_n)_{n\in\N}$ converge simplement vers $f\colon\left[0,\frac{\pi}{4}\right]\to\R$ qui vaut $0$ partout sauf en $\frac{\pi}{4}$ où elle vaut 1.
        Soit $n\in\N$ et $x\in\left[0,\frac{\pi}{4}\right]$. On a $\left\lvert f_n(x)\right\rvert\leqslant1$ intégrable sur $\left[0,\frac{\pi}{4}\right]$. D'après le théorème de convergence dominée,
        \begin{equation}
            \boxed{
                \lim\limits_{n\to+\infty}I_{n}=0.
            }
        \end{equation}

        \item D'après ce qui précède, on a 
        \begin{equation}
            \everymath={\displaystyle}
            \boxed{
                \begin{array}[]{rcl}
                    \frac{\pi}{4}&=&\sum_{n=0}^{+\infty}\frac{(-1)^{k}}{2k+1},\\
                    \ln(2)&=&\sum_{k=0}^{+\infty}\frac{(-1)^{k}}{k+1}.
                \end{array}
            }
        \end{equation}
    \end{enumerate}
\end{proof}

\begin{remark}
    On peut donner un équivalent de $I_{n}$. Comme pour tout $x\in\left[0,\frac{\pi}{4}\right]$, on a $0\leqslant\tan(x)\leqslant1$, on a $I_{n+2}\leqslant I_{n}$. Ainsi, 
    \begin{equation}
        2I_{n+2}\leqslant I_{n}+I_{n+2}=\frac{1}{n+1}\leqslant 2I_{n},
    \end{equation}
    et donc 
    \begin{equation}
        \frac{1}{2(n+1)}\leqslant I_n\leqslant\frac{1}{2(n-1)},
    \end{equation}
    d'où
    \begin{equation}
        \boxed{
            I_{n}\underset{n\to+\infty}{\sim}\frac{1}{2n}.
        }
    \end{equation}
\end{remark}

\begin{proof}
    \phantom{}
    \begin{enumerate}
        \item D'après l'inégalité de Cauchy-Schwarz appliquée à $\sqrt{f}$ et $\frac{1}{\sqrt{f}}$, on a 
        \begin{equation}
            \int_{a}^{b}f\times\int_{a}^{b}\frac{1}{f}\geqslant\left(\int_{a}^{b}1\right)^{2}=(b-a)^{2}.
        \end{equation}
        $f\colon x\mapsto 1$ pour tout $x\in[a,b]$ donne l'égalité, et d'après le cas d'égalité de l'inégalité de Cauchy-Schwarz, on a égalité si et seulement si $\sqrt{f}$ et $\frac{1}{\sqrt{f}}$ sont proportionnelles, donc si et seulement si $f$ est constante.

        \item Soit $\alpha\in\R_{+}^{*}\setminus\left\lbrace1\right\rbrace$ et $c<a$. Soit \function{f_{\alpha,c}}{[a,b]}{\R_+^*}{t}{(t-c)^\alpha}
        On a
        \begin{align}
            \phi(f_{\alpha,c})
            &=\frac{1}{\alpha^{2}-1}\left[(b-c)^{\alpha+1}-(a-c)^{\alpha+1}\right]\left[(a-c)^{-\alpha+1}-(b-c)^{-\alpha+1}\right],\\
            &\underset{\alpha\to+\infty}{\sim}\frac{1}{\alpha^{2}}\left[(b-c)^{\alpha+1}\times\frac{1}{(a-c)^{\alpha-1}}\right],\\
            &\underset{\alpha\to+\infty}{\sim}\frac{(b-a)(a-c)}{\alpha^{2}}\left(\frac{b-c}{a-c}\right)^{\alpha}\xrightarrow[\alpha\to+\infty]{}+\infty,
        \end{align}
        car $b-c>a-c$.

        \item Soit $f,g\in E^{2}$ et $\lambda\in[0,1]$. $\lambda f+(1-\alpha)g$ est continue et strictement positive. $E$ est convexe dans $\left(\mathcal{C}^{0}\left([a,b],\R_{+}^{*}\right),\left\lVert\cdot\right\rVert_{\infty}\right)$ donc connexe par arcs.
        
        Soit $f\in E$ et $(f_n)_{n\in\N}$ suite de fonctions convergent uniformément vers $f$. Par convergence uniforme, on a $\int_{a}^{b}f_n\xrightarrow[n\to+\infty]{}\int_{a}^{b}f$. De plus, pour tout $x\in[a,b]$, on a 
        \begin{equation}
            \left\lvert\frac{1}{f_n(x)}-\frac{1}{f(x)}\right\rvert=\frac{\left\lvert f_n(x)-f(x)\right\rvert}{f_n(x)\times f(x)}\leqslant\frac{\left\lVert f_n-f\right\rVert_{\infty}}{\min_{y\in[a,b]f_n(y)\times f(y)}}.
        \end{equation}
        Il existe $n_0\in\N$ tel que pour tout $n\geqslant n_0$, $\left\lVert f_n-f\right\rVert_{\infty}\leqslant\frac{\min f}{2}$ et pour tout $x\in[a,b]$, pour tout $n\geqslant n_0$, $f_n(x)\geqslant\frac{\min f}{2}$. Alors 
        \begin{equation}
            \left\lVert \frac{1}{f_n}-\frac{1}{f}\right\rVert_{\infty}\leqslant\frac{2\left\lVert f_n-f\right\rVert_{\infty}}{(\min f)^{2}}\xrightarrow[n\to+\infty]{}0.
        \end{equation}
        Ainsi, $\int_{a}^{b}\frac{1}{f_n}\xrightarrow[n\to+\infty]{}\int_{a}^{b}\frac{1}{f}$ et $\phi(f_n)\xrightarrow[n\to+\infty]{}\phi(f)$. $\phi$ est donc continue. D'après le théorème des valeurs intermédiaires, on a donc 
        \begin{equation}
            \boxed{
                \phi(E)=[(b-a)^{2},+\infty[.
            }
        \end{equation}
    \end{enumerate}
\end{proof}

\begin{proof}
    Soit \function{f}{]0,+\infty}{\R}{x}{\frac{\sqrt{x}\ln(x)}{(1+x)^{2}}}
    $f$ est continue. On a $f(x)\xrightarrow[x\to0]{}0$ donc $\int_{0}^{1}f$ converge. On a $f(x)\underset{x\to+\infty}{\sim}\ln(x)\frac{1}{x^{\frac{3}{2}}}=\underset{x\to+\infty}{O}\left(\frac{1}{x^{\frac{5}{4}}}\right)$ donc $\int_{1}^{+\infty}f$ converge.

    On pose $x=u^{2}$ et on obtient 
    \begin{align}
        I
        &=\int_{0}^{+\infty}\frac{2u^{2}\ln(u^{2})}{(1+u^{2})^{2}}\mathrm{d}u,\\
        &=4\int_{0}^{+\infty}\frac{u^{2}\ln(u)}{(1+u^{2})^{2}}\mathrm{d}u,\\
        &=2\left(\left[-\frac{1}{(1+u^{2})}\times u\ln(u)\right]_{0}^{+\infty}+\int_{0}^{+\infty}\frac{1}{1+u^{2}}\left(\ln(u)+0\right)\mathrm{d}u\right),\label{it:ipp}\\
        &=2\left(\int_{0}^{+\infty}\frac{\ln(u)}{1+u^{2}}\mathrm{d}u+\int_{0}^{+\infty}\frac{1}{1+u^{2}}\mathrm{d}u\right).
    \end{align}

    Noter que l'intégration par parties faite en~\ref{it:ipp} est correcte car tout converge en 0 et $+\infty$ (passer à la limite $\alpha,\beta\to0,+\infty$ pour être plus rigoureux).

    La première intégrale est nulle. En effet, on pose $x=\frac{1}{u}$ d'où $\mathrm{d}x=-\frac{\mathrm{d}u}{u^{2}}$ et donc 
    \begin{equation}
        \int_{0}^{+\infty}\frac{\ln(u)}{1+u^{2}}\mathrm{d}u=-\int_{+\infty}^{0}\frac{\ln\left(\frac{1}{x}\right)}{1+\frac{1}{x^{2}}}\frac{\mathrm{d}x}{x^{2}}=-\int_{0}^{+\infty}\frac{\ln(x)}{1+x^{2}}\mathrm{d}x.
    \end{equation}
    La deuxième intégrale vaut $\frac{\pi}{2}$. Finalement, on a 
    \begin{equation}
        \boxed{
            I=\pi.
        }
    \end{equation}
\end{proof}

\begin{proof}
    On note $f$ la fonction intégrande. $f$ est continue négative. On a $\left\lvert f(t)\right\rvert\underset{t\to0}{\sim}\left\lvert\frac{\ln(t)}{\sqrt{t}}\right\rvert=\underset{t\to0}{O}\left(\frac{1}{t^{\frac{3}{4}}}\right)$ donc $\int_{0}^{\frac{1}{2}}f$ converge. On a $\left\lvert f(t)\right\rvert\underset{t\to1}{\sim}\frac{1}{\sqrt{1-t}}$ donc $\int_{\frac{1}{2}}^{1}f$ converge.

    On a 
    \begin{equation}
        I=\int_{0}^{1}\frac{\ln(t)}{1-t}\frac{\mathrm{d}t}{\sqrt{t(1-t)}}.
    \end{equation}
    Comme $t(1-t)=-(t^{2}-t)=-\left(\left(t-\frac{1}{2}\right)^{2}-\frac{1}{4}\right)=\frac{1}{4}\left(1-(2t-1)^{2}\right)$, on pose $2t-1=\cos\theta$. On a alors $t=\frac{\cos\theta+1}{2}$ et $\mathrm{d}\theta=\frac{-2\mathrm{d}t}{\sqrt{1-(2t-1)^{2}}}=\frac{-\mathrm{d}t}{\sqrt{t(1-t)}}$. Ainsi, 
    \begin{equation}
        I=\int_{0}^{\pi}\frac{\ln\left(\frac{\cos\theta+1}{2}\right)}{\frac{1-\cos\theta}{2}}\mathrm{d}\theta.
    \end{equation}
    On a $\frac{1+\cos\theta}{2}=\cos^{2}\left(\frac{\theta}{2}\right)$ et $\frac{1-\cos\theta}{2}=\sin^{2}\left(\frac{\theta}{2}\right)$. En posant $u=\frac{\theta}{2}$, on a donc 
    \begin{equation}
        I=4\int_{0}^{\frac{\pi}{2}}\frac{\ln(\cos u)}{\sin^{2}u}\mathrm{d}u.
    \end{equation}
    En fixant $0<\varepsilon<\alpha<1$ et en posant $I_{\varepsilon,\alpha}=\int_{\varepsilon}^{\alpha}f$, on a en faisant une intégration par parties:
    \begin{equation}
        I_{\varepsilon,\alpha}=4\left(\left[-\cot u\times\ln(\cos u)\right]_{\varepsilon}^{\alpha}-\int_{\varepsilon}^{\alpha}1\mathrm{d}u\right).
    \end{equation}
    Le deuxième terme tend vers $\frac{\pi}{2}$,. Pour le premier, si $\alpha=\frac{\pi}{2}-h$, on a 
    \begin{equation}
        -\cot\alpha\ln\cos\alpha=-\tan h\ln\sin h=-\tan h\left[\ln h+\underset{h\to0}{o}(1)\right]\underset{h\to0}{\sim}-h\ln(h)\xrightarrow[h\to0]{}0.
    \end{equation}
    De même, on a 
    \begin{equation}
        -\cot\varepsilon\ln\cos\varepsilon\underset{\varepsilon\to0}{\sim}-\frac{1}{\varepsilon}\times \frac{-\varepsilon^{2}}{2}\underset{\varepsilon\to0}{\sim}\frac{\varepsilon}{2}\xrightarrow[\varepsilon\to0]{}0.
    \end{equation}

    Ainsi, 
    \begin{equation}
        \boxed{
            I=-2\pi.
        }
    \end{equation}
\end{proof}

\begin{proof}
    On note $f$ la fonction intégrande. Si $h=\frac{\pi}{4}-t$, on a $\cos(2t)=\cos\left(\frac{\pi}{2}-2h\right)=\sin(2h)\underset{h\to0}{\sim}2h$. Ainsi, 
    \begin{equation}
        f(t)\underset{t\to\frac{\pi}{4}}{\sim}\frac{\frac{1}{2\sqrt{2}}}{\sqrt{2\left(\frac{\pi}{4}-t\right)}},
    \end{equation}
    donc l'intégrale existe (critère de Riemann).

    En posant $u=\sin(t)$, puis $v=\sqrt{2}u$, puis $\theta=\arcsin(v)$, on a 
    \begin{align}
        I
        &= \int_{0}^{\frac{\pi}{4}}\frac{\left(1-\sin^{2}(t)\right)\cos(t)}{\sqrt{1-2\sin^{2}(t)}}\mathrm{d}t,\\
        &= \int_{0}^{\frac{\sqrt{2}}{2}}\frac{1-u^{2}}{\sqrt{1-2u^{2}}}\mathrm{d}u,\\
        &= \int_{0}^{1}\frac{1-\frac{u^{2}}{2}}{\sqrt{2}}\frac{\mathrm{d}u}{\sqrt{1-u^{2}}},\\
        &= \int_{0}^{\frac{\pi}{2}}\frac{1-\frac{\sin^{2}\theta}{2}}{\sqrt{2}}\mathrm{d}\theta,\\
        &= \frac{1}{\sqrt{2}}\left(\frac{\pi}{2}-\frac{1}{4}\int_{0}^{\frac{\pi}{2}}(1-\cos(2\theta))\mathrm{d}\theta\right),\\
        &=\frac{3\pi-1}{8\sqrt{2}}.
    \end{align}
\end{proof}

\begin{proof}
    Si $f=c\in\C$ est constante, on a 
    \begin{equation}
        \gamma=\int_{a}^{b}f(t)g(\lambda t)\mathrm{d}t=c\int_{a}^{b}g(\lambda t)\mathrm{d}t.
    \end{equation}
    On pose $u=\lambda t$ et on pose $k(\lambda)=\left\lfloor\frac{\lambda b-\lambda a}{T}\right\rfloor\underset{\lambda\to+\infty}{\sim}\frac{\lambda(b-a)}{T}$. Alors 
    \begin{equation}
        \gamma=\frac{c}{\lambda}k(\lambda)\int_{0}^{T}g+\frac{c}{\lambda}\int_{\lambda a+k(\lambda)T}^{\lambda b}g.
    \end{equation}
    Le deuxième terme est majoré par $\frac{\left\lvert c\right\rvert}{\lambda}T\left\lVert g\right\rVert_{\infty}\xrightarrow[\lambda\to+\infty]{}0$. Finalement,
    \begin{equation}
        \lim\limits_{\lambda\to+\infty}\gamma=\frac{c(b-a)}{T}\int_{0}^{T}g=\frac{1}{T}\int_{0}^{T}g\int_{a}^{b}f.
    \end{equation}

    C'est la même chose pour les fonctions en escalier (par combinaison linéaire).

    Pour une fonction quelconque $f$ continue par morceaux, soit $\varepsilon>0$. Il existe $f_{\varepsilon}$ une fonction en escalier telle que $\left\lVert f-f_{\varepsilon}\right\rVert_{\infty}\leqslant\varepsilon$. On forme 
    \begin{equation}
        \Gamma=\left\lvert\int_{a}^{b}(f(t)g(\lambda t))\mathrm{d}t-\frac{1}{T}\int_{0}^{T}g\int_{a}^{b}f\right\rvert.
    \end{equation}

    On a 
    \begin{align}
        \Gamma
        &=\left\lvert \int_{a}^{b}f_{\varepsilon}(t)g(\lambda t)\mathrm{d}t+\int_{a}^{b}(f(t)-f_{\varepsilon}(t))g(\lambda t)\mathrm{d}t-\frac{1}{T}\int_{0}^{T}g\int_{a}^{b}f_{\varepsilon}-\frac{1}{T}\int_{0}^{T}g\int_{a}^{b}(f-f_{\varepsilon})\right\rvert,\\
        &\leqslant \left\lvert \int_{a}^{b}f_{\varepsilon}(t)g(\lambda t)\mathrm{d}t-\frac{1}{T}\int_{0}^{T}g\int_{a}^{b}f_{\varepsilon}\right\rvert+\left\lvert \int_{a}^{b}(f(t)-f_{\varepsilon}(t))g(\lambda t)\mathrm{d}t\right\rvert+\left\lvert\frac{1}{T}\int_{0}^{T}g\int_{a}^{b}(f-f_{\varepsilon})\right\rvert.
    \end{align}

    Il existe $\lambda_{0}\in\R$ tel que pour tout $\lambda\geqslant\lambda_{0}$,
    \begin{equation}
        \left\lvert\int_{a}^{b}f_{\varepsilon}(t)g(\lambda t)\mathrm{d}t-\frac{1}{T}\int_{0}^{T}\int_{a}^{b}f_{\varepsilon}\right\rvert\leqslant\frac{\varepsilon}{3}.
    \end{equation}

    Ainsi, $\Gamma\leqslant\frac{\varepsilon}{3}\times 3=\varepsilon$. Donc 
    \begin{equation}
        \boxed{
            \lim\limits_{\lambda\to+\infty}\int_{a}^{b}f(t)g(\lambda t)\mathrm{d}t=\frac{1}{T}\int_{0}^{T}g\int_{a}^{b}f.
        }
    \end{equation}

    Pour le cas particulier, on a $g(t)=\frac{1}{3+2\cos(t)}$. $g$ est $2\pi$-périodique, paire et strictement positive. On pose $x=\tan\left(\frac{t}{2}\right)$, on a $\cos(t)=\frac{1-x^{2}}{1+x^{2}}$ et $\sin(t)=\frac{2x}{1+x^{2}}$. Par parité, on a $\int_{0}^{2\pi}g=2\int_{0}^{\pi}g$, et 
    \begin{align}
        \int_{0}^{\pi}g(t)\mathrm{d}t
        &=\int_{0}^{+\infty}\frac{2\mathrm{d}x}{(1+x^{2})\left(3+2\left(\frac{1-x^{2}}{1+x^{2}}\right)\right)},\\
        &= 2\int_{0}^{+\infty}\frac{\mathrm{d}x}{x^{2}+5},\\
        &= \frac{2}{\sqrt{5}}\int_{0}^{+\infty}\frac{\frac{\mathrm{d}x}{\sqrt{5}}}{\left(\frac{x}{\sqrt{5}}\right)^{2}+1},\\
        &= \frac{2}{\sqrt{5}}\times\frac{\pi}{2},\\
        &=\frac{\pi}{\sqrt{5}}.
    \end{align}

    Donc 
    \begin{equation}
        \boxed{
            \lim\limits_{n\to+\infty}\int_{0}^{2\pi}\frac{f(t)}{3+2\cos(nt)}=\frac{1}{\sqrt{5}}\int_{0}^{2\pi}f.
        }
    \end{equation}
\end{proof}

\begin{remark}
    Pour calculer $I=\int_{0}^{2\pi}\frac{\mathrm{d}t}{3+2\cos(t)}$, on peut écrire
    \begin{equation}
        \frac{1}{3+2\cos(t)}=\frac{1}{3+\e^{\i t}+\e^{-\i t}}=\frac{\e^{\i t}}{\e^{2\i t}+3\e^{\i t}+1}.
    \end{equation}
    On décompose $F(X)=\frac{X}{X^{2}+3X+1}=\frac{\alpha}{X-\lambda}+\frac{\beta}{X-\mu}$ avec $\lambda=\frac{-3+\sqrt{5}}{2}\in]-1,0[$, $\mu=\frac{-3-\sqrt{5}}{2}\in]-\infty,-1[$, $\alpha=\frac{\lambda}{\lambda-\mu}$, $\beta=\frac{\mu}{\mu-\lambda}$ avec $\lambda-\mu=\sqrt{5}$ et $\lambda=\frac{1}{\mu}$. Ainsi, 
    \begin{align}
        \frac{1}{3+2\cos(t)}
        &=\frac{\lambda}{\sqrt{5}}\frac{1}{\e^{\i t}-\lambda}-\frac{\mu}{\sqrt{5}}\frac{1}{\e^{\i t}-\mu},\\
        &=\frac{\lambda}{\sqrt{5}}\frac{\e^{-\i t}}{1-\lambda\e^{-\i t}}+\frac{1}{\sqrt{5}}\frac{1}{1-\frac{\e^{\i t}}{\mu}},\\
        &=\frac{1}{\sqrt{5}}\sum_{n\in\Z}\lambda^{n}\e^{\i nt},
    \end{align}
    car $\left\lvert \lambda\e^{-\i t}\right\rvert<1$ et $\left\lvert \frac{\e^{\i t}}{\mu}\right\rvert<1$. Comme on a $\left\lvert \lambda^{n}\e^{\i nt}\right\rvert\leqslant\left\lvert\lambda\right\rvert^{n}$, on a convergence normale sur $[0,2\pi]$ car $\left\lvert\lambda\right\rvert<1$. Ainsi,
    \begin{equation}
        \int_{0}^{2\pi}\frac{\mathrm{d}t}{3+2\cos(nt)}=\frac{1}{\sqrt{5}}\sum_{n\in\Z}\lambda^{n}\int_{0}^{2\pi}\e^{\i nt}\mathrm{d}t=\frac{2\pi}{\sqrt{5}}.
    \end{equation}
\end{remark}

\begin{proof}
    \phantom{}
    \begin{enumerate}
        \item Si $f$ ne tend pas vers 0 en $+\infty$, il existe $\varepsilon_{0}>0$ tel que pour tout $A>0$, il existe $x_{A}\geqslant A$ tel que $\left\lvert f(x_{A})\right\rvert>\varepsilon_{0}$. On sait qu'il existe $\alpha_{0}>0$ tel que pour tout $(x_{1},x_{2})\in\left(\R_{+}\right)^{2}$, si $\left\lvert x_{1}-x_{2}\right\rvert\leqslant\alpha_{0}$ alors $\left\lvert f(x_{1})-f(x_{2})\right\rvert\leqslant\frac{\varepsilon_{0}}{2}$. Alors pour tout $A\geqslant0$, pour tout $x\in[x_{A}-\alpha_{0},x_{A}+\alpha_{0}]$, on a $\left\lvert f(x)-f(x_{A})\right\rvert\leqslant\frac{\varepsilon_{0}}{2}$. Donc $f(x)$ est du signe de $f(x_{A})$ et $\left\lvert f(x)\right\rvert>\frac{\varepsilon_{0}}{2}$. Alors on a 
        \begin{equation}
            \left\lvert\int_{x_{A}-\alpha_{0}}^{x_{A}+\alpha_{0}}f(x)\mathrm{d}x\right\rvert=\int_{x_{A}-\alpha_{0}}^{x_{A}+\alpha_{0}}\left\lvert f(x)\right\rvert\d x>\varepsilon_{0}\alpha_{0}>0.
        \end{equation}
        Or 
        \begin{equation}
            \left\lvert \int_{x_{A}-\alpha_{0}}^{x_{A}+\alpha_{0}}f(x)\d x\right\rvert = \left\lvert \int_{x_{A}-\alpha_{0}}^{+\infty}f(x)\d x-\int_{x_{A}+\alpha_{0}}^{+\infty}f(x)\d x\right\rvert\xrightarrow[A\to+\infty]{}0.
        \end{equation}
        C'est absurde, donc
        \begin{equation}
            \boxed{
                \lim\limits_{x\to+\infty}f(x)=0.
            }
        \end{equation}

        \item Il existe $x_{0}>0$ tel que pour tout $x>x_{0}$, on ait $\left\lvert f(x)\right\rvert<1$. Donc pour tout $x>x_{0}$, on a $\left\lvert f^{2}(x_{0})\right\rvert\leqslant\left\lvert f(x)\right\rvert$ d'où $f^{2}=\underset{+\infty}{O}(f)$ et $f^{2}$ est intégrable.
    \end{enumerate}
\end{proof}

\begin{remark}
    Si $f$ est à valeurs dans $\C$, alors il faut raisonner sur $\Im(f)$ et $\Re(f)$ et le résultat reste vrai.
\end{remark}

\begin{proof}
    \phantom{}
    \begin{enumerate}
        \item Si $x=0$, $f_n(0)=\frac{n}{\sqrt{\pi}}\xrightarrow[n\to+\infty]{}+\infty$. Si $x\neq0$, alors $f_n$ converge simplement vers 0. On n'a pas convergence uniforme sur $\R_{+}^{*}$ car on pourrait intervertir les limites en 0.
        Soit $a>0$, soit $x\in[a,+\infty[$. $f$ étant décroissante sur $\R_{+}^{*}$, on a $\left\lvert f_n(x)\right\rvert\leqslant\frac{n}{\sqrt{\pi}}\e^{-n^{2}a^{2}}\xrightarrow[n\to+\infty]{}0$. On a donc convergence uniforme sur $[a,+\infty[$.

        Notons que $f_n$ est intégrable sur $\R$ et que son intégrable vaut 1. Enfin, pour tout $a>0$, on a $\int_{0}^{+\infty}f_n(x)\d x=\int_{n_0}^{+\infty}\frac{1}{\sqrt{\pi}}\e^{-u^{2}}\d u\xrightarrow[n\to+\infty]{}0$ (reste d'intégrale convergente).

        \item Notons $g_n(u)=\frac{g\left(\frac{u}{n}\right)}{\sqrt{\pi}}\e^{-u^{2}}$ de telle sorte que $\int_{-\infty}^{+\infty}g(t)\frac{n}{\sqrt{\pi}}\e^{-n^{2}t^{2}}\d t=\int_{-\infty}^{+\infty}g_n(u)\d u$.
        
        Soit $u$ fixé dans $\R$, on a $\lim\limits_{n\to+\infty}g_n(u)=\frac{g(0)}{\sqrt{\pi}}\e^{-u^{2}}$ par continuité de $g$, et pour tout $n\geqslant1$, pour tout $u\in\R$, on a $\left\lvert g_n(u)\right\rvert\leqslant\frac{\left\lVert g\right\rVert_{\infty}}{\sqrt{5}}\e^{-u^{2}}$ intégrable sur $\R$. D'après le théorème de converge dominée, on peut intervertir limite et intégrale, donc 
        \begin{equation}
            \boxed{
                \lim\limits_{n\to+\infty}\int_{\R}g(t)f_n(t)\d t=g(0)
            }
        \end{equation}
    \end{enumerate}
\end{proof}

\begin{remark}
    Généralement, pour tout $x\in\R$, $(f_n\star g)(x) = \int_{\R}g(x-t)f_n(t)\d t\xrightarrow[n\to+\infty]{}g(x)$ par théorème de convergence dominée.
\end{remark}

\begin{remark}
    Si $g$ est bornée et uniformément continue sur $\R$, soit $\varepsilon>0$ et $\alpha>0$ tel que si $\left\lvert t\right\rvert\leqslant\alpha$ alors pour tout $x\in\R$, $\left\lvert g(x-t)-g(x)\right\rvert\leqslant\frac{\varepsilon}{2}$. Alors 
    \begin{equation}
        \left\lvert (f_n\star g)(x)-g(x)\right\rvert\leqslant\int_{-\alpha}^{\alpha}\left\lvert g(x-t)-g(x)\right\rvert f_n(t)\d t+\int_{\R\setminus[-\alpha, \alpha]}2\left\lVert g\right\rVert_{\infty}f_n(t)\d t.
    \end{equation}
    Le deuxième terme tend vers 0 quand $n\to+\infty$, donc $(f_n\star g)_{n\in\N}$ converge uniformément vers $g$ sur $\R$.
\end{remark}

\begin{remark}
    Soit $f\colon\R\to\R_{+}$ continue par morceaux telle que $\int_{\R}f=1$. Soit pour $n\geqslant1$, \function{f_n}{\R}{\R^{+}}{t}{nf(nt)}
    Par changement de variable, on a $\int_{\R}f_n=1$ et $\lim\limits_{n\to+\infty}\int_{\alpha}^{+\infty}f=0$ pour $\alpha>0$. $(f_n)_{n\in\N}$ est une approximation de l'unité.
\end{remark}

\begin{proof}
    Si $x\geqslant2$, on a $\frac{1}{x}\in]0,1]$ donc on peut définir \function{f}{[1,+\infty[}{\R}{x}{\frac{1}{x}-\arcsin\left(\frac{1}{x}\right)}
    $f$ est continue et $\arcsin(t)\underset{t\to0}{=}t+\frac{t^{3}}{6}+o(t^{3})$ implique $f(x)\underset{x\to+\infty}{\sim}\frac{-1}{6x^{3}}$, donc d'après le critère de Riemann, $\int_{1}^{+\infty}f$ converge.

    Soit $A\geqslant1$, on pose $I_{A}=\int_{1}^{A}\frac{1}{x}-\arcsin\left(\frac{1}{x}\right)\d x=\ln(A)-\int_{1}^{A}\arcsin\left(\frac{1}{x}\right)\d x$. On a 
    \begin{align}
        \int_{1}^{A}\arcsin\left(\frac{1}{x}\right)\d x
        &=[x\arcsin\left(\frac{1}{x}\right)]_{1}^{A}+\int_{1}^{A}\frac{1}{\sqrt{x^{2}-1}}\d x,\\
        &=\arcsin\left(\frac{1}{A}\right)+\ln(A+\sqrt{A^{2}-1})-\frac{\pi}{2},\\
        &\underset{A\to+\infty}{=}1+\ln(A)+\ln(2)-\frac{\pi}{2}+o(1),
    \end{align}
    donc 
    \begin{equation}
        \boxed{
            I=\lim\limits_{A\to+\infty}=-1+\frac{\pi}{2}-\ln(2).
        }
    \end{equation}
\end{proof}

\begin{proof}
    On a $\ln(\sin(t))\underset{t\to0}{\sim}\ln(t)\underset{t\to0}{=}O\left(\frac{1}{\sqrt{t}}\right)$ donc $I$ existe, et en posant $u=\frac{\pi}{2}-t$, on a $I=J$. On a 
    \begin{align}
        I+J
        &=\int_{0}^{\frac{\pi}{2}}\ln\left(\frac{\sin(2t)}{2}\right)\d t,\\
        &=\int_{0}^{\frac{\pi}{2}}\ln(\sin(2t))\d t-\int_{0}^{\frac{\pi}{2}}\ln(2)\d t,\\
        &=\frac{1}{2}\int_{0}^{\pi}\ln(\sin(u))\d u-\frac{\pi}{2}\ln(2),\\
        &=I+\int_{\frac{\pi}{2}}^{\pi}\ln(\sin(u))\d u-\frac{\pi}{2}\ln(2).
    \end{align}
    On a 
    \begin{equation}
        I+\int_{\frac{\pi}{2}}^{\pi}\ln(\sin(u))\d u=\frac{1}{2}I+\frac{1}{2}\int_{0}^{\frac{\pi}{2}}\ln(\sin(v))\d v=I.
    \end{equation}
    Finalement, on a $I+J=I-\frac{\pi}{2}\ln(2)$ donc 
    \begin{equation}
        \boxed{
            I=J=-\frac{\pi}{2}\ln(2).
        }
    \end{equation}
\end{proof}

\begin{proof}
    $f_\alpha$ est positive, continue et $f_\alpha\leqslant1$. $f_\alpha$ est intégrable si et seulement si $\sum u_k$ converge avec 
        \begin{align}
            u_k
            &= \int_{k\pi-\frac{\pi}{2}}^{k\pi+\frac{\pi}{2}}\frac{\d x}{1+x^{\alpha}\left\lvert\sin(x)\right\rvert},\\
            &=\int_{-\frac{\pi}{2}}^{\frac{\pi}{2}}\frac{\d t}{1+(t+k\pi)^{\alpha}\left\lvert\sin(t)\right\rvert},\\
            &\leqslant\int_{-\frac{\pi}{2}}^{\frac{\pi}{2}}\frac{\d t}{1+\left(k\pi-\frac{\pi}{2}\right)^{\alpha}\frac{2}{\pi}\left\lvert t\right\rvert},\\
            &=2\int_{0}^{\frac{\pi}{2}}\frac{\d t}{1+\left(k\pi-\frac{\pi}{2}\right)^{\alpha}\frac{2}{\pi}t},\\
            &=\frac{\pi}{\left(k\pi-\frac{\pi}{2}\right)^{\alpha}}\ln\left(1+\left(k\pi-\frac{\pi}{2}\right)^{\alpha}\right),\\
            &\underset{k\to+\infty}{\sim}\frac{\alpha\ln(k)}{(k\pi)^{\alpha}}\underset{k\to+\infty}{=}O\left(\frac{1}{k^{\frac{1+\alpha}{2}}}\right).
        \end{align}
        Donc $\sum u_k$ converge et $\int_{0}^{+\infty}f_\alpha(t)\d t$ converge.
\end{proof}

\begin{proof}
    \phantom{}
    \begin{enumerate}
        \item Pour tout $P\in\R[X]$, $\int_{a}^{b}P(t)f(t)\d t=0$. D'après le théorème de Weierstrass, il existe $(P_n)\in\left(\R[X]\right)^{\N}$ telle que $\left\lVert P_n-f\right\rVert_{\infty}\xrightarrow[n\to+\infty]{}0$ sur $[a,b]$. $(P_nf)_{n\in\N}$ converge uniformément vers $f^{2}$ sur $[a,b]$, donc pour tout $n\in\N$, $\int_{a}^{b}P_nf=0$ donne par convergence uniforme $\int_{a}^{b}f^{2}=0$. Comme $f^{2}$ est continue positive, on a $f^{2}=0$ donc $f=0$.
        
        \item Pour tout $n\geqslant1$, on a par intégration par parties,
        \begin{equation}
            I_n=\frac{n}{1-\i}I_{n-1}.
        \end{equation}
        On a $I_{0}=\frac{1}{1-\i}$. Par récurrence, on a 
        \begin{equation}
            I_n=\frac{n!}{(1-\i)^{n+1}}=\frac{n!}{\left(\sqrt{2}\right)^{n+1}}\e^{\i(n+1)\frac{\pi}{4}}.
        \end{equation}
        
        \item Pour $n=4k-1$ pour $k\in\N^{*}$, on a 
        \begin{equation}
            \Im(I_{4k-1})=0=\int_{0}^{+\infty}t^{4k-1}\sin(t)\e^{-t}\d t.
        \end{equation}
        On pose $u=t^{4}$, $t=u^{\frac{1}{4}}$ et $\d t=\frac{1}{4}u^{-\frac{3}{4}}\d u$. Ainsi, en posant $f(u)=\sin\left(u^{\frac{1}{4}}\right)\e^{-u^{\frac{1}{4}}}$,
        \begin{equation}
            \boxed{
                0=\int_{0}^{+\infty}f(u)u^{k-1}\d u.
            }
        \end{equation}
    \end{enumerate}
\end{proof}

\begin{proof}
    \phantom{}
    \begin{enumerate}
        \item $g$ est continue, $\mathcal{C}^{1}$ par morceaux et en tout point de continuité de $f$, on a $g'(t)=\e^{-at}f(t)$. On a $g(0)=0$ et $\lim\limits_{t\to+\infty}g(t)=\mathcal{L}f(a)$. 
        
        Soit $X\geqslant0$, on a grâce à une intégration par parties,
        \begin{align}
            \int_{0}^{X}\e^{-bt}f(t)\d t
            &=\int_{0}^{X}\e^{-(b-a)t}\e^{-at}f(t)\d t,\\
            &=\left[g(t)\e^{-(b-a)t}\right]_{0}^{X}+(b-a)\int_{0}^{X}\e^{-(b-a)t}g(t)\d t.
        \end{align}
        Le terme entre crochet s'annule car $g(0)=0$, et $b>a$ donc $g(X)\xrightarrow[X\to+\infty]{}\mathcal{L}f(a)$. $g$ est continue, admet une limite finie en $+\infty$, donc est bornée sur $\R_{+}$. Ainsi,
        \begin{equation}
            \left\lvert\e^{-(b-a)t}g(t)\right\rvert\leqslant\left\lVert g\right\rVert_{\infty}\e^{-(b-a)t},
        \end{equation}
        qui est intégrable sur $\R_{+}$. Finalement, 
        \begin{equation}
            \int_{0}^{+\infty}\e^{-(b-a)}g(t)\d t,    
        \end{equation}
        converge absolument et $\int_{0}^{+\infty}\e^{-bt}f(t)\d t$ converge et 
        \begin{equation}
            \boxed{
                \mathcal{L}f(b)=(b-a)\int_{0}^{+\infty}\e^{-(b-a)t}g(t)\d t.
            }
        \end{equation}

        \item En raisonnant sur $f-h$, on se ramène à $\mathcal{L}f=0$. Pour tout $b>a$, $\int_{0}^{+\infty}\e^{-(b-a)t}g(t)\d t=0$, donc pour tout $x>0$, $\int_{0}^{+\infty}\e^{-xt}g(t)\d t=0$. Si $g=0$, alors en dérivant, on a $f=0$. On pose $u=\e^{-t}$ qui est un $\mathcal{C}^{1}$ difféomorphisme de $[0,+\infty[\to]0,1]$. On a donc, pour tout $x>0$, $\int_{0}^{1}u^{x-1}g\left(-\ln(u)\right)\d u=0$. Donc pour tout $n\in\N$, $\int_{0}^{1}u^{n}g\left(-\ln(u)\right)\d u=0=\int_{0}^{1}u^{n}k(u)\d u$ avec $k\colon[0,1]\to\R$ définie par $k(0)=\mathcal{L}f(a)$ et $k(x)=g\left(-\ln(x)\right)$ si $x\in]0,1]$. $k$ est continue sur $[0,1]$. D'après le théorème de Weierstrass, $k=0$ donc $g=0$ puis $f=0$.
    \end{enumerate}
\end{proof}

\begin{proof}
    \phantom{}
    \begin{enumerate}
        \item Pour tout $(x,t)\in\R^{2}$, $\left\lvert\e^{\i xt}f(t)\right\rvert=\left\lvert f(t)\right\rvert$ est intégrable sur $\R$ car $f(t)\underset{\left\lvert t\right\rvert\to+\infty}{\sim}0$. $\widehat{f}$ est définie et $\widehat{f}(x)=\int_{-A}^{A}\e^{\i tx}f(t)\d t$.
        
        Posons \function{g}{\R^{2}}{\C}{(x,t)}{\e^{\i tx}f(t)}
        Pour tout $t\in\R$, $x\mapsto g(x,t)$ est $\mathcal{C}^{\infty}$ et pour tout $k\in\N$, $\frac{\partial^{k}g}{\partial x^{k}}=(\i t)^{k}g(x,t)$. On a 
        \begin{equation}
            \left\lvert\frac{\partial^{k}g(x,t)}{\partial x^{k}}\right\rvert=\left\lvert t\right\rvert^{k}\left\lvert f(t)\right\rvert,
        \end{equation}
        majoration indépendante de $x$ et intégrable sur $[-A,A]$. Donc $\widehat{f}$ est $\mathcal{C}^{\infty}$ et pour tout $k\in\N$, pour tout $x\in\R$, $\widehat{f}^{(k)}(x)=\int_{-A}^{A}(\i t)^{k}\e^{\i tx}f(t)\d t$.

        \item Soit $B>0$ tel que si $x>B$, $\widehat{f}(x)=0$. Soit $x_{0}=B+1$, alors $\widehat{f}=0$ sur $]x_{0}-1,+\infty[$. Pour tout $k\in\N$, on a $\widehat{f}^{(k)}(x_{0})=0=\int_{-A}^{A}t^{k}\e^{\i tx_{0}}f(t)\d t$. D'après le théorème de Weierstrass, on a $f(t)=0$ pour tout $t\in[-A,A]$.
    \end{enumerate}
\end{proof}

\begin{proof}
    Si $f$ est affine avec $f(x)=\alpha x+\beta$. On a $\int_{a}^{b}f(t)\d t=\alpha\frac{(b^{2}-a^{2})}{2}+\beta(b-a)$ et $(b-a)f\left(\frac{a+b}{2}\right)=(b-a)\left(\alpha\frac{(a+b)}{2}+\beta\right)$ d'où $(b-a)f\left(\frac{a+b}{2}\right)=\int_{a}^{b}f(t)\d t$.

    Notons que l'inégalité de l'énoncé équivaut à pour tout $x\in\mathring{I}$, pour tout $h>0$, on a $a=x-h$ et $b=x+h\in I^{2}$, $2hf(x)\leqslant\int_{x-h}^{x+h}f(t)\d t$.

    Si $f$ est convexe, soit $a<b\in I^{2}$. Soit $\varphi$ affine sur $[a,b]$ telle que $\varphi\left(\frac{a+b}{2}\right)=f\left(\frac{a+b}{2}\right)$. On a 
    \begin{equation}
        \varphi'=\lambda=\frac{1}{2}\left(f'_{g}\left(\frac{a+b}{2}\right)+f'_{d}\left(\frac{a+b}{2}\right)\right)\geqslant f'_{g}\left(\frac{a+b}{2}\right),
    \end{equation}
    par convexité et en notant $\varphi(t)=\lambda t+\mu$. En notant $\varphi_{1}$ la demi-tangente à $f$ en $\frac{a+b}{2}$, on a pour tout $t\in\left[a,\frac{a+b}{2}\right]$,
    \begin{equation}
        \varphi(t)\leqslant\varphi_1(t)\leqslant f(t).
    \end{equation}
    $\varphi_{1}$ est affine sur $\left[a,\frac{a+b}{2}\right]$, $\varphi_{1}\left(\frac{a+b}{2}\right)=f\left(\frac{a+b}{2}\right)$ et $\varphi_{1}'\left(\frac{a+b}{2}\right)=f'_{g}\left(\frac{a+b}{2}\right)$. De la même façon, pour tout $t\in\left[\frac{a+b}{2},b\right]$, on a 
    \begin{equation}
        \varphi(t)\leqslant\varphi_2(t)\leqslant f(t),
    \end{equation}
    avec $\varphi_{2}$ affine sur $\left[\frac{a+b}{2},b\right]$, $\varphi_{2}\left(\frac{a+b}{2}\right)=f\left(\frac{a+b}{2}\right)$ et $\varphi'_{2}\left(\frac{a+b}{2}\right)=f'_{d}\left(\frac{a+b}{2}\right)$. 
    
    On a donc $\int_{a}^{b}\varphi(t)\d t=(b-a)f\left(\frac{a+b}{2}\right)\leqslant\int_{a}^{b}f$.

    Réciproquement, si pour tout $a<b$, $(b-a)f\left(\frac{a+b}{2}\right)\leqslant\int_{a}^{b}f(t)\d t$, soient $x<y\in I^{2}$ fixés. On pose $g=f-\varphi$ avec $\varphi(z)=\frac{f(y)-f(x)}{y-x}(z-x)+f(x)$. $g$ vérifie l'inégalité de l'énoncé car pour $\varphi$ on a égalité (car affine). On veut montrer que $g\leqslant0$ sur $[x,y]$. On a $g(x)=g(y)=0$. Soit $g(x_{0})=\max\limits_{t\in[x,y]}g(t)$. Si $g(x_{0})>0$, on a $x_{0}\in]x,y[$ car $g(x)=g(y)=0$. Soit $h>0$ tel que $x_{0}-h$ et $x_{0}+h\in[x,y]$. On applique l'inégalité de l'énoncé à $g$:
    \begin{equation}
        2hg(x_{0})\leqslant\int_{x_{0}-h}^{x_{0}+h}g(t)\d t=2g(x_{0}),
    \end{equation}
    donc 
    \begin{equation}
        \int_{x_{0}-h}^{x_{0}+h}(g(x_{0})-g(t))\d t=0,
    \end{equation}
    et l'intégrande est positive et continue. Donc pour tout $t\in[x_{0}-h,x_{0}+h]$, on a $g(t)=g(x_{0})$. On pose $h=\min(y-x_{0},x-x_{0})$. On obtient $g(x)=0=g(x_{0})>0$ (ou $g(y)=g(x_{0})$) ce qui est absurde. Donc $g\leqslant0$ sur $[x,y]$ et $f$ est convexe.
\end{proof}

\begin{remark}
    Notons que si pour tout $(h,x)\in\R_{+}\times\mathring{I}$ tels que $(x-h,x+h)\in I^{2}$, $2hf(x)=\int_{x-h}^{x+h}f(t)\d t$, alors pour $x\in\mathring{I}$ et $h$ fixé, $x\mapsto\int_{x-h}^{x+h}f$ est $\mathcal{C}^{1}$ donc $f$ l'est. Par récurrence, $f$ est $\mathcal{C}^{\infty}$, et en dérivant deux fois par rapport à $h$ (pour $x\in \mathring{I}$ fixé), on a 
    $0=f'(x+h)-f'(x-h)$ donc $f$ est affine.
\end{remark}

\begin{proof}
    \phantom{}
    \begin{enumerate}
        \item Soit $f_n\colon]0,+\infty[\to\R$ définie par $f_n(t)=0$ si $t>n$ et $f_n(t)=\left(1-\frac{t}{n}\right)^{n}t^{x-1}$ si $t\leqslant n$. $f_n$ est continue par morceaux, positive, intégrable sur $\R_{+}$ car équivalente à 0 en $+\infty$ et à $t^{x-1}$ en 0.
        
        Soit $t\in]0,+\infty[$ fixé, il existe $N_{0}\in\N$ tel que pour tout $n\geqslant N_{0}$, $n>t$ et pour tout $n\geqslant N_{0}$, 
        \begin{align}
            f_n(t)
            &=\left(1-\frac{t}{n}\right)^{n}t^{x-1},\\
            &=\e^{n\ln\left(1-\frac{t}{n}\right)}t^{x-1},\\
            &\underset{n\to+\infty}{=}\e^{n\left(-\frac{t}{n}+o\left(\frac{1}{n}\right)\right)}t^{x-1},\\
            &\xrightarrow[n\to+\infty]{}\e^{-t}t^{x-1}.
        \end{align}

        On a donc convergence simple vers $f(t)=\e^{-t}t^{x-1}$, fonction continue sur $\R_{+}^{*}$ intégrable sur $\R_{+}^{*}$. Soit $n\geqslant1$ et $t\in[0,n[$, on a 
        \begin{equation}
            0\leqslant f_n(t)\leqslant f(t),
        \end{equation}
        car $\ln(1+x)\leqslant x$ pour tout $x>-1$.
        D'après le théorème de convergence dominée, on a bien
        \begin{equation}
            \boxed{
                \lim\limits_{n\to+\infty}I_n=\lim\limits_{n\to+\infty}\int_{0}^{+\infty}f_n(t)\d t=\int_{0}^{+\infty}f(t)\d t.
            }
        \end{equation}

        \item On pose $u=\frac{t}{n}$ et on a 
        \begin{align}
            I_n(x)
            &= \int_{0}^{1}(1-u)^{n}(nu)^{x-1}n\d u,\\
            &=n^{x}\int_{0}^{1}(1-u)^{n}u^{x-1}\d u,\\
            &=n^{x}\left(\left[(1-u)^{n}\frac{u^{x}}{x}\right]_{0}^{1}+\int_{0}^{1}n(1-u)^{n-1}\frac{u^{x}}{x}\d u\right).
        \end{align}
        Le terme entre crochets est nul car $u\geqslant1$ et $x>0$.

        Si on pose $B_n(x)=\int_{0}^{1}(1-u)^{n}u^{x-1}\d u$, on a 
        \begin{equation}
            B_n(x)=\frac{n}{x}B_{n-1}(x+1)=\dots=\frac{n!}{x(x+1)\dots(x+n-1)}B_{0}(x+n)=\frac{n!}{x(x+1)\dots(x+n)}.
        \end{equation}
        On a donc 
        \begin{equation}
            \boxed{
                \Gamma(x)=\lim\limits_{n\to+\infty}\frac{n!n^{x}}{x(x+1)\dots(x+n).}
            }
        \end{equation}

        \item Par définition, on a $\gamma=\lim\limits_{n\to+\infty}\left(1+\frac{1}{2}+\dots+\frac{1}{n}-\ln(n)\right)$. On a 
        \begin{align}
            \frac{x(x+1)\dots(x+n)}{1\times2\times\dots\times n}\e^{-x\ln(n)}
            &\underset{n\to+\infty}{=} x\left[\prod_{k=1}^{n}\left(1+\frac{x}{k}\right)\right]\e^{-x\left(1+\frac{1}{2}+\dots+\frac{1}{n}+o(1)\right)}\e^{\gamma x},\\
            &\underset{n\to+\infty}{=} x\e^{\gamma x}\prod_{k=1}^{n}\left(1+\frac{x}{k}\right)\e^{-\frac{x}{k}}\times\underbrace{\e^{o(1)}}_{\xrightarrow[n\to+\infty]{}1}.
        \end{align}
        Donc on a 
        \begin{equation}
            \frac{1}{\Gamma(x)}=x\e^{\gamma x}\prod_{k=1}^{+\infty}\left(1+\frac{x}{k}\right)\e^{-\frac{x}{k}}.
        \end{equation}

        \item On remarque que $\frac{\Gamma'(x)}{\Gamma(x)}=(\ln\Gamma)'(x)$. On a 
        \begin{equation}
            \ln\Gamma(x)=-\ln(x)-\gamma x+\sum_{k=1}^{+\infty}\underbrace{\left(\frac{x}{k}-\ln\left(1+\frac{x}{k}\right)\right)}_{f_k(x)\geqslant0}.
        \end{equation}
        On a convergence simple sur $\R_{+}^{*}$. On a pour tout $k>1$, $f_{k}$ est $\mathcal{C}^{1}$, et pour tout $k\geqslant1$, pour tout $x\in]0,+\infty[$, $f'_k(x)=\frac{1}{k}-\frac{1}{k+x}\geqslant0$ car $x>0$.

        Soit $A>0$, pour tout $x\in]0,A]$, on a 
        \begin{equation}
            0<f_k(x)\leqslant\frac{1}{k}-\frac{1}{k+A}=\frac{A}{k(k+A)}\underset{k\to+\infty}{O}\left(\frac{1}{k^{2}}\right).
        \end{equation}
        Donc pour tout $A>0$, $\sum f_k(x)$ converge normalement sur $]0,A]$. $\ln\Gamma$ est donc $\mathcal{C}^{1}$ (en tant que série de fonction) et on a 
        \begin{equation}
            \boxed{
            \left(\frac{\Gamma'}{\Gamma}\right)(x)=-\frac{1}{x}-\gamma+\sum_{k=1}^{+\infty}\left(\frac{1}{k}-\frac{1}{k+x}\right).}
        \end{equation}
    \end{enumerate}
\end{proof}

\begin{remark}
    En particulier, comme $\Gamma(1)=1$, on a 
        \begin{equation}
            \Gamma'(1)=\left(\frac{\Gamma'}{\Gamma}\right)(1)=-1-\gamma+\sum_{k=1}^{+\infty}\left(\frac{1}{k}-\frac{1}{k+1}\right)=-\gamma,
        \end{equation}
    car la série est téléscopique.
\end{remark}

\begin{remark}
    On a 
    \begin{equation}
        \left(\ln\Gamma\right)''(x)=\frac{\Gamma''\Gamma-\Gamma'^{2}}{\Gamma^{2}}.
    \end{equation}
    Par ailleurs,
    \begin{equation}
        0\leqslant\Gamma'^{2}(x)=\left(\int_{0}^{+\infty}\ln(t)t^{x-1}\e^{-t}\d t\right)^{2}\leqslant\int_{0}^{+\infty}\ln^{2}(t)t^{x-1}\e^{-t}\d t\times\int_{0}^{+\infty}t^{x-1}\e^{-t}\d t=\Gamma''(x)\Gamma(x),
    \end{equation}
    d'après l'inégalité de Cauchy-Schwarz. Ona égalité stricte car $\ln(t)$ n'est pas constante. Ainsi, $\ln\Gamma$ est strictement convexe.
\end{remark}

\begin{remark}
    On peut vérifier que $\ln\Gamma$ est l'unique fonction de $\R_{+}^{*}\to\R$ telle que 
    \begin{enumerate}
        \item $\ln\Gamma$ est convexe,
        \item $\forall x>0$, $\left(\ln\Gamma\right)(x+1)=\left(\ln\Gamma\right)(x)+\ln(x)$,
        \item $\left(\ln\Gamma\right)(1)=0$.
    \end{enumerate}
\end{remark}

\begin{proof}
    \phantom{}
    \begin{enumerate}
        \item On a 
        \begin{equation}
            \e^{2\i\pi d(f)}=\exp\left(\int_{0}^{2\pi}\frac{f'(t)}{f(t)}\d t\right).
        \end{equation}
        Posons $g(x)=\exp\left(\int_{0}^{x}\frac{f'(t)}{f(t)}\d t\right)$. $g$ est $\mathcal{C}^{1}$ sur $\R$. Pour tout $x\in\R$, on a $f'(x)=\frac{f'(x)}{f(x)}g(x)$. On a $\left(\frac{g}{f}\right)'=\frac{g'f-f'g}{g^{2}}=0$ donc $\frac{g}{f}=\alpha\in\C$. En particulier, $g(0)=g(2\pi)=1$ donc $d(f)\in\Z$.

        \item $f_0$ est constante égale à $P(0)$ donc $d(f_0)=0$ car c'est une fonction constante.
        Soit $r\geqslant0$, on a 
        \begin{equation}
            d(f_r)=\frac{1}{2\i\pi}\int_{0}^{2\pi}\frac{f_r'(t)}{f_r(t)}\d t=\frac{1}{2\i\pi}\int_{0}^{2\pi}\frac{\i r\e^{\i t}P'(r\e^{\i t})}{P(r\e^{\i t})}\d t.
        \end{equation}
        On note $g(r,t)$ la fonction intégrande définit sur $\R_{+}\times[0,2\pi]\to\C$. $r\mapsto g(r,t)$ est continue sur $\R_{+}$, et soit $z\in\C$, on a 
        \begin{equation}
            P(z)=a_{0}+a_{1}z+\dots+a_nz^{n}.
        \end{equation}
        Alors 
        \begin{equation}
            h(z)=\frac{zP'(z)}{P(z)}=\frac{a_1 z+\dots+na_nz^{n}}{a_0+a_1z+\dots+a_nz^{n}}
        \end{equation}
        est continue sur $\C$ et pour $z\neq0$, on a 
        \begin{equation}
            \frac{z P'(z)}{P(z)}=\frac{\frac{a_1}{z^{n-1}}+\dots+na_n}{\frac{a_0}{z^{n}}+a_n}\xrightarrow[\left\lvert z\right\rvert\to+\infty]{}n.
        \end{equation}
        Donc $h$ est bornée sur $\C$, soit $M=\left\lVert h\right\rVert_{\infty}$. On a $\left\lvert g(r,t)\right\rvert\leqslant M\in L^{1}\left([0,2\pi]\right)$. Donc $r\mapsto d(f_r)$ est continue et pour $t$ fixé, on a $\lim\limits_{r\to+\infty}g(r,t)=n$. Par convergence dominée, on a $\lim\limits_{r\to+\infty}d(f_r)=n$. $r\mapsto d(f_r)$ est continue à valeurs dans $\Z$ donc constante et $d(f_{0})=0$, $\lim\limits_{r\to+\infty}d(f_r)=n\neq0$: c'est absurde. Donc $P$ s'annule.
    \end{enumerate}
\end{proof}

\begin{remark}
    Le théorème de relèvement permet d'écrire $f(t)=\rho(t)\e^{\i\theta(t)}$ avec $\rho(t)=\left\lvert f(t)\right\rvert$ et $(\rho,\theta)\colon\R\to\left(\R_{+}^{*},\R\right)$ est $\mathcal{C}^{1}$. On a alors 
    \begin{equation}
        \int_{0}^{2\pi}\frac{f'}{f}=\int_{0}^{2\pi}\frac{\rho'}{\rho}+\i\left(\theta(2\pi)-\theta(0)\right).
    \end{equation}
    Le premier terme vaut $\left[\ln(\rho)\right]_{0}^{2\pi}=0$ car $\rho=\left\lvert f\right\rvert$ est $2\pi$-périodique, et le deuxième terme vaut $2\i\pi\times$ le nombre de tours que décrit $f$ autour de l'origine.
\end{remark}

\begin{proof}
    En appliquant l'inégalité de Taylor avec reste intégral à $f$ de classe $\mathcal{C}^{n}$, on a 
    \begin{equation}
        R_n=f(b)-f(a)-\sum_{k=1}^{n-1}\frac{f^{(k)}(a)}{k!}=\int_{a}^{b}\frac{(b-t)^{n-1}}{(n-1)!}f^{(n)}(t)\d t.
    \end{equation}
    Soit $m_n=\min\limits_{[a,b]}f^{(n)}$ et $M_n=\max\limits_{[a,b]}f^{(n)}$. Alors 
    \begin{equation}
        m_n\frac{\left\lvert b-a\right\rvert^{n}}{n!}\leqslant\left\lvert\int_{a}^{b}\frac{(b-t)^{n-1}}{(n-1)!}f^{(n)}(t)\d t\right\rvert\leqslant M_n\frac{\left\lvert b-a\right\rvert^{n}}{n!}.
    \end{equation}
    D'après le théorème des valeurs intermédiaires, il existe $\xi\in]a,b[$ tel que $R_n=\frac{(b-a)^{n}}{n!}f^{(n)}(\xi)$.

    On a 
    \begin{equation}
        v_n-\int_{0}^{1}=\sum_{k=0}^{n-1}\frac{1}{2n}\left[f\left(\frac{k}{n}\right)+f\left(\frac{k+1}{n}\right)\right]-\int_{\frac{k}{n}}^{\frac{k+1}{n}}f.
    \end{equation}

    On prend d'abord $a=\frac{k}{n}$ et $b=\frac{k+1}{n}$, il existe $\xi_k\in\left]\frac{k}{n},\frac{k+1}{n}\right[$ tel que 
    \begin{equation}
        F\left(\frac{k+1}{n}\right)-F\left(\frac{k}{n}\right)=\frac{1}{n}f\left(\frac{k}{n}\right)+\frac{1}{2n^{2}}f'\left(\frac{k}{n}\right)+\frac{1}{6n^{3}}f^{(2)}(\xi_k).
    \end{equation}
    Puis avec $a=\frac{k+1}{n}$ et $b=\frac{k}{n}$, il existe $\eta_k\in\left]\frac{k}{n},\frac{k+1}{n}\right[$ tel que 
    \begin{equation}
        F\left(\frac{k}{n}\right)-F\left(\frac{k+1}{n}\right)=-\frac{1}{n}f\left(\frac{k+1}{n}\right)+\frac{1}{2n^{2}}f'\left(\frac{k+1}{n}\right)-\frac{1}{6n^{3}}f^{(2)}(\eta_k).
    \end{equation}

    En faisant la différence des deux égalités et en divisant par deux, on a 
    \begin{align}
        F\left(\frac{k+1}{n}\right)-F\left(\frac{k}{n}\right)
        &=\int_{\frac{k}{n}}^{\frac{k+1}{n}}f(t)\d t,\\
        &=\frac{1}{2n}\left(f\left(\frac{k+1}{n}\right)+f\left(\frac{k}{n}\right)\right)+\frac{1}{4n^{2}}\left(f'\left(\frac{k}{n}\right)-f'\left(\frac{k+1}{n}\right)\right),\\
        &\qquad+\frac{1}{12n^{3}}\left(f^{(2)(\xi_k)}+f^{(2)}(\eta_k)\right).
    \end{align}
    En sommant, on obtient (par le théorème de Riemann car on a une subdivision pointée)
    \begin{align}
        v_n-\int_{0}^{1}f
        &=\sum_{k=0}^{n-1}\frac{1}{4n^{2}}\left(f'\left(\frac{k+1}{n}\right)-f'\left(\frac{k}{n}\right)\right)-\frac{1}{12n^{3}}\left(f^{(2)}(\xi_k)+f^{(2)}(\eta_k)\right),\\
        &\underset{n\to+\infty}{=}\frac{1}{4n^{2}}\left[f'(1)-f'(0)\right]-\frac{1}{6n^{2}}\left[\int_{0}^{1}f'^{(2)}(t)\d t+o(1)\right],\\
        &\underset{n\to+\infty}{=}\frac{1}{4n^{2}}\left[f'(1)-f'(0)\right]-\frac{1}{6n^{2}}\left(f'(1)-f'(0)+o(1)\right).
    \end{align}

    Donc on a
    \begin{equation}
        \boxed{
            v_n=\int_{0}^{1}f+\frac{1}{12n^{2}}\left(f'(1)-f'(0)\right)+\underset{n\to+\infty}{o}\left(\frac{1}{n^{2}}\right).
        }
    \end{equation}
\end{proof}

\begin{proof}
    On a 
    \begin{align}
        I_n
        &=\int_{0}^{\frac{\pi}{4}}\left(\tan^{2}(x)+1-1\right)\tan^{n-2}(x)\d x,\\
        &=\int_{0}^{\frac{\pi}{4}}\left(\tan'(x)-1\right)\tan^{n-2}(x)\d x,\\
        &=\frac{1}{n-1}\left[\tan^{n-1}(x)\right]_{0}^{\frac{\pi}{4}}-I_{n-2}.
    \end{align}
    Donc
    \begin{equation}
        I_n+I_{n-2}=\frac{1}{n-1}.
    \end{equation}
    On a $I_0=\frac{\pi}{4}$ et $I_1=\left[-\ln\left\lvert\cos\right\rvert\right]_{0}^{\frac{\pi}{4}}=\frac{1}{2}\ln(2)$.

    On a donc 
    \begin{equation}
        I_{2p}=(-1)^{p}\left(\frac{\pi}{4}-1+\frac{1}{3}-\dots+\frac{(-1)^{p}}{2p+1}\right),
    \end{equation}
    et
    \begin{equation}
        I_{2p+1}=\frac{(-1)^{p}}{2}\left(\ln(2)-1+\frac{1}{2}-\frac{1}{3}+\dots+\frac{(-1)^{p}}{p}\right).
    \end{equation}

    D'après le théorème de convergence dominée, on a $I_n\xrightarrow[n\to+\infty]{}0$ donc 
    \begin{equation}
        \boxed{
            \frac{\pi}{4}=\sum_{k=0}^{+\infty}\frac{(-1)^{k}}{2k+1}.
        }
    \end{equation}

    Comme on a $2I_{n}\leqslant I_{n}+I_{n-2}=\frac{1}{n-1}\leqslant 2I_{n-2}$ d'où 
    \begin{equation}
        \frac{1}{2(n+1)}\leqslant I_n\leqslant\frac{1}{2(n-1)},
    \end{equation}
    ainsi 
    \begin{equation}
        \boxed{
            I_n\underset{n\to+\infty}{\sim}\frac{1}{2n}.
        }
    \end{equation}
\end{proof}

\begin{proof}
    On note $g$ la fonction intégrande. $g$ est $\mathcal{C}^{\infty}$ sur $]0,1]$. On a 
    \begin{equation}
        f(x)=\int_{\frac{1}{2}}^{x}g(t)\d t-\int_{\frac{1}{2}}^{x^{2}}g(t)\d t,
    \end{equation}
    donc $f$ est $\mathcal{C}^{\infty}$ et $f'(x)=g(x)-2xg'(x^{2})$.

    En 0, $\e^{t}$ se comporte comme 1 et $\frac{1}{\arcsin(t)}$ comme en $\frac{1}{t}$. Donc, au voisinage de 0,
    \begin{equation}
        h(t)=\frac{\e^{t}}{\arcsin(t)}-\frac{1}{t}=\frac{t\e^{t}-\arcsin(t)}{t\arcsin(t)}\underset{n\to+\infty}{=}\frac{t^{2}+o(t^{2})}{t^{2}+o(t^{2})}\xrightarrow[t\to0^{+}]{}1.
    \end{equation}
    Donc $h$ est bornée sur $]0,1]$, soit $M=\sup\limits_{t\in]0,1]}\left\lvert h(t)\right\rvert$. On a 
    \begin{equation}
        \left\lvert\int_{x^{2}}^{x}h(t)\d t\right\rvert\leqslant\int_{x}^{x^{2}}\left\lvert h(t)\right\rvert \d t\leqslant M(x-x^{2})\xrightarrow[x\to0]{}0.
    \end{equation}
    Comme $\int_{x^{2}}^{x}\frac{\d t}{t}=-\ln(x)$, on a $f(x)\underset{x\to0}{=}-\ln(x)+o(1)$.

    Pour aller plus loin dans le développement limité, on pousse plus loin le développement limité de $h(t)$ dans $0^{+}$.
\end{proof}

\begin{proof}
    On note $f$ la fonction intégrande. $x\mapsto x^{2}+x+1$ ne s'annule pas sur $\R$. $f$ est continue sur $\R$ et positive, $f(x)\underset{x\to+\infty}{\sim}\frac{1}{x^{2n}}$. Donc d'après le critère de Riemann, l'intégrale converge absolument.

    Pour le calcul, on on 
    \begin{equation}
        x^{2}+x+1=\left(x+\frac{1}{2}\right)^{2}+\frac{3}{4}=\frac{3}{4}\left(\left(\frac{2}{\sqrt{3}}(x+1)\right)^{2}+1\right).
    \end{equation}
    On pose $u=\frac{2}{\sqrt{3}}(x+1)$ et on a 
    \begin{equation}
        I_n=\frac{\sqrt{3}}{2}\left(\frac{4}{3}\right)^{n}\int_{0}^{+\infty}\frac{\d u}{(1+u^{2})^{n}}.
    \end{equation}

    On note $J_n$ l'intégrale:
    \begin{equation}
        J_n=\int_{0}^{+\infty}\frac{1}{(u^{2}+1)^{n-1}}\frac{\d u}{1+u^{2}}.
    \end{equation}
    On pose $\theta=\arctan(u)$, $\mathcal{C}^{1}$-difféomorphisme de $[0,+\infty[\to[0,\frac{\pi}{2}[$. On a $\d\theta=\frac{\d u}{1+u^{2}}$ et $\frac{1}{(1+u^{2})^{n-1}}=\cos^{2n-2}(\theta)$.

    On retrouve les intégrales de Wallis, d'où on en tire 
    \begin{equation}
        J_n=\frac{(2n-1)!}{2^{2(n-1)}(n-1)!}\frac{\pi}{2}.
    \end{equation}
    Donc 
    \begin{equation}
        \boxed{
            I_n=\frac{\sqrt{3}}{2}\left(\frac{4}{3}\right)^{n}\frac{(2n-1)!}{2^{2(n-1)}(n-1)!}\frac{\pi}{2}.
        }
    \end{equation}
\end{proof}

\begin{proof}
    Soit $M_{0}=\sup\limits_{[0,1]}\left\lvert f\right\rvert$ et $M_{1}=\sup\limits_{[0,1]}\left\lvert f'\right\rvert$. D'après l'inégalité des accroissements finis, on a
    \begin{equation}
        \left\lvert f\left(\frac{i+1}{n}\right)-f\left(\frac{i}{n}\right)\right\rvert\leqslant\frac{M_1}{n}.
    \end{equation}
    Donc, par la formule de la somme de Riemann, on a 
    \begin{equation}
        \left\lvert u_n-\underbrace{\frac{1}{n}\sum_{i=0}^{n-1}f\left(\frac{i+1}{n}\right)f'\left(\frac{i+1}{n}\right)}_{\xrightarrow[n\to+\infty]{}\int_{0}^{1}ff'}\right\rvert\leqslant\underbrace{\frac{M_1^{2}}{n}}_{\xrightarrow[n\to+\infty]{}0},
    \end{equation}
    donc 
    \begin{equation}
        \boxed{
            u_n\xrightarrow[n\to+\infty]{}\frac{1}{2}\left(f^{2}(1)-f^{2}(0)\right).
        }
    \end{equation}
\end{proof}

\begin{proof}
    Ici, l'intégrale diverge, mais comme on fait tendre l'intervalle d'intégration à un singleton, cela aura une limite finie.

    Formons \function{f}{\R_{+}^{*}\setminus\left\lbrace1\right\rbrace}{\R}{t}{\frac{1}{\ln(t)}}
    Si $x<1$, $x^{a}$ et $x^{b}$ sont $<1$, et si $x>1$, alors $x^{a}$ et $x^{b}$ sont $>1$. $\int_{x^{a}}^{x^{b}}f(t)\d t$ est donc bien définie.

    On a 
    \begin{equation}
        \frac{1}{\ln(t)}-\frac{1}{t-1}=\frac{1}{\ln(1+(t-1))}-\frac{1}{t-1}=\frac{(t-1)-\ln(1+(t-1))}{(t-1)\ln(1+(t-1))}\underset{t\to1}{\sim}\frac{\frac{(t-1)^{2}}{2}}{(t-1)^{2}}=\frac{1}{2}.
    \end{equation}
    Soit $h\colon\R_{+}^{*}\to\R$ définie par $h(t)=\frac{1}{\ln(t)}-\frac{1}{t-1}$ si $t\neq1$ et $h(1)=\frac{1}{2}$. $h$ est continue donc bornée au voisinage de 1. Il existe $\alpha_{0}>0$ et $M_{0}\geqslant0$ tels que pour tout $t\in[1-\alpha_{0},1+\alpha_{0}]$, on ait 
    \begin{equation}
        \left\lvert \int_{x^{a}}^{x^{b}}\frac{\d t}{\ln(t)}-\int_{x^{a}}^{x^{b}}\frac{\d t}{t-1}\right\rvert\leqslant M_{0}\left\lvert x^{b}-x^{a}\right\rvert\xrightarrow[x\to1]{}0.
    \end{equation}

    Or, si $x=1+x'$, on a
    \begin{align}
        \int_{x^{a}}^{x^{b}}\frac{\d t}{t-1}
        &=\left[\ln\left\lvert t-1\right\rvert\right]_{x^{a}}^{x^{b}},\\
        &=\ln\left\lvert x^{b}-1\right\rvert-\ln\left\lvert x^{a}-1\right\rvert,\\
        &=\ln\left\lvert (1+x')^{b}-1\right\rvert-\ln\left\lvert (1+x')^{a}-1\right\rvert,\\
        &\underset{x'\to0}{=}\ln\left\lvert bx"+o(x')\right\rvert-\ln\left\lvert ax'+o(x')\right\rvert,\\
        &\underset{x'\to0}{=}\ln(b)+\ln(x')+o(1)-\left[\ln(a)+\ln(x')+o(1)\right],\\
        &\underset{x'\to0}{=}\ln\left(\frac{b}{a}\right)+o(1)\xrightarrow[x'\to0]{}\ln\left(\frac{a}{b}\right).
    \end{align}

    D'où le résultat.
\end{proof}

\begin{proof}
    \phantom{}
    \begin{enumerate}
        \item Soit $n\geqslant1$, on a 
        \begin{equation}
            S_n=\frac{b-a}{n}\sum_{k=0}^{n-1}f\left(a+k\frac{b-a}{n}\right)\xrightarrow[n\to+\infty]{}\int_{a}^{b}f.
        \end{equation}
        Par convexité, on a 
        \begin{equation}
            \varphi\left(\frac{1}{b-a}S_n\right)\leqslant\sum_{k=0}^{n-1}\frac{1}{n}\varphi\left(f\left(a+k\frac{b-a}{n}\right)\right),
        \end{equation}
        donc en passant à la limite $n\to+\infty$, par continuité, on a 
        \begin{equation}
            \boxed{
            \varphi\left(\frac{1}{b-a}\int_{a}^{b}f(t)\d t\right)\leqslant\frac{1}{b-a}\int_{a}^{b}\varphi\circ f(t)\d t.}
        \end{equation}

        \item Soit $c\in]a,b[$. En cas d'égalité dans ce qui précède, on a 
        \begin{align}
            \varphi\left(\frac{1}{b-a}\int_{a}^{b}f(t)\d t\right)
            &=\varphi\left(\frac{c-a}{b-a}\frac{1}{c-a}\int_{a}^{c}f(t)\d t+\frac{b-c}{b-a}\frac{1}{b-c}\int_{c}^{b}f(t)\d t\right),\\
            &\leqslant\frac{c-a}{b-a}\varphi\left(\frac{1}{c-a}\int_{a}^{c}f(t)\d t\right)+\frac{b-c}{b-a}\varphi\left(\frac{1}{b-c}\int_{c}^{b}f(t)\d t\right),\\
            &\leqslant\frac{1}{b-a}\left(\int_{a}^{c}\varphi\circ f(t)\d t+\int_{c}^{b}\varphi\circ f(t)\d t\right)=\frac{1}{b-a}\int_{a}^{b}\varphi\circ f(t)\d t,
        \end{align}
        par convexité et par ce qui précède. Par hypothèse, on a égalité partout, Par stricte convexité, on a 
        \begin{equation}
            \frac{1}{c-a}\int_{a}^{c}f(t)\d t=\frac{1}{b-c}\int_{c}^{b}f(t)\d t,
        \end{equation}
        d'où $(b-c)\int_{a}^{c}g(t)\d t=(c-a)\int_{c}^{b}f(t)\d t$. En dérivant par rapport à $c$, on obtient
        \begin{equation}
            (b-a)f(c)-\int_{a}^{c}f(t)\d t=-(c-a)f(c)+\int_{c}^{b}f(t)\d t,
        \end{equation}
        soit $(b-a)f(c)=\int_{c}^{b}f(t)\d t$ et $f(c)=\frac{1}{b-a}\int_{a}^{b}f(t)\d t$ pour tout $c\in]a,b[$. Donc $f$ est constante sur $[a,b]$.
    \end{enumerate}
\end{proof}

\begin{remark}
    Pour la première question, on aurait aussi pu passer par des fonctions en escaliers qui converge uniformément vers $f$.
\end{remark}

\begin{proof}
    Si $f=0$, ça marche. Sinon, il existe $y_0\in\R$ tel que $f(y_0)\neq0$ et donc, pour tout $x\in\R$, on a 
    \begin{equation}
        f(x)=\frac{1}{f(y_0)}\int_{x-y_0}^{x+y_0}f(t)\d t.
    \end{equation}
    Par récurrence, $f$ est $\mathcal{C}^{1}$ (d'après l'expression) et si $f$ est $\mathcal{C}^{k}$, alors elle est $\mathcal{C}^{k+1}$, donc $f$ est $\mathcal{C}^{\infty}$. Par ailleurs, $f(y_{0})f(-x)=-f(y_{0})f(y)$ donc $f$ est impaire. On dérive par rapport à $x$: $f(x+y)-f(x-y)=f'(x)f(y)$, et en dérivant à nouveau par rapport à $x$, on a $f'(x+y)-f'(x-y)=f''(x)f(y)$.

    Même chose par rapport à $y$: $f(x+y)-f(x-y)=f(x)f'(y)$ puis $f'-x+y)-f'(x-y)=f(x)f''(y)$. On pose alors $\alpha=\frac{f''(y_0)}{f(y_0)}$, on a $f''(x)-\alpha f(x)=0$.

    Si $\alpha=0$, comme $f$ est impaire, on a $f(x)=ax$ avec $a\in\R$ et en reportant, on a
    \begin{equation}
        \int_{x-y}^{x+y}f(t)\d t=a\left[\frac{u^{2}}{2}\right]_{x-y}^{x+y}=2axy.
    \end{equation}
    Or $f(x)f(y)=a^{2}xy$ donc ou bien $a=0$, ce qui est exclu, ou bien $a=2$.

    Si $\alpha>0$, on a $f(x)=a_{1}\sinh(\sqrt{\alpha}x)$. En reportant, on a 
    \begin{equation}
        \int_{x-y}^{x+y}f(t)\d t=\frac{2a_{1}}{\sqrt{\alpha}}\sinh(\sqrt{\alpha}x)\sinh(\sqrt{\alpha}y),
    \end{equation}
    et $f(x)=f(y)=a_{1}^{2}\sinh(\sqrt{\alpha}x)\sinh(\sqrt{\alpha}y)$ d'où $a_{1}=\frac{2}{\sqrt{\alpha}}$.

    Si $\alpha<0$, on trouve $f(x)=a_{2}\sin(\sqrt{-\alpha}x)$ avec $a_{2}=\frac{2}{\sqrt{-\alpha}}$.
\end{proof}

\begin{proof}
    Soit $f$ une fonction constante égale à $c$ sur $[a,b]$. On a alors $I_n=(b-a)^{\frac{1}{n}}\left\lvert c\right\rvert\xrightarrow[n\to+\infty]{}\left\lvert c\right\rvert$.

    Plus généralement, on a 
    \begin{equation}
        I_n\leqslant\left(\int_{a}^{b}\left\lVert f\right\rVert_{\infty}^{n}\right)^{\frac{1}{n}}=(b-a)^{\frac{1}{n}}\left\lVert f\right\rVert_{\infty}\xrightarrow[n\to+\infty]{}\left\lVert f\right\rVert_{\infty}.
    \end{equation}

    Soit $\varepsilon>0$, il existe $N_{1}\in\N$ tel que pour tout $n\geqslant N_1$, on a $I_{n}\leqslant\left\lVert f\right\rVert_{\infty}+\varepsilon$. $\left\lvert f\right\rvert$ est continue sur le compact $[a,b]$, donc il existe $t_{0}\in[a,b]$ tel que $\left\lvert f(t_{0})\right\rvert=\left\lVert f\right\rVert_{\infty}$. Par continuité de $\left\lvert f\right\rvert$, il existe $(c,d)\in[a,b]^{2}$ avec $c<d$ tel que pour tout $t\in[c,d]$, on ait $\left\lvert f(t)\right\rvert\geqslant\left\lVert f\right\rVert_{\infty}-\frac{\varepsilon}{2}$.

    On a alors
    \begin{equation}
        \int_{a}^{b}\left\lvert f\right\rvert^{n}\geqslant\int_{c}^{b}\left\lvert f\right\rvert^{n}\geqslant(d-c)\left(\left\lVert f\right\rVert_{\infty}-\frac{\varepsilon}{2}\right)^{n},
    \end{equation}
    donc 
    
    \begin{equation}
        I_n\geqslant\left(\left\lVert f\right\rVert_{\infty}-\frac{\varepsilon}{2}\right)(d-c)^{\frac{1}{n}}\xrightarrow[n\to+\infty]{}\left\lVert f\right\rVert_{\infty}-\frac{\varepsilon}{2}.
    \end{equation}

    Il existe donc $N_{2}\in\N$ tel que pour tout $n\geqslant N_{2}$, on a $I_{n}\geqslant\left\lVert f\right\rVert_{\infty}-\varepsilon$. Donc pour tout $n\geqslant\max(N_{1},N_{2})$, on a 
    \begin{equation}
        \left\lVert f\right\rVert_{\infty}-\varepsilon\leqslant I_{n}\leqslant \left\lVert f\right\rVert_{\infty}+\varepsilon,
    \end{equation}
    donc 
    \begin{equation}
        \boxed{
            \lim\limits_{n\to+\infty}I_{n}=\left\lVert f\right\rVert_{\infty}.
        }
    \end{equation}
\end{proof}

\begin{remark}
    Soit $u_n=I_{n}^{n}$ avec $f$ continue non nulle. On a $u_n>0$ et $u_n^{\frac{1}{n}}\xrightarrow[n\to+\infty]{}\left\lVert f\right\rVert$. Avec l'inégalité de Cauchy-Schwarz appliquée à $\left\lvert f\right\rvert^{\frac{n}{2}}$ et $\left\lvert f\right\rvert^{\frac{n}{2}+1}$, on a 
    \begin{equation}
        0<u_{n+1}=\int_{a}^{b}\left\lvert f\right\rvert^{n+1}\leqslant\sqrt{u_n}\sqrt{u_n+2}
    \end{equation}
    d'où 
    \begin{equation}
        \frac{u_{n+1}}{u_n}\leqslant\frac{u_{n+2}}{u_{n+1}}.
    \end{equation}

    $\left(\frac{u_{n+1}}{u_{n}}\right)_{n\geqslant0}$ est croissante et strictement positive, donc converge vers $l\in\overline{\R_{+}^{*}}$. On a 
    \begin{equation}
        \ln(u_{n+1})-\ln(u_n)=\ln\left(\frac{u_{n+1}}{u_{n}}\right)\xrightarrow[n\to+\infty]{}\ln(l).
    \end{equation}

    D'après le théorème de Césaro, on a donc 
    \begin{equation}
        \frac{\ln(u_{n})}{n}\xrightarrow[n\to+\infty]{}\ln(l),
    \end{equation}
    d'où $\ln\left(u_{n}^{\frac{1}{n}}\right)\xrightarrow[n\to+\infty]{}\ln\left\lVert f\right\rVert_{\infty}=\ln(l)$ par unicité de la limite.

    Donc 
    \begin{equation}
        \lim\limits_{n\to+\infty}\frac{u_{n+1}}{u_{n}}=\left\lVert f\right\rVert_{\infty}.
    \end{equation}
\end{remark}

\begin{proof}
    \phantom{}
    \begin{enumerate}
        \item Pour tout $t\in[-\pi,\pi]$, comme $\rho\neq1$, on a $\e^{\i t}\neq\rho\e^{\i\theta}$, donc $t\mapsto\left\lvert\e^{\i t}-\rho\e^{\i \theta}\right\rvert>0$ et $t\mapsto\ln\left\lvert\e^{\i t}-\rho\e^{\i\theta}\right\rvert$ est continue, $2\pi$-périodique sur $[-\pi,\pi]$ donc $F(\rho,\theta)$ existe.
        \item On a 
        \begin{equation}
            F(\rho,\theta)=\int_{-\pi}^{\pi}\ln\left\lvert\e^{\i(t-\theta)}-\rho\right\rvert\d t=\int_{-\pi-\theta}^{\pi-\theta}\ln\left\lvert\e^{\i u}-\rho\right\rvert\d u,
        \end{equation}
        et comme l'intégrande est une fonction $2\pi$-périodique,
        \begin{equation}
            F(\rho,\theta)=\int_{0}^{2\pi}\ln\left\lvert\e^{\i u}-\rho\right\rvert\d u,
        \end{equation}
        est indépendant de $\theta$.

        \item Soit $n\geqslant1$, on a 
        \begin{equation}
            S_n=\frac{2\pi}{n}\sum_{k=0}^{n-1}\ln\left\lvert\e^{\frac{2\i k\pi}{n}}-\rho\right\rvert=\frac{2\pi}{n}\ln\left(\left\lvert\prod_{k=0}^{n-1}\left(\rho-\e^{\frac{2\i k\pi}{n}}\right)\right\rvert\right)=\frac{2\pi}{n}\ln\left(\left\lvert\rho^{n}-1\right\rvert\right).
        \end{equation}

        Si $\rho>1$, on a 
        \begin{align}
            S_n
            &=\frac{2\pi}{n}\ln\left(\rho^{n}-1\right),\\
            &=\frac{2\pi}{n}\left[\ln(\rho^{n})+\ln\left(1-\left(\frac{1}{\rho}\right)^{n}\right)\right]\xrightarrow[n\to+\infty]{}2\pi\ln(\rho).
        \end{align}

        Donc $F(\rho,\theta)=2\pi\ln(\rho)$.

        Si $\rho<1$, $S_n=\frac{2\pi}{n}\ln(1-\rho^{n})\xrightarrow[n\to+\infty]{}0$ donc $F(\rho,\theta)=0$.
    \end{enumerate}
\end{proof}

\begin{remark}
    On a 
    \begin{align}
        F(\rho,0)
        &=\int_{0}^{2\pi}\ln\left\lvert\cos(u)-\rho+\i\sin(u)\right\rvert\d u,\\
        &=\frac{1}{2}\int_{0}^{2\pi}\ln\left(\rho^{2}-2\rho\cos(u)+1\right)\d u,\\
        &=2\pi\ln(\rho)+F\left(\frac{1}{\rho},0\right).
    \end{align}
\end{remark}

\begin{remark}
    On peut se demander si l'on a convergence de $F(1,0)=\frac{1}{2}\int_{-\pi}^{\pi}\ln\left(2(1-\cos(u))\right)\d u$. On vérifie que 
    \begin{equation}
        \left\lvert\ln(2(1-\cos(u)))\right\vert\underset{u\to0}{\sim}2\left\lvert\ln u\right\rvert=\underset{u\to0}{o}\left(\frac{1}{\sqrt{u}}\right).
    \end{equation}
    Donc $F(1,0)$ converge. Pour le calcul, on a 
    \begin{align}
        2F(1,0)
        &=2\pi\ln(2)+2\int_{0}^{\pi}\ln(1-\cos(u))\d u,\\
        &=2\pi\ln(2)+4\int_{0}^{\pi}\ln\left(\sin\left(\frac{u}{2}\right)\right)\d u,\\
        &=2\pi\ln(2)+8\int_{0}^{\frac{\pi}{2}}\ln\left(\sin\left(v\right)\right)\d v.
    \end{align}

    D'après un exercice précédent, l'intégrale vaut $-\frac{\pi}{2}\ln(2)$ et finalement $F(1,0)=0$.
\end{remark}

\begin{proof}
    Toutes les intégrales existent car les fonctions sont à support compact.
    \begin{enumerate}
        \item Montrons la contraposée. Soit $\delta\in\R$ tel que $f(\delta)\neq0$. On suppose que $f(\delta)>0$. Par continuité, il existe $\eta\>0$ tel que $f\geqslant0$ sur $[\delta-\eta,\delta+\eta]$. $f\times\varphi$ est continue sur $[\delta-\eta,\delta+\eta]$, positive et $(f\varphi)(\delta)>0$ donc $\int_{\R}f\varphi>0$ (en choisissant $\varphi\geqslant0$ définie sur le support $[\delta-\eta,\delta+\eta]$).
        \item Montrons un petit lemme: si $\psi\in C_{0}$, il existe $\varphi\in C_{1}$ tel que $\psi=\varphi'$ si et seulement si $\int_{\R}\psi=0$. Pour le sens direct, on a $\int_{\R}\psi=\int_{\R}\varphi'=\lim\limits_{t\to+\infty}\varphi(t)-\lim\limits_{t\to-\infty}\varphi(t)=0$. Pour le sens indirect, on définit $\varphi(x)=\int_{-\infty}^{x}\psi$ (possible car $\psi\in C_{0}$). $\varphi$ est $\mathcal{C}^{1}$ et $\varphi'=\psi$. $\varphi$ est $\mathcal{C}^{1}$ et $\varphi'=\psi$. Soit $A\geqslant0$ tel que pour tout $\left\lvert t\right\rvert\geqslant A$, on a $\psi(t)=0$. Alors pour tout $t\leqslant-A$, on a $\varphi(t)=0$ et pour tout $t\geqslant A$, $\varphi(t)=\int_{-\infty}^{t}\psi=\int_{-A}^{A}\psi=0$. Donc $varphi$ est à support compact.
        
        Pour montrer le résultat, montrons la contraposée. Supposons $f$ non constante, il existe $(x_{1},x_{2})\in\R^{2}$ distincts tel que $f(x_{1})\neq f(x_{2})$. Quitte à remplacer $f$ par $f-\frac{f(x_1)+f(x_2)}{2}$, on peut supposer que $f(x_{2})=-f(x_{1})$. Il existe $\eta>0$ tel que pour tout $t\in[x_{1}-\eta,x_{1}+\eta]$, $f(t)\geqslant0$ et pour tout $t\in[x_{2}-\eta,x_{2}+\eta]$, $f(t)\leqslant0$.

        On a $\int_{\R}\psi=0$. On pose $\varphi(t)=\int_{-\infty}^{t}\psi(x)\d x$. Alors $\varphi\in C_{1}$ et $\int_{\R}f\varphi'>0$.

        \item Soit $G$ une primitive de $g$. On a alors, pour tout $\varphi\in C_{1}$,
        \begin{equation}
            \int_{\R}g\varphi=\left[G\varphi\right]_{-\infty}^{+\infty}-\int_{\R}G\varphi'=\int_{\R}f\varphi'.
        \end{equation}
        D'après ce qui précède, on a, pour tout $\varphi\in C_{1}$, $\int_{\R}(f+G)\varphi'=0$ et donc $f=-G$ à une constante près.
    \end{enumerate}
\end{proof}

\begin{proof}
    On note \function{f}{\R_{+}^{*}}{\R}{t}{\frac{\e^{-t}-\e^{-2t}}{t}}
    $f$ est continue, tend vers $1$ en 0 et $f(t)=\underset{t\to+\infty}{o}\left(\frac{1}{t^{2}}\right)$ donc $I$ existe.

    Soit $\varepsilon>0$. On a 
    \begin{align}
        \int_{\varepsilon}^{+\infty}\frac{\e^{-t}-\e^{-2t}}{t}\d t
        &=\int_{\varepsilon}^{+\infty}\frac{\e^{-t}}{t}\d t-\int_{\varepsilon}^{+\infty}\frac{\e^{-2t}}{t}\d t,\\
        &=\int_{\varepsilon}^{+\infty}\frac{\e^{-t}}{t}\d t-\int_{2\varepsilon}^{+\infty}\frac{\e^{-t}}{t}\d t,\\
        &=\int_{\varepsilon}^{2\varepsilon}\frac{\e^{-t}}{t}\d t.
    \end{align}

    La fonction $\frac{\e^{-t}-1}{t}$ tend vers -1 quand $t\to0$, donc elle est bornée au voisinage de 0 et est continue. Donc 
    \begin{equation}
        \int_{\varepsilon}^{2\varepsilon}\frac{\e^{-t}-1}{t}=\int_{\varepsilon}^{2\varepsilon}\frac{\e^{-t}}{t}\d t-\ln(2)\xrightarrow[\varepsilon\to0]{}0,
    \end{equation}
    donc 
    \begin{equation}
        \boxed{
            I=\ln(2).
        }
    \end{equation}

    On note maintenant $f$ la fonction intégrande de $J$. $f$ est continue sur $\R_{+}^{*}$, on a 
    \begin{equation}
        f(t)\xrightarrow[t\to0]{}0,    
    \end{equation}
    et 
    \begin{equation}
        \int_{1}^{X}\frac{\cos(t)}{t}\d t=\underbrace{\left[\frac{\sin(t)}{t}\right]_{1}^{X}}_{\xrightarrow[X\to+\infty]{}-\sin(1)}+\int_{1}^{X}\frac{\sin(t)}{t^{2}}\d t,
    \end{equation}
    et l'intégrale de droite converge absolument. Donc $\int_{1}^{+\infty}\frac{\cos(t)}{t}\d t$ converge, et c'est donc aussi le cas pour $\int_{1}^{+\infty}\frac{\cos(2t)}{t}\d t$. Donc $J$ existe.

    Soit $\varepsilon>0$ et $X\geqslant \varepsilon$, on a 
    \begin{align}
        \int_{\varepsilon}^{X}\frac{\cos(t)-\cos(2t)}{t}\d t
        &=\int_{\varepsilon}^{X}\frac{\cos(t)}{t}\d t-\int_{\varepsilon}^{X}\frac{\cos(2t)}{t}\d t,\\
        &=\int_{\varepsilon}^{2\varepsilon}\frac{\cos(t)}{t}\d t-\int_{X}^{2X}\frac{\cos(t)}{t}\d t.
    \end{align}
    La deuxième intégrale tend vers 0 quand $X\to+\infty$ (car l'intégrale est semi-convergente), et de même, on a $\int_{\varepsilon}^{2\varepsilon}\frac{\cos(t)-1}{t}\d t\xrightarrow[\varepsilon\to0]{}0$ donc 
    \begin{equation}
        \boxed{
            J=\ln(2).
        }
    \end{equation}
\end{proof}

\begin{remark}
    Généralement, pour $a<b$ et $f\colon\R_{+}\to\R$ continue, dérivable en 0 et telle que $\int_{1}^{+\infty}\frac{f(t)}{t}\d t$ converge. Alors on a $f(u)=f(0)+uf'(0)+\underset{u\to0}{o}(u)$ et 
    \begin{equation}
        \int_{0}^{+\infty}\frac{f(at)-f(bt)}{t}\d t,
    \end{equation}
    existe. En notant $g$ la fonction intégrande, $g$ tend vers $(b-a)f'(0)$ en 0. Et en séparant pour $\varepsilon>0$, on a 
    \begin{equation}
        \int_{\varepsilon}^{+\infty}\frac{f(at)-f(bt)}{t}\d t=\int_{a\varepsilon}^{b\varepsilon}\frac{f(u)}{u}\d u\xrightarrow[\varepsilon\to0]{}f(0)\ln\left(\frac{b}{a}\right).
    \end{equation}
\end{remark}

\begin{proof}
    $f$ est continue par morceaux, positive. On a $0\leqslant f\leqslant2$ donc $f$ est intégrable sur $]0,1]$. On écrit 
    \begin{equation}
        I=\sum_{k=1}^{+\infty}\int_{\frac{1}{k+1}}^{\frac{1}{k}}\frac{\d t}{t}-\frac{k}{k(k+1)},
    \end{equation}
    et en prenant les sommes partielles et en passant à la limite, on trouve 
    \begin{equation}
        \boxed{
            I=1-\gamma.
        }
    \end{equation}
\end{proof}

\begin{proof}
    On note \function{g}{\R_{+}^{*}}{\R}{t}{\frac{1}{\e^{t}-1}}
    $g$ est continue positive sur $\R_{+}^{*}$ et $g(t)=\underset{t\to+\infty}{O}\left(\frac{1}{t^{2}}\right)$/ Donc $f$ est bien définie sur $\R_{+}^{*}$.

    Pour tout $x\in\R_{+}^{*}$, on a 
    \begin{align}
        f(x)
        &=\int_{x}^{+\infty}\frac{\e^{-t}}{1-\e^{-t}}\d t,\\
        &=\left[\ln\left(1-\e^{-t}\right)\right]_{x}^{+\infty},\\
        &=-\ln\left(1-\e^{-x}\right).
    \end{align}
    $f$ est continue et positive sur $\R_{+}^{*}$. On a $f(x)\underset{x\to0}{=}-\ln(x+o(x))\underset{x\to0}{\sim}-\ln(x)=\underset{x\to0}{O}\left(\frac{1}{\sqrt{x}}\right)$ et $f(x)\underset{x\to+\infty}{\sim}\e^{-x}=\underset{x\to+\infty}{O}\left(\frac{1}{x^{2}}\right)$. D'où l'existence de $I$. On pose $u=\e^{-x}$ soit $x=-\ln(u)$ et $\d x=-\frac{\d u}{u}$. C'est un $\mathcal{C}^{1}$-difféomorphisme de $\R_{+}^{*}$ dans $]0,1[$. On a alors 
    \begin{equation}
        I=-\int_{0}^{1}\frac{-\ln(1-u)}{u}\d u=\int_{0}^{1}\frac{-\ln(1-u)}{u}\d u.
    \end{equation}
    On pose $v=1-u$ pour avoir 
    \begin{equation}
        I=\int_{0}^{1}\frac{-\ln(v)}{1-v}\d v.
    \end{equation}
    Pour $v\in]0,1[$, on développe
    \begin{equation}
        \frac{-\ln(v)}{1-v}=\sum_{k=0}^{+\infty}-v^{k}\ln(v)=\sum_{k=0}^{+\infty}f_k(v).
    \end{equation}
    $f_k$ est positive intégrable sur $]0,1[$.

    On forme $u_k=\int_{0}^{1}\left\lvert f_k(t)\right\rvert\d t$ pour $k\in\N$. Alors $u_k=\int_{0}^{1}-t^{k}\ln(t)\d t=\frac{1}{(k+1)^{2}}$. Donc $\sum u_{k}$ converge et on peut intervertir. Finalement, on a 
    \begin{equation}
        \boxed{
            I=\frac{\pi^{2}}{6}.
        }
    \end{equation}
\end{proof}

\begin{remark}
    On peut aussi appliquer le théorème de convergence dominée à $\left(\sum_{k=0}^{n}f_k\right)$ car les $f_k$ sont positifs.
\end{remark}

\begin{remark}
    On a 
    \begin{equation}
        f(x)=\underbrace{\int_{x}^{1}\frac{\d t}{\e^{t}-1}}_{\underset{x\to0}{\sim}\int_{x}^{1}\frac{1}{t}\d t}+\int_{1}^{+\infty}\frac{\d t}{\e^{t}-1}\underset{x\to0}{\sim}-\ln(x)=\underset{x\to0}{O}\left(\frac{1}{\sqrt{x}}\right),
    \end{equation}
    car la fonction est positive, et 
    \begin{equation}
        f(x)\underset{x\to+\infty}{\sim}\int_{x}^{+\infty}\e^{-t}\d t=\e^{-x}=\underset{x\to+\infty}{O}\left(\frac{1}{x^{2}}\right).
    \end{equation}
\end{remark}

\begin{proof}
    \phantom{}
    \begin{enumerate}
        \item On pose $f_n(t)=\left(1+\frac{t}{n}\right)^{n}\e^{-t}=\underset{t\to+\infty}{O}\left(\frac{1}{t^{2}}\right)$, continue positive sur $[-n,+\infty[$. $I_n$ est donc bien définie. On pose $u=t+n$, et alors 
        \begin{align}
            I_n
            &=\int_{0}^{+\infty}\frac{u^{n}}{n^{n}}\e^{-u}\e^{n}\d u,\\
            &=\frac{\e^{n}}{n^{n}}\int_{0}^{+\infty}u^{n}\e^{-u}\d u,\\
            &=\frac{\e^{n}}{n^{n}}\Gamma(n+1),\\
            &=\frac{\e^{n}}{n^{n}}n!.
        \end{align}

        \item On pose $t=\sqrt{n}u$ et on a 
        \begin{equation}
            J_n=\sqrt{n}\int_{-\sqrt{n}}^{\sqrt{n}}\e^{n\ln\left(1+\frac{u}{\sqrt{n}}\right)-\sqrt{n}u}\d u.
        \end{equation}
        On définit, pour tout $n\in\N^{*}$, $f_n\colon\R\to\R$ par $f(u)=\e^{n\ln\left(1+\frac{u}{\sqrt{n}}\right)-\sqrt{n}u}$ sur $[-\sqrt{n},\sqrt{n}]$ et $0$ ailleurs. $f_n$ est continue par morceaux (intégrable) sur $\R$. Soit $u\in\R$, il existe $N_0\in\N$ tel que pour tout $n\geqslant N_{0}$, $\sqrt{n}\geqslant\left\lvert u\right\rvert$. Alors pour tout $n\geqslant N_0$, on a 
        \begin{align}
            f_n(u)
            &\underset{n\to+\infty}{=}\e^{n\left(\frac{u}{\sqrt{n}}-\frac{u^{2}}{2n}+O\left(\frac{1}{n^{\frac{3}{2}}}\right)\right)-\sqrt{n}u},\\
            &\underset{n\to+\infty}{=}\e^{-\frac{u^{2}}{2}+O\left(\frac{1}{\sqrt{n}}\right)}\xrightarrow[n\to+\infty]{}\e^{-\frac{u^{2}}{2}}.
        \end{align}

        Ainsi, on a convergence simple de $f_n\xrightarrow[n\to+\infty]{}f$ sur $\R$ avec $f(u)=\e^{-\frac{u^{2}}{2}}$.

        C'est un cas particulier où trouver la dominante (pour appliquer le théorème de convergence dominée) est difficile. On propose $\varphi(u)=\e^{-\frac{u^{2}}{4}}$ (plus grand que $f$ et toujours intégrable). Soit $n\geqslant1$ et $u\in]-\sqrt{n},\sqrt{n}]$, soit 
        \begin{equation}
            g_n(u)=n\ln\left(1+\frac{u}{\sqrt{n}}\right)-\sqrt{n}u-\left(-\frac{u^{2}}{4}\right),
        \end{equation}
        dérivable sur $]-\sqrt{n},\sqrt{n}]$. On a 
        \begin{equation}
            g'_n(u)=\frac{-\frac{u}{2}\left(1-\frac{u}{\sqrt{n}}\right)}{1+\frac{u}{\sqrt{n}}},
        \end{equation}
        donc $g_n'\leqslant0$ si $u\geqslant0$ et $g_n>0$ si $u<0$. Comme $g_n(0)=0$, on a $g_n(u)\leqslant0$ pour tout $u\in]-\sqrt{n},\sqrt{n}]$ et donc pour tout $u\in\R$, $0\leqslant f_n(u)\leqslant\e^{-\frac{u^{2}}{4}}$.

        On peut donc appliquer le théorème de convergence dominée et on a
        \begin{equation}
            \boxed{
                \frac{J_n}{\sqrt{n}}=\int_\R\e^{-\frac{u^{2}}{2}}\d u=\sqrt{2\pi}.
            }
        \end{equation}

        \item On pose pour $n\in\N^{*}$,
        \begin{align}
            K_n
            &=\int_{-n}^{+\infty}\left(1+\frac{t}{n}\right)^{n}\e^{-t}\d t,\\
            &=\frac{\e^{n}}{n^{n}}\int_{2n}^{+\infty}u^{n}\e^{-u}\d u.
        \end{align}
        On écrit $u^{n}\e^{-u}=u^{n}\e^{-\frac{u}{2}}\e^{-\frac{u}{2}}=h(u)\e^{-\frac{u}{2}}$. Alors $h$ est dérivable sur $]2n,+\infty[$ et $h'(u)=u^{n-1}\left(n-\frac{u}{2}\right)\e^{-\frac{u}{2}}$ donc 
        \begin{equation}
            0\leqslant h(u)\leqslant (2n)^{n}\e^{-n}.
        \end{equation}
        Donc 
        \begin{equation}
            K_n\leqslant 2^{n}\int_{2n}^{+\infty}\e^{-\frac{u}{2}}\d u =2^{n}\times 2\e^{-n}.
        \end{equation}
        Or $\e>2$ donc $\lim\limits_{n\to+\infty}K_n=0$. Donc $K_n=\underset{n\to+\infty}{o}(I_n)$ et $I_n=J_n+K_n$ donc $I_n\underset{n\to+\infty}{\sim}J_n$ et 
        \begin{equation}
            \boxed{
                n!\underset{n\to+\infty}{\sim}\frac{n^{n}}{\e^{n}}\sqrt{2\pi n}.
            }
        \end{equation}
    \end{enumerate}
\end{proof}

\begin{proof}
    \phantom{}
    \begin{enumerate}
        \item Soit $x\in\R$, on a $f_x(t)=\underset{t\to+\infty}{O}\left(\frac{1}{t^{2}}\right)$ et $f_x(t)\xrightarrow[t\to0]{}\frac{x^{2}}{2}$ donc $I(x)$ est bien définie.
        \item Pour $t\in\R_{+}^{*}$, $x\mapsto f_x(t)$ est de classe $\mathcal{C}^{2}$. D'après l'inégalité de Taylor-Lagrange, pour tout $a\neq0$, on a 
        \begin{equation}
            \left\lvert\cos(a)-1\right\rvert=\left\lvert\cos(a)-\cos(0)-a\cos'(0)\right\rvert\leqslant\frac{a^{2}}{2}\sup\limits_{[0,a]}\left\lvert\cos''\right\rvert\leqslant\frac{a^{2}}{2}.
        \end{equation}
        Donc $\left\lvert 1-\cos(tx)\right\rvert\leqslant\frac{t^{2}x^{2}}{2}$. Ainsi, pour tout $t>0$, on a $\left\lvert f_x(t)\right\rvert\leqslant\frac{x^{2}}{2}\e^{-t}$. Fixons $a\geqslant0$, on a pour tout $x\in[-a,a]$, pour tout $t>0$, $\left\lvert f_x(t)\right\rvert\leqslant\frac{a^{2}}{2}\e^{-t}$, fonction indépendante de $x$ et intégrable sur $\R_{+}$.

        D'après le théorème de continuité, $I$ est continue sur $[-a,a]$, pour tout $a\geqslant0$, donc sur $\R$. On a $\frac{\partial f_x}{\partial x}(t)=\frac{\sin(t)x}{t}\e^{-t}$, $\frac{\partial^{2}f_x}{\partial x^{2}}(t)=\cos(tx)\e^{-t}$. Pour tout $u\in\R$, $\left\lvert\sin(u)\right\rvert\leqslant\left\lvert u\right\rvert$ donc pour tout $t>0$, pour tout $x\in[-a,a]$,
        \begin{equation}
            \left\lvert\frac{\partial f_x}{\partial x}(t)\right\rvert\leqslant\left\lvert x\right\rvert\e^{-t}\leqslant a\e^{-t},
        \end{equation}
        \begin{equation}
            \left\lvert\frac{\partial^{2} f_x}{\partial x^{2}}(t)\right\rvert\leqslant\e^{-t}.
        \end{equation}
        Donc d'après le théorème de dérivation, $I$ est de classe $\mathcal{C}^{2}$ et on a 
        \begin{align}
            I''(x)
            &=\int_{0}^{+\infty}\cos(tx)\e^{-t}\d t,\\
            &=\Re\left(\int_{0}^{+\infty}\e^{-(1-\i x)t}\d t\right),\\
            &=\Re\left(\frac{1}{1-\i x}\right),\\
            &=\frac{1}{1+x^{2}}.
        \end{align}
        Donc $I'(x)=\lambda+\arctan(x)$ avec $\lambda\in\R$. Comme $I'(0)=0$, $I'(x)=\arctan(x)$. On a $I(0)=0$ donc 
        \begin{align}
            I(x)
            &=\int_{0}^{x}\arctan(u)\d u,\\
            &=\left[u\arctan(u)\right]_{0}^{x}-\int_{0}^{x}\frac{u}{1+u^{2}}\d u,\\
            &=x\arctan(x)-\frac{1}{2}\ln\left(1+x^{2}\right).
        \end{align}
    \end{enumerate}
\end{proof}

\begin{remark}
    En posant $v=xt$ pour $x>0$, on a 
    \begin{equation}
        I(x)=x\int_{0}^{+\infty}\frac{1-\cos(v)}{v^{2}}\e^{-\frac{v}{x}}\d v,
    \end{equation}
    et 
    \begin{equation}
        \lim\limits_{x\to+\infty}\frac{I(x)}{x}=\int_{0}^{+\infty}\frac{1-\cos(v)}{v^{2}}\d v=\frac{\pi}{2},
    \end{equation}
    et par intégration par parties, cette intégrale vaut $\int_{0}^{+\infty}\frac{\sin(v)}{v}\d v$. C'est une autre preuve de l'intégrale de Dirichlet.
\end{remark}

\begin{proof}
    Soit $x\geqslant0$, $t\mapsto\frac{\sin(t)}{t}$ définie continue sur $\R$. On a 
    \begin{equation}
        \int_{x}^{X}\frac{\sin(t)}{t}\d t=\underbrace{\left[\frac{1-\cos(t)}{t}\right]_{x}^{X}}_{\xrightarrow[X\to+\infty]{}\frac{1-\cos(x)}{x}}+\int_{x}^{X}\frac{1-\cos(t)}{t^{2}}\d t.
    \end{equation}
    L'intégrale est absolument convergente, donc $f(x)$ est définie et on a 
    \begin{equation}
        f(x)=\frac{\cos(x)-1}{x}+\int_{x}^{+\infty}\frac{1-\cos(t)}{t^{2}}\d t.
    \end{equation}
    $f$ est de classe $\mathcal{C}^{1}$ et $f'(x)=-\frac{\sin(x)}{x}$. Soit $X>0$ fixé, on a 
    \begin{align}
        \int_{0}^{X}f(x)\d x
        &=\left[xf(x)\right]_{0}^{X}-\int_{0}^{X}f'(x)x\d x,\\
        &=Xf(X)+\int_{0}^{X}\sin(x)\d x,\\
        &=Xf(X)+(1-\cos(X)),\\
        &=X\int_{X}^{+\infty}\frac{1-\cos(t)}{t^{2}}\d t,\\
        &=1-X\int_{X}^{+\infty}\frac{\cos(t)}{t^{2}}\d t.
    \end{align}
    Par intégrations par parties, on a 
    \begin{align}
        \int_{X}^{+\infty}\frac{\cos(t)}{t^{2}}\d t
        &=\left[\frac{1}{t^{2}}\sin(t)\right]_{X}^{+\infty}+\int_{X}^{+\infty}\frac{2}{t^{2}}\sin(t)\d t,\\
        &=-\frac{\sin(X)}{X^{2}}+\int_{X}^{+\infty}\frac{2\sin(t)}{t^{3}}\d t,\\
        &=\underset{X\to+\infty}{O}\left(\frac{1}{X^{2}}\right),
    \end{align}
    car la deuxième intégrale est majorée par $\int_{X}^{+\infty}\frac{2}{t^{3}}\d t=\frac{1}{X^{2}}$. Finalement, on a 
    \begin{equation}
        \int_{0}^{X}f(x)\d x=1+\underset{X\to+\infty}{O}\left(\frac{1}{X}\right),
    \end{equation}
    donc 
    \begin{equation}
        \boxed{
            \int_{0}^{+\infty}f(x)\d x=1.
        }
    \end{equation}
\end{proof}

\begin{proof}
    Définissons $f_h$ pour $h>0$ par $f_h(t)=f(nh)$ si $t\in[nh,(n+1)h[$ ($n\left\lfloor\frac{t}{h}\right\rfloor$). Pour $h$ fixé, $f(nh)=\underset{n\to+\infty}{O}\left(\frac{1}{n^{2}}\right)$ donc $\phi(h)$ est bien définie. $f_h$ est continue par morceaux et $f_h(t)=\underset{t\to+\infty}{O}\left(\frac{1}{t^{2}}\right)$ donc $f_h$ est intégrable sur $\R_{+}$ et 
    \begin{equation}
        \phi(h)=\int_{0}^{+\infty}f_h(t)\d t.
    \end{equation}

    Soit $t$ fixé, on a $\lim\limits_{h\to0^{+}}h \left\lfloor\frac{t}{h}\right\rfloor=t$. Donc par continuité de $f$, $f_h(t)\xrightarrow[h\to0^{+}]{}f(t)$. 

    On sait qu'il existe $M\in\R_{+}$ tel que pour tout $x>0$, $\left\lvert f(x)\right\rvert\leqslant\frac{M}{x^{2}}$. Donc 
    \begin{equation}
        \left\lvert f_h(t)\right\rvert\leqslant\frac{M}{\left(h\left\lfloor\frac{t}{h}\right\rfloor\right)^{2}}\leqslant\frac{M}{\left(t-h\right)^{2}}.
    \end{equation}

    On s'impose $h<1$. Dans ce cas, pour tout $t>2$, $\left\lvert f_h(t)\right\rvert\leqslant\frac{M}{\left(t-1\right)^{2}}$ est intégrable sur $[2,+\infty[$ et indépendant de $h$. Pour tout $t\in[0,2]$, $\left\lvert f_h(t)\right\rvert\leqslant\left\lVert f\right\rVert_{\infty}$ intégrable sur $[0,2]$ et indépendant de $h$.

    Soit \function{\varphi}{\R_+}{\R}{t}{
        \left\lbrace
        \begin{array}[]{ll}
            \left\lVert f\right\rVert_{\infty} &\text{si }t\in[0,2],\\
            \frac{M}{(t-1)^{2}} &\text{si }t>2.
        \end{array}
        \right.
    }
    $\varphi$ est continue par morceaux intégrable sur $\R_{+}$ et indépendante de $h$. Donc, par convergence dominée,
    \begin{equation}
        \boxed{
            \lim\limits_{h\to0}\phi(h)=\int_{0}^{+\infty}f(t)\d t.
        }
    \end{equation}
\end{proof}

\begin{proof}
    $f$ est impaire donc on se limite à $x>0$. On pose $g(x,t)$ l'intégrande. $t\mapsto g(x,t)$ est continue sur $\R_{+}^{*}$, $g(x,t)\xrightarrow[t\to0]{}x$ et $g(x,t)\underset{t\to+\infty}{\sim}\frac{\e^{(x-1)t}}{2t}$, donc $t\mapsto g(x,t)$ est intégrable sur $\R_{+}^{*}$ si et seulement si $x<1$. Donc le domaine de définition est $]-1,1[$. Enfin, $x\mapsto g(x,t)$ est $\mathcal{C}^{\infty}$.

    On a $\frac{\partial g}{\partial x}(x,t)=\cosh(xt)\e^{-t}=\frac{\e^{(x-1)t}+\e^{-(x+1)t}}{2}$. Fixons $a\in[0,1[$, soit $x\in[0,a]$. Si $t\geqslant1$, on a $0\leqslant \sinh(xt)\leqslant\frac{\e^{xt}}{2}$ et
    \begin{equation}
        0\leqslant g(x,t)\leqslant\frac{\e^{(x-1)t}}{2t}\leqslant\frac{\e^{(a-1)t}}{2t}.
    \end{equation}

    Par ailleurs, $\lim\limits_{u\to0}\frac{\sinh(u)}{u}=1$, donc il existe $M\geqslant0$ tel que si $\left\lvert u\right\vert\leqslant a$, $\left\lvert\frac{\sinh(u)}{u}\right\rvert\leqslant M$. Si $t\in]0,1]$, $xt\in]0,a]$, $0\leqslant g(x,t)\leqslant M_a$. En définissant 
    \function{\phi_0}{\R_{+}^{*}}{\R_{+}}{t}{
        \left\lbrace
            \begin{array}[]{ll}
                M_a & \text{si }t\in]0,1],\\
                \frac{\e^{(a-1)t}}{2t} &\text{si } t>1.    
            \end{array}
    \right.
    }
    $\phi_0$ est intégrable sur $\R_{+}^{*}$ et $\left\lvert g(x,t)\right\rvert\leqslant\phi_0(t)$. Or $\sinh$ est croissante et $\left\lvert g(x,t)\right\rvert\leqslant\frac{\sinh(at)}{t}\e^{-t}$, intégrable sur $\R_{+}^{*}$. Par ailleurs, $\left\lvert\frac{\partial g}{\partial x}(x,t)\right\rvert\leqslant\cosh(at)\e^{-t}$ est intégrable sur $\R_{+}^{*}$ car $a<1$. D'après le théorème de continuité dérivabilité, $f$ est de classe $\mathcal{C}^{1}$ sur $[0,a]$ pour tout $a<1$, donc sur $[0,1[$.

    Alors 
    \begin{align}
        f'(x)
        &=\int_{0}^{+\infty}\cosh(xt)\e^{-t}\d t,\\
        &=\frac{1}{2}\int_{0}^{+\infty}\left(\e^{(x-1)t}+\e^{-(x+1)t}\right)\d t,\\
        &=\frac{1}{2}\left(\frac{1}{1-x}+\frac{1}{x+1}\right).
    \end{align}
    Comme $f(0)=0$, pour $x\in[0,1[$, $f(x)=\frac{1}{2}\ln\left(\frac{1+x}{1-x}\right)$ et si $x\in]-1,0]$, $f(x)=-f(-x)=\frac{1}{2}\ln\left(\frac{1+x}{1-x}\right)$.
\end{proof}

\begin{proof}
    On a $\left\lvert f(x,t)\right\rvert=\frac{\e^{-t}}{\sqrt{t}}=\underset{t\to+\infty}{O}\left(\frac{1}{t^{2}}\right)$ et est équivalent ) $\frac{1}{\sqrt{t}}$ en 0. Donc $F$ existe et est continue sur $\R$. De plus,
    \begin{equation}
        \left\lvert\frac{\partial f}{\partial x}(x,t)\right\rvert=\sqrt{t}\e^{-t},
    \end{equation}
    est intégrable sur $\R_{+}$. Donc $F$ est $\mathcal{C}^{1}$ et 
    \begin{align}
        F'(x)
        &=\int_{0}^{+\infty}\i\sqrt{t}\e^{t(\i x-1)}\d t,\\
        &=\left[\frac{\i\sqrt{t}\e^{t(\i x-1)}}{\i x-1}\right]_{0}^{+\infty}-\int_{0}^{+\infty}\frac{\i \e^{t(\i x-1)}}{2\sqrt{t}(\i x-1)}\d t,\\
        &=-\frac{\i}{2(\i x-1)}F(x).
    \end{align}
    On a
    \begin{equation}
        \frac{\i}{2(\i x-1)}=\frac{x}{2(x^{2}+1)}-\frac{\i}{2(x^{2}+1)},
    \end{equation}
    donc 
    \begin{equation}
        F(x)=A\exp\left(-\frac{1}{4}\ln\left(x^{2}+1\right)+\frac{\i}{2}\arctan(x)\right).
    \end{equation}
    Comme 
    \begin{equation}
        F(0)=A=\int_{0}^{+\infty}\frac{\e^{-t}}{\sqrt{t}}\d t=\sqrt{\pi},
    \end{equation}
    on a
    \begin{equation}
        \boxed{
            F(x)=\sqrt{\pi}\exp\left(-\frac{1}{4}\ln\left(x^{2}+1\right)+\frac{\i}{2}\arctan(x)\right).
        }
    \end{equation}
\end{proof}

\begin{proof}
    \phantom{}
    \begin{enumerate}
        \item On a 
        \begin{equation}
            \left\lvert\widehat{f}(\nu)\e^{\i\nu x}\e^{-\lambda\left\lvert\lambda\right\rvert}\right\rvert\leqslant\left\lvert \widehat{f}(\nu)\right\rvert.
        \end{equation}
        Donc $g_x$ est bien définie. On pose 
        \begin{equation}
            h(t,\nu)=f(t)\e^{\i\nu(x-t)}\e^{-\lambda\left\lvert\nu\right\rvert}.
        \end{equation}
        On a 
        \begin{equation}
            g_x(\lambda)=\frac{1}{2\pi}\int_{-\infty}^{+\infty}\left(\int_{-\infty}^{+\infty}h(t,\nu)\d t\right)\d\nu,
        \end{equation}
        et comme 
        \begin{equation}
            \int_{-\infty}^{+\infty}\int_{-\infty}^{+\infty}\left\lvert h(t,\nu)\right\rvert\d t\d\nu=\int_{\R}\e^{-\lambda\left\lvert\nu\right\rvert}\d\nu\times\int_{\R}\left\lvert f(t)\right\rvert\d t=\frac{2}{\lambda}\int_{\R}\left\lvert f(t)\right\rvert\d t<+\infty,
        \end{equation}
        d'après le théorème de Fubini,
        \begin{align}
            g_x(\lambda)
            &=\frac{1}{2\pi}\int_{\R}\left(\int_{\R}\e^{\i\nu(x-t)}\e^{-\lambda\left\lvert\nu\right\rvert}\d \nu\right)f(t)\d t,\\
            &=\frac{1}{2\pi}\int_{\R}\left(\int_{-\infty}^{0}\e^{\i\nu(x-t)}\e^{\lambda\nu}\d \nu+\int_{0}^{+\infty}\e^{\i\nu(x-t)}\e^{-\lambda\nu}\right)f(t)\d t,\\
            &=\frac{1}{2\pi}\int_{\R}\left(\frac{1}{\lambda-\i(t-x)}+\frac{1}{\lambda-\i(x-t)}\right)f(t)\d t,\\
            &=\frac{1}{2\pi}\int_{\R}\frac{2\lambda f(t)}{\lambda^{2}+(t-x)^{2}}\d t.
        \end{align}

        On pose $t'=t-x$ et on a bien 
        \begin{equation}
            \boxed{
                g_x(\lambda)=\frac{1}{2\pi}\int_{\R}\frac{2\lambda f(t+x)}{\lambda^{2}+t^{2}}\d t.
            }
        \end{equation}

        \item On pose $u=\frac{t}{\lambda}$ pour $\lambda>0$. Alors 
        \begin{equation}
            g_x(\lambda)=\frac{1}{2\pi}\int_{\R}\frac{f(\lambda u+x)}{1+u^{2}}\d u,
        \end{equation}
        et pour $u$ fixé, $\lim\limits_{\lambda\to0}\frac{2f(\lambda u+x)}{1+u^{2}}=\frac{2f(x)}{1+u^{2}}$ par continuité de $f$. Comme 
        \begin{equation}
            \left\lvert \frac{2f(\lambda u+x)}{1+u^{2}}\right\rvert\leqslant\frac{2\left\lVert f\right\rVert_{\infty}}{1+u^{2}},
        \end{equation}
        fonction de $u$ intégrable sur $\R$ indépendante de $\lambda$. Par le théorème de continuité, on a
        \begin{equation}
            \boxed{
                \lim\limits_{\lambda\to0}g_x(\lambda)=\frac{1}{2\pi}\int_{\R}\frac{2f(x)}{1+u^{2}}\d u=f(x).
            }
        \end{equation}

        \item Pour tout $\nu\in\R$, $\lambda\mapsto\widehat{f}(\nu)\e^{\i\nu x}\e^{-\lambda\left\lvert \nu\right\rvert}$ est continue sur $\R_{+}$, et on a 
        \begin{equation}
            \left\lvert \widehat{f}(\nu)\e^{\i\nu x}\e^{-\lambda\left\lvert\nu\right\rvert}\right\rvert\leqslant\left\lvert \widehat{f}(\nu)\right\rvert,
        \end{equation}
        fonction indépendante de $\lambda$ intégrable sur $\R$. Par le théorème de continuité, $g_x$ est continue sur $\R_{+}$, donc en 0. Ainsi,
        \begin{equation}
            \boxed{
                g_x(0)=f(x)=\lim\limits_{\lambda\to0}g_x(\lambda)=\frac{1}{2\pi}\int_{\R}\widehat{f}(\nu)\e^{\i\nu x}\d\nu.
            }
        \end{equation}
    \end{enumerate}
\end{proof}

\begin{remark}
    En exemple, soit $a>0$ et \function{f}{\R}{\R}{x}{\e^{-a\left\lvert x\right\rvert}}
    $f$ est continue, intégrable et bornée. On a 
    \begin{align}
        \widehat{f}(\nu)
        &= \int_{\R}f(t)\e^{-\i\nu t}\d t,\\
        &=\frac{2a}{a^{2}+\nu^{2}}.
    \end{align}
    $\widehat{f}$ est intégrable sur $\R$, donc pour tout $x\in\R$, on a 
    \begin{equation}
        2\pi\e^{-a\left\lvert x\right\rvert}=\int_{\R}\frac{2a}{a^{2}+\nu^{2}}\e^{\i\nu x}\d\nu.
    \end{equation}
\end{remark}

\begin{remark}
    Si $\widehat{f}=0$, elle est intégrable et d'après la formule, on a $f=0$.
\end{remark}

\begin{proof}
    \phantom{}
    \begin{enumerate}
        \item $D_n=1+\sum_{k=1}^{n}2\cos(kt)$ est paire, $2\pi$-périodique et $\mathcal{C}^{\infty}$. Sa moyenne est 1, et comme pour $t\in\R\setminus2\pi\Z$, $\e^{\i t}\neq1$, on a 
        \begin{align}
            D_n(t)
            &=\e^{-\i nt}\sum_{k=0}^{2n}\e^{\i kt},\\
            &=\e^{-\i nt}\left(\frac{1-\e^{\i(2n+1)t}}{1-\e^{\i t}}\right),\\
            &=\e^{-\i nt}\frac{\e^{\i \frac{(2n+1)t}{2}}}{\e^{\i \frac{t}{2}}}\left(\frac{\e^{-\i\frac{(2n+1)t}{2}}-\e^{\i\frac{(2n+1)t}{2}}}{\e^{-\i\frac{t}{2}}-\e^{\i\frac{t}{2}}}\right),\\
            &=\e^{-\i nt}\e^{\i nt}\left(\frac{-2\i\sin\left(\frac{(2n+1)t}{2}\right)}{-2\i\sin\left(\frac{t}{2}\right)}\right),\\
            &=\frac{\sin\left(\frac{(2n+1)t}{2}\right)}{\sin\left(\frac{t}{2}\right)}.
        \end{align}

        \item On pose $u=\frac{2t}{2n+1}$.
        \item Elle est de classe $\mathcal{C}^{1}$ sir $]0,\pi]$. Pour tout $u>0$,
        \begin{equation}
            \varphi(u)=\frac{\frac{u}{2}-\sin\left(\frac{u}{2}\right)}{\left(\frac{u}{2}\right)\sin\left(\frac{u}{2}\right)}\underset{u\to0}{\sim}\frac{\frac{1}{6}\left(\frac{u}{2}\right)^{3}}{\left(\frac{u^{2}}{4}\right)}\xrightarrow[u\to0]{}0.
        \end{equation}
        On pose $\varphi(0)=0$. $\varphi$ ainsi prolongée est continue sur $[0,2\pi]$. On a 
        \begin{align}
            \varphi'(u)
            &=-\frac{1}{2}\frac{\cos\left(\frac{u}{2}\right)}{\sin^{2}\left(\frac{u}{2}\right)}+\frac{2}{u^{2}},\\
            &=\frac{4\sin^{2}\left(\frac{u}{2}\right)-u^{2}\cos\left(\frac{u}{2}\right)}{2u^{2}\sin^{2}\left(\frac{u}{2}\right)},\\
            &\underset{u\to0}{\sim}\frac{u^{4}\left(-\frac{1}{12}+\frac{1}{8}\right)}{\left(\frac{u^{4}}{2}\right)}\xrightarrow[u\to0]{}\frac{1}{12}.
        \end{align}

        D'après le théorème de prolongement de la dérivée, $\varphi$ est $\mathcal{C}^{1}$ sur $[0,\pi]$.

        \item On a 
        \begin{align}
            \int_{0}^{\pi}\varphi(u)\sin\left((2n+1)\frac{u}{2}\right)\d u
            &=\left[-\frac{2}{n+1}\cos\left((2n+1)\frac{u}{2}\right)\varphi(u)\right]_{0}^{\pi}\nonumber\\
            &\qquad+\frac{2}{n+1}\int_{0}^{\pi}\varphi'(u)\cos\left((2n+1)\frac{u}{2}\right)\d u,\\
            &\xrightarrow[n\to+\infty]{}0,
        \end{align}
        car $\varphi$ et $\varphi'$ sont bornées. De plus,
        \begin{align}
            \int_{0}^{\pi}\varphi(u)\sin\left((2n+1)\frac{u}{2}\right)\d u
            &=\int_{0}^{\pi}D_n(u)\d u-\int_{0}^{\pi}\frac{2\sin\left((2n+1)\frac{u}{2}\right)}{u}\d u,\\
            &=\frac{1}{2}\int_{-\pi}^{\pi}D_n(u)\d u-2u_n,\\
            &=\pi-2u_n.
        \end{align}
        Donc
        \begin{equation}
            \boxed{
                \lim\limits_{n\to+\infty}u_n=\frac{\pi}{2}=\int_{0}^{+\infty}\frac{\sin(t)}{t}\d t.
            }
        \end{equation}
    \end{enumerate}
\end{proof}

\begin{remark}
    Soit $f\colon I\to\R$ ou $\C$ intégrable sur un intervalle $I$. On a $\lim\limits_{\lambda\to+\infty}\int_{I}f(t)\e^{\i\lambda t}\d t=0$.

    En effet, soit $\varepsilon>0$, il existe $[a,b]\subset I$ tel que $\int_{I\setminus[a,b]}\left\lvert f\right\rvert\leqslant\frac{\varepsilon}{2}$. Alors 
    \begin{equation}
        \left\lvert\int_{I}f(t)\e^{\i\lambda t}\d t\right\rvert\leqslant\int_{I\setminus[a,b]}\left\lvert f\right\rvert+\left\lvert\int_{[a,b]}f(t)\e^{\i\lambda t}\d t\right\rvert,
    \end{equation}
    et le deuxième terme tend vers 0 quand $\lambda\to \pm\infty$, donc inférieur à $\frac{\varepsilon}{2}$ pour $\lambda$ suffisamment grand. En particulier, si $f$ est intégrable sur $\R$, on a $\lim\limits_{\lambda\to+\infty}\widehat{f}(\lambda)=0$.
\end{remark}

\begin{proof}
    \phantom{}
    \begin{enumerate}
        \item $g$ est de classe $\mathcal{C}^{1}$. On a $g'(t)=f(t)\e^{-at}$, $g(0)=0$ et $\lim\limits_{t\to+\infty}g(t)=Lf(a)$. Donc $g$ est bornée sur $\R_{+}$. Soit $X>0$, on a
        \begin{equation}
            \int_{0}^{X}f(t)\e^{-(a+x)t}\d t=\left[\e^{-xt}g(t)\right]_{0}^{X}+x\int_{0}^{X}g(t)\e^{-xt}\d t.
        \end{equation}
        Le terme entre crochet tend vers $0$ quand $X\to+\infty$ car $x>0$ et $\left\lvert g(t)\e^{-xt}\right\rvert\leqslant\left\lVert g\right\rVert_{\infty}\left\lvert\e^{-xt}\right\rvert=\underset{t\to+\infty}{O}\left(\frac{1}{t^{2}}\right)$. Donc $t\mapsto g(t)\e^{-xt}$ est intégrable sur $\R_{+}$. Donc $Lf(a+x)=\int_{0}^{+\infty}f(t)\e^{-\i t(a+x)}\d t$ existe et vaut $x\int_{0}^{+\infty}g(t)\e^{-xt}\d t$.

        \item On pose $u=\e^{-t}$, $\mathcal{C}^{1}$-difféomorphisme de $[0,+\infty[\to]0,1]$. On a alors 
        \begin{equation}
            Lf(a+x)=x\int_{0}^{1}g\left(-\ln(u)\right)u^{x-1}\d u.
        \end{equation}
        Or $\lim\limits_{u\to0^{+}}g\left(-\ln(u)\right)=Lf(a)$. On définit \function{h}{[0,1]}{\R \text{ ou }\C}{u}{
            \left\lbrace
                \begin{array}[]{ll}
                    g\left(-\ln(u)\right) & \text{si }u\neq 0,\\
                    LF(a) & \text{si }u=0.
                \end{array}
            \right.
        }
        $h$ est continue et $Lf(a+x)=x\int_{0}^{1}h(u)u^{x-1}\d u$. Si pour tout $x>0$, $Lf(a+x)=0$, on a pour tout $>0$, $\int_{0}^{1}h(u)u^{x-1}\d u=0$. Par combinaison linéaire, pour tout $P\in\C[X]$, $\int_{0}^{1}\ln(u)P(u)=0$. D'après le théorème de Weierstrass, on prend $(P_n)_{n\in\N}$ une suite de polynômes qui converge uniformément sur $[0,1]$ vers $\overline{h}$. Donc $\int_{0}^{1}\left\lvert h\right\rvert^{2}=0$ donc $h=0$. Ainsi, $g=0$ et $g'(t)=0=f(t)\e^{-at}$ donc $f=0$.
    \end{enumerate}
\end{proof}

\begin{remark}
    Il suffit qu'il existe $a_{0}\in\R$ tel que pour tout $n\in\N^{*}$, $Lf(a_{0}+n)=0$.
\end{remark}

\begin{proof}
    Pour $x>0$, $\sum g_n(x)$ est alternée car $\left\lvert g_n(x)\right\rvert=\e^{-a_n x}$ est décroissante car $a_n\leqslant a_{n+1}$ et tend vers 0 car $\lim\limits_{n\to+\infty}a_n=+\infty$. Donc $g$ est bien définie sur $\R_{+}^{*}$.

    Pour tout $n\in\N$, $g_n$ est continue sur $\R_{+}^{*}$ et pour tout $N\in\N$, d'après le critère spécial des séries alternées,
    \begin{equation}
        \left\lvert\sum_{n=N}^{+\infty}(-1)^{n}\e^{-a_n x}\right\rvert\leqslant\e^{-a_N x}.
    \end{equation}

    Soit $\alpha>0$. Alors pout tout $x\geqslant\alpha$, pour tout $N\in\N$,
    \begin{equation}
        \left\lvert\sum_{n=N}^{+\infty}(-1)^{n}\e^{-a_n x}\right\rvert\leqslant\e^{-a_N x}\leqslant\e^{-a_N \alpha}\xrightarrow[N\to+\infty]{}0,
    \end{equation}
    car $\alpha>0$ et $\lim\limits_{N\to+\infty}a_N=+\infty$.

    Ainsi, $\sum g_n$ converge uniformément sur $[\alpha,+\infty[$, donc $g$ est continue sur $[\alpha,+\infty[$ pour tout $\alpha>0$ donc sur $\R_{+}^{*}$. De plus, toujours d'après le critère spécial des séries alternées, pour tout $x>0$,
    \begin{equation}
        \left\lvert g(x)\right\rvert\leqslant\e^{-a_0 x},
    \end{equation}
    fonction de $x$ intégrable sur $\R_{+}^{*}$ car continue et est $\underset{x\to+\infty}{O}\left(\frac{1}{x^{2}}\right)$. Donc $g$ est intégrable sur $\R_{+}^{*}$. 
    
    On forme $u_n=\int_{0}^{+\infty}\left\lvert g_n(x)\right\rvert\d x=\int_{0}^{+\infty}\e^{-a_n x}\d x=\frac{1}{a_n}$. On n'est pas sûr que $\sum\frac{1}{a_n}$ converge. Mais on a 
    \begin{equation}
        0\leqslant\sum_{n=0}^{N}(-1)^{n}\e^{-a_n x}=S_N(x)\leqslant\e^{-a_0 x},
    \end{equation}
    fonction intégrable sur $\R_{+}^{*}$. D'après le théorème de convergence dominée, on a 
    \begin{equation}
        \lim\limits_{N\to+\infty}\int_{0}^{+\infty}S_N(x)\d x=\sum_{n=0}^{+\infty}\frac{(-1)^{n}}{a_n}=\int_{0}^{+\infty}g(x)\d x.
    \end{equation}
\end{proof}

\begin{proof}
    Soit $x>0$, notons $f_n(x)=(-1)^{n}\e^{-a_n x}$ continue. $\sum_{n\in\N}f_n(x)$ est une série alternée. Pour tout $n\in\N$, on a $\left\lvert f_{n+1}(x)\right\rvert=\e^{-a_{n+1} x}\leqslant\left\lvert f_n(x)\right\rvert$ car $(a_n)_{n\in\N}$ est croissante, $\lim\limits_{n\to+\infty}\left\lvert f_n(x)\right\rvert=0$ car $\lim\limits_{n\to+\infty}a_n=+\infty$ et $x>0$. Donc $S(x)$ est définie sur $\R_{+}^{*}$.

    Pour tout $n\in\N$, $f_n$ est continue. Soit $\alpha>0$, pour tout $x\in[\alpha,+\infty[$, $\left\lvert\sum_{n=N}^{+\infty}f_n(x)\right\rvert\leqslant\e^{-a_{N}x}\leqslant\e^{-a_N \alpha}\xrightarrow[N\to+\infty]{}0$ d'après le critère spécial des séries alternées. Il y a donc convergence uniforme sur $[\alpha,+\infty[$, donc $S$ est continue sur $\R_{+}^{*}$. On a $\left\lvert S(x)\right\rvert\leqslant\e^{-a_0 x}$ intégrable sur $\R_{+}^{*}$ (encore une fois via le critère spécial des séries alternées). Donc $S$ est intégrable sur $\R_{+}^{*}$. Enfin, d'après le critère spécial des séries alternées, $\sum_{n=0}^{+\infty}\frac{(-1)^{n}}{a_n}$ existe.

    Problème : $\sum\int\left\lvert f_n\right\rvert$ ne converge pas forcément, on ne peut pas appliquer le théorème d'interversion.
    On le fait donc à la main : pour tout $N\in\N$, on a 
    \begin{equation}
        \sum_{n=0}^{N}\int_{0}^{+\infty}f_n(x)\d x\xrightarrow[N\to+\infty]{}\sum_{n=0}^{+\infty}\frac{(-1)^{n}}{a_n}.
    \end{equation}
    Soit $R_N(x)=\sum_{n=N}^{+\infty}f_n(x)\d x$. On a $\left\lvert R_N(x)\right\rvert\leqslant\e^{-a_N x}\leqslant \e^{-a_0 x}$ indépendant de $N$ et intégrable sur $\R_{+}^{*}$. D'après le théorème de convergence dominée, on peut intervertir, d'où $\lim\limits_{N\to+\infty}\int_{0}^{+\infty}\sum_{n=N}^{+\infty}f_n(x)\d x=0$.
\end{proof}

\end{document}
\documentclass[12pt]{article}
\usepackage{style/style_sol}

\begin{document}

\begin{titlepage}
	\centering
	\vspace*{\fill}
	\Huge \textit{\textbf{Solutions MP/MP$^*$\\ Séries Entières}}
	\vspace*{\fill}
\end{titlepage}

\begin{proof}
    \phantom{}
    \begin{enumerate}
        \item On pose, pour tout $n\geqslant1$, $u_n=\left(\cosh\left(\frac{1}{n}\right)\right)^{n^{\alpha}}>0$. On va chercher un équivalent. On a $u_n=\e^{n^{\alpha}\ln\left(\cosh\left(\frac{1}{n}\right)\right)}$. Comme $\cosh(x)\underset{0}{=}1+\frac{x^{2}}{2}+O(x^{4})$, on a 
        \begin{align}
            \ln\left(\cosh\left(\frac{1}{n}\right)\right)
            &\underset{+\infty}{=}\ln\left(1+\frac{1}{2n^{2}}+O\left(\frac{1}{n^{4}}\right)\right),\\
            &\underset{+\infty}{=}\frac{1}{2n^{2}}+O\left(\frac{1}{n^{4}}\right).
        \end{align}

        Ainsi, $u_n\underset{+\infty}{=}\e^{\frac{n^{\alpha-2}}{2}+O\left(n^{\alpha-4}\right)}$. Donc :
        \begin{itemize}
            \item si $\alpha<2$, $\lim\limits_{n\to+^infty}u_n=1$ et $R=1$,
            \item si $\alpha=2$, $\lim\limits_{n\to+^infty}u_n=\e^{\frac{1}{2}}$ et $R=1$,
            \item si $\alpha>2$, on a 
            \begin{equation}
                \frac{u_{n+1}}{u_n}\underset{+\infty}{=}\e^{\left(\frac{(n+1)^{\alpha-2}}{2}\right)-\frac{n^{\alpha-2}}{2}+O\left(n^{\alpha-4}\right)}.
            \end{equation}
            Or
            \begin{align}
                \left(n+1\right)^{\alpha-2}-n^{\alpha-2}
                &= n^{\alpha-2}\left(\left(1+\frac{1}{n}\right)^{\alpha-2}-1\right),\\
                &\underset{+\infty}{=}n^{-2}\left(\frac{\alpha-2}{n}+O\left(\frac{1}{n^{3}}\right)\right),\\
                &\underset{+\infty}{=}\left(\alpha-2\right)n^{\alpha-3}+O\left(n^{\alpha-4}\right).
            \end{align}

            Donc $\frac{u_{n+1}}{u_{n}}\underset{+\infty}{=}\e^{\frac{(\alpha-2)n^{\alpha-3}}{2}+O\left(n^{\alpha-4}\right)}$.
            Ainsi,
            \begin{itemize}
                \item si $\alpha<3$, $\lim\limits_{n\to+\infty}\frac{u_{n+1}}{u_{n}}=1$ et $R=1$,
                \item si $\alpha=3$, $\lim\limits_{n\to+\infty}\frac{u_{n+1}}{u_{n}}=\e^{\frac{1}{2}}$ et $R=\e^{-\frac{1}{2}}$,
                \item si $\alpha>3$, comme $\frac{(\alpha-2)n^{\alpha-3}}{2}+O\left(n^{\alpha-4}\right)\underset{+\infty}{\sim}\frac{(\alpha-2)}{2}n^{\alpha-3}$, il existe $N_{0}\in\N$ tel que pour tout $n\geqslant N_{0}$,
                \begin{equation}
                    \frac{(\alpha-2)n^{\alpha-3}}{2}+O\left(n^{\alpha-4}\right)\geqslant \frac{\alpha-2}{4}n^{\alpha-3}\xrightarrow[+\infty]{}+\infty.
                \end{equation}
                Ainsi, $\lim\limits_{n\to+\infty}\frac{u_{n+1}}{u_{n}}=+\infty$ et $R=0$.
            \end{itemize}
        \end{itemize}

        \item On note $u_n=\e^{n^{2}\ln\left(1+\frac{(-1)^{n}}{n^{2}}\right)}>0$. Comme $\ln(1+x)\underset{0}{=}x+O(x^{2})$, on a $u_n\underset{+\infty}{=}\e^{(-1)^{n}n+O\left(\frac{1}{n}\right)}\underset{+\infty}{\sim}\e^{(-1)^{n}n}=v_n$. On ne peut pas appliquer la règle de d'Alembert à $v_n$, ça ne va pas converger. Mais on peut encadrer $v_n$ : $0<v_n\leqslant\e^{n}$ et donc $R\geqslant\frac{1}{\e}$. On a $\frac{u_n}{\e^{n}}\underset{+\infty}{=}\e^{n\left((-1)^{n}-1\right)+O\left(\frac{1}{n}\right)}$ et $\frac{u_{2n}}{\e^{2n}}\xrightarrow[n\to+\infty]{}1$ donc $\sum \frac{u_{n}}{\e^{n}}$ diverge. Ainsi, $R=\frac{1}{\e}$.
    \end{enumerate}
\end{proof}

\begin{proof}
    \phantom{}
    \begin{enumerate}
        \item On remarque
        \begin{equation}
            \underbrace{\begin{pmatrix}
                u_n\\
                u_{n+1}\\
                \dots\\
                u_{n+p-1}
            \end{pmatrix}}_{X_n}
            =
            \underbrace{
                \begin{pmatrix}
                    m_1 & \dots & m_p\\
                    m_{1}\e^{\i\theta_{1}} & \dots & m_p\e^{\i\theta_{p}}\\
                    \vdots & & \vdots\\
                    m_1\e^{\i(p-1)\theta_p} & & m_p\e^{\i(p-1)\theta_p}
                \end{pmatrix}
            }_{A}
            \underbrace{
                \begin{pmatrix}
                    \e^{\i n\theta_1}\\
                    \vdots\\
                    \e^{\i n\theta_p}
                \end{pmatrix}
            }_{Y_n}.
        \end{equation}

        $A$ est inversible car $\det(A)=\left(\prod_{i=1}^{p}m_{i}\right)\times\text{VdM}(\e^{\i\theta_1},\dots,\e^{\i\theta_p})\neq0$. Donc si $u_n\xrightarrow[n\to+\infty]{}0$, on a $X_n\xrightarrow[n\to+\infty]{}0$ et $Y_n=A^{-1}X_{n}\xrightarrow[n\to+\infty]{}0$ ce qui n'est pas car $\left\lVert Y_n\right\rVert_{\infty}=1$.

        \item On pose $\rho(A)=\max\limits_{\lambda\in\Sp(A)}\left\lvert\lambda\right\rvert$. Si $\chi_{A}=(X-\lambda_{1})\dots(X-\lambda_{p})$ avec $\left\lvert \lambda_{1}\right\rvert=\dots=\left\lvert\lambda_{j}\right\rvert=\rho(A)$ et $\left\lvert\lambda_{i}\right\rvert<\rho(A)$ pour tout $i\in\left\lbrace j+1,\dots,p\right\rbrace$. On écrit $a_n=\sum_{i=1}^{p}\lambda_{i}^{n}=\sum_{i=1}^{j}\lambda_{i}^{n}+\sum_{i=j+1}^{p}\lambda_{i}^{n}$. D'après la règle de d'Alembert, on a $R\geqslant\frac{1}{\rho(A)}$ (et $R=+\infty$ si $\rho(A)=0$ et $A$ est nilpotente). De plus, on a 
        \begin{equation}
            \frac{a_n}{\rho(A)^{n}}=\sum_{k=1}^{j}m_{k}\e^{\i n\theta_k}+\sum_{i=j+1}^{p}\left(\frac{\lambda_{i}}{\rho(A)}\right)^{n},
        \end{equation}
        et le premier terme ne tend pas vers 0 d'après ce qui précède tandis que le deuxième tend vers 0. Donc $\sum \frac{a_{n}}{\rho(A)}$ diverge grossièrement, donc $R=\frac{1}{\rho(A)}$.

        Soit $z\in\C$ avec $\left\lvert z\right\rvert<\frac{1}{\rho(A)}$, on a 
        \begin{align}
            \sum_{n=0}^{+\infty}a_{n}z^{n}
            &=\sum_{i=1}^{p}\left(\sum_{n=0}^{+\infty}\lambda_{i}^{n}z^{n}\right),\\
            &=\sum_{i=1}^{p}\frac{1}{1-\lambda_{i}z},\\
            &=\Tr\left(I_{p}-z A\right)^{-1},
        \end{align}
        car pour tout $i\in\left\llbracket1,p\right\rrbracket,\left\lvert \lambda_{i}z\right\rvert<1$ et on peut trigonaliser dans la même base $A$ et $I_{p}-zA$.
    \end{enumerate}
\end{proof}

\begin{proof}
    D'après la règle de d'Alembert, on a $R=1$. De plus, $\left\lvert a_n\right\rvert=\underset{+\infty}{O}\left(\frac{1}{n^{3}}\right)$ donc il y a convergence uniforme sur $\overline{D(0,1)}$. Ainsi, la somme $S$ est continue sur $\overline{D(0,1)}$. Soit $t\in]-1,1[$, comme $\frac{1}{X(X+1)(2X+1)}=\frac{a}{X}+\frac{b}{X+1}+\frac{c}{2X+1}$ avec $a=b=1$ et $c=4$, on a 
    \begin{equation}
        \frac{S(t)}{6}=\sum_{n=1}^{+\infty}\left(\frac{t^{n}}{n}+\frac{t^{n}}{n+1}-4\frac{t^{n}}{2n+1}\right)=-\ln(1-t)+\left(\frac{-\ln(1-t)}{t}-1\right)-4\underbrace{\sum_{n=1}^{+\infty}\frac{t^{n}}{2n+1}}_{g(t)}.
    \end{equation}

    Si $t>0$, on a $\sqrt{t}g(t)=\sum_{n=1}^{+\infty}\frac{(\sqrt{t})^{2n+1}}{2n+1}$. On pose $h(x)=\sum_{n=1}^{+\infty}\frac{x^{2n+1}}{2n+1}$, $\mathcal{C}^{\infty}$ sur $[0,1[$ et $h(0)=0$. On a $h'(x)=\sum_{n=1}^{+\infty}x^{2n}=\frac{x^{2}}{1-x^{2}}=-1-\frac{1}{2}\frac{1}{x+1}+\frac{1}{2}\frac{1}{x-1}$ donc $h(x)=-x-\frac{1}{2}\ln(x+1)+\frac{1}{2}\ln(x-1)$ d'où $g(t)=-\sqrt{t}-\frac{1}{2}\ln(\sqrt{t}+1)+\frac{1}{2}\ln(\sqrt{t}-1)$.

    Si $t<0$, $\sqrt{-t}g(t)=\sum_{n=1}^{+\infty}\frac{(-1)^{n}\sqrt{-t}^{2n+1}}{2n+1}=\arctan(\sqrt{-t})-\sqrt{-t}$. Donc $g(t)=\frac{\arctan(\sqrt{-t})}{\sqrt{-t}}-1$. L'expression de $S$ reste valable en -1 et 1 par continuité de $S$.
\end{proof}

\begin{proof}
    Soit $t\in]-1,1[$, on a 
    \begin{align}
        I(t)
        &=\int_{0}^{1}\e^{u\ln(1+t)}\d u,\\
        &=\left[\frac{1}{\ln(1+t)}\e^{u\ln(1+t)}\right]_{u=0}^{u=1},\\
        &=\frac{1+t}{\ln(1+t)}-\frac{1}{\ln(1+t)},\\
        &=\frac{t}{\ln(1+t)}=f(t).
    \end{align}

    Soit $u\in[0,1]$, on a $(1+t)^{u}=\sum_{n=0}^{+\infty}\frac{u(u-1)\dots(u-n+1)}{n!}t^{n}=\sum_{n=0}^{+\infty}f_n(u)$. $f_n$ est continue sur $[0,1]$. On a 
    \begin{align}
        \left\lvert f_n(u)\right\rvert
        &=\frac{u(1-u)\dots(n-1-u)}{n!}\left\lvert t\right\rvert^{n},\\
        &\leqslant\frac{(n-1)!}{n!}\left\lvert t\right\rvert^{n},\\
        &=\frac{1}{n}\left\lvert t\right\rvert^{n},\\
        &\leqslant\left\lvert t\right\rvert^{n},
    \end{align}
    car pour tout $u\in[0,1]$, $0\leqslant k-u\leqslant k$. Comme $\left\lvert t\right\rvert<1$, $\left\lvert t\right\rvert^{n}$ est le terme général d'une série convergente. Donc $\sum f_{n}$ converge normalement sur $[0,1]$ et on peut intervertir :
    \begin{equation}
        f(t)=\sum_{n=0}^{+\infty}\underbrace{\left(\int_{0}^{1}\frac{u(u-1)\dots(u-n+1)}{n!}\d u\right)}_{a_n}t^{n},
    \end{equation}
    encore vrai pour $t=0$ car $a_{0}=1$. Donc $f$ est développable en série entière sur $]-1,1[$ et $f$ est $\mathcal{C}^{\infty}$ sur $]-1,1[$. Par ailleurs, $f$ est $\mathcal{C}^{\infty}$ sur $\left[\frac{1}{2},+\infty\right[$, donc $f$ est $\mathcal{C}^{\infty}$ sur $]-1,+\infty[$.
\end{proof}

\begin{remark}
    On a 
    \begin{equation}
        a_n=\frac{(-1)^{n}}{n!}\int_{0}^{1}u(1-u)\dots(n-1-u)\d u.
    \end{equation}
    De plus,
    \begin{align}
        \left\lvert a_{n+1}\right\rvert
        &=\frac{1}{(n+1)!}\int_{0}^{1}u(1-u)\dots(n-1-u)\underbrace{(n-u)}_{\leqslant n}\d u,\\
        &\leqslant \frac{1}{n!}\int_{0}^{1}u(1-u)\dots(n-1-u)\d u=\left\lvert a_n\right\rvert.
    \end{align}

    Enfin, $\left\lvert a_n\right\rvert\leqslant\frac{(n-1)!}{n!}=\frac{1}{n}$. D'après le critère spécial des séries alternées, $\sum a_{n}$ converge. Puis $\sum a_{n}t^{n}$ converge uniformément sur $[0,1]$ (majorer le reste par le critère spécial des séries alternées), donc il y a convergence et continuité en 1. On vérifie que $\left\lvert a_n\right\rvert=\frac{1}{n}\int_{0}^{1}u\e^{\sum_{k=1}^{n-1}\ln\left(1-\frac{u}{k}\right)}\d u=\frac{1}{n}\int_{0}^{1}\e^{-\ln(n)u+g_n(u)}$, où $g_n(u)$ est majorée par $M$ indépendant de $n$ et de $u$. Ainsi, par convergence dominée, $\left\lvert a_n\right\rvert\underset{+\infty}{\sim}\frac{1}{n}\int_{0}^{1}\frac{u}{n^{u}}\d u$, terme général d'une série divergente.
\end{remark}

\begin{proof}
    On a $a_n=\e^{\ln(n)\ln\left(\sum_{k=1}^{n}\frac{1}{k}\right)}\underset{+\infty}{=}\e^{\ln(n)\ln\left(\ln(n)+\gamma+o(1)\right)}$.
    On a 
    \begin{align}
        \ln\left(\ln(n)+\gamma+o(1)\right)
        &=\ln(\ln(n))+\ln\left(1+\frac{\gamma}{\ln(n)}+O\left(\frac{1}{\ln(n)}\right)\right),\\
        &=\ln(\ln(n))+\frac{\gamma}{\ln(n)}+o\left(\frac{1}{\ln(n)}\right).
    \end{align}

    Donc $a_n=\e^{\ln(n)\ln(\ln(n))+\gamma+o(1)}\underset{+\infty}{\sim}\e^{\gamma}\underbrace{\e^{\ln(n)\ln(\ln(n))}}_{b_n}$. On a 
    \begin{equation}
        \frac{b_{n+1}}{b_n}=\e^{\ln(n+1)\ln(\ln(n+1))-\ln(n)\ln(\ln(n))},
    \end{equation}
    mais 
    \begin{equation}
        \ln(n+1)=\ln(n)+\ln\left(1+\frac{1}{n}\right)\underset{+\infty}{=}\ln(n)+O\left(\frac{1}{n}\right),    
    \end{equation}
    et
    \begin{equation}
        \ln(n+1)\ln(\ln(n+1))=\ln(n)\ln(\ln(n+1))+\underbrace{O\left(\frac{\ln(\ln(n+1))}{n}\right)}_{=o(1)\xrightarrow[+\infty]{}0},
    \end{equation}
    puis
    \begin{align}
        \ln(\ln(n+1))
        &\underset{+\infty}{=}\ln\left(\ln(n)+O\left(\frac{1}{n}\right)\right),\\
        &\underset{+\infty}{=}\ln(\ln(n))+\ln\left(1+O\left(\frac{1}{n\ln(n)}\right)\right),\\
        &\underset{+\infty}{=}\ln(\ln(n))+O\left(\frac{1}{n\ln(n)}\right).
    \end{align}

    Donc $\ln(n+1)\ln(\ln(n+1))-\ln(n)\ln(\ln(n))=\underset{+\infty}{o}(1)$, et $\frac{b_{n+1}}{b_{n}}\lim\limits_{n\to+\infty}1$, d'où $R=1$.

    De plus, $\lim\limits_{n\to+\infty}a_n=+\infty$ donc il y a divergence sur le cercle de convergence.
\end{proof}

\begin{remark}
    On peut aussi écrire $a_n\leqslant n^{\ln(n)}=\e^{(\ln(n))^{2}}=c_n$, et 
    \begin{equation}
        \frac{c_{n+1}}{c_{n}}=\e^{\left(\ln(n+1)\right)^{2}-\left(\ln(n)\right)^{2}}\underset{+\infty}{=}\e^{\left(\ln(n)+O\left(\frac{1}{n}\right)\right)^{2}-(\ln(n))^{2}}\xrightarrow[n\to+\infty]{}1.
    \end{equation}
    Donc $\sum c_{n}z^{n}$ a pour rayon de convergence 1, donc $R\geqslant1$, et $\sum a_{n}$ diverge donc $R=1$.
\end{remark}

\begin{proof}
    Le nombre de diviseurs est compris entre 1 et $n$. Comme $\sum z^{n}$ et $\sum nz^{n}$ ont un rayon de convergence égal à 1, on a $R=1$ par encadrement.
\end{proof}

\begin{proof}
    On pose $u_n=\frac{a_{n+1}}{a_{n}}$. Alors $\frac{a_{n-1}a_{n+1}}{a_{n}^{2}}=\frac{u_{n}}{u_{n-1}}$. 
    \begin{itemize}
        \item Si $l<1$, alors d'après la règle de d'Alembert, $\sum u_{n}$ converge donc $\lim\limits_{n\to+\infty}u_{n}=0$ donc $R=+\infty$.
        \item Si $l>1$, il existe $N_{0}\in\N$ tel que pour tout $n\geqslant N_{0}$, $\frac{u_{n}}{u_{n-1}}\geqslant\frac{l+1}{2}$ et pour tout $n\geqslant N_{0}$, $u_{n}\geqslant u_{N_{0}}\times\left(\frac{l+1}{2}\right)^{n-N_{0}}\xrightarrow[n\to+\infty]{}+\infty$ donc $R=0$.
        \item Si $l=1$ : si $a_n=n!$, on a $u_n=n+1$ donc $R=0$, si $a_n=\frac{1}{n!}$, on a $u_n=\frac{1}{n+1}$ donc $R=+\infty$, si $a_n=\lambda^{n}$ avec $\lambda>0$, on a $u_n=\lambda$ et $R=\frac{1}{\lambda}$. Donc on ne peut rien dire.
    \end{itemize}
\end{proof}

\begin{proof}
    D'après la règle de d'Alembert, avec $a_{n}=\frac{(-1)^{n}}{n}$, on a $\left\lvert \frac{a_{n+1}}{a_{n}}\right\rvert=\frac{n}{n+1}\xrightarrow[n\to+\infty]{}1$ donc le rayon de convergence de $\phi$ est $R=1$ donc $\phi$ est bien définie.

    Fixons $z\in\C^{*}$ avec $\left\lvert z\right\rvert<1$, formons \function{f}{[0,1]}{\C}{t}{\e^{\phi(tz)}}
    $z$ étant fixé, le rayon de convergence de la série entière $\sum_{n\geqslant1}(-1)^{n-1}\frac{t^{n}z^{n}}{n}=\phi(tz)$ vaut $\frac{1}{\left\lvert z\right\rvert}>1$, donc l'application $t\mapsto\phi(tz)$ est $\mathcal{C}^{\infty}$ sur $[0,1]\subset\left]-\frac{1}{\left\lvert z\right\rvert},\frac{1}{\left\lvert z\right\rvert}\right[$. $f$ est donc $\mathcal{C}^{\infty}$ sur $[0,1]$ et pour tout $t\in[0,1]$,
    \begin{equation}
        f'(t)=\sum_{n=1}^{+\infty}(-1)^{n-1}z^{n}t^{n-1}\times f(t)=\frac{z}{1+tz}f(t),
    \end{equation}
    car $\left\lvert zt\right\rvert<1$ et $f(0)=1$. On pose $g(t)=1+tz$. Alors $g'(t)=z=\frac{z}{1+tz}g(t)$ et $g(0)=1$. Ainsi, par unicité (d'après le théorème de Cauchy-Lipschitz), pour tout $t\in[0,1]$, $f(t)=g(t)$. En particulier, $f(1)=\e^{\phi(z)}=1+z$.
\end{proof}

\begin{remark}
    On vient de définir, pour $\left\lvert z\right\rvert<1$, $\phi(z)$ qui est un logarithme complexe continue de $1+z$. Si $1+z=\rho\e^{\i\theta}$ avec $\theta\in]-\pi,\pi[$, $\phi(z)=\ln(\rho)+\i\theta$.
\end{remark}

\begin{proof}
    On a $a_n=\frac{1}{\cos\left(\frac{2n\pi}{3}\right)}$ et $1\leqslant\left\lvert a_n\right\rvert\leqslant2$ donc $R=1$. Si $\left\lvert z\right\rvert<1$, on a 
    \begin{equation}
        \sum_{n=0}^{+\infty}z^{3n}-2\left(\sum_{n=0}^{+\infty}-z^{3n+1}+z^{3n+2}\right)=\frac{1}{1-z^{3}}+\frac{2z}{1-z^{3}}-\frac{2z^{2}}{1-z^{3}}=\frac{1+2z-2z^{2}}{1-z^{3}}.
    \end{equation}
\end{proof}

\begin{proof}
    \phantom{}
    \begin{enumerate}
        \item On a $b_n\geqslant0$ donc $g$ est croissante sur $[0,1[$. $g$ admet donc une limite $l\in\overline{\R}_{+}$ en $1^{-}$. Pour tout $x<1$, $g(x)\leqslant l$. Pour tout $N\in\N$, pour tout $x\in[0,1[$, comme $b_nx^{n}\geqslant0$, on a $\sum_{n=0}^{N}b_{n}x^{n}\leqslant g(x)\leqslant l$. $N$ étant fixé, quand $x\to1$, on a $\sum_{n=0}^{N}b_n\leqslant l$ et quand $N\to+\infty$, on a $l=+\infty$.
        \item Soit $\varepsilon>0$, il existe $n_0\in\R$, pour tout $n\geqslant n_0$, $\left\lvert a_n-b_n\right\rvert<\frac{\varepsilon}{2}\times b_n$. Pour tout $x\in[0,1[$, on a $\left\lvert f(x)-g(x)\right\rvert\leqslant\sum_{n=0}^{n_{0}-1}\left\lvert a_n-b_n\right\rvert x^{n}$ + $\sum_{n=n_0}^{+\infty}\left\lvert a_n-b_n\right\rvert x^{n}$. Le terme de gauche est en polynôme en $x$ qui a une limite finie en $1^{-}$, le terme de droite majoré par $\frac{\varepsilon}{2}\sum_{n=n_0}^{+\infty}b_nx^{n}\leqslant\frac{\varepsilon}{2}g(x)$, car les $b_n$ sont positifs. Ainsi, ce terme de droite est un $\underset{x\to1^{-}}{O}(g(x))$ donc majoré par $\frac{\varepsilon}{2}g(x)$ pour $x$ suffisamment proche de 1, d'où $\left\lvert f(x)-g(x)\right\rvert\leqslant\varepsilon g(x)$ et $f(x)\underset{1^{-}}{\sim}g(x)$.
        \item On a $n^{p}\underset{+\infty}{\sim}n(n-1)\dots(n-p+1)$, donc 
        \begin{equation}
            h_p(x)\underset{1}{\sim}\sum_{n=0}^{+\infty}n(n-1)\dots(n-p+1)x^{n}=\sum_{n=p}^{+\infty}n(n-1)\dots(n-p+1)x^{n},
        \end{equation}
        et $f(x)=\frac{1}{1-x}=\sum_{n=0}^{+\infty}x^{n}$, $f'(x)=\frac{1}{(1-x)^{2}}=\sum_{n=1}^{+\infty}nx^{n-1}$. De proche en proche, on a $f^{(p)}(x)=\frac{p!}{(1-x)^{p+1}}=\sum_{n=p}^{+\infty}n\dots(n-p+1)x^{n-p}$, d'où 
        \begin{equation}
            \boxed{
                h_p(x)\underset{1}{\sim}\frac{p!}{(1-x)^{p+1}}.
            }
        \end{equation}
    \end{enumerate}
\end{proof}

\begin{proof}
    Soit $\varepsilon>0$. Il existe $N_{0}\in\N$ tel que pour tout $n\geqslant N_{0}$, $\left\lvert a_n\right\rvert\leqslant\frac{\varepsilon}{n}$. Alors si $S_n=\sum_{h=0}^{n}a_h$, on a 
    \begin{equation}
        \left\lvert S_n-S\right\rvert\leqslant \left\lvert S_n-f\left(1-\frac{1}{n}\right)\right\rvert+\left\lvert f\left(1-\frac{1}{n}\right)-S\right\rvert.
    \end{equation}
    Puisque $f\left(1-\frac{1}{n}\right)\xrightarrow[n\to+\infty]{}S$, il existe $N_{1}\in\N$ tel que pour tout $n\geqslant N_{1}$, $\left\lvert f\left(1-\frac{1}{n}\right)-S\right\rvert\leqslant\frac{\varepsilon}{4}$. Pour $n\geqslant N_{0}$, on a alors 
    \begin{equation}
        \left\lvert S_n-f\left(1-\frac{1}{n}\right)\right\rvert\leqslant A_n+B_n+C_n,
    \end{equation}
    avec $A_n=\sum_{h=0}^{N_{0}}\left\lvert a_h\right\rvert\left(1-\left(1-\frac{1}{n}\right)^{h}\right)\xrightarrow[n\to+\infty]{}0$ et il existe $N_{1}$ pour tout $n\geqslant N_{1}$, $A_n\leqslant\frac{\varepsilon}{4}$. On a 
    \begin{align}
        B_n
        &= \sum_{h=N_{0}+1}^{n}\left\lvert a_{h}\right\rvert\left(1-\left(1-\frac{1}{n}\right)^{h}\right),\\
        &\leqslant\frac{\varepsilon}{4}\sum_{h=N_{0}+1}^{n}\left(\frac{1}{h}\times h\left(1-\left(1-\frac{1}{n}\right)\right)\right),\\
        &\leqslant\frac{\varepsilon}{4}\sum_{h=N_{0}}^{n}\frac{1}{n},\\
        &\leqslant\frac{\varepsilon}{4}\times\frac{n-N_{0}}{n},\\
        &\leqslant\frac{\varepsilon}{4}.
    \end{align}
    Cela est dû au fait que $x\mapsto1-x^{h}$ est concave sur $[0,1]$ donc $\left(1-\left(1-\frac{1}{n}\right)^{h}\right)\leqslant h\left(1-\left(1-\frac{1}{n}\right)\right)$ (ou par accroissement fini). Enfin, on a 
    \begin{align}
        C_n
        &=\sum_{h\geqslant n}a_h\left(1-\frac{1}{n}\right)^{h},\\
        &\leqslant\frac{\varepsilon}{4}\sum_{h\geqslant n}\frac{\left(1-\frac{1}{n}\right)^{h}}{h},\\
        &\leqslant\frac{\varepsilon}{4n}\sum_{h\geqslant n}\left(1-\frac{1}{n}\right)^{h},\\
        &\leqslant\frac{\varepsilon}{4n}\sum_{h=0}^{+\infty}\left(1-\frac{1}{n}\right)^{h},\\
        &=\frac{\varepsilon}{4}.
    \end{align}

    Ainsi, on a $\left\lvert S_n-S\right\rvert\leqslant\varepsilon$ et donc $S_n\xrightarrow[n\to+\infty]{}S$.
\end{proof}

\begin{remark}
    C'est une réciproque du lemme d'Abel radial i.e.~si $\sum a_n$ converge alors \begin{equation}
        \lim\limits_{x\to1^{-}}\sum_{n=0}^{+\infty}a_nx^{n}=\sum_{n=0}^{+\infty}a_n.
    \end{equation}
\end{remark}

\begin{remark}
    Ce n'est pas valable par exemple pour $a_n=(-1)^{n}$, car $f(x)=\frac{1}{1+x}\xrightarrow[x\to1^{-}]{}\frac{1}{2}$ mais $\sum(-1)^{n}$ diverge.
\end{remark}

\begin{proof}
    On note $f(z)=\sum_{n=0}^{+\infty}a_{n}z^{n}$ avec $a_0=f(0)=\rho\e^{\i\theta}\neq0$. Alors
    \begin{equation}
        f(z)=f(0)\left(1+\underbrace{\sum_{n=1}^{+\infty}\frac{a_n}{a_0}z^{n}}_{=g(z)}\right),
    \end{equation}
    avec $g(z)\xrightarrow[z\to0]{}0$ car $g$ est somme d'une série entière donc continue. Il existe $r>0$, si $\left\lvert z\right\rvert<r$, $\left\lvert g(z)\right\rvert<1$. Alors on a vu, d'après l'Exercice 8, que l'on a 
    \begin{equation}
        f(z)=\exp\left(\ln\rho+\i\theta+\sum_{p=1}^{+\infty}(-1)^{p-1}\frac{g(z)^{p}}{p}\right).
    \end{equation}
    Pour $p\in\N$ fixé, on peut développer chaque terme $g(z)^{p}=\sum_{n=0}^{+\infty}a_{n,p}z^{n}$ (produit de Cauchy). On vérifie alors (théorème de Fubini) que l'on peut intervertir les sommations.
\end{proof}

\begin{remark}
    Autre méthode : si $T$ existe avec $T(z)=\sum_{n=0}^{+\infty}b_nz^{n}$. Pour $t\in]-r,r[$, on a $f(t)=\e^{T(t)}$. En dérivant, on a $f'(t)=T'(t)f(t)=\left(\sum_{n}(n+1)b_{n+1}t^{n}\right)f(t)$. Par unicité de développement, et par produit de Cauchy, pour tout $n\in\N$, on a 
    \begin{align}
        (n+1)a_{n+1}
        &=\sum_{h=0}^{n}(h+1)b_{h+1}a_{n-h},\\
        &=(n+1)b_{n+1}\underbrace{a_{0}}_{\neq0}+\sum_{h=1}^{n}hb_{h}a_{n-h+1}.
    \end{align}
    On a $b_{0}=T(0)$, on choisit $b_{0}$ tel que $\e^{b_{0}}=a_{0}\neq0$ et on définit univoquement $(b_n)_{n\in\N}$ par récurrence. On vérifie alors, en majorant, que $\sum b_{n}z^{n}$ a un rayon de convergence $r>0$ (montrer qu'il existe $M\geqslant0,A\geqslant0$ tels que pour tout $n\in\N$, $\left\lvert b_n\right\rvert AM^{n}$). Alors $f'(t)=T'(t)f(t)$ et en posant $g(t)=\e^{T(t)}$, on a $g=f$ par unicité via le théorème de Cauchy-Lipschitz.
\end{remark}

\begin{proof}
    \phantom{}
    \begin{enumerate}
        \item Pour tout $n\geqslant1$, on a $\left\lvert\frac{1}{\sin(n\pi a)}\right\rvert\geqslant1$, donc $R_a\leqslant1$.
        \item On rappelle que si $a$ est irrationnel algébrique de degré $d\geqslant2$, il existe $C>0$ tel que pour tout $\frac{p}{q}\in\Q$, on a $\left\lvert a-\frac{p}{q}\right\rvert\geqslant\frac{C}{q^{d}}$. Soit $n\in\N^{*}$. On fixe $p\in\N$ tel que $n\pi a-p\pi\in\left]-\frac{\pi}{2},\frac{\pi}{2}\right[$. On a alors
        \begin{align}
            \left\lvert \sin(n\pi a)\right\rvert
            &=\left\lvert\sin(n\pi a-p\pi)\right\rvert,\\
            &\geqslant\frac{2}{\pi}\left\lvert n\pi a-p\pi\right\rvert,\\
            &\geqslant2\left\lvert na-p\right\rvert,\\
            &\geqslant2n\frac{C}{n^{d}}=\frac{2C}{n^{d-1}},
        \end{align}
        car par concavité, on a pour tout $t\in\left[-\frac{\pi}{2},\frac{\pi}{2}\right],\left\lvert\sin(t)\right\rvert\geqslant\frac{2}{\pi}\left\lvert t\right\rvert$. On a donc $\left\lvert a_n\right\rvert\leqslant\frac{n^{d-1}}{2C}$, et comme le rayon de convergence de $\sum \frac{n^{d-1}}{2C}z^{d-1}$ vaut 1, on a $R_a=1$.
        \item On a $\left\lvert\sin(n!\pi e)\right\vert=\left\lvert\sin\left(n!\pi\sum_{k=0}^{+\infty}\frac{1}{k!}\right)\right\rvert\underset{+\infty}{=}\left\lvert\sin\left(\frac{\pi}{n+1}+O\left(\frac{1}{n^{2}}\right)\right)\right\rvert\underset{+\infty}{\sim}\frac{\pi}{n}$. Pour $x\in]0,1]$, $\sum nx^{n!}$ converge. L'idée est donc de former $a$ tel que pour tout $x\in]0,1]$, on puisse extraire 
        \begin{equation}
            \left(\frac{x^{\sigma(n)}}{\sin\left(\sigma(n)\pi a\right)}\right)_{n\in\N},
        \end{equation} 
        qui ne tend pas vers 0.
        
        \begin{lemma}
            \label{lem:serie_entiere_1}
            Soit $(a_n)_{n\in\N}\in\left(\N^{*}\right)^{\N}$ strictement croissante, et 
            \begin{equation}
                a=\sum_{n=0}^{+\infty}\frac{1}{a_0\dots a_n}.
            \end{equation}
            On a 
            \begin{equation}
                a-\sum_{k=0}^{N}\frac{1}{a_0\dots a_k}\underset{N\to+\infty}{\sim}\frac{1}{a_0\dots a_{N+1}}.
            \end{equation}
        \end{lemma}
        \begin{proof}[Preuve du Lemme~\ref{lem:serie_entiere_1}]
            On a pour tout $n\in\N^{*}$, $\frac{1}{a_{0}\dots a_{n}}\leqslant\frac{1}{a_{0}a_{1}^{n}}$ et $a_{1}\geqslant2$ donc $\sum_{n\geqslant0}\frac{1}{a_{0}\dots a_{n}}$ converge. On a
            \begin{equation}
                \left\lvert a-\sum_{n=0}^{N}\frac{1}{a_{0}\dots a_{n}}\right\rvert=\sum_{k=N+1}^{+\infty}\frac{1}{a_0\dots a_k},
            \end{equation}
            donc 
            $\frac{1}{a_0\dots a_{N+1}}\leqslant \sum_{k=N+1}^{+\infty}\frac{1}{a_0\dots a_k}\leqslant\frac{1}{a_0\dots a_N}\sum_{k=1}^{+\infty}\frac{1}{a_{N+1}^{k}}=\frac{1}{a_0\dots a_N}\times\frac{1}{a_{N+1}}\times\frac{1}{1-\frac{1}{a_{N+1}}}$. Donc 
            \begin{equation}
                a-\sum_{k=0}^{N}\frac{1}{a_0\dots a_k}\underset{N\to+\infty}{\sim}\frac{1}{a_0\dots a_{N+1}}.
            \end{equation}
        \end{proof}

        On a donc $(a_{0}\dots a_{N})a-\underbrace{(a_0\dots a_{N})\sum_{k=0}^{N}\frac{1}{a_{0}\dots a_{k}}}_{\in\N}\underset{N\to+\infty}{\sim}\frac{1}{a_{N+1}}$. Ainsi,
        \begin{equation}
            \left\lvert\sin\left(\underbrace{(a_0\dots a_N)}_{=\sigma(N)}\pi a\right)\right\rvert=\left\lvert\sin\left((a_0\dots a_N)\pi a-(a_0\dots a_N)\sum_{k=0}^{N}\frac{\pi}{a_{0}a_{k}}\right)\right\rvert\underset{N\to+\infty}{\sim}\frac{\pi}{a_{N+1}}.
        \end{equation}
        Pour $x\in]0,1]$, on a $\frac{x^{\sigma(N)}}{\left\lvert\sin(\sigma(N)\pi a)\right\rvert}\underset{N\to+\infty}{\sim}\frac{1}{\pi}\exp\left(\sigma(N)\ln(x)+\ln(a_{N+1})\right)$. Il suffit de choisir $a_{N+1}$ tel que $\ln(a_{N+1})\geqslant N(a_{0}\dots a_{N})$, par exemple $a_{N+1}=\left\lfloor \e^{N(a_{0}\dots a_{N})}\right\rfloor+1$. Donc pour tout $x\in]0,1]$, $\lim\limits_{N\to+\infty}\frac{x^{\sigma(N)}}{\left\lvert\sin(\sigma(N)\pi a)\right\rvert}=+\infty$. Ainsi, $R_a=0$.
    \end{enumerate}
\end{proof}

\begin{proof}
    Pour $\left\lvert z\right\rvert<1$, par produit de Cauchy, ces séries sont définies et absolument convergentes, par sommabilité,
    \begin{equation}
        \left(\sum_{p_1=0}^{+\infty}z^{a_1 p_1}\right)\times\dots\times\left(\sum_{p_N=0}^{+\infty}z^{a_Np_N}\right)-\frac{1}{(1-z^{a_1})\dots(1-z^{a_N})}=\sum_{(p_1,\dots,p_N)\in\N^{N}}z^{a_1p_1+\dots+a_Np_N}.
    \end{equation}
    Par associativité, on regroupe selon les valeurs de l'exposant et on note l'expression précédente $\sum_{n=0}^{+\infty}c_nz^{n}$. On factorise la fraction rationnelle [les pôles sont des racines de l'unité] :
    \begin{equation}
        \frac{1}{\prod_{\xi\in\U}(z-\xi)^{m\left(\xi\right)}},
    \end{equation}
    avec $m(1)=N$, $m\left(\xi\right)<N$ si $\xi\neq1$ car $a_1\wedge\dots\wedge a_N=1$ : si $\xi^{a_1}=\dots=\xi^{a_N}=1$, l'ordre de $\xi$ divise $a_1,\dots,a_N$ donc divise $a_1\wedge\dots\wedge a_N=1$. Cette expression vaut alors 
    $\sum_{k=1}^{N}\frac{\alpha_{1,k}}{(-z+1)^{k}}+\sum_{\xi\in\U\setminus\left\lbrace1\right\rbrace}\left(\sum_{k=1}^{N-1}\frac{\alpha_{\xi,k}}{(-z+\xi)^{k}}\right)$ (somme finie). Pour $\left\lvert z\right\rvert<1$, on a 
    \begin{equation}
        \frac{1}{\left(-z+\xi\right)^{k}}=\left(-\frac{1}{\xi}\right)^{k}\sum_{n=0}^{+\infty}\frac{(n-k+1)\dots(n+1)}{(k-1)!}\left(\frac{z}{\xi}\right)^{n}.
    \end{equation}
    Ainsi, le coefficient en $z^{n}$ et équivalent à $\frac{n^{k-1}}{(k-1)!}\left(-\frac{1}{\xi}\right)^{k}$ en $+\infty$. Donc $c_n$ est un polynôme en $n$, équivalent en $+\infty$ à $\alpha_{1,N}\times \frac{n^{N-1}}{(n-1)!}$.

    Si $F=\frac{1}{(1-X^{a_1})\dots (1-X^{a_N})}$, en évaluant $(1-X)^{N}F$ et en prenant la limite en $X\to 1$, on a $\frac{X^{a_k}-1}{X-1}=1+X+\dots+X^{a_k-1}\xrightarrow[X\to1]{}a_k$. Finalement, $\alpha_{1,N}=\frac{1}{\prod_{k=1}^{n}a_k}$ et $c_n\geqslant1$ pour $n$ suffisamment grand. Ainsi,
    \begin{equation}
        \boxed{
            c_n\underset{+\infty}{\sim}\frac{n^{N-1}}{\left(\prod_{k=1}^{N}a_k\right)(N-1)!}.
        }
    \end{equation}
\end{proof}

\begin{proof}
    $f$ est $\mathcal{C}^{\infty}$ sur $\R$ par somme et composée. Pour $x\neq1$, on a 
    \begin{equation}
        f(x)=\sqrt{\frac{1-x^{3}}{1-x}}=\sqrt{1-x^{3}}\times\sqrt{\frac{1}{1-x}},
    \end{equation}
    produit de deux fonctions développable en série entière sur $]-1,1[$. Il existe donc $(a_n)_{n\in\N}\in\R^{\N}$ telle que pour tout $x\in]-1,1[$, $f(x)=\sum_{n=0}^{+\infty}a_nx^{n}$. On a $f^{2}(x)=1+x+x^{2}$ et $(f^{2})'(x)=2f'(x)f(x)=1+2x$ d'où pour tout $x\in]-1,1[$,
    \begin{equation}
        2\left(\sum_{n=0}^{+\infty}(n+1)a_{n+1}x^{n}\right)\left(\sum_{n=0}^{+\infty}a_nx^{n}\right)=1+2x,
    \end{equation}
    encore vrai pour $z\in D(0,1)$ par unicité du développement en série entière.

    Si $R>1$, me rayon de convergence de $\sum(n+1)a_{n+1}z^{n}$ est $R$. On aurait alors pour tout $z\in D(0,R)$
    \begin{equation}
        2\left(\sum_{n=0}^{+\infty}(n+1)a_{n+1}z^{n}\right)\left(\sum_{n=0}^{+\infty}a_nz^{n}\right)=1+2z,
    \end{equation}
    i.e.~si $S(z)=\sum_{n=0}^{+\infty}a_nz^{n}$, alors $2S'(z)S(z)=1+2z$. En $\j$, on a $2S'(\j)S(\j)=1+2\j$. Comme pour tout $x\in]-1,1[$, $S^{2}(x)=1+x+x^{2}$, par unicité, on a pour tout $z\in D(0,R)$, $S^{2}(z)=1+z+z^{2}$. Donc $S^{2}(\j)=1+\j+\j^{2}=0$ d'où $S(\j)=0$ : impossible car sinon $0=1+2\j$. Ainsi, $R=1$.
\end{proof}

\begin{proof}
    \phantom{}
    \begin{enumerate}
        \item $\sum_{k\in\N}\frac{f^{(k)(0)}}{k!}x^{k}$ est une série à termes positifs, d'après la formule de Taylor reste intégral, on a 
        \begin{equation}
            f(x)=\underbrace{\sum_{k=0}^{n}\frac{f^{(k)}(0)}{k!}x^{k}}_{S_n(x)}+\underbrace{\int_{0}^{x}\frac{(x-t)^{n}}{n!}f^{(n+1)}(t)\d t}_{R_n(x)\geqslant0}.
        \end{equation}
        On a $0\leqslant S_n(x)\leqslant f(x)$, donc la série converge et la suite $(R_n(x))_{n\in\N}$ converge aussi.

        \item On pose $t=xu$ et on a 
        \begin{equation}
            R_n(x)=x^{n+1}\int_{0}^{1}\frac{(1-u)^{n}}{n!}f^{(n+1)}(u)\d u.
        \end{equation}
        Pour tout $t\in[0,A[$, $f^{(n+2)}(t)\geqslant0$, $f^{(n+1)}$ est croissante. On a donc 
        \begin{equation}
            0\leqslant R_n(x)\leqslant\frac{x^{n+1}}{y^{n+1}}y^{n+1}\int_{0}^{1}\frac{(1-u)^{n}}{n!}f^{(n+1)}(xu)\d u,    
        \end{equation}
        d'où $0\leqslant R_n(x)\leqslant\left(\frac{x}{y}\right)^{n+1}R_n(y)$.

        \item $(R_n(y))_{n\in\N}$ est bornée d'après a), donc $R_n(x)\xrightarrow[x\to0]{}0$ d'où $f(x)=\sum_{k=0}^{+\infty}\frac{f^{(k)}(0)}{k!}x^{k}$.
        
        \item On a $\tan\geqslant0$ sur $\left[0,\frac{\pi}{2}\right[$ et $\tan'=1+\tan^{2}\geqslant0$. Soit $n\in\N$, on suppose que pour tout $k\in\left\llbracket0,n\right\rrbracket$, $\tan^{(k)}\geqslant0$ sur $\left[0,\frac{\pi}{2}\right[$. On dérive $n$ fois, d'après la formule de Leibniz, on a 
        \begin{equation}
            \tan^{(n+1)}=\sum_{k=0}^{n}\binom{n}{k}\tan^{(k)}\tan^{(n-k)}\geqslant0.
        \end{equation}
        Par imparité, on a pour tout $t\in\left[0,\frac{\pi}{2}\right[$
        \begin{equation}
            \tan(x)=\sum_{k=0}^{+\infty}\frac{\tan^{(k)}(0)}{k!}x^{k}=\sum_{q=0}^{+\infty}\frac{\tan^{(2p+1)}(0)}{(2p+1)!}x^{2p+1}.
        \end{equation}
        Par imparité, c'est aussi vrai sur $\left]-\frac{\pi}{2},\frac{\pi}{2}\right[$.
    \end{enumerate}
\end{proof}

\begin{remark}
    Si $\tan(x)=\sum_{k=0}^{+\infty}a_k x^{k}$, $\tan'=1+\tan^{2}$ fournit, pour tout $n\geqslant1$, $(n+1)a_{n+1}=\sum_{k=0}^{n}a_{k}a_{n-k}$.
\end{remark}

\begin{proof}
    \phantom{}
    \begin{enumerate}
        \item D'après le critère spécial des séries alternées, $a_n$ est du signe de $(-1)^{n}$, et $a_n\xrightarrow[n\to+\infty]{}0$. De plus, il existe $M>0$ tel que pour tout $n\geqslant1$, $\left\lvert a_n\right\rvert\leqslant M$. Donc par comparaison, $R\geqslant1$.
        
        D'autre part, on a $\left\lvert a_n\right\rvert+\left\lvert a_{n+1}\right\rvert=\frac{1}{n}$ et $\left\lvert a_n\right\rvert=\sum_{k=0}^{+\infty}\frac{1}{(n+2k)(n+1+2k)}$. On a $\left\lvert a_{n+1}\right\rvert\leqslant\left\lvert a_n\right\rvert$. On a alors $2\left\lvert a_{n+1}\right\rvert\leqslant\frac{1}{n}\leqslant2\left\lvert a_n\right\rvert$, et $\frac{1}{2n}\leqslant\left\lvert a_n\right\rvert\leqslant\frac{1}{2(n-1)}$. D'où $\left\lvert a_n\right\rvert\underset{+\infty}{\sim}\frac{1}{2n}$ et $R=1$. $a_n(-1)^{n}=\left\lvert a_n\right\rvert$ est le terme général d'une série divergente.

        Pour tout $\theta\in\R$ tel que $\e^{\i\theta}\neq1$, on a $a_n\left(-\e^{\i\theta}\right)^{n}=\left\lvert a_n\right\rvert\e^{\i n\theta}$. $n\mapsto\left\lvert a_n\right\rvert$ est décroissante tandis que $n\mapsto\e^{\i n\theta}$ est bornée. D'après la règle d'Abel, $\sum a_{n}\left(-\e^{\i\theta}\right)^{n}$ converge. On a convergence sur le cercle sauf en -1.

        \item On a toujours $\left\lvert a_n\right\rvert\leqslant3^{\frac{n-1}{3}}$. Si $b_n=3^{\frac{n-1}{3}}$, $\sum b_nz^{n}$ a un rayon de convergence égal à $\frac{1}{\sqrt[3]{3}}$. Donc $R\geqslant\frac{1}{\sqrt[3]{3}}$. De plus, $a_{3p+1}\left(\frac{1}{\sqrt[3]{3}}\right)^{3p+1}=3^{p}\times\frac{1}{3^{p}}\times\frac{1}{\sqrt[3]{3}}=3^{-\frac{1}{3}}$. Donc $\sum a_n\left(\frac{1}{\sqrt[3]{3}}\right)^{n}\not\to0$ quand $\nrightarrow+\infty$. Donc $R=3^{-\frac{1}{3}}$.
        
        Sur le cercle, si $z=3^{-\frac{1}{3}}\e^{\i\theta}$ avec $\theta\in\R$, on a $\left\lvert a_{3p+1}z^{3p+1}\right\rvert=3^{-\frac{1}{3}}$ donc $a_{n}z^{n}\nrightarrow0$ : il y a divergence.

        Pour le calcul effectif, si $z\in\C$ tel que $\left\lvert z\right\rvert<3^{-\frac{1}{3}}$, on a 
        \begin{equation}
            \sum_{n=0}^{+\infty}a_{n}z^{n}=\sum_{p=0}^{+\infty}\frac{(-1)^{p}}{2^{p}}z^{3p}+\sum_{p=0}^{+\infty}3^{p}z^{3p+1}=\frac{1}{1+\frac{z^{3}}{2}}+\frac{z}{1-3z^{3}}.
        \end{equation}

        \item Soit $n\geqslant0$, on a $\frac{1}{3}\int_{0}^{1}t^{n}\d t\leqslant a_n\leqslant\int_{0}^{1}t^{n}\d t$. Ainsi, d'après la règle de d'Alembert, on a $R=1$.
        \item 
        Comme $a_n\geqslant\frac{1}{3}\frac{1}{n+1}$, en $x=R$, $\sum a_nx^{n}=\sum a_n$ est divergente. 
        
        $\sum a_n(-R)^{n}$ est alternée, et comme $t^{n+1}\leqslant t^{n}$ pour tout $t\in[0,1]$, $n\mapsto\left\lvert a_n\right\rvert$ décroît vers 0. Donc $\sum a_n(-R)^{n}$ est convergente d'après le critère spécial des séries alternées. 

        Pour le calcul, soit $x\in]-1,1[$. Soit \function{f_n}{[0,1]}{\R}{t}{\frac{(tx)^{n}}{1+t+t^{2}}}
        $f_n$ est continue sur $[0,1]$ et $\left\lvert f_n(t)\right\rvert\leqslant\left\lvert x\right\rvert^{n}$ terme général d'une série à termes positifs convergente. Donc $\sum f_n$ converge normalement donc uniformément sur $[0,1]$, on peut intervertir :
        \begin{equation}
            \sum_{n=0}^{+\infty}\int_{0}^{1}\frac{t^{n}}{1+t+t^{2}}\d t x^{n}=\int_{0}^{1}\frac{1}{1+t+t^{2}}\times \frac{1}{1-tx}\d t.
        \end{equation}
        On pose $F(X)=\frac{1}{1+X+X^{2}}\times\frac{1}{1-Xx}=\frac{\alpha X+\beta}{1+X+X^{2}}+\frac{\gamma}{1-Xx}$. Si $x\neq0$, on a $\gamma=\frac{x^{2}}{1+x+x^{2}}$ et $\lim\limits_{X\to+\infty}XF(X)=0=\alpha-\frac{\gamma}{x}$ et $\alpha=\frac{x}{1+x+x^{2}}$. Enfin, $F(0)=1=\beta+\gamma$ donc $\beta=\frac{1}{1+x+x^{2}}$. Finalement, on a 
        \begin{equation}
            S(x)=\frac{1}{1+x+x^{2}}\times\left[\int_{0}^{1}\frac{xt+1+x}{1+t+t^{2}}\d t+x^{2}\int_{0}^{1}\frac{\d t}{1-tx}\right].
        \end{equation}
        Le calcul est laissé aux soins du lecteur.

        Pour la valeur en -1, on note que pour tout $x\in[0,1[$, $\sum_{n=0}^{+\infty}(-1)^{n}a_nx^{n}=S(-x)$. D'après le critère spécial des séries alternées, le $n$-ième reste est majoré par $a_n\to0$ donc on a convergence uniforme et $\lim\limits_{x\to1}S(-x)=\sum_{n=0}^{+\infty}(-1)^{n}a_n$ (continuité en -1).
    \end{enumerate}
\end{proof}

\begin{proof}
    \phantom{}
    \begin{enumerate}
        \item On partitionne les relations d'équivalence sur $\left\llbracket1,n+1\right\rrbracket$ selon le cardinal de la classe de $n+1$, $k$. On a alors $\omega_{n+1}=\sum_{k=0}^{n}\binom{n}{k}\omega_{n-k}=\sum_{k=0}^{n}\binom{n}{k}\omega_k$ (choisir les $k$ éléments en relation avec $n+1$). On a $\omega_{0}=1=0^{0}$. Soit $n\in\N$, supposons que pour tout $k\leqslant n$, on ait $\omega_{k}\leqslant k^{k}$. Alors 
        \begin{equation}
            \omega_{n+1}\leqslant\sum_{k=0}^{n}\binom{n}{k}k^{k}\leqslant\sum_{k=0}^{n}\binom{n}{k}n^{k}=(1+n)^{n}\leqslant(n+1)^{n+1}.
        \end{equation}
        Donc pour tout $n\in\N$, on a $\omega_{n}\leqslant n^{n}$.

        On a $\frac{\omega_{n}}{n!}\leqslant\frac{n^{n}}{n!}\underset{+\infty}{\sim}\frac{\e^{n}}{\sqrt{2\pi n}}$ donc $R\geqslant\frac{1}{\e}>0$.

        \item Pour tout $n\leqslant n_0$, $\frac{\omega_n r^{n}}{n!}\leqslant A$. Soit $n\geqslant n_0$, supposons que pour tout $k\leqslant n$, $\omega_k x^{k}\leqslant Ak!$. Alors on a 
        \begin{align}
            \omega_{n+1}r^{n+1}
            &=\sum_{k=0}^{n}\binom{n}{k}\omega_{k}r^{k}r^{n+1-k},\\
            &\leqslant n!Ar\sum_{k=0}^{n}\frac{r^{n-k}}{(n-k)!},\\
            &\leqslant n! A r\e^{r}\leqslant (n+1)! A.
        \end{align}
        Donc pour tout $n\in\N$, $\frac{\omega_n}{n!}\leqslant \frac{A}{r^{n}}$. On a donc $R\geqslant r$ pour tout $r\geqslant 1$ donc $R=+\infty$.

        \item Soit $x\in\R$, on a 
        \begin{align}
            f'(x)
            &=\sum_{n=0}^{+\infty}(n+1)\frac{\omega_{n+1}}{(n+1)!}z^{n},\\
            &=\sum_{n=0}^{+\infty}\frac{\omega_{n+1}}{n!}z^{n},\\
            &=\sum_{n=0}^{+\infty}\sum_{k=0}^{n}\frac{\omega_{k}}{k!}\frac{1}{(n-k)!}x^{n},\\
            &=\e^{x}f(x).
        \end{align}

        Donc il existe $K\in\R$ tel que pour tout $x\in\R$, $f(x)=K\e^{\e^{x}}$, et $K=\frac{f(0)}{\e}=\frac{1}{\e}$. On a donc 
        \begin{equation}
            f(x)=\frac{1}{\e}\e^{\e^{x}}=\frac{1}{\e}\sum_{k=0}^{+\infty}\frac{\e^{kx}}{k!}=\frac{1}{\e}\sum_{k=0}^{+\infty}\sum_{n=0}^{+\infty}\underbrace{\frac{1}{k!}\frac{(kx)^{n}}{n!}}_{a_{k,n}}.
        \end{equation}
        On a $\sum_{k=0}^{+\infty}\sum_{n=0}^{+\infty}\left\lvert a_{k,n}\right\rvert=\e^{\e^{\left\lvert x\right\rvert}}<+\infty$. D'après le théorème de Fubini, on a donc 
        \begin{equation}
            f(x)=\frac{1}{\e}\sum_{n=0}^{+\infty}\sum_{k=0}^{+\infty}\frac{(kx)^{n}}{k!n!},
        \end{equation}
        et donc pour tout $n\in\N$, on a 
        \begin{equation}
            \boxed{
                \omega_{n}=\frac{1}{\e}\sum_{k=0}^{+\infty}\frac{k^{n}}{k!}.
            }
        \end{equation}
    \end{enumerate}
\end{proof}

\begin{proof}
    \phantom{}
    \begin{enumerate}
        \item Pour tout $n\geqslant1$, on a $p_n\leqslant\sum_{j=1}^{n}p_{n-j}$ (car $p_{n-j}$ est le nombre maximal de possibilité si le premier terme vaut $j$, $t_{1}=j$). On a $p_0=2^{0}$ et par récurrence forte, soit $n\in\N^{*}$, supposons que pour tout $k\in\left\llbracket0,n-1\right\rrbracket$, $p_{k}\leqslant2^{k}$, alors 
        \begin{equation}
            p_n\leqslant\sum_{j=0}^{n-1}p_j\leqslant\sum_{j=0}^{n-1}2^{j}=2^{n}-1.
        \end{equation}
        Donc pour tout $n\in\N$, $p_n\leqslant 2^{n}$. D'après la règle de d'Alembert, $R\geqslant\frac{1}{2}$. De plus, pour tout $n\in\N$, $p_n\geqslant1$ donc $R\leqslant 1$.

        \item Soit $x\in[0,R[$, on a $x<1$. Alors $0\leqslant-\ln\left(1-x^{k}\right)\underset{k\to+\infty}{\sim}x^{k}$, terme générale d'une série à termes positifs convergente car $x<1$. Donc $\prod_{k\geqslant1}\frac{1}{1-x^{k}}$ converge.
        
        Soit $N\geqslant1$, on a 
        \begin{align}
            \prod_{k=1}^{N}\frac{1}{1-x^{k}}
            &=\prod_{k=1}^{N}\left(\sum_{n_k=0}^{+\infty}(x^{k})^{n_k}\right),\\
            &=\sum_{(n_{1},\dots,n_{N})\in\N^{N}}x^{n_{1}+2n_{2}+\dots+Nn_{N}},\\
            &=\sum_{n=0}^{+\infty}\alpha_{n,N}x^{n},
        \end{align}
        où $\alpha_{n,N}=\left\lvert\left\lbrace(n_1,\dots,n_N)\in\N^{N}\middle| n_1+2n_2+\dots+Nn_{N}=n\right\rbrace\right\rvert$. Par sommabilité, on a $\alpha_{n,N}=\left\lvert\left\lbrace\text{partitions }(t_k)_{k\geqslant1}\text{ de }n\middle| t_1\leqslant N\right\rbrace\right\rvert\leqslant p_n$, et si $n\leqslant N, \alpha_{n,N}=p_n$. 
        
        On a $f(x)=\sum_{n=0}^{N}p_nx^{n}+\sum_{n=N+1}^{+\infty}\alpha_{n,N}x^{n}$ d'où $\prod_{k=1}^{N}\frac{1}{1-x^{k}}\leqslant f(x)$. Ainsi,
        \begin{equation}
            0\leqslant f(x)-\prod_{k=1}^{N}\frac{1}{1-x^{k}}\leqslant\sum_{n=N+1}^{+\infty}\left(p_n-\alpha_{n,N}\right)x^{n}\leqslant\sum_{n=N+1}^{+\infty}p_n x^{n}\xrightarrow[N\to+\infty]{}0,
        \end{equation}
        reste d'une série convergente. Donc 
        \begin{equation}
            f(x)=\prod_{k=1}^{+\infty}\frac{1}{1-x^{k}}.
        \end{equation}

        Soit $z\in D(0,R)$, 
        \begin{equation}
            \left\lvert f(z)-\prod_{k=1}^{N}\frac{1}{1-z^{k}}\right\rvert\leqslant\sum_{n=N+1}^{+\infty}\left(p_n-\alpha_{n,N}\right)\left\lvert z\right\rvert^{n}\leqslant\sum_{n=N+1}^{+\infty}p_n\left\lvert z\right\rvert^{n}\xrightarrow[N\to+\infty]{}0.
        \end{equation}

        Cela reste vrai sur $D(0,R)$.

        \item Si $x\in[0,1[$, on peut développer et on obtient
        \begin{equation}
            \prod_{k=1}^{+\infty}\frac{1}{1-x^{k}}=\sum_{n=0}^{+\infty}a_nx^{n},
        \end{equation}
        et par unicité, $a_n=p_n$ pour tout $n\in\N$ donc $R=1$.
    \end{enumerate}
\end{proof}

\begin{proof}
    \phantom{}
    \begin{enumerate}
        \item On a $f(z_{0}+r\e^{\i t})=\sum_{n=0}^{+\infty}a_n r^{n}\e^{\i nt}=\sum_{n=0}^{+\infty}f_n(t)$. On a pour tout $n\in\N$, $\left\lvert f_n(t)\right\rvert=\left\lvert a_n\right\rvert r^{n}$, et comme $r<d(z_{0},\partial U)$ donc $\sum \left\lvert a_n\right\rvert r^{n}$ converge. On a convergence normale des $(f_n)_{n\in\N}$ sur $[0,2\pi]$.
        Ainsi,
        \begin{equation}
            \int_{0}^{2\pi}f\left(z_{0}+r\e^{\i t}\right)\d t=\sum_{n=0}^{+\infty}a_n r^{n}\int_{0}^{2\pi}\e^{\i nt}\d t=2\pi a_0=2\pi f(z_{0}).
        \end{equation}
        Donc 
        \begin{equation}
            \boxed{
                f(z_{0})=\frac{1}{2\pi}\int_{0}^{2\pi}f\left(z_0+r\e^{\i t}\right)\d t.
            }
        \end{equation}

        \item $\overline{U}$ est un compact donc $\left\lvert f\right\rvert$ atteint son maximum sur $\overline{U}$. De plus, pour tout $r\in[0,d(z_{0},\partial U)]$ et pour tout $t\in[0,2\pi]$, on a $\left\lvert f\left(z_{0}+r\e^{\i t}\right)\right\rvert\leqslant\left\lVert f\right\rVert_{\infty}$, intégrable sur $[0,2\pi]$ et $f$ continue. Donc d'après le théorème de continuité, $f(z_{0})=\frac{1}{2\pi}\int_{0}^{2\pi}f\left(z_0+R\e^{\i t}\d t\right)$ où $R=d(z_{0},\partial U)$.
        
        Si $\left\lvert f\right\rvert$ atteint son maximum en $z_{0}\in U$, on a 
        \begin{equation}
            \left\lvert f(z_{0})\right\rvert=\left\lVert f\right\rVert_{\infty}=\left\lvert \frac{1}{2\pi}\int_{0}^{2\pi}f\left(z_{0}+R\e^{\i t}\right)\d t\right\rvert\leqslant\frac{1}{2\pi}\int_{0}^{2\pi}\left\lvert f\left(z_{0}+R\e^{\i t}\right)\right\rvert\d t\leqslant\left\lVert f\right\rVert_{\infty}.
        \end{equation}

        On a donc égalité partout : $\frac{1}{2\pi}\int_{0}^{2\pi}\left(\left\lVert f\right\rVert_{\infty}-\left\lvert f\left(z_0+R\e^{\i t}\right)\right\rvert\right)\d t=0$. Comme l'intégrande est une fonction continue positive, donc pour tout $t\in[0,2\pi]$, $\left\lVert f\right\rVert_{\infty}=f\left(z_{0}+R\e^{\i t}\right)$. On a $d\left(C(0,R),\partial U\right)=0$ et comme $C(0,R)$ est un compact la distance est atteinte : il existe $t_{0}\in[0,2\pi]$ tel que $z_{0}+R\e^{\i t_0}\in\partial U$, donc $\left\lvert f\right\rvert$ atteint son maximum et son minimum sur $\partial U$.

        \item Si $f=0$ sur $\partial U$, alors $f=0$ sur $\overline{U}$.
    \end{enumerate}
\end{proof}

\begin{remark}
    S'il existe $z_{0}\in U$ tel que $\left\lvert f(z_{0})\right\rvert=\left\lVert f\right\rVert_{\infty}$, on a pour tout $r\leqslant R$,
    \begin{equation}
        \left\lvert f(z_{0})\right\rvert=\left\lVert f\right\rVert_{\infty}=\left\lvert \frac{1}{2\pi}\int_{0}^{2\pi}f\left(z_{0}+r\e^{\i t}\right)\d t\right\rvert\leqslant\frac{1}{2\pi}\int_{0}^{2\pi}\left\lvert f\left(z_{0}+r\e^{\i t}\right)\right\rvert\d t\leqslant\left\lVert f\right\rVert_{\infty}.
    \end{equation}

    On en déduit que pour tout $t\in[0,2\pi]$, $\left\lvert f\left(z_0+r\e^{\i t}\right)\right\rvert=\left\lvert f(z_{0})\right\rvert$, et on a aussi $\arg\left(f\left(z_{0}+r\e^{\i t}\right)\right)\equiv\arg\left(f(z_0)\right)[2\pi]$. Donc $f\left(z_{0}+r\e^{\i t}\right)=f(z_{0})$, et on peut vérifier que $f$ est constante.
\end{remark}

\begin{proof}
    \phantom{}
    \begin{enumerate}
        \item Passer au $\ln$ de la valeur absolue, équivalent, convergence géométrique.
        \item On cherche une équation fonctionnelle satisfaite par $f$.
        On a $f(qz)=\prod_{k=1}^{+\infty}\left(1-q^{k+1}z\right)=\prod_{k=2}^{+\infty}\left(1-q^{k}z\right)$ donc $(1-qz)f(qz)=f(z)$.
        Si $f$ est développable en série entière avec $f(z)=\sum_{n=0}^{+\infty}a_nz^{n}$ pour tout $z\in D(0,R)$ avec $R>0$, on a par unicité du développement, $a_n(q^{n}-1)=a_{n-1}q^{n}$ et comme $\left\lvert q\right\rvert<1$, pour tout $n\in\N^{*}$, $q^{n}\neq1$ d'où $a_{n}=a_{n-1}\frac{q^{n}}{q^{n}-1}=\prod_{i=1}^{n}\frac{q^{i}}{q^{i}-1}$.

        Réciproquement si pour tout $n\in\N$, $a_n=\prod_{i=1}^{n}\frac{q^{i}}{q^{i}-1}$, alors pour tout $n\in\N$, $a_n\neq0$ et par la règle de d'Alembert, $R=+\infty$. Si $S\colon\C\to\C$ est définie apr $\S(z)=\sum_{n=0}^{+\infty}a_n z^{n}$, en reportant les calculs, $S$ vérifie la même équation fonctionnelle que $f$. En itérant, on a $S(z)=\prod_{i=1}^{n}(1-q^{i}z)S(q^{n}z)$. $S$ étant continue en 0 (car développable en série entière sur $\C$), on a $S(q^{n}z)\xrightarrow[n\to+\infty]{}S(0)=a_0=1$. En passant à la limite, on a donc $S(z)=f(z)$ et $f$ est développable en série entière.

        \item Si $f(z)\neq0$ et $f(qz)\neq0$, on pose $g(z)=\frac{1}{f(z)}$. On a alors $g(qz)=(1-qz)g(z)$, et on procède de même façon qu'à la question précédente. On trouve alors $R=\frac{1}{\left\lvert q\right\rvert}$.
    \end{enumerate}
\end{proof}

\begin{proof}
    \phantom{}
    \begin{enumerate}
        \item Par continuité, $a_0=f(z_{0})=0$. Supposons qu'il existe $n\geqslant1$ tel que $a_n\neq0$. Soit $n_{0}=\min\left\lbrace n\in\N\middle| a_n\neq0\right\rbrace$. Il vient, si $\left\lvert h\right\rvert<r_0$, 
        \begin{equation}
            f(z_{0}+h)=\sum_{k\geqslant n_{0}}^{+\infty}a_k h^{k}=a_{n_{0}}h^{n_{0}}\left(1+\underbrace{\sum_{k=1}^{+\infty}\frac{a_{k+n_0}}{a_{n_{0}}}h^{k}}_{g(h)}\right).
        \end{equation}

        $g$ est continue (série entière de rayon de convergence plus grand que $r_0>0$) et $g(0)=0$ donc il existe $\alpha_{0}>0$ tel que si $\left\lvert h\right\rvert\leqslant\alpha_{0}$, alors $\left\lvert g(h)\right\rvert\leqslant\frac{1}{2}$. Alors $1+g(h)\neq0$ et si $h\neq0$, $f(z_{0}+h)=a_{n_{0}}h^{n_{0}}\left(1+g(h)\right)\neq0$. Il existe $N\in\N$ tel que pour tout $k\geqslant N$, $\left\lvert \xi_{k}-z_{0}\right\rvert\leqslant\alpha_{0}$, d'où pour tout $k\geqslant N$, $f(\xi_{k})\neq0$, ce qui est absurde. Les $(a_n)_{n\in\N}$ sont donc tous non nuls.

        \item Soit $z_1\in U$. Il existe $\gamma\colon[0,1]\to U$ continue telle que $\gamma(0)=z_{0}$ et $\gamma(1)=z_{1}$. Soit $t_{0}=\sup\left\lbrace t\in[0,1]\middle|\forall x\in[0,t],f(\gamma(x))=0\right\rbrace$. Supposons $t_{0}\neq1$. Par définition de la borne supérieure, pour tout $x\in[0,t_0[$, $f(\gamma(x))=0$. On peut appliquer ce qui précède à $\gamma(t_0)$ à la place de $z_{0}$ : il existe $\alpha_{0}$ tel que pour tout $z\in D(\gamma(t_{0}),\alpha_{0})$ tel que $f(z)=0$. Par continuité de $\gamma$, il existe $\beta>0$ tel que si $\left\lvert t-t_{0}\right\rvert<\beta$, alors $\left\lvert\gamma(t)-\gamma(t_{0})\right\rvert\leqslant\alpha_{1}$/ Pour $t=t_{0}+\frac{\beta}{2}$, on a $f(\gamma(t))=0$ pour tout $x\in[0,t]$. C'est absurde. Donc $t_{0}=1$ et $f(z_{1})=0$.
    \end{enumerate}
\end{proof}

\begin{remark}
    Deux fonctions analytiques définies sur un ouvert connexe par arcs et qui coïncident sur une suite injective convergente sont égales.
\end{remark}

\begin{proof}
    \phantom{}
    \begin{enumerate}
        \item On a $f(x)=\ln\left(\left(x-\cos(\theta)\right)^{2}+\sin^{2}(\theta)\right)$. L'argument est positif et égal à 0 si et seulement si $x=\cos(\theta)$ et $\sin(\theta)=0$, ce qui est absurde car $\theta\in]0,\pi[$. $f$ est définie sur $\R$ et est $\mathcal{C}^{\infty}$.
        On a
        \begin{equation}
            f'(x)=\frac{2(x-\cos(\theta))}{1-2x\cos(\theta)+x^{2}}=\frac{2(x-\cos(\theta))}{\left(x-\e^{\i\theta}\right)\left(x+\e^{\i\theta}\right)}=\frac{a}{x-\e^{\i\theta}}+\frac{b}{x-\e^{-\i\theta}}.
        \end{equation},
        où $a=\frac{2\left(\e^{\i\theta}-\cos(\theta)\right)}{\e^{\i\theta}-\e^{-\i\theta}}=\frac{2\i\sin(\theta)}{2\i\sin(\theta)}=1$ et $b=\overline{a}=1$.

        On sait alors que $f'(x)=\frac{1}{x-\e^{\i\theta}}+\frac{1}{x-\e^{-\i\theta}}$ est développable en série entière avec un rayon de convergence égal à 1 (fonction rationnelle dont 0 n'est pas un pôle). Soit $x\in]-1,1[$, on a 
        \begin{align}
            f'(x)
            &=-\e^{-\i\theta}\times\frac{1}{1-x\e^{-\i\theta}}-\e^{\i\theta}\times\frac{1}{1-x\e^{\i\theta}},\\
            &=-\e^{-\i\theta}\sum_{k=0}^{+\infty}\left(x\e^{-\i\theta}\right)^{k}-\e^{\i\theta}\sum_{k=0}^{+\infty}\left(x\e^{\i\theta}\right)^{k},\\
            &=\sum_{k=0}^{+\infty}\left(-\left(\e^{-\i\theta}\right)^{k+1}-\left(\e^{\i\theta}\right)^{k+1}\right)x^{k},\\
            &=\sum_{k=0}^{+\infty}\left(-2\cos\left((k+1)\theta\right)\right)x^{k}.
        \end{align}
        On a $f(0)=0$. Ainsi, pour tout $x\in]-1,1[$, $f(x)=\sum_{k=1}^{+\infty}\frac{x^{k}}{k}\times\left(-2\cos(k\theta)\right)$.

        \item D'après la règle d'Abel, on a convergence pour $x=1$ donc $\sum_{n\geqslant1}\frac{\cos(n\theta)}{n}$ converge, et on a convergence uniforme sur $[0,1]$. $f$ est alors continue en 1 et on a 
        \begin{equation}
            \sum_{k=1}^{+\infty}\frac{\cos(k\theta)}{k}=-\frac{1}{2}f(1)=-\frac{1}{2}\ln\left(2-2\cos(\theta)\right).
        \end{equation}

        \item Pour $x\in]-1,1[$ pour tout $\theta\in[0,\pi]$, $x^{2}-2x\cos(\theta)+1=\left(x-\cos(\theta)\right)^{2}+\sin^{2}(\theta)$ qui est égal à 0 si et seulement si $x=\cos(\theta)$ et $\theta\in\left\lbrace0,\pi\right\rbrace$ : impossible car $x\in]-1,1[$. On a 
        \begin{equation}
            I(x)=\int_{0}^{\pi}\left(-2\sum_{k=1}^{+\infty}\frac{x^{k}}{k}\cos(k\theta)\right)\d\theta=\int_{0}^{\pi}f_k(\theta)\d\theta.
        \end{equation}

        On pose $u_k=\int_{0}^{\pi}\left\lvert f_k(\theta)\right\rvert\d\theta\leqslant\frac{x^{k}}{k}\times\pi$, terme général d'une série convergente car $\left\lvert x\right\rvert<1$, donc $\sum u_{k}$ converge. On peut donc intervertir :
        \begin{equation}
            I(x)=-2\sum_{k=1}^{+\infty}\frac{x^{k}}{k}\int_{0}^{\pi}\cos(k\theta)\d\theta=0.
        \end{equation}
        Pour $\left\lvert x\right\rvert>1$, on a 
        \begin{equation}
            I(x)=\int_{0}^{\pi}\left(\ln(x^{2})+\ln\left(1-\frac{2\cos(\theta)}{x}+\frac{1}{x^{2}}\right)\right)\d\theta=2\pi\ln\left(\left\lvert x\right\rvert\right).
        \end{equation}
    \end{enumerate}
\end{proof}

\begin{remark}
    Pour $x=1$, on a 
    \begin{align}
        \ln(2-2\cos(\theta))
        &=\ln(2)+\ln(1-\cos(\theta)),\\
        &\underset{\theta\to0}{=}\ln(2)+\ln\left(\frac{\theta^{2}}{2}+o(\theta^{2})\right),\\
        &\underset{\theta\to0}{=}2\ln(\theta)+O(1)+\ln(2),\\
        &\underset{\theta\to0}{\sim}2\ln(\theta),\\
        &\underset{\theta\to0}{O}\left(\frac{1}{\sqrt{\theta}}\right).
    \end{align}

    $I(1)$ est donc bien définie et $I(1)=\pi\ln(2)+\int_{0}^{\pi}\ln\left(2\sin^{2}\left(\frac{\theta}{2}\right)\right)$. On se ramène à $\int_{0}^{\frac{\pi}{2}}\ln(\sin(\theta))\d\theta$.
\end{remark}

\begin{proof}
    \phantom{}
    \begin{enumerate}
        \item On pose $a_k=0$ si $k\not\in\left\lbrace p_n,n\in\N\right\rbrace$ et $a_k=1$ sinon. On a toujours $\left\lvert a_k\right\rvert\leqslant 1$, donc $R\geqslant1$. De plus, $\sum_{n\geqslant}1^{p_n}=+\infty$, donc $R=1$.
        \item Soit $\varepsilon>0$, il existe $N_{0}\in\N$ tel que pour tout $n\geqslant N_0$, $\frac{n}{p_n}\leqslant\frac{\varepsilon}{2}$ et donc $p_n\geqslant\frac{2n}{\varepsilon}$. Soit $x\in[0,1[$. Pour tout $n\geqslant N_0$, on a $x^{p_n}\leqslant x^{\frac{2n}{\varepsilon}}$. Ainsi,
        \begin{equation}
            f(x)=\sum_{n=0}^{N_0-1}x^{p_n}+\sum_{n=N_0}^{+\infty}x^{p_n}\leqslant\sum_{n=0}^{N_0-1}x^{p_n}+\sum_{n=N_0}^{+\infty}x^{\frac{2n}{\varepsilon}}=\sum_{n=0}^{N_0-1}x^{p_n}+\frac{1}{1-x^{\frac{2}{\varepsilon}}}.
        \end{equation}
        On a $\lim\limits_{x\to1^{-}}(1-x)\sum_{n=0}^{N_0-1}x^{p_n}=0$, donc il existe $\alpha_{1}>0$ tel que pour tout $x\in[1-\alpha_{1}[$, $(1-x)\sum_{n=0}^{N_0-1}x^{p_n}\leqslant\frac{\varepsilon}{3}$, et 
        \begin{equation}
            \frac{1-x}{1-x^{\frac{2}{\varepsilon}}}=\frac{u}{1-\left(1-\frac{2u}{\varepsilon}+\underset{u\to0}{o}(u)\right)}\xrightarrow[u\to0]{}\frac{\varepsilon}{2},
        \end{equation}
        en posant $u=1-x$. Ainsi, il existe $\alpha_{2}>0$ tel que pour tout $x\in[1-\alpha_{2},1[$, $\frac{1-x}{1+x^{\frac{2}{\varepsilon}}}\leqslant\frac{2\varepsilon}{3}$. Ainsi, en posant $\alpha=\min\left(\alpha_{1},\alpha_{2}\right)$, si $x\in[1-\alpha,1[$, alors $f(x)(1-x)\leqslant\varepsilon$.

        \item On suppose $\lim\limits_{x\to1^{-}}(1-x)f(x)=0$. On pose, pour tout $k\geqslant1,x_{k}=1-\frac{1}{k}$. Alors on a 
        \begin{align}
            (1-x_k)f(x_k)
            &=\frac{1}{k}\left(\sum_{n=0}^{N_0-1}\left(1-\frac{1}{k}\right)^{p_n}\right)+\sum_{n=k}^{+\infty}\left(1-\frac{1}{k}\right)^{p_n},\\
            &\geqslant \sum_{n=k}^{+\infty}\frac{1}{k}\left(1-\frac{1}{k}\right)^{p_n},\\
            &\geqslant\sum_{n=k}^{+\infty}\frac{1}{n}\left(1-\frac{1}{k}\right)^{p_n}.
        \end{align}
        Donc $\lim\limits_{k\to+\infty}\sum_{n=k}^{+\infty}\frac{1}{n}\left(1-\frac{1}{k}\right)^{p_n}=0$.

        %% A voir
    \end{enumerate}
\end{proof}

\begin{proof}
    On suppose que $R_{1}$ et $R_{2}$, les rayons de convergence de $U(z)$ et $V(z)$ sont strictement positifs. Soit $R=\min(R_1,R_2)>0$. Pour tout $n\in\N$, pour tout $z\in D(0,R)$, on a $u_{n+1}z^{n+1}=\left(u_nz^{n}-v_nz^{n}\right)z$ et $v_{n+1}z^{n+1}=\left(u_nz^{n}-2v_nz^{n}\right)z$. On somme sur $\N$ et on obtient
    \begin{equation}
        \begin{array}[]{rcl}
            U(z)-u_0 &=& z\left(U(z)-V(z)\right),\\
            V(z)-v_0 &=& z\left(U(z)-2V(z)\right).
        \end{array}
    \end{equation}

    Ainsi,
    \begin{equation}
        \begin{array}[]{rcl}
            U(z)(z-1)-zV(z) &=& -U_0,\\
            zU(z)-V(z)(2z+1) &=& -V_0.
        \end{array}
    \end{equation}

    Le déterminant du système est 
    \begin{equation}
        \begin{vmatrix}
            z-1 & -z\\
            z -(2z+1)
        \end{vmatrix}=-z^{2}+z+1.
    \end{equation}
    Le discriminant est $\Delta=5$. Soit $z_{1}=\frac{1-\sqrt{5}}{2}$ et $z_2=\frac{1+\sqrt{5}}{2}$. On a $\left\lvert z_1\right\rvert=\frac{\sqrt{5}-1}{2}<\left\lvert z_2\right\rvert$.

    Si $z\not\in\left\lbrace z_1,z_2\right\rbrace$, on peut utiliser les formules de Cramer. Ainsi,
    \begin{equation}
        \begin{array}[]{l}
            U(z) = \frac{
                \begin{vmatrix}
                    -u_0 & -z\\
                    -v_0 & -(2z+1)
                \end{vmatrix}
            }{-z^{2}+z+1}=\frac{(2z+1)u_0-v_0z}{-z^{2}+z+1},\\
            V(z)= \frac{
                \begin{vmatrix}
                    z-1 & -u_0\\
                    z & -v_0
                \end{vmatrix}
            }{-z^{2}+z+1}=\frac{-v_0(z-1)+u_0z}{-z^{2}+z+1}.
        \end{array}
    \end{equation}

    Réciproquement, en définissant ainsi $U$ et $V$, ce sont des fractions rationnelles de $z_1$ et $z_2$ donc développable en séries entières avec un rayon de convergence à $\left\lvert z_1\right\rvert=\frac{\sqrt{5}-1}{2}$. En remontant les calculs, les coefficients vérifient les relations de récurrence.
\end{proof}

\begin{proof}
    \phantom{}
    \begin{enumerate}
        \item Si $\theta\in\left\lbrace0,\pi\right\rbrace$, $f(z)=0$. Sinon, on a $f(z)=\frac{\sin(\theta)}{\left(z-\e^{\i\theta}\right)\left(z-\e^{-\i\theta}\right)}\in\R(z)$. On prend $\left\lvert z\right\rvert<1$. Il existe $(A,B)\in\C^{2}$ tel que $f(z)=\frac{A}{z-\e^{\i\theta}}+\frac{B}{z-\e^{-\i\theta}}$. On trouve $A=-\frac{\i}{2}$ et $B=\overline{A}=\frac{\i}{2}$. En remplaçant, on a 
        \begin{align}
            f(z)
            &= \frac{\i}{2}\left(\frac{1}{z-\e^{-\i\theta}}-\frac{1}{z-\e^{\i\theta}}\right),\\
            &= \frac{\i}{2}\left(-\frac{\e^{\i\theta}}{1-z\e^{\i\theta}}+\frac{\e^{-\i\theta}}{1-z\e^{-\i\theta}}\right),\\
            &=\frac{\i}{2}\left(\sum_{n=0}^{+\infty}\left(-\e^{\i\theta}\left(z\e^{\i\theta}\right)^{n}+\e^{-\i\theta}\left(z\e^{-\i\theta}\right)^{n}\right)\right),\\
            &=\sum_{n=0}^{+\infty}\sin\left((n+1)\theta\right)z^{n}.
        \end{align}

        \item Pour $\left\lvert z\right\rvert<1$, défini car $z\not\in\left\lbrace\e^{\i\theta},\e^{-\i\theta}\right\rbrace$, on a 
        \begin{equation}
            I(z)=\int_{0}^{+\infty}\sum_{n=0}^{+\infty}\underbrace{\sin\left((n+1)\theta\right)z^{n}}_{f_n(\theta)}\d\theta.
        \end{equation}
        Comme $\left\lvert f_n(\theta)\right\rvert\leqslant\left\lvert z\right\rvert^{n}$, terme général d'une série à termes positifs convergentes car $\left\lvert z\right\rvert<1$, $\sum f_n$ convergent normalement sur $[0,\pi]$. On peut donc intervertir, et on a 
        \begin{equation}
            I(z)=\sum_{n=0}^{+\infty}\int_{0}^{\pi}\sin\left((n+1)\theta\right)\d\theta=2\sum_{k=0}^{+\infty}\frac{z^{2k}}{2k+1}.
        \end{equation}
        Pour $x\in]-1,1[$, soit $g(x)=xI(x)=2\sum_{k=0}^{+\infty}\frac{x^{2k+1}}{2k+1}$. On a $g'(x)=\frac{2}{1-x^{2}}=\frac{1}{1-x}+\frac{1}{1+x}$. On a $g(0)=0$, donc $g(x)=\ln\left(\frac{1+x}{1-x}\right)$ et $I(x)=\frac{1}{x}\ln\left(\frac{1+x}{1-x}\right)$.
    \end{enumerate}
\end{proof}

\begin{remark}
    Si $\left\lvert x\right\rvert>1$, on a $I(x)=\frac{1}{x^{2}}I\left(\frac{1}{x}\right)=\frac{1}{x}\ln\left(\frac{x+1}{x-1}\right)$.
\end{remark}

\begin{proof}
    \phantom{}
    \begin{enumerate}
        \item On a $\mathbb{E}(Y^{k})=\sum_{p=1}^{n}p^{k}\mathbb{P}\left(Y=p\right)$ pour tout $k\in\left\llbracket1,n\right\rrbracket$. Ainsi,
        \begin{equation}
            \begin{pmatrix}
                \mathbb{E}(Y)\\
                \vdots\\
                \mathbb{E}(Y^{n})\\
            \end{pmatrix}=
            \underbrace{
                \begin{pmatrix}
                    1&2&\dots&n\\
                    \vdots&2^{2}&\dots&n^{2}\\
                    \vdots & \vdots &\vdots & \vdots\\
                    1&2^{n}&\dots&n^{n}
                \end{pmatrix}
            }_{A}
            \begin{pmatrix}
                \mathbb{P}(Y=1)\\
                \dots\\
                \mathbb{P}(Y=n)
            \end{pmatrix}.
        \end{equation}
        On a $\det(A)=n! \text{VdM}(1,\dots,n)\neq0$. $A$ est inversible puis
        \begin{equation}
            \begin{pmatrix}
                \mathbb{P}(Y=1)\\
                \dots\\
                \mathbb{P}(Y=n)
            \end{pmatrix}=
            A^{-1}
            \begin{pmatrix}
                \mathbb{E}(Y)\\
                \vdots\\
                \mathbb{E}(Y^{n})\\
            \end{pmatrix}.
        \end{equation}

        Donc $\left(\mathbb{E}(Y^{k})\right)_{k\in\left\llbracket1,n\right\rrbracket}$ caractérise la loi de $Y$.

        \item Soit $n\geqslant1$. On a $k^{n}\mathbb{P}(Y=k)=\underset{K\to+\infty}{O}\left(\frac{1}{k^{2}}\right)$ (car $a<1$). Donc $Y$ possède un moment à tout ordre. Formons la série génératrice de $Y$: $G_Y(t)=\sum_{k=0}^{+\infty}\mathbb{P}\left(Y=k\right)t^{k}$ de rayon $R$ supérieur à $\frac{1}{a}$. $G_Y$ est $\mathcal{C}^{\infty}$ sur $\left[0,\frac{1}{a}\right[$. Ainsi,
        \begin{equation}
            \begin{array}[]{rcl}
                G_Y'(1) &=&\sum_{k=0}^{+\infty}y\mathbb{P}(Y=k)=\mathbb{E}(Y),\\
                G_Y''(1) &=& \sum_{k=0}^{+\infty}k(k-1)\mathbb{P}(Y=k)=\mathbb{E}(Y^{2})-\mathbb{E}(Y),\\
                \vdots\\
                G_Y^{(n)}(1) &=& \sum_{k=0}^{+\infty}k(k-1)\dots(k-n+1)\mathbb{P}(Y=k)=\mathbb{E}(Y^{n})+\sum_{k=0}^{+\infty}A(k)\mathbb{P}(Y=k),
            \end{array}
        \end{equation}
        avec $A\in\Z[X]$ de degré inférieur ou égal à $n-1$. Donc $\left(\mathbb{E}(Y^{n})\right)_{n\in\N^{*}}$ déterminent $\left(G_Y(1)\right)_{n\in\N^{*}}$.

        \begin{lemma}
            \label{lem:2}
            Soit $f(z)=\sum_{n=0}^{+\infty}a_nz^{n}$ de rayon de convergence supérieur ou égal à $R$. Soit $z_0\in D(0,R)$, alors il existe $(b_n)_{n\in\N}\in\C^{\N}$ tel que pour tout $h\in D(0,R-\left\lvert z_0\right\rvert)$, $f(z_{0}+h)=\sum_{n=0}^{+\infty}b_n h^{n}$.
        \end{lemma}
        \begin{proof}[Preuve du lemme~\ref{lem:2}]
            Soit $h\in D(0,R-\left\lvert z_0\right\rvert)$, on a $\left\lvert z_{0}+h\right\rvert\leqslant\left\lvert z_0\right\rvert+\left\lvert h\right\rvert<R$. On a donc 
            \begin{equation}
                f(z_{0}+h)=\sum_{n=0}^{+\infty}a_n (z_{0}+h)^{n}=\sum_{n=0}^{+\infty}a_n\sum_{k=0}^{+\infty}\binom{n}{k}z_{0}^{n-k}h^{k}=\sum_{n=0}^{+\infty}\sum_{k=0}^{+\infty}\alpha_{n,k}h^{k},
            \end{equation}
            avec $\alpha_{n,k}=\binom{n}{k}a_nz_{0}^{n-k}$ si $k\leqslant n$ et 0 sinon. Or 
            \begin{equation}
                \sum_{n=0}^{+\infty}\sum_{k=0}^{+\infty}\left\lvert \alpha_{n,k}\right\rvert\left\lvert h\right\rvert^{k}=\sum_{n=0}^{+\infty}\left\lvert a_n\right\rvert\sum_{k=0}^{n}\binom{n}{k}\left\lvert z_{0}\right\rvert^{n-k}\left\lvert h\right\rvert^{k}=\sum_{n=0}^{+\infty}\left\lvert a_n\right\rvert\left(\left\lvert h\right\rvert+\left\lvert z_{0}\right\rvert\right)^{n}<+\infty,
            \end{equation}
            
            car $\left\lvert h\right\rvert+\left\lvert z_{0}\right\rvert<R$. D'après le théorème de Fubini, on a $f(z_{0}+h)=\sum_{k=0}^{+\infty}\underbrace{\left(\sum_{n=0}^{+\infty}\alpha_{n,k}\right)}_{b_k}h^{k}$.
        \end{proof}

        On a pour tout $h\in\R$ tel que $\left\lvert h\right\rvert<\frac{1}{a}-1$. On a $G_Y(1+h)=\sum_{n=0}^{+\infty}b_nh^{n}$ et $b_n=\frac{G_Y^{(n)}(1)}{n!}$. Or $\mathbb{P}(Y=k)=k!G_Y^{(k)}(0)$. On peut encore développer $G_Y$ au voisinage de $2-\frac{1}{a}$, et de proche en proche, au voisinage de $1-2^{k}\left(\frac{1}{a}-1\right)$, jusqu'à 0. On retrouve ainsi la loi de $Y$.
    \end{enumerate}
\end{proof}

\begin{proof}
    \phantom{}
    \begin{enumerate}
        \item Pour tout $t\in[0,2\pi]$, on a $\left\lvert r\e^{\i t}\right\rvert>\left\lvert z\right\rvert$, donc $r\e^{\i t}-z\neq0$ et $g$ est bien définie. Soit 
        \function{F}{[0,1]\times[0,2\pi]}{\C}{(\lambda,t)}{
            \frac{f\left((1-\lambda)z+\lambda r\e^{\i t}\right)-f(z)}{r\e^{\i t}-z}r\e^{\i t}
        }
        $F$ est continue sur $[0,1]\times[0,2\pi]$ et $[0,1]\times[0,2\pi]$ est compact donc $F$ est bornée. Ainsi, $g$ est continue (théorème de continuité des intégrales à paramètres). 

        On a $\frac{\partial F}{\partial \lambda}(\lambda,t)=f'\left((1-\lambda)z+\lambda r\e^{\i t}\right)r\e^{\i t}$. C'est une fonction continue sur $[0,1]\times[0,2\pi]$, donc bornée. D'après le théorème de Leibniz, $g$ est de classe $\mathcal{C}^{1}$. Ainsi, pour tout $\lambda\in]0,1]$, on a 
        \begin{equation}
            g'(\lambda)=\int_{0}^{2\pi}f'\left((1-\lambda)z+\lambda r\e^{\i t}\right)r\e^{\i t}\d t=\left[\frac{1}{\i\lambda}f\left((1-\lambda)z+\lambda r\e^{\i t}\right)\right]_{0}^{2\pi}=0,
        \end{equation}
        par continuité de $g'$, c'est aussi vrai en $\lambda=0$. Donc $g$ est constante sur $[0,1]$. De plus, $g(0)=0=g(1)=\int_{0}^{2\pi}\frac{f\left(r\e^{\i t}\right)-f(z)}{r\e^{\i t}-z}r\e^{\i t}\d t$. Donc 
        \begin{equation}
            f(z)\int_{0}^{2\pi}\frac{r\e^{\i t}}{r\e^{\i t}-z}\d t=\int_{0}^{2\pi}\frac{f\left(r\e^{\i t}\right)-f(z)}{r\e^{\i t}-z}r\e^{\i t}\d t.
        \end{equation}
        Pour tout $t\in[0,2\pi]$, on a $\frac{r\e^{\i t}}{r\e^{\i t}-z}=\frac{1}{1-\frac{z}{r}\e^{\i t}}$. Comme $\left\lvert\frac{z}{r}\right\rvert<1$, on a 
        \begin{equation}
            \frac{r\e^{\i t}}{r\e^{\i t}-z}=\sum_{n=0}^{+\infty}\left(\frac{z\e^{-\i t}}{r}\right)^{n}.
        \end{equation}
        De plus, $\left\lvert \frac{z\e^{-\i t}}{r}\right\rvert^{n}=\left\lvert\frac{z}{r}\right\rvert^{n}$, terme général d'une série à termes positifs convergente indépendant de $t$, et on a 
        \begin{equation}
            \left\lvert f\left(r\e^{\i t}\right)\frac{z\e^{-\i t}}{r}\right\rvert^{n}\leqslant\left\lVert f\right\rVert_{\infty,\mathcal{C}(0,r)}\left\lvert\frac{z}{r}\right\rvert^{n}.
        \end{equation}

        Ainsi, on a 
        \begin{equation}
            f(z)\times\sum_{n=0}^{+\infty}\int_{0}^{2\pi}\left(\frac{z\e^{-\i t}}{r}\right)^{n}\d t=\sum_{n=0}^{+\infty}\int_{0}^{2\pi}\left(f\left(r\e^{\i t}\right)\frac{z\e^{-\i t}}{r}\right)^{n}\d t=\sum_{n=0}^{+\infty}\left(\frac{z}{r}\right)^{k}\underbrace{\int_{0}^{2\pi}\e^{-\i nt}\d t}_{2\pi\delta_{n,0}}.
        \end{equation}

        Ainsi,
        \begin{equation}
            f(z)=\sum_{n=0}^{+\infty}\left(\frac{z}{r}\right)^{n}\int_{0}^{2\pi}f\left(r\e^{\i t}\right)\e^{-\i nt}\d t.
        \end{equation}

        Ceci valant pour $t\in]0,R[$ fixé, pour tout $z\in D(0,r)$, $f(z)=\sum_{n=0}^{+\infty}a_n z^{n}$. Donc pour tout $n\in\N$, $a_n=\frac{f^{(n)}(0)}{n!}$, qui ne dépend pas de r. Ainsi, $f$ est développable en série entière sur tout $D(0,R)$.

        \item On applique ce qui précède à $h\mapsto f(z_{0}+h)$.
    \end{enumerate}
\end{proof}

\begin{remark}
    $f$ est $\mathcal{C}^{1}$ au sens complexe si et seulement si $f$ est développable en série entière au voisinage de tout point $z_{0}\in U$ (avec un rayon de convergence plus grand que $d(z_{0},\partial U)$) si et seulement si $f$ est $\mathcal{C}^{\infty}$ au sens complexe.
\end{remark}

\begin{remark}[Théorème de Liouville]
    Si $f$ est $\mathcal{C}^{1}$ au sens complexe de $\C\to\C$, alors il existe $(a_n)_{n\in\N}$ telle que pour tout $z\in\C$, $f(z)=\sum_{n=0}^{+\infty}a_nz^{n}$ (rayon de convergence $+\infty$) et pour tout $r>0$,
    \begin{equation}
        a_n=\frac{1}{2\pi r^{n}}\int_{0}^{2\pi}f\left(r\e^{\i t}\right)\e^{-\i nt}\d t.
    \end{equation}
    Si de plus $f$ est bornée sur $\C$ et pour tout $n\geqslant1$, pour tout $r>0$, on a $\left\lvert a_n\right\rvert\leqslant\frac{\left\lVert f\right\rVert_{\infty}}{r^{n}}$. Quand $r\to+\infty$, on a $a_n=0$ donc $f$ est constante.

    Application : soit $P\in\C[X]$ non constant. Comme $\lim\limits_{\left\lvert z\right\rvert}\left\lvert P(z)\right\rvert=+\infty$. On sait qu'il existe $m=\min\limits_{z\in\C}\left\lvert P(z)\right\rvert$, et si $m>0$, $f=\frac{1}{P}$ est $\mathcal{C}^{1}$ au sens complexe et bornée sur $\C$ donc constante : impossible. On vient de redémontrer le théorème de d'Alembert Gauss.
\end{remark}

\begin{proof}
    Soit $x\neq0$. Si, pour tout $p\in\N$, $u_p=\left\lvert\frac{x^{3p}}{(3p)!}\right\rvert>0$, on a 
    \begin{equation}
        \frac{u_{p+1}}{u_{p}}=\frac{x^{3}}{(3p+1)(3p+2)-3p+3}\xrightarrow[p\to+\infty]{}0.
    \end{equation}
    Donc le rayon de convergence est $R=+\infty$.

    Notons $S_1(x)=\sum_{n=0}^{+\infty}\frac{x^{3n+1}}{(3n+1)!}$ et $S_2(x)=\sum_{n=0}^{+\infty}\frac{x^{3n+2}}{(3n+2)!}$. On a 
    \begin{equation}
        \begin{array}[]{rcl}
            S_0(x)+S_1(x)+S_2(x) &=& \sum_{n=0}^{+\infty}\frac{x^{n}}{n!}=\e^{x},\\
            S_0(x)+\j S_{1}(x)+\j^{2}S_{2}(x) &=& \e^{\j x},\\
            S_0(x)+\j^{2} S_{1}(x)+\j S_{2}(x) &=& \e^{\j^{2} x}.
        \end{array}
    \end{equation}

    En effet, $\j=\j^{3n+1}$ et $\j^{2}=\j^{3n+2}$ pour tout $n\in\N$. En sommant, on a $3S_{0}(x)=\e^{x}+\e^{\j x}+\e^{\j^{2}x}$. Donc 
    \begin{equation}
        \boxed{
            S_0(x)=\frac{1}{3}\left(\e^{x}+\e^{-\frac{1}{2}x}+2\cos\left(\frac{\sqrt{3}}{2}x\right)\right).
        }
    \end{equation}
\end{proof}

\begin{remark}
    Autre méthode possible : on a $S_2'(x)=S_1(x)$ et $S_1'(x)=S_0(x)$. Donc $S_{2}''(x)+S_2'(x)+S_2(x)=\e^{x}$. L'équation homogène a pour solution générale $\lambda \e^{\j x}+\mu\e^{\j^{2}x}$, avec une solution particulière $P(x)\e^{x}$, avec $P$ constante car $1$ n'est pas racine de $X^{2}+X+1$. On trouve $\frac{\e^{x}}{3}$, donc $S_2(x)=\frac{\e^{x}}{3}+\lambda\e^{\j x}+\mu\e^{\j^{2}x}$, on identifie $S_2(0)=0$ et $S_2'(0)=0$, puis $S_0=S_2''$.
\end{remark}

\begin{proof}
    \phantom{}
    \begin{enumerate}
        \item Si \function{f_1}{\R}{\R}{x}{f(x)}
        on a pour tout $n\in\N$, $a_n=\frac{f_1^{(n)}(0)}{n!}\in\R$.
        \item Soit $\theta\in[0,\pi]$, on a 
        \begin{align}
            v\left(r\e^{\i\theta}\right)
            &=\sum_{n=0}^{+\infty}a_n\Im\left(r^{m}\e^{\i\theta n}\right),\\
            &=\sum_{n=0}^{+\infty}a_n r^{n}\sin\left(n\theta\right),
        \end{align}
        car les $a_n\in\R$.

        Pour $m\geqslant1$ fixé, on a
        \begin{equation}
            v\left(r\e^{\i\theta}\right)\sin(m\theta)=\sum_{n=0}^{+\infty}a_n r^{n}\sin(n\theta)\sin(m\theta)=\sum_{n=0}^{+\infty}f_n(\theta),
        \end{equation}
        avec $f_n$ continue sur $[0,\pi]$. Pour tout $n\in\N$, pour tout 
        $\theta\in[0,\pi]$, $\left\lvert f_n(\theta)\right\rvert\leqslant\left\lvert a_n r^{n}\right\rvert$, terme général d'une série à termes positifs convergente car $R=+\infty$. Donc $\sum f_n$ converge normalement sur $[0,\pi]$, on peut intervertir
        \begin{align}
            \int_{0}^{\pi}v\left(r\e^{\i\theta}\right)\sin(m\theta)\d\theta
            &=\sum_{n=0}^{+\infty}a_n r^{n}\int_{0}^{\pi}\sin(n\theta)\sin(m\theta)\d\theta,\\
            &=a_m r^{m}\frac{\pi}{2}.
        \end{align}
        
        \item
        \begin{lemma}
            \label{lem:3}
            Pour tout $m\in\N$, on a $\left\lvert\sin(m\theta)\right\rvert\leqslant m\left\lvert\sin(\theta)\right\rvert$.
        \end{lemma}
        \begin{proof}[Preuve du lemme~\ref{lem:3}]
            Par récurrence, car 
            \begin{align}
                \left\lvert\sin\left((m+1)\theta\right)\right\rvert
                &=\left\lvert\sin(m\theta)\cos(\theta)+\sin(\theta)\cos(m\theta)\right\rvert,\\
                &\leqslant \left\lvert\sin(m\theta)\right\rvert+\left\lvert\sin(\theta)\right\rvert,\\
                &\leqslant (m+1)\left\lvert\sin(\theta)\right\rvert.
            \end{align}
        \end{proof}

        Donc 
        \begin{equation}
            \left\lvert r^{m}a_m\right\rvert\leqslant\frac{2}{\pi}\int_{0}^{\pi}\left\lvert v(r\e^{\i\theta})\right\rvert m\left\lvert \sin(\theta)\right\rvert\d\theta.
        \end{equation}
        $\sin$ est positif sur $[0,\pi]$, et pour tout $\theta\in]0,\pi[$, si $v(r\e^{\i\theta})=0$, $f(r\e^{\i\theta})\in\R$ donc $r\e^{\i\theta}\in\R$ ce qui est exclu. Ainsi, $v(r\e^{\i\theta})$ a un signe constant sur $[0,pi]$, et 
        \begin{equation}
            \left\lvert\int_{0}^{\pi}v(r\e^{\i\theta})\sin(m\theta)\d\theta\right\rvert=\int_{0}^{\pi}\left\lvert v(r\e^{\i\theta})\right\rvert\sin(\theta)\d\theta.
        \end{equation}

        Finalement, on a $\left\lvert r^{m}a_m\right\rvert\leqslant mr\left\lvert a_1\right\rvert$, d'où $\left\lvert a_m\right\rvert\leqslant\frac{m}{r^{m-1}}\left\lvert a_1\right\vert$. Pour $m\geqslant2$, lorsque $r\to+\infty$, on obtient $a_m=0$. Donc $f$ est affine.
    \end{enumerate}
\end{proof}

\begin{proof}
    \phantom{}
    \begin{enumerate}
        \item On a 
        \begin{equation}
            \int_{0}^{2\pi}r^{\left\lvert k\right\rvert}f\left(\e^{\i(x-t)}\right)\d t=\int_{0}^{2\pi}\sum_{n=0}^{+\infty}\underbrace{a_n\e^{\i n(x-t)+\i kt}}r^{\left\lvert k\right\rvert}_{f_n(t)}\d t.
        \end{equation}
        $f_n$ est continue sur $[0,2\pi]$, avec $\left\lvert f_n(t)\right\rvert\leqslant\left\lvert a_n r^{\left\lvert k\right\rvert}\right\rvert$ terme général d'une série à termes positifs convergente. donc $\sum_{n\geqslant0}f_n(t)$ converge normalement sur $[0,2\pi]$. Ainsi,
        \begin{align}
            \int_{0}^{2\pi}r^{\left\lvert k\right\rvert}\e^{\i kt}f\left(\e^{\i(x-t)}\right)\d t
            &=\sum_{n=0}^{+\infty}a_n\e^{\i nx}r^{\left\lvert k\right\rvert}\int_{0}^{2\pi}\e^{\i kt}\e^{-\i nt}\d t,\\
            &=
            \left\lbrace
                \begin{array}[]{ll}
                    2\pi r^{\left\lvert k\right\rvert}a_{k}\e^{\i kx} &\text{si }k\geqslant0,\\
                    0 &\text{sinon.}
                \end{array}
            \right.
        \end{align}

        Puis 
        \begin{equation}
            \int_{0}^{+\infty}P_r(t)f\left(\e^{\i(x-t)}\right)\d t=\int_{0}^{2\pi}\sum_{k=-\infty}^{+\infty}\underbrace{r^{\left\lvert k\right\rvert}\e^{\i kt}f\left(\e^{\i (x-t)}\right)}_{g_k(t)}\d t.
        \end{equation}
        $g_k$ est continue sur $[0,2\pi]$, et $\left\lvert g_k(t)\right\rvert\leqslant r^{\left\lvert k\right\rvert}\left\lVert f\right\rVert_{\infty,\overline{D(0,1)}}$ et $\sum_{k\in\Z}r^{\left\lvert k\right\rvert}<\infty$. On a donc convergence normale sur $[0,2\pi]$, et 
        \begin{align}
            \int_{0}^{2\pi}P_r(t)f\left(\e^{\i(x-t)}\right)\d t
            &=\sum_{k=-\infty}^{+\infty} r^{\left\lvert k\right\rvert}\int_{0}^{2\pi}\e^{\i kt}f\left(\e^{\i(x-t)}\right)\d t,\\
            &=\sum_{k=0}^{+\infty} 2\pi r^{k}a_k\e^{\i kx},\\
            &=2\pi f\left(r\e^{\i x}\right).
        \end{align}

        \item On a 
        \begin{align}
            P_r(x)
            &=\sum_{k=0}^{+\infty}r^{k}\e^{\i kx}+\sum_{k=0}^{+\infty}r^{k}\e^{-\i kx}-1,\\
            &=\frac{1}{1-r\e^{\i x}}+\frac{1}{1-r\e^{-\i x}}-1,\\
            &=\frac{1-r^{2}}{1+r^{2}-2r\cos(x)},
        \end{align}
        et $1+r^{2}-2r\cos(x)=\left(1-r\cos(x)\right)^{2}+r^{2}\sin^{2}(x)>0$, donc $P_r>0$. On applique le résultat du a) pour $f=1$ et on obtient 
        \begin{equation}
            \frac{1}{2\pi}\int_{0}^{2\pi}P_r(t)\d t=1.
        \end{equation}

        \item Si $f(\U)\subset\U$, prenons $z\in D(0,1)$, soit $z+r\e^{\i x}$, $r\in[0,1[$ et $x\in\R$,
        \begin{align}
            \left\lvert f\left(r\e^{\i x}\right)\right\rvert 
            &= \left\lvert \frac{1}{2\pi}\int_{0}^{2\pi}P_r(t)f\left(\e^{\i(x-t)}\right)\d t\right\rvert,\\
            &\leqslant\frac{1}{2\pi}\int_{0}^{2\pi}P_r(t)\left\lvert f\left(\e^{\i(x-t)}\right)\right\rvert\d t,\\
            &\leqslant\frac{1}{2\pi}\int_{0}^{2\pi}P_r(t)\d t=1.
        \end{align}
        Donc $f(z)\in\overline{D(0,1)}$ et $f\left(\overline{D(0,1)}\right)\subset\overline{D(0,1)}$.
    \end{enumerate}
\end{proof}

\begin{proof}
    \phantom{}
    \begin{enumerate}
        \item L'espérance vaut la série harmonique $H_n\underset{n\to+\infty}{\sim}\ln(n)$ par linéarité. Par indépendance, la variance vaut $\sum_{k=1}^{n}\frac{1}{k}\left(1-\frac{1}{k}\right)\underset{n\to+\infty}{\sim}\ln(n)$.
        \item On a 
        \begin{equation}
            \left(\left\lvert \frac{R_n}{\ln(n)}-1\right\rvert\right)\subset \underbrace{\left(\left\lvert \frac{R_n}{\ln(n)}-\frac{\mathbb{E}(R_n)}{\ln(n)}\right\rvert>\frac{\varepsilon}{2}\right)}_{B_n}\bigcup\underbrace{\left(\left\lvert\frac{\mathbb{E}(R_n)}{\ln(n)}-1\right\rvert>\frac{\varepsilon}{2}\right)}_{C_n}.
        \end{equation}
        $C_n$ est nul à partir d'un certain rang car $\lim\limits_{n\to+\infty}\frac{\mathbb{E}(R_n)}{\ln(n)}=1$. De plus, 
        \begin{equation}
            B_n=\left(\left\lvert R_n-\mathbb{E}(R_n)\right\rvert\frac{\varepsilon}{2}\ln(n)\right),    
        \end{equation}
        donc d'après l'inégalité de Bienaymé-Tchebychev, $B_n<\frac{4\mathbb{V}(R_n)}{\varepsilon^{2}\ln^{2}(n)}\underset{n\to+\infty}{\sim}\frac{4}{\varepsilon^{2}\ln(n)}\xrightarrow[n\to+\infty]{}0$.

        \item On a 
        \begin{align}
            G_{R_n}(t)
            &=\mathbb{E}\left(t^{R_n}\right),\\
            &=\prod_{k=1}^{n}\mathbb{E}(t^{\chi_{A_k}}),\\
            &=\prod_{k=1}^{n}\left(1-\frac{1}{k}+\frac{t}{k}\right),\\
            &=\frac{t}{n!}\prod_{k=1}^{n}\left(k-1+t\right),
        \end{align}
        car les $(\chi_{A_k})_{k\geqslant1}$ sont indépendants. $\mathbb{P}(R_n=1)$ est le coefficient en $t$ de $G_{R_n}$, et vaut donc $\frac{(n-1)!}{n!}=\frac{1}{n}$. De même, $\mathbb{P}(R_n=2)$ est le coefficient en $t^{2}$ de $G_{R_n}$ et vaut donc $\frac{1}{n!}\sum_{k=2}^{n}\frac{(n-1)!}{k-1}=\frac{1}{n}\sum_{k=1}^{n-1}\frac{1}{k}$.

        \item On a $T_n=\sum_{k=na+1}^{nb}\chi_{A_k}$, donc 
        \begin{equation}
            G_{T_n}(t)=\prod_{k=na+1}^{nb}\left(1-\frac{1}{k}+\frac{t}{k}\right).
        \end{equation}
        Ainsi, $\ln\left(G_{T_n}(t)\right)=\sum_{k=na+1}^{nb}\ln\left(1+\frac{t-1}{k}\right)$. Pour $x>1$, soit $g(x)=\ln(1+x)-x+\frac{x^{2}}{2}$. On a $g'(x)=\frac{x^{2}}{1+x}\geqslant0$ et $g(0)=0$ donc $g\geqslant0$.

        On a 
        \begin{equation}
            \sum_{k=na+1}^{nb}\frac{t-1}{k}-\frac{1}{2}\sum_{k=na+1}^{nb}\left(\frac{t-1}{k}\right)^{2}\leqslant\ln\left(G_n(t)\right)\leqslant\sum_{k=na+1}^{nb}\frac{t-1}{k}.
        \end{equation}
        Comme $0\leqslant\frac{1}{2}\sum_{k=na+1}^{nb}\left(\frac{t-1}{k}\right)^{2}\leqslant\frac{(t-1)^{2}}{2}\sum_{k=na+1}^{+\infty}\frac{1}{k^{2}}\xrightarrow[n\to+\infty]{}0$, et $\sum_{k=na+1}^{nb}\frac{t-1}{k}=(t-1)(H_{nb}-H_{na})\xrightarrow[n\to+\infty]{}(t-1)\ln\left(\frac{b}{a}\right)$.
        Donc $\lim\limits_{n\to+\infty}G_{T_n}(t)=\e^{\ln\left(\frac{b}{a}\right)(t-1)}$. Il s'agit de la fonction génératrice d'une variable aléatoire suivant une loi de Poisson de paramètre $\ln\left(\frac{a}{b}\right)$.
    \end{enumerate}
\end{proof}

\begin{proof}
    \phantom{}
    \begin{enumerate}
        \item 
        \begin{enumerate}
            \item Soit $a_n=\P(S_n=0)$. On a $0\leqslant a_n\leqslant1$ donc le rayon de convergence de $f$ est plus grand que 1 et $f$ est définie et $\mathcal{C}^{\infty}$, de même pour $g$.
            \item Soit $g_k(t)=\P(T=k)t^{k}$. Pour tout $t\in[0,1]$, on a $\left\lvert g_k(t)\right\rvert\leqslant\P(T=k)$, terme général d'une série à termes positifs convergente indépendant de $t$, car $\sum_{k=0}^{+\infty}\P(T=k)=\pi\leqslant1$, donc $\sum g_k$ converge normalement sur $[0,1]$. Donc $\lim\limits_{t\to1^{-}}g(t)=\sum_{k=1}^{+\infty}\P(T=k)=\pi$.
        \end{enumerate}

        \item On a $(S_{m+1}-S_m,\dots,S_{m+k}-S_{m})=(X_{m+1},X_{m+1}+X_{m+2},\dots,X_{n+1}+\dots+X_{m+k})$. $X_{m+1}+\dots X_{m+r}$ a pour loi la convoluée de la loi de $X$ r fois par elle-même car $X_i\sim X$ et les $(X_i)_{1\geqslant i}$ sont indépendants. Donc $X_m+\dots+X_{m+r}\sim X_1+\dots+X_r$, d'où le résultat.
        
        \item 
        \begin{enumerate}
            \item $\P(S_n=0)=\sum_{k=1}^{n}\P(T=k)\P_{T=k}(S_n=0)$ car si on revient à 0 à l'instant $n$, le 1er retour en 0 a eu lieu à un instant $k\in\left\llbracket1,n\right\rrbracket$. D'après ce qui précède, on a donc 
            \begin{equation}
                \P_{T=k}(S_n=0)=\P_{S_k=0}(S_n-S_k=0)=\P(S_{n-k}=0).
            \end{equation}

            \item Posons $P(T=0)=0$, d'où $g(t)=\sum_{k=0}^{+\infty}\P(T=k)t^{k}$ et pour tout $n\geqslant1$,
            \begin{equation}
                \sum_{k=0}^{n}\P(T=k)\P(S_{n-k}=0)=\P(S_n=0),
            \end{equation}
            puis par produit de Cauchy, 
            \begin{equation}
                f(t)=\P(S_0=0)+\sum_{n=1}^{+\infty}\P(S_n=0)t^{n}=1+f(t)g(t).
            \end{equation}
        \end{enumerate}

        \item 
        \begin{enumerate}
            \item Soit $j=\left\lvert\left\lbrace i\in\left\llbracket1,n\right\rrbracket\middle|X_i=1\right\rbrace\right\rvert$. On a $S_n=j-(n-j)=2j-n$ (on est allé $j$ fois à droite et $n-j$ fois à gauche). Si $n=2p+1$, $S_n=S_{2p+1}=2j-2p-1\neq0$. Pour que $S_{2n}=0$, il faut et il suffit que $j=n$, ce qui revient à une loi binomiale $\mathcal{B}(2n,p)$ d'où $\P(S_{2n}=0)=\binom{2n}{n}(pq)^{n}$.
            
            \item On a $f(t)=\sum_{n=0}^{+\infty}\P(S_n=0)t^{n}=\sum_{n=0}^{+\infty}\binom{2n}{n}(pq)^{n}t^{2n}$. Comme $p\in]0,1[$, $p(1-p)=pq\leqslant\frac{1}{4}$, et si $t\in[0,1[$, 
            \begin{align}
                \frac{1}{\sqrt{1-4pqt^{2}}}
                &=(1-4pq^{2})^{-1},\\
                &=\sum_{n=0}^{+\infty}\frac{(-1)^{n}\left(-\frac{1}{2}\right)\left(-\frac{1}{2}-1\right)\dots\left(-\frac{1}{2}-n+1\right)}{n!}(4pqt^{2})^{n},\\
                &=\sum_{n=0}^{+\infty}\frac{\frac{1}{2}\left(1+\frac{1}{2}\right)\dots\left(n-1+\frac{1}{2}\right)}{n!}(4pqt^{2})^{n},\\
                &=\sum_{n=0}^{+\infty}\frac{1\times 3\dots \times (2n-1)}{2^{n}n!}(4pqt^{2})^{n},\\
                &=\sum_{n=0}^{+\infty}\frac{(2n)!}{2^{2n}(n!)^{2}}(4pqt^{2})^{n},\\
                &=\sum_{n=0}^{+\infty}\binom{2n}{n}p^{n}q^{n}t^{2n}=f(t).
            \end{align}
        \end{enumerate}

        \item 
        \begin{enumerate}
            \item Pour tout $t\in]0,1[$, on a 
            \begin{equation}
                g(t)=\frac{f(t)-1}{f(t)}=1-\sqrt{1-4pqt^{2}},
            \end{equation}
            car $f(t)\neq0$. D'après 1.(b), on a 
            \begin{equation}
                \pi=\lim\limits_{t\to1^{-}}g(t)=1-\sqrt{1-4p(1-p)}=1-\left\lvert 2p-1\right\rvert=1-\left\lvert p-q\right\rvert,
            \end{equation}
            et $\pi=1$ si et seulement si $p=q=\frac{1}{2}$.

            \item Pour tout $t\in[0,1]$, $g(t)=1-\sqrt{1-4p(1-p)}$. Or $\left\lvert pqt^{2}\right\rvert<1$ donc $g$ est développable en série entière sur $[0,1[$, et on a 
            \begin{align}
                g(t)
                &=1-\left(1+\sum_{n=1}^{+\infty}\frac{\frac{1}{2}\left(\frac{1}{2}-1\right)\dots\left(\frac{1}{2}-n+1\right)}{n!}(-4pqt^{2})^{n}\right),\\
                &=\sum_{n=1}^{+\infty}\frac{1\times 1\times 3\times\dots\times(2n-3)}{2^{n}n!}(4pqt^{2})^{n},\\
                &=\sum_{n=1}^{+\infty}\frac{(2n-2)!}{2^{n-1}(n-1)!2^{n}n!}4^{n}(pq)^{n}t^{2n},\\
                &=\sum_{n=1}^{+\infty}\frac{(2n-2)!}{n(n-1)!^{2}}4(pq)^{n}t^{2n}.
            \end{align}
            Par unicité du développement, on a $\P(T=2n+1)=0$ et $\P(T=2n)=\binom{2n-2}{n-1}\frac{4(pq)^{n}}{n}$.
        \end{enumerate}

        \item 
        \begin{enumerate}
            \item Si $p=\frac{1}{2}$, on a $\pi=1$ d'où $\P(T=+\infty)=0$. Donc 
            \begin{equation}
                \mathbb{E}(T)=\sum_{n=1}^{+\infty}\binom{2n-2}{n-1}\frac{2n}{n4^{n-1}}=\sum_{n=0}^{+\infty}\binom{2n}{n}\frac{1}{4^{n}}.
            \end{equation}
            Or $\binom{2n}{n}\frac{1}{4^{n}}=\frac{(2n)!}{(n!)^{2}}\frac{1}{4^{n}}\underset{n\to+\infty}{\sim}\frac{(2n)^{2n}}{n^{2n}}\frac{\sqrt{4\pi n}}{2\pi n}\frac{1}{4^{n}}\underset{n\to+\infty}{\sim}\frac{1}{\sqrt{\pi n}}$, terme général d'une série divergente, d'où le résultat.

            \item Si $p\neq\frac{1}{2}$, on a 
            \begin{equation}
                \mathbb{E}(T\times \mathbf{1}_{T<+\infty})=\sum_{k=1}^{+\infty}k\P(T=k)=g'(1)=\frac{4pq}{\sqrt{1-4pq}},
            \end{equation}
            car $g(t)=1-\sqrt{1-4pqt^{2}}$. On a $\P(T<+\infty)=\pi=1-\left\lvert p-q\right\rvert$ et comme $\sqrt{1-4pq}=\left\lvert p-q\right\rvert$, d'où 
            \begin{equation}
                \mathbb{E}_{T<+\infty}(T)=\frac{4pq}{\left\lvert p-q\right\rvert(1-\left\lvert p-q\right\rvert)}.
            \end{equation}
        \end{enumerate}

        \item $\pi=1$ si et seulement si $p=\frac{1}{2}$ d'après 6.
        
        \item $f$ est croissante sur $[0,1[$ donc il existe $l\in\overline{\R}$ telle que $\lim\limits_{t\to1^{-}}f(t)=l=\sup\limits_{t<1}f(t)$.
        
        \begin{itemize}
            \item Si $\sum_{n=0}^{+\infty}\P(S_n=0)=+\infty$, soit $N\in\N$, on a $\sum_{n=0}^{N}\P(S_n=0)t^{n}\leqslant f(t)\leqslant l$. $N$ étant fixé, on peut faire tendre vers 1 d'où $\sum_{n=0}^{+\infty}\P(S_n=0)\leqslant l$. Lorsque $N\to+\infty$, on obtient $\lim\limits_{t\to1^{-}}f(t)=+\infty$.
            
            \item Si $(S_n)_{n\geqslant0}$ n'est pas récurrente, alors $\pi\neq1$ et $p\neq\frac{1}{2}$. Pour tout $t\in[0,1[$,
            \begin{equation}
                f(t)=\frac{1}{\sqrt{1-4pqt^{2}}}\xrightarrow[t\to1^{-}]{}\frac{1}{\sqrt{1-4pq}},
            \end{equation}
            défini car $4pq=4p(1-p)<1$.

            \item Si $p=\frac{1}{2}$, $\P(S_n=0)=\binom{2n}{n}\frac{1}{4^{n}}$ d'après 4.(a), et $\P(S_n=0)\underset{n\to+\infty}{\sim}\frac{1}{\sqrt{\pi n}}$, terme général d'une série divergente donc $\sum_{n=0}^{+\infty}\P(S_n=0)=+\infty$.
        \end{itemize}

        \item 
        \begin{enumerate}
            \item On a 
            \begin{align}
                \mathbb{E}(N)
                &=\lim\limits_{n\to+\infty}\mathbb{E}(N_n),\\
                &=\sum_{k=1}^{+\infty}\mathbb{E}(\mathbf{1}_{\left\lbrace S_n=0\right\rbrace}),\\
                &=\sum_{k=1}^{+\infty}\P(S_k=0),\\
                &=f(1)-\P(S_0=0),\\
                &=f(1)-1.
            \end{align}
            Or $f(1)=1+f(1)g(1)$ et $g(1)=\pi$, donc $f(1)=\frac{1}{1-\pi}$, d'où 
            \begin{equation}
                \mathbb{E}(N)=\frac{\pi}{1-\pi}.
            \end{equation}

            \item Si $p=\frac{1}{2}$, $\sum\P(S_n=0)$ diverge donc $\mathbb{E}(N)=+\infty$.
        \end{enumerate}

        \item Pour tout $k\geqslant1$, on a 
        \begin{equation}
            \P(S_k=x)=\sum_{j=0}^{k}\P(T^{x}=j)\P_{T^{x}=j}(S_k=x).
        \end{equation}
        Par produit de polynômes, $f_{n,x}(t)=\P(S_0=x)+\sum_{k=1}^{n}\P(S_k=x)t^{k}$ et $\P(S_0=x)=0$ car $x\neq0$. Soit 
        \begin{equation}
            f_{n,x}(t)=\sum_{k=1}^{n}\P(T^{x}=j)\P(X_{k-j}=0)t^{k},
        \end{equation}
        pour $t\in[0,1]$. Soit $f_n(t)=\sum_{k=0}^{n}\P(S_k=0)t^{k}$, on a 
        \begin{equation}
            f_n(t)g_{n,x}(t)=\left(\sum_{l=0}^{n}\P(S_l=0)t^{k}\right)\left(\sum_{j=0}^{n}\P(T^{x}=j)t^{j}\right),
        \end{equation}
        donc $f_{n,x}(t)\leqslant f_n(t)g_{n,x}(t)$. Or $N_{n,x}=\sum_{j=1}^{n}\mathbf{1}_{\left\lbrace S_j=x\right\rbrace}$ et 
        \begin{equation}
            \mathbb{E}(N_{n,x})=f_{n,x}(1)\leqslant f_n(1)g_{n,x}(1)=\mathbb{E}(N_{n})\sum_{k=1}^{n}\P(T^{x}=k)\leqslant\mathbb{E}(N_{n}),
        \end{equation}
        donc $\mathbb{E}(N_{n,x})\leqslant\mathbb{E}(N_{n})$.

        \item 
        \begin{equation}
            \left\lbrace \left(\left\lVert S_n\right\rVert\right)_{n\geqslant1}\right\rbrace=\left\lbrace\exists A\geqslant0\middle|\left\lbrace n\in\N\middle|\left\lVert S_n\right\rVert\leqslant A\right\rbrace\text{ est fini}\right\rbrace=\mathcal{A}.    
        \end{equation}
        Comme $S_n$ est à valeur dans $\Z^{d}$, il y a un nombre fini de points dans $\overline{B(0,A)}$. On a 
        \begin{equation}
            \mathcal{A}=\left\lbrace x\in\Z^{d}\middle|\left\lbrace n\in\N\middle| S_n=x\right\rbrace\text{ est fini}\right\rbrace.
        \end{equation}
        Soit $x\in\Z^{d}$ fixé, et $\mathcal{A}^{x}=\left\lbrace\left\lbrace n\in\N\middle| S_n=x\right\rbrace\text{ est infini}\right\rbrace$, on a 
        \begin{equation}
            \mathcal{A}\biguplus_{x\in\Z^{d}}\mathcal{A},
        \end{equation}
        (union disjointe) et $\P(\mathcal{A})=\sum_{x\in\Z^{d}}\P(\mathcal{A}^{x})$ dénombrable.

        Soit $N^{x}=\left\lvert\left\lbrace j\in\N\middle| S_j=x\right\rbrace\right\rvert$ (à valeurs dans $\N\cup \lbrace+\infty\rbrace$). $(N_{n}^{x})_{n\geqslant1}$ converge vers $N^{x}$ en croissant. Or $(\mathbb{E}(N_{n}^{x}))_{n\geqslant1}$ converge vers $\mathbb{E}(N^{x})$ dans $\R_{+}\cup\lbrace+\infty\rbrace$, et comme il s'agit d'une marche transitoire, on a 
        \begin{equation}
            \mathbb{E}(N_{n}^{x})\leqslant\mathbb{E}(N_{n})\leqslant\mathbb{E}(N)<+\infty,
        \end{equation}
        donc $\mathbb{E}(N^{x})$ est fini et 
        \begin{equation}
            \mathbb{E}(N^{x})=\sum_{k=1}^{+\infty}k\P(N^{x}=k)+(+\infty)\underbrace{\P(N^{x}=+\infty)}_{\P(\mathcal{A}^{x})}.
        \end{equation}
        Nécessairement, $\P(\mathcal{A}^{x})=0$ pour tout $x\in\Z^{d}$ et $\P(\mathcal{A})=0$.
    \end{enumerate}
\end{proof}

\end{document}
\documentclass[12pt]{article}
\usepackage{style/style_sol}

\begin{document}

\begin{titlepage}
	\centering
	\vspace*{\fill}
	\Huge \textit{\textbf{Solutions MP/MP$^*$\\ Espaces préhilbertiens}}
	\vspace*{\fill}
\end{titlepage}

\begin{proof}
	\phantom{}
	\begin{enumerate}
		\item $\varphi$ est une forme bilinéaire symétrique. Soit $f\in E$, on a 
		\begin{equation}
			\varphi(f,f)=\int_{0}^{1}f^{2}+f'^{2}\geqslant 0.
		\end{equation}
		Si $\varphi(f,f)=0$, $f^{2}$ étant continue et positive, on a $f=0$. 

		\item Soit $(f,g)\in V\times W$. On a 
		\begin{align}
			\int_{0}^{1}fg +f'g'
			&=\int_{0}^{1}fg+\int_{0}^{1}f'g',\\
			&=\int_{0}^{1}fg+[fg']_{0}^{1}-\int_{0}^{1}fg'',\\
			&=\int_{0}^{1}fg-\int_{0}^{1}fg,\\
			&=0.
		\end{align}

		Donc $V$ et $W$ sont orthogonaux.

		Soit $h\in E$. Supposons qu'il existe $(f,g)\in V\times W$ tel que $h=f+g$. Il existe $(a,b)\in\R^{2}$ tel que pour tout $x\in[0,1]$, $g(x)=a\e^{x}+b\e^{-x}$ et $f(0)=f(1)=0$. Donc $h(0)=a+b$ et $h(1)=a\e+\frac{b}{\e}$. On trouve donc 
		\begin{equation}
			\begin{array}[]{rcl}
				a &=& \frac{\frac{h(0)}{\e}-h(1)}{\frac{1}{\e}-\e},\\
				b &=& \frac{eh(0)-h(1)}{\e-\frac{1}{\e}}.
			\end{array}
		\end{equation}

		Réciproquement, en définissant $a$ et $b$ comme précédemment, on pose $f=h-g$. On a bien $f\in V$ et $h=f+g$. Finalement, $E= V\overset{\perp}{\oplus} W$.

		\item Si $h_0\in W\cap E_{\alpha,\beta}$, il existe $(a,b)\in\R^{2}$ tel que pour tout $x\in[0,1]$, $h_0(x)=a\cosh(x)+b\sinh(x)$, $h_0()=0=a=\alpha$ et $h_0(1)=a\cosh(1)+b\sinh(1)=\beta$ d'où 
		\begin{equation}
			b=\frac{\beta-\alpha\cosh(1)}{\sinh(1)}.
		\end{equation}

		Réciproquement, $h_0$ ainsi défini est dans $W\cap E_{\alpha,\beta}$.

		Pour tout $h\in E_{\alpha,\beta}$, $h-h_{0}\in V$, d'après le théorème de Pythagore, on a $\left\lVert h_0\right\rVert\leqslant\left\lVert h\right\rVert$. Ainsi, la borne supérieure est $\left\lVert h_0\right\rVert^{2}$.
	\end{enumerate}
\end{proof}

\begin{proof}
	\phantom{}
	\begin{enumerate}
		\item Si $P\in\ker(\Delta)$, on a $P(X+a)=P(X)$ et par itération, pour tout $k\in\N$, $P(ka)=P(0)$, donc $P$ est constant. Ainsi, $\ker(\Delta)=\R_{0}[X]$. On a 
		\begin{equation}
			\Delta(X^{k})=(X+a)^{k}-X^{k}=\sum_{i=0}^{k-1}\binom{k}{i}X^{i}a^{k-i},
		\end{equation}
		de degré $k-1$ et de coefficient dominant $ka$. Ainsi, si $P=\sum_{i=0}^{n}X^{i}$ avec $a_n\neq0$, on a $\Delta(P)=\sum_{k=0}^{n}a_k\Delta(X^{k})$ de degré $n-1$ et de coefficient dominant $a_n na$.

		\item Si $k\geqslant\deg(P)+1$, on a $\Delta^{k}P=0$, et $\Delta^{\deg(P)}(P)=a_n\times n!a^{n}\neq0$. $\varphi$ est une somme finie, une forme bilinéaire symétrique, $\varphi(P,P)\geqslant0$ et si $P\neq0$, $\left(\Delta^{\deg(P)}P(0)\right)^{2}>0$ donc $\varphi(P,P)>0$.
		
		\item Soit $n\in\N$, cherchons $P_n$ de coefficient dominant strictement positif avec $\deg(P_n)=n$ tel que pour tout $k\in\left\llbracket0,\right\rrbracket$, $\Delta^{k}P_n(0)=\delta_{n,k}$. On a $P(0)=0$ pour $k=0$, et $\Delta(P)(0)=P(a)-P(0)=0$ donc $P(a)=0$. De proche en proche, pour tout $k\in\left\llbracket0,n-1\right\rrbracket$, $P(ka)=0$. Donc $P_n=\alpha_{n}X(X-a)\dots(X-(n-1)a)$, et $\Delta^{n}(P_n)(0)=1$ d'où 
		\begin{equation}
			\alpha_n = \frac{1}{n!a^{n}}.
		\end{equation}

		Réciproquement, en définissant $P_n$ comme ci-dessus, avec $P_0=1$, on a 
		\begin{align}
			\Delta(P_n)
			&=\alpha_n\left[(X+a)(X-a)\dots(X-(n-2)a)-X(X-a)\dots(X-(n-1)a)\right],\\
			&=\alpha_n X(X-a)\dots(X-(n-2)a)(na),\\
			&=P_{n-1}.
		\end{align}
		Par récurrence, $\Delta^{k}(P_n)(0)=\delta_{n,k}$ donc $(P_n)$ est orthonormée.
	\end{enumerate}
\end{proof}

\begin{proof}
	On choisit $E=\mathcal{C}^{0}\left(\left[0,\frac{\pi}{2}\right],\R\right)$ muni du produit scalaire $\int_{0}^{\frac{\pi}{2}}f(t)g(t)\d t=(f|g)$. Soit $f_0\colon x\mapsto 1$ et $f_1\colon x\mapsto x\in E$, on note $F=\Vect(f_0,f_1)$. Trouver $I(a,b)$ revient à calculer $p_F(\sin)=a_0 f_0+b_0 f_1$ avec $\sin-p_{F}(\sin)\in F^{\perp}$.

	On a 
	\begin{equation}
		(\sin -a_{0}f_1-b_{0}f_0|f_{0})=(\sin -a_{0}f_1-b_{0}f_0|f_{1})=0.
	\end{equation}
	D'où 
	\begin{equation}
		\begin{array}[]{rcl}
			\int_{0}^{\frac{\pi}{2}}\left(\sin(x)-a_0 x-b_{0}\right)\d x&=&0,\\
			\int_{0}^{\frac{\pi}{2}}\left(x\sin(x)-a_0 x^{2}-b_{0}x\right)\d x&=&0.
		\end{array}
	\end{equation}
	On trouve ainsi les valeurs de $b_0=\frac{8}{\pi^{2}}(\pi-3)$ et de $a_0=\frac{96}{\pi^{3}}\left(1-\frac{\pi}{4}\right)$ en résolvant un système de deux équations à deux inconnues en calculant les intégrales (utiliser une intégration par partie pour le calcul de celle d'intégrande $x\sin(x)$). Enfin,
	\begin{align}
		I(a,b)
		&=\left\lVert\sin-p_F(\sin)\right\rVert^{2},\\
		&=\left\lVert \sin\right\rVert^{2}-\left\lVert p_F(\sin)\right\rVert^{2}.
	\end{align}
	On finit par trouver $I(a,b)=1$.
\end{proof}

\begin{remark}
	Ce genre nde problème se résout souvent en se ramenant à un espace euclidien et en utilisant nos connaissances sur le projeté orthogonal.
\end{remark}

\begin{proof}
	\phantom{}
	\begin{enumerate}
		\item Si $k\geqslant\min(\deg(P),\deg(Q))$, $P^{(k)}(a_k)Q^{(k)}(a_k)=0$, donc $(\cdot|\cdot)$ est définie et est une forme bilinéaire symétrique positive.
		Soit $P\in E\setminus\left\lbrace0\right\rbrace$ avec $\deg(P)=k_{0}$. Si $(P|P)=0$, alors $\sum_{k=0}^{+\infty}\left(P^{(k)}(a_k)\right)^{2}=0$. En particulier, $P^{(k_0)}(a_{k_0})=0$ ce qui est absurde par définition de $k_0$. Donc $(\cdot|\cdot)$ est un produit scalaire.

		\item On applique le procédé d'orthogonalisation de Schmidt à $(X^{n})_{n\in\N}$.
		\item Soit $Q_n=\alpha_{0,n}+\alpha_{1,n}X+\dots+\alpha_{n,n}X^{n}$ de degré $n$. On a 
		\begin{equation}
			0=Q_n(a_0)=Q_n'(a_1)=\dots=Q_n^{(n-1)}(a_{n-1}),
		\end{equation}
		et $Q_n^{(n)}(a_n)=1$ si et seulement si 
		\begin{equation}
			\begin{array}[]{rcl}
				\alpha_{0,n}+\alpha_{1,n}a_{0}+\dots + \alpha_{n,n}a_{0}^{,}&=&0,\\
				\alpha_{1,n}+2\alpha_{2,n}a_{1}+\dots+n\alpha_{n,n}a_{1}^{n-1}&=&0,\\
				\vdots&\vdots&\vdots,\\
				(n-1)!\alpha_{n-1,n}+(n\times\dots\times2)\alpha_{n,n}a_{n-1}&=&0,\\
				n!\alpha_{n,n}&=&1.
			\end{array}
		\end{equation}
		Il y a une unique solution car c'est un système triangulaire. Ainsi, pour tout $k\in\left\lbrace0,\dots,n\right\rbrace$, $Q_n^{(k)}(a_k)=\delta_{n,k}$ et pour $k>n+1$, c'est vrai aussi car $Q_n^{(k)}=0$.

		On obtient ainsi une famille de polynômes telle que pour tout $n\in\N$, $\deg(Q_n)=0$ et le coefficient dominant de $Q_n$ est strictement positif. De plus pour tout $(n,m)\in\N^{2}$,
		\begin{equation}
			(Q_n|Q_m)=\sum_{k=0}^{+\infty}\underbrace{Q_n^{(k)}(a_k)Q_m^{(k)}(a_k)}_{\delta_{n,k}\delta_{m,k}}=\delta_{n,m}.
		\end{equation}

		Par unicité, $Q_n=P_n$ et $P_n^{(k)}(a_k)=\delta_{n,k}$.

		\item Comme $\int_{a_{n-1}}^{t_{n-1}}d t_{n}$ est un polynôme en $t_{n-1}$ de degré 1, si 
		\begin{equation}
			A_n(x)=\int_{a_{0}}^{x}\int_{a_1}^{t_1}\dots\int_{a_{n-1}}^{t_{n-1}}\d t_{n}\d t_{n-1}\dots\d t_{2}\d t_{1},
		\end{equation}
		alors c'est un polynôme en $x$ de degré $n$ et de coefficient dominant $\frac{1}{n!}$. De plus, pour tout $k\in\left\lbrace0,\dots,n\right\rbrace$,
		\begin{equation}
			A_n^{(k)}(t_k)=\int_{a_k}^{t_k}\dots\int_{a_{n-1}}^{t_{n-1}}\d t_{n}\dots\d t_{k+1}.
		\end{equation}
		Donc si $k\leqslant n-1$, $A_n^{(k)}(a_k)=0$, et si $k>n$, $A_{n}^{(k)}=0=A_n^{(k)}(a_k)$. Enfin, $A_n^{(n)}=1$. Donc $A_n=P_n$ par unicité.

		\item On a $P_0=1$, $P_1(x)=x$. On trouve $P_2(x)=\frac{1}{2}x(x-2\alpha)$ et $P_3(x)=\frac{x}{6}\left(x^{2}-6\alpha x+9\alpha^{2}\right)$. On vérifie alors par récurrence que $P_n=\frac{x}{n!}\left(x-n\alpha\right)^{n-1}$.
	\end{enumerate}
\end{proof}

\begin{proof}
	On note $\varphi(a,b,c)$ l'intégrale. On pose 
	\begin{equation}
		E=\left\lbrace f\in\mathcal{C}^{0}(\R_{+},\R)\middle| f^{2}(x)\e^{-x}\in\mathcal{L}^{1}(\R)\right\rbrace.
	\end{equation}
	$0\in E$, si $f\in E$, alors $\lambda f\in E$ pour tout $\lambda\in\R$ et si $(f,g)\in E^{2}$, alors $\left\lvert fg\right\rvert\leqslant\frac{1}{2}(f^{2}+g^{2})$ donc $g(x)g(x)\e^{-x}\in\mathcal{L}^{1}(\R_+)$, et $(f+g)^{2}=f^{2}+g^{2}+2fg$ donc $f+g\in E$.

	On définit pour tout $(f,g)\in E^{2}$,
	\begin{equation}
		(f|g)=\int_{0}^{+\infty}f(x)g(x)\e^{-x}\d x,
	\end{equation}
	qui est un produit scalaire sur $E$.

	Soit $f_k\colon x\mapsto x^{k}$ de $\R_{+}\to\R$, $F=\Vect(f_0,f_1,f_2)$. On a $\varphi(a,b,c)=\left\lVert\sin-af_2-bf_1-cf_0\right\rVert^{2}$ minimum pour $af_2-bf_1-cf_0=p_F(\sin)$.

	Par définition, $(a,b,c)$ vérifient $(\sin-af_2-bf_1-cf_0|f_k)=0$ pour tout $k\in\left\lbrace0,1,2\right\rbrace$.

	On a 
	\begin{equation}
		(\sin|f_0)=\int_{0}^{+\infty}\sin(x)\e^{-x}\d x=\Im\left(\int_{0}^{+\infty}\e^{-x(1-\i)}\d x\right)=\Im\left(\frac{1}{1-\i}\right)=\frac{1}{2},
	\end{equation}
	\begin{align}
		(\sin|f_1)
		&=\int_{0}^{+\infty}x\sin(x)\e^{-x}\d x,\\
		&=\Im\left(\int_{0}^{+\infty}x\e^{-x(1-\i)}\d x\right),\\
		&=\Im\left(\left[\frac{x\e^{-x(1-\i)}}{-(1-\i)}\right]_{0}^{+\infty}+\frac{1}{1-\i}\int_{0}^{+\infty}\e^{-x(1-\i)}\d x\right),\\
		&=\Im\left(0+\frac{1}{(1-\i)^{2}}\right),\\
		&=\frac{1}{2},
	\end{align}
	\begin{align}
		(\sin|f_2)
		&=\int_{0}^{+\infty}x^{2}\sin(x)\e^{-x}\d x,\\
		&=\Im\left(\int_{0}^{+\infty}x^{2}\e^{-x(1-\i)}\d x\right),\\
		&=\Im\left(0+\frac{2}{(1-\i)^{3}}\right),\\
		&=-\frac{1}{2}.
	\end{align}
	Soit $(i,j)\in\left\llbracket0,2\right\rrbracket^{2}$, on a 
	\begin{align}
		(f_i|f_j)
		&=\int_{0}^{+\infty}x^{i+j}\e^{-x}\d x,\\
		&=\Gamma(i+j+1),\\
		&=(i+j)!.
	\end{align}
	On résout ensuite le système 
	\begin{equation}
		\left\lbrace
			\begin{array}[]{rcl}
				a+b+2c &=&\frac{1}{2},\\
				a+2b+6c &=& \frac{1}{2},\\
				2a+6b+24c &=& -\frac{1}{2},
			\end{array}
		\right.
	\end{equation}
	et on finit par calculer $\left\lVert\sin-p_F(\sin)\right\rVert^{2}=\left\lVert\sin\right\rVert^{2}-\left\lVert p_F(\sin)\right\rVert^{2}$.
\end{proof}

\end{document}
\documentclass[12pt]{article}
\usepackage{style/style_sol}

\begin{document}

\begin{titlepage}
	\centering
	\vspace*{\fill}
	\Huge \textit{\textbf{Solutions MP/MP$^*$\\ Espaces euclidiens}}
	\vspace*{\fill}
\end{titlepage}

\begin{proof}
	\phantom{}
	\begin{enumerate}
		\item Soit $Y\in\mathcal{M}_{n,1}(\R)$. $XX^{\mathsf{T}}Y=(X|Y)X$ est la projection orthogonale de $Y$ sur $\R X$. Donc $H_X$ est la matrice de la réflexion par rapport à $X^{\perp}$.
		\item C'est une conséquence du théorème de réduction.
	\end{enumerate}
\end{proof}

\begin{proof}
	\phantom{}
	\begin{enumerate}
		\item $A\in SO_{3}(\R)$ si et seulement si 
		\begin{equation}
			\label{eq:1}
			\begin{array}[]{rcl}
				a^{2}+b^{2}+c^{2} &=& 1,\\
				ab+ac+bc &=& 0,\\
				a^{3}+b^{3}+c^{3}-3abc &=& 1,
			\end{array}
		\end{equation}
		(vecteurs colonnes unitaires, vecteurs colonnes orthogonaux, déterminant égal à 1).
		$a,b,c$ racines de $X^{3}-X^{2}+p$ si et seulement si $X^{3}-X^{2}+p=(X-a)(X-b)(X-c)=X^{3}-X^{2}(a+b+c)+X(ab+bc+ac)-abc$ si et seulement 
		\begin{equation}
			\label{eq:2}
			\begin{array}[]{rcl}
				a+b+c &=& 1,\\
				ab+bc+cd &=& 0,\\
				-abc&\in&\left[0,\frac{4}{27}\right].
			\end{array}
		\end{equation}

		Ainsi, si on a~\eqref{eq:1}, on a $(a+b+c)^{2}=a^{2}+b^{2}+c^{2}+2(ab+ac+bc)=1$ donc $a+b+c=\pm1=\varepsilon\in\left\lbrace-1,1\right\rbrace$.
		De plus,
		\begin{align}
			(a+b+c)^{3}
			&= a^{3}+b^{3}+c^{3}+3(ab^{2}+ba^{2}+ac^{2}+ca^{2}+bc^{2}+cb^{2})+6abc,\\
			&=1+3abc+6abc+3a(1-a^{2})+3b(1-b^{2})+3c(1-c^{2}),\\
			&=1+3abc-3-9abc+3(a+b+c)+6abc,\\
			&=3(a+b+c)-2,
		\end{align}
		donc $\varepsilon^{2}=3\varepsilon-2$ donc $\varepsilon=1$ et $a+b+c=1$.

		On a $b+c=1-a,bc=-ab-ac=-a(b+c)=a(a-1)$, et $-abc=a^{2}(1-a)=\varphi(a)\geqslant0$, car $a^{2}+b^{2}+c^{2}=1$ donc $a\in[-1,1]$. On a $-abc=\varphi(a)=\varphi(b)=\varphi(c)$, et $a+b+c)=1$ donc un des trois au moins est positif. Comme $\varphi$ est compris entre $0$ et $\frac{4}{27}$ sur $[0,1]$, on a $-abc\in\left[0,\frac{4}{27}\right]$.

		Si on a~\eqref{eq:2}, on a $(a+b+c)^{2}=1=a^{2}+b^{2}+c^{2}=2(ab+bc+ac)=a^{2}+b^{2}+c^{2}$. On a $(a+b+c)^{3}=1=a^{3}+b^{3}+c^{3}-3(a^{3}+b^{3}+c^{3})+3(a+b+c)+6abc$ donc $a^{3}+b^{3}+c^{3}-3abc=1$.

		\item On a $A\begin{pmatrix}
			1\\1\\1
		\end{pmatrix}=(a+b+c)\begin{pmatrix}
			1\\1\\1
		\end{pmatrix}=\begin{pmatrix}
			1\\1\\1
		\end{pmatrix}$ donc l'axe de rotation est $\R\begin{pmatrix}
			1\\1\\1
		\end{pmatrix}$. On a $\Tr(A)=3a=1+2\cos(\theta)$, donc $\cos(\theta)=\frac{3a-1}{2}$, et $\sin(\theta)=(Af_1|f_2)=[f_3,f_1,Af_2]$ avec $f_3=\frac{1}{\sqrt{3}}\begin{pmatrix}
			1\\1\\1
		\end{pmatrix}$ et $f_1=\frac{1}{\sqrt{2}}\begin{pmatrix}
			1\\-1\\0
		\end{pmatrix}$ et $f_2=f_3\wedge f_1$. On laisse les calculs au lecteur.
	\end{enumerate}
\end{proof}

\begin{proof}
	\phantom{}
	\begin{enumerate}
		\item $A_n\in S_n(\R)$ donc est diagonalisable sur $\R$.
		\item Soit $X=\begin{pmatrix}
			x_1\\ \dots\\ x_n
		\end{pmatrix}$. On a 
		\begin{align}
			X^{\mathsf{T}}A_n X
			&=\sum_{(i,j)\in\left\llbracket1,n\right\rrbracket^{2}}\frac{x_i x_j}{\lambda_i+\lambda_j},\\
			&=\int_{0}^{1}\sum_{(i,j)\in\left\llbracket1,n\right\rrbracket^{2}}x_{i}t^{\lambda_{i}-\frac{1}{2}}x_jt^{\lambda_{j}-\frac{1}{2}}\d t,\\
			&=\int_{0}^{1}\left(\sum_{i=1}^{n}x_it^{\lambda_{i}-\frac{1}{2}}\right)^{2}\d t\geqslant0.
		\end{align}
		Si $X^{\mathsf{T}}A_nX=0$, alors pour tout $t\in]0,1]$, $\sum_{i=1}^{n}x_it^{\lambda_i-\frac{1}{2}}=0$ donc pour tout $y\in]-\infty,0]$, $\sum_{i=1}^{n}x_i \e^{\left(\lambda_{i}-\frac{1}{2}\right)y}=0$. Or $\left(y\mapsto\e^{\left(\lambda_{i}-\frac{1}{2}\right)y}\right)_{1\leqslant i\leqslant n}$ forme une famille libre comme vecteurs propres de la dérivation. Donc pour tout $i\in\left\llbracket1,n\right\rrbracket$, $x_i=0$ et $X=0$.

		\item On a $A_n\in S_n^{+}(\R)$ donc d'après l'inégalité d'Hadamard, on a 
		\begin{equation}
			0\leqslant\det(A_n)\leqslant\prod_{k=1}^{n}\frac{1}{2k-1}\xrightarrow[n\to+\infty]{}0,
		\end{equation}
		car si $u_n=\prod_{k=1}^{n}\frac{1}{2k-1}$, on a $\frac{u_{n+1}}{u_{n}}\xrightarrow[n\to+\infty]{}0$ donc $\sum u_n$ converge donc $u_n\xrightarrow[n\to+\infty]{}0$.
	\end{enumerate}
\end{proof}

\begin{remark}
	On rappelle que si $A$ est symétrique complexe, elle n'est pas nécessairement diagonalisable, par exemple $A=\begin{pmatrix}
		\i &1\\ 1 &-\i
	\end{pmatrix}$. On a $\chi_{A}=X^{2}$ et $A\neq 0$.
\end{remark}

\begin{proof}
	\phantom{}
	\begin{enumerate}
		\item On a 
		\begin{equation}
			E_{i,i}=\frac{1}{2}\left(I_n+\diag(-1,\dots,-1,1,-1\dots,-1)\right)\in\Vect(O_n(\R)),
		\end{equation}
		où le 1 est à l'indice $i$.
		De plus, si $i\neq j$, on a 
		\begin{equation}
			E_{i,j}=\frac{1}{2}(A+B),
		\end{equation}
		où 
		\begin{equation}
			A = \left(
				\begin{array}{*{11}c}
				1       & 0      & \dots & \dots & \dots & \dots & \dots & \dots & \dots & \dots & 0\\
				0       & \ddots &\\
				\vdots  &        & 1\\
				\vdots  &        &   & 0 & \dots & \dots & \dots &1\\
				\vdots  &        &   & \vdots  & 1 & & &\vdots\\
				\vdots  &        &   & \vdots  &   &\ddots & &\vdots\\
				\vdots  &        &   & \vdots  &   &      & 1 &\vdots\\
				\vdots  &        &   & 1 & \dots & \dots & \dots &0\\
				\vdots  &        &   &       &  & & & &1\\
				\vdots  &        &   &       &  &  & & & &\ddots \\
				0       & \dots  & \dots  & \dots  & \dots  & \dots &\dots &\dots &\dots & 0 & 1
				\end{array}
			\right),
		\end{equation}
		et 
		\begin{equation}
			B = 
			\left(
			\begin{array}{*{11}c}
				-1       & 0      & \dots & \dots & \dots & \dots & \dots & \dots & \dots & \dots & 0\\
				0       & \ddots &\\
				\vdots  &        & -1\\
				\vdots  &        &   & 0 & \dots & \dots & \dots &1\\
				\vdots  &        &   & \vdots  & -1 & & &\vdots\\
				\vdots  &        &   & \vdots  &   &\ddots & &\vdots\\
				\vdots  &        &   & \vdots  &   &      & -1 &\vdots\\
				\vdots  &        &   & -1 & \dots & \dots & \dots &0\\
				\vdots  &        &   &       &  & & & &-1\\
				\vdots  &        &   &       &  &  & & & &\ddots \\
				0       & \dots  & \dots  & \dots  & \dots  & \dots &\dots &\dots &\dots & 0 & -1
				\end{array}
			\right),
		\end{equation}
		avec les changements aux quadrants correspondants aux $j$-èmes et $i$-èmes lignes et colonnes. Comme $A$ et $B$ sont des matrices de permutation, on a $E_{i,j}\in\Vect(O_n(\R))$.

		\item $O_n(\R)$ est compact, et $U\mapsto \Tr(AU)$ est continue sur $O_n(\R)$, donc bornée et donc $N$ est bien définie. $N$ vérifie l'homogénéité et l'inégalité triangulaire. Vérifions la séparation : soit $A\in\mathcal{M}_{n}(\R)$ telle que $N(A)=0$. Pour tout $U\in O_n(\R)$, $\Tr(AU)=0$. Par combinaison linéaire, on a pour tout $(i,j)\in\left\llbracket1,n\right\rrbracket^{2}$, $\Tr(AE_{i,j})=a_{i,j}=0$ donc $A=0$. Donc $N$ est bien une norme.
		\item Soit \function{\iota}{O_n(\R)}{O_n(\R)}{U}{UV}
		$\iota$ est bijective car $O_n(\R)$ est un groupe. Donc 
		\begin{align}
			N(VA)
			&= \sup\limits_{U\in O_n(\R)}\left\lvert \Tr(AUV)\right\rvert,\\
			&= \sup\limits_{U\in O_n(\R)}\left\lvert \Tr(AU)\right\rvert,\\
			&= N(A).
		\end{align}

		\item Soient $\lambda_{1},\dots,\lambda_{n}$ valeurs propres (positives) de $S$. Soit $(\varepsilon_{1},\dots,\varepsilon_{n})$ une base orthonormée de $\R^{n}$ telle que pour tout $i\in\left\llbracket1,n\right\rrbracket$, $S\varepsilon_{i}=\lambda_{i}\varepsilon_{i}$. Soit $U\in O_n(\R)$, on a 
		\begin{align}
			\left\lvert \Tr(Su)\right\rvert
			&=\left\lvert \Tr(US)\right\rvert,\\
			&=\left\lvert\sum_{i=1}^{n}(US\varepsilon_{i}|\varepsilon_{i})\right\rvert,\\
			&\leqslant\sum_{i=1}^{n}\lambda_{i}\left\lVert U\varepsilon_{i}\right\rVert\left\lVert \varepsilon_{i}\right\rVert,\\
			&\leqslant\sum_{i=1}^{n}\lambda_{i},
		\end{align}
		et la borne supérieur est atteinte pour $U=I_n$. Donc $N(S)=\Tr(S)$.

		\item Soit $S=\sqrt{AA^{\mathsf{T}}}\in S_n^{+}(\R)$. D'après la décomposition polaire, il existe $O\in O_n(\R)$ telle que $A=SO$. Alors on a $N(A)=N(S)=\Tr(\sqrt{AA^{\mathsf{T}}})$.
	\end{enumerate}
\end{proof}

\begin{proof}
	$A$ et $B$ sont symétriques réelles donc diagonalisables. Si $Ax=\lambda X$ avec $X\neq0$, alors $X^{\mathsf{T}}AX=\lambda\left\lVert X\right\rVert^{2}\geqslant0$ donc $\lambda\geqslant0$ : les valeurs propres de $A$ et $B$ sont positives.

	Si $A\not\in GL_{n}(\R)$, $\det(A)=0$ et $\det(B)=\prod_{\mu\in\Sp(B)}\mu\geqslant0$. Si $A\in GL_{n}(\R)$, on a $A\in S_n^{++}(\R)$, d'où 
	\begin{equation}
		A^{-1}B=\sqrt{A^{-1}}\sqrt{A^{-1}}B\sqrt{A^{-1}}\sqrt{A}=\sqrt{A^{-1}}C\sqrt{A},
	\end{equation}
	car $\sqrt{A^{-1}}=\sqrt{A}^{-1}$ (preuve en diagonalisant). Soit $X$ un vecteur unitaire. On a 
	\begin{equation}
		X^{\mathsf{T}}CX=\underbrace{X^{\mathsf{T}}\sqrt{A^{-1}}}_{Y^{\mathsf{T}}}B\underbrace{\sqrt{A^{-1}}X}_{Y}\geqslant Y^{\mathsf{T}}AY=X^{\mathsf{T}}\sqrt{A^{-1}}A\sqrt{A^{-1}}X=X^{\mathsf{T}}X=1.
	\end{equation}
	Si $\lambda\in\Sp(B)$, soit $X$ unitaire tel que $CX=\lambda X$. Il vient $X^{\mathsf{T}}CX=\lambda\geqslant1$.
	Comme $C\in S_n(\R)$, on a $\det(C)=\prod_{\lambda\in\Sp(B)}\lambda\geqslant1$ donc $\det(B)\geqslant\det(A)$.
\end{proof}

\begin{remark}
	Si on a égalité, alors $\Sp(C)=\left\lbrace1\right\rbrace$, donc $C=I_n$ et $A=B$.
\end{remark}

\begin{proof}
	$SO(\R^{3})$ est un groupe donc $r'\in SO(\R^{3})$. Si $r$ est la rotation d'axe orienté par $f_{3}$ (unitaire) et d'angle $\theta$, alors $r'(s(f_{3}))=s(f_{3})$ donc $r'$ est une rotation d'axe orienté par $s(f_{3})$ d'angle $\theta'$. On a $\Tr(r')=\Tr(r)$ donc $\theta'=\pm\theta$. Soit $f_{1}\in f_{3}^{\perp}$ unitaire et $f_{2}=f_{3}\wedge f_{1}$. On a $\sin(\theta)=(r(f_{1})|f_2)=[f_3,f_1,r(f_1)]$. Comme $s$ est une isométrie, $s(f_1)$ est unitaire et orthogonal à $s(f_3)$ donc 
	\begin{equation}
		\sin(\theta')=[s(f_3),s(f_1),\underbrace{s(r(f_1))}_{r'(s(f_1))}]=\underbrace{\det(s)}_{1}\times\underbrace{[f_3,f_1,r(f_1)]}_{\sin(\theta)},
	\end{equation}
	donc $\theta=\theta'$.

	Supposons que $r$ et $s$ commutent alors $r'=r$, donc $s(f_{3})\in\Vect(f_{3})$ et $s$ est une isométrie donc $s(f_{3})\in\left\lbrace f_3,-f_3\right\rbrace$. Si $s(f_3)=f_3$, $r$ et $s$ ont même axe. Si $s(f_3)=-f_3$, $-1\in\Sp(s)$ et $s$ est un retournement et $r$ aussi (car $r$ et $s$ jouent des rôles symétriques), et l'axe de $r$ est perpendiculaire à celui de $s$.

	Réciproquement, si $r$ et $s$ sont de même axe, elles commutent. Si ce sont deux retournements par rapport à deux axes orthogonaux, dans une base orthonormée directe adaptée, elles ont pour matrice $\begin{pmatrix}
		1&0&0\\0&-1&0\\0&0&-1
	\end{pmatrix}$ et $\begin{pmatrix}
		-1&0&0\\0&1&0\\0&0&-1
	\end{pmatrix}$, et donc $r$ et $s$ commutent.
\end{proof}

\begin{proof}
	\phantom{}
	\begin{enumerate}
		\item D'après le théorème de réduction, il existe $P\in O_n(\R)$ tel que 
		\begin{equation}
			\underbrace{A}_{\in D}=P\diag(R_{\theta_1},\dots,R_{\theta_r},1\dots,1)\underbrace{P^{-1}}_{P^{\mathsf{T}}},
		\end{equation}
		car $-1\not\in\Sp_\R(A)$, où $R_{\theta}$ une matrice de rotation d'angle $\theta$. Donc $\det(A)=1$ et donc $A\in SO_n(\R)$ et $D\subset O_n(\R)$.

		\item Soit $M\in\mathcal{M}_n(\R)$, on a $\varphi(A)=M$ si et seulement si $M(I_n+A)=I_n-A$. Si c'est le cas, en transposant, on a
		\begin{align}
			\left(M(I_n+A)\right)^{\mathsf{T}}
			&=\left(I_n+A\right)^{\mathsf{T}}M^{\mathsf{T}},\\
			&=\left(I_n+A^{-1}\right)M^{\mathsf{T}},\\
			&=\left(I_n-A\right)^{\mathsf{T}},\\
			&=I_n-A^{-1},
		\end{align}
		et $A\in GL_n(\R)$, donc $\left(A+I_n\right)^{\mathsf{T}}M=A-I_n$ donc 
		\begin{equation}
			M^{\mathsf{T}}=(A+I_n)^{-1}(A-I_n)=(A-I_n)(A+I_n)^{-1},
		\end{equation}
		car si $BC=CB$ et $C$ inversible, alors $BC^{-1}=C^{-1}B$.

		Ainsi, $M^{\mathsf{T}}=-M$ donc $\varphi$ est bien définie de $D$ dans $D'$.

		Soit $M\in D'$, on a $M=\varphi(A)$ si et seulement si $M(I_n+A)=I_n-A$ si et seulement si $(M+I_n)A=I_n-M$.
		\begin{lemma}
			\label{lem:1}
			Si $\lambda\in\Sp_{\R}M$, alors $\lambda=0$.
		\end{lemma}
		\begin{proof}[Preuve du~\ref{lem:1}]
			Soit $X$ vecteur propre associé à $\lambda$. On a 
			\begin{equation}
				\underbrace{X^{\mathsf{T}}MX}_{\in\R}=\lambda\underbrace{\left\lVert X\right\rVert^{2}}_{>0}=\left(X^{\mathsf{T}}MX\right)^{\mathsf{T}}=X^{\mathsf{T}}M^{\mathsf{T}}X=-X^{\mathsf{T}}MX=-\lambda\underbrace{\left\lVert X\right\rVert^{2}}_{>0},
			\end{equation}
			donc $\lambda=0$.
		\end{proof}

		On en déduit que $M+I_n$ est inversible, et donc $A=(M+I_n)^{-1}(I_n-M)$. Il vient 
		\begin{align}
			A^{\mathsf{T}}
			&=(I_n+M)(I_n-M)^{-1},\\
			&=(I_n-M)^{-1}(I_n+M),\\
			&=A^{-1},
		\end{align}
		et donc $A$ est orthogonale.

		Si $I_n+A$ n'est pas inversible, il existe $X\neq0$ tel que $AX=-X$ et $0=M(I_n+A)X=(I_n-A)X$ donc $AX=X$ : impossible car $X\neq0$. Donc $I_n+A$ est inversible.
	\end{enumerate}
\end{proof}

\begin{proof}
	Soit $A$ inversible, $A\in S_n^{++}(\R)$ et $\sqrt{A^{-1}}=\sqrt{A}^{-1}$. Alors 
	\begin{equation}
		A^{-1}B=\sqrt{A^{-1}}\underbrace{\sqrt{A^{-1}}B\sqrt{A^{-1}}}_{C}\sqrt{A},
	\end{equation}
	donc $A^{-1}B$ est semblable à $C$.

	On a $X^{\mathsf{T}}CX=\underbrace{X^{\mathsf{T}}\sqrt{A^{-1}}}_{Y^{\mathsf{T}}}B\underbrace{\sqrt{A^{-1}}X}_{Y}\geqslant0$ donc $C\in S_n^{+}(\R)$.

	On a l'inégalité de l'énoncé si et seulement si $1+\sqrt[n]{\det(A^{-1}B)}\leqslant\sqrt[n]{\det(I_n+A^{-1}B)}$ si et seulement si $1+\sqrt[n]{\det(C)}\leqslant\sqrt[n]{\det(I_n+C)}$. Notons $(\lambda_{1},\dots,\lambda_{n})\in\Sp_\R(C)\subset\R$. L'inégalité équivaut à 
	\begin{equation}
		1+\left(\prod_{i=1}^{n}\lambda_{i}\right)^{\frac{1}{n}}\leqslant\left(\prod_{i=1}^{n}\left(1+\lambda_{i}\right)\right)^{\frac{1}{n}}.
	\end{equation}
	S'il existe $i\in\left\llbracket1,n\right\rrbracket$ tel que $\lambda_{i}=0$, l'inégalité est vraie. Si pour tout $i\in\left\llbracket1,n\right\rrbracket$, $\lambda_{i}>0$, alors l'inégalité équivaut à 
	\begin{equation}
		\underbrace{\ln\left(1+\exp\left(\frac{1}{n}\sum_{i=1}^{n}\ln(\lambda_{i})\right)\right)}_{\varphi\left(\frac{1}{n}\sum_{i=1}^{n}\ln(\lambda_{i})\right)}\leqslant\underbrace{\frac{1}{n}\sum_{i=1}^{n}\ln\left(1+\exp\left(\ln(\lambda_{i})\right)\right)}_{\frac{1}{n}\sum_{i=1}^{n}\varphi\left(\ln(\lambda_{i})\right)}.
	\end{equation}

	Comme $\varphi'(x)=\frac{\e^{x}}{1+\e^{x}}=1-\frac{1}{1+\e^{x}}$ et $\varphi''(x)=\frac{\e^{x}}{1+\e^{x}}>0$, $\varphi$ est strictement convexe d'où l'inégalité.

	De plus, si on a égalité, $\lambda_{1}=\dots=\lambda_{n}$, et $C$ étant diagonalisable, il existe $\lambda\geqslant0$ tel que $C=\lambda I_n$, d'où $B=\lambda A$.

	Si $A$ n'est pas inversible, soit pour $p\geqslant1$, $A_p=\frac{1}{p}I_n+A\in S_n^{++}(\R)$ car $\Sp(A_p)=\Sp(A)+\frac{1}{p}\subset\R_{+}^{*}$. Alors pour tout $p\in\N^{*}$, on a 
	\begin{equation}
		\sqrt[n]{\det(A_p)}+\sqrt[n]{\det(B)}\leqslant\sqrt[n]{\det(A_p+B)},
	\end{equation}
	et en passant à la limite $p\to+\infty$, on obtient l'inégalité.
\end{proof}

\begin{remark}
	On a $\sqrt[n]{\det\left(\frac{A+B}{2}\right)}=\frac{1}{2}\sqrt[n]{\det(A+B)}\geqslant\frac{1}{2}\left(\sqrt[n]{\det(A)}+\sqrt[n]{\det(B)}\right)$. On peut en déduire (par continuité et dichotomie) que $A\mapsto\sqrt[n]{\det(A)}$ de $S_n^{+}(\R)$ dans $\R$ est concave.
\end{remark}

\begin{proof}
	\phantom{}
	\begin{enumerate}
		\item On a $\left(AX\right)_{i}=\sum_{j=1}^{n}a_{i,j}x_j$ et 
		\begin{equation}
			X^{\mathsf{T}}AX=\sum_{i=1}^{n}\sum_{j=1}^{n}x_{i}x_{j}a_{i,j}=\sum_{i=1}^{n}x_{i}^{2}a_{i,i}+\sum_{i\neq j}x_{i}x_{j}a_{i,j}.
		\end{equation}
		Ainsi, comme $A\in S_n^{+}(\R)$, 
		\begin{equation}
			0\leqslant \left\lvert X\right\rvert^{\mathsf{T}}A\left\lvert X\right\rvert=\sum_{i=1}^{n}\left\lvert x_i\right\rvert^{2}a_{i,i}+\sum_{i\neq j}\left\lvert x_i\right\rvert\left\lvert x_j\right\rvert a_{i,j}.
		\end{equation}
		Or, pour $i\neq j$, $\left\lvert x_i\right\rvert\left\lvert x_j\right\rvert a_{i,j}\leqslant x_i x_j a_{i,j}$. Donc 
		\begin{equation}
			\left\lvert X\right\rvert^{\mathsf{T}}A\left\lvert X\right\rvert\leqslant X^{\mathsf{T}}AX.
		\end{equation}

		\item Si $AX=0$, d'après ce qui précède on a $\left\lvert X\right\rvert^{\mathsf{T}}A\left\lvert X\right\rvert=0$. Formons \function{\varphi}{\mathcal{M}_{n,1}(\R)^{2}}{\R}{(X,Y)}{Y^{\mathsf{T}}AX}
		$\varphi$ est une forme bilinéaire symétrique positive de forme quadratique associée $q$. D'après l'inégalité de Cauchy-Schwarz, on a 
		\begin{equation}
			\left\lvert\varphi(Y,\left\lvert X\right\rvert)\right\rvert\leqslant\sqrt{q(Y)}\underbrace{\sqrt{q(\left\lvert X\right\rvert)}}_{=0}=0.
		\end{equation}
		Donc $Y^{\mathsf{T}}A\left\lvert X\right\rvert=0$ pour tout $Y\in\R^{n}$. Donc $A\left\lvert X\right\rvert\in\left(\R^{n}\right)^{\perp}=\left\lbrace0\right\rbrace$ d'où $A\left\lvert X\right\rvert=0$.

		Pour tout $i\in\left\llbracket1,n\right\rrbracket$, $\sum_{j=1}^{n}a_{i,j}\left\lvert x_j\right\rvert=0$ donc $\sum_{j\neq i}a_{i,j}\left\lvert x_j\right\rvert+a_{i,i}\left\lvert x_i\right\rvert=0$. Si $\left\lvert x_i\right\rvert=0$, pour tout $j\neq i$, $\left\lvert x_j\right\rvert=0$ : impossible. Donc pour tout $i\in\left\llbracket1,n\right\rrbracket x_i\neq0$.

		\item Soit $X=\begin{pmatrix}
			x_1\\\vdots\\x_n
		\end{pmatrix}$ et $Y=\begin{pmatrix}
			y_1\\\vdots\\y_n
		\end{pmatrix}\in\left(\ker(A)\setminus\left\lbrace0\right\rbrace\right)^{2}$. Alors $Y-\frac{y_1}{x_1}X\in\ker(A)$ et sa première coordonnée est nulle donc $Y=\frac{y_1}{x_1}X$, donc $\dim(\ker(A))\leqslant A$ et $\rg(A)\geqslant n-1$.

		\item Soit $A'=A-\lambda I_n$. Soit $\lambda_{1}\in\Sp(A')$, on a $\Sp(A')=\Sp(A)-\lambda$. Or $\lambda=\min\Sp(A)$, donc pour tout $\lambda'\in\Sp(A')$, $\lambda'\geqslant0$ et donc $A'\in S_n^{+}(\R)$ et vérifie les hypothèses de $A$. On a $0<\dim(\ker(A'))\leqslant1$ et 0 est valeur propre donc $\dim(\ker(A-\lambda I_n))=1$ : $\lambda$ est une valeur propre simple.
	\end{enumerate}
\end{proof}

\begin{proof}
	Par récurrence sur $\dim(E)=n$ : c'est vrai si $\dim(E)=1$ car dans ce cas, $u=0$. Soit $n\geqslant1$, supposons le résultat vrai en dimension $n$ et soit $E$ de dimension $n+1$. Soit $(\varepsilon_{1},\dots,\varepsilon_{n+1})$ une base orthonormée de $E$. On a $\Tr(u)=\sum_{i=1}^{n}(u(\varepsilon_{i})|\varepsilon_{i})=0$. Soit \function{f}{S(0,1)}{\R}{x}{(u(x)|x)}. $f$ est continue et $S(0,1)$ est connexe par arc. Nécessairement, il existe $x\in S(0,1)$ tel que $(u(x)|x)=0$. On pose $e_1=x$, et dans une base orthonormée adaptée à $E=\R_{1}\overset{\perp}{\oplus}\left(\R e_{1}\right)^{\perp}$,
	\begin{equation}
		\mat_{B}(u)=\begin{pmatrix}
			0&\star\\\star &A
		\end{pmatrix}.
	\end{equation}
	$\Tr(A)=0$, et par hypothèse de récurrence, il existe $B_{1}$ une base orthonormée de $(\R e_1)^{\perp}$ (de dimension $n$) telle que $\mat(p\circ u, B_1)=\begin{pmatrix}
		0&\star\\
		\star&0
	\end{pmatrix}$ où $p$ est la projection orthogonale sur $(\R e_{1})^{\perp}$. D'où le résultat.
\end{proof}

\begin{proof}
	\phantom{}
	\begin{enumerate}
		\item Si $u$ est antisymétrique, avec $y=x$, on a $(u(x)|x)=0$. Réciproquement, si pour tout $x\in E$, $(u(x)|x)=0$, alors pour tout $(x,y)\in E^{2}$, $(u(x+y)|x+y)=0=(u(x)|x)+(u(y)|y)+(u(x)|y)+(u(y)|x)$, d'où $(u(x)|y)=-(x|u-y)$.
		\item Soit $B$ une base orthonormée et $A=\mat_{B}(u)$. $u$ est antisymétrique si et seulement si pour tout $(x,y)\in(\R^{n})^{2}$, $Y^{\mathsf{T}}AX=-X^{\mathsf{T}}AY=-X^{\mathsf{T}}A^{\mathsf{T}}X$. Donc, pour $X$ et $Y$ les vecteurs dans la base canonique, on a $A^{\mathsf{T}}=A$, et la réciproque est vraie.
		\item Soit $\lambda\in\Sp(u)$ et $x\neq0$ vecteur propre associé. On a $(u(x)|x)=0=\lambda\left\lVert x\right\rVert^{2}$. Comme $x\neq0$, on a $\lambda=0$, donc $\Sp(u)\subset\left\lbrace0\right\rbrace$.
		
		Si $\dim(E)$ est impair, $\chi_{u}$ est de degré impair, donc admet une racine réelle (par le théorème des valeurs intermédiaires), donc $0\in\Sp(u)$.

		\item Par récurrence sur $\dim(E)=n$. Si $n=1$, $\mat_B(u)=(0)$. Soit $n\in\N$, supposons le résultat vrai pour $\dim(E)\leqslant n$. Soit $E$ de dimension $n+1$ et $u\in\mathcal{L}(E)$ antisymétrique.
		\begin{lemma}
			\label{lem:2}
			Si $F$ est stable par $u$, $F^{\perp}$.
		\end{lemma}
		\begin{proof}[Preuve du~\ref{lem:2}]
			Soit $x\in F^{\perp}$ et $y\in F$. On a $(u(x)|y)=-(\underbrace{x}_{\in F^{\perp}}|\underbrace{u(y)}_{\in F})=0$.
		\end{proof}

		Rappelons par ailleurs qu'il existe $F$ stable par $u$ de dimension 1 ou 2, dans une base orthonormée $B_1$ de $F$ : $\mat_{B_1}(u_{\mid F})=(0)$ si $\dim(F)=1$, et $\mat_{B_2}(u_{\mid F})=\begin{pmatrix}
			0&a\\ -a&0
		\end{pmatrix}$ avec $a\in\R$ si $\dim(F)=2$. On applique l'hypothèse de récurrence à $F^{\perp}$.

		\item Soit $A\in\mathcal{A}_n(\R)$. On a 
		\begin{equation}
			\exp(A)^{\mathsf{T}}=\sum_{k=0}^{+\infty}\frac{\left(A^{k}\right)^{\mathsf{T}}}{k!}=\sum_{k=1}^{+\infty}\frac{(-A)^{k}}{k!}=\exp(-A)=\exp(A)^{-1},
		\end{equation}
		car $A\mapsto A^{\mathsf{T}}$ est linéaire et $\mathcal{A}_n(\R)$ de dimension finie donc continue.

		$\exp(A)\in O_n(\R)$, $\det(\exp(A))=\exp(\Tr(A))=\exp(0)=1$ en trigonalisant sur $\C$. Ainsi, $\exp(A)\in SO_n(\R)$.

		Soit $M\in SO_{n}(\R)$, il existe $P\in O_n(\R)$ et $\sigma_1,\dots,\sigma_k\in\R^{k}$, il existe $n_1\in\N$ tel que 
		\begin{equation}
			M=P\diag(R_{\theta_1},\dots,R_{\theta_k},-1,\dots,-1,1,\dots,1),
		\end{equation}
		où $-1$ apparaît $n_1$ fois, avec $n_1$ pair car $\det(M)=1$, donc 
		\begin{equation}
			M=P\diag(R_{\theta_1},\dots,R_{\theta_k},R_{\pi},\dots,R_{\pi},1,\dots,1),
		\end{equation}
		où l'on rappelle que mes $R_{\theta}$ représente une matrice de rotation d'angle $\theta$ en dimension 2. Soit $a\in\R$, on a 
		\begin{equation}
			\begin{pmatrix}
				0&-a\\a&0
			\end{pmatrix}=aR_{\frac{\pi}{2}}.
		\end{equation}
		Comme $R_{\frac{\pi}{2}}^{2}=-I_2, R_{\frac{pi}{2}}^{3}-R_{\frac{\pi}{2}}$ et $R_{\frac{\pi}{2}}^{4}=I_2$, on a pour tout $k\in\N$,
		\begin{equation}
			\begin{pmatrix}
				0&-a\\a&0
			\end{pmatrix}^{2k}=(-1)^{k}a^{2k}I_2,
		\end{equation}
		et
		\begin{equation}
			\begin{pmatrix}
				0&-a\\a&0
			\end{pmatrix}^{2k+1}=(-1)^{k}a^{2k+1}\begin{pmatrix}
				0&-1\\1&0
			\end{pmatrix}.
		\end{equation}

		Donc 
		\begin{align}
			\exp\begin{pmatrix}
				0&-a\\a&0
			\end{pmatrix}
			&=\sum_{k=0}^{+\infty}(-1)^{k}\left(\frac{a^{2k}}{(2k)!}I_2+\frac{a^{2k+1}}{(2k+1)!}\begin{pmatrix}
				0&-1\\1&0
			\end{pmatrix}\right),\\
			&=\cos(a)I_2+\sin(a)\begin{pmatrix}
				0&-1\\1&0
			\end{pmatrix}=R_a.
		\end{align}

		Ainsi, $M=P\exp(\underbrace{\diag(R_{\theta_1},\dots,\theta_k,R_\pi,\dots,R_\pi,0,\dots,0)}_{A'\in\mathcal{A}_n(\R)})P^{-1}=\exp(\underbrace{PA'P^{-1}}_{\in \mathcal{A}_n(\R)})$, donc $M\in\exp(\mathcal{A}_n(\R))$.
	\end{enumerate}
\end{proof}

\begin{proof}
	\phantom{}
	\begin{enumerate}
		\item Soit $A\in S_n(\R)$. Il existe $P\in O_n(\R)$ tel que 
		\begin{equation}
			A=P\diag(\lambda_{1},\dots,\lambda_{n})P^{-1},
		\end{equation}
		d'où 
		\begin{equation}
			\exp(A)=P\diag(\e^{\lambda_{1}},\dots,\e^{\lambda_{n}})P^{-1} \in S_n^{++}(\R).
		\end{equation}

		Soit $B\in S_n^{++}(\R)$, alors il existe $P\in O_n(\R)$ tel que 
		\begin{equation}
			B=P\diag(\mu_{1},\dots,\mu_{n})P^{-1},
		\end{equation}
		avec $\mu_{i}>0$ pour tout $i\in\left\llbracket 1,n\right\rrbracket$. Soit 
		\begin{equation}
			A=P\diag(\ln(\mu_{1}),\dots,\ln(\mu_{n}))P^{-1}.
		\end{equation}
		Alors $\exp(A)=B$.

		Soit $(A_{1},A_{2})\in S_n(\R)$ tel que $\exp(A_1)=\exp(A_2)=B$. Soient $(u_1,u_2)\in\mathcal{L}(\R^{n})^{2}$ correspondant à $A_1$ et $A_2$, et $v\in\mathcal{L}(\R^{n})$ correspondant à $B$. On vérifie que les sous-espaces propres de $u_1$ et $u_2$ sont ceux de $v$. Il s'ensuit que $u_1=u_2$. En effet, si $A\in\Sp(u_1)$, et si $u_1(x)=\lambda_{1}x$, alors $\exp(u_1)(x)=v(x)=\e^{\lambda}x$ donc $\ker(u_1-\lambda_i id)\subset\ker(v-\e^{\lambda}id)$. $u_1$ étant diagonalisable, si les valeurs propres distinctes sont $\lambda_{1},\dots,\lambda_{r}$, alors 
		\begin{equation}
			\R^{n}=\bigoplus_{i=1}^{r}\ker(u_i-\lambda_i id)\subset\bigoplus_{i=1}^{r}\ker(v-\e^{\lambda_i}id)\subset\R^{n}.
		\end{equation}
		D'où $\ker(u_i-\lambda_i id)=\ker(v-\e^{\lambda_i}id)$ pour tout $i\in\left\lbrace 1,\dots,r\right\rbrace$.

		\item On a $\exp(A)=\sum_{k=0}^{+\infty}\frac{A^{k}}{k!}$, c'est la somme d'une série de fonctions continues qui converge normalement sur les compacts.
		
		\item On munit $\mathcal{M}_n(\R)$ de $\vertiii{M}=\sup\limits_{\left\lVert X\right\rVert=1}\left\lVert MX\right\rVert$. Soit $X\in S(0,1)$, pour tout $k\in\N$, on a 
		\begin{align}
			\left\lvert X^{\mathsf{T}}M_kX-X^{\mathsf{T}}MX\right\rvert
			&=\left\lvert\left((M_k-M)(X)|X\right)\right\rvert,\\
			&\leqslant \left\lVert (M_k-M)(X)\right\rVert,\\
			&\leqslant\vertiii{M_k-M}.
		\end{align}
		Il existe $k_{0}\in\N$ tel que pour tout $k\geqslant k_0$, $\vertiii{M_k-M}\leqslant\min\left(\frac{\alpha}{2},1\right)$. On a pour tout $k\geqslant k_0$, $\alpha\leqslant X^{\mathsf{T}}MX\leqslant \beta$ d'où 
		\begin{equation}
			\alpha-\frac{\alpha}{2}=\frac{\alpha}{2}\leqslant X^{\mathsf{T}}M_k X\leqslant \beta+1.
		\end{equation}

		\item 
		\begin{lemma}
			\label{lem:3}
			Soit $A\in S_n(\R)$, on a $\vertiii{A}=\max\limits_{\lambda\in\Sp(A)}\left\lvert\lambda\right\rvert$, noté $\rho(A)$.	
		\end{lemma}
		\begin{proof}[Preuve du lemme~\ref{lem:3}]
			Soit $(\varepsilon_1,\dots,\varepsilon_n)$ une base orthonormale qui diagonalise $A$ avec $A\varepsilon_{i}=\lambda_{i}\varepsilon_{i}$. Soit $X=\sum_{i=1}^{n}x_i\varepsilon_i\in S(0,1)$, on a $AX=\sum_{i=1}^{n}\lambda_i x_i\varepsilon_i$. Alors 
			\begin{equation}
				\left\lVert AX\right\rVert^{2}=\sum_{i=1}^{n}\left(\lambda_i x_i\right)^{2}\leqslant\rho(A)^{2}\underbrace{\left\lVert X\right\rVert^{2}}_{=1},
			\end{equation}
			et $\left\lVert AX\right\rVert^{2}=\rho(A)$ pour $X$ vecteur propre associé à une des valeurs propres de valeur absolue maximale.
		\end{proof}

		D'après ce qui précède, pour tout $k\geqslant k_0$, $\Sp(\mu_k)\subset\left[\frac{\alpha}{2},\beta+1\right]$. Donc 
		\begin{equation}
			\Sp(\ln(M_k))\subset\left[\ln\left(\frac{\alpha}{2}\right),\ln(\beta+1)\right].
		\end{equation}
		Alors $\vertiii{\ln(M_k)}\leqslant\max\left(\left\lvert\ln\left(\frac{\alpha}{2}\right)\right\rvert,\left\lvert\ln\left(\beta+1\right)\right\rvert\right)$, pour tout $k\geqslant k_0$.

		\item $\left(\ln(M_k)\right)_{k\in\N}$ est bornée en dimension finie, donc admet une valeur d'adhérence $A$. Pr, pour tout $k\in\N$, $\ln(M_k)\in S_n(\R)$ fermé car sous-espace vectoriel de $\mathcal{M}_n(\R)$ en dimension finie. Donc $A\in S_n(\R)$. En notant l'extraction $\sigma$, on a $\ln\left(M_{\sigma(k)}\right)\xrightarrow[k\to+\infty]{}A$ donc $M_{\sigma(k)}\xrightarrow[k\to+\infty]{}\exp(A)=M$ par continuité de l'exponentielle.
		
		De plus, par injectivité, on a bien $A=\ln(M)$. La suite $\left(\ln(M_k)\right)_{k\in\N}$ admet une unique valeur d'adhérence $\ln(M)$, donc converge vers $\ln(M)$.
	\end{enumerate}
\end{proof}

\begin{remark}
	Généralement, soient $E$ et $F$ de dimension finie. Soit $A$ un fermé de $E$, et $f\colon A\subset E\to B\subset F$ bijective continue. On suppose que si $(\eta_k)_{k\in\N}\in B^{\N}$ est bornée, alors $(f^{-1}(\eta_k))_{k\in\N}\in A^{\N}$ est bornée. Alors $f^{-1}$ est continue.
\end{remark}

\begin{proof}
	$S_F(0,1)$ est compacte, et $X\mapsto(AX|X)$ est continue, donc admet un maximum sur $S_F(0,1)$ et $\Phi(F)$ est bien définie. 

	Soit $(\varepsilon_{1},\dots,\varepsilon_{n})$ une base orthonormée qui diagonalise $A$ : pour tout $i\in\left\llbracket 1,n\right\rrbracket$, $A\varepsilon_{i}=\lambda_i\varepsilon_{i}$.

	Soit $F$ un sous-espace vectoriel de $\R^{n}$ tel que $\dim(F)=k$. Soit $X)\sum_{i=1}^{n}X_i\varepsilon_{i}\in S_F(0,1)$. Alors $(AX|X)=\sum_{i=1}^{n}\lambda_i x_i^{2}$.

	Soit $E_k=\Vect(\varepsilon_k,\dots,\varepsilon_{n})$, $\dim(E_k)=n-k+1$. Nécessairement, $E_k\cap F\neq\left\lbrace0\right\rbrace$, car sinon $\dim(E_k+F)=n+1$. 

	Soit $x=\sum_{i=k}^{n}x_i\varepsilon_i\in E_k\cap F$ unitaire. Alors 
	\begin{equation}
		(AX|X)=\sum_{i=k}^{n}\lambda_i x_i^{2}\geqslant \lambda_k\sum_{i=k}^{n}x_i^{2}=\lambda_k.
	\end{equation}
	Donc $\Phi(F)\geqslant \lambda_k$.

	Soit $F_k=\Vect(\varepsilon_1,\dots,\varepsilon_k)$ de dimension $f$. Pour tout $x\in F_k$ unitaire, on a $(AX|X)=\sum_{i=}^{k}\lambda_i x_i^{2}\leqslant \lambda_k$.

	$\lambda_k$ est atteint pour $x=\varepsilon_k$, d'où $\lambda_k=\min\limits_{\substack{F\text{ sev de }\R^{n}\\ \dim(F)=k}}\Phi(F)$.
\end{proof}

\begin{proof}
	Comme les valeurs propres de $A$ sont $a_{1,1},\dots,a_{n,n}$, on a
	\begin{equation}
		\Tr(A^{2})=\sum_{i=1}^{n}a_{i,i}^{2}=\Tr(A^{\mathsf{T}}A)=\sum_{(i,j)\in\left\lbrace1,\dots,n\right\rbrace}a_{i,j}^{2},
	\end{equation}
	et donc pour tout $i\neq j$, $a_{i,j}=0$.
\end{proof}

\begin{proof}
	Soit $F'$ un sous-espace vectoriel de dimension $k$ de $\R^{n-1}$, on lui associe $F=F'\times\left\lbrace0\right\rbrace$ de dimension $k$ de $\R^{n}$.

	Soit $X'=\begin{pmatrix}
		x_1\\\vdots\\x_{n-1}
	\end{pmatrix}\in F'$ et $X=\begin{pmatrix}
		x_1\\\vdots\\x_{n-1}\\0
	\end{pmatrix}$. On a 
	\begin{equation}
		(A'X'|X')=\sum_{(i,j)\in\left\lbrace1,\dots,n-1\right\rbrace^{2}}a_{i,j}x_{i}x_{j}=(AX|X),
	\end{equation}
	et $\left\lVert X'\right\rVert=\left\lVert X\right\rVert$.

	Donc $\Phi'(F')=\Phi(F)\geqslant\lambda_{k}$. Ceci est valable pour tout sous-espace vectoriel $F'$ de $\R^{n-1}$ de dimension $k$, donc $\mu_k\geqslant\lambda_k$.

	Soit $G$ un sous-espace vectoriel de dimension $k+1$ de $\R^{n}$. 
	\begin{itemize}
		\item Si $G\subset\R^{n-1}\times\left\lbrace0\right\rbrace$, on a comme précédemment, en notant 
		\begin{equation}
			G'=\left\lbrace(x_1,\dots,x_{n-1})\in\R^{n-1}\middle|(x_1,\dots,x_{n-1},0)\in G\right\rbrace,	
		\end{equation}
		un sous-espace vectoriel de $\R^{n-1}$ de dimension $k+1$ et comme précédemment, on a $\Phi(G)=\Phi'(G')\geqslant\mu_{k+1}\geqslant\mu_k$.
		\item Si $G\not\subset\R^{n-1}\times\left\lbrace0\right\rbrace$, on forme 
		\begin{equation}
			G_1=G\bigcap\R^{n-1}\times\left\lbrace0\right\rbrace.
		\end{equation}
		On a $\dim(G)+\dim(\R^{n-1}\times\left\lbrace0\right\rbrace)-\dim(G_1)=\dim\left(G+\R^{n-1}\times\left\lbrace0\right\rbrace\right)=n$, donc $\dim(G_1)=k$.

		Comme $G_1\subset G$, on a $\Phi(G)\geqslant\Phi(G_1)\geqslant\mu_k$. Dan tous les cas, $\Phi(G)\geqslant\mu_k$ donc $\lambda_{k+1}\geqslant\mu_k$.
	\end{itemize}
\end{proof}

\begin{proof}
	\phantom{}
	\begin{enumerate}
		\item Supposons qu'il existe $(v,w)\in\left(\overline{B_{\vertiii{\cdot}}(0,1)}\right)^{2}$ tel que $u=\frac{v+w}{2}$. Pour tout $x\in\R^{n}$, on a $\left\lVert v(x)\right\rVert\leqslant\left\lVert x\right\rVert$ et $\left\lVert w(x)\right\rVert\leqslant\left\lVert x\right\rVert$, car $\vertiii{v}\leqslant1$ et $\vertiii{w}\leqslant1$. Donc 
		\begin{equation}
			\left\lVert u(x)\right\rVert=\left\lVert x\right\rVert\leqslant\left\lVert\frac{1}{2}\left(v(x)+w(x)\right)\right\rVert\leqslant\frac{\left\lVert v(x)\right\rVert+\left\lVert w(x)\right\rVert}{2}\leqslant\left\lVert x\right\rVert.
		\end{equation}

		On a donc $\left\lVert v(x)\right\rVert=\left\lVert x\right\rVert=\left\lVert w(x)\right\rVert$ et il existe $\lambda_x\geqslant0$ tel que $v(x)=\lambda_x w(x)$ (égalité dans Minkowski). Donc $\lambda_x=1$, et ceci étant pour tout $x\in\R^{n}$, on a $v=w=u$. Donc $u$ est extrémal.

		\item Soit $B=(e_1,\dots,e_n)$ une base orthonormée de $\R^{n}$ et $A=\mat_{B}(u)$. On pose $S=\sqrt{A^{\mathsf{T}}A}\in S_n^{+}(\R)$. On sait qu'il existe $\theta\in O_n(\R)$ tel que $A=\theta\times S$ (décomposition polaire). Pour tout $X\in S(0,1)$, comme $\vertiii{A}\leqslant1$, on a $\left\lVert AX\right\rVert\leqslant1$. Par ailleurs, pour tout $X\in S(0,1)$, $X^{\mathsf{T}}S^{2}X=(AX|AX)=\left\lVert AX\right\rVert^{2}\leqslant1$. Donc $\Sp(S^{2})\subset[0,1]$ et $\Sp(S)\subset[0,1]$ car $S\in S_n(\R)$. Si $\Sp(S)=\left\lbrace1\right\rbrace$, on a $S=I_n$, et $A=\theta\in O_n(\R)$ ce qui n'est pas. Donc il existe $\lambda\in\Sp(S)$ tel que $\lambda\in[0,1[$.
		
		Dans une base orthonormée $B'$ qui diagonalise $S$, on a $A'=\mat_{B'}(u)=\theta '\times\diag(\lambda_1,\dots,\lambda_n)$ avec $\lambda_i\in[0,1]$ et $\lambda_1<1$.

		Si $\lambda_{1}\neq0$, soit $\varepsilon=\min\left(\lambda_1,1-\lambda_1\right)$, on pose $S^{+}=\diag(\lambda_1-\varepsilon,\lambda_2,\dots,\lambda_n)$ et $S^{-}=\diag(\lambda_1+\varepsilon,\lambda_2,\dots,\lambda_n)$, $B=\theta'\times S^{+}$, $C=\theta'\times S^{-}$, $A'=\frac{B+C}{2}$ et $B\neq C\neq A$. Comme les valeurs propres de $S^{+}$ et $S^{-}$ sont dans $[0,1]$, on a $\vertiii{S^{+}}\leqslant 1$, $\vertiii{S^{-}}\leqslant1$. Et $\vertiii{\theta'}=1$, d'où $\vertiii{B}\leqslant1$ et $\vertiii{C}\leqslant1$. Donc $u$ n'est pas extrémal.

		Si $\lambda_1=0$, $S^{+}=\diag(-1,\lambda_2,\dots,\lambda_n)$ et $S^{-}=\diag(1,\lambda_2,\dots,\lambda_n)$, et parallèlement, $u$ n'est pas extrémal. Les points extrémaux sont les isométries.

		\item Soit $\left\lVert\cdot\right\rVert_2$ une norme euclidienne. Si $\left\lVert X\right\rVert_{2}<1$, alors $X$ n'est pas extrémal et il existe $(\lambda,\mu)\in[0,1[^{2}$ tel que $X=\frac{\lambda X+\mu X}{2}$.
		
		Soit $X$ tel que $\left\lVert X\right\rVert_{2}=1$, si $X=\frac{Y+Z}{2}$ avec $\left\lVert Y\right\rVert_{2}\leqslant1$ et $\left\lVert Z\right\rVert_{2}\leqslant1$. On a 
		\begin{equation}
			\left\lVert X\right\rVert_{2}=1=\left\lVert\frac{Y+Z}{2}\right\rVert_{2}\leqslant\frac{\left\lVert Y\right\rVert_{2}+\left\lVert Z\right\rVert_{2}}{2}\leqslant1.
		\end{equation}
		On a égalité partout, comme pour la première question, on a $Y=Z=X$. Les points extrémaux sont les points de la sphère unité. En prenant $A=\diag(1,0,\dots,0)$, on a $\vertiii{A}=1$ mais $A$ n'est pas une isométrie pour $n\geqslant2$. Donc la norme triple n'est pas une norme euclidienne.
	\end{enumerate}
\end{proof}

\begin{proof}
	Si $A=P\diag(\lambda_1,\dots,\lambda_n)P^{-1}$ alors $A^{3}=P\diag(\lambda_1^{3},\dots,\lambda_n^{3})P^{-1}$. $\sqrt[3]{}$ étant injectif, on a $\Sp_{\R}(A)=\Sp_{\R}(B)$. Soit $\lambda_1,\dots,\lambda_i$ les valeurs propres distinctes de $A$. Soient $u$ et $v$ canoniquement associés à $A$ et $B$. $u$ et $v$ sont diagonalisables. Soit $x\in\ker(u-\lambda_i id)$. On a $u(x)=\lambda_i x$, on a $u^{3}(x)=\lambda_{i}^{3}x$, donc $\ker(u-\lambda_i id)\subset\ker(u^{3}-\lambda_i^{3}id)$ car les $(\lambda_j^{3})_{1\leqslant j\leqslant i}$ sont distincts. On a 
	\begin{equation}
		\R^{n}=\bigoplus_{i=1}^{r}\ker(u-\lambda_i id)\subset\bigoplus_{i=1}^{r}\ker(u^{3}-\lambda_i^{3}id)\subset\R^{n},
	\end{equation}
	donc $\ker(u-\lambda_i id)=\ker(u^{3}-\lambda_i^{3}id)=\ker(v^{3}-\lambda_i^{3}id)=\ker(v-\lambda_i id)$. $u$ et $v$ ont les mêmes valeurs propres et même sous-espaces propres, donc sont égaux et $A=B$.
\end{proof}

\begin{proof}
	Soit \function{\varphi}{(\R^{n+1})^{2}}{\R}{(x=(x_0,\dots,x_n),y=(x_0,\dots,y_n))}{\sum_{(i,j)\in\left\llbracket0,n\right\rrbracket^{2}}\frac{x_iy_i}{i+j+1}}
	C'est une forme bilinéaire symétrique. $q$ dérive de $\varphi$. Soit $x\in\R^{n+1}\setminus\left\lbrace0\right\rbrace$, on a 
	\begin{align}
		q(x)
		&=\sum_{(i,j)\in\left\llbracket0,n\right\rrbracket^{2}}x_{i}x_{j}\int_{0}^{1}t^{i+j}\d t,\\
		&=\int_{0}^{1}\sum_{(i,j)\in\left\llbracket0,n\right\rrbracket^{2}}x_ix_jt^{i+j}\d t,\\
		&=\int_{0}^{1}\left(\sum_{i=0}^{n}x_it^{i}\right)^{2}\d t\geqslant0.
	\end{align}
	Si l'intégrale est nulle, alors $\sum_{i=0}^{n}x_it^{i}=0$ pour tout $t\in[0,1]$. C'est un polynôme en $t$ ayant une infinité de racines sur $[0,1]$, c'est donc le polynôme nul donc pour tout $i\in\left\llbracket0,n\right\rrbracket$, $x_i=0$ donc $x=0$.
\end{proof}

\begin{proof}
	Pour $n=1$, on considère $x_0=0$ et $\left\lVert x_1\right\rVert=1$. Si $c_1=\frac{x_1+x_2}{2}=\frac{x_2}{2}$. Alors 
	\begin{equation}
		\left\lVert c_1-x_1\right\rVert=\left\lVert c_1-x_2\right\rVert=\frac{1}{2}.
	\end{equation}

	Soit pour $n\geqslant1$, $H_n$ : \og Pour $E$ de dimension $n$, il existe $(x_1,\dots,x_{n+1})\in E^{n+1}$, pour tout $i\neq j$, $\left\lVert x_i-x_j\right\rVert=1$ et pour $c_n=\frac{(x_1+\dots+x_{n+1})}{n+1}$, il existe $r_n$ tel que pour tout $i\in\left\llbracket1,n+1\right\rrbracket$, $\left\lVert x_i-c_n\right\rVert=r_n$\fg.

	Supposons $H_n$ est vraie. Soit $E_n$ de dimension $n+1$ et soit $H$ un hyperplan de $E$. Il existe $(x_1,\dots,x_{n+1},c_n,r_n)$ vérifiant $H_n$. Soit $u$ un vecteur unitaire orthogonal à $H$. Soit $D$ la droite passant par $c_n$ et de vecteur directeur $u$ : $D=\left\lbrace c_n+tu\middle| t\in\R\right\rbrace$. Pour tout $i\in\left\llbracket1,n+1\right\rrbracket$, pour tout $t\in\R$, 
	\begin{equation}
		\left\lVert x_i-(x_n+tu)\right\rVert^{2}=\left\lVert x_i-c_n\right\rVert^{2}+t^{2}=r_n^{2}+t^{2}.
	\end{equation}
	Posons $x_{n+2}=c_n+\sqrt{1-r_n^{2}}u$. Pour tout $i\in\left\llbracket 1,n+1\right\rrbracket$, $\left\lVert x_{n+2}-x_i\right\rVert=1$. Soit 
	\begin{equation}
		c_{n+1}=\frac{x_1+\dots+x_{n+2}}{n+2}=\frac{n+1}{n+2}c_n+\frac{1}{n+2}x_{n+2}=c_n+\frac{\sqrt{1-r_n^{2}}}{n+2}u.
	\end{equation}

	Pour $i\in\left\llbracket1,n+1\right\rrbracket$, on a 
	\begin{align}
		\left\lVert c_{n+1}-x_{i}\right\rVert^{2}
		&=\frac{1-r_n^{2}}{(n+2)^{2}}+r_n^{2},\\
		&=\frac{1+((n+2)^{2}-1)r_n^{2}}{(n+2)^{2}},\\
		&=\frac{1+(n+1)(n+3)r_n^{2}}{(n+2)^{2}}.
	\end{align}

	On pose $r_{n+1}=\sqrt{\frac{1+(n+1)(n+3)r_n^{2}}{(n+2)^{2}}}$, puis 
	\begin{equation}
		\left\lVert c_{n+1}-x_{n+2}\right\rVert^{2}=\left(\frac{\sqrt{1-r_n^{2}}}{n+2}-\sqrt{1-r_n^{2}}\right)^{2}=\frac{1-r_n^{2}}{(n+2)^{2}}(n+1)^{2}.
	\end{equation}

	On a 
	\begin{align}
		r_n^{2}
		&=\left\lVert \frac{x_1+\dots+x_{n+1}}{n+1}-x_1\right\rVert^{2},\\
		&=\frac{1}{(n+1)^{2}}\left\lVert\sum_{i=2}^{n+1}(x_{i}-x_1)\right\rVert^{2},\\
		&=\frac{1}{(n+2)^{2}}\times\left(n+2\sum_{2\leqslant i<j\leqslant n+1}(x_i-x_1|x_j-x_i)\right).
	\end{align}
	On a, pour tout $i\neq j\neq 1$, 
	\begin{equation}
		\left\lVert x_i-x_j\right\rVert^{2}=1=\left\lVert (x_i-x_1)+(x_1-x_j)\right\rVert^{2}=2+2(x_i-x_1|x_1-x_j),
	\end{equation}
	d'où $(x_i-x_1|x_j-x_1)=\frac{1}{2}$, puis 
	\begin{equation}
		r_n^{2}=\frac{1}{(n+1)^{2}}\left(n+\frac{n(n-1)}{2}\right)=\frac{n}{2(n+1)},
	\end{equation}
	et
	\begin{align}
		r_{n+1}^{2}
		&=\frac{1+\frac{n(n+3)}{2}}{(n+2)^{2}},\\
		&=\frac{n(n+3)+2}{2(n+2)^{2}},\\
		&=\frac{n+1}{2(n+2)}\in[0,1[.
	\end{align}
	Et en reportant, 
	\begin{align}
		\left\lVert c_{n+1}-x_{n+2}\right\rVert^{2}
		&=\frac{(n+1)^{2}}{(n+2)^{2}}(1-r_n^{2}),\\
		&=\frac{(n+1)^{2}}{(n+2)^{2}}(1-\frac{n}{2(n+1)}),\\
		&=\frac{(n+2)(n+1)^{2}}{(n+2)^{2}2(n+1)},\\
		&=\frac{n+1}{2(n+2)},\\
		&=r_{n+1}.
	\end{align}
\end{proof}

\begin{remark}[Méthode directe]
	Soit $E$ euclidien de dimension $n+1$. Soit $(e_1,\dots,e_{n+1})$ une base orthonormée de $E$. On a $\frac{\left\lVert e_i-e_j\right\rVert}{2}=1$ pour $i\neq j$. Soit 
	\begin{equation}
		H=\left\lbrace x=x_1 e_1+\dots+x_{n+1}e_{n+1}\in E\middle| x_{1}+\dots+x_{n+1}=0\right\rbrace,	
	\end{equation} 
	hyperplan de $E$. On a $\dim(E)=n$, pour tout $i\in\left\lbrace1,\dots,n+1\right\rbrace$, soit $y_i=\frac{1}{\sqrt{2}}\left(e_i-c\right)\in H$ avec $c=\frac{e_1+\dots+e_{n+1}}{n+1}$ et pour tout $i\neq j$, $\left\lVert y_i-y_j\right\rVert=1$.
\end{remark}

\begin{proof}
	On définit
	\function{\varphi}{(\R^{n})^{2}}{\R}{((x_1,\dots,x_n),(y_1,\dots,y_n))}{\sum_{i=1}^{n}x_iy_i-\alpha\left(\sum_{i=1}^{n}\alpha_{i}\right)\left(\sum_{i=1}^{n}y_i\right)}
	Alors $\varphi(x,x)=q(x)$, et d'après l'inégalité de Cauchy-Schwarz pour le produit scalaire canonique de $\R^{n}$, pour tout $(x_1,\dots,x_n)\in\R^{n}$, on a 
	\begin{equation}
		\left(\sum_{i=1}^{n}x_{i}\right)^{2}\leqslant n\sum_{i=1}^{n}x_{i}^{2},
	\end{equation}
	d'où $\sum_{i=1}^{n}x_{i}^{2}\geqslant\frac{1}{n}\left(\sum_{i=1}^{n}x_{i}\right)^{2}$.

	\begin{itemize}
		\item Si $\alpha<\frac{1}{n}$, on a $q(x_1,\dots,x_n)\geqslant\left(1-n\alpha\right)\sum_{i=1}^{n}x_{i}^{2}\geqslant0$ et so $q(x_1,\dots,x_n)=0$, alors $\sum_{i=1}^{n}x_{i}^{2}=0$ donc les $x_i$ sont nuls.
		\item Si $\alpha\geqslant\frac{1}{n}$, on a $q(1,\dots,1)=n-\alpha n^{2}=n(1-\alpha n)\leqslant0$.
	\end{itemize}
	Finalement, $q$ est une forme quadratique définie positive si et seulement si $\alpha<\frac{1}{n}$.
\end{proof}

\begin{proof}
	\phantom{}
	\begin{enumerate}
		\item Si $\sum_{i=1}^{n}\lambda_i e_i=0$, on a 
		\begin{equation}
			\left\lVert\sum_{i=1}^{n}\lambda_i e_i\right\rVert^{2}=0=\sum_{i=1}^{n}\lambda_i^{2}\left\lVert e_i\right\rVert^{2}+\sum_{\substack{i\neq j\\i=1}}^{n}\lambda_i \lambda_j(e_i|e_j).
		\end{equation}
		On a alors 
		\begin{equation}
			0\leqslant\left\lVert\sum_{i=1}^{n}\left\lvert\lambda_i\right\rvert e_i\right\rVert^{2}=\sum_{i=1}^{n}\lambda_{i}^{2}\left\lVert e_i\right\rVert^{2}+\sum_{\substack{i\neq j\\ i=1}}^{n}\left\lvert \lambda_i\right\rvert\left\lvert \lambda_j\right\rvert(e_i|e_j)\leqslant0,
		\end{equation}
		car $\left\lvert\lambda_i\right\rvert\left\lvert\lambda_j\right\rvert\geqslant\lambda_i\lambda_j$ et $(e_i|e_j)\leqslant0$ donc $\left\lvert\lambda_i\right\rvert\left\lvert\lambda_j\right\rvert(e_i|e_j)\leqslant\lambda_i\lambda_j(e_i|e_j)$. Ainsi, $\sum_{i=1}^{n}\left\lvert\lambda_i\right\rvert e_i=0$.

		Notons que $\sum_{i\neq j}\left(\left\lvert\lambda_i\right\rvert\left\lvert\lambda_j\right\rvert-\lambda_i\lambda_j\right)(e_i|e_j)=0$ et chaque terme est négatif, donc pour tout $i\neq j$, $\left(\left\lvert\lambda_i\right\rvert\left\lvert\lambda_j\right\rvert-\lambda_i\lambda_j\right)(e_i|e_j)=0$. Si $(e_i|e_j)<0$, $\lambda_i$ et $\lambda_j$ sont donc de mêmes signes.

		\item On suppose que $\sum_{i=1}^{p}\lambda_i e_i=0$, alors $\sum_{i=1}^{p}\left\lvert \lambda_i\right\rvert e_i=0$ et $(\varepsilon|\sum_{i=1}^{n}\left\lvert \lambda_i\right\rvert e_i)=0=\sum_{i=1}^{n}\left\lvert\lambda_i\right\rvert(\varepsilon|e_i)$ et chaque terme de la somme est positif, donc pour tout $i\in\left\lbrace1,\dots,p\right\rbrace$, on a $\lambda_i=0$ et $(e_1,\dots,e_p)$ est libre.
		
		\item On a $(-x|e_i)<0$ donc $(-x,e_1,\dots,e_p)$ vérifie l'hypothèse. On a $1\times(-x)+\sum_{i=1}^{p}x_i e_i=0$ et d'après ce qui précède, $-x+\sum_{i=1}^{p}\left\lvert x_i\right\rvert e_i=0$ donc $x=\sum_{i=1}^{p}\left\lvert x_i\right\rvert e_i=\sum_{i=1}^{p}x_i e_i$ et par unicité, pour tout $i\in\left\lbrace1,\dots,\right\rbrace$, $\left\lvert x_i\right\rvert=x_i\geqslant0$.
		
		Soit $i_0\in\left\lbrace1,\dots,p\right\rbrace$ tel que $x_{i_0}=0$. On a 
		\begin{equation}
			(x|e_{i_0})=\sum_{\substack{i=1\\ i\neq i_0}}^{p}x_i(e_i|e_{i_0})>0,
		\end{equation}
		ce qui est absurde donc pour tout $i\in\left\lbrace1,\dots,p\right\rbrace$, $x_i>0$.
	\end{enumerate}
\end{proof}

\begin{proof}
	\begin{itemize}
		\item En dimension 1, soit $E=\Vect(u)$ avec $u$ unitaire. Soit $(x_1,\dots,x_p)\in E^{p}$, $\dim(E)=1$ donc pour tout $i\left\llbracket1,p\right\rrbracket$, $x_i=\lambda_i u$ avec $\lambda_i\in\R$, et pour $i\neq j\in\left\llbracket1,p\right\rrbracket^{2}$, $(x_i|x_j)=\lambda_i\lambda_j$, d'où $p\leqslant2$ si on veut $(x_i|x_j)<0$ pour tout $i\neq j\in\left\llbracket1,p\right\rrbracket^{2}$. Or $(u,-2)$ est obtusangle donc $r_1=2$.
		\item En dimension 2, on suppose $r_2=3$.
	\end{itemize}

	Par récurrence, supposons $r_n=n+1$ pour $n\in\N^{*}$. Soit $E$ un espace euclidien de dimension $n+1$ et soit $(x_1,\dots,x_p)\in E^{p}$ une famille obtusangle maximale (avec $p\geqslant2$). En particulier, $x_1\neq0$. Soit $H=x_1^{\perp}$ de dimension $n$ et pour tout $i\in\left\llbracket2,p\right\rrbracket$, $x_i'=p_H(x_i)$. Pour tout $i\in\left\llbracket2,p\right\rrbracket$, on a $x_i=x_i'+y_i$ avec $y_i=\lambda_i x_1$ avec $(x_i|x_1)=\lambda_i\left\lVert x_1\right\rVert^{2}<0$ donc $\lambda_i<0$, et pour tout $i\neq j\in\left\llbracket2,p\right\rrbracket^{2}$, $(x_i|x_j)=(x_i'|x_j')+\underbrace{\lambda_i\lambda_j\left\lVert x_1\right\rVert^{2}}_{>0}<0$, donc $(x_{i}'|x_{j}')<0$.

	Par hypothèse de récurrence, on a donc $p-1\leqslant n+1$ d'où $p\leqslant n+2$ d'où $r_{n+1}\leqslant n+2$.

	De plus soit $H$ un hyperplan (quelconque) de $E$. Par hypothèse de récurrence, il existe alors $(x_2',\dots,x_{n+2}')\in H^{n+1}$ obtusangle.
	Soit $x_1$ un vecteur orthogonal à $H$. Soit $\varepsilon>0$ et pour tout $i\in\left\llbracket2,n+2\right\rrbracket$, $x_{i}=x_{i}'-\varepsilon x_1$. On a $(x_i'|x_1)<0$ et $(x_i|x_j)=(x_i'|x_j')+\varepsilon^{2}$ pour tout $i\neq j\in\left\llbracket2,n+2\right\rrbracket$. Il suffit de prendre 
	\begin{equation}
		\varepsilon=\frac{1}{2}\min_{i\neq j\in\left\llbracket2,n+2\right\rrbracket^{2}}\left(\sqrt{-(x_i'|x_j')}\right)>0,
	\end{equation}
	donc on a bien $r_{n+1}=n+2$.
\end{proof}

\begin{proof}
	\phantom{}
	\begin{enumerate}
		\item En posant $u_0=0$ cela revient à trouver $(u_0,\dots,u_n)\in E^{n+1}$ tel que pour tout $i\neq j$, $\left\lVert u_i-u_j\right\rVert=1$. On sait que dans $\R^{n+1}$ euclidien, soit la base canonique de $(e_1,\dots,e_{n+1})$ on a pour tout $i\neq j$, $\left\lVert\frac{e_i}{\sqrt{2}}-\frac{e_j}{\sqrt{2}}\right\rVert=1$. Soit donc 
		\begin{equation}
			c=\frac{e_1+\dots+e_{n+1}}{(n+1)\sqrt{2}},
		\end{equation}
		et $H=\left\lbrace(x_1,\dots,x_{n+1})\in\R^{n+1}\middle|\sum_{i=1}^{n+1}x_i=0\right\rbrace$ hyperplan. Soit $v_i=\frac{e_i}{\sqrt{2}}-c\in H$ et pour tout $i\neq j$, $\left\lVert v_i-v_j\right\rVert=1$.

		On a ainsi $n+1$ vecteurs dans $H$ (de dimension $n$) tels que $\left\lVert v_i-v_j\right\rVert=1$. On pose pour tout $i\in\left\lbrace1,\dots,\right\rbrace$, $u_i=v_i-v_{n+1}$ unitaires et pour tout $i\neq j\in\left\lbrace1,\dots,n\right\rbrace$, $\left\lVert u_i-u_j\right\rVert=1$.

		\item Soit $(\lambda_1,\dots,\lambda_n)\in\R^{n}$ tel que $\sum_{i=1}^{n}\lambda_i u_i=0$. On a $\left\lVert u_i-u_j\right\rVert^{2}=1=\left\lVert u_i\right\rVert^{2}-2(u_i|u_j)+\left\lVert u_j\right\rVert^{2}$ donc $(u_i|u_j)=\frac{1}{2}$. Pour $j\in\left\llbracket1,n\right\rrbracket$, on a $\sum_{i=1}^{n}\lambda_i(u_i|u_j)=0$ donc $\lambda_j+\frac{1}{2}\sum_{\substack{i=1\\i\neq j}}\lambda_i=0$.
		Posons $S=\sum_{i=1}^{n}\lambda_i$. On a $\lambda_j=-S$, donc $\sum_{j=1}^{n}\lambda_j=-nS=S$ donc $S=0$ et pour tout $j\in\left\llbracket,1,n\right\rrbracket$, $\lambda_j=0$. Ainsi, $(u_1,\dots,u_n)$ est une base.

		\item A priori, on peut écrire 
		\begin{equation}
			u_j=\sum_{i=1}^{j-1}b_{i,j}e_i+a_{j}e_{j}=\sum_{i=1}^{j-1}(eu_j|e_i)e_i+a_{j}e_j.
		\end{equation}
		Soit $i\in\left\llbracket1,n-1\right\rrbracket$,, montrons que pour $j\neq k>i$, $(u_j|e_i)=(u_k|e_i)$ si et seulement si $(u_j-u_k|e_i)=0$. On a $e_i\in\Vect(u_1,\dots,u_i)$ (procédé de Gram-Schmidt) et pour tout $j\in\left\llbracket1,i\right\rrbracket$, 
		\begin{equation}
			(u_j-u_k|u_l)=(u_j|u_l)-(u_k|u_l)=\frac{1}{2}-\frac{1}{2}=0,
		\end{equation}
		car $j\neq l$, $k\neq l$ et $l\leqslant i<j,k$. Par combinaison linéaire, $(u_j-u_k|e_i)=0$, d'où le résultat.
	\end{enumerate}
\end{proof}

\begin{proof}
	S'il existe $u\in O(E)$ tel que pour tout $i\in\left\lbrace1,\dots,p\right\rbrace$, $y_i=u(x_i)$, alors on a directement 
	\begin{equation}
		(y_i|y_j)=(u(x_i)|u(x_j))=(x_i|x_j),
	\end{equation}
	pour tout $(i,j)\in\left\lbrace1,\dots,p\right\rbrace^{2}$.

	Réciproquement, si pour tout $(i,j)\in\left\llbracket1,p\right\rrbracket^{2}$, $(x_i|x_j)=(y_i|y_j)$, alors soient $F=\Vect(x_{i})_{1\leqslant i\leqslant p}$ et $(x_1,\dots,x_n)$ une base de $F$ (quitte à renuméroter).

	\begin{lemma}
		\label{lem:4}
		Soit $\mathrm{Gram}(z_1,\dots,z_p)=((z_i|z_j))_{1\leqslant i,j\leqslant p}$ de colonnes $C_1,\dots,C_n$. Soit $(\alpha_{1},\dots,\alpha_{p})\in\C^{p}$, alors on a $\alpha_{1}C_1+\dots+\alpha_{p}C_p=0$ si et seulement si $\alpha_{1}z_{1}+\dots+\alpha_{p}z_{p}=0$.
	\end{lemma}
	\begin{proof}[Preuve du lemme~\ref{lem:4}]
		On a $\alpha_{1}C_{1}+\dots+\alpha_{p}C_{p}=0$ si et seulement si $\sum_{j=1}^{p}\alpha_{j}z_{j}\in\left\lbrace z_1,\dots,z_p\right\rbrace^{\perp}$ si et seulement si $\sum_{j=1}^{p}\alpha_j z_j=0$; car $C_j=\begin{pmatrix}
			(z_1|z_j)\\\vdots\\(z_p|z_j)
		\end{pmatrix}$ pour tout $j\in\left\llbracket1,p\right\rrbracket$.
	\end{proof}

	D'après le lemme, on a $\mathrm{Gram}(y_1,\dots,y_r)=\mathrm{Gram}(x_1,\dots,x_r)\in GL_r(\R)$ donc $(y_1,\dots,y_r)$ est libre. D'autre part, pour tout $i\in\left\llbracket r+1,p\right\rrbracket$, il existe $(\alpha_{1,i},\dots,\alpha_{p,i})\in\R^{p}$,
	\begin{equation}
		x_i=\alpha_{1,i}x_1+\dots+\alpha_{r,i}x_r.
	\end{equation}
	D'après le lemme, on a $y_i=\alpha_{1,i}y_1+\dots+\alpha_{r,i}y_r$. Soit $(\varepsilon_{r+1},\dots,\varepsilon_p)$ une base orthonormée de $\Vect(x_{1},\dots,x_r)^{\perp}=F^{\perp}$ et $(f_{r+1},\dots,f_{n})$ une base orthonormée de $\Vect(y_1,\dots,y_r)^{\perp}$.

	Soit $u$ telle que pour tout $i\in\left\llbracket1,r\right\rrbracket$, $u(x_i)=y_i$, et pour tout $i\in\left\llbracket r+1,n\right\rrbracket$, $u(\varepsilon_i)=f_i$, $u\in\mathcal{L}(E)$.

	On a bien pour tout $i\in\left\llbracket r+1,p\right\rrbracket$, $u(x_i)=y_i$. Soit enfin $x\in E$, avec 
	\begin{equation}
		x=\underbrace{\alpha_1 x_1+\dots+\alpha_r x_r}_{\in F}+\underbrace{\alpha_{r+1}\varepsilon_{r+1}+\dots+\alpha_{n}\varepsilon_{n}}_{\in F^{\perp}}.	
	\end{equation}
	
	On a alors 
	\begin{equation}
		u(x)=\underbrace{\alpha_1 y_1+\dots+\alpha_{r}y_r}_{\in\Vect(y_i)_{1\leqslant i\leqslant r}}+\underbrace{\alpha_{r+1}f_{r+1}+\dots+\alpha_n f_n}_{\in \Vect(y_i)_{1\leqslant i\leqslant r}^{\perp}}.
	\end{equation}

	Enfin,
	\begin{align}
		\left\lVert u(x)\right\rVert^{2}
		&=\left\lVert \alpha_{1}y_{1}+\dots+\alpha_{r}y_{r}\right\rVert^{2}+\left\lVert \alpha_{r+1}f_{r+1}+\dots+\alpha_{n}f_{n}\right\rVert^{2},\\
		&=\sum_{(i,j)\in\left\llbracket1,r\right\rrbracket^{2}}\alpha_{i}\alpha_{j}\underbrace{(y_i|y_j)}_{(x_i|x_j)}+\sum_{i=r+1}^{n}\alpha_{i}^{2},\\
		&=\left\lVert \alpha_{1}x_{1}+\dots+\alpha_{r}x_{r}\right\rVert^{2}+\left\lVert \sum_{i=r+1}^{n}\alpha_{i}\varepsilon_{i}\right\rVert^{2},\\
		&=\left\lVert x\right\rVert^{2},
	\end{align}
	donc $u\in O(E)$.
\end{proof}

\begin{proof}
	\begin{lemma}
		\label{lem:5}
		S'il existe $M\geqslant0$ tel que pour tout $k\in\N$, $\vertiii{f^{k}}\leqslant M$, alors $E=\ker(f-id)\oplus\im(f-id)$ et $\left(\frac{id+f+\dots+f^{k}}{k+1}\right)_{k\in\N}$ converge vers $\pi$, projecteur sur $\ker(f-id)$ parallèlement à $\im(f-id)$.
	\end{lemma}
	\begin{proof}[Preuve du lemme~\ref{lem:5}]
		Soit $x\in E$, on a 
		\begin{equation}
			(id-f)\left(\frac{id+f+\dots+f^{k}}{k+1}\right)(x)=\frac{(id-f^{k+1})}{k+1}(x)\xrightarrow[k\to+\infty]{}0,
		\end{equation}
		car $\frac{\left\lVert f^{k+1}(x)\right\rVert}{k+1}\leqslant\frac{M\left\lVert x\right\rVert}{k+1}\xrightarrow[k\to+\infty]{}0$.

		Soit $y\in\ker(f-id)\cap\im(f-id)$, il existe $x\in E$ tel que $f(x)-x=y$ et $f(y)=y$, donc 
		\begin{equation}
			\frac{(id+f+\dots+f^{k})}{k+1}(y)=y=\left(\frac{id+f+\dots+f^{k}}{k+1}\right)(f-id)(x)\xrightarrow[k\to+\infty]{}0.
		\end{equation}
		Donc $y=0$. Comme on est en dimension finie, on a 
		\begin{equation}
			E=\ker(f-id)\oplus\im(f-id).
		\end{equation}

		Soit $x\in E$, il existe $(y,z)\in\ker(f-id)\times\im(f-id)$ tel que $x=z+y$. Il existe $x_1\in E$ tel que $z=f(x_1)-x_1$. Alors 
		\begin{equation}
			\frac{(id+f\dots+f^{k})}{k+1}(x)=y+\frac{(f^{k+1}-id)}{k+1}(x_1)\xrightarrow[k\to+\infty]{}y.
		\end{equation}
	\end{proof}
	
	Ici, on a pour tout $k\in\N$, pour tout $x\in E$, $\left\lVert f^{2}(x)\right\rVert=\left\lVert f\circ f(x)\right\rVert\leqslant\left\lVert f(x)\right\rVert\leqslant\left\lVert x\right\rVert$. Par récurrence, pour tout $k\in\N$, $\left\lVert f^{k}(x)\right\rVert\leqslant\left\lVert x\right\rVert$. On peut donc appliquer le lemme précédent. De plus, pour tout $k\in\N$, pour tout $x\in E$, $\left\lVert\frac{id+f+\dots+f^{k}}{k+1}(x)\right\rVert\leqslant\left\lVert x\right\rVert$.

	\begin{lemma}
		\label{lem:6}
		Si $E=F\oplus G$ et $F$ et $G$ ne sont pas orthogonaux. Soit $\Pi_{F\sslash G}$. Alors il existe $x\in E$ tel que $\left\lVert \Pi(x)\right\rVert\geqslant\left\lVert x\right\rVert$.
	\end{lemma}
	\begin{proof}[Preuve du leùùe~\ref{lem:6}]
		Soit $(y,z)\in F\times G$ tel que $(y|z)\neq0$. Supposons (quitte à remplacer $z$ par $-z$) $(y|z)<0$. Soit $t\in\R$, on a 
		\begin{equation}
			\left\lVert y+tz\right\rVert^{2}-\left\lVert y\right\rVert^{2}=2t(y|z)+t^{2}\left\lVert z\right\rVert^{2}\underset{t\to0^{+}}{\sim}2t(y|z)<0.
		\end{equation}
		Comme $\left\lVert y\right\rVert^{2}=\left\lVert\Pi(y+tz)\right\rVert^{2}$, il existe $t>0$ tel que $\left\lVert y-tz\right\rVert\leqslant\left\lVert\Pi(y+tz)\right\rVert$.
	\end{proof}

	D'après le lemme précédent, $\ker(f-id)$ et $\im(f-id)$ sont orthogonaux.
\end{proof}

\begin{proof}
	\phantom{}
	\begin{enumerate}
		\item Soit $y\in C$ et $K=\overline{B(x,\left\lVert y-x\right\rVert)}\cap C$. $K$ est un compact, car fermé borné en dimension fini, et non vide car $y\in K$. Soit $z\mapsto d(x,z)=\left\lVert x-z\right\rVert$ de $\K$ dans $\R$. Elle est continue (car 1-Lipschitzienne) sur un compact donc admet un minimum atteint en $z_0$. On a $\left\lVert x-z_0\right\rVert\leqslant\left\lVert x-y \right\rVert$. Si $z\in C\setminus K$, on a $\left\lVert z-x\right\rVert>\left\lVert x-y\right\rVert\geqslant\left\lVert z_0-x\right\rVert$.
		
		Pour l'unicité, soient $z_1,z_2\in C$ tels que $d(x,C)=\left\lVert x-z_1\right\rVert=\left\lVert x-z_2\right\rVert$. On a $\frac{z_1+z_2}{2}\in C$ par convexité, on a
		\begin{align}
			\left(x-\frac{z_1+z_2}{2}|z_1-z_2\right)
			&=\frac{1}{2}\left((x-z_1)+(x-z_2)|(x-z_2)-(x-z_1)\right),\\
			&=\frac{1}{2}\left\lvert\left\lVert x-z_1\right\rVert^{2}-\left\lVert x-z_2\right\rVert^{2}\right\rvert,\\
			&=0,
		\end{align}
		donc $z_1-z_2$ est orthogonal à $x-\frac{z_1+z_2}{2}$. D'après le théorème de Pythagore, on a 
		\begin{equation}
			\left\lVert x-z_1\right\rVert^{2}=\left\lVert x-\frac{z_1+z_2}{2}\right\rVert^{2}+\left\lVert \frac{z_1-z_2}{2}\right\rVert^{2}\geqslant\left\lVert x-z_1\right\rVert^{2}+\left\lVert \frac{z_1-z_2}{2}\right\rVert^{2}.
		\end{equation}
		Nécessairement, $z_1=z_2$

		\item Soit $y\in C$. Pour tout $t\in[0,1]$, on a 
		\begin{equation}
			\left\lVert tp_C(x)+(1-t)y-x\right\rVert^{2}=\left\lVert (1-t)(y-p_C(x))-(x-p_C(x))\right\rVert^{2}\geqslant\left\lVert x-p_C(x)\right\rVert^{2},
		\end{equation}
		et le terme de gauche vaut 
		\begin{equation}
			\left\lVert x-p_C(x)\right\rVert^{2}+\underbrace{(1-t)^{2}\left\lVert y-p_C(x)\right\rVert^{2}-2(1-t)\left(x-p_C(x)|y-p_C(x)\right)}_{\varphi(t)}.
		\end{equation}
		On a donc $\varphi(t)\geqslant0$ pour tout $t\in[0,1]$. Si $(x-p_C(x)|y-p_C(x))>0$, on aurait $\varphi(t)\underset{t\to1^{-}}{\sim}-2(1-t)(x-p_C(x)|y-p_C(x))<0$: impossible. Donc $(x-p_C(x)|y-p_C(x))\leqslant0$.

		Soit $z\in C$ tel que pour tout $y\in C$, $(x-z|y-z)\leqslant0$, alors pour tout $y\in C$, on a 
		\begin{equation}
			\left\lVert x-y\right\rVert^{2}=\left\lVert x-z\right\rVert^{2}+\left\lVert z-y\right\rVert^{2}+2(x-z|z-y)\geqslant\left\lVert x-z\right\rVert^{2},
		\end{equation}
		donc par unicité de $p_C(x)$, $z=p_C(x)$.

		\item Soit $x_1,x_2\in E$. Si $p_C(x_1)=p_C(x_2)$, on a $0=\left\lVert p_C(x_1)+p_C(x_2)\right\rVert\leqslant\left\lVert x_1-x_2\right\rVert$. Si non, soit $H=\Vect(p_C(x_2)-p_C(x_1))^{\perp}$, on a 
		\begin{equation}
			\begin{array}[]{rcl}
				x_1-p_C(x_1)&=&\lambda_1(p_C(x_1)-p_C(x_2))+y_1,\\
				x_2-p_C(x_2)&=&\lambda_2(p_C(x_1)-p_C(x_2))+y_2,\\
			\end{array}
		\end{equation}
		avec $y_1,y_2\in H$. Alors 
		\begin{equation}
			0\geqslant(x_1-p_C(x_1)|p_C(x_2)-p_C(x_1))=\lambda_1\left\lVert p_C(x_2)-p_C(x_1)\right\rVert^{2},
		\end{equation}
		donc $\lambda_1\leqslant0$ et de même, $\lambda_2\leqslant0$.
		Alors on a
		\begin{align}
			\left\lVert x_1-x_2\right\rvert^{2}
			&=\left\lVert x_1-p_C(x_1)+p_C(x_1)-p_C(x_2)+p_C(x_2)-x_2\right\rVert^{2},\\
			&=\left\lVert (1-\lambda_1-\lambda_2)(p_C(x_1)-p_C(x_2)+y_1-y_2)\right\rVert^{2},\\
			&=\underbrace{\left\lvert1-\lambda_1-\lambda_2\right\rvert^{2}}_{\geqslant1}\times\left\lVert p_C(x_1)-p_C(x_2)\right\rVert^{2}+\left\lVert y_1-y_2\right\rVert^{2},\\
			&\geqslant\left\lVert p_C(x_1)-p_C(x_2)\right\rVert^{2},
		\end{align}
		d'après le théorème de Pythagore. Donc $p_C\colon E\to C$ est $1$-Lipschitzienne.
	\end{enumerate}
\end{proof}

\begin{remark}
	Dans la question 2), si $x\not\in C$, on considère $H$ l'hyperplan passant par $p_C(x)$ et orthogonal à $x-p_C(x)$. $C$ est de l'autre côté de $H$ par rapport à $x$.
\end{remark}

\begin{proof}
	\phantom{}
	\begin{enumerate}
		\item $\varphi$ est linéaire par rapport à la seconde variable car $G\subset\mathcal{L}(\K^{n})$. De plus, on a 
		\begin{equation}
			\varphi(y,x)=\sum_{g\in G}(g(y)|g(x))=\sum_{g\in G}\overline{(g(x)|g(y)}=\overline{\varphi(x,y)},
		\end{equation}
		et $\varphi(x,x)=\sum_{g\in G}\left\lVert g(x)\right\rVert^{2}\geqslant0$. Enfin, si $\varphi(x,x)=0$ alors pour tout $g\in G$, $\left\lVert g(x)\right\rVert=0$. En particulier, pour $g=id$, on a $x=0$. Donc $\varphi$ est un produit scalaire.

		Soit $g_0\in G$. Comme $g\mapsto g\circ g_0$ est bijectif de réciproque $g\mapsto g\circ g_0^{-1}$, le résultat en découle.

		\item Soit $B$ une base de $\K^{n}$ orthonormée pour $\varphi$ (existe d'après le procédé de Gram-Schmidt). Soit $f\in G$ et $M=\mat_{B}(f)$ est orthogonale (si $\K=\R$) ou unitaire (si $\K=\C$). Donc $M^{\mathsf{T}}M=I_n$ (respectivement $\overline{M}^{\mathsf{T}}M=I_n$), d'où $M^{-1}=M^{\mathsf{T}}$ (respectivement $M^{-1}=\overline{M}^{\mathsf{T}}$), donc $\Tr(f^{-1})=\overline{\Tr(f)}$.
		
		\item Soit $B$ base de $\R^{2}$ orthonormée pour $\varphi$ associée à $G$, $P$ la matrice de passage de la base canonique de $\R^{2}$ à $B$. Pour tout $M\in G$, $P^{-1}MP\in SO_{2}(\R)$ et $G'=\left\lbrace P^{-1}MP\middle| M\in G\right\rbrace$ est un sous-groupe fini de $SO_2(\R)$. OR $(SO_2(\R),\times)$ est isomorphe à $(\U,\times)$ (via $R_\theta\mapsto\e^{\i\theta}$)/ Spot $M$ un sous-groupe de cardinal $n$ de $(\U,\times)$, d'après le théorème de Lagrange, pour tout $z\in H$, $z^{n}=1$ donc $H\subset\U_n$ et par isomorphisme, $G$ est cyclique.
	\end{enumerate}
\end{proof}

\begin{remark}
	On a aussi, pour tout $f\in G,\left\lvert\det(f)\right\rvert=1$ car $\overline{M}^{\mathsf{T}}M=I_n$.
\end{remark}

\begin{remark}
	Il existe des sous-groupes finis de $GL_2(\R)$ non commutatifs. Par exemple, le groupe des isométries du triangle (3 rotations, 3 symétries), isomorphe à $(\sigma_3,\circ)$ non-commutatif.
\end{remark}

\end{document}
\documentclass[12pt]{article}
\usepackage{style/style_sol}

\begin{document}

\begin{titlepage}
	\centering
	\vspace*{\fill}
	\Huge \textit{\textbf{Solutions MP/MP$^*$\\ Calcul différentiel}}
	\vspace*{\fill}
\end{titlepage}

\end{document}
\documentclass[12pt]{article}
\usepackage{style/style_sol}

\begin{document}

\begin{titlepage}
	\centering
	\vspace*{\fill}
	\Huge \textit{\textbf{Solutions MP/MP$^*$\\ Équations différentielles linéaires}}
	\vspace*{\fill}
\end{titlepage}

\begin{proof}

	L'équation différentielle est linéaire homogène sous forme résolue du second ordre. D'après le théorème de Cauchy-Lipschitz, l'ensemble solution est un $\R$-espace vectoriel de dimension 2.

	Soit $\varphi$ de classe $\mathcal{C}^{1}(\R,\R)$ et \function{u}{\mathcal{C}^{1}(\R,\R)}{\mathcal{C}^{0}(\R,\R)}{y}{y'+\varphi y}
	On définit ensuite 
	\function{u\circ u}{\mathcal{C}^{2}(\R,\R)}{\mathcal{C}^{0}(\R,\R)}{y}{(y'+\varphi y)'+\varphi(y' + \varphi y)=y''+ y'(2\varphi)+(\varphi' +\varphi^{2})y}
	On pose $\varphi(x)=x$. Alors l'équation différentielle équivaut à $u\circ u(y)=0$. On a $u(z)=0$ si et seulement si $z'+xz=0$ si et seulement s'il existe $c\in\R$ tel que pour tout $x\in\R$, $z(x)=C\e^{-\frac{x^{2}}{2}}$.

	On cherche la solution générale sous la forme $y(x)=d(x)\e^{-\frac{x^{2}}{2}}$. En reportant, cela équivaut à $d'(x)\e^{-\frac{x^{2}}{2}}=c\e^{-\frac{x^{2}}{2}}$, et cela équivaut au fait qu'il existe $d\in\R$ tel que $d(x)=cx+d$.
	Donc l'ensemble solution est 
	\begin{equation*}
		\left\lbrace x\mapsto(cx+d)\e^{-\frac{x^{2}}{2}}\middle| (c,d)\in\R^{2}\right\rbrace.
	\end{equation*}
\end{proof}

\begin{proof}
	C'est une équation homogène linéaire. Soit 
	\begin{equation*}
		A=
		\begin{pmatrix}
			1&-3&3\\
			-2&-6&13\\
			-1&-4&8
		\end{pmatrix}.
	\end{equation*}

	Le système équivaut à $tY'=aY$ où \function{Y}{I=\R_{+}^{*}\text{ ou }\R_{-}^{*}}{\R^{3}}{t}{
		\begin{pmatrix}
			x(t)\\y(t)\\z(t)
		\end{pmatrix}
	}

	Sur $I$, le système équivaut à $Y'=\frac{1}{t}AY$, équation homogène à valeurs dans $\R^{3}$. D'après le théorème de Cauchy-Lipschitz, l'ensemble solution est un $\R$-espace vectoriel de dimension 3. On a 
	\begin{align*}
		\chi_{A}
		&=
		\begin{vmatrix}
			X-1&3&-3\\
			2&X+6&-13\\
			&1&4&X-8
		\end{vmatrix},\\
		&=
		\begin{vmatrix}
			X-1 &3&0\\
			2&X+6&X-7\\
			1&4&X-4	
		\end{vmatrix},\\
		&=
		\begin{vmatrix}
			X-1 &-4X+7&0\\
			2&X-2&X-7\\
			1&0&X-4
		\end{vmatrix},\\
		&=(-4X+7)(X-7)+(X-4)\left((X-1)(X-2)-2(-4X+7)\right),\\
		&=X^{3}-3X^{2}+3X-1,\\
		&=(X-1)^{3}.
	\end{align*}
	$A$ est trigonalisable mais non diagonalisable car non semblable à $I_{3}$. On a 
	\begin{equation*}
		(A-I_3)\begin{pmatrix}
			x\\y\\z
		\end{pmatrix}=\begin{pmatrix}
			0\\0\\0
		\end{pmatrix}\Longleftrightarrow
		\left\lbrace
			\begin{array}[]{rcl}
				-3y+3z &=& 0,\\
				-2x-7y+13z &=& 0,\\
				-x-4y+7z &=&0. 
			\end{array}
		\right.
		\Longleftrightarrow
		\left\lbrace
			\begin{array}[]{rcl}
				y&=&z,\\
				x&=&3y
			\end{array}
		\right.
	\end{equation*}
	On prend pour vecteur propre $f_1=\begin{pmatrix}
		3\\1\\1
	\end{pmatrix}$. On a $(A-I_3)^{3}=0$ d'après le théorème de Cayley-Hamilton, et $\dim(\ker(A-I_3))=1$. On a 
	\begin{equation*}
		(A-I_3)^{2}=
		\begin{pmatrix}
			0&-3&3\\
			-2&-7&13\\
			-1&-4&7
		\end{pmatrix}
		\begin{pmatrix}
			0&-3&3\\
			-2&-7&13\\
			-1&-4&7
		\end{pmatrix}
		=
		\begin{pmatrix}
			3&9&-18\\
			1&3&-6\\
			1&3&-6
		\end{pmatrix}.
	\end{equation*}

	On choisit $f_3$ tel que $(A-I_3)^{2}f_3\neq\begin{pmatrix}
		0\\0\\0
	\end{pmatrix}$, par exemple $f_3=\begin{pmatrix}
		1\\0\\0
	\end{pmatrix}$. On pose $f_2=(A-I_3)f_3=\begin{pmatrix}
		0\\-2\\-1
	\end{pmatrix}$, et on a $f_1=(A-I_3)^{2}f_3
	=
	\begin{pmatrix}
		3\\1\\1
	\end{pmatrix}$.

	Soit 
	\begin{equation*}
		P=\begin{pmatrix}
			3&0&1\\
			1&-2&0\\
			1&-1&0
		\end{pmatrix}\in GL_{3}(\R).
	\end{equation*}
	Alors 
	\begin{equation*}
		A_{1}=P^{-1}AP=\begin{pmatrix}
			1&1&0\\
			0&1&1\\
			0&0&1
		\end{pmatrix}.
	\end{equation*}

	On pose $Y_1=P^{-1}Y=\begin{pmatrix}
		x_1\\y_1\\z_1
	\end{pmatrix}$. Alors le système équivaut à 
	\begin{equation*}
		\left\lbrace
			\begin{array}[]{rcl}
				tx_1' &=& x_1+y_1,\\
				ty_1' &=& y_1+z_1,\\
				tz_1' &=& z_1.
			\end{array}
		\right.
	\end{equation*}
	On trouve $z_1(t)=\alpha\e^{\ln\left\lvert t\right\rvert}=C t$ pour tout $t\in I$ (avec $C=\pm\alpha$). En reportant, on a $y_1'=\frac{1}{t}y_1+C$, donc si $y_1(t)=D(t)\times t$, on a $D'(t)\times t=C$ d'où $D(t)=C\ln\left\lvert t\right\rvert+D$. Enfin, on a $x_1'=\frac{1}{t}x_1+C\ln\left\lvert t\right\rvert+D$.

	Donc si $x_1(t)=E(t)\times t$, on a $E'(t)\times t=C\ln\left\lvert t\right\rvert+D$. Si $I=\R_{+}^{*}$, on a 
	\begin{equation*}
		E(t)=C\int_{1}^{t}\frac{\ln(u)}{u}\d u+D\ln(t)+E,
	\end{equation*}
	avec $\int_{1}^{t}\frac{\ln(u)}{u}\d u=\frac{1}{2}\ln^{2}(t)$. Ainsi, on a $E(t)=\frac{C}{2}\ln^{2}(t)+D\ln(t)+E$, d'où 
	\begin{equation*}
		x_1(t)=\frac{C}{2}t\ln^{2}\left\lvert t\right\rvert+Dt\ln\left\lvert t\right\rvert+E\times t.
	\end{equation*}

	Puis $Y=PY_{1}$, prolongeable (avec une classe $\mathcal{C}^{1}$) en 0 si et seulement si $C=D=0$ si et seulement si $Y_1(t)=\begin{pmatrix}
		tE\\0\\0
	\end{pmatrix}$.
\end{proof}

\begin{remark}
	Sur $I=\R_{+}^{*}$ ou $\R_{-}^{*}$, on a 
	\begin{align*}
		tY_1'=A_1Y_1
		&\Longleftrightarrow Y_1'-\frac{1}{t}A_1Y_1=0,\\
		&\Longleftrightarrow \exp(-\ln(t)A_1)(Y_1'-\frac{1}{t}A_1 Y_1)=(Y_1(t)\exp(-\ln(t)A_1))'=0,\\
		&\Longleftrightarrow \exists Y_0\in\R^{3}, \forall t\in I, \exp(-\ln\left\lvert t\right\rvert A_1)Y_1(t)=Y_0,\\
		&\Longleftrightarrow \exists Y_0\in\R^{3}, \forall t\in I, Y_1(t)=\exp(\ln\left\lvert t\right\rvert A_1)Y_0.
	\end{align*}
	On a $A_1=I_3+\underbrace{
		\begin{pmatrix}
			0&1&0\\0&0&1\\0&0&0
		\end{pmatrix}
	}_{N}$ avec $N^{2}=
	\begin{pmatrix}
		0&0&\\0&0&0\\0&0&0
	\end{pmatrix}$ et $N^{3}=0$. Ainsi, 
	\begin{equation*}
		\exp(\ln\left\lvert t\right\rvert A_1)=\underbrace{\e^{\ln\left\lvert t\right\rvert}}_{\pm t}\times\left(I_3+\ln\left\lvert t\right\rvert N+\frac{\ln^{2}\left\lvert t\right\rvert}{2}N^{2}\right).
	\end{equation*}
\end{remark}

\begin{proof}
	\phantom{}
	\begin{enumerate}
		\item On a $V(x)=\e^{xA}u$ sur $\R$. Pour tout $x\in\R$, on a $xA\in\mathcal{A}_n(\R)$ et 
		\begin{align*}
			\exp(xA)^{\mathsf{T}}
			&=\exp((xA)^{\mathsf{T}}),\\
			&=\exp(-xA),\\
			&=\exp(xA)^{-1},
		\end{align*}
		donc $\exp(xA)\in SO_n(\R)$ et $\left\lVert V(x)\right\rVert_{2}=\left\lVert u\right\rVert_{2}$.

		\item D'après le théorème de Cauchy-Lipschitz, pour tout $x_0\in\R$, \function{\Theta_{x_0}}{S_{\R}}{\R^{n}}{V}{V(x_0)} est un isomorphisme (où $S_{\R}$ est l'ensemble solution).
		
		Ainsi,
		\begin{itemize}
			\item ou bien $(V_1,\dots,V_n)$ est liée et pour tout $x\in\R$, $W(x)=0$,
			\item ou bien $(V_1,\dots,V_n)$ est libre et pour tout $x\in\R$, $B(x)=(V_1(x),\dots,V_n(x))$ est une base de $\R^{n}$ et $W(x)\neq0$. Alors 
			\begin{align*}
				W'(x)
				&=\sum_{i=1}^{n}\det_{B}(V_1(x),\dots,V_i'(x),\dots,V_n(x)),\\
				&=\sum_{i=1}^{n}\det_{B(x)}(V_1(x),\dots,AV_i(x),\dots,V_n(x))W(x),\\
				&=\sum_{i=1}^{n}a_{i,i}W(x),\\
				&=W(x)\times\Tr(A),\\
				&=0.
			\end{align*}
			Donc $W(x)=c$.
		\end{itemize}

		\item On suppose $u\neq0$. Comme pour tout $x\in\R$, $\exp(xA)\in O_n(\R)$. $(u,\exp(xA)u)$ est liée si et seulement s'il existe $\varepsilon(x)\in\left\lbrace-1,1\right\rbrace$ telle que $\varepsilon(x)\in\left\lbrace-1,1\right\rbrace$, $\exp(xA)u=\varepsilon(x)u$. On a $(\exp(xA)u|u)=\varepsilon(x)\left\lVert u\right\rVert_{2}^{2}$ donc $x\mapsto\varepsilon(x)$ est continue à valeurs dans $\left\lbrace-1,1\right\rbrace$ donc constante.
		
		\begin{lemma}
			\label{lem:1}
			On a $\Sp_{\R}A\subset\left\lbrace0\right\rbrace$, et il existe $P\in O_n(\R)$ et $(\alpha_1,\dots,\alpha_p)\in(\R^{*})^{p}$ tel que 
			\begin{equation*}
				P^{-1}AP=P^{\mathsf{T}}AP=\begin{pmatrix}
					0&-\alpha_1\\
					\alpha_1&0\\
					&&\ddots\\
					&&&0&-\alpha_p\\
					&&&\alpha_p&0\\
					&&&&&\ddots\\
					&&&&&&0
				\end{pmatrix}=A_1.
			\end{equation*}
		\end{lemma}
		\begin{proof}[Preuve du lemme~\ref{lem:1}]
			Si $Ax=\lambda X$, alors 
			\begin{equation*}
				(AX|X)=X^{\mathsf{T}}AX=\lambda\left\lVert X\right\rVert_{2}^{2}=(X^{\mathsf{T}}AX)^{\mathsf{T}}=X^{\mathsf{T}}(-A)X=-\lambda\left\lVert X\right\rVert_{2}^{2}.
			\end{equation*}
			Donc $\lambda=0$.

			Le deuxième résultat s'obtient par récurrence sur $n$.
		\end{proof}

		On a donc 
		\begin{equation*}
			\exp(xA)=P\exp(xA_1)P^{-1}=P\begin{pmatrix}
				R_{x\alpha_1}\\
				&\ddots\\
				&&R_{x\alpha_p}\\
				&&&1\\
				&&&&\ddots\\
				&&&&&1
			\end{pmatrix}P^{-1},
		\end{equation*}
		avec $\alpha_{i}\neq0$, où $R_{\theta}$ indique la matrice de rotation en dimension 2 d'angle $\theta$. Ainsi, pour que $\exp(xA)u=u$ pour tout $x\in\R$, il faut et il suffit que $u\in\ker(A)$ (pour ne pas être affecté par les matrices de rotation).
	\end{enumerate}
\end{proof}

\begin{remark}
	Si $(V_1(0),\dots,V_n(0))$ est une base orthonormée directe, pour tout $x\in\R$, pour tout $i\in\left\lbrace1,\dots,n\right\rbrace$, $\left\lVert V_i(x)\right\rVert_{2}=\left\lVert V_i(0)\right\rVert_{2}=1$ et en dérivant, on a $(V_i(x)|V_j(x))=\varphi_{i,j}(x)$.

	On a 
	\begin{align*}
		\varphi_{i,j}'(x)
		&=(V_i'(x)|V_j(x))+(V_i(x)|V_j'(x)),\\
		&=V_j(x)^{\mathsf{T}}AV_i(x)+V_j^{\mathsf{T}}\underbrace{A^{\mathsf{T}}}_{-A}V_i(x),\\
		&=0.
	\end{align*}
	Donc $\varphi_{i,j}=0$ donc $\varphi_{i,j}(x)=0$ pour tout $x\in\R$. Enfin, \begin{equation*}
		\det_{B}(V_1(x),\dots,V_n(x))=\det_{B}(V_1(0),\dots,V_n(0))=1.	
	\end{equation*}
	Donc pour tout $x\in\R$, $(V_1(x),\dots,V_n(x))$ est une base orthonormée directe.
\end{remark}

\begin{proof}
	On résout sur $I=\R_{+}^{*}$ ou $\R_{-}^{*}$. Posons 
	\begin{equation*}
		A=\begin{pmatrix}
			-4&-2\\
			6&3
		\end{pmatrix},
	\end{equation*}
	$Y\colon t\mapsto\begin{pmatrix}
		x(t)\\y(t)
	\end{pmatrix}$, $B\colon t\mapsto\frac{1}{\e^{t}-1}\begin{pmatrix}
		2&-3
	\end{pmatrix}$.

	$(x,y)$ est solution du système différentiel sur $I$ si et seulement si pour tout $t\in I$, $Y'(t)=AY(t)+B(t)$.

	On réduit $A$ : $\chi_{A}=X^{2}+X=X(X+1)$ est scindé à racines simples, donc $A$ est diagonalisable. On a 
	\begin{equation*}
		A\begin{pmatrix}
			a\\b
		\end{pmatrix}=0\Longleftrightarrow 
		\left\lbrace
			\begin{array}[]{rcl}
				-4a-2b&=&0,\\
				6a+3b&=&0,
			\end{array}
		\right.
	\end{equation*}
	si et seulement si $2a=b$. On pose $f_0=\begin{pmatrix}
		1\\-2
	\end{pmatrix}$, vecteur propre de $A$ associé à 0. On a 
	\begin{equation*}
		(A+I_2)\begin{pmatrix}
			a\\b
		\end{pmatrix}=0\Longleftrightarrow
		\left\lbrace
			\begin{array}[]{rcl}
				-4a-2b &=&0,\\
				6a+3b &=&0,
			\end{array}
		\right.
	\end{equation*}
	si et seulement si $3x=-2y$. On pose $f_{-1}=\begin{pmatrix}
		2\\-3
	\end{pmatrix}$.

	Soit $P=\begin{pmatrix}
		1&2\\
		-2&-3
	\end{pmatrix}$, on a $P^{-1}AP=A_1=\begin{pmatrix}
		0&0\\
		0&-1
	\end{pmatrix}$, et on pose $Y_1=P^{-1}Y=\begin{pmatrix}
		x_1\\y_1
	\end{pmatrix}$. De plus, on a 
	\begin{equation*}
		B(t)=\frac{1}{\e^{t}-1}\begin{pmatrix}
			2\\-3
		\end{pmatrix}=\frac{1}{\e^{t}-1}f_{-1},
	\end{equation*}
	donc $P^{-1}B(t)=\frac{1}{\e^{t}-1}\begin{pmatrix}
		0\\1
	\end{pmatrix}=B_1(t)$.

	Ainsi, le système différentiel équivaut sur $I$ à pour tout $t\in I$, $Y_1'(t)=A_1 Y_1(t)+B_1(t)$, d'où pour tout $t\in I$,
	\begin{equation*}
		\left\lbrace
			\begin{array}[]{rcl}
				x_1'(t)&=&0,\\
				y_1'(t)&=&-y_1(t)+\frac{1}{\e^{t}-1}.
			\end{array}
		\right.
	\end{equation*}
	Ainsi, il existe $\alpha\in\R$ tel que pour tout $t\in I$, $x_1(t)=\alpha$. D'autre part, on trouve $y_1(t)=\e^{t}\left(\ln(\left\lvert\e^{t}-1\right\rvert)+\gamma\right)$, avec $\gamma\in\R$.

	Pour déterminer $x$ et $y$, on calcule ensuite $Y=PY_{1}$.
\end{proof}

\begin{proof}
	\phantom{}
	\begin{enumerate}
		\item On résout sur $I=\R_{+}^{*}$ou $]-1,0[$. Sur $I$, l'équation différentielle équivaut à 
		\begin{equation*}
			f'(x)+\frac{\lambda}{x}f(x)=\frac{1}{x(x+1)},
		\end{equation*}
		d'équation homogène associée $y'=-\frac{\lambda}{x}y$. Les solutions de l'équation homogène sont $x\mapsto\beta\e^{-\lambda\ln\left\lvert x\right\rvert}=\frac{\beta}{\left\lvert x\right\rvert^{\lambda}}$ où $\beta\in\R$.
		Pour une solution générale de la forme $y(x)=\frac{\beta(x)}{\left\lvert x\right\rvert^{\lambda}}$ avec $x\mapsto\beta(x)$ de classe $\mathcal{C}^{1}$ sur $I$, on a $\frac{\beta'(x)}{\left\lvert x\right\rvert^{\lambda}}=\frac{1}{x(x+1)}=\frac{1}{x}-\frac{1}{x+1}$. Commencent les disjonctions de cas où l'on note $f(x)=\frac{\beta(x)}{\left\lvert x\right\rvert^{\lambda}}$ une solution.

		\begin{itemize}
			\item \underline{Si $I=\R_{+}^{*}$}, on a $\beta'(x)=x^{\lambda-1}-\frac{x^{\lambda}}{x+1}$.
			
			\begin{itemize}
				\item \underline{Si $\lambda\neq0$}, il existe $\beta\in\R$ tel que $\beta(x)=\frac{x^{\lambda}}{\lambda}-\int_{1}^{x}\frac{u^{\lambda}}{u+1}\d u+\beta$ et $f(x)=\frac{1}{\lambda}-\frac{1}{x^{\lambda}}\int_{1}^{x}\frac{u^{\lambda}}{1+u}\d u+\frac{\beta}{x^{\lambda}}$. $\lim\limits_{x\to0}\frac{\beta}{x^{\lambda}}$ est finie si et seulement si $\lambda>0$. Comme $\frac{u^{\lambda}}{u+1}\underset{u\to0}{\sim}u^{\lambda}$ donc $\int_{1}^{0}\frac{u^{\lambda}}{u+1}\d u$ converge si et seulement si $\lambda>-1$ (critère de Riemann).
				
				\begin{itemize}
					\item \underline{Si $\lambda\in]-1,0[$}, $\frac{1}{x^{\lambda}}\xrightarrow[x\to0]{}0$ et $\int_{1}^{x}\frac{u^{\lambda}}{1+u}\d u\xrightarrow[x\to0]{}\int_{1}^{0}\frac{u^{\lambda}}{1+u}\d u$ donc $f(x)\xrightarrow[x\to0]{}\frac{1}{\lambda}$ qui est une limite finie (sans condition sur $\beta$).
					
					\item \underline{Si $\lambda>0$}, notons que si $f$ a une limite finie en 0, il faut que 
					\begin{equation*}
						\frac{1}{x^{\lambda}}\left(\int_{1}^{x}\frac{u^{\lambda}}{1+u}\d u-\beta\right)\xrightarrow[x\to0]{}\text{quelque chose de fini.}
					\end{equation*}
					Or $\frac{1}{x^{\lambda}}\xrightarrow[x\to0]{}+\infty$, donc il faut 
					\begin{equation*}
						\left(\int_{1}^{x}\frac{u^{\lambda}}{1+u}\d u-\beta\right)\xrightarrow[x\to0]{}0,
					\end{equation*}
					d'où 
					\begin{equation*}
						\beta=-\int_{0}^{1}\frac{u^{\lambda}}{1+u}\d u.
					\end{equation*}

					Réciproquement, si $\beta=-\int_{0}^{1}\frac{u^{\lambda}}{1+u}\d u$, on a 
					\begin{align*}
						f(x)
						&=\frac{1}{\lambda}-\frac{1}{x^{\lambda}}\left(\int_{1}^{x}\frac{u^{\lambda}}{1+u}\d u+\int_{0}^{1}\frac{u^{\lambda}}{1+u}\d u\right),\\
						&=\frac{1}{\lambda}-\frac{1}{x^{\lambda}}\int_{0}^{x}\frac{u^{\lambda}}{1+u}\d u,\\
						&=\frac{1}{\lambda}-\int_{0}^{x}\frac{\left(\frac{u}{x}\right)^{\lambda}}{1+u}\d u,\\
						&=\frac{1}{\lambda}-\int_{0}^{1}\frac{v^{\lambda}}{1+vx}x\d v.
					\end{align*}

					Or 
					\begin{equation*}
						\int_{0}^{1}\frac{v^{\lambda}}{1+vx}x\d v=x\int_{0}^{1}\frac{v^{\lambda}}{1+vx}\d v,
					\end{equation*}
					et pour tout $(x,v)\in I\times[0,1]$, $\left\lvert\frac{v^{\lambda}}{1+vx}\right\rvert\leqslant v^{\lambda}$, intégrable sur $[0,1]$. D'aès le théorème de convergence dominée, on a donc 
					\begin{equation*}
						\int_{0}^{1}\frac{v^{\lambda}}{1+vx}\d v\xrightarrow[x\to0]{}\int_{0}^{1}v^{\lambda}\d v=\frac{1}{\lambda+1},
					\end{equation*}
					d'où $x\int_{0}^{1}\frac{v^{\lambda}}{1+v}\d v\xrightarrow[x\to0]{}0$ et $f(x)\xrightarrow[x\to0]{}\frac{1}{\lambda}$. 

					Donc $f$ a une limite finie en 0 si et seulement si $\beta=-\int_{0}^{1}\frac{u^{\lambda}}{1+u}\d u$.

					\item \underline{Si $\lambda<-1$}, on a $\frac{\beta}{x^{\lambda}}\xrightarrow[x\to0]{}0$ et $\frac{u^{\lambda}}{1+u}\underset{u\to0}{\sim}u^{\lambda}$. Par intégration des relations de comparaisons (applicable car les intégrandes sont positives), on a 
					\begin{equation*}
						\int_{x}^{1}\frac{u^{\lambda}}{1+u}\d u\underset{x\to0}{\sim}\int_{x}^{1}u^{\lambda}\d u=\frac{1}{\lambda+1}\left(1-x^{\lambda+1}\right)\underset{x\to0}{\sim}\frac{x^{\lambda+1}}{\lambda+1},
					\end{equation*}
					et
					\begin{equation*}
						-\frac{1}{x^{\lambda}}\int_{x}^{1}\frac{u^{\lambda}}{1+u}\d u\underset{x\to0}{\sim}\frac{1}{1+\lambda}\frac{x^{\lambda+1}}{x^{\lambda}}\xrightarrow[x\to0]{}0,
					\end{equation*}
					d'où $f(x)\xrightarrow[x\to0]{}\frac{1}{\lambda}$.

					\item \underline{Si $\lambda=-1$}, on a 
					\begin{align*}
						f(x)
						&=-1-x\int_{1}^{x}\frac{\d u}{1+u}+\beta x,\\
						&=-1-x\ln(x+1)+\ln(2)+\beta x,\\
						&\xrightarrow[x\to0]{}\ln(2)-1.
					\end{align*}
				\end{itemize}

				\item \underline{Si $\lambda=0$}, on a 
				\begin{equation*}
					\beta'(x)=\frac{1}{x}-\frac{1}{x+1}
				\end{equation*}
				et $\beta(x)=\ln\left(\frac{x}{1+x}\right)+\beta$. On a alors $f(x)=\frac{\beta(x)}{x^{0}}=\ln\left(\frac{x}{1+x}\right)+\beta\xrightarrow[x\to0]{}-\infty$, sans condition sur $\beta$.
			\end{itemize}

			\item \underline{Si $I=]-1,0[$}, on vérifie que c'est la même chose.
		\end{itemize}

		Si $f(x)=\sum_{n\in\N}a_n x^{n}$ est solution avec un rayon de convergence $R>0$, on a $xf'(x)=\sum_{n\in\N}na_nx^{n}$. Ainsi, pour tout $x\in]-R,R[$, on a
		\begin{equation*}
			xf'(x)+\lambda f(x)=\sum_{n\in\N}(n+\lambda)a_nx^{n}=\frac{1}{1+x}=\sum_{n\in\N}(-1)^{n}x^{n}.
		\end{equation*}
		Par unicité du développement en série entière, on a pour tout $n\in\N$,
		\begin{equation*}
			a_n=\frac{(-1)^{n}}{\lambda+n},
		\end{equation*}
		donc si $\lambda\not\in\Z_{-}$, on a une solution développable en série entière autour de 0.

		Réciproquement, avec cette définition des $(a_n)$ et de $f$, on a un rayon de convergence $R=1$ (par la règle de d'Alembert) et $f$ est solution de l'équation différentielle sur $]-1,1[$.

		\item On choisit $\lambda=\frac{1}{3}>0$. Les $(a_n)_{n\in\N}$ sont donc définis. Soit 
		\begin{equation*}
			S(x)=\sum_{n\in\N}a_nx^{n}=\sum_{n\in\N}\frac{(-1)^{n}x^{n}}{\frac{1}{3}+n}.
		\end{equation*}
		$S$ est solution de l'équation différentielle sur $]-1,1[$, et on connaît sa forme d'après l'étude menée à la première question. Comme $\lambda>0$, $S$ a une limite finie en 0 donc $S$ est entièrement déterminée (car on n'a pas le choix pour la constante $\beta$) :
		\begin{equation*}
			S(x)=3+\frac{1}{x^{\frac{1}{3}}}\int_{0}^{x}\frac{u^{\frac{1}{3}}}{1+u}\d u.
		\end{equation*}
		On pose $v=u^{\frac{1}{3}}$, d'où 
		\begin{equation*}
			\int_{0}^{x}\frac{u^{\frac{1}{3}}}{1+u}\d u=3\int_{0}^{x^{3}}\frac{3v^{3}\d v}{v^{3}+1}=9\left(\int_{0}^{x^{3}}\d v-\int_{0}^{x^{3}}\frac{\d v}{v^{3}+1}\right).
		\end{equation*}
		On décompose ensuite $\frac{1}{X^{3}+1}$ en éléments simples pour calculer l'intégrale.
	\end{enumerate}
\end{proof}

\begin{proof}
	\phantom{}
	\begin{enumerate}
		\item Pour le sens indirect, on a $\exp(tA)=\sum_{k\in\N}\frac{t^{k}A^{k}}{k!}$. Pour $i\neq j$, $(\exp(tA))_{i,j}$ est une série entière en $t$ et on a 
		\begin{equation*}
			(\exp(tA))_{i,j}=0+t a_{i,j}+t^{2}(A^{2})_{i,j}+\dots\underset{t\to0}{\sim}t a_{i,j}.
		\end{equation*}
		Par hypothèse, $(\exp(tA))_{i,j}\geqslant 0$ donc pour $t\to0^{+}$, on a $a_{i,j}\geqslant0$.
		
		Réciproquement, on considère $\beta=\max\limits_{1\leqslant i\leqslant n}(-a_{i,i})$. Posons $A'=A+\beta I_n\in\mathcal{M}_n(\R^{+})$. Pour tout $t\geqslant 0$, $tA'\in\mathcal{M}_n(\R^{+})$ donc $\exp(tA')\in\mathcal{M}_n(\R_{+})$. Comme $A$ et $I_n$ commutent, on a 
		\begin{align*}
			\exp(tA')
			&=\exp(tA+\beta tI_n),\\
			&=\exp(tA)\exp(t\beta I_n),\\
			&=\exp(tA)\times\e^{t\beta},
		\end{align*}
		donc $\exp(tA)=\underbrace{\e^{-t B}}_{\in\R_{+}}\exp(tA')\in\mathcal{M}_n(\R_{+})$.

		\item Le théorème de Cauchy-Lipschitz s'applique. Posons $\varphi\colon t\mapsto\exp(-t A)x(t)$, définie et dérivable sur $\R_{+}$. $x$ est solution du problème de Cauchy
		\begin{align*}
			&\Longleftrightarrow 
			\left\lbrace
				\begin{array}[]{rcll}
					x'(t)&=&A x(t)+f(t), &\forall t\in\R_{+},\\
					x(0)&=&0,
				\end{array}
			\right.,\\
			&\Longleftrightarrow 
			\left\lbrace
				\begin{array}[]{rcll}
					\exp(-tA)(x'(t)-Ax(t))&=&\exp(-tA)f(t), &\forall t\in\R_{+},\\
					\varphi(0)&=&0,
				\end{array}
			\right.,\\
			&\Longleftrightarrow 
			\varphi(t)=x_0+\int_{0}^{t}\exp(-u A)f(u)\d u,\forall t\in\R_{+},\\
			&\Longleftrightarrow 
			x(t)=\exp(tA)\left(x_0+\int_{0}^{t}\exp(-u A)f(u)\d u\right),\forall t\in\R_{+},\\
			&\Longleftrightarrow 
			x(t)=\exp(tA)+\exp(tA)\int_{0}^{t}\exp(-u A)f(u)\d u,\forall t\in\R_{+}.
		\end{align*}
		Or $\exp(t A)x_0\in(\R_{+})^{n}$ d'après la première question, et 
		\begin{equation*}
			\exp(tA)\int_{0}^{t}\exp(-u A)f(u)\d u=\int_{0}^{t}\exp((t-u)A)f(u)\d u.
		\end{equation*}
		Pour tout $u\in[0,t]$, $(t-u)>0$ donc $\exp((t-u)A)\in\mathcal{M}_n(\R_{+})$ et ainsi, $c(t)\in(\R_{+})^{n}$.
	\end{enumerate}
\end{proof}

\begin{proof}
	Le sens indirect est normalement du cours, il suffit de considérer l'isomorphisme \function{\Theta_{t_0}}{S_{(H),]a,b[}}{\R^{n}}{f}{(f(x),f'(x),\dots,f^{(n-1)}(x))}
	où $S_{(H),]a;b[}$ est l'ensemble des solutions de l'équation homogène sur $]a,b[$ avec une condition particulière en $t_{0}$.

	Réciproquement, si $W$ ne s'annule pas, notons $L_i(x)=\begin{pmatrix}
		f_{1}^{(i)(x)}\\ f_{2}^{(i}(x)\\\vdots\\f_n^{(i)}(x)
	\end{pmatrix}$ (ce sont les lignes de $W$ mises en colonne). On a 
	\begin{equation*}
		W(x)=\det(L_0(x),L_1(x),\dots, L_{n-1}(x)),
	\end{equation*}
	et comme $W$ ne s'annule pas, pour tout $x\in]a,b[$, $(L_0(x),\dots,L_{n-1}(x))$ est une base de $\R^{n}$. Ainsi, il existe $a_0(x),\dots,a_{n-1}(x))\in\R^{n}$ telle que 
	\begin{align*}
		\begin{pmatrix}
			f_{1}^{(n)}(x)\\\vdots\\ f_{n}^{(n)}(x)
		\end{pmatrix}
		&=\sum_{i=1}^{n}a_{i}(x) L_i(x),\\
		&=\begin{pmatrix}
			L_0(x),L_1(x),\dots, L_{n-1}(x)
		\end{pmatrix}
		\begin{pmatrix}
			a_0(x)\\\vdots\\a_{n-1}(x)
		\end{pmatrix},\\
		&=\underbrace{\begin{pmatrix}
			f_1(x) & f_1'(x)_{R(x)} &\dots & f_{1}^{(x-1)}(x)\\
			\vdots & \vdots & &\vdots\\
			f_{n-1}(x) & f_{n-1}'(x) & \dots & f_{n-1}^{(n-1)}(x)
		\end{pmatrix}}_{R(x)}\begin{pmatrix}
			a_0(x)\\\vdots\\a_{n-1}(x)
		\end{pmatrix}.
	\end{align*}

	Les $f_i$ étant $\mathcal{C}^{n}$, $x\mapsto R(x)$ est continue et $A\mapsto A^{-1}$ est $\mathcal{C}^{0}$ sur $\mathcal{M}_n(\R)$ donc $x\mapsto R(x)^{-1}$ est continue sur $]a,b[$ donc $x\mapsto R(x)^{-1}\begin{pmatrix}
		f_1^{(n)}(x)\\\vdots\\f_{n}^{(n)}(x)
	\end{pmatrix}=\begin{pmatrix}
		a_0(x)\\\vdots\\a_{n-1}(x)
	\end{pmatrix}$ est continue sur $]a,b[$. En d'autres termes, les $(a_i)_{i\in\left\llbracket0,n-1\right\rrbracket}$ sont continues sur $]a,b[$.
\end{proof}

\begin{proof}
	$\left\lvert\sin\right\rvert$ est continue sur $\R$, donc le théorème de Cauchy-Lipschitz sur $\R$. L'équation homogène a $(\cos,\sin)$ pour base de solutions. On cherche des solutions sous la forme $y(x)=a(x)\cos(x)+b(x)\sin(x)$, avec $a'(x)\cos(x)+b'(x)\sin(x)=0$.

	$y$ est solution sur $\R$ si et seulement si 
	\begin{equation*}
		\begin{array}[]{rcl}
			a'(x)\cos(x)+b'(x)\sin(x)&=&0,\\
			-a'(x)\sin(x)+b'(x)\cos(x)&=&\left\lvert\sin(x)\right\rvert.
		\end{array}
	\end{equation*}
	
	
	\begin{equation*}
		\begin{array}[]{l}
			\cos(x)\times\text{première ligne}-\sin(x)\times\text{deuxième ligne}\\
			\sin(x)\times\text{première ligne}+\cos(x)\times\text{deuxième ligne}
		\end{array}
	\end{equation*}
	donne 
	\begin{equation*}
		\begin{array}[]{rcl}
			a'(x) &=& -\sin(x)\left\lvert\sin(x)\right\rvert=\varepsilon_{x}\sin^{2}(x),\\
			b'(x) &=& \cos(x)\left\lvert\sin(x)\right\rvert=-\varepsilon_{x}\cos(x)\sin(x),
		\end{array}
	\end{equation*}
	avec $\varepsilon_{x}=1$ si $x\in[k\pi,(k+1)\pi]$ pour $k$ impair, et $\varepsilon_{x}$ si $k$ est pair.

	Sur $I_k=[k\pi,(k+1)\pi]$, on a $a(x)=\varepsilon_k\times\frac{1}{2}\left(x-\frac{\sin(2x)}{2}\right)+a_k$ et $b(x)=\varepsilon_{k}\times\frac{1}{2}\left(-\frac{\cos(2x)}{2}\right)+b_k$. On a 
	\begin{equation*}
		y(x)=\frac{\varepsilon_k}{2}\left(\left(x-\frac{\sin(2x)}{2}\right)\cos(x)-\frac{\cos(2x)}{2}\sin(x)\right)+a_k\cos(x)+b_k\sin(x).
	\end{equation*}

	Par continuité, $\lim\limits_{x\to k\pi^{-}}y(x)=\frac{\varepsilon_{k}}{2}\left(k\pi(-1)^{k}\right)+a_k(-1)^{k}$ et $\lim\limits_{x\to k\pi^{+}}y(x)=-\frac{\varepsilon_k}{2}(k\pi(-1)^{k})+a_{k+1}(-1)^{k}$ (on a $\varepsilon_{k+1}=-\varepsilon_k$). Donc $a_{k+1}=a_{k}+\varepsilon_{k}k\pi$. De même pour les $b_k$, on étudie la continuité de la dérivée.

	On détermine ainsi $a_k$ et $b_k$ en fonction de $a_0$ et $b_0$, par exemple pour tout $k\in\Z$, $a_k=a_0+\sum_{j=0}^{k-1}\varepsilon_j	(j\pi)$.
\end{proof}

\begin{remark}
	Autre méthode : $\left\lvert\sin\right\rvert$ est $\mathcal{C}^{1}$-PM continue $2\pi$-périodique paire. On admet que pour tout $x\in\R$, 
	\begin{equation*}
		\left\lvert\sin(x)\right\rvert=\sum_{n=0}^{+\infty}\alpha_n \cos(nx),
	\end{equation*}
	avec 
	\begin{equation*}
		\alpha_n=\frac{2}{\pi}\int_{0}^{\pi}\sin(t)\cos(nt)\d t.
	\end{equation*}
	On résout ensuite $y''+y=\cos(nx)$ pour tout $n\in\N$, et on somme en vérifiant que la solution obtenue est de classe $\mathcal{C}^{2}$.
\end{remark}

\begin{proof}
	On pose $\varphi(t)=X(t)^{\mathsf{T}}X(t)$. En dérivant, on a 
	\begin{equation*}
		\varphi'(t)=X(t)^{\mathsf{T}}X(t)+X(t)^{\mathsf{T}}X'(t)=-X^{\mathsf{T}}A(t)X(t)+X^{\mathsf{T}}A(t)X(t)=0.
	\end{equation*}
	Comme $\varphi(0)=I_n$, on a pour tout $t\in\R$, $\varphi(t)=I_n$ donc $X(t)\in O_n(\R)$.
\end{proof}

\begin{remark}
	Soit $Y\colon \R\to\mathcal{M}_{n,1}(\R)$ solution de $Y'(t)=A(t)Y(t)$ avec $Y(0)=Y_0$, de même $Y(t)^{ \mathsf{T}}Y(t)=\left\lVert Y(t)\right\rVert^{2}=\left\lVert Y_0\right\rVert^{2}$ donc $Y(t)$ est tracé sur une sphère.
\end{remark}

\begin{remark}
	Réciproquement, soit $X\colon\R\to O_n(\R)$ de classe $\mathcal{C}^{1}$. En dérivant 
	\begin{equation*}
		X(t)^{\mathsf{T}}X(t)=\mathrm{I_n},	
	\end{equation*}
	on a $X'(t)X(t)^{\mathsf{T}}+X(t)X'(t)^{\mathsf{T}}=0$ et $X(t)^{\mathsf{T}}=X(t)^{-1}$, donc 
	\begin{equation*}
		X'(t)X(t)^{-1}=-X(t)X'(t)^{\mathsf{T}}=-(X'(t)X(t)^{-1})^{\mathsf{T}},
	\end{equation*}
	donc $X'(t)=A(t)X(t)$ avec $A(t)$ antisymétrique.
\end{remark}

\begin{proof}
	Sur $I=\R_{+}^{*}$ ou $\R_{-}^{*}$, le théorème de Cauchy-Lipschitz s'applique. Si $y(x)=\sum_{n=0}^{+\infty}a_n x^{n}$ est solution sur $]-R,R[$ avec $R>0$, on a $y'(x)=\sum_{n=0}^{+\infty}(n+1)a_{n+1}x^{n}$ et $y''(x)=\sum_{n=1}^{+\infty}n(n+1)a_{n+1}x^{n}$. En reportant, et par unicité du développement en série entière, on a 
	\begin{equation*}
		2(n+1)n a_{n+1}+(n+1)a_{n+1}-a_n=0.
	\end{equation*}
	Donc pour tout $n\in\N$, $a_{n+1}=\frac{a_n}{(n+1)(2n+1)}$ donc 
	\begin{equation*}
		a_n=\frac{2^{n}}{(2n)!}a_0.
	\end{equation*}

	Réciproquement, définissons ainsi les $a_n$, avec par exemple $a_0=1$. On a $R=+\infty$ (règle de d'Alembert). En remontant les calculs, $y_1(x)=\sum_{n=0}^{+\infty}\frac{2^{n}x^{n}}{(2n)!}$ est solution sur $I$.

	Si $I=\R_{+}^{*}$, on a $y_1(x)=\cosh(\sqrt{2x})$. On vérifie alors que $y_2(x)=\sinh(\sqrt{2x})$ est solution.

	Si $I=\R_{-}^{*}$, on a $y_1(x)=\cos(\sqrt{-2x})$. On vérifie que $\sin(\sqrt{-2x})$ est solution.

	Les solutions maximales sont donc :
	\begin{itemize}
		\item sur $\R_{+}^{*}$, $\lambda\cosh(\sqrt{2x})+\mu\sinh(\sqrt{2x})$ avec $\mu\neq0$,
		\item sur $\R_{-}^{*}$, $\alpha\cos(\sqrt{-2x})+\beta\sin(\sqrt{-2x})$ avec $\beta\neq0$,
		\item sur $\R$, $\lambda\cosh(\sqrt{2x})$ sur $\R_{+}$ et $\lambda\cos(\sqrt{-2x})$ sur $\R_{-}$ d'où $\lambda\sum_{n=0}^{+\infty}\frac{x^{n}2^{n}}{(2n)!}$, de classe $\mathcal{C}^{\infty}$ car développable en série entière.
	\end{itemize}
\end{proof}

\begin{proof}
	Le théorème de Cauchy-Lipschitz s'applique sur $\R$. $(\sinh,\cosh)$ est nue base de l'ensemble solutions de l'équation homogène. Soit $\varphi(x)=\lambda(x)\cosh(x)+\mu(x)\sinh(x)$ avec la condition $\lambda'\cosh+\mu'\sinh=0$. $\varphi$ est solution si et seulement si 
	\begin{equation*}
		\left\lbrace
			\begin{array}[]{rcl}
				\lambda'(x)\cosh(x)+\mu'(x)\sinh(x) &=&0,\\
				\lambda'(x)\sinh(x)+\mu'(x)\cosh(x) &=&\frac{1}{\cosh(x)}.
			\end{array}
		\right.
	\end{equation*}

	$\cosh(x)\times\text{première ligne}-\sinh(x)\times\text{deuxième ligne}$ et $\sinh(x)\times\text{première ligne}-\cosh(x)\times\text{deuxième ligne}$ donne 
	\begin{equation*}
		\left\lbrace
			\begin{array}[]{rcl}
				\lambda'(x) &=& -\tanh(x),\\
				\mu'(x) &=& 1.
			\end{array}
		\right.
	\end{equation*}
	Donc $\lambda(x)=-\ln(\cosh(x))+\lambda$ et $\mu(x)=x+\mu$.

	On a 
	\begin{equation*}
		\begin{array}[]{rcl}
			\varphi(x) &=& \cosh(x)(\lambda-\ln(\cosh(x)))+\sinh(x)(x+\mu),\\
			\varphi'(x) &=& \sinh(x)\left(\lambda-\ln(\cosh(x))\right)+\cosh(x)(x+\mu).
		\end{array}
	\end{equation*}

	Et $\varphi(0)=0$ si et seulement si $\lambda=0$ et $\varphi'(0)=0$ si et seulement si $\mu=0$.
\end{proof}

\begin{proof}
	D'après la décomposition de Dunford, il existe $D$ diagonalisable et $N$ nilpotente qui commutent telles que $A=D+N$, avec $\chi_{D}=\chi_{A}$.
	Alors 
	\begin{equation*}
		\exp(tA)=\underbrace{\exp(tD)}_{P^{-1}\diag(\e^{t}\lambda_i)_{1\leqslant i\leqslant}P}\underbrace{\exp(tN)}_{\left(I_n+tN+\dots+\frac{t^{n-1}N^{n-1}}{(n-1)!}\right)}\xrightarrow[t\to+\infty]{}0.
	\end{equation*}
\end{proof}

\begin{proof}
	$(\sin,\cos)$ est une base de solution de l'équation homogène sur $\R$. Soit 
	\begin{equation*}
		\varphi(t)=\lambda(t)\sin(\omega t)+\mu(t)\cos(\omega t),
	\end{equation*}
	avec $\lambda'(t)\sin(\omega t)+\mu'(t)\cos(\omega t)=0$. $\varphi$ est solution si et seulement si $\varphi''+\omega^{2}\varphi=f$ et 
	\begin{equation*}
		\lambda'(t)\cos(\omega t)-\mu'(t)\sin(\omega t)=\frac{f(t)}{\omega}.
	\end{equation*}

	On fait $\sin(\omega t)$ fois la première ligne + $\cos(\omega t)$ fois la deuxième ligne donne 
	\begin{equation*}
		\lambda'(t)=\frac{f(t)}{\omega}\cos(\omega t).
	\end{equation*}
	$\cos(\omega t)$ fois la première ligne - $\sin(\omega t)$ fois la deuxième ligne donne 
	\begin{equation*}
		\mu'(t)=-\frac{f(t)}{\omega}\sin(\omega t).
	\end{equation*}
	Ainsi,
	\begin{equation*}
		\varphi(t)=\int_{0}^{t}\frac{f(u)}{\omega}\sin(\omega(t-u))\d u+\lambda\sin(\omega t)+\mu\cos(\omega t).
	\end{equation*}
	$\varphi$ est $T$-périodique si et seulement si pour tout $t\in\R$, $\varphi(t+T)=\varphi_1(t)=\varphi(t)$. Or $\varphi_1(t)$ est solution car $f$ est $T$-périodique. On a $\varphi_1=\varphi$ si et seulement si $\varphi_1(0)=\varphi(0)$ et $\varphi_1(T)=\varphi(T)$ d'après le théorème de Cauchy-Lipschitz, si et seulement si $\varphi(T)=\varphi(0)$ et $\varphi'(T)=\varphi'(0)$.

	Ainsi, on doit avoir 
	\begin{equation*}
		\int_{0}^{T}\frac{f(u)}{\omega}\sin(\omega(T-u))\d u+\lambda\sin(\omega T)+\mu\cos(\omega T)=\mu.
	\end{equation*}
	Comme $\varphi'(t)=\lambda(t)\omega\cos(\omega t)-\mu(t)\omega\sin(\omega t)$, donc 
	\begin{equation*}
		\varphi'(t)=\int_{0}^{t}f(u)\cos(\omega(t-u))\d u+\lambda\omega\cos(\omega T)-\mu\omega\sin(\omega t).
	\end{equation*}
	Donc on doit avoir 
	\begin{equation*}
		\int_{0}^{T}f(u)\cos(\omega(T-u))\d u+\lambda\omega\cos(\omega T)-\mu\omega\sin(\omega T)=\lambda\omega.
	\end{equation*}

	C'est un système de deux équations à deux inconnues et admet une unique solution $T$-périodique si et seulement si le déterminant 
	\begin{align*}
		\begin{vmatrix}
			\sin(\omega T)&\cos(\omega T)-1\\
			\omega(\cos(\omega T)-1)& -\omega\sin(\omega t)
		\end{vmatrix}
		&=\omega\left(-\sin^{2}(\omega T)-\left(\cos(\omega T)-1\right)^{2}\right),\\
		&=\omega\left(-2+2\cos(\omega T)\right),
	\end{align*}
	est non nul si et seulement si $\cos(\omega T)\neq1$.
\end{proof}

\begin{proof}
	Soit $I=\R_{+}^{*}$ ou $\R_{-}^{*}$. Le théorème de Cauchy-Lipschitz s'applique sur $I$ et la dimension de l'espace des solutions de l'équation homogène est 2. Notons que si une solution est polynomiale de degré $n$, alors le coefficient en $x^{n+1}$ de $x^{2}y''(x)-2x(1+x)y'(x)+2(1+x)y(x)$ est $0=-2na_n+2a_n$. Nécessairement $n=1$ et $y_1$ est affine. On vérifie que $y_1(x)=x$ est solution. On cherche ensuite une solution de la forme $y_2(x)=C(x)y_1(x)=C(x)x$ avec $C$ non constante. En reportant, on trouve 
	\begin{equation*}
		C''(x)+\left(2\left(1+\frac{2}{x}\right)\right)C'(x)=0.
	\end{equation*}

	On trouve par exemple $C(x)=\int_{\varepsilon}^{x}\frac{\e^{-2 u}}{u^{4}}\d u$. On choisit $\varepsilon=1$ si $I=\R_{+}^{*}$ et $\varepsilon=-1$ si $I=\R_{-}^{*}$.

	\begin{equation*}
		\int_{\varepsilon}^{x}\frac{\e^{-2u}}{u^{4}}\d u\underset{x\to0}{\sim}\int_{\varepsilon}^{x}\frac{\d u}{u^{4}}\underset{x\to0}{\sim}\frac{-1}{3x^{3}}.
	\end{equation*}
	Donc $y_2$ n'a pas de limite en 0. $\lambda y_1$ sont les seules solutions maximales sur $\R$.
\end{proof}

\begin{proof}
	On pose $g(t)=f'(t)+f(t)$. L'équation homogène a pour solution $y(t)=\lambda\exp(-t)$ d'où $f(t)=\lambda(t)\exp(-t)$ avec 
	\begin{equation*}
		(f'+f)(t)=g(t)=\lambda'(t)\exp(-t).
	\end{equation*}
	On a $\lambda(t)=\int_{0}^{t}\exp(t) g(u)\d u+\lambda$. Si $F(t)=\int_{0}^{t}g(u)\exp(u)\d u$, soit $\varepsilon>0$. Il existe $A>0$ tel que pour tout $t>A$, $\left\lvert g(t)\right\rvert\leqslant\varepsilon$. Alors 
	\begin{equation*}
		F(t)=\underbrace{\e^{-t}\int_{0}^{A}g(u)\e^{u}\d u}_{\xrightarrow[t\to+\infty]{}0}+\int_{A}^{t}g(u)\e^{u-t}\d u,
	\end{equation*}
	et le second terme est majoré en valeur absolue par $\frac{\varepsilon}{2}\int_{A-t}^{0}\e^{u}\d u=\frac{\varepsilon}{2}\left(1-\e^{A-t}\right)\leqslant\frac{\varepsilon}{2}$. D'où le résultat.

	Contre exemple pour la deuxième question : $\e^{t}$.
\end{proof}

\begin{proof}
	Soit \function{\varphi}{\mathcal{M}_n(\K)}{\mathcal{M}_n(\K)}{M}{MB-BM}
	On a $A'(t)=\varphi(A(t))$, c'est une équation différentielle homogène linéaire. $\varphi$ est à coefficients constants, on sait alors que 
	\begin{equation*}
		A(t)=\exp(t\varphi)(A(0)).
	\end{equation*}
	On a $\exp(t\varphi)=\sum_{k=0}^{+\infty}\frac{t^{k}}{k!}\varphi^{k}$. Soit \function{\varphi_1}{\mathcal{M}_n(\K)}{\mathcal{M}_n(\K)}{M}{MB} et \function{\varphi_2}{\mathcal{M}_n(\K)}{\mathcal{M}_n(\K)}{M}{-BM}
	On a $\varphi=\varphi_1+\varphi_2$, et 
	\begin{equation*}
		\left(\varphi_1\circ\varphi_2\right)(M)=-BMB=(\varphi_2\circ\varphi_1)(M).
	\end{equation*}
	Ainsi, $\exp(t\varphi)=\exp(\varphi_1)\exp(t\varphi_2)$. On a \function{\varphi_1^{k}}{\mathcal{M}_n(\K)}{\mathcal{M}_n(\K)}{M}{MB^{k}} et \function{\varphi_1^{k}}{\mathcal{M}_n(\K)}{\mathcal{M}_n(\K)}{M}{(-1)^{k}B^{k}M} 

	On Si $A(0)=A_0$, on a
	\begin{equation*}
		\exp(t\varphi_1)\left(\exp(t\varphi_2)(A(0))\right)=\exp(t\varphi_1)\exp(-tB)(A_0).
	\end{equation*}
	On a 
	\begin{equation*}
		\exp(t\varphi_{1})(M)=M\exp(tB).
	\end{equation*}
	Ainsi,
	\begin{equation*}
		A(t)=\exp(-tB)A_0\exp(tB),
	\end{equation*}
	donc $A(t)$ est semblable à $A_0$.
\end{proof}

\begin{remark}
	Si $A_0$ et $B$ commutent alors $A(t)=A_0$ donc pour tout $t\in\R$, $A(t)$ et $B$ commutent.
\end{remark}

\begin{remark}
	Onp eut aussi résoudre en écrivant 
	\begin{equation*}
		\underbrace{\e^{tB}(A'(t)+BA(t))}_{C'(t)}=\underbrace{\e^{tB}A(t)}_{C(t)}B.
	\end{equation*}

	Donc $C'(t)=C(t)B$ puis $C'(t)\exp(-tB)-C(t)B\exp(-tB)=0=D'(t)$ avec $D(t)=\exp(-tB)$. Ainsi, $D(t)=D(0)$, d'où $C(t)=C(0)\exp(tB)$ puis 
	\begin{equation*}
		A(t)=\exp(-tB)A(0)\exp(tB).
	\end{equation*}
\end{remark}

\begin{remark}
	Si on a maintenant $A'(t)=A(t)B(t)-B(t)A(t)$, soit pour $k\in\N$, $\varphi_k(t)=\Tr(A^{k}(t))$. Alors 
	\begin{align*}
		\varphi_k'(t)
		&=\Tr(-\sum_{i=0}^{k-1}A^{i}(t)A'(t)A^{k-1-i}(t)),\\
		&=\sum_{i=0}^{k-1}\Tr(A'(t)A^{k-1}(t)),\\
		&=k\Tr(A'(t)A^{k-1}(t)),\\
		&=k\left(\Tr(A(t)B(t)A^{k-1}(t))-\Tr(B(t)A^{k}(t))\right).
	\end{align*}
	Donc $\varphi_k'(t)=0$, donc $t\mapsto\Tr(A^{k}(t))$ est constant. Or les coefficients de $\chi_{A}$ sont des polynômes en $(\Tr(A^{k})_{1\leqslant k\leqslant n-1})$, donc $\chi_{A(t)}$ est constant. Si $\chi_{A_{0}}=\prod_{k=1}^{n}(X-\lambda_k)$ est scindé à racines simples, alors pour tout $t\in\R$, $A(t)$ est semblable à $\diag(\lambda_i)$ donc à $A_0$.
\end{remark}

\begin{proof}
	\phantom{}
	\begin{enumerate}
		\item On a 
		\begin{align*}
			X_3'(t)
			&=-\exp(-t(A+B))(A+B)\exp(tB)\exp(tA)\\
			&\qquad+\exp\left(-t(A+B)\right)\left(B\exp(tB)\exp(tA)+\exp(tB)A\exp(tA)\right)\nonumber,\\
			&=\exp(-t(A+B))\left(-(A+B)+B+\exp(tB)A\exp(-tB)\right)\exp(tB)\exp(tA).
		\end{align*}

		Donc $\varphi(t)=-A+\exp(tB)A\exp(-tB)$ est de classe $\mathcal{C}^{1}$. De plus, on a 
		\begin{align*}
			\varphi'(t)
			&=\exp(tB)BA\exp(-tB)-\exp(tB)AB\exp(-tB),\\
			&=\exp(tB)[B,A]\exp(-tB).
		\end{align*}

		\item $[B,[A,B]]=0$ donc $B$ commute avec $[B,A]$. Ainsi, $\varphi'(t)=[B,A]$ et 
		\begin{equation*}
			\varphi(t)=t(BA-AB)+\varphi(0)=t(AB-BA).
		\end{equation*}
		Puis on a ($A$ et $B$ commutent avec $[A,B]$)
		\begin{align*}
			\chi_{3}'(t)
			&=t\exp(-t(A+B))[B,A]\exp(tB)\exp(tA),\\
			&=t[B,A]\chi_3(t).
		\end{align*}

		Ainsi,
		\begin{equation*}
			\exp\left(-\frac{t^{2}}{2}[B,A]\right)\left(X_3'(t)-t[B,A]\chi_3(t)\right)=C'(t)=0,
		\end{equation*}
		avec $C(t)=\exp\left(-\frac{t^{2}}{2}[B,A]\right)\chi_{3}(t)$, donc 
		\begin{equation*}
			\chi_3(t)=\exp\left(\frac{t^{2}}{2}[B,A]\right)\chi_3(0)=\exp\left(\frac{t^{2}}{2}[B,A]\right).
		\end{equation*}

		Ainsi,
		\begin{equation*}
			\exp\left(t(A+B)\right)=\exp(tB)\exp(tA)\exp\left(-\frac{t^{2}}{2}[B,A]\right),
		\end{equation*}
		et pour $t=1$,
		\begin{equation*}
			\exp(A+B)=\exp(B)\exp(A)\exp\left(-\frac{1}{2}[B,A]\right).
		\end{equation*}
	\end{enumerate}
\end{proof}

\begin{proof}
	\phantom{}
	\begin{enumerate}
		\item Si $X=\emptyset$, c'est bon. Sinon, soit $x_0\in X$. Si $y'(x_0)=0$, $y$ est solution de l'équation différentielle avec $y(x_0)=y'(x_0)=0$ et 0 est aussi solution. Par unicité venant du théorème de Cauchy-Lipschitz, on a $y=0$ ce qui n'est pas. Donc $y'(x_0)\neq0$ et par continuité de $y'$ $y'>0$ au voisinage de $x_0$ donc $y$ est localement injective.
		
		\item Supposons $\left\lvert X\right\rvert=+\infty$. Soit $(x_n)_{n\in\N}\in X^{\N}$ injective. Comme $X_n\subset I$, $x_n\in I$ pour tout $n\in\N$, donc il existe $\sigma\colon\N\to\N$ strictement croissante telle que $(x_{\sigma(n)})_{n\in\N}$ converge vers $x\in I$.
		
		Or $y(x_{\sigma(n)})=0$ pour tout $n\in\N$ donc par continuité de $y$, on a $y(x)=0$. Ainsi, pour tout $a>0$, il existe $x_n\in X$ tel que $x_n\in]x-a,x+a[$, impossible d'après la première question.

		\item Stratégie : on va montrer que $X$ est dénombrable, qu'il existe $x_0\in X$ tel que pour tout $x\in X$, $x_0\leqslant x$, et ainsi de suite par récurrence sur $X\setminus\left\lbrace x_0\right\rbrace$.
		
		Pour tout $B<0$, soit $\widetilde{I})[a,B]$. On a $\left\lvert X\cap\widetilde{I}\right\rvert<\infty$. On a 
		\begin{equation*}
			I=\bigcup_{n\in\N}\underbrace{[B_n, B_{n+1}]}_{I_n},
		\end{equation*}
		avec $B_0=a$ et $(B_n)$ strictement croissante, $B_n\xrightarrow[n\to+\infty]{}B$. Alors 
		\begin{equation*}
			X=\bigcup_{n\in\N}\underbrace{I_n\cap X}_{{\text{fini}}}.
		\end{equation*}
		Donc $X$ est dénombrable. On a $X_n=I_n\cap X$. Chaque $X_n$ s'ordonne en $x_1^{(n)}<\dots<x_{r_n}^{(n)}$.
	\end{enumerate}
\end{proof}

\begin{proof}
	\phantom{}
	\begin{enumerate}
		\item Le Wronskien $W_{y_1,y_2}(t)=(y_1 y_2'-y_1' y_2)(t)$ garde un signe constant. On a $W_{y_1,y_2}(a)=-y_1'(a)y_2(a)$ et $W_{y_1,y_2}(0)=-y_1'(b)y_2(b)$. $y_1'(a)$ et $y_1'(b)$ sont différents de 0 par unicité du théorème de Cauchy-Lipschitz (sinon $y_1=0$).
		
		Si $y_1>0$ sur $]a,b[$ : si $y_1'(a)<0$, par continuité de $y_1'$, $y_1'$ reste négatif à droite de $a$ donc $y_1$ y est strictement décroissante donc négative : impossible. Donc $y_1(a)>0$. De même, $y_1'(b)<0$. Or le Wronskien ne change pas de signe et 
		\begin{equation*}
			y_1'(a)y_1'(b)y_2(a) y_2(b)=W_{y_1,y_2}(a)\times W_{y_1,y_2}(b)>0.
		\end{equation*}
		Donc $y_2(a) y_2(b)<0$. Comme $y_2$ est continue, le théorème des valeurs intermédiaires s'applique et $y_2$ s'annule sur $]a,b[$. 

		Si $y_1<0$, on applique ce qui précède à $-y_1$.

		Si $y_2$ s'annulait deux fois sur $]a,b[$, comme $y_1$ et $y_2$ jouent des rôles symétriques, $y_1$ s'annulerait une fois sur $]a,b[$ : impossible.

		\item Soit $H=y_1 y_2'-y_2 y_1'$. On a 
		\begin{equation*}
			H'=y_1 y_2''-y_2 y_1''=(r_1-r_2)y_1 y_2.
		\end{equation*}

		Supposons que $y_1>0$ sur $]a,b[$. Su $y_2$ ne s'annule pas sur $]a,b[$, supposons par exemple que $y_2>0$ sur $]a,b[$. Alors $H'<0$ sur $]a,b[$, $H$ est strictement décroissante sur $[a,b]$, et $H(0)=-y_2(a) y_1'(a)<0$, $H(b)=-y_2(b) y_1'(b)>0$ : impossible. Donc $y_2$ s'annule au moins une fois sur $]a,b[$.

		Application : si pour tout $t\in I$, $r_1(t)<\omega^{2}$, soit $a<b$ deux zéros consécutifs de $y_1$ et $y_2(t)=\sin(\omega(t-a))$. Les zéros de $y_2$ sont les $a+\frac{k\pi}{\omega}$ d'où un écart plus grand que $\frac{\pi}{\omega}$.

		Soit $a$ un zéro de $y_1$. En échangeant les rôles joués par $r_1$ et $r_2$ : $y=\sin(\omega'(t-a))$ s'annule en $0$ et $a+\frac{\pi}{\omega}$ (deux zéros consécutifs). Donc l'écart entre deux zéros consécutifs de $y_1$ est plus petit que $\frac{\pi}{\omega}$.
	\end{enumerate}
\end{proof}

\begin{proof}
	\phantom{}
	\begin{enumerate}
		\item Il est clair que $\mathcal{T}_{T}$ est linéaire. Pour tout $Y\in S$, pour tout $x\in\R$, on a 
		\begin{equation*}
			(\mathcal{T}_{T}(y))''(x)+p(x)\mathcal{T}_{T}(y)(x)=y''(x+T)+p(x+T)y(x+T)=0,
		\end{equation*}
		donc $\mathcal{T}_{T}(y)\in\mathcal{L}(S)$. Via le théorème de Cauchy-Lipschitz, $\dim(S)=2$. Posons $A=\frac{\Tr(\mathcal{T}_T)}{2}$. D'après le théorème de Cayley-Hamilton, 
		\begin{equation*}
			X^{2}-2AX+\det(\mathcal{T}_T)
		\end{equation*}
		annule $\mathcal{T}_T$. Soit alors $(y_1,y_2)$ la base de $S$ telle que $y_1(0)=1$, $y_1'(0)=0$, $y_2(0)=0$ et $y_2'(0)=1$.

		Si $y=\alpha y_1+\beta y_2\in S$, alors $y(0)=\alpha$ et $y'(0)=\beta$ donc $y=y(0)y_1+y'(0)y_2$. Ainsi, 
		\begin{equation*}
			\mathcal{T}_T(y_1)=y_1(T)y_1+y_1'(T)y_2=\mathcal{T}_T(y_1)(0) y_1+\mathcal{T}_T(y_1)'(0) y_2,
		\end{equation*}
		d'où 
		\begin{equation*}
			\mat_{(y_1,y_2)}(\mathcal{T}_T)=\begin{pmatrix}
				y_{1}(T) & y_2(T)\\
				y_1'(T) & y_2'(T)
			\end{pmatrix}.
		\end{equation*}

		Ainsi, $\det(\mathcal{T}_{T})=y_1(T)y_2'(T)-y_1'(T)y_2(T)=W_{y_1,y_2}(T)$ où $W$ est le Wronskien. On a 
		\begin{align*}
			W_{y_1,y_2}'(x)
			&=y_1(x)y_2''(x)-y_1''(x)y_2(x),\\
			&=-y_1(x)p(x)y_2(x)+y_1(x)p(x)y_2(x),\\
			&=0.
		\end{align*}
		Donc $W_{y_1,y_2}$ est constant et $W_{y_1,y_1}(0)=1$ donc $\det(\mathcal{T}_T)=1$. Ainsi, 
		\begin{equation*}
			\chi_{\mathcal{T}_T}=X^{2}-2AX+1.
		\end{equation*}

		On a $A=\frac{\Tr(\mathcal{T}_T)}{2}=\frac{1}{2}(y_1(T)+y_2'(T))$ donc pour tout $y\in S$, pour tout $x\in\R$, $y(x+2T)-2Ay(x+T)+y(x)=0$.

		\item On a $\chi_{\mathcal{T}_T}=X^{2}-2AX+1$. On a $\Delta=4(A^{2}-1)<0$ si $\left\lvert A\right\rvert<1$. On a deux racines complexes conjuguées $\mu$ et $\overline{\mu}$. De plus, $\mu\overline{\mu}=1=\det(\mathcal{T}_T)$ donc $\mu\in\U$. Ainsi, il existe $\theta\in]0,\pi[$ tel que $\Sp_{\C}(\mathcal{T}_T)=\left\lbrace\e^{\i\theta},\e^{-\i\theta}\right\rbrace$. Donc $\mat_{(y_1,y_2)}(\mathcal{T}_T)$ est semblable sur $\R$ à
		\begin{equation*}
			R_{\theta}=\begin{pmatrix}
				\cos(\theta) & -\sin(\theta)\\
				\sin(\theta) & \cos(\theta)
			\end{pmatrix}
		\end{equation*}

		Soit $(f_1,f_2)$ la base de $S$ telle que $\mat_{(f_1,f_2)}(\mathcal{T}_{T})=R_{\theta}$. Pour tout $n\in\Z$, 
		\begin{equation*}
			\mat_{(f_1,f_2)}(\mathcal{T}_{T}^{n})=R_{n\theta}.
		\end{equation*}

		Si $f=af_1+bf_2$, on a 
		\begin{equation*}
			\mathcal{T}_T^{n}(f)=(a\cos(n\theta)-b\sin(n\theta))f_1+(a\sin(\theta)+b\cos(n\theta))f_2=f(x+nT).
		\end{equation*}

		Pour tout $x\in[0,T]$, pour tout $n\in\Z$,
		\begin{equation*}
			\left\lvert f(x+nT)\right\rvert\leqslant
			\sqrt{a^{2}+b^{2}}
			\left(\left\lVert f_{1}\right\rVert_{\infty,[0,T]}+\left\lVert f_2\right\rVert_{\infty,[0,T]}\right),
		\end{equation*}
		donc $f$ est bornée.

		\item Si $\left\lvert A\right\rvert>1$, on a $\delta>0$ et 
		\begin{equation*}
			\Sp(\mathcal{T}_T)=\left\lbrace\lambda,\frac{1}{\lambda}\right\rbrace,
		\end{equation*}
		avec $\left\lvert\lambda\right\rvert\in]0,1[$. Il existe $(f_1,f_2)$ base de $S$ telle que $\mathcal{T}_{T}(f_1)=\lambda f_1$ et $\mathcal{T}_{T}(f_2)=\frac{1}{\lambda}f_2$. 

		Ainsi, si $f=af_1+bf_2$, pour tout $x\in[0,T]$, pour tout $n\in\Z$, on a 
		\begin{equation*}
			\left\lvert f(x+nT)\right\rvert=\left\lvert \lambda^{n}af_1(x)+\frac{b}{\lambda^{n}}f_2(x)\right\rvert\xrightarrow[n\to+\infty]{}+\infty,
		\end{equation*}
		donc toutes les solutions non nulles sont non bornées.

		Si $A=1$, on a $\chi_{\mathcal{T}_{T}}=(X-1)^{2}$. Ou bien $\mathcal{T}_{T}=id$ et dans ce cas toutes les solutions sont $T$-périodiques donc bornées (car continues). Ou bien il existe une base $(f_1,f_2)$ de $S$ telle que 
		\begin{equation*}
			\mat_{(f_1,f_2)}(\mathcal{T}_{T})=\begin{pmatrix}
				1&1\\
				0&1
			\end{pmatrix}.
		\end{equation*}
		On a 
		\begin{equation*}
			\mat_{(f_1,f_2)}(\mathcal{T}_{T}^{n})=\begin{pmatrix}
				1&n\\0&1
			\end{pmatrix}.
		\end{equation*}

		Ainsi, il existe des solutions non nulles périodiques et des solutions non bornées.
	\end{enumerate}
\end{proof}

\begin{proof}
	\phantom{}
	\begin{enumerate}
		\item $(x\mapsto\e^{x},x\mapsto\e^{-x})$ est une base de $S$ (espace des solutions de l'équation différentielle). On cherche la solution générale sous la forme 
		\begin{equation*}
			y(x)=\lambda(x)\e^{x}+\mu(x)\e^{-x},
		\end{equation*}
		avec $\lambda'(x)\e^{x}+\mu'(x)\e^{-x}=0$ et $\lambda'(x)\e^{x}-\mu'(x)\e^{-x}=f(x)$.

		Donc $\lambda'(x)=\frac{1}{2}f(x)\e^{-x}$ et $\mu'(x)=-\frac{1}{2}f(x)\e^{x}$. Donc il existe $(\lambda,\mu)\in\R^{2}$ tel que pour tout $x\in\R$,
		\begin{equation*}
			y(x)=\frac{1}{2}\left(\left(\int_{0}^{x}f(t)\e^{-t}\d t\right)\e^{x}+\lambda\e^{x}+\left(\int_{0}^{x}f(t)\e^{t}\d t+\mu\right)\e^{-x}\right).
		\end{equation*}

		Soit $\varepsilon>0$. Il existe $A\geqslant0$ tel que pour tout $t\geqslant A$, $\left\lvert f(t)\right\rvert\leqslant\varepsilon$. Alors pour tout $x\geqslant A$, on a 
		\begin{equation*}
			\left\lvert\int_{0}^{x}f(t)\e^{t}\d t\e^{-x}\right\rvert\leqslant\varepsilon\left\lvert 1-\e^{-x}\right\rvert\leqslant\varepsilon,
		\end{equation*}
		donc $\lim\limits_{x\to+\infty}\int_{0}^{x}f(t)\e^{t}\d t\e^{-x}=0$.

		Si $y$ est bornée, nécessairement $\lim\limits_{x\to+\infty}\int_{0}^{x}f(t)\e^{-t}\d t+\lambda=0$. Donc 
		\begin{equation*}
			\lambda=-\int_{0}^{+\infty}f(t)\e^{-t}\d t,
		\end{equation*}
		définie car $f$ est bornée. De même, 
		\begin{align*}
			\lim\limits_{x\to-\infty}\int_{0}^{x}f(t)\e^{-t}\d t\e^{x}
			&=\lim\limits_{x'\to+\infty}\left(-\int_{0}^{x'}f(-u)\e^{u}\d u\right)\e^{-x'},\\
			&=0.
		\end{align*}
		Donc $\mu=\int_{-\infty}^{0}f(t)\e^{t}\d t$ (définie car $f$ est bornée). Alors 
		\begin{equation*}
			y(x)=\frac{1}{2}\left(-\int_{x}^{+\infty}f(t)\e^{-t}\d t\e^{x}+\int_{-\infty}^{x}f(t)\e^{t}\d t\e^{-x}\right).
		\end{equation*}

		Réciproquement, posons 
		\begin{equation*}
			y_0(x)=\frac{1}{2}\left(-\int_{x}^{+\infty}f(t)\e^{-t}\d t\e^{x}+\int_{-\infty}^{x}f(t)\e^{t}\d t\e^{-x}\right).
		\end{equation*}

		On a 
		\begin{equation*}
			\int_{x}^{+\infty}f(t)\e^{-t}\d t\e^{x}=\int_{x}^{+\infty}f(t)\e^{x-t}\d t=\int_{0}^{+\infty}f(u+x)\e^{-u}\d u.
		\end{equation*}
		Pour tout $x\in\R$, $\left\lvert f(u+x)\e^{-u}\right\rvert\leqslant\left\lVert f\right\rVert_{\infty,\R}\e^{-u}$, intégrable. D'après le théorème de convergence dominée, on a 
		\begin{equation*}
			\lim\limits_{\left\lvert x\right\rvert\to+\infty}\int_{x}^{+\infty}f(t)\e^{-t}\d t\e^{x}=0.
		\end{equation*}
		De même, on a 
		\begin{equation*}
			\lim\limits_{\left\lvert x\right\rvert\to+\infty}\int_{-\infty}^{x}f(t)\e^{t}\d t\e^{-x}=0.
		\end{equation*}
		Donc $y_0(x)\xrightarrow[\left\lvert x\right\rvert\to+\infty]{}0$. Donc $y_0$ est bornée et sa limite est 0.
	\end{enumerate}
\end{proof}

\begin{proof}
	\phantom{}
	\begin{enumerate}
		\item Comme $p\colon[a,+\infty[\to\R_{+}^{*}$, l'équation différentielle équivaut à $x''+\frac{p'}{p}x'+\frac{q}{p}x=0$ et le théorème de Cauchy-Lipschitz s'applique.
		
		La première partie vient de l'unicité du théorème de Cauchy-Lipschitz. La deuxième vient du théorème de relèvement.

		\item Il vient
		\begin{equation*}
			\begin{array}[]{rcl}
				(px')'&=&px''+p'x'=r'\cos\theta-r\theta'\sin\theta,\\
				x' &=& r'\sin\theta+r\theta'\cos\theta=\frac{r\cos\theta}{p}.
			\end{array}
		\end{equation*}
		$x$ est solution si et seulement si $(xp)'=-qx=-qr\sin\theta$ si et seulement si 
		\begin{equation*}
			\left\lbrace
				\begin{array}[]{rcl}
					r'\cos\theta+r(q-\theta')\sin\theta &=&0,\\
					r'\sin\theta+r\left(\theta'-\frac{1}{p}\right)\cos\theta&=&0,
				\end{array}
			\right.
		\end{equation*}
		si et seulement si 
		\begin{equation*}
			\left\lbrace
			\begin{array}[]{rcl}
				r'&=&r\sin\theta\cos\theta\left(\frac{1}{p}-q\right),\\
				\theta'&=&q\sin^{2}\theta+\frac{1}{p}\cos^{2}\theta.
			\end{array}
			\right.
		\end{equation*}

		\item Si $p=1$, on a 
		\begin{equation*}
			\left\lbrace
				\begin{array}[]{rcl}
					\theta'&=&q\sin^{2}\theta+\cos^{2}\theta,\\
					r'=r\sin\theta\cos\theta\left(1-q\right).
				\end{array}
			\right.
		\end{equation*}
		On a $\theta'>0$ donc $\theta$ est strictement croissante et admet une limite $l\in\overline{\R}$ en $+\infty$. Si $l<+\infty$, on a 
		\begin{equation*}
			\int_{a}^{t}\theta'(t)\d u=\theta(t)-\theta(a)\xrightarrow[t\to+\infty]{}l-\theta(a).
		\end{equation*}

		De plus, 
		\begin{align*}
			\int_{a}^{t}\theta'(u)\d u
			&=\int_{a}^{t}q(u)\sin^{2}(\theta(u))\d u+\int_{a}^{t}\cos^{2}(\theta(u))\d u,\\
			&\geqslant\int_{a}^{t}q(u)\sin^{2}(\theta(u))\d u,\\
			&\underset{u\to+\infty}{\sim}q(u)\sin^{2}(l).
		\end{align*}

		Comme $\int_{a}^{t}q(u)\d u$ diverge, nécessairement, $\int_{a}^{t}\theta'(u)\d u$ étant finie, on a $\sin^{2}(l)=0$ donc $\cos^{2}(l)=1$ et 
		$\int_{a}^{t}\cos^{2}(\theta(u))\d u\xrightarrow[t\to+\infty]{}+\infty$ : contradiction. 

		Nécessairement, $l=+\infty$, puis par le théorème des valeurs intermédiaires, pour tout $k\in\N$ tel que $k\pi\geqslant a$, il existe un unique $t_k\in[a,+\infty[$ tel que $\theta(t_k)=k\pi$ et $x(t_k)=0$. Donc $x$ s'annule une infinité de fois.
	\end{enumerate}
\end{proof}

\begin{proof}
	Si (ii), soit $a\in\R$, alors pour tout $f\in E$, $(\mathcal{T}_a(f))'=\mathcal{T}_a(f')$. Alors pour tout $x\in\R$,
	\begin{equation*}
		f^{(n)}(x+a)+a_{n-1}f^{(n-1)}(x+a)+\dots+a_{0}f(x+a)=0,
	\end{equation*}
	donc $\mathcal{T}_{a}(f)\in E$, d'où (iii).

	Si (i), on note $\chi_{\Delta}(X)=\sum_{i=0}^{n-1}a_{i}X^{i}+X^{n}$ le polynôme caractéristique de $\Delta\colon f\mapsto f'\in\mathcal{L}(E)$. D'après le théorème de Cayley-Hamilton, on a $\chi_{\Delta}(\Delta)=0_{\mathcal{L}(E)}$. Donc pour tout $f\in E$, 
	\begin{equation*}
		\chi_{\Delta}(\Delta)(f)=f^{(n)}+a_{n-1}f^{(n-1)}+\dots+a_{0}f=O_{E},
	\end{equation*}
	donc $E$ est inclus dans l'ensemble solution. Puis, d'après le théorème de Cauchy-Lipschitz, la dimension de l'espace des solutions est $n=\dim(E)$ donc on a bien égalité. D'où (ii).

	Si (iii), notons que s'il existe $(_1,\dots,x_n)\in\R^{n}$ tel que pour tout $f\in E$, $f(x_1)=\dots=f(x_n)=0$, alors $f=0$. En effet, soit pour tout $x\in\R$, \function{\delta_x}{\mathcal{C}^{\infty}(\R,\C)}{\C}{f}{f(x)} une forme linéaire sur $E$.
	D'après le théorème de caractérisation des formes linéaires, il existe $g_x\in E$ tel que pour tout $f\in E$, $\delta_x(f)=f(x)=(g_x|f)$ (produit scalaire complexe a priori). Soit $f\in E$, si pour tout $x\in\R$, $(g_x|f)=0$ alors $f=0$. Ainsi, $\left(\Vect((g_x)_{x\in\R})\right)^{\perp}=\left\lbrace0\right\rbrace$. Donc $\Vect((g_x)_{x\in\R})=E$. Donc $(g_x)_{x\in\R}$ est une famille génératrice de $E$, ainsi il existe $(x_1,\dots,x_n)\in\R^{n}$ tel que $(g_{x_1},\dots,g_{x_n})$ est une base de $E$, donc $(\delta_{x_1},\dots,\delta_{x_n})$ est une base de $\mathcal{L}(E,\C)$ (ensemble des formes linéaires sur $E$ de dimension $n$). En effet, c'est une famille libre car si $\sum_{i=1}^{n}\lambda_i \delta_{x_i}=0$ alors pour tout $f\in E$, $\left(\sum_{i=1}^{n}\lambda_i g_{x_i}\middle| f\right)=0$ donc $\sum_{i=1}^{n}\lambda_i g_{x_i}=0$ et $\lambda_1=\dots=\lambda_n=0$.
	Alors pour tout $x\in\R$, il existe $(\lambda_1,\dots,\lambda_n)\in\C^{n}$ tel que $\delta_x=\lambda_1(x)\delta_{x_1}+\dots+\lambda_n(x)\delta_{x_n}$. Donc si $f(x_1)=\dots=f(x_n)=0$, alors pour tout $x\in\R$, $f(x)=\delta_x(f)=\sum_{i=1}^{n}\lambda_i\delta_{x_i}(f)=0$ d'où $f=0$.

	Ensuite, notons qu'il existe $(h_1,\dots,h_n)$ base de $E$ telle que pour tout $f\in E$, $f=\sum_{i=1}^{n}f(x_i)h_i$. En admettant ce résultat, on définit 
	\begin{equation*}
		g=\sum_{i=1}^{n}f'(x_i)h_i,
	\end{equation*}
	et pour tout $i\in\left\llbracket1,n\right\rrbracket$, $f'(x_i)=g(x_i)$. Pour tout $x\in E$, on a 
	\begin{equation*}
		f'(x)=\lim\limits_{p\to+\infty}p\left(\mathcal{T}_{\frac{1}{p}}(f)(x)-f(x)\right).
	\end{equation*}
	Si $\delta_x=\sum_{i=1}^{n}\lambda_i \delta_{x_i}$, on a 
	\begin{align*}
		p\left(\mathcal{T}_{\frac{1}{p}}(f)(x)-f(x)\right)
		&=p\left(f\left(x+\frac{1}{p}\right)-f(x)\right),\\
		&=\delta_x\left(\mathcal{T}_{\frac{1}{p}}(f)-f\right),\\
		&=\sum_{i=1}^{n}\lambda_i\delta_{x_i}\left(p\left(\mathcal{T}_{\frac{1}{p}}(f)-f\right)\right),\\
		&=\sum_{i=1}^{n}\lambda_i p\left(f\left(x_i+\frac{1}{p}\right)-f(x_i)\right),\\
		&\xrightarrow[p\to+\infty]{}\sum_{i=1}^{n}\lambda_i f'(x_i),\\
		&=\sum_{i=1}^{n}\lambda_i g(x_i),\\
		&=g(x),\\
		&=f'(x).
	\end{align*}

	D'où (i).
\end{proof}

\begin{remark}
	En notant le polynôme minimal $\Delta$ $\Pi_{\Delta}$, on a $\deg(\Pi_{\Delta})=n$. En effet, si $\Pi_{\Delta}=b_0+b_1 X+\dots+b_{m-1}X^{m-1}+X^{m}$ avec $m\leqslant n$ (d'après le théorème de Cayley-Hamilton), alors $E$ est inclus dans l'ensemble solution de l'équation différentielle $b_0+b_1 y+\dots+b_{m-1}y^{(m-1)}+y^{(m)}=0$ qui est de dimension $m$. Or $\dim(E)=n$ et $m\leqslant n$, donc $m=n$ et $\chi_{\Delta}=\pi_{\Delta}$.
\end{remark}

\begin{proof}
	\phantom{}
	\begin{enumerate}
		\item Il existe $(m,M)\in(\R_{+}^{*})^{2}$ tel que $m\leqslant \Delta\leqslant M$. Si $\lambda=0$, $f$ est affine et $f(0)=f(1)=0$ implique $f=0$. Si $\lambda>0$, on a 
		\begin{equation*}
			\lambda mf\leqslant f''=\lambda\Delta f\leqslant \lambda Mf.
		\end{equation*}
		Posons $g$ solution de $g''=\lambda mg$ et $h$ solution de $f''=\lambda Mh$, avec $g(0)=h(0)=0$, $g'(0)=h'(0)=f'(0)$. On a 
		\begin{equation*}
			\begin{array}[]{rcl}
				g(t) &=& \frac{f'(0)}{\sqrt{\lambda m}}\sinh\left(\sqrt{\lambda m}t\right),\\
				h(t) &=& \frac{f'(0)}{\sqrt{\lambda M}}\sinh\left(\sqrt{\lambda M}t\right).
			\end{array}
		\end{equation*}
		Donc $g(1)\neq0$ et $h(1)\neq0$. On a 
		\begin{equation*}
			0\leqslant (f-g)''-\lambda m(f-g)=f''-\lambda mf.
		\end{equation*}
		Si $f_1=f-g$, on a $f_1''-\lambda m f_1=\varepsilon\geqslant0$ et $f_1(0)=f_1'(0)=0$. Résolvons $f_1''-\lambda mf_1=\varepsilon_{1}$ avec $f_1'(0)=f_1(0)=0$. On a 
		\begin{equation*}
			f_1(t) = \lambda(t)\sinh\left(\sqrt{\lambda m}t\right)+\mu(t)\cosh\left(\sqrt{\lambda m}t\right),
		\end{equation*}
		avec $\lambda'(t)\sinh\left(\sqrt{\lambda m}t\right)+\mu'(t)\cosh\left(\sqrt{\lambda m}t\right)=0$. Il vient 
		\begin{equation*}
			\sqrt{\lambda m}\left(\lambda'(t)\cosh\left(\sqrt{\lambda m}t\right)\right)+\mu'(t)\sinh\left(\sqrt{\lambda m}t\right)=\varepsilon_{1}(t).
		\end{equation*}
		D'où 
		\begin{equation*}
			\begin{array}[]{rcl}
				\lambda'(t) &=& \frac{1}{\sqrt{\lambda m}}\cosh\left(\sqrt{\lambda m}t\right)\varepsilon_{1}(t),\\
				\mu'(t) &=& -\frac{1}{\sqrt{\lambda m}}\sinh\left(\sqrt{\lambda m}t\right)\varepsilon_{1}(t).
			\end{array}
		\end{equation*}
		On a $f_1(0)=0$ donc $\mu(0)=0$ et $f_1'(0)=0$ donc $\lambda(0)=0$. Finalement,
		\begin{align*}
			f_1(t)
			&=\frac{1}{\sqrt{\lambda m}}\int_{0}^{t}\left(\sinh\sqrt{\lambda m}u\cosh\sqrt{\lambda m}u-\cosh\sqrt{\lambda m}u\sinh\sqrt{\lambda m}u\right)\varepsilon_{1}(u)\d u,\\
			&=\frac{1}{\sqrt{\lambda m}}\int_{0}^{t}\sinh\sqrt{\lambda m}(t-u)\varepsilon_{1}(u)\d u\geqslant0.
		\end{align*}

		Donc $f\geqslant g$. De même, $f\leqslant h$. Donc quelle que soit la valeur de $f'(0)$, on a $f(1)>0$ ou $f(1)<0$. Ainsi, $\lambda\leqslant0$.

		On pose $\left\langle f,g\right\rangle=\int_{0}^{1}\Delta fg$. C'est un produit scalaire car $\Delta>0$. Vérifions que $v$ est autoadjoint pour ce produit scalaire : 
		\begin{equation*}
			\left\langle v(f), g\right\rangle=\int_{0}^{1}f''(t)g(t)\d t=\underbrace{\left[f(t)g(t)\right]_{0}^{1}}_{=0\text{ car }g\in E}-\int_{0}^{1}f'(t)g'(t)\d t,
		\end{equation*}
		expression symétrique en $f$ et $g$. Donc $\left\langle v(f), g\right\rangle=\left\langle f, v(g)\right\rangle$. Si $v(f)=\lambda f$ et $v(g)=\lambda g$, on a alors $\lambda\left\langle f,g\right\rangle=\mu\left\langle f,g\right\rangle$ donc si $\lambda\neq\mu$, on a $\left\langle f,g\right\rangle=0$.

		\item C'est une conséquence immédiate du théorème de Cauchy-Lipschitz.
		\item Sur $[2,+\infty[$ on a $f''=\gamma f$ et $\gamma<0$ d'après la première question. Donc il existe $(A,\varphi)\int\R\times\R$ tel que pour tout $t\in[2,+\infty[$, $f(t)=A\sin\left(\sqrt{-\gamma}t+\varphi\right)$.
		
		Si $A=0$, $f$ est solution du problème de Cauchy $f''=\gamma\Delta f$ avec $f(2)=f'(2)=0$ donc $f=0$ par unicité du théorème de Cauchy-Lipschitz, ce qui est absurde car $f'(0)=1$. Donc $A\neq0$ et $f$ s'annule en $\frac{k\pi-\varphi}{\sqrt{-\gamma}}$ avec $k\in\N$ sur $[2,+\infty[$.

		Sur $[0,2]$, si $f$ s'annule une infinité de fois, il existe $(a_n)_{n\in\N}$ une suite injective de $[0,2]$ telle que $f(a_n)=0$ pour tout $n\in\N$. On extrait $(a_{\sigma(n)})_{n\in\N}$ qui converge vers $a\in[0,2]$. $f$ étant continue sur $[0,2]$, $f(a)=0$ et d'après le théorème de Rolle, pour tout $n\in\N$, il existe $b_n\in]a,a_{\sigma(n)}[$ (ou bien $]a_{\sigma(n)},a[$) tel que $f'(b_n)=\gamma$. Par continuité de $f'$, puisque $b_n\to0$, on a $f'(a)=0$. $f$ est alors solution du problème de Cauchy $y''=\gamma\Delta y$ avec $y(a)=y'(a)=0$. Par unicité du théorème de Cauchy-Lipschitz, $f=0$ ce qui est absurde car $f'(0)=1$. Donc $f$ s'annule un nombre fini de fois sur $[0,2]$.

		\item Soit $A>0$. Sur $[0,A]$, notons $M=\sup\limits_{[0,A]}\left\lvert \Delta\right\rvert$. Sur $[0,x_{1}(\gamma)]$, $f_{\gamma}$ est positive (car ne change pas de signe et $f_{\gamma}'(0)=1$). Notons $t_{\gamma}\in[0,x_{1}(\gamma)]$ tel que $f_{\gamma}(t_{\gamma})=\max\limits_{t\in[0,x_{1}(\gamma)]}f_{\gamma}(t)$. Pour tout $t\in]0,x_{1}(\gamma)[$, on a 
		\begin{equation*}
			f_{\gamma}''(t)=\Delta(t)\gamma f_{\gamma}(t)<0,
		\end{equation*}
		donc $f_{\gamma}$ est concave sur $[0,x_{1}(\gamma)]$. Ainsi, pour tout $t\in[0,x_{1}(\gamma)]$, $f_{\gamma}(t)\leqslant t$ (en-dessous de la tangente en 0). Donc $f_{\gamma}(t_{\gamma})\leqslant t_{\gamma}\leqslant x_{1}(\gamma)$. Alors pour tout $t\in[0,x_{1}(\gamma)]$, on a 
		\begin{equation*}
			0\leqslant f(t)\leqslant x_1(\gamma)(\gamma)\leqslant A,
		\end{equation*}
		et $\gamma MA\leqslant f_{\gamma}''(t)\leqslant0$. D'après l'inégalité des accroissements finis, on a 
		\begin{align*}
			1=\left\lvert f'(t_{\gamma})-f'(0)\right\rvert
			&\leqslant\left\lvert\gamma\right\rvert MAt_{\gamma},\\
			&\leqslant\left\lvert\gamma\right\rvert MAx_{1}(\gamma),
		\end{align*}
		donc 
		\begin{equation*}
			x_{1}(\gamma)\geqslant\frac{1}{MA\left\lvert\gamma\right\rvert}\xrightarrow[\gamma\to0]{}+\infty.
		\end{equation*}
	\end{enumerate}
\end{proof}

\begin{remark}
	Autre méthode pour la première question : comme $f(0)=f(1)=0$ et $f\neq0$, il existe $x_0\in]0,1[$, $f(x_0)\neq0$. Quitte à remplacer $f$ par $-f$ on suppose $f(x_0)>0$. Alors $\max\limits_{[0,1]}f>0$ et il existe $x_1\in]0,1[$ tel que $f(x_1)=\max\limits_{[0,1]}f$. Il vient $f'(x_1)=0$ et si $\lambda>0$, on a $f''(x_1)=\lambda\Delta(x_1)f(x_1)>0$. Un développement limité fournit 
	\begin{equation*}
		f(x_1+h)-f(x_1)\underset{h\to0}{\sim}\frac{h^{2}}{2}f'(x_1)>0,
	\end{equation*}
	ce qui contredit le fait que $f(x_1)=\max\limits_{t\in[0,1]}f(t)$.
\end{remark}

\end{document}

\cleardoublepage
\listoffigures

\end{document}