\section{Suites et séries de fonctions}

\begin{exercise}
    Pour $x\geqslant0$ et $n\in\N^{*}$, on note 
    $$F_{n}(x)=\frac{1}{n}\prod_{k=1}^{n}\left(n+kx\right)^{\frac{1}{n}}$$

    Étudier la convergence simple et la convergence uniforme de $(F_{n})_{n\geqslant1}$.
\end{exercise}

\begin{exercise}
    Soit $-\alpha\notin\N^{*}$. Pour $n\geqslant1$, soit 
    $$u_{n}(x)=\frac{1\times 3\times\dots\times(2n-1)}{(1+\alpha)\times\dots\times(2n-1+\alpha)}x^{n}$$
    pour $x\in\R$.

    \begin{enumerate}
        \item Donner la nature de $\sum_{n\geqslant1}u_{n}(x)$.
        \item Trouver les valeurs de $\alpha$ telle que $\sum_{n\geqslant1}u_{n}$ converge uniformément sur $[0,1[$.
        \item Trouver les valeurs de $\alpha$ telle que $\sum_{n\geqslant1}u_{n}$ converge uniformément sur $]-1,0]$
    \end{enumerate}
\end{exercise}

\begin{exercise}
    On forme 
    $$f(x)=\sum_{k=0}^{+\infty}\arctan(k+x)-\arctan(k)=\sum_{k=0}^{+\infty}f_{k}(x)$$
    Quel est le domaine de définition de $f$ ? Montrer que $f$ est $\mathcal{C}^{1}$ sur ce domaine. Donner un équivalent de $f$ en $+\infty$.
\end{exercise}

\begin{exercise}
    On pose 
    $$f(t)=\sum_{n=1}^{+\infty}\ln\left(1-\e^{-nt}\right)$$
    Montrer que $f$ est définie pour $t>0$ et donner un équivalent de $f(t)$ quand $t\to0^{+}$. On admet l'existence de $I=\lim\limits_{x\to+\infty}\int_{0}^{x}\ln\left(1-\e^{-u}\right)du$.
\end{exercise}

\begin{exercise}
    Soit \function{f_n}{\R}{\R}{x}{\frac{n^{2}x^{2}}{1+n^{4}x^{4}}}
    $(a_{p})_{p\in\N}\in\R^{\N}$ et $g_{n}(x)=\sum_{p=0}^{+\infty}\frac{1}{2^{p}}f_{n}(x-a_{p})$.
    \begin{enumerate}
        \item $g_{n}$ est-elle définie ? Étudier la convergence simple de $(g_{n})_{n\in\N}$.
        \item Soit $[a,b]\subset\R$. Montrer que $(g_{n})_{n\in\N}$ converge uniformément sur $[a,b]$ si et seulement si pour tout $p\in\N$, $a_{p}\notin[a,b]$.
    \end{enumerate}
\end{exercise}

\begin{exercise}
    Convergence simple de 
    \begin{equation}
        f(x)=\sum_{n=1}^{+\infty}\frac{x}{x^{2}+n^{2}}=\sum_{n=1}^{+\infty}f_{n}(x).
    \end{equation}
    $f$ est-elle $\mathcal{C}^{1}$ ? Donner la limite de $f$ en $0$ et $+\infty$. Donner un équivalent en $0$.
\end{exercise}

\begin{exercise}
    Soit $(f_n)_{n\in\N}$ une suite de fonctions $\mathcal{C}^{1}$ sur $[a,b]$ telle qu'il existe $M\geqslant0$ tel que pour tout $n\in\N$, $\left\lVert f_{n}'\right\rVert_{\infty,[a,b]}\leqslant M$. On suppose que $(f_{n})_{n\in\N}$ converge simplement vers $f$ sur $[a,b]$. Montrer qu'il y a convergence uniforme.
\end{exercise}

\begin{exercise}
    Soit $x\geqslant1$. Soit $f_{n}(x)=\sum_{p=1}^{n}\frac{1}{\sqrt{n^{2}+p^{x}}}$. Étudier la convergence.
\end{exercise}

\begin{exercise}[Produit Eulérien]
    \phantom{}
    \begin{enumerate}
        \item Soit $(\mathcal{A},\left\lVert\cdot\right\rVert)$ une algèbre normée et pour $n\geqslant1$ \function{f_n}{\mathcal{A}}{\mathcal{A}}{a}{\left(1_{\mathcal{A}}+\frac{a}{n}\right)^{n}.}
        
        Montrer que 
        \begin{equation}
            \left\lVert\sum_{k=0}^{n}\frac{a^{k}}{k!}-f_{n}(a)\right\rVert\leqslant\sum_{k=0}^{n}\frac{\left\lVert a\right\rVert^{k}}{k!}-\left(1+\frac{\left\lVert a\right\rVert}{n}\right)^{n}.
        \end{equation}

        On pourra montrer que pour tout $(n,k)\in\N\times\left\llbracket0,n\right\rrbracket$, $\frac{1}{n^{k}}\binom{n}{k}\leqslant\frac{1}{k!}$. En déduire que $(f_n)_{n\geqslant1}$ converge simplement vers $\exp$, avec convergence uniforme sur les compacts de $\mathcal{A}$.

        \item On pose pour tout $n\in\N$,
        \begin{equation}
            P_{n}(X)=\frac{\left(1+\frac{\i X}{2n+1}\right)^{2n+1}-\left(1-\frac{\i X}{2n+1}\right)^{2n+1}}{2\i}.
        \end{equation}
        Montrer que $(P_{n})_{n\in\N}$ converge simplement vers $\sin$ sur $\C$.

        \item Déterminer le degré de $P_{n}$, les racines de $P_{n}$ et son coefficient en $X$. En déduire que 
        \begin{equation}
            P_{n}=X\prod_{k=1}^{n}\left(1-\frac{X^{2}}{(2n+1)^{2}\tan^{2}\left(\frac{k\pi}{2n+1}\right)}\right).
        \end{equation}

        \item Soit $(a_{n,p})_{(n,p)\in\N^{2}}\in\C^{\N^{2}}$. On suppose que 
        \begin{enumerate}
            \item [(i)] Il existe $(\alpha_{n})_{n\in\N}\in\R_{+}^{N}$ avec $\sum_{n=0}^{+\infty}\alpha_n<+\infty$ et pour tout $(n,p)\in\N^{2}$, $\left\lvert a_{n,p}\right\rvert\leqslant\alpha_{n}$.
            \item [(ii)] $\forall n\in\N$, il existe $\beta_{n}=\lim\limits_{p\to+\infty}a_{n,p}\in\C$.
        \end{enumerate}
        Montrer que $\lim\limits_{p\to+\infty}\sum_{n=0}^{+\infty}a_{n,p}=\sum_{n=0}^{+\infty}\beta_{n}$.

        \item En déduire que pour tout $x\in\R$, $\sin(x)=x\prod_{k=1}^{+\infty}\left(1-\frac{x^{2}}{k^{2}\pi^{2}}\right)$. On pourra montrer que pour tout $t\in\left]-\frac{\pi}{2},\frac{\pi}{2}\right[$, $\left\lvert\tan(t)\right\rvert\geqslant\left\lvert t\right\rvert$.
    \end{enumerate}
\end{exercise}

\begin{exercise}
    Soit $[a,b]\subset]0,1[$ et \function{f}{\R}{\R}{x}{2x(1-x)}
    On définit $f^{1}=f$ et pour tout $n\in\N$, $f_{n+1}*f\circ f_{n}$.
    \begin{enumerate}
        \item Montrer que $(f_{n})_{n\in\N}$ converge uniformément vers $\frac{1}{2}$ sur $[a,b]$. A-t-on convergence uniforme sur $[0,1]$ ?
        \item Soit $\Q_{2}=\left\lbrace\frac{p}{2^{n}}\middle|p\in\Z,n\in\N\right\rbrace$. Montrer que pour tout $P\in\R[X]$, pour tout $\varepsilon>0$, il existe $Q\in\Q_{n}[X]$ tel que $\left\lVert P-Q\right\rVert_{\infty,[a,b]}\leqslant\varepsilon$.
        \item En déduire que pour tout $f\in\mathcal{C}^{0}\left([a,b],\R\right)$, il existe $A\in\Z[X]$ telle que 
        \begin{equation}
            \left\lVert f_{n}-A\right\rVert_{\infty,[a,b]}\leqslant \varepsilon.    
        \end{equation}
        Peut-on généraliser à d'autres intervalles ?
    \end{enumerate}
\end{exercise}

\begin{exercise}
    Soit $(u_{n})_{n\in\N}$ une suite d'applications convexes de $I\subset\R\to\R$ qui converge simplement vers $u\colon I\to\R$.
    \begin{enumerate}
        \item Soit $[a,b]\subset\mathring{I}$. Montrer qu'il existe $A\in\R$ tel que pour tout $(x,y)\in[a,b]$,
        \begin{equation}
            \left\lvert u_{n}(x)-u_{n}(y)\right\rvert\leqslant A\left\lvert x-y\right\rvert.
        \end{equation}
        On pourra former $(\alpha,\beta)\in I^{2}$, $\alpha<a<b<\beta$, et étudier les taux d'accroissements des $u_{n}$.

        \item En déduire que $(u_{n})_{n\in\N}$ converge uniformément vers $u$ sur $[a,b]$.
    \end{enumerate}
\end{exercise}

\begin{exercise}
    Soit $[a,b]\subset\R$ et $E=\mathcal{C}^{0}\left([a,b],\R\right)$ muni de $\left\lVert\cdot\right\rVert_{\infty}$. Soit $\varphi\colon:\R\to\R$ continue. Montrer que \function{\psi}{E}{E}{f}{\varphi\circ f}
    est continue.
\end{exercise}

\begin{exercise}
    On pose, sous réserve d'existence,
    \begin{equation}
        f(t)=\sum_{n=0}^{+\infty}\frac{\e^{-nt}}{1+n^{2}}=\sum_{n=0}^{+\infty}f_{n}(t).
    \end{equation}
    \begin{enumerate}
        \item Donner le domaine de définition $E$ de $f$.
        \item $f$ est-elle continue sur $E$ ? Évaluer $\lim\limits_{t\to+\infty}f(t)$.
        \item Montrer que $f$ est $\mathcal{C}^{\infty}$ sur $E\setminus\left\lbrace0\right\rbrace$. Donner l'équation différentielle satisfaite par $f$.
    \end{enumerate}
\end{exercise}

\begin{exercise}
    Soit $a\in\R$ et \function{u_n}{[0,1]}{\R}{x}{\frac{x\e^{-nx}}{n^{a}}}
    \begin{enumerate}
        \item Montrer que $\sum_{n=1}^{+\infty} u_n$ converge simplement sur $[0,1]$. On note $S(x)=\sum_{n=1}^{+\infty}u_{n}(x)$.
        \item Pour quelles valeurs de $a$ a-t-on convergence normale sur $[0,1]$ ?
        \item Calculer $S$ pour $a=1$ et $a=2$.
    \end{enumerate}
\end{exercise}

\begin{exercise}
    Donner le domaine de définition de
    \begin{equation}
        f(x)=\sum_{n=1}^{+\infty}\frac{1}{\sqrt{n}}\times\frac{1}{1+nx^{\frac{3}{2}}}.
    \end{equation}
    Étudier la continuité de $f$ sur son domaine de définition. $f$ est-elle intégrable sur son domaine de définition ? 
\end{exercise}

\begin{exercise}
    \phantom{}
    \begin{enumerate}
        \item Donner le domaine de définition de 
        \begin{equation}
            S(x)=\sum_{n=2}^{+\infty}\frac{x\e^{-nx}}{\ln(n)}=\sum_{n=2}^{+\infty}f_{n}(x).
        \end{equation}
        \item Montrer que l'on a converge uniforme sur $[0,\infty[$. A-t-on convergence normale ?
        \item Montrer que $S$ est $\mathcal{C}^{1}$ sur $\R_{+}^{*}$, mais n'est pas dérivable à droite en $0$.
        \item Montrer que pour tout $k\in\N$, $S(x)=\underset{x\to+\infty}{o}\left(\frac{1}{x^{k}}\right)$.
    \end{enumerate}
\end{exercise}

\begin{exercise}[Théorème de Weierstrass trigonométrique]
    On pose, pour $k\in\N$, \function{Q_k}{\R}{\R}{t}{c_k\left(\frac{1+\cos(t)}{2}\right)^{k}} où $c_{k}\in\R$ tel que $\frac{1}{2\pi}\int_{-\pi}^{\pi}Q_{k}(t)\mathrm{d}t=1$.
    \begin{enumerate}
        \item Montrer que pour tout $\delta\in]0,\pi]$, $\lim\limits_{k\to+\infty}\sup\limits_{\delta\leqslant\left\lvert t\right\rvert\leqslant\pi}Q_{k}(t)=0$.
        \item Soit $f$ continue $2\pi$-périodique de $\R$ dans $\C$. On définit 
        \function{P_k}{\R}{\C}{t}{\frac{1}{2\pi}\int_{-\pi}^{\pi}f(t-s)Q_k(s)\mathrm{d}s.}
        Montrer que $(P_k)_{k\in\N}$ converge uniformément vers $f$ sur $\R$. On utilisera, en la justifiant, la continuité uniforme de $f$ et son caractère borné sur $\R$.
        \item On note, pour tout $k\in\Z$, \function{\varepsilon_k}{\R}{\C}{t}{\e^{\i kt}.}
        On pose $F=\Vect(\varepsilon_{k})_{k\in\Z}$ (\og polynômes trigonométriques\fg $2\pi$-périodiques). Montrer que $F$ est dense dans $E$, $\C$-espace vectoriel des fonctions continues $2\pi$-périodiques pour $\left\lVert\cdot\right\rVert_{\infty}$.
    \end{enumerate}
\end{exercise}

\begin{exercise}
    Soit $(u_n)_{n\in\N}$ une suite monotone de fonctions continues qui converge simplement vers $u$ continue sur un compact $K\subset E$ où $E$ est un espace vectoriel normé.
    \begin{enumerate}
        \item Montrer que l'on peut se ramener au cas d'une suite $(f_n)_{n\in\N}$ décroissante de fonctions continues qui converge simplement vers 0.
        \item Soit $\varepsilon>0$ et pour tout $n\in\N$, $F_{n,\varepsilon}=\left\lbrace x\in K\middle| f_{n}(x)\geqslant\varepsilon\right\rbrace$. Montrer que $F_{n,\varepsilon}$ est fermé, que $F_{n+1,\varepsilon}\subset F_{n,\varepsilon}$ et que $\cap_{n\in\N}F_{n,\varepsilon}=\emptyset$. En déduire qu'il existe $N\in\N$ tel que $F_{N,\varepsilon}=\emptyset$, puis que $(f_{n})_{n\in\N}$ converge uniformément vers 0 sur $K$.
        \item Prouver le résultat en considérant pour tout $n\in\N$, $x_{n}\in K$ tel que $f_{n}(x_{n})=\max\limits_{x\in K}f_n(x)$.
    \end{enumerate}
\end{exercise}